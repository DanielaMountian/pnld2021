
\begin{comment}
TRADUÇÃO DA VERSÃO ALEMÃ DE \textit{A DEMANDA DO SANTO GRAAL} (Códice 147 da
\textit{Bibliotheca Palatina Germaniae}, de Heidelberg, fins do século XIII). 
\end{comment}

\chapter{O despontar} 

\textsc{Na noite sacra} de Pentecostes, tendo os convivas da Távola Redonda vindo a
Camelot e assistido missa, e devendo-se pôr a mesa pois era tempo da refeição do
meio-dia, veio, a cavalo, uma donzela ao salão e muito se aprestou, pois seu
cavalo por toda parte suava. Desmontou e veio perante o rei, e o saudou. E ele
disse que Deus lhe pagasse e que, ante Deus e ele, ela era muito bem-vinda. Ela
disse: “Senhor, dizei-me por Deus, Lancelot está aqui?”

“Sim”, disse o rei, “vede onde ele está!” E ela sabia qual ele era e foi para
onde ele se encontrava e diz: “Lancelot, eu lhe digo da parte do rei Pellis,
por quem deveis acompanhar-me àquela floresta”. E ele pergunta a quem ela
obedece. “Eu obedeço àquele”, disse ela, “de quem vos falei” “E por qual razão
me convocais para cavalgar convosco?” “Isso logo vereis”, disse ela. “Em nome
de Deus”, disse ele, “pois desejo com prazer seguir convosco”. Então disse a um
de seus servos que selasse seu corcel e lhe trouxesse suas armas, e aquele
cumpriu. E porque o rei e todos os que estavam no palácio viram que ele queria
partir, lamentaram e queixaram-se muito a ele na ocasião. E porque viram que
ele não desejava permanecer, deixaram-nos cavalgar, e a rainha disse:
“Lancelot, deixai-nos, pois, nesta ocasião de festejo?” “Senhora”, disse a
donzela, “sabei que o tereis de pronto novamente amanhã para a ceia”. “Pois,
ide”, disse ela, “se não retornardes amanhã, isto será contra minha boa
vontade”. Ele e a donzela levantaram-se, apartaram-se então, sem descanso ou
outra companhia, senão a de dois escudeiros que vieram com a donzela. E
passaram à frente da cidade de Camelot, e ao passarem à floresta, cavalgaram por
ampla vereda bem meia milha até que adentraram um vale.

Lá viram um convento de freiras. E a donzela volveu pelo mesmo caminho tão
rápido quanto pôde. E ao vir à porta, bateram-lhe os servos. E se a abriu, e
eles desmontaram e entraram. E quando os que lá estavam foram avisados de que
Lancelot veio, foram todos a seu encontro e lhe ofertaram grande honraria e o
conduziram a uma câmara e o desarmaram. E ao ter sido desarmado, viu em um
leito seus dois primos Bohort e Leonel, e isso o maravilhou muito, e os
acordou. E quando o viram, abraçaram-no e o beijaram, e nutriram grande alegria
por se verem uns aos outros. 

“Caro Senhor,” falou Bohort a Lancelot, “que aventura vos trouxe cá, Senhor?
Esperávamos encontrar-vos em Camelot.” E então lhe disse como a donzela o havia
para lá trazido e não sabia por quê. E enquanto conversavam, vieram três
freiras a ele, que perante ele trouxeram Galaat, o belo garoto. Esse era tão
bem feito de membros, que nunca se teria encontrado outro igual sobre a terra.
E aquela mulher que o tinha às mãos, chorava alegremente. E ao virem a
Lancelot, então disse: “Senhor, eu vos trago nossa criação e toda a alegria
que temos e todo o nosso consolo e toda a nossa esperança, para que queirais
torná-lo cavaleiro, porque ele vem de homens mais nobres que vós, segundo nos
parece. Portanto ele deve receber a ordem da cavalaria”. Ele achou o rapaz
demasiadamente completo de toda a beleza que nele via, e tanto bem, que lhe
agradaria deveras torná-lo cavaleiro. E respondeu às mulheres que cumpriria com
imenso gosto o pedido, se o desejassem. “Senhor,” disse ela, “assim queremos
que aconteça ainda hoje ou amanhã”. “Em nome de Deus”, disse ele, “quando
quiserdes”. Lancelot lá pernoitou e fez que o rapaz permanecesse acordado por
toda a noite no mosteiro. E, às primas horas da manhã, tornou-o cavaleiro e
atou-lhe uma espora por isso, e Bohort uma outra.

E depois Lancelot lhe afivelou sua espada por esta razão, e lhe deu um golpe e
disse que Deus o tornaria um nobre, se não lhe faltasse a beleza. E porque nele
fez tudo aquilo que se deve fazer a um novo cavaleiro, então disse: “Caro
Senhor, não deveis cavalgar comigo para a corte do rei Arthur?”
“Senhor,” disse ele, “convosco não cavalgarei para lá”. Então disse Lancelot à
abadessa: “Senhora, é vossa vontade que vosso novo cavaleiro conosco cavalgue
para a corte do meu Senhor, o rei Arthur?” “Senhor,” disse ela, “ele ainda não
deve ir para lá! Portanto assim que soubermos que é o tempo, nós o deveremos
enviar para lá”. Então se apartaram Lancelot e seus camaradas e cavalgaram
tanto que se encontravam em Camelot ao tempo das terças. O rei tinha então ido à
missa com grande séquito de altos homens. E os três primos tinham vindo ter à
corte, desmontaram e subiram ao salão. 

Então se puseram a dizer do garoto que Lancelot havia feito cavaleiro. Então
disse Bohort que ele nunca vira algum homem que tanto se parecesse com
Lancelot. “E”, disse ele, “eu não quero crer em outra coisa senão que seja
Galaat, que foi concebido com a bela filha do Rei Pescador, se ele tanto se
assemelha à sua linhagem quanto à nossa”. “Valha-me Deus”, disse Leonel, “nisso
quero de bom grado crer, pois ele se parece de toda forma com nosso senhor, Lancelot”.

Durante um bom tempo conversaram sobre aquilo, para tentar que Lancelot
retirasse qualquer palavra de sua boca. Mas nenhuma palavra disse por conta
daquilo que se falou. Então deixaram de falar sobre isso e observaram os
assentos à Távola Redonda e encontraram em cada um escrito: “Aqui deve este se
sentar, aqui deve aquele se sentar”. E assim foram a observar até que chegaram
ao grande assento, que era chamado o Assento Perigoso. E lá encontraram letras,
e eram recentemente escritas. Observaram o que falam, e elas falavam pois: 
“Quatrocentos e cinquenta e quatro anos após o martírio de Deus, ao dia de
Pentecostes, então este assento deve encontrar seu mestre”. Quando eles o
viram, um disse ao outro: “Em verdade, há aqui maravilhosa aventura!”
“Certamente”, disse Lancelot, “aquele que contar corretamente nesta missiva os
termos da ressurreição de Nosso Senhor, encontrará que é hoje, e neste dia
deverá ocorrer, pois que hoje é o dia de Pentecostes que deverá ser após
quatrocentos e cinquenta e quatro anos. Eu gostaria que ninguém visse esta
letra até que o rei, que deve ultimar essa aventura, venha”. 

Eles disseram que desejavam ocultar e trouxeram um lençol de seda e encobriram a
escrita que lá estava. E quando o rei Arthur veio do mosteiro e viu que
Lancelot havia regressado e trazido Bohort e Leonel, alegrou-se muito e os
recebeu com grande alegria, e disse que eram bem-vindos aos olhos de Deus e aos
seus. A alegria elevou-se maravilhosamente entre os camaradas da Távola
Redonda, quando também viram, com muito gáudio, os dois irmãos. E o senhor
Gawin perguntou como eles estavam desde então. Eles disseram: “Muito bem, pela
Graça de Deus!”, que estiveram desde então sempre fortes e saudáveis.
“Seguramente”, disse o senhor Gawin, “estou por hora todo contente e isso me
alegra”. Grande era a alegria dos convivas da corte, Bohort e Leonel quando por
muito tempo não tinham visto os homens. E o rei ordenou que se pusesse a mesa,
pois lhe parecia ser tempo de cear. Então disse o Senhor Key: “Sentai-vos agora
para cear, assim me parece que não cumpris o costume da corte pois vimos de
toda forma que não vos sentais para nenhum festejo, eis que vai ocorrer na
corte uma aventura aos olhos de todos os vossos heróis”. “Seguramente”, disse o
rei, “dizeis a verdade, esse costume tenho de toda forma conservado e devo
conservá-lo quanto tempo me aprouver; tive, pois, grande alegria que Lancelot
e seus primos tenham vindo à corte saudáveis e contentes como não imaginava”. 

“Então homenageai”, disse Key. Enquanto eles falavam, veio um servo e disse ao
rei: “Senhor, mais maravilhas vos trago”. “Dizei-me o que mais!” “Senhor, lá
abaixo, próximo a vosso palácio, encontra-se uma coluna, que singra as águas,
vinde e vede, pois que sei que é uma maravilhosa aventura.” E o rei foi lá para
observar a aventura, e assim fizeram todos os outros. Ao virem até a água,
então viram que a coluna estava fora da água e era de mármore vermelho. E na
coluna viram que uma espada estava fixa e belamente embainhada. E a bainha da
espada era um rubi e eram douradas as letras preciosamente encravadas. E os
heróis observavam as letras, que diziam: “Ninguém deve retirar-me daqui senão
aquele que me deve ter por direito e que deve ser o melhor cavaleiro do mundo”
. Quando o rei viu as letras, então falou a Lancelot: “Senhor, tomai a
espada, pois é vossa por direito, sabendo bem eu que sois o melhor cavaleiro
que há no mundo”. E ele respondeu asperamente: “Senhor, não pertence isso a mim
e nem poderia dela querer lançar mão, pois não sou digno de tomá-la e por isso
devo conter-me, e seria tolice eu querer tocá-la”. 

“Tentai, porém!”, disse o rei, “se quiserdes retirá-la”. “Senhor,” falou ele,
“eu não o farei, pois bem sei que quem tentar e falhar sairá encantado”. “O que
sabeis?”, disse o rei. “Senhor,” disse ele, “eu o sei bem e vos digo, e quero
que saibais que ainda hoje, neste dia, devem alçar-se as grandes aventuras e a
grande maravilha do Santo Graal”. E como o rei ouviu que Lancelot não
o queria fazer, falou ao senhor Gawin: “Caro sobrinho, tentai vós”. “Senhor,
com a vossa licença, eu não o farei: se meu senhor Lancelot não quer tentar,
seria então tolice tentar lançar mão, quando sabeis que ele é muito melhor
cavaleiro que eu”. “Deveis, porém, tentá-lo porque eu o quero, não para
possuirdes a espada!” E ele lançou mão à coluna e agarrou a espada com toda
força e puxou o quanto pôde, mas não logrou sacá-la. O rei disse atônito: “Caro
sobrinho, deixai estar, pois cumpriste meu mandamento”. “Senhor Gawin”, disse
Lancelot, “sabeis por certo que esta espada deverá cortar-vos de tão perto que
daríeis um castelo para que não a houvesses tocado”. “Senhor,” disse o senhor
Gawin, “nisto não faço gosto; se devo agora por tal razão morrer, então o faço
para cumprir a vontade do meu senhor”. E porque o rei ouviu esta fala,
arrependeu-se do que Gawin tinha feito, e disse a Parsifal que tentasse. E ele
respondeu: “Com prazer, para fazer corte ao meu senhor”, e lançou mão à espada
e puxou o quanto pôde, mas não conseguiu tê-la. E então acreditaram
todos que estavam no lugar que era verdade o que Lancelot dissera, e que as
letras na espada eram certeiras, e ninguém jamais poderia ser temerário para
fazê-lo. Então disse o senhor Key: “Senhor, valha-me Deus, então podeis ir
cear, se o quiserdes, pois me parece que não vos faltará aventura antes da
ceia”. 

“Vamos então”, disse o rei, “já é tempo”. Então se foram todos dali e deixaram a
coluna restar sobre a água. E o rei chamou a soprar a comida e foi sentar-se
para cear, e cada qual foi sentar-se em seu lugar. No mesmo dia serviram à mesa
quatro reis coroados e com eles tantos homens elevados que seria maravilhoso dizer.
No mesmo dia sentou-se o Rei Arthur sobre seu elevado assento no palácio e foi
servido junto a grande cortejo de príncipes. Assim aconteceu, estando
assentados todos, por toda parte, pois tinham vindo todos os convivas da Távola
Redonda e todos os assentos estavam preenchidos. Então lhes aconteceu uma
maravilha, pois todas as portas e janelas do palácio se fecharam, sem que
ninguém pusesse a mão, e por isso o salão escureceu. E todos se espantaram com
estas coisas, fosse o sábio ou o parvo. E o Rei Arthur, com a primeira fala,
disse: “De fato vimos hoje um milagre e cremos que deveremos amanhã ver o que é
isso”. Enquanto o rei assim falava, adentrou um nobre com trajes alvos, velho e
sisudo, e nenhum cavaleiro presente soube informar de onde ele tinha vindo. E o
fidalgo vinha a pé e conduzia com a mão um cavaleiro, com armadura escarlate,
sem espada ou escudo. 

E falou tão logo adentrou o palácio: “A paz esteja aqui!” E disse ao rei,
quando o viu: “Rei Arthur, eu te trago o cavaleiro, por quem há tanto tempo se
aspira, e ele procede da alta linhagem do Rei Davi e de José de Arimateia. É
aquele com quem os prodígios desta e de outras terras devem ocorrer, vede-o
aqui!” E o rei ficou muito feliz e disse ao fidalgo: “Sede bem-vindo, e sendo
isso mais verdadeiro, então bem-vindo seja vosso cavaleiro! Se é aquele há
tanto reclamado, que deverá nos realizar as aventuras do Santo Graal, e nunca
maior alegria nos foi dada por nenhum homem, que aquela que dele devemos ter.`` 

“Seja ele ou outro, então desejo que se lhe favoreça, pois é de tão alta
linhagem como dizeis.” “Em verdade”, disse o nobre, “deveis ver um belo início
a partir dele”, e o fez desarmar. Então ficou com uma saia de cindel vermelho,
e se lhe deu portanto uma capa de samítico vermelho, guarnecida com arminho,
que trajou sobre o pescoço. Quando estava trajado e pronto: “Segui-me, senhor
cavaleiro”, disse ele, e esse o fez, e o conduziu ao Assento Perigoso, ao lado
do qual se sentava Lancelot. E alçou o pano de seda, que sobre ele permanecia;
lá havia letras escritas, que diziam: “Aqui é o assento de Galaat”. O fidalgo
observou que as letras haviam sido recentemente inscritas, segundo lhe pareceu.
Então reconheceu o nome e clamou tão alto que todos os que estavam no castelo
ouviram: “Senhor, aqui cavaleiro, sentai-vos aqui, pois é vosso o lugar”. E
sentou-se sorridente. E diz ao nobre senhor: “podeis bem seguir vosso caminho,
pois que fizeste o que vos foi pedido, e saudai por mim intensamente todos
aqueles que [são] da santa corte e meu tio rei Pellis e meu avô, o rico Rei Pescador
e dizei-lhe por minha causa que desejo ir tão logo possa, ou quando for
devido”. E o nobre despediu-se e encomendou a Deus o rei Arthur e todos os
outros.  Pois se lhe perguntou quem ele era, ele bem respondeu asperamente, que
não o saberiam agora, se deveriam saber ao tempo propício em que poderiam
perguntar. Então dirigiu-se à porta principal do palácio, que estava fechada, e
a abriu e precipitou-se para a corte, pois que seus cavaleiros e servos eram
bem quinze que haviam com ele vindo, e montou e despediu-se da corte, de tal
sorte que não, nenhum homem à vez que dele soubesse. 

Pois que todos os que estavam no salão viram que o cavaleiro assentava-se no
assento, pois que muitos mais corajosos o teriam temido, e pois que grande
aventura poderia ocorrer, nenhum havia que não se admirasse que era um jovem
varão, e ignoravam por graça de quem ele poderia ter vindo, então era por
vontade de Nosso Senhor. Então houve grande alegria, e todos que estavam na
corte ofertaram ao cavaleiro mercê e honra, pois que bem pensaram que era ele
aquele pelo qual a aventura do Santo Graal deveria ser superada. Então
bem perceberam junto ao assento, pois ninguém se tinha assentado, que para
todos, a não ser para ele, haveria malefício em o fazer. 

Então lhe ofereceram grande honra e o serviram tanto quanto podiam, e o
consideravam o mais alto sobre todos os cavaleiros da Távola Redonda. E
Lancelot, que o via com grande prazer por mercê da maravilha, quando ele
reconheceu que era aquele a que havia feito cavaleiro no dia, teve grande
alegria. E portanto fez-lhe  a máxima honra que conhecia e falou-lhe por
algumas vezes e perguntou-lhe de seu ser, e que de muitas maneiras o tornasse
conhecido, o que ele não deveria recusar. E informou-lhe muito acerca do que
lhe tinha perguntado. E Bohort também se alegrou, de modo que não poderia estar
mais feliz, quando bem reconheceu que era Galaat, filho de Lancelot, que
deveria levar ao fim a aventura e a maravilha. Então falou Leonel a seu irmão:
“Amado irmão, sabeis quem é o cavaleiro lá sentado sobre o Assento Perigoso?”
“Dele bem não sei”, falou Bohort, “além de que é aquele que meu senhor
Lancelot há pouco fez cavaleiro com sua mão e seja aquele de quem ele e eu
dizíamos, aquele que meu senhor Lancelot ganhou com a filha do rico Rei
Pescador”. “Sabei seguramente”, falou Leonel, que é nosso parente, e desta
aventura devemos com direito estar felizes.  Se estou sem engano, ele ainda
deve vir na mais alta honra, pois nenhum cavaleiro aproxima-se dele, que
tenhamos conhecido, pois que se portou muito belamente”. 

Assim falaram os dois irmãos de Galaat, e assim fizeram todos os outros da
corte. E assim levaram os outros para cima e para baixo, tanto que a rainha,
que na câmara comia, ouviu dizer. Quando um servo lhe falou: “Senhora, mais
maravilha aconteceu aqui dentro”. “Como”, perguntou ela. “Por minha verdade”,
falou ele, “um cavaleiro está na corte, que preencheu a aventura do Assento
Perigoso, e todo o mundo maravilhou-se por ter-lhe vindo a graça”.
“Seguramente”, disse ela. “Pode isto ser veraz?” “Sim”, disse ele,
“seguramente”. “Valha-me Deus”, falou ela, “pois cabalmente lhe aconteceu,
quando essa aventura não poderia ocorrer a ninguém, pois estaria morto ou
ferido, e o teria levado ao fim”. “Ah”, disseram as damas “como nasceu em boa
hora o cavaleiro! Se nenhum cavaleiro foi mais nobre para isso lhe acontecer.
Nesta aventura pode-se bem reconhecer que esse é quem deve concluir a aventura
da Grã-Bretanha, por meio da qual o rei ferido deve convalescer”. “Caro
amigo”, falou a rainha ao servo, “dizei-me pois, que Deus o ajude, de que
figura ele era?” “Senhora”, falou ele, “é um dos mais formosos cavaleiros no
mundo, sem medida jovem e idêntico a Lancelot e à linhagem do rei Ban, tão
maravilhosamente que todos os que daqui falam tomam por verdadeiro que ele vem
dele”. E então desejou a rainha vê-lo muito mais que antes. Quando ouviu falar
de sua aparência, bem pensou que ele era quem Lancelot havia ganho com a filha
do rico Rei Pescador, portanto que se lhe dissesse de que maneira ele o
ganhara. E foi por causa do garoto que ela tanto se zangou, como se dele fosse
a culpa. 

Quando o rei tinha comido, e os convivas da Távola Redonda se tinham levantado
de seus assentos, e o rei mesmo foi até o Assento Perigoso e suspendeu o pano
de seda e encontrou o nome de Galaat, que ele tanto desejou saber. Então o
mostrou o rei ao senhor Gawin e disse: “Caro sobrinho, eis que recebemos
Galaat, o bom cavaleiro, que todos da Távola Redonda tanto desejamos ver.
Então reverenciemos em honra por tê-lo conosco, já que não deve permanecer
muito tempo conosco, pois bem sei que a busca do Santo Graal logo se eleva,
como estou certo. Eis o que Lancelot entendia [que devesse] nos falar, quando ele não teria
dito, se não tivesse sabido por diversa maneira”. “Senhor,” disse meu senhor
Gawin, “vós e nós estamos com ele em dívida para servi-lo como aquele a quem
Deus nos enviou, por causa de quem se redime a terra das grandes maravilhas e
das grandes aventuras que tanto já perduraram”. 

Então veio o rei a Galaat e falou: “Senhor, sejais bem-vindo, pois muito
desejávamos ver-vos. Eis que vos temos aqui, pelo que damos graças a Deus e a
vós por ter desejado vir para cá”. “Pois Senhor,” falou ele, “é meu dever
dizê-lo, que é daqui que deverão vir os companheiros da busca do Santo Graal,
que certamente deve elevar-se”. “Senhor,” disse o rei, “vossa vinda nos dá
muito a fazer por causa desta grande maravilha que deve nesta terra ser levada
a termo e por um prodígio que hoje ocorreu, em que todos falharam. Porque sei
que não falhareis, pois sois aquele que deve levar tudo ao fim, em que os
outros falharam. Por isto Deus o enviou, Senhor, para que completásseis o que
ninguém pôde ultimar”. “Senhor,” falou Galaat, “onde está a aventura de que me
dizeis? Pois anseio muito por vê-la”. “Eu devo indicar-vos”, falou o rei e o
tomou pela mão e desceu do palácio. E os convivas da corte seguiram para ver
como deveria findar a aventura da coluna. Para lá correram estes e aqueles, de
modo que nenhum cavaleiro permaneceu no palácio. E as novas vieram perante a
rainha. E tão logo quanto ouviu, ergueu as tábuas e falou para quatro das
principais damas que com ela estavam: “Caras Senhoras, ide comigo até lá, pois
não deixai, de forma alguma, preciso ver a aventura terminar, se conseguir
chegar a tempo”. 

A rainha retirou-se do palácio e com ela grande parte de damas e donzelas. Assim
que se aproximaram da água e os cavaleiros a viram chegar, principiaram a falar:
“Hoje, hoje, minha Senhora rainha!”, e deram-lhe passagem. E o rei falou a
Galaat: “Senhor, vede cá a aventura, da qual vos disse para retirar a espada
desta coluna; pois que todos os melhores de minha casa falharam, pois não a
souberam extrair”. “Senhor,” falou Galaat, “não é prodígio, pois a aventura é
minha e não deles, e por esta certeza de ter a espada, não preciso de nenhum
senhor comigo, como podereis ver por vós mesmos”. Então pôs as mãos à
espada e a retirou levemente, como se não estivesse atada, e tomou a bainha e a
enfiou e a afivelou consigo e falou ao rei: “Senhor, assim está melhor que
antes. Então não se me aprovisiona um escudo, que eu não possuo?” “Caro
Senhor,” falou o rei, “um escudo há Deus de vos conceder, como vos fez com a
espada”. Então viram tudo rio abaixo, de onde vinha uma donzela em um palafrém
negro e veio depressa para eles. E quando veio ter, saudou o rei e seu cortejo
e perguntou se Lancelot estaria lá, e ele estava de todo próximo a ela.
Respondeu e falou: “Donzela, estou aqui!” Ela o contemplou, o conheceu e lhe
falou, chorando: “Ah Lancelot, muito se inverteu vosso ser desde ontem de
manhã”. E quando falou isso, Lancelot respondeu-lhe e falou “Donzela, como é
isto que me dizeis?” “Por minha verdade”, falou ela, “com prazer vos devo
dizer, à escuta de todos os que cá estão: Hoje de manhã éreis o melhor
cavaleiro que aí vive; e quem o tivesse chamado melhor cavaleiro teria dito o
verdadeiro, pois o éreis. E se o dissesse agora, dever-se-ia considerar
mentira. Um melhor do que sois é bem visitado com a aventura da espada, pois
não teríeis ousado dela lançar mão. E por isso vosso nome foi confundido e
invertido, por isso vos lembro para não crer que ainda sois o melhor cavaleiro
do mundo”. E ele falou que não mais queria acreditar que o fosse, “Pois essa
aventura mo tirou do coração”. 

Então se virou a donzela para o rei e falou: “Rei Arthur, Mathias, o Eremita, vos
exora comigo que neste dia, ainda hoje, deve ocorrer a maior honra que jamais
se deu para qualquer cavaleiro da Bretanha, e não por vossa causa, é por causa
de outra pessoa, e sabeis por que causa? Pelo Santo Graal, que deve aparecer em
vossa casa e fartar a todos os convivas da Távola Redonda”. E
tão logo o falou, virou-se e seguiu a via por onde tinha vindo. E suficientes
cavaleiros e convivas a teriam lá retido, para saber quem ela era e de que lado
tinha vindo. E, pois, ela não quis dizê-lo a nenhum dos que lhe pediram. Então
falou o rei para os heróis de sua corte: “Caros Senhores, é pois da demanda do
Santo Graal que tivemos verdadeiramente sinais de que logo a ela vireis. E por
causa disso bem sei que não mais vos verei aos montes como aqui estais, então
quero que no gramado de Camelot haja agora um torneio tão cortante que após
vossa morte ainda dele digam aqueles que vos sucederem”. E eles o seguiram
todos e foram para a cidade e se armaram, e uma parte armou-se quase a enfiar [a armadura],
e os outros não mais tomaram que cobertura e escudo e abandonaram-se à sua
força. E o rei, que tudo isso fizera, não o fizera senão para ver a força de
Galaat, pois pensou que ele não retornaria em breve quando deles se separasse. 

E lá estavam todos reunidos no gramado de Camelot, os grandes e os pequenos,
Galaat a pedido do rei e da rainha colocou sua malha de pescoço e alçou seu
elmo, sem o escudo, pois não queria receber de ninguém admoestação. E meu
senhor, senhor Gawin, que também estava muito feliz, falou que queria
conduzir-lhe a lança. E deste modo falou também meu senhor Ywain e Bohort de
Ganna, e a rainha tinha ido além-muros com grande séquito de damas e donzelas. 
E Galaat veio ao gramado com os outros cavaleiros e começou a quebrar
lanças tão nervosamente que todos os que o viram tiveram grande maravilha e o
tomaram pelo melhor cavaleiro dentre todos. E falaram todos aqueles que o viram
que ele sobejamente portava a cavalaria, que bem parecia pelo que havia feito
que no futuro superaria todos os demais cavaleiros em bravura. Quando ocorreu o
torneio, acharam todos os companheiros da Távola Redonda que lá portaram armas,
que ninguém restara a não ser dois, ele derrubou a todos. Um era Lancelot, o
outro Parsifal. Assim durou o torneio até as nonas e durou até que o rei mesmo
se preocupasse que não se desaviessem e os separou. E fez Galaat retirar seu
elmo e o fez dá-lo a vestir a Bohort de Ganna, e o conduziram pelas
principais alamedas com semblante aberto, pelo qual as pessoas mesmo o viram. E
a rainha que o contemplava e falou que era certo que fosse ele o filho de
Lancelot, pois dois homens bem não se assemelhavam como eles. Por isto não
seria maravilha que fosse de tão grande cavalaria, quando outra coisa se lhe
passaria de forma surpreendente. E uma dama ouviu dessa fala uma parte e
respondeu logo de pronto: “É devido a ele por direito ser bom cavaleiro como
dizei?” “Por certo”, falou a rainha, “pois ele procede de todos os lados dos
melhores cavaleiros do mundo e da melhor linhagem que se conhece no mundo”. 

Com isto seguiram as damas e queriam ouvir as vésperas, pois era uma grande
celebração. E por que o rei tinha vindo da igreja e estava no alto palácio,
pediu que se cobrisse a távola. E foram sentar-se os cavaleiros, cada qual em
seu lugar, como se haviam sentado pela manhã. Tão logo estavam sentados e
permaneciam em calmo silêncio, ouviram uma trovoada tão maravilhosamente grande
que acharam que o palácio cairia. Com isso veio um belo sol, que brilhou tão
claro que brilhou de modo sete vezes mais belo do que havia brilhado e ficaram
todos os que dentro estavam como se estivessem agora cheios do Espírito Santo,
e começou um a olhar o outro, e não sabiam como aquilo lhes havia acontecido, e
não havia homem na corte que soubesse falar qualquer palavra para fora de sua
boca, estavam todos emudecidos, pequenos e grandes. E permaneceram sentados por
um bom momento, e não puderam senão olhar-se uns aos outros. Enquanto
permaneciam sentados, adentrou o Santo Graal e estava coberto com um samítico
branco, e ninguém pôde ver quem o trazia, e veio através das grandes portas ao
palácio. E tão logo estava dentro, todo o palácio estava pleno de bom odor,
como se todas as ervas do mundo se tivessem lá espalhado, e foi pelo palácio de
confim a confim. E assim que isso aconteceu, foram todas as mesas preenchidas
das melhores iguarias que alguém poderia imaginar. Pois que tinha servido de um
ao outro, despediu-se deles o Santo Graal, de modo que ninguém soube de onde
viera e por onde retornara. Então ganharam poder para falar como antes e
agradeceram ao Nosso Senhor pela grande Graça e honra que Ele lhes havia feito,
por que Ele os havia saciado da santa graça do Santo Graal. E sobre todos eles
o Rei Arthur era o mais feliz, pela misericórdia que Nosso Senhor lhe fizera
perante todos os reis que à sua frente estavam. 

Por essas coisas alegraram-se os forasteiros e os pátrios, quando pensaram que
Nosso Senhor não os havia esquecido, quando Nosso Senhor fez a eles tão grande
graça e muito falaram sobre aquilo, tanto quanto durou a ceia. E o rei mesmo
falou sobre isso com os que lhe estavam próximos e falou: “Senhores, devemos
simplesmente ter grande alegria, pois Nosso Senhor nos demonstrou grande
amizade, que ele nos queira enviar de sua graça em um tão alto dia como no dia
santo de Pentecostes”. “Senhor,” falou o Senhor Gawin, “ainda há outra coisa
aqui, que não sabeis: que cada homem foi servido de tudo aquilo que seu coração
desejava, como se fosse na casa dos Reis Magos, sem que fôssemos decepcionados
por não termos podido verdadeiramente vê-lo, quando tudo nos estava encoberto.
Por isso eu juro de toda forma começar amanhã cedo, sem mais tardar, todas as
coisas que devo entreter até realmente ter visto o que me foi aqui revelado, se
de qualquer forma puder ser, assim poderei retornar”. 

Pois que os da Távola Redonda ouvissem isso, levantaram-se todos de seus
assentos e fizeram o mesmo juramento que meu senhor Gawin havia feito, e
disseram que não retornariam nem mais descansariam até que devessem estar
assentados à mesa santa, onde estariam, de todo modo, todas as doçuras, como lá
tinham experimentado. E assim que o rei ouviu que assim tinham jurado, foi-lhe
muito desconfortável, pois sabia bem que não poderia impedir aquela demanda. E
falou para o meu senhor Gawin: “Hei, Gawin, mataste-me por causa do juramento
que fizeste! Tiraste-me a maior e mais bela companhia que já ganhei, e é a
sociedade da Távola Redonda. Pois bem sei que, tão logo vos afasteis de mim,
que nunca tão logo e tão prontamente retornareis quanto cavalgastes para fora.
Pois bem sei que a maioria de vós morrerá nessa demanda, pois ela não tomará
fim tão logo quanto pensais. Portanto não me dói pouco, pois todos os
meus dias vos honrei e incentivei a todas as minhas fortunas, como se fôsseis
meus filhos ou meus irmãos. Por tal me dói tanto vossa separação, pois me
acostumei a estar convosco e vossa companhia e não posso saber como deverei me
consolar”. Após esta fala, levantou-se o rei em pensamento muito duro,
e nesse pensamento saíram-lhe as lágrimas dos olhos, que todos os que na corte
estavam bem notaram. E quando pôde falar, clamou tão alto que todos da corte
ouviram: “Hei, Gawin, Gawin, fizeste-me o maior sofrimento e o maior lamento,
que nunca posso superar, até que saiba ao certo que fim deve tomar a demanda;
pois tenho grande preocupação de que nenhum meu amigo jamais retorne”. “Hei
Senhor,” falou Lancelot, “o que dizeis!? Um tal homem como vós não deve levar
preocupações em seu coração, pois justiça e equidade têm boa esperança e bom
consolo, e devem consolar-vos. E deveras, se todos morrermos nessa demanda,
seria para nós maior honra que em qualquer outro lugar”. “Lancelot”, falou o
rei, “foi o grande amor que vos tive todos os meus dias, que me fez falar essa
fala, não admira que eu esteja irado. Se nenhum rei na Cristandade ganhou
tantos bons cavaleiros para sua távola como hoje eu tive neste dia, e nunca
mais voltarem como hoje estiveram, então é o maior desassossego o que tenho”. 

A esta fala não soube Gawin responder, e bem sabia que o rei dizia verdade, e
teria lamentado a fala que pronunciou, se tivesse ousado. Então não pôde ser,
pois viera temerariamente, e viera em todas as câmaras, que estavam na corte,
como eles assumiram a demanda do Santo Graal. E quiseram despedir-se pela manhã
os que deveriam ser companheiros e havia muitos mais que estavam mais irados
por isso que pela boa coragem. E por causa dos companheiros da corte, estava o
rei Arthur temeroso sobre todos os reis. 

Pois que as damas e donzelas que com o rei estavam assentadas, mais isso
ouviram, zangaram-se e entristeceram-se muito, sobretudo as que eram esposas ou
enamoradas dos companheiros da Távola Redonda. E isso não foi maravilha, pois
elas estavam temerosas por eles, pois temiam que morressem na demanda. Então
tiveram grande lamento a externar, e a rainha perguntou ao servo que ante ela
se encontrava: “Dizei-me, estáveis lá, quando a demanda foi jurada?” “Sim”,
falou ele, “Senhora, meu senhor Gawin o jurou e Lancelot e todos os
companheiros. E o senhor Gawin o jurou primeiro e meu senhor Lancelot, e assim
fizeram todos os outros da Távola Redonda que lá estavam”. Pois que ouviu essas
coisas, entristeceu-se por causa de Lancelot, pois que pensou que ela [própria] fosse
morrer, e não pôde conter-se, as lágrimas saíram-lhe dos olhos. E respondeu por um
longo momento, estando tão triste, que mais triste não poderia estar: “É
deveras uma grande pena”, falou ela, “que sem a morte assim de alguns nobres
cavaleiros nunca se vai completar, porque tantos cavaleiros se maravilharam
dessa demanda. E muito me impressionou que meu senhor, o rei, que é como um
sábio homem, tenha-o permitido. Se a melhor porção de seus heróis deve
separar-se tão maravilhosamente, poucos bons deverão permanecer”. E principiou
a chorar muito maravilhosamente, e todas as damas que com ela estavam. 

E estavam todos da corte tristes por causa dos mais que se deveriam separar. E
tão logo as mesas foram retiradas no palácio e as damas se haviam levantado e
se juntado aos cavaleiros, então se alçou o lamento de cada dama ou donzela,
fosse esposa ou enamorada, falou a seu cavaleiro que queria acompanhá-lo na
demanda. E havia suficientes na corte que facilmente se teriam deixado superar
ou convencer, e não fosse um nobre ancião feito, que estava vestido em trajes
santos e veio depois da ceia. E tão logo veio perante o rei, então clamou tão
alto que todos puderam bem ouvir, os que ali estavam: “Ouvi, companheiros da
Távola Redonda, Nascius, o eremita, vos apregoa que ninguém nesta demanda
levará dama ou donzela, sem que caia em pecado mortal, e que nenhum homem deve
adentrar, sem antes se penitenciar. Porque ninguém deve adentrar tão alto
serviço, que não tenha antes se purificado e lavado de todo o pecado mortal e
toda iniquidade. Pois essa demanda não deve ser de coisas maravilhosas, senão
que deve ser de arraigado amor de Nosso Senhor, do alto ensinamento que o
Mestre brevemente deve ensinar aos bons cavaleiros, que, por excelentes, tomou
para seu serviço dentre outros cavaleiros do mundo, a quem ele deve franquear
as grandes maravilhas do Santo Graal e deve deixá-los ver o que nenhum coração
mortal poderia imaginar. Por virtude dessas coisas, não conduza ninguém sua
esposa ou enamorada consigo”. 

E o rei fez hospedar o bom homem, bem e corretamente, e o rei lhe perguntou
muito sobre seu ser. E ele lhe respondeu pouco, pois que tinha outras coisas a
pensar por causa daquilo. E a rainha veio a Galaat e sentou-se junto a ele e
lhe perguntou muito de onde ele era. E ele disse a ela uma grande parte do que
ele sabia ser, sem que ele fosse filho de Lancelot, nisto não expulsou nenhuma
palavra. Ainda então em sua fala percebeu a rainha seguramente bem que ele era
filho de Lancelot, e que ele o tinha ganho com a filha do rei Pellis,  pois
que disso tinha ouvido fartamente falar. Por causa disso, queria ouvi-lo de sua
própria boca, e por isso lhe perguntou a verdade de seu pai. E ele respondeu
que não o sabia muito bem de quem era filho. “Hei, Senhor,” falou ela,
“escondeis-me o fato, por que o fazes? Nunca de vosso pai deveis ganhar
vergonha, quando é o mais belo cavaleiro do mundo, e veio, de todo lado, de
reis e rainhas e da mais alta linhagem que se sabe, e teve até agora o louvor
das outras e dos melhores do mundo. Por causa disso ireis perante os melhores
cavaleiros do mundo, se o igualares tão bem quanto nenhum simplório cavaleiro
daqui, ele vos reconhecerá bem quando vos vir”. Quando Galaat ouviu
essa fala, enrubesceu-se muito e muito se envergonhou, e de pronto lhe
respondeu: “Senhora, estais tão seguramente sabida de quem eu seja, então bem
podeis me dizer. E é aquele a quem considero que seja meu pai, quero que
tenhais dito verdadeiro; se não o for, então não posso seguir por causa de
nenhuma coisa do que me dizeis”. “Em verdade”, falou ela, “desde que não o
quisestes dizer, então quero eu vos dizer: aquele que vos ganhou, é o meu
senhor Lancelot do Lago e o mais belo cavaleiro e o mais prezado cavaleiro que
já nasceu em nossos anos. Por essa causa penso que não deveis esconder nem de
mim nem de ninguém, quando de melhor cavaleiro e mais nobre não poderíeis ter
sido ganho”. “Senhora”, falou ele, “desde que tão bem sabeis, por que vos devo
dizer? Isso se tornará sabido a tempo”.

Longamente falaram um com o outro e por tanto tempo até um bom momento da noite.
E porque era tempo de ir dormir, e o rei tomou Galaat por sua mão e o conduziu
à sua câmara e o fez jazer sobre a cama em que ele mesmo costuma ajustar-se,
por grande honra a ele e por que nele via grande maravilha. E após foram o rei
e Lancelot dormir e todos os grandes da corte. E pela noite esteve o rei em
grande apreensão e pensou muito que os bons e os nobres de sua corte pela manhã
deveriam separar-se e cavalgar para onde ele bem pensava que fossem permanecer
longamente. E por causa de sua longa permanência, estava ele em tristeza,
quando ele bem pensou que deles a maior parte deveria ficar na demanda. Essa
era a coisa pela qual ele estava triste.

Assim em grande lamento e desta maneira estavam os heróis todos e todos aqueles
que eram do reino de Logres. E porque Nosso Senhor Deus quis que o escuro da
noite passasse e que despontasse o dia e brilhasse luz, os cavaleiros
levantaram-se de pronto, todos que haviam assumido a demanda e vestiram-se e
prepararam-se. E pois que se tinham prontos, levantou-se o rei de sua cama. E
quando se tinha vestido, foi para a câmara onde o senhor Gawin e Lancelot
permaneceram, um ao lado do outro, à noite. E ao vir ele, eles se tinham
vestido e aprontado e queriam ir para a missa. E o rei, que lhes tinha amor
como se fossem seus filhos, saudou-os quando foi a seu encontro. Eles se
levantaram perante ele e o chamaram por bem-vindo, e ele os fez sentarem-se e
sentou-se junto a eles. E começou a contemplar o senhor Gawin e falou: “Gawin,
Gawin, vós me traístes, quando minha honra nunca foi por vós tão melhorada
quanto agora é diminuída, quando eu nunca mais serei assim honrado por uma tão
grande companhia quanto vós me tomastes com vossa insensatez. Ainda estou
triste por causa deles quanto estou por vós dois, quando todo o amor que jamais
nenhum homem ganhou por outros eu vos tive, e agora não se levanta, senão do
tempo em que conheci a grande nobreza que em vós estava albergada”. 

Pois que o rei tinha dito essa fala, calou-se então e pensou seriamente e no
pensamento caíram-lhe as lágrimas sobre seu rosto. E quando eles viram essa
coisa, que ele estava irado, ficaram tão tristes como ninguém poderia dizer e
não se atreveram a responder, pois eles o viram em cólera. E ele permaneceu um
bom momento em ira, e quando pôde falar, então falou muito tristemente: “Hei,
Senhor Deus, nunca pensei que jamais devesse merecer separar-me dessa companhia
que Deus me emprestou”. Depois falou a Lancelot: “Eu vos exorto pela verdade
entre mim e vós que me auxilieis a aconselhar sobre essas coisas”. “Senhor,”
falou ele, “eu vi jurar tanta gente nobre que não creio que isso possa ir à
frente de qualquer maneira, pois não haveria ninguém que não tivesse vergonha e
seria grande inverdade que deixasse a demanda”. “Em minha verdade”,
falou o rei, “sei bem que dizeis o verdadeiro, quando o grande amor que vos
tenho mo fez dizer; quando se não fosse isso o jurado, bem queria que se
entendesse, o quanto me dói daqui separar-me”.

Assim longamente falaram juntos até que foi alto dia, e que o sol tivesse em
grande medida derretido o orvalho, e o palácio começou a encher-se dos heróis
do reino. E a rainha se tinha levantado e veio até o rei e disse: “Senhor, os
cavaleiros esperam lá dentro para que vão à missa”. E ele se levantou e secou
os olhos, para que os que o vissem não percebessem o grande lamento que ele
tinha sofrido. Meu senhor Gawin deixou trazerem-lhe suas armas, e meu senhor
Lancelot. E por estarem bem armados, sem seus escudos, e vieram ao
palácio e acharam os companheiros, eles também estavam prontos e queriam ouvir
missa. E ao ouvirem missa no monastério, assim armados, foram de novo ao
palácio e assentaram-se, uns e outros que eram companheiros para a demanda.
“Senhor,” falou o rei Bandirs para o rei Arthur, “desde que a coisa foi tão
duramente empreendida, que não se pode sair, então sugiro que se faça trazer os
santos, que os companheiros façam tal juramento, que é devido, aqueles que
querem ir à demanda”. “Eu bem confesso”, falou o rei, “é como dizeis e não se
pode conter”. Então se fez o escrivão trazer os santos com que se costumava
jurar na corte. E pois que se os tinha trazido perante o mestre, então falou o
rei: “Senhor Gawin, vinde para cá, sois o primeiro e o impulso na demanda, por
isso fareis o primeiro juramento!” “Senhor,” falou o rei Bandirs, “ele não
deve fazê-lo por primeiro, pois o que deve fazê-lo é aquele que consideramos
por senhor e mestre da Távola Redonda. É o meu senhor, senhor Galaat”. Tão logo
o faça, devemos todos fazer esse juramento como ele fez, sem contradição”.
Então se chamou Galaat, e esse veio e ajoelhou-se perante os santos e jurou
como verdadeiro cavaleiro que ele por ano e dia e ainda mais, que se não o
fizesse, nunca deveria vir à corte, ele sabia, pois, ao certo do Santo Graal,
devesse isso ser ou não de alguma maneira. Depois jurou Lancelot tal
juramento como Galaat tinha jurado, e depois meu senhor Gawin e Parsifal e
Bohort e Leonel e Ylays, o Belo. E então juraram os companheiros da
Távola Redonda, um após o outro. E pois que tinham todos jurado e subscrito,
acharam que lhes eram cento e cinquenta, que todos eram bons cavaleiros e
nenhum hesitava dentre eles. Então comeram um pouco por causa do rei, pois
assim lhes pedira. E quando tinham comido, retiraram seu elmo e apoiaram sobre
sua armadura, e era uma coisa certa de que não mais poderiam ficar. E tomaram
licença do rei e encomendaram a rainha a Deus com olhos gritantes. E pois que
ela viu que queriam separar-se e não mais podiam permanecer, começou a padecer
de grande lamento, como se seus amigos estivessem mortos ante seus olhos. E por
que ela não queria que se o percebesse, então adentrou sua câmara e caiu sobre
sua cama. Então começou a padecer do maior lamento, que nenhum homem tanto se
endureceria, se o tivesse visto, que não se teria compadecido. E porque
Lancelot estava pronto para montar, teve grande tristeza por causa da ira de
sua dama, que ninguém poderia ter estado mais colérico, voltou-se para a câmara
aonde a vira entrar, e foi-se para ela lá dentro. E porque a rainha o viu vir
armado, começou a gritar: “Hei, Lancelot, bem me mataste, pois deixais a corte
do meu senhor e seguis para terra estranha de onde ninguém retorna”. “Senhora,”
falou ele, “devo muito mais brevemente, se Deus quiser, retornar para cá, do
que pensais”. “Hei, Senhor Deus,” falou a rainha, “meu coração não me diz isso,
estou no maior lamento e medo do mundo ao qual uma dama nobre pode vir por
causa de um homem”. “Senhora, para lá devo sair com vossa licença.” “Quando
quiserdes”, falou ela, “nunca seguireis para lá com minha vontade; porque isso
tem que ser, assim ide sob a guarda daquele que se deixou martirizar na Santa
Cruz, para redimir o pecador da morte eterna, que vos acompanhe e proteja em
todos os confins a que vieres!” “Senhora”, falou ele, “Deus o faz por sua
santa misericórdia”. 

Com isso separou-se Lancelot da rainha e baixou à corte e achou que seus
companheiros outra coisa não fazem senão esperar por sua montaria. E ele foi ao
seu cavalo e montou. E o rei viu que Galaat queria seguir sem escudo na
demanda, como os outros. Veio até ele e falou: “Senhor, parece-me que não
fazeis o suficiente se não conduzires daqui um escudo, como os outros
companheiros”. “Senhor,” diz ele, “eu agiria de modo errado se tomasse um
daqui, quando nenhum nunca pendeu de meu pescoço, já que a aventura me daria”.
“Pois vos acompanhe Deus”, falou o rei, “quando me calo se não pode ser de
outro modo”. Estavam os grandes montados, e os cavaleiros, e cavalgaram
embora uns e outros, e cavalgaram pela cidade, até que viessem ao campo, e
nunca viram tamanho lamento e gritaria quanto a que faziam os da cidade
conjuntamente. E porque viram que os companheiros que estavam na demanda do
Santo Graal deles se separavam, não houve ninguém, nem pobre nem rico, de todos
os graves que lá deviam permanecer, que não derramasse quentes lágrimas. Quando
tiveram assim grande tristeza da separação, quando cada um dos que queriam
partir não fez como se estivessem tristes, e fizeram como se estivessem
felizes, como também estavam. E pois que tinham vindo à floresta ante o castelo
Nagari, detiveram-se perante uma cruz. Então falou meu senhor Gawin para o rei:
“Já cavalgaste suficientemente longe, deveis retornar e deveis fazê-lo, pois
sois aquele que não nos deve acompanhar”. “O retornar pesa-me mais que o vir
para cá, quando também me dói separar-me de vós. Pois eu vejo, porém, que não
pode ser diferente, então devo retornar”. Então meu senhor Gawin
retirou seu elmo de sua cabeça, e assim fizeram todos os outros. E depois,
porque tinham todos seu elmo afrouxado, então encomendaram uns aos outros ao
Nosso Senhor, em meio a quentes lágrimas. E de pronto separaram-se de uma tal
maneira que o rei cavalgou para Camelot e os companheiros para a floresta. E
cavalgaram até que vieram ao castelo Nagares\footnote{ O texto
alterna os nomes Nagari e Nagares, ora conservados em suas posições originais
na narrativa.}. Nagares era um nobre e de boa vida e era um dos
melhores cavaleiros do mundo quando estava em juventude. E quando viu que os
companheiros cavalgavam por sua fortaleza, fez fechar os portões em todos os
lados e falou, pois Nosso Senhor lhe tinha feito a graça, de que os tivesse em
seu poder, assim nunca viriam adiante, que ele não lhes tivesse feito a maior
honra e o maior serviço, com todas as suas possibilidades. E desta maneira os
reteve em seu poder e os fez desarmar, e os serviu pela noite tão bela e
ricamente que todos eles se maravilharam de onde ele tomava os bens. Então se
aconselharam pela noite o que gostariam de fazer de manhã. E veio a conselho
que queriam pela manhã separar-se, e cada um queria cavalgar por sua via
especial, pois pensaram que lhes acarretaria vergonha que cavalgassem uns com
os outros.

Pela manhã, tão logo o dia brilhou, levantaram-se os companheiros e armaram-se e
foram ouvir missa, em uma capela, que estava na casa. E quando tinham ouvido a
missa, sentaram-se sobre seu corcel e encomendaram o senhor na casa ao Nosso
Senhor Deus e lhe agradeceram muito pela honra que lhes havia feito. E
cavalgaram para fora do castelo e na hora se separaram uns dos outros como
tinham dito. E cavalgaram através da floresta, um aqui, outro acolá, onde viram
de todo espesso por todos os confins que viam, caminho ou vias. E choraram
muito na separação aqueles que, mais que todos, consideravam ter coração duro
ou os mais corteses. Então se silencia a fala adiante sobre os outros e diz de
Galaat, quando era autor desta demanda. 
\oneside

\chapter[O escudo branco com a cruz vermelha]{O escudo branco\break com a cruz vermelha}

\textsc{Aqui se dizem} mais coisas, pois Galaat estava separado de seus companheiros, que
ele cavalgou três dias ou quatro, que ele não encontrou nenhuma aventura que
não esperasse. E no quinto dia, após as vésperas, assim aconteceu-lhe que seu
caminho o portasse a um certo convento. E quando lá entrou, bateu aos portões.
E os irmãos de dentro correram para fora e o fizeram desmontar com autoridade,
pois bem reconheceram que ele era um dos valorosos cavaleiros. Um tomou seu
cavalo, outro o conduziu ao salão, que se elevava sobre a terra, para
desarmá-lo. E pois que o tinham desarmado, ele viu dois companheiros da Távola
Redonda, um era o rei Bandirs e outro, Ywan. E tão logo o reconheceram,
correram para lá com braços abertos e o abraçaram e o proclamaram bem-vindo e
lhe fizeram a maior honra que sabiam quando estavam alegres que o tinham
achado, e se lhe fizeram reconhecer. E ele lhes fez em troca grande alegria,
pois os tinha por irmãos e companheiros. 

Pela noite, pois que tinham comido e ido jogar lá embaixo, sob uma árvore,
então lhes perguntou Galaat que aventura os tinha ali portado. “Valha-me Deus”,
falaram eles, “viemos cá para contemplar uma aventura; é por vez maravilhoso o
que se nos fez entender, que há neste convento um escudo que ninguém pode
portar em seu pescoço. Pois tão logo o tome e o queira portar, vai-lhe mal no
primeiro dia ou no segundo, seja morto, ou ferido ou encantado. E por isso
viemos cá para descobrir se é verdadeiro ou não o que se diz”. “Quando quero
seguir amanhã cedo”, falou o rei Bandirs, “então devo saber se a aventura é
verdadeira como se diz”. 

“Em verdade”, falou Galaat, “dizei-me maravilha! A aventura é como me narraram e
disseram, que não lograste conduzi-lo, então que sou aquele que o deve
conduzir, quando não tenho nenhum escudo”.  “Senhor,” falaram eles, “queremos
deixá-lo para vós, quando não falhardes na aventura”. “Quero”, falou ele, “que
façais a primeira tentativa, para saber se a aventura é assim verdadeira como
se diz”. Então o seguiram ambos. 

À noite se o fez confortável e serviu-se-lhe de tudo o que tinham conhecido.  E
muito vastamente honraram os irmãos a Galaat, pois ouviram a palavra que dele
davam os cavaleiros. Então o deitaram rica e magnificamente, como se deve fazer
simplesmente a um homem como ele era. E de noite deitou-se junto a ele
o rei Bandirs e seu companheiro na câmara. E de manhã, pois que tinham ouvido
missa, pergunta o rei Bandirs a um dos irmãos onde estava o escudo, do qual se
diz tanta maravilha na terra. “Senhor,” falou o bom homem, “por que causa me
perguntais por ele?” “Pela causa”, falou ele, “de que quero levá-lo comigo,
para saber se ele tem tal força como dele se diz”. “Eu não vos aconselho”,
disse o bom homem, “que o leveis daqui, quando creio que não vos acontecerá nem
virá nada a não ser vergonha”. “Seja como for, ainda assim o quero ter”, disse
ele. Então o conduziu para trás do altar principal do convento e achou
um escudo branco com uma cruz vermelha. Então falou o bom homem: “Aqui está o
escudo, pelo qual perguntaste”. E ele o contemplou e falou que nunca tinha
visto escudo tão precioso ou belo, e cheirava tão bem como se todas as ervas do
mundo tivessem sido espalhadas sob ele. Pois que Ywan o contemplou, então falou
ele: “Então me ajude Deus, este é o escudo que ninguém deve levar em seu
pescoço, pois que seria muito melhor cavaleiro que um outro. E por causa disso
nunca ele virá ao meu pescoço, quando não sou engrandecido e assim nobre para
que o deva levar”. “Em verdade”, falou o rei Bandirs, “eu devo
portá-lo, aconteça-me o que acontecer”, e o pendurou em seu pescoço e o portou
para fora da igreja. E pois que veio a seu cavalo, então falou ele para Galaat:
“Senhor, eu bem queria que aqui me esperásseis, para que soubésseis dizer o que
me aconteceu desta aventura, quando, assim que me for mal, queria bem que o
tivésseis, quando facilmente superardes a aventura”. “Devo esperá-lo
com prazer”, falou Galaat. E de pronto sentou-se, e os irmãos no convento lhe
emprestaram um escudeiro, que lhe fez companhia e que lhe trouxesse de volta o
escudo, se preciso fosse. 

Assim permaneceram Galaat e o senhor Ywan, que lhe fez companhia, por tanto
tempo até que soubessem a verdade destas coisas. E o rei Bandirs levantou-se
para seu caminho, ele e seu escudeiro, e bem cavalgaram duas milhas e vieram a
um penhasco, na frente de uma ermida. Ele viu que de lá vinha um cavaleiro,
armado com arma branca, e veio tão logo cavalgando quanto pôde e segurou sua
lança à frente e veio caminhando perante ele. O rei Bandirs virou-se perante
ele tão logo o viu vir e trouxe sua lança sobre ele, de modo que os pedaços
voassem ocultos. E o bom cavaleiro o encontrou duro, de forma que se lhe
romperam os anéis da coifa, que o ferro foi através do lado esquerdo e o
encontrou tão duramente como aquele que tem grande força, que o conduziu do
cavalo ao chão. Com a queda que ele caiu, então tomou o cavaleiro branco o
escudo ao rei do pescoço, e falou tão alto que se podia bem ouvir e que o
escudeiro o compreendesse: “Senhor cavaleiro”, falou ele, “fizeste como um tolo
e bufão, que pendurásseis este escudo em vosso pescoço, quando não é permitido
a nenhum cavaleiro portar, a não ser ao melhor do mundo. E por causa do pecado
que fizestes, então me enviou para cá Nosso Senhor Deus para tomar vingança por
causa deste equívoco”. E pois que o tinha feito, falou ao escudeiro:
“Vê este escudo e o porta ao servo de Nosso Senhor Jesus Cristo, ao bom
cavaleiro que se chama Galaat, que deixaste no convento. E diz-lhe que lhe roga
o mais alto mestre que ele o leve. Ele o deve de todas as formas achar tão
fresco e novo e pois bom quanto está agora. E isto é uma coisa que o vincula a
ter-lhe amor, e saudai-o por causa de mim tão logo o virdes”.  E ele
lhe perguntou como se chamava, para que pudesse dizer ao cavaleiro, quando a
ele viesse. “De meu nome não podes saber, quando não é coisa que se possa dizer
a qualquer homem terrestre, e por causa disto deves dele carecer e fazer como
te chamei”. “Senhor,” falou o escudeiro, “porque não me quereis dizer o
vosso nome, peço-vos por causa do maior amor que tendes no mundo que me
queirais dizer a verdade desse escudo e como ele foi trazido a esta terra e por
qual causa alguma maravilha pode daí vir, quando ninguém em nosso tempo em seu
pescoço o pendura, que não lhe venha daí um mal”. “Como muito me reclamaste que
eu devo dizê-lo, e não apenas a ti, quero que aqui me tragas o bom
cavaleiro, aquele que deve portar o escudo”. E ele diz que o quer fazer com
prazer, e que ele lhe dissesse onde deveria encontrá-lo quando para ali
retornasse. “Neste sítio”, falou o cavaleiro, “onde agora estamos”.  

Então veio o escudeiro ao rei Bandirs e perguntou-lhe se ele estava ferido.
“Sim”, falou ele, “muito duramente!” “Gostaríeis bem de cavalgar?”, falou o
escudeiro. E ele se endireitou e falou que queria tentar, por ferido que
estivesse. E o escudeiro o ajudou por tanto tempo até que ele veio a seu
corcel. E ele montou, e o escudeiro sentou-se atrás dele para mantê-lo, quando
de outro modo não conseguiria se ter mantido, teria certamente caído por terra.
Pois que fizeram isso, então se separaram do sítio, pois o rei estava ferido, e
cavalgaram por tanto tempo até que vieram ao convento, do qual se tinham
apartado. E porque foram alertados de que eles vinham, saltaram perante eles e
ajudaram o rei Bandirs a desmontar e o conduziram a uma câmara e ataram suas
feridas, que eram grandes e horríveis. E Galaat perguntou a um dos irmãos, que
se indagava se ele poderia convalescer. Ele falou: “Ele bem deve convalescer,
se Deus quiser, e ele está quase muito ferido, e isso ele não deve queixar a
ninguém, pois lhe predissemos que a quem tomasse o escudo, viria mal. E ele
o levou apesar de nosso mandamento, por isso deve ser havido por tolo”. E pois
que lhe fizeram no convento o melhor que podiam, falou o escudeiro a Galaat tão
alto que todos os que lá estavam ouviram: “Senhor, a vós faz saudar o bom
cavaleiro com a arma branca, que feriu o rei Bandirs, e vos envia este escudo e
vos pede que o conduzis no futuro por causa do Alto Mestre, quando não sabe
ninguém que o porte com mais direito que vós mesmo; por causa disso, ele vo-lo
enviou comigo. E se quereis saber de onde vem a aventura que assim aconteceu
aqui fartamente, então vinde a ele, e ele vos deve narrar como me prometeu”.  

E pois que os irmãos ouviram as novas, ofertaram a Galaat grande honra e falaram
que abençoada fosse a hora que lá o trouxera, quando bem reconheceram que a
grande e maravilhosa aventura por meio dele deveria ser ultimada. E o
senhor Ywan falou a Galaat: “Pendurai este escudo em vosso pescoço, quando não
foi feito para nenhuma pessoa senão para vós! Assim minha vontade é em alguma
medida realizada, quando nenhuma coisa desejei mais saber que conhecer o
cavaleiro que fosse digno de levar esse escudo”. E Galaat falou que queria
pendurá-lo em seu pescoço, porque ele lhe tinha sido enviado. E chamou-se [a si mesmo] para
a primeira arma e chamou a trazer sua arma. E quando estava armado, então
pendurou o escudo em seu pescoço e montou em seu corcel e separou-se deles e
encomendou os irmãos a Nosso Senhor Deus. E Ywan estava armado e assentado
sobre seu cavalo e disse que queria fazer companhia a Galaat. E ele respondeu
que não poderia ser, que para lá não precisava de nenhuma companhia que não
deste escudeiro. E assim se separou dos outros, e cada qual cavalgou por seu
caminho. E o senhor Ywan cavalgou para a floresta. E o cavaleiro com a arma
branca, que tinha visto o escudeiro, estava próximo dele. Eles o saudaram, e
pois que viu o senhor Galaat, validou sua saudação, tão bonita e belamente
quanto só ele podia, e ganharam conhecimento e um entreteve o outro. Então
falou Galaat: “Por causa do escudo que devo levar ocorreu alguma aventura e
muito maravilhosa nesta terra, como ouvi dizer. Assim quero com prazer vos
pedir por reto amor, que me digais a verdade como e por que isso aconteceu,
quando bem creio que bem o sabeis”. “Senhor,” falou ele, “bem quero com prazer
dizer-vos, se o quereis, quando bem o sei”. “Senhor Galaat”, falou ele,
“aconteceu doze anos após o martírio de Nosso Senhor, que José de
Arimateia,\footnote{ Apesar de nossa opção por conservar os nomes próprios das
personagens e localidades narradas no original alemão, como grafado no século
\textsc{xiii} nomes canônicos como José de Arimateia e as personagens bíblicas foram
traduzidos para o português, para facilitar a identificação do leitor.}, 
o nobre cavaleiro, que alçou Nosso Senhor da cruz, separou-se de
Jerusalém com uma grande parte da linhagem de que ele era. E longamente erraram
até que vieram a um caminho que os portou para a cidade de Saras, onde estava o
rei Evalles, que era pagão. E quando José veio a Saras, teve Evallet guerra com
um seu vizinho, um rei rico e poderoso, que confinava com sua terra, que era
chamado Thulomeus. E pois que Evallet tinha vindo próximo e queria submeter
Thulomeus, que para si ambicionava sua terra, Josephus, filho de José, falou
para ele que se viesse desaconselhado para a batalha, seria ferido e
envergonhado por seus inimigos. “Como me aconselhais?” falou Evallet. “Quero
bem dizer-vos”, falou Josephus e principiou e diz-lhe da Nova Aliança e do
verdadeiro Evangelho e da crucificação de Nosso Senhor e da ressurreição e
diz-lhe a verdade, e fez trazer-lhe um escudo branco com uma cruz vermelha de
cindel e falou: ‘Senhor Rei Evallet, quero indicar-vos como deveis reconhecer o
poder e a força da Santa Cruz. Isto é verdade, Thulomeus deve ter poder sobre
vós, três dias e três noites, e tanto que não pensai em escapar. Assim deveis
descobrir a Cruz e deveis falar: ‘Prezado Senhor, de cuja morte trago sinais,
ajuda-me deste medo e escolta-me são e sem dano, então creio servir e conservar
tua aliança’. Depois separou-se deles o rei e rumou para Thulomeus. E aconteceu
certeiramente como lhe tinha dito Josephus. E lá viu tal medo que pensou que
morreria, daí descobriu o escudo onde, no meio, estava um crucifixo, e falou a
palavra que Josephus lhe tinha ensinado. Então lhe aconteceu sorte e honra, e
ficou protegido de todos os seus inimigos e ganhou a vitória sobre Thulomeus e
todos os seus homens. E quando veio ao sítio de Saras, ele diz a todo o seu povo
a verdade do que tinha achado. E José cobriu então o escudo com a cruz, e falou
que queria tornar-se cristão\footnote{ O nome contido na versão original do
documento é, de fato, \textit{Joseph}, apesar de apresentar-se mais coerente a
correção apresentada pelo tradutor alemão contemporâneo Hans Hugo Steinhoff,
que consigna o nome \textit{Evallet}. A mencionada tradução alemã prestou-se a
uma comparação, muitas vezes necessária, para a presente faina em língua
portuguesa.}. E antes que fossem batizados, veio um homem perante
eles, com uma mão decepada e trazia o pulso na outra mão. E José o chamou, e
ele veio até ele. E tão logo ele o tocou com a cruz que estava sobre o escudo,
então achou sua mão de novo convalescida, que tinha perdido. E ainda aconteceu
uma maravilhosa aventura, quando a cruz, que estava sobre o escudo, dele então
se separou de tal maneira que ninguém conseguiu perceber onde estaria. 

Então recebeu Evallet o batismo e ficou chamado Mordelas e tornou-se servo de
Jesus Cristo e fez manter em honra o escudo. E depois aconteceu que Josephus
separou-se do sítio de Saras, ele e seu pai. E pois que tinham vindo à
Grã-Bretanha, lá acharam um rei, que era fartamente cruel e mau, que tomou a
ambos como prisioneiros, e uma grande parcela de cristãos. Quando Josephus
estava na prisão, então vieram as novas ao longe, quando ao tempo ninguém tinha
tão grande nome como ele. E posto que o rei Mordelas foi alertado, pôs-se a
caminho e seu povo e o senhor Natages seu cunhado e vieram à Grã-Bretanha por
cima de quem mantinha Josephus preso. E desertificaram e apodreceram a ele e
toda sua terra, e tornou-se a terra cristã. E tiveram Josephus em tanto amor
que nunca depois desta vez se separou e com ele permaneceram, e o serviram
sobre toda a terra para onde se conduzia. E então veio que Josephus estava
deitado em sua cama mortuária, e Evallet, que era chamado Mordelas, viu que ele
queria se separar do mundo, e então foi perante ele e chorou muito amargamente
e falou: ‘Senhor, porque quereis pois nos deixar, devo então aqui permanecer só
nesta terra, e por causa de vosso amor  deixei toda minha terra e todo o meu
reino? Por meio de Deus, porque deveis daqui vos separar, deixai-me algum
sinal, para que eu tenha algo que o lembre!’ ‘Senhor,’ falou Josephus, ‘devo
bem fazê-lo’. E começou a refletir sobre o que poderia deixar-lhe. E posto que
muito tinha pensado, então disse ele: ‘Rei Mordelas, faz-me trazer o mesmo
escudo que levaste sobre Thulomeus’. E o rei falou que queria com prazer
fazê-lo, ‘quando não está longe daqui’, quando ele o levava consigo de todas as
maneiras, levasse ele ou quem ele quisesse em qual terra. E fez trazer o escudo
perante Josephus. Pela mesma hora em que o escudo foi trazido perante Josephus,
aconteceu que Josephus sangrasse assim tanto pelo nariz, que ninguém o soube
acalmar, e tomou o escudo de pronto e faz uma cruz com o mesmo sangue no escudo
que agora vedes aqui. E sabei seguramente que é o mesmo escudo de que vos
disse. E quando tinha feito a cruz como aqui bem podeis ver, falou ele: ‘Tomai
este escudo o qual vos deixo para uma memória minha, que bem vedes que esta
cruz é feita do meu sangue. Deve de toda sorte ser assim fresco e novo e
vermelho tanto quanto durar o escudo, e isso não deve ser curto. E sabei que
ninguém deve levá-lo, quem quer que seja o cavaleiro, que vai se arrepender,
até que o bom cavaleiro Galaat [venha], que é da linhagem de Natigen, que deve
pendurar em seu pescoço. E por causa disso ninguém seja audacioso para
pendurá-lo em seu pescoço, senão aquele a quem nosso Senhor Deus  deve apanhar,
e por cuja vontade assim muito mais maravilha será vinda do escudo que de
outros, assim deve também mais nobreza e mais maravilha vir daquele que o deve
levar que de outros’. 

`Por que me deixais assim tão boa memória de vós’, falou o rei, ‘dignai-vos
também a dizer-me onde devo deixar o escudo, quando bem quero que ele seja
feito em tal sítio que o cavaleiro o encontre’. Respondeu ele: ‘Aí deveis ver
onde Natiges se fez enterrar quando morreu, lá deveis fazer o escudo, pois que
lá virá o bom cavaleiro no décimo-quarto dia, assim tendo recebido a ordem da
cavalaria’. E assim veio, quando hoje é o décimo-quinto dia em que viestes a
este convento, onde Nascius\footnote{ O documento altera o
nome do cadáver.}  se encontra enterrado. Então vos narrei por que
causa a aventura aconteceu e os cavaleiros de tola ousadia que por cima do
mandamento quiseram levar o escudo, que não era permitido a ninguém senão a
vós”. E pois que o falara, então desapareceu, que Galaat nem
soube [de] onde ele viera. E posto que o escudeiro que junto a ele estava escutou
esta aventura, então caiu ele de seu cavalo e caiu aos pés de Galaat e
pediu-lhe chorando, por meio do amor que ele tinha por aquele cujo sinal ele
levava no escudo, que o deixasse com ele cavalgar e que o fizesse cavaleiro.
“Seguramente”, falou Galaat, “se devo ter companhia, não vos negarei”.
“Senhor,” falou ele, “assim vos peço por Deus que me façais cavaleiro, quando
bem vos digo que a cavalaria em mim se deve constituir, se Deus quiser”. Galaat
contemplou o escudeiro, que misericordiosamente chorava, e dele muito se
apiedou, e pediu-lhe que quisesse fazê-lo. “Senhor,” falou o escudeiro, “então
nos deixai retornar para o lugar de onde viemos, quando lá tenho arma e corcel,
e deveis fazê-lo com justiça, não por causa de minha vontade, senão porque há
aí uma aventura, que ninguém consegue levar ao fim”. “Quero com prazer o
fazer”, falou Galaat e retornou por isso ao convento. Posto que os do convento
viram que ele voltava, então tiveram alegria e perguntaram ao escudeiro por que
causa tinham eles retornado. “Para fazer-me cavaleiro”, falou o escudeiro, e
ficaram muito alegres por ele. E o bom cavaleiro pergunta onde seria a
aventura. E os do convento falaram: “Sabeis o que é?” “Não”, falou ele. “Assim
sabei que é uma voz que vem de um sarcófago da terra para fora e é tão forte
que ninguém pode ouvi-la, que não perca força de seu corpo por um longo tempo
depois”. “E sabeis de onde vem esta aventura?” “Não”, falaram eles, “vem,
pois, do mau espírito”. “Então me conduzi para lá”, falou ele, “quando vejo com
prazer”. “Então ide conosco”, falaram eles. Então o conduziram a um canto da
igreja, assim armado, sem o elmo. Então falou um irmão: “Senhor, vede lá a
grande árvore e o sarcófago embaixo?” Sim”,  falou ele. “Então vos
quero dizer”, falou o irmão, “o que deveis fazer. Ide ao sarcófago e o erguei.
Eu vos digo que deveis embaixo encontrar uma letra”. Com isso foi
Galaat para lá e ouviu uma voz, que fez um grito tão grande que foi maravilha e
clamou tão alto que todos ouviram, os que lá estavam, e falou: “Servo de Jesus
Cristo, não te aproximes de mim, quando me fazes expulsar do sítio onde tenho
estado por longo tempo”. Quando Galaat o ouviu,  então não se assustou e
caminhou para o sarcófago. E posto que o quisesse  erguer em um canto, então
viu de lá sair uma fumaça e uma chama, e viu depois de lá sair uma figura, a
mais horrível e mais cruel em uma figura de homem, e fez uma cruz à sua frente,
quando bem sabia que era o Inimigo. Então ouviu uma voz, que falou: “Galaat,
santa pessoa, eu te vejo aqui com os santos anjos, que meu poder nunca pode ser
perante a sua força, eu te deixo este sítio”. E quando Galaat o ouviu, ele se
persignou e deu graças a Nosso Senhor Deus e ergueu o sarcófago para cima e
olhou para lá. Então viu um cavaleiro armado e tudo de que se precisa
para um cavaleiro. E quando o deduziu, então chamou os irmãos e falou: “Vinde
cá e vede o que achei e dizei-me o que é, pois estou pronto a fazer mais, se
devo fazê-lo”.  E foram todos para lá. E quando viram o cadáver lá
jazer, enterraram-no e falaram: “Senhor, não podeis fazer mais do que já
fizestes, o cadáver, que aqui padecia, não deve nunca ser transferido do sítio,
como consideramos”. “Não” falou o velho, que tinha dito a aventura para Galaat,
“quando ele deve ser retirado e arremessado para fora, quando esta terra é
abençoada e consagrada, e por causa disso o cadáver dos falsos e maus cristãos
aqui não pode permanecer jazendo”. E então chamaram os servos do
convento para que o erguessem para fora do túmulo e o jogaram para frente do
cemitério. E então falou Galaat ao irmão: “Senhor, fiz tudo isso que pertence a
esta aventura, que estou compelido a fazer?” “Sim”, falou ele, “quando nunca
vem voz de lá de dentro, de que aconteça tanto mal como desta”. “E
sabeis”, falou Galaat, “por qual razão tantas maravilhas de lá vieram?”
“Senhor,” falou ele, “sim senhor, e quero de bom grado dizer-vos, quando vos é
obrigatório saber e significa grande coisa”.

Com isto, separaram-se então do cemitério e retornaram ao convento. E Galaat
falou ao servo que ele deveria fazer vigília à noite na igreja, se deveria
mesmo fazê-lo cavaleiro pela manhã. E aquele, que nada desejava senão aquilo
que dele se demandava, preparou-se para receber a alta ordem da cavalaria,
quando tanto a desejou. E o bom homem tomou Galaat e o conduziu a uma
câmara e o fez desarmar e o fez sentar-se sobre uma cama e falou: “Senhor,
perguntava-me agora da aventura, que trouxestes ao fim, e quero de bom grado
vos dizer, quando nesta aventura repousavam três coisas que são para muito se
temer: o sarcófago não era fácil de se levantar, o cadáver do cavaleiro
precisava ser jogado para fora de sua forma, e a voz, que todo aquele que a
ouvisse perdia sua força e seu poder; e dessas três coisas devo eu dizer-lhe o
sentido. 

O sarcófago, que cobria o morto, significa a dureza do mundo, que nosso Deus
achou tão grande que veio para a terra, quando o filho não amava o pai, nem o
pai ao filho; por esta razão conduziu o Inimigo ao Inferno aqueles que do mundo
se separavam. E porque o Pai do Reino do Céu viu que era tão grande a dureza
sobre a terra, que um não reconhecia ao outro, nem acreditava no outro, nem
pelas prédicas que se lhe diziam, que achavam todos os dias novos deuses, e
enviou Seu Filho ao mundo, que o achou duro e pecaminoso. E para amolecer essa
dureza, e para amolecer os pecadores e os renovar, conduziu-se ao mundo e o
achou assim duro e assim cheio de pecados mortais, que se poderia mais rápido
amolecer uma pedra dura que o seu coração. Quando ele falou pela boca de Davi,
o profeta, ‘não serei vitorioso até que eu morra’, isto é tanto quanto ter dito
‘pai, terás bem pouca conversão deste povo antes da minha morte’. E assim, como
o Pai enviou Seu Filho para redimir o povo e é renovado convosco, quando assim
como o erro e a tolice fugiram à Sua chegada, e a Verdade brilhou e foi
reconhecida, assim também nosso senhor Deus vos elegeu sobre todo outro
cavaleiro para enviar pelas terras para completar as aventuras, e para
reconhecer como elas aconteceram. Por tal razão o vosso futuro deve igualar o
de Nosso Senhor Jesus Cristo, não de tão alto direito. Como os profetas foram
muito antes do advento de Nosso Senhor, e tinham predito sua vinda e falavam
que Ele deveria redimir o povo do tormento infernal, também se profetizou vosso
futuro há mais de vinte anos. E falaram muitos que ninguém deveria ouvir a
aventura de Logres até que vós viésseis. E desta forma, pois, bem nos aconteceu
que viestes pela graça de Deus”.  

 “Pois me dizei”, falou Galaat, “o que o corpo significa; do sarcófago me
tínheis dito a verdade”. Ele falou: “Eu devo dizer-vos isso. O corpo significa
a grande dureza do povo, que eles estavam todos mortos através dos grandes
pecados que tinham feito de dia a dia, quando bem parece que estavam cegos do
advento de Nosso Senhor Jesus Cristo. Quando então viram o rei acima de todo
rei, tomaram-no por um pecador e consideraram que fosse como eles eram e o
condenaram à morte pelo conselho do Diabo, que os enfeitiçou a todos, e que
lhes tinha ido aos lábios. E por isso fizeram isso. Então Vespasiano os
deserdou e expulsou tão logo soube a verdade dos profetas perante os quais eles
foram falsos. E então foram trazidos à morte com o Inimigo e seu conselho”. 

“Então devo também de bom grado saber como o transtorno e o lamento vieram ao
corpo, que à vez estava morto, e a voz do sarcófago”. Ele falou: “Isso
significa a terrível fala que veio à frente de Pilatos, o juiz, que Seu sangue
deveria ir sobre nós e nossos filhos. E por esta causa então eles foram
perdidos e perderam tudo que tinham. Assim podeis bem ver que essa aventura
significa a Paixão de Nosso Senhor e sinal de sua vinda futura. E ainda vos
digo mais: quando onde\footnote{ O texto original apresenta os dois pronomes
justapostos, exatamente como na tradução.}  cavaleiros forasteiros
para cá vinham e iam ao caixão e o Inimigo os reconhecia por pecadores de muito
tempo e os via impuros, e via que estavam completamente sem castidade, assim
lhes fazia ter grande medo de ouvir da voz e tanto se assustar que perdiam seu
poder de todo seu corpo ermo. E se seu poder nunca fosse superado, então por
causa dos pecadores, que estavam cheios de impureza, não teria Nosso Senhor vos
enviado para cá, por causa de trazer a aventura ao seu fim. Então tão logo para
cá viestes, o Diabo, que vos conhecia tão casto e tão limpo de todos os pecados
como nenhum homem terrestre pode ser, não ousou deter vossa companhia e fugiu e
perdeu todo seu poder de vossa vinda, e a aventura vos passou, pois que tantos
nobres cavaleiros a tinham tentado. Assim vos tenho dito a verdade dessas
coisas”. 

E Galaat falou que lá haveria maior significado do que ele próprio consideraria.
À noite foi Galaat servido do melhor que os irmão puderam. E pela manhã fez o
servo cavaleiro, como era costumeiro para o horário. E como tinha feito tudo
que estava obrigado a fazer, perguntou-lhe como se chamava. E ele falou que se
chamava Meliant e era filho do rei da Dinamarca. “Caro senhor cavaleiro”, falou
Galaat, “porque sois cavaleiro e vindes de tão alta linhagem como de reis, vede
então que a cavalaria em vós está bem constituída e honrais vossa linhagem,
então por causa de nenhum tormento, que se possa padecer, deve-se deixá-la''.
“Seguramente,” falou o novo cavaleiro, “Senhor, devo bem conservá-la, se Deus
quiser”. E então chamou Galaat suas armas, e se lhe trouxe. E Meliant falou:
“Senhor, por graça de Deus e vossa me fizestes cavaleiro, pois eu tenho tanta
alegria disso que não consigo dizer. Quando bem sabeis, quem faz um cavaleiro,
que ele não tem que recusar o primeiro pedido que lhe pede, se lhe pedir pedido
possível”. “Falais o verdadeiro”, falou Galaat. “Por que o dissestes?” “Por
que” falou ele, “eu vos quero desejar um dom e pedir que mo deis, quando é
coisa de que nunca virá mal”. “Eu vos dou”, falou Galaat, “mesmo que deva me
ferir”. “Grande graça.” falou Meliant, “Então quero pedir que me
deixeis seguir convosco na demanda até que alguma aventura se nos depare. E
depois, se a aventura nos portar juntos, que não me negueis vossa companhia
pela vontade de outrem”. “Quero de bom grado fazê-lo”, falou Galaat. Então
chamou Meliant que se lhe trouxesse um corcel, quando queria partir com Galaat.
E se lhe trouxe de pronto, e separou-se deles com Galaat e cavalgaram o dia
inteiro e a semana inteira. E então lhes sucedeu em uma terça-feira cedo que
viessem a uma cruz e encontrassem letras,  que estavam cortadas em uma madeira
e falavam: “Cavaleiro, escuta! Tu, que buscas aventura, vê aqui dois caminhos,
um para a mão direita e um para a mão esquerda. Pois o lado esquerdo te
proibimos, que por aí não venhas, quando pois precisa ser um nobre para aí vir,
para que logo deva sair”. Quando Meliant viu as letras, então falou para
Galaat: “Nobre cavaleiro, deixai-me cavalgar o caminho para o lado esquerdo,
quando lá posso tentar minha força e reconhecer se alguma vez devo ter nobreza
e por causa disso dever ter o nome de cavaleiro”. “Se o quiserdes”,
falou Galaat, “assim deixai-me ir lá dentro, quando bem penso que eu mais
facilmente saio de lá que vós”. E ele falou que ninguém além dele deveria
entrar, e com isto um se separou do outro, e cada um cavalgou seu caminho e sua
via e procurou aventura o melhor que pôde. Com isto calam-se as notícias sobre
Galaat e se diz de Meliant como lhe ocorreu.

Então dizem as notícias que, como Meliant estava separado de Galaat, que estava
a dois dias de distância, e cavalgou tanto até vir, às primas horas da manhã, a
um gramado e viu no meio da grama uma bela poltrona, e rica, e à frente da
poltrona estavam távolas cobertas e estavam preenchidas com as melhores
iguarias que se possam pensar, e estava sobre a poltrona uma coroa de todo
rica. E ele a contemplou e não desejou nenhuma outra coisa que lá estava, senão
a coroa, que era tão bela, que em boa hora tinha nascido aquele que a deveria
portar. Então a tomou e falou que deveria conduzi-la consigo e a fez em seu
braço direito e retornou à floresta. E não tinha cavalgado longe quando viu vir
em sua direção um cavaleiro sobre um grande corcel negro, e o chamou e falou:
“Senhor cavaleiro, deixai estar a coroa, pois que não vos pertence, e saibais
que em muito má hora a tomais!” E quando o viu, virou por esta causa, quando
bem sabia que com ele precisaria ferir-se e falou: “Amado Senhor Deus,
ajuda teu cavaleiro!” E aquele lhe veio e o encontrou tão duro pelo escudo e
pela proteção do pescoço, e fincou-lhe a lança através do lado e o feriu de
novo por terra e assim o tinha pronto que ferro ficasse em seu corpo e um
pedaço da madeira. E desmontou sobre ele e lhe tomou a coroa, quando nenhum
direito tinha a ela, e retornou pelo mesmo caminho por onde tinha vindo. 
E Meliant permaneceu jazendo como não podia levantar-se e achou que estava
ferido de morte e muito se admoestou que não tivesse seguido Galaat, porque
disso lhe veio mal. E nisto que estava em tal lamento, ocorreu que o caminho de
Galaat de tal maneira o portou que veio até ele. E pois que viu que Meliant
jazia sobre a terra, assim ferido, entristeceu-se bastante, quando consigo
pensou que ele estivesse ferido de morte. Então desmontou e foi até ele e
falou: “Hei, Meliant, quem lhe fez isso, que Deus vos regenere!” E com isto
Meliant o ouviu, então o reconheceu ao falar, e falou: “Hei, Senhor, por Deus,
não me deixai morrer aqui nesta floresta, mas conduzi-me para qualquer
convento, onde possa me acontecer meu direito e que lá mesmo eu morra como um
bom cristão”. “Como”, falou Galaat, “como? Estais então tão ferido, que julgais
morrer?” “Senhor,” falou ele, “eu sim!” Então estava Galaat à vez triste e
perguntou onde estariam os que lhe tinham feito aquilo. E com isto assim veio o
cavaleiro para fora da floresta, que havia ferido Melians,\footnote{ Também aqui
o texto original alterna as formas \textit{Meliant} e \textit{Melians}.
 } e falou para Galaat: “Senhor cavaleiro, precavei-vos de mim, pois quero
lhe fazer o maior mal que puder”. “Hei, Senhor,” falou Melians, “é ele aquele
que me matou, por Deus, precavei-vos dele!” E Galaat não respondeu nenhuma
palavra e dirigiu-se para frente do cavaleiro, e ele veio tão ligeiro quanto
pôde. E por ter vindo muito impetuosamente, perdeu o seu [adversário]. Mas
Galaat o encontrou tão duro que lhe enfiou a lança pelos ombros e o trouxe, a
ele e ao corcel juntos, por terra, e a lança quebrou-se. E Galaat saltou por
cima dele. E com isto, por cuja causa ele deveria volver, então viu de onde
vinha um outro cavaleiro bem armado, que clamou: “Senhor cavaleiro, deveis
deixar-me aqui o cavalo”. E Galaat foi a seu encontro e colou a lança sobre seu
escudo e a quebrou, pois que antes estava partida, e não conseguiu derrubá-lo
da sela; e Galaat lhe cortou a mão direita com a espada. E porque ele sentiu
que estava ferido, então volveu para a fuga, quando tinha grande medo de
morrer. E Galaat não o caçou mais que o necessário para que não tivesse como
fazer mais mal do que já tinha feito. E volveu para Meliant e não contemplou
mais o cavaleiro que tinha derrubado o que mais queria fazer, quando de bom
grado, por sua vontade, faria o que pudesse fazer. “Senhor,” falou ele, “se eu
pudesse suportar cavalgar, bem queria que perante vós me erguêsseis e me
dirigísseis a um convento não longe daqui, que eu bem sei, se estivesse eu lá,
far-se-ia tudo que se pudesse para me salvar”. Galaat falou que de bom grado
queria fazê-lo, “quando seria bom que se vos retirasse o ferro”. “Hei, Senhor,”
falou Meliant, “não me faço bem em tal temor, eu teria então me confessado,
pois que creio que eu morra de pronto. Por isto conduzi-me daqui de imediato”.
E então Galaat o tomou, o mais suavemente que pôde, e o assentou diante de si e
o tomou pelo braço, de sorte que não caísse, quando estava muito doente; e
alçou seu caminho e cavalgou por tanto tempo até que viesse à porta, e chamou
os irmãos que lá estavam, que eram pessoas muito santas. E eles abriram a porta
o receberam muito boamente e o portaram a uma câmara. E pois que tinha tirado
seu elmo, então desejou o corpo de Nosso Senhor. E pois que tinha confessado e
rezado por graça como um bom cristão, recebeu ele Nosso Senhor. E pois que O
tinha recebido, falou para Galaat: ``Que venha a morte quando ela quiser, quando
estou bem preparado para ela. Então podeis tentar o que quiserem para retirar o
ferro”. Galaat retirou-lhe o ferro e ele desmaiou pela dor que padecia. E
Galaat perguntou se ninguém ali que pudesse com feridas. “Sim,” falaram eles,
“Senhor”. E chamaram a vir um velho monge, que tinha sido cavaleiro, e lhe
mostraram as feridas. E ele as contemplou e falou que queria deixá-lo saudável
em um mês, se Deus quisesse. Por essas notícias estava Galaat feliz e fez-se
desarmar e ele falou que queria permanecer no dia e mais um dia para contemplar
se Meliant poderia convalescer. E assim permaneceu Galaat lá três dias, e então
perguntou a Meliant como estava. E ele disse que se volvia para convalescer.
“Assim posso daqui me separar amanhã”, falou Galaat. E ele lhe respondeu bem
triste: “Hei, Senhor, quereis deixar-me aqui? Ainda sou aquele que deseja tanto
a vossa companhia, mais que qualquer homem no mundo, se pudesse tê-la”.
“Senhor,” falou Galaat, “não mais vos sou de valia e tinha muito com o que me
ocupar além de descansar, sobretudo procurar o Santo Graal, quando a demanda
por minha causa começou”. “Como”, falou um dos irmãos, “ela já começou?”
“Sim”, falou Galaat, “somos ambos companheiros nela”. “Por minha verdade”,
falou o irmão, “assim bem vos digo, cavaleiro seguro, que esta má sorte vos
aconteceu por causa de vossos pecados. Se me mostrardes vosso ser desde que a
demanda foi iniciada, eu queria bem vos dizer através de qual pecado isso vos
teria acontecido”. “Em verdade”, falou Meliant, “eu quero de bom grado
dizer-vos”. 

Então lhe contou Meliant como Galaat o tinha feito cavaleiro e das letras que
eles encontraram na cruz, que proibiam o caminho para a mão esquerda, e como
ele cavalgou pelo caminho e o que lhe aconteceu. E o bom homem era de visa
santa e bem instruído e falou: “Senhor cavaleiro, seguramente é um sinal do
Santo Graal, quando nada me dissestes que não tenha grande significado e isso
também deve significar! O que deveríeis, para tornar-vos cavaleiro, antes de
ter feito sua confissão, era vir para a ordem da cavalaria puro e claro de toda
a impureza e de todo o pecado que vós sabidamente mancháveis. E assim viríeis
para a demanda do Santo Graal não como estaríeis obrigado a estar. E pois que o
Diabo isso viu, foi-lhe muito tormento e pensou que deveria atacar-vos tão logo
vistes seu sítio, e vos fez como eu devo dizer-vos quando foi''.

Pois que saístes do convento, e que vos havíeis tornado cavaleiro, o primeiro
encontrar que vos encontrou foi o sinal da Santa Cruz. É um sinal de que todo
cavaleiro está obrigado a deixar-se sobre o fato, e mais foram as letras que
vos indicaram dois caminhos: um para a mão direita, o outro para a mão esquerda.
Pelo da direita deveríeis entender o caminho para Nosso Senhor Jesus Cristo,
que os cavaleiros de Deus são obrigados a seguir noite e dia, o dia por causa
da alma, a noite por causa do corpo. O da mão esquerda devíeis entender o
caminho do pecador, e lá se ressente um grande temor para aquele se alça para
lá dentro. E porque não era assim seguro como o outro, por isso proibiram as
letras que ninguém para lá se alçasse, que não fosse mais nobre que o outro. E
isso é tanto quanto dizer como que ele seria tão repleto do amor de Deus que
ele por nenhuma aventura poderia cair em medo. E pois que viste\footnote{ Neste
período do parágrafo em tela, o religioso altera o pronome de tratamento de
\textit{ir}  para \textit{du}.}  as letras, foi-te
grande maravilha o que podia ser, e de pronto o Inimigo te enganou e bem sabia
de tua cortesia,\footnote{ O religioso Hans-Hugo Steinhoff traduz, de modo muito
indiciário, o étimo medieval \textit{hoffart,} que assimilamos a “cortesia”, ou
 “modo cortês”, por alta coragem ou bravura.}  quando
pensaste bem atravessar por tua valentia. E também te enganou a tua presunção,
quando a inscrição significava a cavalaria espiritual, assim pensaste e
entendeste como cavalaria mundana. E por isso vieste em cortesia e com isso
caíste em pecado mortal. Quando te separaste de Galaat, o Inimigo, que te achou
fraco, veio a ti e pensou que tinha feito pouco, ele te fez ainda incorrer em
pecado mortal por outra vez, assim que ele te fez cair de pecado em pecado. E
pois ele preparou à tua frente uma coroa de ouro e te fez cair em cobiça
injusta. Tão logo quanto o viste, na hora caíste em pecado mortal. E porque ele
viu que levavas embora a coroa, ele veio a ti na forma de um cavaleiro
pecaminoso e pensava te derrotar como alguém que te tivesse matado, quando a cruz
que fizeste à tua frente e pensou corretamente que não estavas a Seu serviço.
Assim Ele te trouxe em medo de morte, para que na outra vez te abandonasses ao
auxílio de Deus, mais que à tua força. E para que tivesses socorro de pronto,
ele te enviou este santo cavaleiro Galaat, que te vingou em dois cavaleiros,
que lá significavam os dois pecados, que em ti estavam abrigados, e não teriam
conseguido ser contra ele, quando ele está sem pecado mortal. Então vos contei
por qual sentido essa aventura aconteceu”. E falaram que esse
significado seria belo e maravilhoso.

“Pois nos ajude Deus, bom homem”, falou Galaat, “quando preciso daqui me
separar”. E lá contemplou suas armas, se lhe faltava alguma coisa. E porque viu
que nada lhe faltava, então se armou, pediu licença a Meliant e separou-se do
convento. E por tanto tempo quanto a aventura o conduziu, ele veio ao caminho
que ia para o castelo das moças,\footnote{ O nome medieval para o castelo das
moças é \textit{megdeburg}, que pode estar na origem do nome da cidade de
Magdeburg, já que \textit{Magd} é o termo alemão contemporâneo
correspondente a \textit{megde}. } e não tinha cavalgado mais e lhe vieram
ao encontro sete moças que ricamente tinham cavalgado e falaram: “Senhor
cavaleiro, ultrapassastes o vau\footnote{ Trecho de um rio pelo qual se pode
atravessar o mesmo a pé ou montado.}  e deveis vir ao castelo''. E
ele cavalgou todo para ele, até que um servo o encontrou e lhe falou: “Senhor,
os do castelo vos proíbem que adiante cavalgueis, e vós dizeis pois o que
quereis”. “Eu não quero outra coisa,'' falou ele, “que o costume deste
castelo”. “Seguramente”, falou ele, “é alguma coisa vos desejastes para mal, e
deveis tê-la de modo que nenhum cavaleiro possa atacar. Esperai-me pois aqui,
deveis ter o que buscais”. “Pois cavalgai logo”, falou Galaat, “por causa de
minha ocupação”.

Então se separou o servo de Galaat e cavalgou para o castelo. E não demorou
muito Galaat viu de lá saírem sete cavaleiros, que eram irmãos e chamaram por
Galaat: “Senhor cavaleiro, precavei-vos de nós, pois que não vos asseguramos
senão a morte!” “Como”, falou Galaat, “quereis todos lutar comigo juntos?”
“Sim”, eles falaram, “quando assim é o costume do castelo”. E pois que ele o
ouviu, sacou sua lança e deixou-se correr sobre eles e encontrou o primeiro,
que o conduziu por terra e tinha quase quebrado o seu pescoço. E todos eles
juntos vieram de encontro ao seu escudo, pois que não o queriam derrubar da
sela. E da força de sua lança, quase o tinham levado por terra e quebrado seu
pescoço. No começo quebraram todos suas lanças. E Galaat tinha derrubado três
com sua lança, e sacou a espada e deixou-se correr sobre eles, que viu
conservarem-se diante de si, e foi sobre eles. E então se alçou entre eles uma
grande e temerosa briga e durou por tanto tempo até que estivessem sentados
aqueles que tinha derrubado, e então a confusão alçou-se ainda maior que antes.


Então aquele que era o melhor de todos os cavaleiros fortaleceu-se assim tanto
que os impeliu para trás de si, e os preparou com espada cortante que nenhuma
arma os pode ajudar, e fez o sangue sair-lhes do corpo. E o acharam de tal
poder que não pensaram que fosse qualquer homem terrestre, porque nenhum homem no
mundo seria aquele que poderia ter sofrido metade do que ele sofreu. E isso
muito os assustou, que não quiseram movê-lo de seu sítio, e o acharam o dia
todo em tal força como a que de primeiro se alçou. Se a verdade dele era
aquela, como a história do Santo Graal demonstra, por nenhum trabalho nem
por nenhuma cavalaria nenhum homem o veria fatigado. Em tal medida durou a
contenda até o meio-dia, quando os sete irmãos eram de grande valentia. Então,
porque veio o tempo, acharam-se tão cansados e prontos de mal, que não tinham
poder para proteger seus corpos. E aquele que nunca ficava cansado, foi-lhes ao
encontro, que caíram de seus cavalos. E porque viram que não podiam ajudar,
então se viraram para a fuga. E porque ele viu isso, não os caçou e cavalgou
para a ponte que ia para a frente do castelo. E lá encontrou ele um homem
grisalho, vestido com roupas espirituais, e trazia a chave do castelo e falou:
“Senhor, tomai a chave! Então podeis fazer a todos que estão lá dentro conforme
a vossa vontade, quando tanto fizestes que o castelo é vosso”. E ele tomou a
chave e cavalgou ao castelo. E tão logo ele veio, então viu através das
alamedas tantas moças, que não pôde contar, que todas falaram: “Senhor, sede
por Deus bem-vindo, muito aspiramos pela vossa salvação, louvado seja Deus, que
vos enviou, Senhor, quando de outra forma não seríamos salvas desta lamúria do
castelo”. E ele respondeu que Deus lhes pagasse. E então o tomaram pela rédea e
ofereceram-lhe que desmontasse. E ele lhes respondeu, que ainda não seria tempo
de tomar albergue. E uma donzela lhe falou: “Hei, Senhor, por que o dizeis?
Seguramente, se daqui vos separardes, aqueles que por vossa valentia foram
golpeados, viriam à noite de novo para cá e de novo alçariam o costume
lamurioso que por tento tempo acalentaram neste castelo e teríeis trabalhado à
toa.\footnote{ A expressão alemã medieval empregada pelo cronista é \textit{umb
sůß}, que no alemão contemporâneo redundaria em \textit{wegen der Süßigkeit},
ou “por causa da doçura”, ou “por doçura”.} “O que quereis que eu
faça”, falou Galaat, “quando estou pronto para fazer todas as vossas vontades
tanto quanto eu veja que está bem feito?” “Nós queremos”, falou a donzela,
“todos os cavaleiros que aqui dentro estão e todos os que bem contivestes do
castelo, que os façais jurar e também todos os que estão aqui, que nunca mais
manterão esse costume”. E ele falou que de bom grande o queria fazer. 

E porque o conduziram até a casa principal, lá desmontou, retirou seu elmo e
subiu ao palácio. E de pronto saiu da câmara uma donzela e trouxe um chifre de
marfim, muito ricamente contido, e deu a Galaat e falou: “Senhor, se quereis
que venham todos aqueles que de vós deverão adiante ter o castelo, então soprai
o chifre, quando se pode bem ouvi-lo a dez milhas todo ao redor e ao redor”. E
ele respondeu que seria bom fazê-lo, e o deu a um cavaleiro que viu estar à sua
frente. E ele tomou o chifre e o soprou tão alto que se podia ouvir a dez
milhas de distância sobre toda a terra. Pois que o tinha feito, sentaram-se ao
redor de Galaat, e ele pergunta ao que lhe tinha trazido a chave, se era padre.
E ele respondeu “sim”. “Então me dizei”, falou ele, “o costume daqui e onde
todas estas donzelas foram tomadas”. “De muito grado”, falou o bom homem. “É
verdadeiro que faz sete anos, vieram os mesmos cavaleiros que derrotastes, a
este castelo por aventura e tomaram albergue do duque Lyvor, que era senhor em
toda esta terra, e era o mais santo que se soubesse. E à noite, quando se tinha
ceado, alçou-se uma briga entre os sete irmãos e o duque por uma sua filha, que
queriam ter com violência. E sucedeu que o duque foi morto e um seu filho. E
então tomaram aqueles por cuja causa a briga tinha começado, e os irmãos que o
fizeram, então tomaram todos o tesouro que aqui estava e enviaram a cavaleiros
e a servos e alçaram novamente a guerra com os desta terra e tanto fizeram que
os venceram e precisaram reter todo o seu feudo\footnote{ O termo encontrado no
documento original é\textit{ lehen}, que permanece no alemão contemporâneo
, ao lado de \textit{das Feudum}.} deles. E porque
a filha do duque o viu, ficou muito irada e assim falou de traição:
‘Seguramente, vós Senhores, que tendes o domínio deste castelo, não vos temo!
Pois, tal como o possuís por causa de uma mulher, assim deveis deixá-lo por
causa de uma donzela, e deverá derrotá-los, aos sete, o lábio de um só
cavaleiro’. E eles responderam, por causa daquilo que ela tinha dito, assim
nunca uma donzela do castelo deveria passar, que deveria ser capturada até que
viesse o cavaleiro que os vencesse; e até então o fizeram e desde então o
castelo se chama Castelo das Moças.” “E a donzela, por cuja causa se fez a
briga, ela ainda está aqui?”, falou Galaat. “Senhor,” falou o bom homem, “não,
ela está morta, e uma sua irmã, que é mais jovem que ela, essa ainda está
aqui”. 

“Como estavam as donzelas aqui?”, falou Galaat. “Senhor, elas padeceram grande
desconforto”. “Então estão para fora, por isso seja Deus louvado”, falou Galaat. 

Perto das nonas horas encheu-se o castelo, quando cada um que soube da notícia
de que o castelo foi ganho, veio e fez grande honra a Galaat, a quem eles
consideravam seu senhor. E ele deu de pronto à filha do duque o castelo e tudo
o que lá pertencia e fez tanto que todos os homens da terra receberam seu feudo
da donzela. E fez todos jurarem que nunca deveriam deixar o costume como ele
lhes mandava e que deixassem partir as donzelas para suas terras.

O dia todo permaneceu Galaat lá, e lhe fizeram grande honra. E pela manhã vieram
as notícias de que os sete irmãos teriam sido mortos. “Quem os matou?”, falou
Galaat. “Senhor,” falou um, “ontem pois que eles se separaram de nós,
encontrou-os meu senhor Gawin e Gaharies, seu irmão, e meu senhor Ywan, e
deixaram correr um sobre o outro e o acidente caiu sobre os sete irmãos''. E lhe
causou grande maravilha a aventura, e desejou suas armas. E se lhe entregaram,
e ele se armou, e separou-se do castelo. E os que lá estavam o escoltaram por
um bom momento, até que se chamaram a voltar. E alçou-se por seu caminho e
cavalgou sozinho. Então se cala a história sobre ele e diz do senhor Gawin. 

Aqui diz a história que, porque meu senhor Gawin  estava separado dos
seus companheiros, que ele cavalgou por alguns dias, que ele não achou nenhuma
aventura que fosse para contar, até que ele veio ao convento onde Galaat tinha
tomado o escudo branco com a cruz vermelha. E pois que o ouviu, então pergunta
por onde ele tinha retornado. E eles lhe informaram. E ele se alçou ao caminho
para ele e cavalgou por tanto tempo até que a aventura o trouxesse para onde
Meliant jazia e ele conheceu o senhor Gawin. Então lhe diz notícias de Galaat,
que ele, pela manhã, se tinha dele separado. “Ah Deus,” falou meu senhor Gawin,
``como sou tão desafortunado, sou bem o mais azarado dos cavaleiros do
mundo, que não posso cavalgar para perto de Galaat e o seguir! Seguramente, dê
Deus que eu o ache, nunca quero me separar de sua companhia, se ele tinha minha
companhia em boa conta, como tenho a dele”. Esta fala ouviu um dos irmãos do
convento e falou ao senhor Gawin: “Seguramente a companhia de vós dois não
seria equânime, quando sois um mau e infiel servidor,  e ele é um cavaleiro
como deve ser”.  “Senhor,” falou meu senhor Gawin,  “do que me dizeis
penso que bem me conheceis”. “E vos conheço melhor”, disse o bom homem, “do que
considerais”. “Prezado Senhor,” falou meu senhor Gawin,  “pois queirais
dizer-me bem, se vos agradar, quanto sou assim como me reprovais”. “Eu não vos
devo dizer”, falou ele, “quando bem deveis descobri-lo, quando for tempo, de se
dever dizer-vos”. 

Enquanto falavam, veio dentro um cavaleiro, armado de toda arma. E os irmãos lhe
mostraram a corte e o receberam e o desarmaram. E pois que estava desarmado,
foi de pronto à câmara onde meu senhor Gawin  dentro estava. E tão logo
Gawin o viu, então viu que era Gaharies, seu irmão. E correu de braços abertos
para ele e lhe fez grande alegria, e ele lhe pergunta se esteve saudável. E ele
respondeu: “Sim, pela graça de Deus!” 

À noite foi ele bem servido pelos irmãos do convento. E pela manhã, pois que
amanhecia, então ouviram missa de todo armados, sem o elmo. E pois que bem
estavam preparados e montados, separaram-se deles e cavalgaram até as primas
horas, então viram o senhor Ywan cavalgar só, e bem o reconheceram por suas
armas e o chamaram, no que ele parou quieto. E porque ouviu chamar-se, parou
quieto e bem os reconheceu em seu falar, quando o saudaram e ficaram muito
contentes e lhe perguntaram o que tinha feito desde então. E ele respondeu que
não tinha achado nenhuma aventura que o encontrasse. “Pois cavalguemos em
diante”, falou Gaharies, “juntos até que Nosso Senhor Deus nos mande aventura”.
E o seguiram e cavalgaram juntos suas vias como por tanto tempo até que vieram
próximos ao Castelo das Moças. Isso foi no mesmo dia em que o mesmo castelo foi
ganho, e os sete irmãos tinham fugido. 

Pois que os sete irmãos viram os três cavaleiros, então falou um para o outro
que deveriam pagar a eles pelo que lhes tinha acontecido. Quando bem
reconheceram que eram os companheiros da aventura, por cuja vontade
apodreceriam, foram atacar por cima dos três companheiros, que se protegeram,
quando teriam vindo à morte. E pois que ouviram esse discurso, dirigiram seu
corcel contra eles, e aconteceu que, na primeira justa, morreram três dos sete
irmãos. Quando o senhor Gawin viu um e Gaharies o outro, e Ywan o terceiro.
Então sacaram a espada e correram por sobre os outros; eles se defenderam tanto
quanto puderam, quando estavam bem cansados e se tinham trabalhado da grande
luta contra Galaat no mesmo dia. E aqueles que lá estavam eram bons cavaleiros
e valentes, e encontraram-nos tanto que em pouco tempo os mataram e os deixaram
jazer na floresta. E cavalgaram para onde a sorte os levasse, e não volveram
para o Castelo das Moças, eles volveram para o caminho da mão direita. E por
isto perderam Galaat, e à hora das vésperas, separaram-se uns dos outros e cada
homem por seu caminho. E meu senhor Gawin  cavalgou por tanto tempo até
que viesse à casa de um eremita e achou o eremita em sua capela e cantou
vésperas de Nossa Senhora. E desmontou de seu corcel e as ouviu e desejou
albergue pela vontade de Deus, e ele lhe deu de boa vontade.

À noite perguntou o bom homem ao senhor Gawin quem ele era e ele lhe diz a
verdade e qual demanda ele tinha assumido. E pois que o bom homem entendeu meu
senhor Gawin,  então bem o conheceu e falou: “Senhor, se vos agrada,
gostaria à vez bem de saber de vosso ser”. Então começou a dizer da confissão e
lhe contou alguns belos exemplos do Evangelho para ensinar que tinha feito sua
confissão; queria consolá-lo de tudo quanto pudesse. “Senhor,” falou o senhor
Gawin, ``se me quereis ensinar uma fala que ontem ao dia foi dita, quero que
digais todo meu ser, se penseis que sou um nobre, quando bem sei que sois um
padre”.  E o bom homem o louvou que bem o queria consolar de tudo quanto
pudesse. E meu senhor Gawin  contemplou o bom homem e o viu grisalho e
velho, e bem o pensou ser um homem santo, e o agradou muito voltar a fazer-lhe
sua confissão. E então começou a lhe dizer que se sabia, mais que todos, culpado
perante Nosso Senhor e nada esqueceu de lhe dizer da fala que ele tinha ouvido
no convento onde Meliant jazia. Então o bom homem achou que fazia mais de
quatro anos que ele lhe tinha feito sua confissão. E então falou: senhor
cavaleiro, pois fostes mau servidor e infiel! E quando viestes para a ordem da
cavalaria, era para que servísseis o maior, e que deveríeis amparar e alertar a
Santa Igreja e que devolvêsseis a Nosso Senhor o tesouro que ele vos deu a
conservar, que era vossa alma. E por essa vontade se vos faz cavaleiro, e
mantivestes mal a vossa cavalaria. Quando muito servistes ao Inimigo, e
deixastes vosso Salvador e levastes a vida mais impura que qualquer cavaleiro
levou. E por isto podeis bem perceber que alguém bem vos conhece, que o chamou
mau cavaleiro e infiel. Seguramente, se não fôsseis tão pecador quanto sois, os
sete irmãos nunca seriam mortos por vós e vossa ajuda, e teriam recebido
penitência do costume impuro que tinham sobre o Castelo das Moças, e se teriam
reconciliado com Deus. E assim não fez Galaat, o bom cavaleiro, que cavalgais
procurando, quando os venceu sem matar. E não foi sem grande significado,
quando os sete irmãos tomaram o costume sobre o castelo, que mantinham todas as
moças que vinham lá para a terra, fosse justo ou injusto”.  

“Como Senhor?” Disse cá Gawin. ``Dizei-me o significado, para que eu possa contar,
quando eu retornar à corte.'' “De bom grado”, falou o bom homem, “quando pelo ao
Castelo das Moças deve-se entender o inferno e pelas moças as boas almas, que
injustamente estavam trancafiadas antes do martírio de Nosso Senhor. E pelos
sete irmãos devem-se entender os sete pecados capitais, que então reinavam no
mundo. Quando tão logo a alma do corpo se separa, fossem nobres ou más, seguiam
na hora para o inferno e lá eram trancafiadas como as moças. Pois que o Pai do
Céu viu que tinha criado mal, enviou Seu Filho aqui para baixo ao mundo, para
desatar as boas moças, que eram as boas almas. E como ele enviou Seu Filho aqui
para baixo ao mundo, assim enviou Seu servo eleito, para que ganhasse o castelo
e resgatar as boas moças, que lá são tão castas e puras como a flor-de-lis, que
o sol nunca enruga”. 

Pois que o senhor Gawin ouviu essa fala, ele não soube responder. O bom homem
falou: ``Gawin, Gawin, se tu queres deixar tua má vida, que longamente levaste,
ainda queres bem te dirigires com Nosso Senhor, quando a Escritura diz que
ninguém é tão pecador, que deseje com coração bom e puro a misericórdia de
Nosso Senhor e não ache Graça. E por isso bem te quero aconselhar em segredo
que tomeis penitência por tudo que fizestes de errado”. E ele falou que não
poderia padecer penitência. Pois viu o bom homem que seu trabalho estava
perdido no ensinamento, então se calou e não lhe falou mais nada daquilo. Pela
manhã, pois que tinha raiado o dia, o senhor Gawin despediu-se dele. E ele
cavalgou por tanto tempo por aventura que Agravant e Gifflet Lefeld o
encontraram e cavalgaram bem quatro dias juntos sem achar aventura, que fosse
para contar, e no quinto dia separaram-se e cavalgou cada um seu caminho. Então
aqui a história se cala sobre ele e diz de Galaat.

Doravante se segue, pois que Galaat estava separado do Castelo das Moças, que
cavalgou por alguns dias, até que veio a um vau deserto. Um dia ocorreu que lhe
vieram Lancelot e Parsifal\footnote{ A grafia encontra-se, nesta passagem,
alterada de \textit{Parsival} para \textit{Parsifal }.},  que
cavalgavam juntos e não o conheceram como não estavam acostumados a ver aquelas
armas. E Lancelot veio-lhe com o primeiro e quebrou sua lança em seu escudo. E
Galaat o encontrou assim que ele o derrubou e ao corcel um por cima do outro.
De outra forma não lhe causou de modo algum dor. E pois que tinha quebrado sua
lança, sacou então sua espada e golpeou Parsifal tanto que lhe cortou o elmo e
a viseira. E não lhe tivesse a espada amolecido na mão, ele o teria matado sem
dúvida, e não teve tanto poder que pudesse permanecer na sela, ele caiu por
terra desmaiado, e como doente ficou da luta, que tinha recebido, que não sabia
se era dia ou noite. E esta justa foi feita à frente de uma clausura, onde
estava dentro um enclausurado. E pois que viu Galaat, ele falou: “Então parti
com Deus que vos escolte! Seguramente se eles vos reconhecessem como eu vos
conheço, nunca teriam a ousadia, que se lançaram contra vós”. Pois que Galaat
ouviu essa fala, então teria grande medo de que se o reconhecesse, e golpeou o
cavalo com a espora e o apressou prontamente ao caminho como o cavalo pudesse
correr. E porque viram que não podiam alcançá-lo a cavalo, volveram de novo,
entristecidos e também tristes que teriam de pronto morrido, de bom grado.
Quando odiavam assim tanto sua vida que se alçaram para a floresta.

Assim permaneceu Lancelot na floresta, entristecido e irado por ter perdido o
cavaleiro, e ele falou a Parsifal: “O que podemos?” E ele respondeu que não
sabia nenhum conselho para estas coisas, quando o cavaleiro apressou-se assim
tão logo que eles não conseguiram alcançá-lo a cavalo; “então bem vede que a
noite nos intimidou em tal sítio, que nunca podemos sair, saímos então com
aventura. E por isto me parece que seria melhor volver para a via correta,
quando se nós aqui começarmos a nos perder, não creio que consigamos vir ao
caminho correto em um bom momento. Então podemos fazer o que nos agrada, quando
nosso melhor é mais retornar que continuar cavalgando”. Lancelot responde que
não seguiria de bom grado o retorno e queria cavalgar perto daquele que
conduzia o escudo branco, quando ele jamais quisesse ter conforto até que
soubesse quem ele era. “Mas por tanto tempo podeis bem esperar”, falou
Parsifal, “até que amanhã venha o dia, então quereremos cavalgar atrás do
cavaleiro”. E Lancelot responde-lhe que não o queria fazer. “Então que Deus
vos escolte hoje”, falou Parsifal, “quando vinde tão longe e quereis voltar
para a clausura”, quando ele falou, deveria reconhecê-lo bem. 

Assim se separaram os companheiros, Parsifal para dentro da clausura, e
Lancelot cavalgou cruzando a floresta atrás do cavaleiro, de tal maneira que
não se deteve em nenhum caminho ou atalho, e cavalgou assim como a aventura o
conduzia, e doeu-lhe duramente que não soubesse, longe ou perto, onde poderia
tomar seu caminho, quando a noite estava, à vez, sombria. E ainda assim tinha
cavalgado longe, tanto que tinha vindo a uma cruz de pedra, que estava em uma
encruzilhada em um campo deserto. E ele contemplou a cruz, pois se aproximou, e
viu que ao lado estava uma coluna de mármore, lá estava uma carta selada. E a
noite estava tão sombria que ele não conseguiu reconhecer o que a carta falava.
E viu que perto da cruz estava uma velha capela, e por isso se dirigiu para lá,
pois considerava encontrar gente lá dentro. E pois que se aproximou, atou seu
cavalo a um carvalho e tirou seu escudo do pescoço e o pendurou em uma árvore e
foi para a capela e a achou deserta. E entrou e viu uma grade de ferro que lá
estava fechada e feita tão apertada que não se conseguia facilmente entrar. E
viu através da grade e viu que lá dentro estava um altar, muito ricamente
vestido com panos de seda e outros panos e outras coisas belas. E lá à frente
estava um candelabro de prata, que continha seis velas incandescentes, e davam
muito grande luz. E pois que isso viu, desejou muito entrar e saber quem lá
estava. Quando não considerava ver coisas assim belas, em tal sítio estranho,
como lá estavam. E deu a volta e contemplou a grade. E pois que viu que não
podia entrar, então se apartou da capela muito entristecido e veio a seu cavalo
e o conduziu com seu arreio à cruz e lhe retirou a sela e o arreio e o deixou
pastar. E desatou seu elmo e o pôs para si e retirou sua espada e se dirigiu
sobre seu escudo à frente da cruz e ficou facilmente adormecido, após o que ele
estava muito cansado, quando não conseguia esquecer o bom cavaleiro com o
escudo branco. E pois que tinha jazido um bom bocado, entorpecido como estava,
então viu que dois cavaleiros traziam uma padiola de cavalo, e lá jazia um
cavaleiro ferido, que se lamentava muito. E quando veio próximo a Lancelot,
então se conteve e não falou uma palavra, quando considerava que ele dormia. E
Lancelot não falou, como não estava nem dormindo nem desperto e assim jazia
entorpecido. E o cavaleiro da padiola de cavalo, que lá permanecia junto à
cruz, começou a lamentar-se quase em lamúria e falou: “Hei, Senhor Deus, este
lamento deve durar pelo caminho inteiro? Hei Deus, quando deve vir o Vaso
Santo, por meio do qual o forte lamento deve acabar? Hei Deus, alguém padece de
tantas dores como eu padeço por causa de um pequeno erro?”

Um bom bocado queixou-se o cavaleiro e voltou-se para Deus de suas dores e seu
lamento. E Lancelot nunca se moveu e não falou uma palavra, assim como jazia
entorpecido, e ainda assim ouviu e bem entendeu a fala. Porque o cavaleiro
tinha rezado em tal medida, Lancelot viu ao redor de si e viu sair da capela o
candelabro com as velas, que ele tinha visto na capela. E ele contemplou o
candelabro, que lá vinha à frente da cruz, quando não conseguiu ver quem o
portava. Isso lhe tinha grande maravilha. E viu depois vir o Santo Vaso, que
ele tinha mais visto junto ao Rei Pescador, o mesmo a que se chama o Santo
Graal.\footnote{ Apesar de termos empregado iniciais maiúsculas ao longo de toda
a tradução, esta é a primeira ocorrência de maiúsculas do texto original, para
referir-se ao Santo Graal, à página 116.} E assim quando o
cavaleiro ferido o viu vir, então se deixou cair tão alto quanto estava para a
terra e dirigiu as mãos juntas para lá e falou: “Caro Senhor Deus, O deste
Santo Vaso, que aqui vejo vir, deu sinal grande nesta terra e em outras,
querido Pai, vede na minha direção por meio do Teu compadecer em tal medida que
este grande tormento com que me retorço, seja-me suavizado em curto tempo, que
eu possa vir à demanda em que vieram os outros valentes!” E arrastou-se sobre
as mãos e sobre os pés por tanto tempo até que veio à coluna em cuja frente
estava a távola e, em cima, o Santo Vaso. E ele se tomou por suas ambas mãos e
se moveu para cima e fez tanto que beijou a távola de prata e fixou seus olhos.
Pois que o tinha feito, sentiu-se aliviado de suas pernas e soltou um grande
grito: “Hei, Senhor Deus, estou convalescido!” E não durou muito até que
dormisse. E pois que o Vaso lá permaneceu só um tempinho, então foi o
candelabro de novo à capela e o Santo Vaso junto, de modo que Lancelot não
soube com quem vem ou com quem se aparta, quem o portou. Por isto lhe aconteceu
por meio disto que estava pesado de pecados, com que estava carregado, que não
se ponderou vir através do Santo Graal, e nunca fez compreensão de que alguma
coisa era por causa daquilo. E por isto achou ele em alguns fins na demanda
muita vergonha, que portanto se lhe mostrou, e foi-lhe mal em alguns fins. 

Pois que o Santo Graal se tinha apartado da cruz, dirigiu-se o cavaleiro para a
padiola são e forte e beijou a cruz e de pronto veio um escudeiro e lhe trouxe
arma assim bela e assim rica. E quando vislumbrou o cavaleiro, então lhe
perguntou como lhe tinha acontecido. “Por minha verdade,” falou ele, “bem, por
graça de Deus, quando estava bem de pronto convalescido, eis que o Santo Graal
veio e me desatou de todos os meus vícios. E me fez maravilha por causa deste
cavaleiro, que aqui dorme, que não despertou com sua vinda”. “Por minha
verdade,” falou o escudeiro, “ele está carregado de diversos pecados, pois que
não fez sua penitência, e facilmente se culpou contra Nosso Senhor Deus, que
não quis que ele visse a bela aventura”. “Seguramente,” falou o cavaleiro,
“seja ele quem quiser, ele é infeliz e, creio eu, um companheiro da Távola
Redondo, que se atribuiu buscar o Santo Graal”.

“Senhor,” falou o escudeiro, “eu também trouxe vossas armas, que podeis tomar
quando vos aprouver”. E o cavaleiro respondeu que de outra coisa não precisava
senão daquilo, e tomou sua viseira de ferro, e sua coifa e armou-se. E o
escudeiro tomou o elmo de Lancelot e sua espada e lhe direcionou, então foi ao
corcel de Lancelot e o selou e o arreou. E quando o tinha pronto, então falou
a seu senhor: “Senhor, montai, quando não vos falta um bom corcel! Seguramente,
eu não vos dei coisa alguma, pois estaria melhor sucedido convosco que com o
mau cavaleiro que lá padece”. 

A lua brilhava bela e iluminada, quando era depois da meia-noite, e o cavaleiro
pergunta ao servo\footnote{ O texto altera o termo \textit{edelknecht},
para \textit{knecht},  o que acompanhamos nesta tradução,
assim nos afastando da opção do adaptador Hans-Hugo Steinhoff, que traduz
também \textit{knecht} por “escudeiro”,  à p. 120.}  como ele reconhecia a espada. E ele
respondeu que lhe parecia que a conhecia pela grande beleza que tem, e a
retirou e a contemplou tão bela que teve dela grande desejo. E porque o
cavaleiro estava pronto e montado sobre o corcel de Lancelot, manteve sua mão
esticada na frente da capela e jurou, como Deus e os santos o ajudassem, que
nunca superaria o cavalgar até que soubesse como o Santo Graal aparecia em
alguns fins do reino de Logris e com quem ele foi trazido à Inglaterra e por
meio de qual negócio, ou seja, se alguém estivesse informado da notícia
verdadeira antes dele. “Ajudai-me Deus”, falou o servo, “dissestes o
suficiente, pois deixai Deus decidir desta demanda, e seguramente, sem medo
mortal não podeis passá-la”. “Se nela eu morrer, deve a minha honra ser maior
que a minha vergonha, quando, por essa demanda, nenhum nobre pode negar-se a
morrer ou a convalescer.” E então se apartou, ele e seu escudeiro, e
conduziu com ele a arma de Lancelot e seu corcel como a aventura o conduziu. E
pois que tinham vindo bem uma meia milha pelo caminho ou mais, então aconteceu
que Lancelot endireitou-se e sentou-se como aquele que primeiro estava
acordado, e refletiu consigo se aquilo que ele viu, se era verdadeiro ou se lhe
fora sonhado, quando não sabia se tinha visto o Santo Graal  ou se lhe foi
sonhado. E foi à capela e viu o candelabro à frente do altar, e não viu aquilo
que desejava ver sobre todas as coisas, que era o Santo Graal, quando teria de
bom grado visto se fora verdadeiro, se tivesse podido ser.

Pois que Lancelot tinha visto um bom bocado a grade por causa disso, se podia
ver daquilo que sobre todas as coisas desejava ver, que era o Santo Graal,
então ouviu uma voz que falou: “Muito mais duro que uma pedra e muito mais
amargo que fel e muito mais despido que um álamo, como serias tão atrevido que
em nenhum sítio, onde costume estar o Santo Graal, ousasses vir! Vai-te daqui,
quando este sítio está por demais despurificado de tuas obras!” E
pois que Lancelot ouviu esta fala, então não sabia o que deveria saber, e
apartou-se em frente e suspirou de coração e se lhe lacrimejaram os olhos e
maldisse o tempo em que nasceu. Então bem sabia que ali tinha vindo, que ele
nunca ganharia honra para a vontade que lhe faltava de saber a verdade sobre o
Santo Graal. Quando as três falas que lhe foram faladas, ele não esqueceu e
nunca mais conseguiu esquecer. Por longo tempo fora ele sem saber por que causa
fora assim chamado. E então ele veio para a cruz, e lá não achou nem
seu elmo nem seu corcel; de pronto ele bem percebeu que tinha sido verdade.
Quando lá se alçou em um lamento assim grande e maravilhoso e chamou a si mesmo
um triste descortês e falou: “Então apareceram meu pecado e meu mal à vida,
então bem vejo que minha desventura me amaldiçoou mais que outra coisa. Então
porque deve melhorar, então o Inimigo me obstou, que me tomou a visão, que eu
não consegui ver nenhuma coisa que de Deus viesse. Não é maravilha se eu não
posso ver iluminado. Quando do tempo em que me tornei cavaleiro, então não foi
nunca tempo, em que não estivesse coberto e carregado de pecado mortal, quando
eu por todo o tempo não pratiquei a castidade do mundo mais que um outro”. 

Assim se repreendeu Lancelot e se rebaixou muito. E porque o dia brilhava belo
e claro e os pássaros cantavam na floresta e o sol começou a brilhar sobre as
árvores, e porque ele ouviu o rouxinol cantar, que ele fartamente se tinha
alegrado, e porque ele se viu desaconselhado de todas as coisas e de suas armas
e de seu corcel e bem sabia que Nosso Senhor Deus estava irado com ele, e
pensou achar as coisas e nunca vir ao sítio onde poderia reencontrar sua
alegria. Quando lá considerou achar caminho e toda honra terrestre,
então lhe faltou e se criticou a si mesmo de sua grande descortesia. 
Então se separou da cruz e foi a pé através da floresta, sem elmo, sem
espada e sem escudo, e não retornou à capela, onde ouviu as três maravilhosas
falas, e volveu para aquele atalho e foi por tanto tempo até que, às primas
horas, veio a uma cela. Lá achou dentro um ermitão, que deveria começar missa e
estava armado com as armas da Santa Igreja. E foi à capela calado e pensativo e
também triste como alguém poderia estar. E ajoelhou-se na igreja e golpeou-se à
frente de seu peito e ofertou graças por causa de Nosso Senhor Deus dos maus
feitos que tinha feito no mundo, e espreitou a missa, que o bom homem cantava e
seus alunos. E quando estava cantada e o bom homem se tinha desarmado das armas
de Nosso Senhor Deus, Lancelot o chamou de pronto e o levou para um fim e lhe
pediu que lhe falasse. E o bom homem perguntou de onde ele era. E ele respondeu
que era da corte do Rei Arthur e um companheiro da Távola Redonda. E o bom homem
perguntou quais conselhos ele queria ter, se era de penitência. E Lancelot
falou “sim”. “Em nome de Deus”, falou o bom homem e o conduziu para junto do
altar, e sentaram-se um ao lado do outro, e o bom homem lhe perguntou como se
chamava. E ele lhe respondeu que se chamava Lancelot do Lago, filho do Rei Ban
de Benuwig. E pois que o bom homem ouviu que ele era Lancelot, um do
Lago,\footnote{ O texto original alterna os termos \textit{Lac }e \textit{Lache}
para referir-se a lago, ambos grafados com iniciais maiúsculas.} 
um do mundo, do qual se falam mais boas coisas, ficou bastante assustado que o
viu atuar grande lamúria, e falou: “Senhor, deveis a Deus grande louvor por
ter-vos feito belo e sensível, quando no mundo não se sabe de beleza e nobreza
que vos igualem, que Ele vos emprestou o senso e a razão que tendes. Assim sois
obrigado a fazer um grande louvor, para conservar em vós Seu amor,  que o Diabo
não possua nada deste suave dom que Ele vos deu. Então servi a Deus de toda
vossa força e cumpri Seu mandamento, e não vos servis do dom que Ele vos deu
contra Seu inimigo mortal, o Diabo. Pois Ele foi mais suave perante vós que a
qualquer outro, e se Ele, pois, vos perder, muito se deve vos admoestar. E quem
é o mau servo, de quem se lê no Evangelho, que nos fez confessar que um bom
homem deu três grandes porções de seus bens? Quando ele deu a um servo um
bizantino de ouro\footnote{ A fonte original refere-se a um \textit{wisant
goltes}, ou \textit{wisant} de ouro, que Hans-Hugo Steinhoff traduz por
\textit{bizantino de ouro}. Mais adiante, o texto original apresenta o termo
\textit{bysant}.} e ao outro dois e ao terceiro cinco. E aquele a
quem deu cinco, ganhou com isso em tal medida, que veio perante seu senhor e
devia fazer conta e direito de seus ganhos, então falou: ‘Vede, Senhor,
outorgastes-me cinco bizantinos, que aqui estão e cinco deles, que eu ganhei’.
E pois que o senhor o entendeu, então falou: ‘Vem cá, verdadeiro servo e bom,
eu te contenho em minha companhia de minha casa’. 

Depois veio o outro, a que se tinham dado duas, e falou a seu senhor que ainda
tinha ganho dois daquilo. E respondeu justo como tinha respondido ao outro
servo. E aconteceu que o que tinha recebido um enterrou o seu sob a terra e
temeu seu senhor e não ousou vir à frente. Era o mau servo e o falsário e o
falso hipócrita de coração, pois o Espírito Santo não lhe veio. Não quis estar
à volta, odiou o mundo do amor de Nosso Senhor, quando recebeu o que a Sagrada
Palavra deixou saber. Quando a Sagrada Escritura diz: ‘Quem se queima, não se
abençoa’. Isso é tanto quanto dizer que o Espírito Santo não aquece aqueles que
não atentam para a palavra do Santo Evangelho, nem aquele que a ouve, ainda que
alce sua devoção a Deus. E esta fala vos contei por causa do suave dom que vos
deu Nosso Senhor, de tal sorte que vos fez mais belo que um outro e melhor, ao
que me parece, nas coisas que vos aparecem. E se do dom que Deus vos deu sois
Seu inimigo, então sabei que Ele deve inverter-vos a nada em curto tempo,
pedi-lhe então brevemente graça e verdadeira penitência em contrição de coração
e melhora de vida. E Ele tanto ama a verdadeira confissão dos pecadores que
vos faz assim poderoso e assim forte, que vos faz mais forte e mais
poderoso do que fostes algum dia no mundo”.  

“Senhor,” falou Lancelot, “esta parábola que me mostrastes dos três servos, que
lá tiveram os bizantinos, desconsola-me mais ainda que qualquer outra coisa.
Quando bem sei que Nosso Senhor aconselhou-me em minha infância de tudo da boa
graça que alguém pode ter. E por causa disto, Ele teve vontade de emprestar, e
quão mal Lhe retribuí de quanto me outorgou. Quando bem sei que devo ser
julgado como o mau servo que escondeu o Bizâncio na terra. Quando servi, por
todos os meus dias, aos Seus inimigos e contra Ele guerreei por meio do meu
pecado. E o Diabo tem a doçura e o mel, quando não me mostrou o tormento
mundano que vem a quem permanece em seu caminho”. 

E pois que o bom homem ouviu esta fala, principiou a chorar e falou: “Senhor,
este caminho de que me dizeis, bem o sei que quem ali permanece está
eternamente perdido. Mas como vedes, que se erra o homem de seu caminho quando
ele adormece e, em contrapartida, vem ao seu reto caminho quando está desperto,
assim acontece ao pecador quando adormece em pecado mortal e se aparta do reto
caminho e retorna a Deus por causa da penitência e direciona-se para Deus, o
mais alto senhor, que grita: ‘Eu sou a fidelidade e verdade, o caminho e a
vida!'” 

Então o eremita contemplou, em uma cruz, que lá estava o sinal de Nosso Senhor
Jesus Cristo pintado, e mostrou a Lancelot e falou: “Vedes a cruz?” “Sim eu”,
respondeu. “Então sabeis seguramente que esta figura abriu seus braços para
receber os pecadores, vós e outros, que se voltam para Ele, e clama por toda
parte ‘vinde, vinde’, e por causa disso, que seja tão misericordioso que Ele
por toda parte está pronto para receber aqueles que se voltam para Ele. E
sabei”, falou o bom homem, “que aqueles que se vitimam em tal medida como vos
digo, com verdadeira penitência fora da boca e confissão do coração e melhora
da vida, que nunca mais peque, então Deus vai com ele partilhar misericórdia. E
por causa disto, dizei de pronto vossa vida e vosso ser à devoção, perante mim
e perante ele, e eu vos ajudarei a desculpar-vos por meu poder e
aconselhar-vos-ei como puder”. E Lancelot refletiu consigo um pouco
como nunca tinha refletido sobre sua vida nem sobre a rainha, grande confissão
a isto o trazia. E suspirou tão profundamente do coração e ficou
calado que nem uma palavra pôde sair de sua boca, e ainda então bem gostaria de
dizer e não se atreveu a começar, como aquele que mais estava negado que audaz.


E o bom homem lhe pediu por toda via que dissesse seu pecado e não o deixasse,
pois de outro modo estaria perdido, que fizesse como ele lhe mostrava. E
prometeu-lhe a eterna alegria, se o dissesse, e o inferno, se ele se calasse. E
dizia a ele tanto com boa fala e com bom conselho, que Lancelot principiou a
dizer: 

“Senhor,” falou Lancelot, “é assim que estou morto com pecado por meio de uma
mulher a que tive amor por toda minha vida, que é a rainha, esposa do rei
Arthur, este é aquele que me deu muito ouro e prata e ricas vestes que eu tenho;
quero dá-las aos cavaleiros pobres. E eis aquela que me alçou em grande, alta
coragem, e aquele por cuja vontade tive grande valentia, de que diz o mundo
inteiro, e aquele que me fez vir da pobreza à riqueza e do desconforto à honra
mundana. Quando bem sei que por meio deste pecado dela, Nosso Senhor
está irado comigo tanto como Ele me indicou nesta noite”. E então lhe contou
como ele tinha visto o Santo Graal e ele nunca se teria ponderado honrar-se
perante ele ou por causa dele, nem por causa do amor de Nosso Senhor. E pois
que tinha contado ao bom homem todo seu ser e sua vida, então lhe pediu por
Deus que o ajudasse e cavalgou. “Seguramente Senhor, nenhum conselho vos pode
ajudar, prometei então a Nosso Senhor que nunca mais caireis neste pecado. Pois
se quereis de tudo se dispensar, assim pedi graça por causa de Nosso Senhor e
deixai-vos arrepender de todo o coração, assim sei bem que Nosso Senhor vos
deve chamar o bom cavaleiro e deve abrir-vos as portas celestes, onde a vida
eterna está pronta para todos que lá entrarem. Nesta medida como agora estais,
que não tendes nenhum conselho a fazer, quando seria tanto como aquele que
construísse, sobre um mau fundamento, uma torre forte e alta. Então acontece
que por muito tempo se construiu e fizeram-se cômodos, e caiu sobre um monte.
Assim estaria totalmente perdido o tormento de Nosso Senhor, se [não] o
receberdes, pois, de bom coração e o fizerdes a tempo, quando seria a semente
que se joga pelo penhasco, que os pássaros comem, com que ninguém para nada
conta.” “Senhor,” falou Lancelot, ``não me dizeis nenhuma coisa
do que devo fazer, se Deus me deixar viver”. “Então crede-me”, falou o bom
homem, “que nunca mais enganareis vosso Salvador, assim como nunca caireis em
pecado mortal com a rainha, nem com outras mulheres, nem fareis coisa alguma
que o possa enfurecer convosco”. E ele lhe acredita como um bom cavaleiro. 
“Então contai-me do Santo Graal”, falou o bom homem, “como se vos
aconteceu”. E ele lhe contou das três falas, que a voz tinha dito na capela,
onde ele foi chamado “pedra, fel e álamo”. “Por Deus, dizei-me este significado
das três falas, quando nunca ouvi fala que de tão bom grado desejasse saber
como dessa. E por esta causa vos peço que seguramente me queirais fazer, quando
bem sei que bem o sabeis.” 

Então começou o bom homem a refletir um momento, e quando ele falou, então
disse: “Seguramente, Lancelot, não me maravilha que vos sejam faladas essas
três falas, quando fostes por todos os vossos dias o homem mais maravilhoso do
mundo, por causa disso não é desmedido se se vos falam falas maravilhosas, e
não a outros. E por esta causa, vos agrada saber a verdade, assim devo eu de
bom grado vos dizer. Então escutai”, falou o bom homem. “Contais-me que se vos
chamou ‘Lancelot, mais duro que uma pedra e mais amargo que um fel e muito
mais nu e despido que um álamo, fugi daqui’, e nisto que se vos chamou ‘mais
duro que uma pedra’, pode-se aí entender uma maravilha, quando todas as pedras
são duras de natureza, e uma mais que as outras. E pela pedra podeis entender o
pecador que está enrijecido e embebido no pecado. E seu coração está assim
empedernido que não pode ser amainado, nem por causa da água, nem por causa do
fogo. Por causa do fogo não pode ser amolecido, nem o fogo do Espírito Santo
pode nele entrar, quando não acha nenhum sítio, para purificar o recipiente que
está impuro e horrendo de velhos pecados, que ele fez e desempenhou dia a dia.
Por causa disto não pode ser amolecido por nenhuma água,  que é a doce água do
Espírito Santo; doce chuva não o pode amolecer. Quando Nosso Senhor Deus não Se
alberga nunca em nenhum sítio onde estiver Seu Inimigo, quando quer que a casa,
que toma por albergue, esteja pura de toda a iniquidade e de todas as coisas. E
por meio desses sentidos é o pecador chamado pedra, por causa da grande dureza
que nele existe. Quando é para se saber por que causa tu\footnote{ Novamente
aqui, no mesmo diálogo, altera-se o pronome de tratamento de “vós”
 para “tu”.}  estás muito mais duro que
uma pedra, assim é tanto como dizer que estás muito mais pecador que outro
pecador.” 

Pois que o bom homem o tinha dito, então começou a pensar, e de pronto
respondeu: “Eu devo te dizer como tu és muito mais pecador que outras pessoas.
Bem ouviste dos três servos, a quem o rico homem deu seu ouro para crescer e
multiplicar. E os dois que mais tinham, foram servos leais e sábios. O
terceiro, que pouco tinha, esse foi servo mau e desleal. Então vê, se tu
poderias ser um servo a quem o Senhor Deus acometeria Seu bem para majorar!
Parece-me que Ele muito mais te acometeu que a outros cavaleiros terrestres,
parece-me que não se acha nenhum homem a que o Senhor Deus deu como tanta graça
como te deu. Ele te deu beleza acima da medida, deu-te senso e discernimento
para reconhecer o bem e o mal. Ele te deu valentia e ousadia, e depois te deu
caridosamente boa sorte, que superaste tudo quanto tivesses em intenção. 
E todas essas coisas Deus te emprestou para que fosses Seu
cavaleiro e Seu servo. E não te as deu para que todas essas coisas em ti fossem
minoradas, mas para crescer e aumentar. E foste tão mau servo e desleal que O
deixaste e serviste Seu Inimigo e por toda via guerreaste contra Ele. E foste o
mau devedor, que se aparta de seu senhor quanto tem seu soldo e vai ajudar seu
inimigo contra ele. Assim fizeste contra Nosso Senhor. Quando tão logo Ele bem
te pagou e ricamente, então O deixaste e foste servo daquele que Lhe faz guerra
toda via. E isso não fez nenhum homem, conforme meu pensamento, que se tivesse
pago tão bem quanto a ti. E por causa disso podes bem ouvir que és mais duro
que uma pedra e muito mais pecador que outros pecadores. E ainda quem lá quiser
pode bem entender outra medida, da pedra vê-se bem vir alguma doçura sobre o
Mar Vermelho, onde o povo israelita permaneceu longo tempo, então se vê
confessamente que o povo teve vontade de beber, e um se queixava ao outro, que
Moisés veio em uma dura penha e falou, ainda que não pudesse ser: ‘Não podemos
fazer vir água para fora desta penha, para que todo o povo tenha de beber?’
Aconteceu que um rio veio de dentro, e ele pediu a Nosso Senhor que todos ali
tivessem de beber. E assim se calou seu murmúrio e foi aplacada sua sede. Assim
se pode bem dizer que da pedra se conta com alguma doçura. Mas de ti nunca veio
nenhuma, e por causa disso podes bem ver abertamente és mais duro que uma
pedra.” 

“Senhor,” falou Lancelot, “então me dizei, por que fui chamado ‘mais amargo que
um fel’?” “Eu te devo dizer”, falou o bom homem, “pois me escuta: eu te
mostrei por que em ti se albergou toda a dureza, daí nenhuma doçura pode vir.
Não deves considerar que aí permanece outra coisa que não amargura, e a
amargura está em ti de tal maneira como deveria ser a doçura. E por causa disto
igualastes bem o fel impuro, onde não está nenhuma doçura e não mais que amargo
e impuro. Então te informei por que tu és muito mais duro que uma pedra e mais
amargo que um fel. Então está a terceira coisa por saber, como és mais nu e sem
serventia que um álamo, de que se diz. Isto nos significa o Evangelho, que se
lê no Dia de Ramos,\footnote{ Atualmente se diria \textit{Domingo de Ramos},
expressão que consta da tradução de Hans-Hugo Steinhoff, à página 139. 
 } que veio Nosso Senhor contra Jerusalém cavalgando sobre um burro, que
aqueles que conheciam abraâmico,\footnote{ O texto original apresenta o vocábulo
\textit{abrahamisch}. Hans-Hugo Steinhoff preferiu proceder à
adaptação para o alemão contemporâneo, traduzindo o étimo por
\textit{Hebräisch}.} cantaram perante Ele a doce
canção, quando da Santa Igreja todo homem faz confessar o dia que se chama Dia
de Ramos. No mesmo dia pregou o Alto Professor e Senhor e o Alto Mestre e
Profeta na cidade de Jerusalém, que albergava toda a dureza. E pois que se
esforçou durante o dia todo, e estava separado da prédica, achou um álamo que
estava no caminho, que estava muito belamente aconselhado com galhos e com
folhas, mas nenhum fruto ele tinha. E então Nosso Senhor veio à árvore e a
achou desaconselhada de frutos, então falou assim irado: ‘Amaldiçoada seja a
árvore que não porta nenhum fruto!’ 

Assim aconteceu do álamo que estava fora de Jerusalém. Então contempla se podes
estar tão despido e nu quanto ele estava. Então quando o Alto Mestre veio à
árvore, achou suficientes folhas, que bem tomou, como quis. Quando então o
Santo Graal veio até tu, Ele te achou assim desaconselhado, que em ti não achou
nem bom pensamento nem boa vontade, mas impureza de pecador e falta de
castidade. Ele te achou desaconselhado de folhas e flores, isso é tanto quanto
dizer de todas as boas obras. Então se te diz a fala que tu me contaste:
‘Lancelot, muito mais duro que uma pedra, muito mais amargo que um fel, muito
mais despido e nu que um álamo, foge daqui!'” 

``Seguramente Senhor,” falou Lancelot, “como muito me dissestes e indicaste
claramente que eu por acerto fui chamado ‘pedra, fel e álamo’. Quando todas as
coisas, que me dissestes, que estão em mim albergadas, então que me mostrastes
que não fui longe demais, eu ainda posso volver, se me precaver de cair em
pecado mortal, e por causa disso prometer a Nosso Senhor e depois a vós que
nunca mais devo voltar à vida que por longo tempo levei, e devo manter-me casto
e  manter meu lábio assim puro quanto eu puder. Quanto a servir à cavalaria
e manter as armas, não posso renunciar enquanto estiver são e pronto”.

E pois que o bom homem ouviu esta fala, ficou muito feliz consigo e falou para
Lancelot: “Seguramente, se quereis deixar o pecado da rainha, eu vos\footnote{
Nesta altura do texto, retoma-se o pronome formal de tratamento \textit{ir},
que corresponde ao pronome \textit{Ihr}, do alemão contemporâneo.
 } digo por certo que Nosso Senhor Deus vos deve ter amor e vos deve enviar
guarida, e volver Sua misericórdia para vós e vos deve dar poder para terminar
alguma coisa que não lograstes levar ao fim, por causa de seus pecados”. 
“Senhor,” falou ele, “eu o deixo em tal medida que nunca mais com ela
pecarei nem com nenhuma outra esposa”. E pois que o bom homem o ouviu, ficou
consigo muito alegre e lhe deu tal penitência que lhe pareceu que ele podia
manter e o abençoou e o absolveu e lhe pediu que passasse a noite ali. E ele
lhe respondeu que precisaria bem fazê-lo, quando não tinha nenhum cavalo sobre
o qual pudesse cavalgar, nem escudo, nem nenhuma espada. “Isto Deus vos deve
aconselhar a partir de amanhã à noite,” falou o bom homem, “quando aqui junto
mora um dos meus irmãos, que é cavaleiro, que deve bem me enviar corcel e armas
e tudo de que precisais, tão logo eu lhe pedir”. E Lancelot responde
que de bom grado queria permanecer junto a ele, e o bom homem ficou muito feliz
consigo.

Assim permaneceu Lancelot junto ao bom homem, que dele muito bem cuidou e lhe
mostrou muitas boas falas, que Lancelot muito se arrependeu da vida que tinha
levado, quando bem viu que lá morreria, que perderia a alma e bem poderia
também por meio disto perder o corpo, se dele saísse. E por causa disso se
arrependeu que já tivesse ganho fala com a rainha e criticou muito a si próprio
e prometeu em seu coração que por todos os dias de sua vida nunca mais queria
cair nisso, e ofertou a Nosso Senhor, de bom grado, glória dos pecados que já
cometeu desde o tempo em que nasceu, e pediu-Lhe que lhe presenteasse um bom
fim. Aqui se calam as notícias dele e se volta para Parsifal. 

\chapter{A tentação de Parsifal}

\textsc{Aqui nos dizem} as notícias que, porque Parsifal estava separado de
Lancelot e porque volveu para a cela, quando ele bem pensou saber notícias
verdadeiras da cruz, de que era a fala. E pois que tinha voltado, então
aconteceu que não conseguiu achar nenhum caminho que lhe indicasse o local,
mesmo assim dirigiu-se o melhor que pôde para aquela direção. E então ele veio
à capela, então bateu à janela da enclausurada. E ela de pronto abriu como
aquela que nunca dorme, e tirou sua cabeça para fora quanto pôde, e
perguntou-lhe quem ele seria. E ele lhe respondeu, que era da corte do Rei
Arthur e se chamava Parsifal de Gales. E pois que ela ouviu seu nome, ficou
muito feliz, quando ela muito lhe teve amor, como era culpada em o fazer,
quando era seu sobrinho, e chamou sua companhia. E mandou que abrissem a porta
ao cavaleiro, que estava fora, e que lhe dessem de comer, se ele estivesse em
precisão, e que lhe servissem em tudo que pudessem, porque não tive mais amor a
nenhum homem no mundo que a ele. E os que estavam lá dentro fizeram o que
ordenou e foram à porta e a abriram e receberam o cavaleiro e o desarmaram e
lhe deram de comer. E perguntou se tanto podia falar com a enclausurada. E eles
falaram: “Não, quando amanhã após a missa, assim bem cremos, vós lhe falais”. E
ele se deixou com isto contentar e deitou-se em uma cama, que lhe estava feita,
e repousou toda a noite, como estava muito cansado. 

Pela manhã, pois que era dia claro, levantou-se então Parsifal e ouviu uma
missa cantada. E pois que tinha ouvido missa, então se armou e foi à
enclausurada e falou: “Por Deus, Senhora, então me dizei notícias do cavaleiro
que aqui à frente ontem cavalgava, para o qual dissestes  e que deveríeis bem
confessá-lo; quando me importa que eu saiba quem ele seja”. Pois que a mulher
ouviu estas notícias, então lhe perguntou por que o seguia. “Porque”, falou
ele, “eu nunca mais vou descansar até que saiba quem ele seja e o ache e com ele
lute, quando tanto me fez que não posso deixar passar com honra”. “Hei,” falou
ela, “o que é isto que dizeis, quereis com ele duelar, quereis morrer como
vossos irmãos, que estão mortos e aniquilados por causa de sua temeridade!
Quando seguramente, se assim morrerdes, seria grande lástima, e vossa linhagem
seria bastante diminuída. Sabeis  que nisto perdeis, se me duelares
com o cavaleiro? Devo dizer-vos,  e é verdade que a grande demanda do Santo
Graal está iniciada, e sois nela um companheiro, como me parece, e deve em
breve ser levada ao fim, se Deus quiser, e isto é que deveis ganhar mais honra
nela que vós mesmos considerais, que se vos contiverdes de que luteis com o
cavaleiro. Quando bem sabeis que nesta terra há muitos outros cavaleiros que
consideram terminá-la, que três são escolhidos, que devem ter o louvor e honra
sobre todos os demais. E os dois devem ser virgens e o terceiro, viver casto. Os
dois que, portanto, devem ser virgens, são, um o cavaleiro que procurais, e
vós, o outro, Bohort de Gália o terceiro. E com os três a demanda deve ser
terminada. E da vontade que Deus vos preparou tal honra, assim seria uma grande
lástima, que vós outros morrêsseis. Então sabei, se duelares com aquele que
buscais, seguramente sem erro, ele deve vos ter morto de pronto, quando é muito
melhor cavaleiro do que sois e que qualquer outro que viva”. 

“Senhora,” falou Parsifal, “parece-me no que me dizeis de meus irmãos, que
bem me reconheceis e bem sabeis quem eu sou”. “Eu bem o sei, e é bem justo que
eu o saiba”, falou a enclausurada, “quando sou vossa tia, e vós sois meu
sobrinho. E não tenhais maravilha se estou em tal sítio pobre, por Deus, sabei
que sou aquela que se chama a rainha da Terra Deserta, e me vistes de outro
modo do que agora me vedes. Quando eu era uma das mais ricas mulheres do mundo,
e ainda então não me aprazia tanto a riqueza, a pobreza me agradava mais, em
que então estou”. 

Pois que Parsifal ouviu esta fala, principiou a chorar da compaixão que
teve, quando bem pensou consigo aquilo que ela lhe tinha dito; e bem a
reconheceu. E então se sentou à frente dela e lhe perguntou mais de sua mãe e
de seus parentes. “Como”, falou ela, “caro sobrinho, não sabeis nenhuma notícia
de vossa mãe?” “Seguramente”, falou ele, “Senhora, não, eu nem sei se esteja
morta ou ainda vivaz, quando fartamente me sucede no sono e me parece como ela
fala comigo, que ela deve com justiça mais se queixar de mim que me honrar,
quando bem a tratei”. E pois que a mulher ouviu esta fala, então suspirou e lhe
respondeu entristecida: “Seguramente, ver vossa mãe no sono, não falhastes;
quando está morta desde que seguistes para a corte do rei Arthur”. “Senhora,”
falou ele, “como é isto?” “Em verdade”, falou ela, “vossa mãe estava tão
triste de vossa separação, que morreu no mesmo dia tão logo se penitenciou”.
“Então que sua alma tenha glória! Quando me é à vez lamentável; porque isso
aconteceu, então devo padecê-lo, quando todos nós à morte devemos ir; quando
seguramente não ouvi nunca mais falar disto. Quando o cavaleiro que busco,
sabei, por Deus, quem ele seja, se seja aquele que veio à corte com a arma
vermelha?” “Sim,” falou ela, “assim é, e devo dizer-vos de que significado se
trata.

Sabeis que após o Advento de Nosso Senhor Jesus Cristo foram três távolas no
mundo. A primeira foi a távola de Jesus Cristo, à qual os apóstolos comeram de
quando fartamente; esta foi a távola que manteve corpo e alma com iguarias do
Céu. À távola sentaram-se os irmãos que lá estavam, que eram uma só coisa, de
que David fala em seu livro uma fala muito bela: ‘É pois uma boa coisa e
maravilhosa de irmãos que se mantinham juntos por uma vontade e uma obra!’
Pelos irmãos, que lá se sentavam à távola, pode-se entender uma concórdia e
humildade e toda boa obra. E a mesma fez-se para o Cordeiro sem máculas, que
foi crucificado por causa de nossa redenção. 

Após da távola foi feita uma outra távola em sua igualdade. Essa foi a
távola do Santo Graal, na qual os grandes sinais aconteceram fartamente nesta
terra, ao tempo de José de Arimateia, no começo da fé cristã foi trazida para
esta terra, pois que toda gente nobre e descrente deveria ter por toda via uma
parábola. Aconteceu que José de Arimateia veio a esta terra e à vez muitos
povos com ele, tantos que podem ser quatro mil homens. E pois que vieram a esta
terra, então se desconsolaram muito, pois temiam que lhes faltasse o alimento,
quando à vez tinham muitos povos entre eles. Um dia seguiram através de uma
floresta. Lá não acharam nem de comer nem ninguém, e ficaram por isso muito
assustados, quando a isso não estavam acostumados, e padeceram o dia. E no
outro dia seguiram para cima e para baixo e acharam uma velha mulher, que lhes
trouxe doze pães retirados de um forno, e eles os compraram. E porque eles
tinham de partilhá-los,  então começo entre eles uma ira e uma guerra, quando
um não queria como o outro.

Esta aventura veio perante José, de que ele estava bem irado, pois o sabia,
e mandou que se trouxesse o pão perante ele. E se os trouxe, e veio cada qual
com o que tinha comprado. Pois que fora alertado sobre sua boca, de que um não
queria como o outro, então mandou ao povo que se sentasse em roda. E ele trouxe
os pães para lá e os colocou no mais alto da távola do Santo Graal e por causa
disso os doze pães cresceram tanto que quatro mil foram com isto saciados e
tiveram todos o suficiente. E pois que isso viram, então agradeceram a Nosso
Senhor Deus da graça, que Ele rapidamente os tinha salvo.

Na távola havia um assento, onde Josephus, o filho de José, deveria
sentar-se. E o assento era feito tal que seu mestre e padre se deveria ali
sentar, e não era permitido a mais ninguém, e foi consagrado e abençoado pela
mão de Nosso Senhor, tal como a história nos informa. E receberam a instrução
de que deveria ser sobre toda a Cristandade e no sítio que Nosso Senhor lhe
tinha assinalado. E por causa disto ninguém foi tão audaz que ousasse lá se
sentar. E o assento era feito à igualdade daquele em que Nosso Senhor, na
Quinta-feira Santa,\footnote{ A expressão alemã medieval para a Quinta-feira
santa é \textit{grún donrstag},  que gerou
a atual expressão \textit{Gründonnerstag}.}  junto a seus
apóstolos, sentou-se. E assim deveria dirigir aqueles que se sentassem à távola
do Santo Graal, e deveria ser seu mestre e seu senhor. Então aconteceu assim,
que ele veio à terra, e estiveram por um bom tempo desorientados pela terra
estranha, que dois irmãos, que lá eram parentes de José, tiveram inveja de que
Nosso Senhor o tivesse assim elevado e o tivesse eleito o melhor na companhia.
E tiveram sua fala com os principais e falaram que não mais
deveriam padecer que ele fosse seu mestre, quando eram de linhagens tão altas
quanto ele. E por causa disso nunca queriam estar abaixo dele e nunca queriam
chamá-lo mestre. E pela manhã, pois que tinham subido a uma montanha alta e as
távolas estavam postas, e deveriam assentar José no mais alto assento, então
contradisseram os dois irmãos. E um se sentou à frente de sua visão, e
aconteceu-lhe um tal sinal, que a terra o tragou. E o sinal foi de pronto dito
por sobre toda a terra, pelo que o assento foi chamado Assento
Amaldiçoado.\footnote{ O texto apresenta a expressão grafada
com iniciais maiúsculas.} E nunca mais foi alguém tão ousado que
se atrevesse a lá se sentar, salvo aquele que Deus havia eleito. 

Depois dessa távola, foi a Távola Redonda feita com o conselho de Merlin, não
sem grande significado, quando assim como a chamaram Távola Redonda, assim é
para entender a novidade do mundo e o percurso dos planetas e dos elementos.
Quando no círculo no céu se veem as estrelas e muitas outras coisas, por causa
disto se pode bem dizer que a Távola Redonda significa esse mundo de direito.
Quando bem podeis ver que de outras terras que costumam tecer cavaleiros, seja
na Cristandade ou no mundo pagão, vêm os cavaleiros para a Távola Redonda.
Quando Deus lhes fez a graça, tal que fossem cavaleiros e companheiros, então
vos considereis felizardos como se tivésseis ganho todo o mundo, quando bem se
vê que deixais vossos pais e mães e esposas e filhos, para ali ser
companheiros. E por vós mesmos bem vistes que vos aconteceu, isso desde que vos
apartastes de vossa mãe e se vos fez companheiro à Távola Redonda. Nunca desde
então desejastes retornar, e fostes de pronto aprisionado pela doçura da
companhia, que é obrigatório ser entre os companheiros. Quando Merlin fez a
Távola Redonda, falou isto para aqueles que deveriam ser companheiros, que se
deveria saber a verdade do Santo Graal, em que se poderiam ver alguns sinais ao
tempo de Merlin. E se pergunta como se poderiam reconhecer aqueles que eram os
melhores, e ele falou que deveriam ser três que o levariam ao fim, e deveriam
ser os dois virgens\footnote{ O vocábulo encontrado no texto original é
\textit{megde}, o mesmo étimo empregado para caracterizar o Castelo das Moças.}  
e o terceiro, casto. E um dentre os três deve estar acima de seu
pai como o leão à frente do leopardo, de valentia e de castidade e de audácia,
que se deve considerar à frente dos mais elevados, à frente dos mestres e à
frente de todos eles. E por toda via a Távola Redonda deve buscar o Santo
Graal, até que Nosso Senhor Deus tão repentinamente envie entre eles, que deve
ser maravilha. E pois que ouviram esta fala, então disseram: ‘Obrigado, Merlin,
que por isso ele vem ser tão valente como dizes, deves fazer-lhe um único
assento, em que ninguém se deve sentar além dele mesmo,  que ele seria sobre os
outros tão grande que todo homem bem o poderia reconhecer’. ‘Assim eu digo,’
falou Merlin, ‘tão grande e tão maravilhoso!’ E então fez seu assento. E pois
que o tinha feito, beijou-o e falou que o tinha feito por causa da boa vontade
do bom cavaleiro que sobre ele deve repousar. E de pronto lhe perguntaram o que
lá deveria acontecer do assento. ‘Seguramente,’ falou ele, `lá deve ainda disto
acontecer alguma grande maravilha, quando nenhum homem jamais se deve sentar
nele, que não seja morto ou ferido, até que o verdadeiro cavaleiro nele se
assente’. ‘Em verdade’, falaram eles, “assim se fazem em grande medo aqueles
que nele se sentam’. ‘Em grande aflição eles vêm’, falou Merlin, ‘e por causa
das maravilhas que dele devem acontecer, então deve chamar-se o Assento
Maravilhoso\footnote{ Também esta designação encontra-se em letras maiúsculas
no texto medieval original.}’. Caro sobrinho”, falou a mulher,
“então vos contei por inteiro que coisas se fizeram pela Távola Redonda e por
que foi feito o assento maravilhoso,\footnote{ Nesta oração, a expressão
anterior apresenta-se em iniciais minúsculas.}  e quando ele foi
feito, por cuja causa alguns cavaleiros estão mortos, que não eram valorosos
para que devessem nele se sentar. Então devo vos dizer inteiramente por que o
cavaleiro veio à corte em armas vermelhas. Vós bem sabeis que Jesus Cristo foi
entre os apóstolos senhor e mestre por mandato e a partir disto foi imaginada a
távola do Santo Graal por José e a Távola Redonda para a vontade dos
cavaleiros. Nosso Senhor Deus prometeu aos apóstolos que após Sua ressurreição
deveria vir e deveria visitá-los e vê-los. E eles esperaram de acordo com Sua
promessa, entristecidos e assustados. Então lhes aconteceu no dia de
Pentecostes que estivessem em uma casa, e estivessem fechados os portões, que
viesse o Espírito Santo dentre eles em igualdade ao fogo e os consolou e os
assegurou naquilo em que estavam em pavor, e os separou e os enviou à terra
para pregar ao mundo e anunciar o Santo Evangelho.

Assim aconteceu no dia de Pentecostes aos santos apóstolos, pois Nosso
Senhor veio consolá-los. Assim veio também o cavaleiro que deveis ter à frente
dos mais altos e como mestre.  E como Nosso Senhor veio em igualdade do fogo,
assim veio o cavaleiro em armas vermelhas na cor do fogo. E assim como as
portas estavam fechadas onde estavam os apóstolos e o Espírito Santo veio até
eles, assim estavam as portas do palácio fechadas antes de o Santo Graal vir
dentro. Ele veio tão de repente entre vós, e entre vós não era ninguém tão
sábio que soubesse de onde ele teria vindo. E no mesmo dia foi começada a
demanda do Santo Graal, que nunca se deve deixar até que se saiba a verdade, e
da lança por cuja causa isto é, que tantas aventuras disso aconteceram nesta
terra. 

Então devo vos dizer a verdade do cavaleiro, por que causa não deveis duelar
com ele, quando bem sabeis que sois obrigado a assim fazer, pois sois seu irmão
e companheiro na Távola Redonda, quando não poderíeis vos proteger contra ele,
quando ele é muito melhor cavaleiro do que sois”. “Em
verdade,” falou Parsifal, “vós me dissestes tanto que nunca me agrada lutar
com ele. Quando por Deus, indicai-me o que posso fazer e onde posso
encontrá-lo. Quando, se o tivesse por companheiro, nunca mais dele me separaria
tanto quanto pudesse servi-lo”. “Para tanto devo aconselhar-vos da
melhor maneira que posso,” falou ela, “quando agora não posso vos dizer onde
ele esteja; então os sinais, pelos quais podeis reconhecê-lo, devo bem vos
dizer. E quando o encontrares, assim mantende sua companhia tanto quanto
possais. 

Deveis cavalgar até um castelo, que se chama Deus, para uma de suas primas.
Ela é a irmãzinha daquele a quem pertence o castelo; pelo amor, eu creio, ele
deve lá tomar albergue esta noite. E se ela vos puder indicar que fim ele
levou, então o segui, tão logo possais. Se ela disso não vos disser, então
cavalgai direto para o castelo de Korpanich,\footnote{ Nas versões bretã e
portuguesa de \textit{A Demanda do Santo Graal}, corresponderia ao castelo de
Corbenic.}  onde o rei ferido mora. Pois bem sei que lá ouvireis
retas notícias dele, se não achares lá dentro”. 

 Assim longamente falaram Parsifal e a enclausurada do cavaleiro, até que
bem eram as sextas horas. E então ela falou: “Caro sobrinho, deveis ainda hoje
permanecer junto a mim, assim vos farei o melhor conforto, quando faz longo
tempo que não mais vos via, e me deve doer vossa separação”. “Senhora,'' falou
ele, “tenho coisas demais a fazer que não posso permanecer, então vos peço por
Deus que bem me deixais cavalgar”. “Seguramente,” falou ela, “com minha licença
assim vos separais daqui; até a manhã, assim que tiverdes ouvido missa, então
bem vos darei licença”. E ele lhe respondeu, que assim de bom grado gostaria de
permanecer, e se fez de pronto desarmar. E os que dentro estavam tinham coberto
a mesa e comeram aquilo que a mulher lhes tinha preparado. Assim permanece lá
Parsifal junto a sua tia e falaram entre eles dois de algumas coisas tanto que
ela falou: “Caro sobrinho, assim é que  vos protegestes até este tempo de uma
tal maneira que vossa virgindade não foi diminuída nem quebrada, e ainda nunca
soubestes o que seja a junção carnal. E disso bem precisais. Quando se tanto
vos tivesse acontecido, que vossa carne se tivesse maculado de algum pecado e
então de vossa posição como o mais alto companheiro na demanda teríeis caído
como Lancelot do Lago, que por sua impura falta de castidade perdeu de
terminar aquilo pelo que os outros estão em trabalho. Por isto vos peço que
mantenhais vosso lábio limpo  na cavalaria, tal que
possais chegar ao Santo Graal imaculado de castidade, que sois de uma tão bela
nobreza que um cavaleiro fizesse. Quando entre todos aqueles que estão na
Távola Redonda, nenhum há que não tenha quebrado sua virgindade, salvo vós e
Galaat, o bom cavaleiro, de quem vos digo”. E ele respondeu
que deveria, pela vontade de Deus, tão bem proteger-se quanto fosse sua
precisão.  

O dia todo permaneceu Parsifal lá, e sua tia muito bem lhe indicou o que
fazer. E, sobretudo, pediu-lhe que mantivesse casta sua carne. E ele lhe
prometeu que o queria fazer. E pois que tinham por bom tempo conversado da
corte do Rei Arthur, então perguntou a ela Parsifal que aventuras a tinham
trazido para sítio tão estranho e pelo que tinha deixado sua terra. “Valha-me
Deus,” falou ela, “isso fez o medo da morte, pelo que para cá fugi, quando
sabeis bem, pois estáveis na corte, que meu senhor, o Rei, guerreou contra o
rei Libram. E então logo aconteceu, assim que meu senhor, o Rei, estava morto,
que eu lá era mulher e tive medo de que me matasse, se me pudesse ter. E então
tomei a maior parte de meus bens e fugi para este ermo, para que não fosse
achada. E fiz por bem fazer esta cela e esta casa, que bem vedes, e tomei junto
a mim meu capelão e meus criados, e me encerrei nesta clausura, em tal medida
que nunca saio dela, se Deus quiser, e devo aqui terminar minha vida e morrer
no serviço de Nosso Senhor Deus”. “Em minha verdade,” falou Parsifal, ``isto é
maravilhosa aventura. Dizei-me o que veio a vosso filho Deabias, quando desejo
muito saber como ele está”. “Seguramente,” falou a mulher, “ele cavalgou a
serviço do rei Pellis, nosso parente por suas armas que tem. E desde então ouvi
dizer que ele o fez um cavaleiro, pois são bem dois anos que não o vejo, pois
ele procura torneios através da Grã-Bretanha, e bem creio que devereis achá-lo
em Korpanig,\footnote{ A própria fonte medieval apresenta esta grafia
alternativa a Korpanich.}  se para lá cavalgares”. “Seguramente,”
falou ele, ``se por outra coisa não cavalgo para lá, senão para vê-lo, assim
quero ir para lá, quando à vez muito desejo vê-lo”. “Valha-me Deus,” falou ela,
“eu queria que o tivésseis aqui achado, quando bem estaria confortável que ele
estivesse conosco”. 

Assim permaneceu Parsifal todo o dia junto a sua tia. E pela manhã, tão
logo quanto tinha ouvido missa e se tinha desarmado, então se separou de lá e
cavalgou todo o dia através da floresta, que lá era grande e maravilhosa, em
tal medida, que não o encontrou homem nem mulher. Após a hora da refeição,
então aconteceu que ele ouviu à mão direita um berrante, e volveu para lá e ele
pensou que seria uma casa espiritual ou uma ermida. E pois que tinha cavalgado
um pouco, então viu que era um convento, que estava fechado com fossas
profundas e muros altos. E ele cavalgou para lá e chamou por tanto tempo à
porta, que se abriram. E pois que os que estavam dentro viram que ele estava
armado, então bem pensaram que ele era cavaleiro ou seria cavaleiro de
aventura, e lhe fizeram desarmar e conduziram seu corcel a um estábulo e lhe
fizeram suficiente. E um dos irmãos conduziu Parsifal a uma câmara para
repousar. À noite ele estava albergado o melhor que os irmãos puderam. E pela
manhã aconteceu-lhe que não despertou antes das primas horas, e foi ouvir missa
no mesmo convento. E pois que ele veio à igreja, então viu, à mão
direita, que lá estava uma grade de ferro, e lá estava um irmão, armado com as
armas de Nosso Senhor, e queria principiar missa. E ele se virou para lá, como
aquele que tivesse vontade de ouvir o serviço de Deus. E pois que ele veio à
grade, então virou-se para ir até lá e conseguiu, como lhe pareceu. E quando o
viu, deixou-se contentar e ajoelhou-se lá à frente e dentro viu uma cama
pronta, muito rica de lençóis de seda e de outras coisas, o melhor que se podia
pensar.

Parsifal contemplou a cama por tanto tempo até que lhe pareceu que lá jazia
um homem ou uma mulher, e ele não sabia quem era, quando o rosto estava coberto
com um pano de seda, de modo que claramente não conseguiu vê-lo. E ele ouviu a
missa que o bom homem tinha principiado. E quando se chegou ao ponto em que o
padre deve erguer Nosso Senhor Deus, então se dirigiu a ele um, que lá jazia, e
sentou-se em sua cama e descobriu seu rosto. E era um homem velho e grisalho e
tinha uma coroa de ouro sobre sua cabeça, e lhe estavam os ombros nus, e
descoberto até o umbigo. E pois que Parsifal o viu, viu que estava muito
ferido e tinha muitas feridas em seu corpo, nos ombros,  nos braços e no
rosto. E então aconteceu que o padre tornou visível o corpo de Jesus Cristo,
então estendeu suas mãos à frente e principiou a chamar: “Querido e doce Pai,
não esqueçais a minha dor!” E ao depois não jazeu de novo e estava tudo em sua
oração e segurou as mãos à frente de seu Salvador, com a coroa de ouro sobre
sua cabeça.

Longamente contemplou Parsifal o homem que jazia sob a coberta, quando lhe
pareceu muito desconfortável demais estar com suas feridas que ele tinha. E o
viu tão velho como lhe pareceu, ele tinha bem quatrocentos anos. E
contemplou todo o tempo, quando lhe pareceu que a coisa fosse maravilhosa. E
ele viu que lá se cantava missa, que o padre tomou o corpo de Nosso Senhor
entre suas mãos e o levou para aquele que jazia na cama e lhe deu para usar. E
de pronto quando o teve, tomou-lhe a coroa da cabeça e colocou sobre o altar, e
aquele se riu de novo em sua cama como dantes e foi coberto, que não se o via.
E de pronto fez-se o padre para fora como aquele que tinha cantado missa. 
Pois que Parsifal viu a coisa, então saiu da igreja e foi à câmara, onde
à noite tinha se deitado, e chamou por um dos irmãos e falou: “Caro Senhor, por
Deus, informai-me o que vos pergunto”. “Senhor cavaleiro, dizei-me o que é, eu
vou dizer-vos de bom grado, se o souber”. “Em verdade,” falou Parsifal, “eu
vos devo dizer. Eu estava mesmo no mosteiro e ouvi o serviço de Nosso Senhor, e
lá vi que em uma grade como em uma cama, um velho homem de grande idade jazia,
e ele tinha uma coroa de ouro sobre sua cabeça. E quando se ergueu e sentou-se
lá, então vi que ele estava de todo ferido. E depois, quando o padre tinha
cantado missa, então vi que ele deu o corpo de Nosso Senhor para usar. E tão
logo o tinha recebido, tomou a coroa e a colocou sobre o altar e parece-me que
seja um grande significado, e devo de bom grado saber, se puder ser. Por causa
disto vos peço que me queirais dizer”. “Seguramente,” falou o bom homem, “pois
de bom grado”.

“É verdade, e bem o ouviram falar muitas pessoas, que José, o nobre
cavaleiro de Arimateia, foi enviado pelo mais alto Senhor para esta terra, que
ele deveria aumentar a Cristandade com ajuda de seu Salvador. E pois que aqui
veio, então padeceu de muito medo e maravilha, que lhe fizeram os inimigos
descrentes. E nessa terra não havia, ao tempo, senão pagãos, e havia nessa
terra um pagão, que se chamava Krúdel e era o mais impuro e mais maldoso do
mundo, sem piedade e sem humildade. E ele ouviu dizer que os cristãos
tinham vindo a sua terra e que tinham trazido consigo o Santo Vaso, que seria
tão maravilhoso que eles viviam de sua graça, sem refeições. E considerou esta
fala como mentira, e se lhe assegurou e se lhe mostrou que era seguro. E então
ele falou que descobriria dentro em pouco, e pegou o filho de José, Josephus, e
dois de seus sobrinhos, e bem elegeu cem que lá estavam para serem mestres
sobre a Cristandade. E pois que os tinha pego, e colocado na prisão, e pois que
eles tinham consigo o Santo Graal, não temeram que se não lhes trouxesse
alimento corporal. E o rei os manteve na prisão em maneira por quarenta dias,
que não lhes dava de comer, e tinha bem proibido que alguém fosse tão ousado
para se entregar a isto até o tempo. Então soaram as notícias pela terra, de
que Josephus com uma grande parte da Cristandade seria prisioneiro do rei
Krudel,\footnote{ A palavra, neste momento do texto, surge grafada sem o acento
agudo.}  tanto que o Rei Morderas,  que lá estava em Jerusalém na
cidade de Saras, que José tinha convertido com suas falas e com suas prédicas,
ouviu disso dizer. E ficou por causa disto muito entristecido, quando tinha
ganho novamente sua terra com o conselho de José, que Ptholomeus\footnote{ O
nome do rei, anteriormente referido como Thulomeus, aparece grafado, neste
momento, como Ptholomeus.}  tomou e tinha tomado, fez o conselho de
Josephus e seu cunhado Seraphin. E como o rei sabia que José\footnote{ Em
defluência do que antes se narra, deveria o texto referir-se a Josephus, mas o
nome apresentado, como se manteve na tradução, é José.}  estava
prisioneiro, falou ele que queria fazer sua força para resgatá-lo e reuniu seu
povo todo que conseguiu ter ao tempo e pôs-se ao mar, guarnecido com arma e
corcel, e tanto fez assim que veio à terra com navios. E pois que veio à terra
com todo o seu povo, ofereceu ao rei Krúdel, se não se restituísse José, ele
quereria apodrecer a ele e tomar-lhe toda a sua terra. E não lhe pediu muito, e
foi perante ele pelo campo, e agrediram-se. Assim aconteceu pela vontade de
Deus que os cristãos se sobrepuseram, e o rei Krúdel foi morto e seu povo; e o
rei Morderas, que era chamado Evallet antes de se tornar cristão, fez tanto na
luta que todo o seu povo aí teve grande maravilha. E pois que o tinham
desarmado, então acharam que ele tinha tantas feridas em seu corpo, que outro
homem disto teria morrido. Então lhe perguntaram como estava. E ele respondeu
que não sentia nem dor nem machucados, que tinha, e libertou José da prisão. E
pois que o viu, isto lhe fez grande alegria, quando lhe tinha amor de todo  seu
coração. E José pergunta-lhe quem o havia ali trazido. E ele respondeu que para
lá tinha vindo para resgatá-lo. E pela manhã aconteceu que os cristãos foram
perante o Santo Graal e falaram sua oração. Quando então José, que era mestre
deles, tinha se vestido para ir ao Santo Graal, e o rei Morderans,\footnote{
Aqui o nome se altera novamente no texto medieval.}  que por toda
via tinha desejado vê-lo claramente, se pudesse ter sido, foi muito mais perto
do que deveria. Então veio uma voz sobre ele e falou: ‘Rei, não ide mais
longe!’ E ele foi tão longe que nenhuma língua de homem ousou dizer, nem
coração terrestre imaginar, e estava ansioso por ver que foi mais longe. E de
pronto veio uma nuvem e lhe tomou a visão dos olhos e a força do corpo em tal
medida que não viu nem um brilho e pouco conseguia se ajustar. E pois que ele
viu que Nosso Senhor tinha feito como grande vingança por ter quebrado Seu
mandamento, falou tão alto que todo o povo ouviu: ‘Amado Senhor Jesus Cristo,
que nesta hora me indicastes que é tolice desrespeitar Teu mandamento, e me
basta o que me enviastes, e eu vou de bom grado padecer. Assim
deixa-me prometer do meu serviço que eu não morra até que o bom cavaleiro, que
deve vir de minha linhagem, que deve levar ao fim a aventura do Santo Graal,
venha me consolar, pois gostaria de abraçá-lo e beijá-lo’. 

Porque o rei tinha desejado este dom ao Nosso Senhor Deus, então lhe
respondeu a voz: ‘Rei, não te deve assustar, Nosso Senhor ouviu tua prece e tua
vontade deve ser realizada destas coisas, quando não deves ver um brilho, até
que o cavaleiro que desejas venha te ver. E nas horas em que ele vier, e deve
ver-te, então te deve ser restituída a luz aos olhos, tanto que o possas ver
claramente, e então devem convalescer as tuas feridas, que antes do tempo nunca
se curarão’. 

Assim falou a voz para o rei e mostrou a ele que deveria ver a vinda do
cavaleiro por quem tanto tinha desejado, e nos parece que eram verdadeiras
todas estas coisas. Quando se padeceu mais de cem anos que esta aventura
aconteceu, desde então ainda nunca viu um brilho nem conseguiu arranjar-se, e
suas feridas ainda estão incuradas. Então o cavaleiro está nesta terra, como se
nos diz, que deve levar esta aventura a um fim, e pelos sinais celebramos que
ele ainda deve ser e que ele deve voltar a ter poder de seus membros, e depois
não deve viver muito. 

Assim aconteceu ao rei Morderas, como vos contei. E sabei por certo que é o
mesmo que eu vi, e desde então viveu aqui dentro tão santa e tão
espiritualmente, que desde então não mais comeu nenhuma refeição terrestre e
nada a não ser o mesmo que o padre nos mostra no sacramento na missa. E isso
podeis bem ver hoje: tão logo o padre cantou missa, ele trouxe então ao rei o
sacramento e lhe deu a usar. Assim o rei desejou por longo tempo o advento do
cavaleiro, que também desejava muito ver, e lhe aconteceu como a Simão, o
Ancião, que longamente aguardou o advento de Nosso Senhor, até que foi trazido
ao Templo. E lá o recebeu o velho homem e o tomou entre suas mãos, feliz e bem
animado por sua promessa ter sido realizada. Quando o Espírito Santo lhe fez
saber que não deveria morrer antes de ter visto Jesus Cristo, o Filho de Deus,
o alto profeta e mestre altíssimo. Assim espera o rei o advento, quando virá
Galaat, o bom cavaleiro. Então vos contei a verdade, como me perguntastes,
assim como aconteceu. Então vos peço que me digais quem sois”.
E ele respondeu que era servo do rei Arthur e companheiro da Távola Redonda “e
se chamava Parsifal de Gales”. E pois que o bom homem entendeu seu nome, ele
lhe fez grande alegria, quando dele fartamente ouviu falar, e lhe pediu assim
permanecesse junto a ele, quando lhe queria fazer grande alegria e honra. E ele
responde que muito tinha a fazer, que não podia permanecer de nenhuma maneira,
e por causa disso precisava dali separar-se, e desejou suas armas. E pois que
estava pronto, montou e tomou licença e separou-se dali e cavalgou através da
floresta até depois das terças. 

Bem em torno do meio-dia, seu caminho o trouxe a um fundo. Lá o encontraram
vinte homens, bem preparados e bem armados, com uma padiola de corcel, onde
jazia um homem e estava há pouco morto. E eles perguntaram a Parsifal de onde
ele seria. E ele respondeu que era da corte do rei Arthur, e então gritaram
todos: “Pra cima dele!” Pois que Parsifal o viu, preparou-se para se defender
o melhor que podia e dirigiu-se contra aquele que lhe veio primeiro, e o
encontrou tão duramente que o conduziu por terra, e o cavalo caiu sobre seu
corpo. E pois que queria completar seu ataque, não o conseguiu, quando mais de
sete o golpearam sobre o escudo, e os outros lhe apunhalaram o corcel, e ele
caiu por terra, e quis levantar-se como aquele que era de grande poder. E os
outros se deixaram correr por sobre ele tão rancorosamente que nenhum auxílio o
pôde ajudar, e golpearam sobre o escudo e sobre o elmo e lhe deram tantos
golpes que não pôde permanecer de pé e veio com o joelho por terra. E golpearam
sobre ele tão maravilhosamente e tão longo o trouxeram, que lhe tinham quebrado
o elmo da cabeça e o teriam ferido de morte, tivesse feito o cavaleiro com a
arma vermelha, que a aventura lá trouxe. E pois que viu este cavaleiro a pé,
sob tantos de seus inimigos, que queriam golpeá-lo, então se virou para lá tão
rapidamente quanto pôde o corcel, e clamou para eles: “Deixai o cavaleiro!” E
correu sob eles com a lança e encontrou o primeiro tão duramente que o conduziu
por terra, e sacou sua espada, quando tinha quebrado sua lança. E então correu
por cima e por baixo e golpeou um e outro tão maravilhosamente que não
encontrou nenhum que não conduzisse por terra e fez tanto em pouco tempo com
os golpes que lhes dava e com a valentia de que era repleto, que nenhum foi tão
ousado que se atrevesse a resistir aos seus golpes e fugiram, um senhor após o
outro, e espalharam-se através da floresta, que era grande e larga, que ele não
pôde mais ver nenhum a não ser três. Um deles Parsifal cortou e feriu, e
Galaat aos outros dois. E pois que então viu que Parsifal estava salvo, foi-se
para a floresta, que viu toda muito densa,  como aquele que não queria de forma
alguma que se o seguisse. E pois que Parsifal viu que ele cavalgou por seu
caminho tão rapidamente, então chamou por ele o mais que pôde e falou: “Senhor,
esperai até que eu cavalgue um pouco até vós!” 

O bom cavaleiro não fez o mesmo que ouviu de Parsifal, e cavalgou para si
tão logo quanto o corcel podia, como aquele que não tinha vontade de retornar.
E Parsifal o seguiu a pé tão logo quanto pôde, quando não tinha nenhum
garanhão, quando aqueles lhe tinham apunhalado o seu. E então o encontrou um
servo sobre um cavalo forte e conduzia para a mão direita um corcel, que era
preto. E pois que Parsifal o viu, não soube o que deveria fazer, quando de bom
grado teria tido o corcel, que teria seguido o cavaleiro. E fez ao servo grande
promessa que ele com vontade gostaria de ter o corcel, quando com violência não
queria seguir caminho, a não ser que lhe obrigasse grande necessidade. E para
que não se o tomasse por não formoso, saudou o servo. E como o ouviu,
responde-lhe: “Deus deve vos pagar!” “Caro amigo,” falou Parsifal, “eu te
peço, por causa de todo serviço, por causa de toda paga e por que sou teu
cavaleiro na primeira cidade em que for por ti exortado, que me emprestas o
corcel por tanto tempo quanto eu persiga um que de mim cavalgou”.
“Senhor,” falou o servo, “não faço isso de maneira nenhuma, quando é de um tal
homem, que me toma o corpo se eu não o devolver”. “Amigo,” falou Parsifal, “eu
te peço por todo o serviço que o faças, quando seguramente nunca ganhei tanta
tristeza como então, se eu perder o cavaleiro, por causa de não ter um corcel”.
“Em verdade,” falou o servo, “eu não faço outra coisa, quando não o deveis ter
por minha vontade tanto quanto esteja em minha guarda, tomais então com
violência”. E pois que isto ouviu, ficou tão irado, que considerou sair de seus
sentidos de sofrimento. Quando por coisa nenhuma faria violência ao servo, e
pensou em seu senso, se perdesse o cavaleiro, que dele cavalgava, seria uma
coisa que nunca mais conseguiria ganhar alegria. Estas duas coisas lhe
trouxeram tão grande tristeza em seu coração que ele não pôde permanecer sobre
seus pés e caiu contra uma árvore, pálido e desmaiado, como se tivesse perdido
poder de todo o seu corpo. E tem tanto lamento que de bom grado teria querido
de pronto que tivesse morrido. E tirou seu elmo de sua cabeça e falou para o
servo: “Caro amigo, porque não queres me ajudar a sair deste sofrer, de que eu
não posso sair sem morrer, eu te peço que tomes minha espada e me mates de
pronto. Assim meu lamento é encurtado. Quando o bom cavaleiro ouvir dizer que
eu estou morto e que cavalguei à sua procura, nunca será falta de formosura,
quando ouvir dizer, que ele peça a Nosso Senhor Deus que Ele tenha graça sobre
minha alma”. 

“Em verdade,” responde o servo, “se Deus quiser não o faço, quando não o
mereceis por causa de mim”, e apressou-se por sua via tanto quanto jamais pôde.
E Parsifal permanece tão irado que considerou morrer. E pois que não mais o
viu nem a ninguém, então principiou a desempenhar tão grande lamento, e soou-se
pobre e funesto e diz: “Desfavorecido, então caíste em tudo o que procuras,
pois que ele cavalgou de ti, quando nunca mais virias ao sítio como bem para
achá-lo, como seria agora”. 

No ressoar, como Parsifal então desempenhou seu lamento em tal medida,
assim cobiçou e ouviu vir um tropel de cavalos. E abriu os olhos e viu vir um
cavaleiro armado, que cavalga bem logo pela floresta e conduzia o corcel na
mão, que o servo lhe havia negado. E Parsifal bem reconheceu o corcel, e não
pensou que ele lhe tivesse tomado com violência. E pois que não mais o viu,
principiou de novo a lamentar sua lamúria. E não demorou muito, depois, para
que visse o servo vir sobre seu cavalo e desempenhou grande lamento. E pois que
viu Parsifal, então falou: “Senhor, vedes algum cavaleiro por aqui cavalgar,
que aí conduzisse o corcel que me desejais?” “Sim,” falou Parsifal, “por que
causa perguntas?” “Por causa de que ele me tomou com violência e quase me matou
e me fez mal, quando meu senhor me mataria em que sítio me achasse”. “O que
queres”, falou Parsifal, “que eu faço por isso, quando não reconheço o que
fazer, quando estou a pé. E se eu tivesse cavalo, eu me confiaria a bem
trazê-lo de volta para cá”. “Senhor, montai em meu cavalo”, falou o servo, “e
se de novo o ganhares, que assim seja vosso”. “E teu cavalo”, falou Parsifal,
“como será novamente para ti, se eu ganhar o corcel?” “Senhor,” falou ele, “eu
vos devo seguir a pé, e se puderes ganhar o corcel, assim me dareis de novo o
meu cavalo, e o corcel será vosso”. E ele respondeu que não desejava outra
coisa. 

Então atou seu elmo sobre a sua cabeça e montou no cavalo do servo e tomou
seu escudo e cavalgou tão logo como podia ter a cavalo, atrás do cavaleiro,
tanto até que o achou em um gramado, que havia alguns na floresta. E pois que
viu o cavaleiro, apressou-se tanto como podia conduzir o cavalo, e chamou-o tão
longe como o viu: “Senhor cavaleiro, voltai e dai ao servo seu corcel, que
maldosamente lhe tomastes!” E pois que o ouviu chamar tão
duramente, então volveu e posicionou a lança. E Parsifal sacou a espada,
quando bem viu que ele tinha vindo para uma briga. E o cavaleiro, que dele
queria logo se despachar, veio a ele tão rapidamente quanto o corcel podia
seguir, e encontrou o cavalo no peito e lhe fincou a lança através dele, e o
cavalo caiu morto, assim que Parsifal caiu sobre seu pescoço. E pois que o
cavaleiro o viu que ele tinha caído, então se virou novamente e correu para
fora do gramado através da floresta, onde era mais densa. E pois que Parsifal
viu esta aventura, então ficou muito entristecido que não soube o que podia
dizer e chamou por aquele que fugia: “Funesto desalentador de coração, voltai e
lutai comigo, estou a pé e estais a cavalo!” E aquele não responde como o que
tivesse pouco medo e correu à floresta. E pois que Parsifal não mais o pôde
ver, então ficou tão entristecido que lançou seu escudo sobre a terra e sua
espada e tirou seu elmo de sua cabeça, e principiou de novo a desempenhar seu
lamento muito mais que antes e chora e gritou de alta voz e chamou a si de
funesto descortês, e o mais azarado cavaleiro de todos os cavaleiros.
E falou que ele bem visse que tinha caído de todos os seus desejos.

Nesse lamento e padecimento permaneceu Parsifal o dia inteiro, que ninguém
lhe veio que o consolasse. Quando veio a noite, estava tão desmaiado e tão
debilitado que lhe pareceu que lhe faltavam todos os seus membros. E então
começou a dormir e adormeceu e não despertou até a meia-noite. E pois que
acordou, então viu à sua frente uma mulher, que lhe perguntou muito de repente:
“Parsifal, o que fazes aqui?” E ele responde que não fazia bem nem mal, e se
tivesse como cavalgar, não permaneceria aí mais tempo. “Se me prometeres”,
falou ela, “que me farás a minha vontade, quando eu te exortar, eu vou de
pronto te dar um bom corcel, que te leve aonde quiseres”. E pois que o ouviu,
ficou tão contente, que ninguém poderia estar mais contente, como aquele que
não testava com que conversava e considerava que fosse uma mulher; era o
Inimigo, que de bom grado o teria ali trazido para que perdesse sua alma. E
pois que ele ouviu que ela lhe prometia o que ele acima de tudo desejava,
respondeu a ela que lhe estaria seguramente pronto para fazer o pudesse, e se
ela lhe desse o corcel, ele quereria fazer tudo que ela chamasse. “Assim jura
como um leal cavaleiro”, falou ela. “Sim, seguramente”, falou ele. “Então me
espera aqui,” falou ela, “devo de pronto retornar”. E então foi ela à floresta
e voltou de pronto para lá, e trouxe um corcel tão grande, que era preto e que
era maravilhoso de se ver.

E pois que Parsifal viu o corcel, então começou a muito aferrá-lo, e ainda
então foi tão ousado que o montou como aquele que não percebeu o engodo do
Inimigo, e tomou seu escudo e sua lança. E aquela que à sua frente estava
falou: “Quereis\footnote{ O texto original mais uma vez converte o pronome de
tratamento informal em pronome formal.}  seguir caminho? Então vos
deixai considerar que me deveis a paga”. E ele lhe responde que deveria
fazê-lo e apressou-se à floresta quanto sempre pôde. E a luz brilha muito
clara. E o corcel o conduziu tão logo e em curto tempo o trouxe à frente da
floresta e o tinha afastado mais que quatro grandes dias de lonjura, e ele
cavalgou por tanto tempo até que viu, à sua frente, um precipício e uma grande
água, que era forte e vigorosa, e o corcel virou-se para lá e queria lá se
lançar. E pois que Parsifal o viu tão forte, então temeu muito porque era
noite. E não viu nem ponte para seguir por cima, e suspendeu sua mão e fez um
sinal com a santa cruz à frente de sua testa.

E pois que o Inimigo sentiu-se carregado com a santa cruz, que lhe era por
demais pesada para levar, então se agitou e se apartou de Parsifal e seguiu
para a água gritando e berrando, e de pronto aconteceu que a água ficou acesa
como muitos fins com claras chamas, que considerou que estivesse queimando. 
Pois que Parsifal viu esta aventura, então percebeu de
pronto que era o Inimigo que lá o tinha levado para o enganar e fazer perder
corpo e alma. Então se abençoou e encomendou-se a Deus e pediu a Nosso Senhor
que não o deixasse cair em nenhuma tentação, para que com isto não perdesse a
companhia dos cavaleiros escolhidos. E ofertou as mãos contra o céu e agradeceu
ao Nosso Senhor de bom coração que o tivesse assim ajudado nesta necessidade.
Quando então o Inimigo queria levá-lo para a água e deixá-lo cair lá dentro,
assim pôde ele bem se ter afogado e teria perdido seu corpo e sua alma. E ele
se foi para fora da água, quando, porém, tinha medo do Inimigo. E ajoelhou-se
perante o oriente e fez sua prece assim como conseguiu e muito desejou o dia,
para saber em qual terra estava, quando bem lhe pareceu que o Inimigo o tinha
carregado para longe do convento onde ele viu o rei ferido. 

Assim estava Parsifal em sua prece e devoção e reza por tanto tempo até que
o sol tivesse feito seu retorno ao céu e que brilhasse sobre todo o mundo e
tivesse derretido o orvalho. E então ele viu à sua volta e viu que estava em
uma montanha grande e maravilhosamente erma, e estava às voltas e por isso e
tão fechado com as notícias que não viu nenhuma terra que não fosse fora
daquela distância. Então ele percebeu que tinha sido conduzido a uma ilha, que
não sabia onde e de bom grado o teria sabido, pois não conseguiu saber como
poderia ficar sabendo, quando não via nem fortaleza nem castelo onde pudesse
estar alguém como lhe pareceu, e ainda então, assim não estava ele sozinho, ele
via ao seu redor animais selvagens, ursos, leopardos e dragões. E pois que se
viu em tal sítio, então não estava bem confortável, quando temia os animais
maravilhosos, que não o deixavam em comodidade, como lhe pareceu, e o matariam,
se não conseguisse se defender. E ainda então Aquele que protegeu Jonas no
ventre do peixe e protegeu Daniel na cova dos leões, vai ajudá-lo e vai ser seu
escudo da paz, assim não tenha ele medo, e ele mais se abandonou ao Seu auxílio
e à Sua consolação que à sua espada. Então ele bem viu que por causa de
valentia ou nenhuma cavalaria terrestre ele conseguiria vir a um fim ou
escapar, se então Nosso Senhor não o ajudasse. Então ele viu no meio da ilha um
muito grande penedo, e lhe pareceu que se estivesse lá em cima, que não teria
nenhum medo de quaisquer animais selvagens, e por isto se voltou para lá. Então
ele viu um dragão, que mantinha com os dentes um leão junto ao pescoço e se
sentava no mais alto da montanha. E atrás do dragão corria um leão gritando e
desempenhava tão grande lamúria que Parsifal pensou que ele fazia o lamento
pelo pequeno leão que o dragão levava embora. E pois que Parsifal o viu, então
correu tão logo quanto pôde montanha acima. Então o leão, que era bem mais leve
que ele, o tinha logo ultrapassado e tinha começado uma luta contra o dragão, antes
que ele ali pudesse vir. E quando veio ao topo da montanha, então viu os dois
animais lutarem, e pensou que queria ajudar o leão, quando seria um animal mais
nobre que o dragão. E sacou sua espada e colocou seu escudo à frente de seu
rosto, para que as chamas não o ferissem. E foi ao dragão e lhe deu um grande
golpe entre suas duas orelhas. E soltou fogo e chamas, tanto que lhe queimou o
escudo e sua coifa. E ainda então lhe teria feito muito mais, quando ele era
leve e ágil, assim que as chamas não o encontraram de certo, e assim por esta
causa muito menos. Pois que o viu, então temeu que o fogo fosse crescer com
veneno, e ainda então assim correu contra o dragão e lhe deu um grande golpe,
que pôde encontrá-lo, e o encontrou no mesmo sítio em que antes o tinha
encontrado. E a espada estava tão leve e boa e o impeliu de pronto através da
cabeça levemente, e tão logo ele lhe tinha golpeado através da pele, então não
estavam os ossos duros e o matou logo no sítio. 

E pois que o leão viu que ele estava desatado do dragão com a ajuda do
cavaleiro, então não fez igual como se não tivesse nenhuma vontade de duelar
com ele e veio à frente dele e beijou-lhe a cabeça e lhe fez a maior alegria
que pôde, que Parsifal bem percebeu que não tinha nenhuma vontade de lhe fazer
mal. E colocou sua espada de novo na bainha e baixou seu escudo, que lá estava
bem chamuscado, e tirou seu elmo de sua cabeça e recebeu o ar, quando o dragão
muito o tinha aquecido. E o leão foi de toda sorte saudando e lhe fez grande
alegria. E pois que Parsifal o viu, então lhe acariciou a cabeça e o pescoço e
ombros e falou que Nosso Senhor lhe tinha enviado o animal para lhe fazer
companhia e considerou isso uma bela aventura. E o leão lhe fez tão grande
companhia e alegria como nenhum outro animal poderia fazer a um homem, e o dia
todo permaneceu ele junto até depois do horário das vésperas; e então desceu o
penhasco e conduziu o pequeno leão a seu ninho. E Parsifal viu-se sozinho na
montanha, sem companhia, não se deveria perguntar a ele se não tinha nenhum
conforto, quando outra coisa não via ao redor de si senão o vaidoso mar. E
teria ainda tido mais desconforto, tivesse feito o consolo de seu Altíssimo que
ele tinha, quando era um homem que lá acreditava em seu mais alto Salvador. E
ainda então estava contra o costume de sua terra, quando ao tempo eram todas as
pessoas incrédulas no reino de Gales e eram bem maldosas. Quando, se o pai
achasse seu filho deitado na cama, tomava-o pela cabeça ou pelos ombros ou
pelos braços e o jogava para fora da cama e o golpeava de pronto. Quando lhe
pareceria ser uma vergonha se seu pai o achasse morto. Quando aconteceu que o
pai matasse o filho, e o filho, o pai, e que todos morressem nas armas, assim
falavam todos na terra que eles eram nobres. 

Todo o dia permaneceu Parsifal sobre o penhasco e olhou sobre o mar, para
ver se nenhum navio ele podia ver vir, quando lhe aconteceu todo o dia que não
conseguia ver muito em cima ou abaixo, que nada viu. E quando o viu, então
recebeu de novo um coração e consolou-se sobre Nosso Senhor Deus e Lhe pediu
que o protegesse, que não caísse em nenhuma em tentação de engodo do diabo ou
com maus pensamentos. Pois como o pai é responsável por proteger o filho, assim
Ele precisa protegê-lo e salvá-lo. E segurou suas mãos contra os céus e falou:
“Querido Senhor Deus, desde que me fizestes em tão alto estado na ordem da
cavalaria e me elegestes para tanto, e eu não teria sido digno, Senhor, através
da Tua Graça não permitais que eu saia deste serviço. E sejas para mim como o
bom lutador que lá bem protege seu senhor contra aquele que o cobiça. Querido
Senhor, assim me ajuda, que eu proteja a minha alma, que é tua propriedade e
tua herança, contra aquele que a deseja por injustiça. Querido, doce Pai, pois
que falas no Evangelho de ti mesmo: ‘Eu sou o bom pastor que lá protege a
ovelha’, e assim não faz o mercador que lá deixa a ovelha sem proteção até que
o lobo conte e a coma e a dilacere. Senhor, seja minha escolta e meu protetor,
assim como eu esteja junto a tuas ovelhas e que eu não seja uma das ovelhas
loucas e impuras, que se apartam das outras e fogem para o ermo. Senhor, tem de
mim graça e não me deixa apodrecer no ermo, e conduz-me na tua direção, que é
para a Santa Igreja e para a Santa Fé, em que as boas ovelhas estão e os homens
fiéis e também onde estão os bons cristãos, assim que o Inimigo, que não deseja
senão a minha alma, não me ache sem proteção”. 

E pois que Parsifal o tinha falado, então viu o leão vir até ele, que tinha
lutado com o dragão. E não fez igual ao que ele lhe queria fazer e foi contra
ele perante ele e lhe fez grande alegria. E pois que Parsifal o viu, então o
chamou. E veio até ele, como se fosse o mais manso dos animais do mundo, e se
colocou perto dele, sobre seus ombros, por tanto tempo até que a noite ficasse
sombria e preta. E Parsifal adormeceu de pronto junto a seu leão, e não lhe
apeteceu qualquer comida, quando bem pensava em outras coisas.

Pois que Parsifal estava adormecido, então lhe aconteceu uma maravilhosa
aventura. Quando lhe pareceu em seu sono que à frente dele viriam duas
mulheres, uma era velha, a outra nem tão velha e muito bela; e não estavam a pé
e estavam sentadas sobre dois animais maravilhosos. Uma estava sentada sobre um
leão e outra sobre um dragão. E ele contemplou as donzelas com maravilha por
causa de assim poderem dominar os dois animais. E a mais jovem veio à frente e
falou: “Parsifal, meu senhor vos faz saudação e vos pede que deveis vos
preparar  o melhor que puderes, quando precisais amanhã lutar contra o lutador
que é para mais se temer no mundo. E se fores vencido, não deveis vos deixar
com um de vossos membros e vos sucederá de ser difamado por todos os dias em
que viveres”. E pois que ele ouviu esta fala, então falou: “Senhora, quem é
vosso senhor?” “Seguramente,” falou ela, “é o homem mais rico de todo o mundo.
Então aguardai que sejais seguro e ousado para a luta, que tenhais sua honra”.
E então desapareceu como santa, que Parsifal não soube de onde ela tinha
vindo.

A outra donzela veio para ele sobre o dragão e falou: “Parsifal, eu me
queixo contra vós, quando muito errastes perante mim e os meus, e eu não o
mereci”. Pois que ouviu esta fala, então respondeu a ela e falou: “Seguramente,
perante vós e perante nenhuma mulher do mundo não considero que tenha errado, e
vos peço que me digais com o que agi errado perante vós; e se eu tiver poder
para melhorar, devo vos melhorar segundo vossa vontade”. “Quero bem vos dizer
em que medida errastes contra mim”, falou ela. “Por longo tempo criei em meu
castelo um dragão, que me servia muito mais do que considerais. E o animal voou
ontem infelizmente até a montanha e achou um leão, que conduziu consigo
penhasco acima, e viestes para lá correndo com vossa espada e o matastes, e ele
não vos tinha feito [mal]. Então me dizei por que causa o matastes, tivesse ele feito
algo errado pelo que assim devêsseis levá-lo à morte. O leão era vosso ou
estava sob vosso cuidado, que fôsseis responsável por lutar à frente dele?
São-vos assim permitidos os animais, que deveis matá-los sem direito? Sois
assim tão grande cavaleiro?” Pois que Parsifal entendeu esta
fala que a donzela lhe tinha falado, respondeu a ela: “Senhora, ele não errou
comigo, nem o leão era meu, nem os animais do céu não me são permitidos, quando
por meio de ser o leão de mais nobre que o dragão. E pois que vi que o leão
menos mal fazia que o dragão, então corri por cima dele e o matei, e por meio
disso não errei perante vós como dizeis”. Pois que a donzela ouviu esta fala,
então falou: “Parsifal, não quereis me fazer mais nada?” Então ele falou:
“Senhora, o que vos faça por causa disso?'' “Eu quero”, falou ela, “que para a
melhora do dragão vos torneis meu homem”. E ele respondeu que não o faria.
Então ela falou: “Mas o fostes; antes de receberdes o feudo de vosso senhor,
éreis meu. E por causa de que éreis antes meu que de um outro, assim não quero
renunciar a vós e vos quero assegurar: em qualquer lugar em que o achar sem
proteção, que vos tomarei como aquele que antes era meu”. 

Depois dessa fala apartou-se dele a mulher, e Parsifal adormeceu, quando
estava muito cansado do sonho, e dormiu a noite inteira, que nunca acordou. E
pela manhã, pois que o dia brilhava mais alto, e o sol estava alto e brilhava
sobre sua cabeça, então abriu os olhos, e viu que era dia. Então se endireitou
e permaneceu assim sentado, e ergueu sua mão e abençoou-se e pediu a Nosso
Senhor Deus que lhe enviasse conselho que lhe fosse útil para a alma. Quando
sobre o corpo não atentava tanto quanto antes tinha feito, por meio disto não
pensou que nunca viria do penhasco em cujo cume estava. E viu à sua volta e não
viu o leão que antes lhe tinha feito companhia, nem o dragão que ele tinha
matado e muito lhe maravilhava aonde teriam ido. 

Nisto que Parsifal assim pensou, então viu no longe mar um navio flutuando.
Tinha estendido a vela e vinha direito para o sítio onde estava Parsifal, que
lá esperava que Deus lhe enviasse aventura que lhe aprouvesse e o ajudasse. E o
navio velejou bem rápido, quando tinha o vento da água, e veio certeiramente
perante ele e aportou junto ao penhasco. E pois que Parsifal, que estava sobre
o penhasco, o viu, então ficou consigo muito contente, quando considerou que lá
estavam muitas pessoas. E por meio disto se levantou e tomou suas armas. E pois
que estava armado,  então escalou para fora do penhasco como aquele que queria
saber quem eram as pessoas que estavam no navio. E pois que se aproximou, então
viu que o navio estava por fora e por dentro guarnecido com samítico branco,
que nenhuma outra coisa lá dentro não brilhava que não o branco. E pois que
veio a bordo do navio, então viu um homem, vestido com um saiote à igualdade de
um padre e tinha uma coroa sobre sua cabeça de samítico branco, bem ampla como
dois dedos, e na coroa estavam escritos os Santos Nomes de Nosso Senhor. E pois
que Parsifal o viu, então o maravilhou muito, e aproximou-se junto a ele e
falou: “Senhor, sede bem-vindo!” “Deus vos melhore, caro amigo”, falou o bom
homem. “E quem sois vós, Senhor?” “Eu sou”, falou ele, “da corte do Rei
Arthur”. “Que aventura vos trouxe para cá?”, falou ele. “Senhor,” falou
Parsifal, “eu desconheço em que medida ou como eu viesse para aqui”. “O que
desejais?”, falou o bom homem. “Senhor,” falou ele, “quer Nosso Senhor Deus,
então bem quero que eu consiga sair e que eu possa ir com meus companheiros da
Távola Redonda à demanda do Santo Graal, quando por outra coisa não me apartei
de meus senhores da casa do rei”. “Valha-me Deus,” falou o bom homem, “se vos
aprouver, assim podeis logo sair daí, e Ele deve logo ter-vos ajudado, quando o
quer. Se Ele vos tivesse por Seu servo, e se Ele visse que estaríeis melhor em
outro lugar que aqui, assim sabei que Ele de pronto vos ajudaria a sair. Pois
Ele vos fez para cá para tentação, para saber e para reconhecer se sois Seu
reto servo e Seu fiel cavaleiro, como a ordem da cavalaria contém. Pois no
momento escalastes tal altura, assim vosso coração não se deve baixar por causa
de nenhum medo nem de nenhuma necessidade terrena. Quando o coração do
cavaleiro deve ser fechado e assim duro perante o inimigo de seu senhor, que
nenhuma coisa o possa assustar. E se for trazido a temor, então não é um
verdadeiro cavaleiro e um verdadeiro lutador que antes se deixa matar na luta
antes de deixar algum para seu senhor”.  

E perguntou-lhe Parsifal de onde ele era e de qual terra. Ele responde que
era de terra estrangeira. “Que aventura aqui vos trouxe”, falou ele, “em tal
sítio estranho e tão ermo como me parece?” “Em verdade”, falou o bom homem,
“eu vim para ver-vos e para consolar e para que por isto me digais de vosso
ser. Se não for nenhuma coisa de que devais aconselhar, eu conseguiria vos
aconselhar tão bem quanto nenhum homem que aí vive”. “Dizeis-me maravilha”,
falou Parsifal, “que aqui viestes par me aconselhar, quando não consigo pensar
como pode ser isto. Quando neste penhasco em que eu estou,
ninguém mais me sabe senão Deus e eu. E se me tivésseis seguido, assim não
creio que soubésseis, quando nunca me vistes em meus dias de vida, e por causa
disso me tem maravilha o que dizeis”. “Hei,” falou o bom homem, “Parsifal, eu
vos conheço muito melhor do que considerais, quando há muito tempo não fizeste
nada que eu não soubesse tão bem como vós”. E pois que Parsifal ouviu que o
bom homem o chamava, então ficou bem assustado, e arrependeu-se porque lhe
tinha dito muito, e respondeu: “Hei, caro Senhor, perdoai-me que vos tenha
falado, quando não considerei que me conhecêsseis. Assim bem vejo que me
reconheceis melhor do que eu a vós, e por causa disso me tenho por tolo e a
vós, por um homem sábio”.

Então se recostou Parsifal sobre o bordo do navio junto ao bom homem, e
falaram de algumas coisas um com o outro, e o achou tão sábio de todas as
coisas que muito o maravilhou quanto poderia ser, e lhe aprouve assim muito a
sua companhia. E se por toda a via estivesse junto a ele, não agradaria comida
ou bebida, tão boa era sua fala e assim alegre. E como falaram um com
o outro por um bom tempo, então falou Parsifal: “Caro Senhor, fazei-me sábio
de uma coisa que hoje me veio em meu sono, pois me parece tão maravilhosa que
eu não posso descansar sem saber a verdade dela”. “Assim me dizei”, falou o bom
homem, “eu vou vos significar de pronto, que bem deveis saber o que deve ser”.
“Eu devo vos dizer”, falou Parsifal. “Aconteceu-me esta noite em meu sono que
duas mulheres vieram à minha frente e uma estava sentada sobre um leão e a
outra sobre um dragão: e a que se sentava sobre o leão era uma donzela, e a que
se sentava sobre o dragão era uma velha, e a mais jovem falou comigo em
primeiro”. E então contou para ele toda a fala que ela lhe tinha dito, tão bem
que nenhuma palavra lhe foi esquecida. E pois que tinha contado seu sonho,
então pediu a ele por Deus que lhe dissesse o significado. E ele lhe responde
que o queria fazer de bom grado, e principiou a falar: “Hei, Parsifal,
Parsifal, as duas mulheres que dizeis que tão estranhamente cavalgavam, que
uma se sentava sobre um leão e a outra sobre um dragão, seu significado é muito
maravilhoso, e eu devo vos significar ou resignar.

A que estava sentada sobre o leão significa a nova fé. Que ela se sentasse
sobre um leão, era sobre Jesus Cristo, em que tomava pé e fundamento, quando
por meio Dele foi elevada a fé e preenchida com Seu advento e em sua proteção
toda a Cristandade e uma verdadeira luz para todos os que abrirem seu coração
para a sua confissão. E a mulher sentava-se sobre o leão,  era sobre Jesus
Cristo. A mulher é a verdade e a fé e batismo e esperança, e a mulher lá é um
castelo e um ferro sobre os quais Jesus Cristo falou que deveria erguer a Santa
Igreja, de que Ele falou: ‘Sobre esta pedra devo erguer o Meu templo e lhe dar
plenos poderes’. E junto à mulher que estava sentada sobre o leão, deve-se
junto a ela entender a nova fé, que Nosso Senhor conservou forte e em poder
como o pai faz a seu filho. E por meio disto que ela lhe pareceu mais jovem que
a outra, não é maravilha, quando da idade nem da igualdade não era a outra,
quando essa mulher nasceu na Paixão de Nosso Senhor e na Ressurreição, e a
outra tinha muito longamente existido e reinado. E aquela veio falar para
ti\footnote{ Mais uma vez, o texto comuta o pronome de tratamento formal para a
modalidade informal.}  como a seu filho, quando todos os bons
cristãos são seus filhos. E ela bem sabia que era tua mãe, quando ela te teve
tão grande medo que veio te alertar daquilo que estava para te acontecer. Ela
veio dizer-te por causa de teu senhor, que foi por causa de Jesus Cristo, que
precisas lutar. E sobre a verdade, que eu te devo fazer, ela não teria tido
amor por ti, não teria vindo a ti para falar, quando não tivesse boato vencido.
Ela veio por causa disto, para te dizer que serias traído, quando devesses
lutar e contra quem, contra o mais temível lutador do mundo. Ele é aquele por
meio de quem Enoch e Elias, que assim eram nobres pessoas, foram tomados do
mundo e foram conduzidos ao céu e não voltaram para cá antes do Dia do Juízo
Final, pois vieram lutar contra aquele que é muito temível. É o lutador e o
Inimigo, que por todas as vias assim se esforça muito e trabalha para que traga
o homem em perdição mortal, que os\footnote{ O texto original apresenta esta
alteração de singular para plural, que mantivemos na tradução.} 
conduza à danação eterna, é o lutador contra quem precisas lutar. E se fores
vencido, como te diz a mulher, então nunca deves sair por causa de perder, pois
deves de toda forma te envergonhar. E bem podes ver em ti mesmo se é verdade;
quando se isto acontecer que o Inimigo te atacar que te conduza em perda de
corpo e alma, e por eles deve te escoltar para a casa da tristeza, que é no
inferno, onde tu então deves padecer em vergonha, lamento e martírio por tanto
tempo como durar o poder de Nosso Senhor, que deve durar por todas as vias.

Então te disse o que a donzela seja, que tu vistes no sonho sobre o leão, e
por meio do que eu te mostrei, assim podes bem saber o significado da outra”. 
“Senhor,” falou Parsifal, “de uma me dissestes, que bem sei
o significado dela. Então me dizei da outra, que cavalgava o dragão, quando
dela não consigo reconhecer significado, vós me dizeis então”. “Devo te
dizer?”, falou o bom homem. “Então me ouve: a mulher que viste cavalgar o
dragão significa a Velha Aliança, e o dragão que ela trouxe é a Escritura que
maldosamente é entendida e maldosamente é conservada, são os hipócritas, os
hereges e os malfeitores e os pecadores mortais, e é o próprio Inimigo e é
serpente e o dragão, que por meio de sua petulância foi lançado para fora do
Paraíso. E esse é o dragão que lá falou para Adão e para sua mulher: ‘Se
comerdes este fruto, sereis iguais a Deus’, e esta fala o trouxe a um desejo
injusto, quando de pronto pensaram em ser muito mais altos do que eram. Pois
acreditaram no conselho do Inimigo e desdenharam d’Aquele de que foram lançados
para fora do Paraíso; e de erro foram lançados em erro, e o erro tinha parte em
todos os seus descendentes e devem todos os dias penitenciar-se. E pois que a
mulher veio perante ti, ela se queixou de seu dragão, que tinhas matado. E
sabias de que dragão ela se queixava? Ela não se queixava do dragão que ontem
matastes. Era aquele que ela cavalgava, o Inimigo. Sabes por que causa ela fez
esta súplica? Quando viestes para fora deste penhasco, na hora em que fizestes
a cruz sobre ti, não podia mais se padecer de qualquer maneira e teve assim
grande medo que considerava estar morto e fugiu tão logo quanto aquele que não
queria te fazer companhia. E assim o golpeastes e o expulsastes e tomastes seu
poder e sua força, de sua luta e de seus membros, quando ele considerava bem te
ter ganho. Disto assim conta a ira que ela tinha sobre ti. E pois que tinhas
respondido a ela o melhor que conseguistes depois de te perguntar, então ela
tencionou que te tornasses seu homem, então respondestes que não o querias
fazer. E ela respondeu que tu por algum tempo o terias sido, antes que tivesses
feudo de seu senhor. E após estas coisas, muito pensastes e serias obrigado a
bem saber; quando sem erro, antes que tu recebestes o batismo e antes que te
tornastes cristão, estavas no cuidado do Diabo. Quando tão logo recebestes o
selo de Nosso Senhor Jesus, que é o batismo e o santo crisma, então tinhas
renegado o Inimigo e estarias fora de seu cuidado, e tinhas então feito
companhia ao teu Salvador. Então te contei o significado de uma mulher e também
de outra. Então vou novamente por minha via, quando muito tenho a fazer, e tu
deves permanecer aqui, e deves refletir a luta que tens a fazer. E se fores
vencido, assim deve te acontecer como te é prometido”. 

“Caro Senhor,” falou ele, “por que quereis seguir caminho tão logo?
Seguramente, vossa fala e vossa companhia me aprazem tão bem que nunca
desejaria apartar-me de vós, e por Deus, se puder ser, assim permanecei comigo
ainda hoje; quando tanto quanto me dissestes, creio que tanto melhor deve ser
meu dia que eu vivo”. “Eu preciso seguir caminho,” falou o bom homem, “quando
muitas pessoas me esperam, e vós\footnote{ Novamente se altera o pronome de
tratamento para a modalidade formal.}  deveis aqui permanecer e
esperai que não estejais desaconselhado contra aquele que deveis combater,
quando se ele vos achar sem alerta, logo vos pode ser desgosto”.

 E pois que o falou, então se apartou de lá. E o vento batia na vela e
conduziu o navio tão rapidamente como se pode ver e tinha em curto tempo tanto
corrido, que Parsifal não pôde mais vê-lo. E pois que à vez o tinha perdido,
então subiu novamente o penhasco assim armado como estava e achou o leão que o
dia inteiro lhe tinha feito companhia. E principiou a acariciá-lo, por meio do
que lhe fez grande alegria. E lá bem permaneceu até depois do meio-dia, então
viu longe, no mar, vir um navio, tão logo quanto todos os ventos o varriam. E à
frente veio uma gralha, e fez o mar se acalmar, e de todos os fins
empilhavam-se saltos. E pois que o viu, então muito o maravilhou o que pudesse
ser, quando a gralha lhe tomou a visão do navio, que lá estava bem coberto com
pano preto, eu não sei se seria seda ou linho,  e veio tão perto dele,
que viu claramente que era um navio. E pois que veio bem próximo, então desceu
para saber o que seria. Quando lhe pareceu bem que seria o bom homem, com o
qual tinha conversado no dia. E lhe aconteceu porém que em toda a montanha
nenhum animal foi tão ousado que se atrevesse a pular sobre ele, eu sei se
seria pela graça de Nosso Senhor ou por outras coisas. E ele desce a montanha e
vem para o navio o mais depressa que pôde. E pois que entrou, então viu uma
donzela sentada lá dentro, que era sem medida bela e estava à vez muito
ricamente, como podia ser um mulher. E tão logo ela o viu vir, então se
levantou perante ele e falou sem saudar: “Parsifal, o que fazeis aqui, e o que
vos trouxe a esta montanha, que lá é tão estranho que nunca sairás, se não for
por aventura, nem ganharás o que comer, e deveis morrer de fome e de sede, pelo
que não acharás ninguém que vos reclame”. “Donzela,” falou ele, “se aqui eu
morresse de fome, então não seria um verdadeiro servo, quando ninguém serve tão
grande senhor quanto eu faço,  assim o sirvo lealmente e de bom coração, que
não faço coisa nenhuma, que não me seja de valia. E ele mesmo fala que sua
porta não está fechada para ninguém, que lá venha e deseje ser lá dentro
recebido, e quem deseja algo, o tem. E se alguém deseja, ele não responde e se
deixa simplesmente achar”. E pois que ela ouviu que ele falava
parábola do Evangelho, então não respondeu à fala e principiou outra e falou:
“Parsifal, sabes\footnote{ Neste momento do texto, altera-se o pronome de
tratamento para o informal.}  de onde eu venho?” “Como, donzela,”
falou ele, “quem vos deu a reconhecer meu nome?” “Eu bem o sei”, falou ela, “e
posso conhecer melhor do que considerais”. “De onde vindes?”, falou ele. “Em
verdade,” falou ela, “eu venho da floresta deserta, onde vi as maravilhosas
aventuras do mundo do bom cavaleiro”. “Hei, donzela, dizei-me sobre a verdade a
que estais obrigada por aquele que mais amais sobre o reino da terra, o que
seja”. “Não vos digo de nenhuma forma o que disto sei, se não me prometeres
pela ordem da cavalaria que quereis fazer minha vontade a qualquer tempo em que
eu te exortar.” E ele lhe respondeu que o queria fazer, se alguma vez pudesse.
“Basta que me dissestes isto, então vou vos dizer”, falou ela.

“É verdade, não faz muito que estive na floresta deserta bem no meio do
mesmo fim onde corre a grande água que se chama Marthoße. Lá vi que o bom
cavaleiro vinha e caçava dois cavaleiros e queria matá-los; eles caíram na água
por meio do medo da morte, e lhe aconteceu tão bem que vieram por cima. E neste
tempo infeliz, e seu cavalo afogou-se, e ele mesmo estaria morto, se não
tivesse de pronto saído. E por meio disto, que ele retornou, então convalesceu.
Então te contei a verdade do cavaleiro depois que me perguntastes. Então quero
que me digas como viveste desde que vieste a esta ilha, pois que deverias estar
como perdido, se daí não saísses. Quando bem vês que aqui ninguém conta de quem
tenhas auxílio, quando precisas sair, ou deves morrer aqui, assim faz minha
vontade para que saias daqui, quando de outra forma não podes sair, senão por
mim. Por esta causa deves fazer tanto pela minha vontade, que eu te ajude a
sair, se és sábio. Quando não conheço maldade maior que a que faz aquele que
bem pode ajudar e não o faz”.

“Donzela,” falou Parsifal, “se eu pensasse que fosse vontade de Nosso
Senhor Deus que eu saísse, então de bom grado quereria sair, se pudesse. E de
outro modo não quero estar fora daqui. Quando não há nenhuma coisa no mundo que
eu bem queira fazer, se pensar que seria contra Sua vontade, quando então teria
enterrado vilmente a cavalaria, se estivesse contra Ele”. “Deixai tudo
estar,” falou ela, “e dizei-me se hoje comestes”. “Seguramente,” falou ele,
“não mordi nenhuma comida terrena hoje, quando agora me veio um bom homem
consolar, que me falou muito boa fala, que me saciou, assim que não me apraz
comer nem beber tanto quanto pense nele”. “Sabei”, falou ela, ``quem ele é? É um
feiticeiro e um mentiroso e faz por toda via de uma fala cem, e nunca fala
verdade, se pode. E se bem o credes, então estais enganado, quando nunca sairás
deste penhasco, e deveis aqui morrer de fome, e os animais selvagens vos devem
arruinar, e podeis ver uma parábola disto: estivestes aqui dois dias e duas
noites, e tanto quanto o dia de hoje passou, que aquele, de quem falais, nunca
vos trouxe de comer e vos deixou  e deixa assim que dele não ganheis nenhum
auxílio. É vossa grande pena e descortesia se aqui morrerdes, pois sois um
jovem e um bom cavaleiro, que ainda podeis ajudar a mim e a outros, se daqui
vos retirares”.

E pois que Parsifal ouviu que ela se lhe encomendava, então falou:
“Donzela, quem sois que me ajudais daqui, se quiserdes?” “Eu sou”, falou ela,
“uma donzela que lá está deserdada, que lá seria a mais rica mulher do mundo,
se não estivesse expulsa de minha herança”. “Donzela,” falou ele, “quem vos
deserdou, quando muito mais me inspirais piedade que antes?” “Eu devo vos
dizer”, falou ela. “É verdade que havia um homem rico, que me tomou em sua casa
para o servir, e o homem era o rei mais rico que se sabia. E eu era tão bela e
tão clara como  ninguém era, ele queria ter maravilha de minha beleza, quando
eu era bela sobre todas as coisas. E na beleza me alcei e falei uma fala que
não lhe caiu bem. E tão logo ele ouviu, ficou à vez irado comigo, que não me
quis padecer em sua companhia, e me lançou fora, pobre, e me deserdou. E desde
então não quis ter nunca piedade de mim, nem de ninguém que estivesse a meu
lado. Assim me expulsa o rico homem e aos meus e me lançou em estrago. E então
principiei uma guerra contra ele, e bem me aconteceu desde então, quando bem a
ganhei. E lhe tomei a maior parte de seus homens, que o deixaram e vieram a
mim, por meio da grande companhia que lhes conservo. Quando de mim não desejam,
eu lhes dou e muito mais.

Assim estou em guerra dia e noite contra aquele que me arruinou. Assim reuni
cavaleiros e escudeiros e servos e toda sorte de gente, e vos digo que não sei
nenhum cavaleiro no mundo, nem nenhum nobre, que eu não peça para os meus, para
que fique ao meu lado. E por meio disto, de que te soube um valente cavaleiro,
por causa disto vim para cá para que me ajudais;\footnote{ Neste segmento do
texto original, na mesma fala se alternam os pronomes de tratamento,
\textit{tu} e \textit{vós}.} e bem sois obrigado a o fazer,
quando sois um companheiro da Távola Redonda, quando ninguém de lá é
companheiro, que seja obrigado a sair-se de uma donzela arruinada, quando ela
lhe pede por auxílio. E sabeis que isto é verdade. Quando lá estáveis sentado,
e o rei lá vos fez, então jurastes o primeiro juramento, que fizestes, que
nunca negaríeis a nenhuma donzela auxílio, que vos pedisse.” E ele respondeu
que tinha feito o juramento sem falha, por causa disso queria de bom grado
ajudá-la, porque ela lhe pedia. E ela lhe agradece muito.

Assim longamente falaram um com o outro que eram as sextas horas e caiu-se
perto das nonas horas. E o sol brilha quente e aquecido. Então falou a donzela
a Parsifal: ``Há a mais bela tenda neste navio, que jamais vistes. Se vos
agrada, devo retirá-la de lá e fazê-la abrir para que o sol não vos cause dor”.
E ele diz ``eu gostaria muito''. E ela foi ao navio e fez dois servos abrirem a
tenda. E pois que tinham aberto o melhor que podiam, então falou a donzela para
Parsifal: “Vinde dentro descansar e sentai-vos por tanto tempo até que venha a
noite, quando o sol muito vos aquece”. E Parsifal foi à tenda e dormiu de
pronto, e fez-se desarmar de seu elmo e de sua coifa e de sua espada. E pois
que estava desarmado, então ela o deixou dormir. E pois que tinha dormido um
bom bocado, então acordou e desejou comer, e ela chamou que se pusesse a mesa e
se o fez. E ele viu que se o servia de tantos pratos que muito o admirou, e ele
e a donzela comiam um com o outro. E pois que ele desejou beber, então se lhe
deu de pronto. E pois que tinha bebido, então julgou que era o vinho mais forte
que jamais bebeu e o melhor, como lhe pareceu, e o maravilhou de onde pudesse
vir. Quando ao tempo não havia na Grã-Bretanha nenhum vinho, a não ser em
sítios muito ricos, e bebiam comumente cerveja e outras bebidas, que eles
faziam. E ele bebeu tanto que foi por isto aquecido mais do que
deveria. E ele contemplou a donzela, que lá era bela fora de medida, como lhe
pareceu, que nunca tinha visto igual a ela de beleza. E ela tanto o agradou e o
deleitou tanto pela indulgência que nela via, e pela doce fala que ela lhe
tinha dito, que estava mais aceso do que deveria.

Pois conversaram os dois de muitas coisas, e falou a ela por causa de seu
amor, que ela era sua e ele, dela. E ela o recusou como pôde, pelo que ele cada
vez mais se inflamava por ela e se agradava dela. E não fez mais que pedir. E
pois que ela viu que ele estava aquecido, então responde e falou: “Parsifal,
sabei que de nenhuma maneira faço o que vos apraz, a não ser que me prometais
que devereis ser meu no futuro e no meu auxílio contra todos, e devereis então
fazer o que eu te chamar”. E ele responde a ela que de bom grado o queria
fazer. “Então prometeis a mim como um verdadeiro cavaleiro?” “Sim”, falou ele.
“Bem me basta com isto, e devo fazer tudo que vos apraz. E sabei seguramente
que nunca me desejastes tanto quanto eu vos desejei, quando sois um cavaleiro
do mundo, por quem mais intimei”. Então pediu a um de seus servos que fizesse
uma cama, a mais rica e mais bela que ele pudesse, no meio da tenda. E ele
respondeu que queria fazer seu pedido. E fizeram de pronto uma cama. E pois que
estava feita, a donzela tirou os sapatos e se deitou e Parsifal junto dela. E
pois que ele jazia, então devia cobrir-se. Então lhe aconteceu uma aventura,
que ele viu sua espada jazer sobre a terra. E moveu sua mão para lá, para
erguê-la, pelo que queria alinhá-la em sua cama. E viu no cabo uma cruz
vermelha, que ali estava gravada. E tão logo quanto o viu, então pensou em si
mesmo e fez o sinal da Santa Cruz em sua testa. E de pronto viu a tenda cair, e
uma neblina, uma fumaça estava em todo o seu redor, tão grande que ele nada
via, e ele cheirou tão grande mau cheiro em todos os fins que lhe pareceu que
estava no inferno. Então chamou com voz alta e falou: “Querido, doce Pai,
Senhor Jesus Cristo, não me deixa aqui perecer, vem por tua misericórdia em meu
auxílio, quando de outro modo estou perdido!” E quando abriu seus olhos, então
não viu a tenda, sob a qual estava antes deitado. E viu na água e viu o navio
em tal medida como antes tinha visto. E a donzela falou: “Parsifal, vós me
traístes”, e de pronto, ela se alçou ao mar. E Parsifal viu um temporal tão
grande, que a seguia, que lhe pareceu que toda a madeira no mundo estivesse
pega. E o navio do fogo seguiu tão flamejante que nenhum sibilar do vento pôde
tão logo navegar como lhe pareceu. 

Pois que Parsifal viu a aventura, então ficou muito entristecido que lhe
pareceu que deveria morrer. E ele viu o navio por tanto tempo quanto pôde ver.
E pois que perdeu a vista daquilo, então falou: “Ah pobre, eis que estou
morto!”, e sacou a espada da bainha e golpeou tão duramente e se encontrou no
pé esquerdo, que o sangue saiu em todos os fins. E pois que o fez, então falou:
“Caro senhor, isto é melhora do que fiz contra ti”. Então ele se contemplou,
que estava nu, com suas roupas íntimas, e viu suas roupas de um lado e suas
armas de outro. Então se repreendeu e falou: “Ah, eu pobre desgraçado, fui tão
mau e impuro, que tão logo fui trazido ao sítio para perder aquilo que ninguém
pode restituir, que é a virgindade, que ninguém pode novamente ganhar, quando
uma vez a perdeu”.  Então pôs sua espada na bainha, e o lamentou muito
mais pelo que considerou que Deus estivesse muito irado com ele, que qualquer
outra coisa, nem que estava ferido. E vestiu-se e preparou-se o melhor que
pôde. E dirigiu-se para cima do penedo e pediu a Nosso Senhor Deus que lhe
enviasse conselho, que pudesse achar misericórdia e graça, quando se sabia
culpado e pecador, que acreditou que nunca mais conseguisse se reconciliar, se
não fosse acontecer por meio de misericórdia. 

Assim esteve Parsifal a noite toda junto à água como aquele que não queria
ir para cima ou para baixo, por meio das feridas que tinha, e pediu a Nosso
Senhor, que lhe enviasse tal conselho, que lhe fosse útil para a alma, quando
não desejava nenhuma outra coisa. E falou: “Caro Senhor Deus, nunca creio sair
daqui, morto nem vivo, que não seja com Tua vontade”.  Assim permaneceu
Parsifal o dia inteiro e verteu seu sangue muito das feridas. E pois que viu a
noite brilhar, assim triste e assim escura no mundo, então deitou sua cabeça
sobre a coifa e fez uma cruz em sua testa e pediu a Nosso Senhor Deus por meio
de Sua graça que Ele o protegesse em tal medida que o Demônio não tivesse
nenhum poder sobre ele. E pois que tinha levado sua oração ao fim, então se
endireitou e ficou sobre seus pés e cortou um bom grado de sua camisa e parou
suas feridas com aquilo,  pelo que elas não sangraram muito. E alçou-se em sua
oração, quando ele conseguiu tanto, e rezou em tal medida, até que o dia veio.
E pois que Nosso Senhor Deus queria que o dia se levantasse, e o sol também
estava alto, então olhou Parsifal ao redor de si e viu em um fim o mar e em
outro, o rochedo. E pois que pensou no Inimigo, que no outro tinha considerado
como uma donzela, quando bem pensou que seria o Inimigo e principiou a
desempenhar a mais maravilhosa e maior lamentação e falou que seguramente
estaria morto, se o Espírito Santo não o tivesse consolado.

Nisto que Parsifal assim se queixava e assim falava, então viu mais longe
no mar contra o oriente e veio vir o navio que ele tinha visto em outro tempo.
Era o navio que estava coberto com samítico branco, onde o bom homem estava
dentro, que lá estava vestido à maneira de um padre. Pois que o viu, então
estava bem consolado por meio da boa fala que ele tinha dele ouvido e grande
sabedoria que nele tinha achado. E pois que o navio tinha vindo à terra e o bom
homem se tinha alinhado à bordo, então se endireitou tanto quanto pôde. E o bom
homem saiu do navio e veio a ele e sentou-se sobre o penhasco e falou:
“Parsifal, como fizestes desde que estive junto de ti?” “Senhor,” falou
Parsifal, “mal, quando uma donzela me caiu perto, tinha trazido um pecado
mortal”, e contou a ele como lhe tinha acontecido. E o bom homem lhe pergunta
se ele a conhecia. “Não,” falou ele, “quando bem sei que o Inimigo a mandou
para mim para me arruinar e me enganar. E teria sido enganado, não tivesse
feito o sinal da Santa Cruz, pois de mim aconteceu que voltei ao meu reto juízo
e à minha reta maneira. Quando tão logo eu fiz uma cruz à minha frente, de
pronto a donzela se apartou de mim, que nunca mais a vi. Então vos peço por
Deus que digais o que devo fazer, quando nunca precisei de conselhos tão bem
quanto agora”. “Hei,” falou o bom homem, “todo tempo serás
incompreendido. Não conheces a donzela, que tão perto te trouxe o pecado
mortal, quando te resgatou o sinal da Santa Cruz?” “Seguramente,” falou ele,
“eu não a conheço bem, eu vos peço por Deus que me digais quem ela seja e de
que terra e quem seja o homem rico que a arruinou, contra quem ela me pediu que
a ajudasse?” “Isto devo bem vos informar”, falou o bom homem, “que tu
deves bem em breve reconhecer quem ela seja”.

“Então escuta: a donzela, com quem conversaste é o Inimigo e o mestre do
Inferno, que tem poder sobre todos os inimigos. E é verdade que antes daqui ela
estava no Céu e na companhia dos anjos e era tão bela e tão luzidia e tão
clara. E pela beleza ele\footnote{ Neste momento da fala, o bom homem passa a
referir-se à donzela como “ele”, vale dizer, o Inimigo.}  se
excedeu e quis se fazer igual ao Maior e falou: “Eu devo subir tão alto e devo
ser igual ao mais alto Senhor”. E tão logo ele o falou, então não quis Nosso
Senhor que sua casa fosse enganada pela envenenada petulância e o precipitou do
elevado assento, onde ele se tinha assentado, e o fez cair na casa da
escuridão, que se chama o Inferno. E pois que ele se viu tão rebaixado do
elevado assento e da grande altura, onde costumava estar, e foi empurrado na
eterna escuridão, que pensou que deveria guerrear com todos que pudesse, e com
aquele que para lá o tinha trazido; quando não conseguiu ver que isto
facilmente aconteceria. E fez-se para a mulher de Adão e fez tanto que a
enganou e a levou a pecado mortal, pelo qual ele foi lançado e caiu da grande
alegria do Céu. Foi com a cobiça que ela fez com sua vontade infiel, que ela
trouxe o fruto da árvore mortal; quando lhes era proibido da boca do
Altíssimo. E pois que o tinha comido, então o deu a comer também para Adão de
tal maneira que todos os seus herdeiros tiveram de se penitenciar por isto. O
Inimigo, que a aconselhou a isso, o dragão, que ontem vistes a donzela
cavalgar. E era a donzela que ontem te veio ver, que te disse que guerreava
noite e dia; então te disse verdade. E bem sabes tu mesmo e sabe que nenhuma
hora se passa que ele não combata o cavaleiro Jesus Cristo e contra as boas
pessoas e servos, em que Nosso senhor está dentro albergado. 

E pois que ela tinha feito frente a ti um pacto com suas falsas falas e seu
engodo, então fez abrir sua tenda e falou: ‘Parsifal, vem cá em baixo e
senta-te aqui por tanto tempo até que venha a noite e saia do sol, quando me
parece que o sol vos causa dor!’ Esta fala, que ela te falou, não é sem grande
significado, quando ela muito mais coisas quer dizer do que entendes. A tenda,
que lá estava aberta na medida e na igualdade do mundo, significa seguramente o
mundo, quando bem nunca está sem pecado. E por causa de estar por toda via
cheia de pecado, então ela não queria que estivesses abrigado fora da tenda, e
te fez prepará-la. E pois que chamou por ti, ela falou: ‘Parsifal, vem cá
sentar e descansar por tanto tempo até que a noite venha!’ Nisto que ela
falou que tu abaixo te sentasses e descansasses, com isso ela quis dizer que
ficasses indolente e preenchesses teu corpo cheio de iguarias terrenas. Ela não
te aconselhou que trabalhasses neste mundo e semeasses a semente no mesmo dia,
que as pessoas mais nobres devem colher no dia do Juízo Final. E te pediu que
descansasses por tanto tempo até que a noite viesse; isto é tanto quanto dizer
até que a morte te tome, que seguramente é chamada noite para todos os tempos,
quando acha o homem em pecado mortal. E ela te bradou que não queria que o sol
te aquecesse. Não é maravilha que tivesse preocupação, pois por sol entendemos
Jesus Cristo; o verdadeiro brilho aquece o pecador do fogo de Jesus Cristo, e
depois pouco podem prejudicá-lo os frios do Inimigo, e se tem o sol em seu
coração, é Jesus Cristo.  

Então te disse tanto da mulher, que estás obrigado a bem saber quem ela
seja. Ela veio te ver, mais por teu mal que por teu bem.” “Senhor,” falou
Parsifal, “falastes tanto da mulher, que bem sei que é o guerreiro contra quem
eu devo duelar”. Então falou o bom homem: “Seguramente, tens razão, então
contempla como duelaste”. “Senhor, mal,” falou Parsifal, “quando me parece que
seria vencido, não tivesse feito a graça do Espírito Santo, que não me deixou
arruinar”. “Como te aconteceu,” falou o bom homem, “então futuramente te
protejas mais; quando se lá caíres por outra vez, não te deves achar tão logo
em altura como então fizeste”. Longamente falou o bom homem contra Parsifal e
lhe mostrou muito bem o que fazer e falou que seu Deus não deveria esquecê-lo e
deveria em breve enviar-lhe auxílio. Então lhe pergunta como lhe tinha
acontecido com suas feridas. “Em verdade,” falou ele, “desde que viestes
perante mim, então nunca recebi dor nem mal, igualmente a como se não tivesse
nenhuma ferida, e desde que me falais não recebo. Também me veio de vossa fala
e de vosso rosto uma como que grande doçura em meus membros, que não creio que
sejais um homem terreno, senão espiritual. E sei bem por verdadeiro, se por
toda via aqui comigo permaneceis, que não posso ter sede ou fome, e se eu
ousasse dizê-lo, eu diria que seríeis o pão que vem do Céu, de que ninguém
frui, que vive eternamente”. E tão logo ele disse isto, então desapareceu o bom
homem em tal medida que Parsifal não soube aonde teria ido. Então falou uma
voz: “Parsifal, venceste e conservaste o campo, entra no navio e navega para
lá aonde a aventura te mostrar, e não teme a qual sítio vens e nenhuma coisa
que vês, quando Deus deve te guiar. E digo-te, tanto bem te aconteceu que em
breve deves ver teus companheiros Bohort e Galaat, que são aqueles que mais
queres ver”. E pois que Parsifal ouviu esta fala, então
ganhou tão grande alegria que ninguém poderia ver, e ergueu suas mãos ao Céu e
agradece a Nosso Senhor Deus pelo que lhe aconteceu. E então Parsifal tomou
suas armas e armou-se de pronto e entrou no navio e conduziu ao mar tão
rapidamente que era maravilha ver. – E aqui se calam as notícias de falar dele
voltam para Lancelot, que tinha permanecido na clausura junto ao bom homem,
que bem lhe tinha dito das três falas que a voz lhe dita dito à frente da
capela. 

\chapter{A penitência de Lancelot}

\textsc{Aqui falam} as notícias que Lancelot permaneceu três dias junto ao bom homem. 
Nisto que o conservou em sua companhia, então lhe pregou tudo e bem o
exortou a fazer e falou: “Seguramente, Lancelot, por nada viestes a esta
demanda, quereis então vos proteger do pecado mortal, e tirai vosso coração das
coisas terrenas e pensamentos e o gozo do mundo. Quando bem deveis
saber que nessa demanda a cavalaria não vos pode ajudar, o Espírito Santo vos
faz o caminho senão para todas as aventuras que trazeis ao fim. Quando bem
sabeis que essa demanda foi assumida por alguma sabida aventura do Santo Graal,
que Nosso Senhor Deus prometeu ao cavaleiro que de bem e de cavalaria deve
superar todos os que foram antes dele e todos que vierem depois. O cavaleiro
vistes no dia de Pentecostes sentar-se no Assento Perigoso da Távola Redonda, e
sobre o mesmo assento ninguém estava sentado, que não devesse morrer. A
aventura vistes acontecer um pouco fartamente. O cavaleiro é o leão que lá deve
saber em sua vida todas as coisas terrenas da cavalaria, e quando tiver tanto
feito tanto que não mais deve ser terreno, senão espiritual, e deve ele deixar
o ser terreno e deve vir para a cavalaria do Céu. 

Assim falou Merlin do cavaleiro que por vez vistes, como aquele que quase
muito sabia das coisas para dizer, que estavam para acontecer, e ainda é que
assim o cavaleiro tem mais nobreza e ousadia que outro tem. Sabei por
verdadeiro, se ele caísse em pecado mortal, de que Deus o protege por Sua
misericórdia, ele não seria nesta demanda nada além de um outro cavaleiro
simplório. Do serviço pelo qual Lhe viestes, não pertence às coisas mundanas,
quando nós mesmos vemos que quem quiser aí vir por completo, precisa
primeiramente se lavar e se purificar de todas as impurezas terrenas, para que
o Inimigo não tenha parte com ele em coisa nenhuma. Em tal medida, quando ele
se tiver purificado do Inimigo e Renegado sobre tudo e se tiver separado de
todos os pecados mortais, então ele pode seguramente vir a essa demanda e a
este alto serviço. E se ele for de fé tão débil que considere fazer mais por
meio de sua cavalaria que da graça de Nosso Senhor, sabei que nunca saíra dela
senão com desonra.” 

Assim falou o nobre homem a Lancelot e o conservou em tal medida três dias
junto dele. Então Lancelot se considerou ainda mais com sorte, que Deus o tinha
para lá mostrado ao nobre homem, que tão bem o tinha instruído, quando bem lhe
parecia que muito melhor deveriam ser os dias que ele viveu. E pois que veio o
quarto dia, então pediu o bom homem a seu irmão que lhe enviasse arma e corcel
para um cavaleiro que tinha estado junto a ele. E ele preencheu seguramente seu
desejo sobre tudo. E no quinto dia, pois que Lancelot tinha ouvido missa, e
pois que se tinha armado, então montou em seu corcel e apartou-se do bom homem
chorando muito; e lhe pediu muito por Deus que perante Ele rezasse para ser que
Nosso Senhor Deus assim não o esquecesse, que ele não caísse na primeira
desgraça. Assim se apartou Lancelot do bom homem. 

E pois que estava separado, então cavalga através da floresta até as primas
horas. Então o encontra um servo que lhe pergunta: “Senhor cavaleiro, de onde
sois?” Respondeu ele: “Eu sou da corte do Rei Arthur”. “Então me dizei”, falou
o servo, “como vos chamais?” Respondeu ele: “Eu me chamo Lancelot do Lago”.
Falou o servo: “Eu vos aconselho a não procurar, quando sois um dos mais
desgraçados cavaleiros do mundo”. “Caro amigo,” falou Lancelot, “de onde o
sabeis?” “Bem o sei,” falou o servo, “não sois aquele que viu o Santo Graal
vir perante si e fazer sinal brilhante e nunca ponderou nunca, não mais do que
se fôsseis um homem incrédulo”. “Em verdade,” falou Lancelot, “eu nunca o vi e
nunca me movi, isto me é mais lamentável que agradável”. “Isto não é
maravilha,” falou o servo, “se vos é lamentável, quando seguramente provastes
que não éreis um homem nobre nem um verdadeiro cavalheiro, senão falso, desleal
e incrédulo. E se não quiserdes fazer-lhe honra por vós mesmos, não deveis vos
maravilhar se disto lhe acontecer infâmia nessa demanda, pois viestes a ela com
os nobres homens. Em verdade, perverso cavaleiro, podeis bem ter grande
arrependimento, costumava-se vos ter por melhor cavaleiro do mundo, então se
vos considera o mais perverso cavaleiro e mais desleal do mundo”. 

Pois que Lancelot ouviu esta fala, então não soube nada que dizer e ficou
assustado do que o servo lhe moveu. No entanto ele falou: “Caro amigo, tu me
falas tão mal quanto te apraz e quanto queres, e eu ouço, quando nenhum
cavaleiro deve irar-se de tais coisas indiferentes que um servo lhe faz, quando
ele não lhe fala tanto mal”. “Para ouvir viestes, Senhor,” falou o
servo, “quando de vós nunca falo melhor. Costumáveis ser uma flor da cavalaria
terrena! Desgraçado, soturno, bem sois capturado pela vontade que não vos tem
amor e pouco atenta sobre vós. Ela vos preparou que perdestes a coroa celeste e
a companhia dos anjos e toda a honra do mundo e viestes para saber de toda a
vergonha”. E Lancelot não se atreveu a responder como aquele que estava
triste, e tinha querida que estivesse morto. E o servo foi sempre em frente
ralhando e ralhando e blasfemando e falando a maior vergonha que podia; e ele
ouviu tudo isto, o que estava assustado que não se atreveu a olhar. E pois que
o servo ficou cansado de falar o que ele quis, e pois que viu ele não queria
responder, então cavalgou sua via. E Lancelot não procurou por ele e cavalgou
assim gritando e pedindo a Nosso Senhor que o guiasse pelo caminho que fosse
útil para a alma. Quando bem que tinha errado tanto neste mundo e tinha se
esquecido tanto de seu Deus. E seja então que a graça de Nosso Senhor é tão
grande, assim nunca deve achar graça, e foi levado a que a primeira vida não o
agradasse bem, esta vida simples o agradava ainda mais. 

E pois que tinha cavalgado até o meio-dia, então viu diante de si, fora do
caminho, uma casa. Ele cavalga para lá, quando bem sabia que era uma clausura.
E pois que veio junto, então viu uma pequena capela e uma pequena casa. E à
frente da porta sentava-se um homem velho, vestido com vestes brancas em
igualdade a um homem espiritual, e tinha maravilhoso lamento e falou: “Querido
Senhor Deus, por que o permitiste, porém ele vos serviu por tanto tempo e muito
se martirizou em vosso serviço!” E pois que Lancelot viu o nobre homem
chorando tanto, então o saudou e falou: “Senhor, Deus vos saúde!” “Assim Ele o
faça a vós também, Senhor cavaleiro”, falou o bom homem, “quando se ele não me
proteger, temo que o Inimigo me possa facilmente enlouquecer. Deus vos desate
dos pecados em que estais, quando seguramente sois um homem mais nobre que
qualquer outro cavaleiro que eu sei”. Pois que Lancelot entendeu o que o bom
homem dizia, então desmontou e pensou que queria tanto dali se apartar, que
deveria aconselhar-se com o bom homem que bem o reconheceu, como lhe pareceu,
na fala que lhe tinha falado. Então amarrou seu cavalo a uma árvore e foi em
frente ao altar onde jazia morto, como lhe pareceu, um homem grisalho, vestido com
uma pequena camisa branca. E junto a ele jazia uma camisa de lã, dura, afiada e
espinhosa. 

E Pois que Lancelot o viu, então o maravilhou quase muito a morte do bom
homem, e pensou que queria tanto dali se apartar, e sentou-se e perguntou como
tinha morrido. O bom homem lhe respondeu: “Senhor cavaleiro, eu não o sei,
quando bem vejo que ele não está morto com Deus nem com a morte correta, quando
em tais roupas como vedes, ninguém morre, que tenha se doado na vida. Por causa
disto bem sei que o Inimigo fez este dano, se ele estiver morto; é grande pena,
como me parece, quando bem ele esteve mais de trinta anos em serviço de Deus”.
“Em minha verdade”, falou Lancelot, “este dano me parece ser assim tão grande
que ele perdeu seu serviço, se foi seduzido em tal idade pelo Inimigo”. Com
isto foi o bom homem à capela e tomou um livro e uma estola e saiu de novo e
principiou a castigar o Inimigo [..] à sua frente em uma tão horrenda figura
que o coração de nenhuma pessoa no mundo não se assustaria. “Tu me martirizas
demais,” falou o Inimigo, “então me tens, o que queres?” “Eu quero”, falou
ele, “que me digas como meu companheiro morreu e se ele se perdeu ou se
conservou”. E o Inimigo lhe respondeu que ele estaria conservado. “Como pode
isto ser?”, falou o bom homem. “Parece-me que me mentes, quando assim a nossa
ordem não manda e proíbe abertamente que ninguém vista nenhum pano de linho,
quando quem o faz quebra a ordem, e quem morre em sacrilégio não é bom, como me
parece”. “Eu te devo mostrar como isto aconteceu. Então entende.

Sabes bem que ele foi um nobre e de grande linhagem e tem ainda sobrinhos e
sobrinhas nesta terra. Aconteceu então um dia que o Conde de Val começou guerra
contra um seu sobrinho, que era chamado Agravant. Pois que a luta estava
iniciada, Agravans,\footnote{ O texto original apresenta, seguidas, as duas
grafias.}  que se viu ao chão, não soube o que deveria fazer, e
veio aconselhar-se com seu tio, que ali jaz até agora, e lhe pediu tão
amistosamente que ele saiu de sua clausura e foi com ele e auxiliou manejar
guerra contra o conde. Sucedeu lá que ele o seguiu e o ajudou a portar as
armas. E pois que vieram um com o outro junto a seus amigos, então fez tão bem
de toda a cavalaria que o conde foi pego no terceiro dia do tempo em que vieram
um junto ao outro. E depois fizeram uma paz o conde e Agravant, e o conde lhe
deu boa segurança de que ele nunca mais quereria guerrear-lhes. E pois que a
luta estava conciliada, então voltou o nobre à sua clausura e começou de novo a
cair em seu serviço, que ele por alguns dias tinha exercitado. E pois que o
conde ficou ciente de que ele tinha sido derrubado por causa da vontade do bom
homem, então pediu a dois seus sobrinhos que o vingassem. Então falaram: ‘Nós o
fazemos de muito bom grado’. Então foram de pronto para lá, e
pois que vieram à capela, então viram que o nobre estava em calma pois cantava
missa. Então não se atreveram a incomodá-lo no ser, quando falaram que queriam
esperar até que ele saísse, e armaram uma tenda lá em frente. E pois que tinha
pronunciado sua oração e tinha saído da capela, então eles lhe falaram que
deveria estar morto, e sacaram sua espada sobre ele. E pois que consideraram
golpear-lhe a cabeça, então provou Aquele, a quem por toda a via tinha servido,
um tão grande sinal aparente, que não puderam sobre ele golpear golpe de que
pudessem causar-lhe dor. E não vestia nada mais que uma saia, e
golpearam por sobre ele como sobre uma bigorna de aço tanto até que se
quebraram suas espadas, e ficaram eles próprios muito cansados e quebrados de
grandes golpes que lhe tinham dado. Então nem lhe tinham causado dor, que ele
nem deixou gota de sangue.

Pois que viram isso, então ficaram perdidos de sentido e sem a coragem que
tinham. Então tomaram madeira e fizeram um fogo e falaram que queriam
queimá-lo, quando não poderia fazer contra o fogo. Então o deixaram desnudo e
lhe tomaram a camisa de lã que aqui vedes. E pois que ele se viu assim nu,
então se envergonhou de si mesmo e pediu-lhes que lhes dessem alguma roupa, que
não ficasse assim reprovável. E eles estavam maus e malvados e falaram que ele
nunca vestiria linho ou lã, pois ele deveria morrer de pronto sem roupa. Pois
que ouviu isso, suspirou e falou: ‘Como, considerais que eu devo morrer por
causa do fogo que aqui fizestes?’ Eles falaram que não deveria dele ter senão
a morte. ‘Seguramente’, falou ele, ‘é vontade de Deus que eu morra, é mais da
graça de Deus do que do fogo, quando o fogo nunca deve ter tanto poder sobre
mim que possa chamuscar um cabelo de meu corpo. Nem seria qualquer camisa no
mundo tão pequena, se eu a vestisse e fosse ao fogo à vez, que jamais fosse
chamuscada ou irritada por causa de um cabelo’. 

Pois que isto ouviram, então o tomaram por notícia o que ele disse. ‘Qual a
causa’, falou um deles, ‘que bem devo ver de pronto se ele tiver dito verdade?’
E retirou sua camisa de sua saia e fez vestir ao bom homem, e de pronto o
lançaram ao fogo que tinham feito tão grande que durou da manhã até a noite bem
tarde. E pois que o fogo estava extinto, então acharam o bom homem morto;
todavia tinha sua carne tão íntegra e tão pura como podeis ver, e também a
camisa, que ele vestia, que não estava em nada irritada além do que podeis ver.
E pois que o viram, então ficaram muito assustados e o tomaram lá e o levaram à
capela, onde agora jaz, e puseram sua camisa de lã junto dele e fizeram sua
via. E com este sinal, que Ele, a quem ele tinha servido tanto, através dele
tinha feito, então deves bem seguramente saber que ele está conservado e não
perdido. E com isso quero daqui me apartar, quando bem informei o que te
deixava em preocupação”.  E tão logo tinha dito isso, então
seguiu sua via e derrubou a árvore diante de si e faz o maior estrondo do
mundo, e pareceu como se todos os inimigos do inferno retumbassem pela
floresta. E pois que o bom homem ouviu esta aventura, então estava muito mais
contente que antes, e tira a estola e coloca o livro aqui e vai ao cadáver. E
começou a olhá-lo e falou para Lancelot: “Em minha verdade, Deus já mostrou
sinais neste homem, que eu considerava que estivesse morto em pecado mortal,
quando ele fez, por graça de Deus, como vós mesmos podeis ter ouvido”.
“Senhor,” falou Lancelot, “quem é aquele que vos falou tanto, que eu não pude
ver? Pois sua voz ouço bem, que era tão assustadora e obscena  como ninguém é,
ele pode bem assustar”.

“Senhor,” falou o bom homem, “bem é para assustar, quando não há nenhuma
coisa que seja para temer tanto como essa, quando ele é aquele que dá ao homem
tal conselho, que perde corpo e alma”. Então soube bem com quem ele
tinha falado, e então lhe pediu o ermitão que o ajudasse ainda hoje a sepultar
o corpo e que lá permanecesse até que o tivesse cometesse. Então respondeu
Lancelot que de bom grado o queria fazer, e estava muito feliz que Deus o
tinha trazido ao sítio, que ele podia servir a um santo como ele era. E tira
suas armas e as conduz à capela e vai ao seu cavalo e tirou-lhe sua sela e o
arreio e volta para o santo homem e lhe faz companhia. E pois que estavam
sentados juntos, então principiou a pregar-lhe: “Senhor cavaleiro, sois
Lancelot do Lago?” E ele lhe respondeu “sim”. “Então o que cavalgais armado a
buscar, como estais/”, falou o bom homem. “Senhor,” falou Lancelot, “eu
cavalgo em busca do Santo Graal com meus companheiros”. “Seguramente”, falou o
bom homem, “buscar bem podeis, mas falhastes a achar, pois se o Santo Graal
viesse perante vós, não creio que o pudésseis ver, não mais do que um cego faz
quando se lhe segura uma espada diante dos olhos. E ainda pois assim muitas
pessoas estiveram na escuridão do pecado e da impureza, e que Deus trouxe de
volta à verdadeira luz tão logo viu que seu coração estava convertido. Nosso
Senhor Deus não deixa de receber o pecador, quando vê que a Ele retornou com o
coração e o senso, então conta de pronto a consolá-lo e a ajudá-lo. E se
prepara e purifica seu castelo, como um pecador está obrigado a fazer. Ele desce
e nele repousa, e nunca mais o pecador precisa se preocupar em separar-se
d’Ele, a não ser que O expulse de sua casa. Sabei, Ele não deixa alguém senão
quando é contra Ele, assim se afasta dele que não pode mais ficar lá, pois ali
está contido aquele que por toda a via luta contra Ele.

Lancelot, esta parábola te disse através da obra que por tanto tempo tens
desempenhado, desde que caístes em pecado mortal, assim isto é tanto quanto
dizer, desde que recebestes a ordem da cavalaria. Se antes te tornasses
cavaleiro, então terias albergado em ti todas as boas virtudes tão naturalmente
que eu não saberia de nenhum jovem que te pudesse igualar. Se de primeiro
tivesses albergado em ti pureza tão completamente que não as teria quebrado com
vontade nem com obras. Principalmente com vontade não amainastes. Quando
fartamente aconteceu isto, que pensaste em coisas pecaminosas, do que foi
perturbada a pureza, e te foi admoestado, e dizes que nunca quisestes cair
nesta impureza, e que estarias numa sólida fé, que nada seria melhor para a
cavalaria que a pureza e evitar a falta de castidade, e conservar seu corpo
casto. E de acordo com essas virtudes que lá são elevadas, terias humildade e
paciência e terias por toda via uma cabeça abnegada. Tu não farias como faz o
hipócrita, que lá quando entrou no templo: ‘Amado Senhor Deus, eu Te louvo e Te
agradeço porque não sou nem tão mau nem desleal como meu vizinho’. Assim não
serias, se igualasses aquele que não se atreveu, por grande arrependimento, a
ver o quadro, que Deus não se irasse com ele porque era assim pecador, e ficou
afastado do altar e deu uma pancada à frente de seu coração e falou: ‘Amado
Senhor, tem graça de mim, pobre pecador’. Desta maneira deve-se conter aquele
que quer preencher a obra da humildade. Assim fizeste quando eras um jovem
nobre, quando tinhas amor ao teu Criador antes de todas as coisas e falavas que
não se estaria obrigado a temer nenhuma coisa terrena. Pois se deveria temer
aquele que tem poder para perder corpo e alma e precipitar ao inferno.

Conforme essas duas virtudes, que eu te signifiquei, terias albergado em ti
paciência. A paciência iguala uma esmeralda, que lá está sempre verde, quando a
paciência nunca terá tão grande tentação, que ela possa ser superada. Quando
ela verdeja por toda a via e todo o tempo está em uma força, não se pode nunca
vir contra ela, que ela não tire a vitória e a honra, quando ninguém pode
superar seu inimigo tão bem quanto com paciência. Quando todo o pecado que
pensaste de cor, então sabe que bem a terias em ti albergado, bem naturalmente.

Depois terias albergado em ti uma outra virtude tão naturalmente como se te
viesse por natureza, que era a justiça. É uma virtude tão grande que através
dela todas as coisas são conservadas em seu reto sítio, e não se transforma por
nenhuma vez e dá a todo homem o que ele merece e lhe mostra seu direito. A
justiça não dá a ninguém por causa da amizade e não toma ninguém por ódio e não
poupa nem amigo nem parente dá por toda a via segundo o reto juízo, em tal
medida que nunca se transforma de seu reto caminho por causa de qualquer
aventura que lá possa acontecer.

Conforme essa virtude, estava albergada em ti uma tão grande suavidade que
era maravilha. Quando terias tido toda a riqueza do mundo em tua mão, a terias
dado à frente pela vontade de teu Criador. Então foi em ti aceso o fogo do
Espírito Santo, e queimava em ti, e estavas com vontade de conservar, com corpo
e alma, o que lhe tinha emprestado as virtudes, assim viestes para a ordem da
cavalaria. Pois que lá o Inimigo fez perder-se, com o primeiro, a humanidade e
a trouxe à perdição, e te viu assim aconselhado e recoberto em tudo, então ele
teve preocupação que não te pudesse prender em nenhuma medida. Quando ele viu
claramente que ele bem conservou seu trabalho, se ele queria te tirar de
qualquer coisa que fosses interiormente. Ele viu que foste ordenado para o
serviço de Nosso Senhor e estavas em tal serviço que nunca poderias ser
rebaixado a servir ao Inimigo, e temeu muito te falar, por causa e temer que
nisso perdesse seu trabalho. E então pensou ele em algumas maneiras como ele
poderia te enganar, tanto tempo até que refletiu que ele poderia, com mulheres,
trazer-te ao pecado mortal muito mais brevemente que com outras coisas. E falou
que o primeiro pai enganado pela vontade da mulher, e Salomão, o mais sábio de
todos os homens terrenos, e Sansão, que lá tinha mais força que nenhum homem, e
Absalão, que lá era filho de Davi, o mais belo homem do mundo. E porque ele
tanto fez que todos dentre eles fossem enganados, e envergonhados, ‘assim não
me pareceu que a criança pudesse ser algo contra com direito’. E então veio ele
à rainha, que nunca tinha se confessado corretamente, e foi aconselhada e a
trouxe, porque estavas em sua corte. E pois que te tornastes
cavaleiro no dia, e entendestes que ela te contemplou, então nisso pensaste e
assim te encontrou o Inimigo com um arco, e te encontrou tão duramente que te
fez tropeçar. E te fez tropeçar tanto que te tirou do caminho correto e te
lançou onde nunca tinhas pensado, que era o caminho da falta de castidade, que
lá perde corpo e alma tão maravilhosamente que ninguém pode bem saber, a não
ser que tente. E lá o Inimigo recebeu o rosto. E tão logo tinhas queimado teus
olhos com o calor da falta de castidade, e de pronto saíste da humildade e
contiveste a cortesia e foste com a cabeça erguida como um leão e pensaste em
teu coração que não deverias louvar ninguém e deverias nunca te louvar, então
tu tiveste o amor daquela que vias tão bela. E pois que o
Inimigo ouviu toda a fala, tão logo ela falou a língua, e confessou que tu
querias pecado mortal com pensamentos e com vontade, então se dirigiu para ti
de pronto, e expulsou Aquele que por tanto tempo tinhas albergado. E assim
Nosso Senhor te perdeu, que por tanto tempo tinha te movido e aconselhado com
todas as boas virtudes e tinha a ti tão elevado que tinha te recebido em Seu
serviço. E pois que considerou que eras Seu servo e querias servi-Lo por causa
do bem que te tinha feito, e emprestado, então de pronto o deixaste e te
tornastes servo do diabo. Ele lançou em ti tanto de seu poder quanto Nosso
Senhor tinha feito do Seu, que contra a pureza, conservaste falta de castidade,
quando uma perdia a outra, e contra a humildade, recebeste cortesia, como
ninguém louva mais que a si mesmo.

Depois expulsaste todas as virtudes que te chamei e conservaste todas
aquelas que eram contra ti. E ainda então tinha Nosso Senhor Deus feito tanto
bem em ti e perfeição que não podia ser, que tinha que permanecer em alguma
medida. E daquilo que Deus te deixou, fizeste a maior nobreza em terras
estrangeiras, de que o mundo todo diz. Então contempla o que poderias ter
feito, se tivesses em ti conservado todas as virtudes: não terias falhado,
terias completado a aventura do Santo Graal, em que todos os outros estão em
trabalho, e terias trazido muito a fim como nenhum homem, a não ser o
verdadeiro cavaleiro. Os olhos não se te teriam parado da face de Nosso Senhor,
e O terias visto claramente. Todas essas coisas eu te disse por causa de que
estou sofrendo e triste por estares tão enganado e envergonhado, que nunca
ganhas honra em nenhum sítio a que vás, e devem todos ressoar, que saibam como
te aconteceu nesta demanda. Ainda então não pecaste tanto, que ainda vens à
graça, se fores de bom coração. Eu não te aconselhei que cavalgasses
futuramente nesta demanda, quando bem deves saber que ninguém nela ingressou,
que se apartasse sem vergonha, a não ser que tenha feito verdadeira confissão.
Quando essa demanda é de coisas celestiais e não terrenas. E quem quer vir ao
Céu impuro, tropeça tão duramente que sente todos os dias que viveu. E assim é
daqueles que vieram a essa demanda impuros e enganados, cheios de pecados
terrenos, assim que não conseguem encontrar nem caminho nem via, e cavalgam
como tolos em terra estrangeira. E com isto lhe aconteceu uma parábola do
Evangelho, que dentro se encontra escrito que havia um homem rico, que tinha
feito preparar uma sociedade e para ela convidou seus amigos e seus vizinhos. E
pois que as mesas estavam postas, então enviou seu servo e seu mensageiro para
aqueles que ele tinha convidado e encarregou-lhes de que tudo estivesse pronto.
Então eles se dissiparam e ficaram tanto tempo, que o bom homem desgostou-se. E
pois que via que eles não vinham, então falou a seus servos: ‘Ide de novo às
vias e às alamedas chamar e dizei aos conhecidos e aos estranhos, aos ricos e
aos pobres, que venham comer, quando todas as coisas estão prontas!’ Então
fizeram o pedido de seu senhor e trouxeram tantas pessoas consigo que a casa
ficou toda cheia. E pois que todos estavam sentados, então viu o senhor, entre
os outros, um homem que não estava vestido com roupas de bodas. Então veio a
ele e falou: ‘O que fazes aqui dentro?’ ‘Senhor,’ falou ele, ‘vim como os
outros’. ‘Em verdade’, falou o senhor, ‘tu não vieste, quando vieram cheios de
alegria e vestidos como se deve vir às bodas, quando não trouxeste coisa
nenhuma que fosse pertencente a uma boda, e de pronto se lhe fez empurrar de
sua casa. Então falaram os que se assentavam à mesa que ele tinha convidado dez
vezes como eles tinham vindo. Por causa disto se pode verdadeiramente falar que
vós muito sois chamados, e poucos são escolhidos. 

Esta parábola, da qual fala o Evangelho, podemos volver para esta demanda.
Quando das bodas, que ele fez chamar, podemos entender a mesa do Santo Graal, à
qual os homens mais nobres e os verdadeiros cavaleiros devem sentar-se, aqueles
que Nosso Senhor encontra vestidos com roupas que se prestam às bodas. Quando
aqueles que Ele encontra desnudos e descurados da verdadeira confissão e de
boas obras, que Ele não deve receber e o faz lançar para fora da companhia dos
outros, assim que ele receba tanta vergonha e fardos como os outros ganham”.  E
com isto se calou e contemplou Lancelot, que lá chorava tão seriamente como se
visse o mundo inteiro jazer morto diante de si, como lá estava tão triste, nem
soube o que deveria tornar-se. E pois que o contemplou longamente, então lhe
perguntou se tinha feito sua confissão alguma vez desde que veio a essa
demanda. E ele respondeu bem baixo “sim” e contou-lhe todo o seu ser e as três
falas, que aquele lhe tinha significado. E pois que o santo homem ouviu que ele
lhe dizia, então falou: “Lancelot, eu te suplico pela cristandade que tens e
pela ordem da cavalaria que recebeste há maior tempo, que me queiras dizer qual
vida melhor te agradou, aquela que até aqui trazido e exercitado, ou esta a que
recentemente viste”. “Por verdadeiro,” falou Lancelot, “Senhor, a nova vida me
agrada melhor cem vezes do que a outra fez. E a quero viver, quero nunca dela
me apartar por causa de nenhuma aventura que pode me acontecer”. “Então sem
dúvida”, falou o bom homem, ``quando de onde Nosso Senhor Deus vir que O amas de
bom coração, Ele deve te enviar tanta graça  que para Ele deves ser um templo e
um abrigo”. Nesta medida desempenharam o dia. E pois que era
noite, então comeram pão e beberam cerveja, que acharam na clausura. E
então se deixaram dormir um pouco, quando mais pensaram em coisas espirituais e
não terrenas.

Na outra manhã, pois que o bom homem queria sepultar o cadáver à frente do
altar, então foi à clausura e falou que nunca por tanto tempo quanto vivesse se
queria apartar, pois queria servir a seu Criador celestial. E pois que viu que
Lancelot queria armar-se, então falou: “Lancelot, eu vos\footnote{ Novamente
se registra a contumaz alteração do pronome de tratamento para a modalidade
formal.} peço em reta penitência que vistais a camisa de lã do
santo cadáver de agora em diante. E bem vos digo que disto vos acontecerá tal
bem que nunca cairás em pecado mortal tanto tempo quanto a vestires. E vos peço
mais que vós, por tanto tempo quanto estejais na demanda, nunca deveis comer
nenhuma carne nem beber vinho, e que ides todo dia ouvir o serviço de Nosso
Senhor,  se estiveres em qualquer sítio onde o possais fazer”. E ele
recebeu o pedido por reta penitência e retirou-se da presença do bom homem e
recebeu a disciplina com boa vontade. E tomou a camisa de lã, que lá estava
afiada e aguda, e a vestiu e pôs sua roupa por cima e se armou e tomou licença
do bom homem. E lhe deu de bom grado e lhe pediu muito que bem fizesse e que
não deixasse de forma nenhuma, que fizesse sua confissão, assim que o Inimigo
não tivesse mal a lhe fazer. E ele respondeu que deveria fazê-lo. E com isso
apartou-se dali e cavalgou o dia todo através da floresta até as vésperas, que
não achou aventura que fosse de se contar. 

E depois das vésperas uma donzela o encontrou, que se sentava sobre um
palafrém branco e cavalgou logo. E pois que viu Lancelot, então o saudou e
falou: “Senhor, o que quereis?” “Seguramente”, falou ele, “eu não sei aonde me
conduz a aventura, quando bem não sei onde devo achar aquilo que cavalgo
procurando”. “Eu bem sei o que procurais,” falou ela, “estais muito mais
próximo dele do que considerais, desde que conserveis o que viestes até ele”.
“Donzela,” falou ele, “as duas falas que dizeis me parecem desiguais”.
“Censurai-me”, falou ela, “quando deveríeis brevemente ver mais do que vedes ou
podeis calar. Então não vos disse coisa nenhuma, bem deveis ainda encontrar”.

Pois que o tinha dito, e queria partir, então ele lhe pergunta onde deveria
abrigar-se à noite. “Não achais assim abrigo,” falou ela, “senão amanhã deveis
achar um tal, como dele precisais, e lá deveis encontrar auxílio, pelo que
estais em preocupação”. E ele a encomendou a Deus, e assim se apartaram um do
outro. E Lancelot cavalgou as vias através da floresta, até que veio entre
dois caminhos. E lá ficavam duas cruzes para separar as vias. E pois que ele
viu a cruz, então ficou muito contente que a tivesse, e falou que queria bem lá
tomar seu abrigo, e aproximou-se da cruz e desmontou e tirou o arreio do corcel
e a sela e o deixou pastar. E tirou seu escudo de seu pescoço e tirou seu elmo
e ajoelhou-se perante a cruz e falou sua oração, pediu Àquele que na cruz foi
golpeado, em cuja honra em igualdade a cruz lá fora colocada, que o protegesse
de tal maneira que não caísse em pecado mortal, quando nada temia tanto. Em tal
medida fez sua oração e pediu a Nosso Senhor por um longo momento e deitou-se
sobre uma pedra, que lá estava à frente da cruz. E lhe aprouve bem dormir,
quando estava muito cansado de jejuar e vigiar, e lá de pronto adormeceu.

Pois que estava adormecido, então lhe pareceu que perante ele veio um homem,
ocupado de estrelas em todo redor, e tinha em sua companhia sete reis e dois
cavaleiros, e ele tinha uma coroa dourada sobre sua cabeça. E então vieram
perante Lancelot e ficaram quietos e inclinaram-se para a cruz e lá à frente
fizeram sua oração. E pois que tinham longamente rezado, então se sentaram
todos e seguraram suas mãos contra o céu e clamaram com voz alta: “Senhor do
Céu, vem nos ver e dá a cada homem o que ele merece e nos coloca em tua casa,
pois muito desejamos entrar”. E pois que o tinham dito, então se calaram todos
calmos. E então viu Lancelot que as nuvens se abriam e de lá saiu um homem com
grande companhia de anjos e desceu e deu para aqueles sua bênção, e os chamou
servos bons e verdadeiros e falou: ``Minha corte está pronta perante todos vós,
vinde à alegria que nunca toma fim!” E pois que o tinha feito, então veio ele
a um dos dois cavaleiros e falou: não foste meu amigo e por toda a via
guerreaste contra mim. Foge daqui, quando perdi tudo aquilo que te encomendei,
e eu devo te afundar, tu me ganhas então de novo meu tesouro”. E pois que ouviu
esta fala, então fugiu dos outros e pediu graça tão triste como podia. E o
homem falou: “Se queres, eu te tenho amor, se quiseres, assim te odeio”, e
aquele se apartou dos outros e da companhia. E o homem que desceu do céu veio
ao mais jovem dos cavaleiros dentre eles todos e lhe deu asas e falou: “Querido
filho, podeis voar sobre toda a cavalaria”. E então ele começou a voar. Então
se tornaram suas asas tão grandes e tão maravilhosas que todo o mundo foi
coberto por elas. E ele se elevou contra as nuvens, e de pronto se fechou o
céu, para acolhê-lo e ele seguiu dentro sem obstáculo. 

Assim veio depois a Lancelot em seu sono que ele viu este significado. E
pois que viu que era dia, então suspendeu sua mão e faz uma cruz à frente de
sua testa e encomendou-se a Nosso Senhor Deus e falou: “Amado Pai Jesus Cristo,
és um verdadeiro forte e um verdadeiro redentor para todos aqueles que te
clamam com todo o coração. Senhor, eu te clamo e te agradeço de me teres
redimido de grandes vergonhas e de grande padecimento e precisei ter sofrido,
não tivesse a graça de teus bens feito. Senhor, eu sou Tua criatura, mostraste
tão grande amor, quando lá estava minha alma em grande inquietação a rumar para
o inferno e para a eterna danação, por meio de Tua misericórdia a resgataste e
conclamaste-a a Te confessar. Senhor, por meio de Tua grande misericórdia, não
a deixa ir a outra parte que não o reto caminho, e protege-me, que eu não me
perca, que o Inimigo não me ache senão fora de suas mãos”. E pois que o tinha
dito, levantou-se de seu assento e veio a seu cavalo e colocou-lhe a sela e o
arreio. Então tomou seu elmo, sua lança e seu escudo, sentou-se em seu cavalo e
pôs-se a caminho, como tinha feito ao dia, e pensou em tudo que lhe veio no
sono. Quando não conseguiu estar ciente daquilo a que o sonho se voltava,
quando de bom grado teria sabido. E como tinha cavalgado até ao meio-dia, então
lhe estava muito quente, então o encontrou em um vale o cavaleiro que lhe tinha
levado adiante suas armas antes de ontem.

E pois que o viu vir, então não o saudou, quando lhe disse: “Protege-te de
mim, Lancelot, quando estás morto, será que não podes te proteger?!” Então
veio a ele com a lança de longe e o picou através do escudo e da coifa, e ele
não lhe sacudiu a carne. E então tenta Lancelot seu poder e sua força e o
pinica tão duramente que derrubou a ele e a seu cavalo tão impiedosamente que
foi maravilha que não lhe tivesse quebrado o pescoço. Então correu à frente e
volveu para trás de si e viu que seu corcel estava de pé. E então tomou o
corcel com o arreio e o atou a uma árvore, por causa da vontade de o cavaleiro
achar quando se levantasse de novo. E pois que o tinha feito, então se pôs a
caminho e cavalgou até a noite. Então estava cansado e desmaiado como aquele
que não tinha de comer durante todo o dia e nem no dia anterior. E então
cavalgou pelo tempo de dois grandes dias, que muito se cansou e trabalhou. E
tanto como cavalgou, veio perante uma clausura, que lá ficava em um caminho.
Então observou e viu sentar-se à frente da porta um velho homem. E então ficou
contente e o saudou, e então o velho o saudou de volta amigável e lindamente. 

“Senhor,” falou Lancelot, “podeis albergar um cavaleiro errante?” “Caro
Senhor,” falou o bom homem, “sendo de vosso agrado, eu vos abrigo esta noite o
melhor que posso e vos dou de comer daquilo que Deus nos deu”. Então ele falou
que não desejava melhor. E então o bom homem tomou o cavalo e o conduziu a um
tapume, que lá ficava à frente de sua casa, e retirou ele mesmo a sela e o
arreio e lhe deu do pão, e lá junto ele tinha partido. Depois tomou a lança e o
escudo de Lancelot e os levou à sua casa. E pois que estava, à vez, desarmado,
então lhe perguntou o bom homem se tinha ouvido as vésperas. Então ele lhe
respondeu que ele não tinha visto nem homem nem mulher nem casa senão um homem
que o encontrou ao meio-dia. Então foi o bom homem à sua capela e chamou seu
aluno e principiou as vésperas do dia e depois as de Nossa Senhora. Pois que
tinha dito o que pertencia ao dia, então saiu da capela e pergunta a Lancelot
quem seria ele e de que terra. Então lhe diz toda a sua vida e não lhe oculta
nenhuma coisa que lhe sucedeu do Santo Graal. E pois que o bom homem ouviu esta
aventura, apiedou-se muito de Lancelot, quando viu que ele começou a chorar da
hora em que começou a dizer e contar a aventura do Santo Graal. Então lhe pediu
por Nossa Senhora e por toda a santa fé que lhe dissesse confissão e toda a sua
vida. Então ele falou que o faria de muito bom grado, porque o queria. Então o
conduz à sua capela e Lancelot lhe contou todo o seu ser como tinha feito das
outras vezes. E depois lhe pediu por Deus que lhe desse conselho.

E pois que o bom homem tinha ouvido sua vida e sua confissão, então começou
a consolá-lo muito e lhe diz tantas boas palavras e falas que ficou muito mais
feliz do que antes tinha estado. Então falou: “Senhor, informai-me daquilo que
vos pergunto, se souberes”. “Dizei,” falou o bom homem, “quando não é coisa
nenhuma, eu vou informo conforme meu poder”. “Senhor,” falou Lancelot,
“pareceu-me hoje à noite, em meu sono, que à minha frente veio um homem,
tomado, à vez, de estrelas e tinha em sua companhia sete reis e dois
cavaleiros”, então lhe contou palavra por palavra como tinha visto. Pois que o
bom homem ouviu a fala, então lhe diz: “Oh Lancelot, podes ter visto o alto
nascimento da linhagem de onde procedes. Pois sabe seguramente que há mais
significado suficiente do que as pessoas consideram. Então me ouve, se
quiseres, e vou te dizer o começo da tua linhagem e de todos os teus pais, que
lá foram muito nobres e muito boas pessoas. Quando isto tomo muito longe daqui,
quando se deve assim fazer.

Então foi depois da morte de Nosso Senhor Jesus Cristo, nos tempos em que
José de Arimateia, o autêntico cavaleiro, o nobre homem, rumou para fora de
Jerusalém pelo mandamento de Nosso Senhor, para pregar e conseguiu fazer a
verdade e a Nova Aliança e o mandamento dos Evangelhos, então veio à cidade de
Saras. Lá encontrou homem pagão, um rei que lá se chamava Anales, que tem
guerra contra seu vizinho, que era rico e violento. Pois que veio junto ao rei,
então lhe aconselhou de maneira que vencesse seu inimigo, e aconteceu no campo
por causa da ajuda que Deus lhe enviou. E de pronto, tão logo retornou à sua
cidade, então recebeu o Batismo das mãos de Josephes,\footnote{ O filho de José
de Arimateia fora denominado, em outros trechos da narrativa,
\textit{Josephus}, mas neste momento se escreve \textit{Josephes}. 
} o filho de José. Ele tinha um cunhado que lá se chamava Seraff tanto tempo
quanto foi um pagão; quando desprezou sua aliança, foi chamado Nasiens.

  Pois que o cavaleiro veio à Cristandade e tinha feito definhar sua
aliança, então crê tão bem em Deus e tanto amor tem ao seu Criador, que ele era
uma pedra e uma fortaleza na fé. E bem era uma coisa aparente que ele era um
homem nobre e verdadeiro, que Nosso Senhor Deus o deixou ver grande honra do
Santo Graal, de que nenhum cavaleiro ouviu falar nos tempos, a não ser José.
Ainda depois foram poucos os que o viram, que não fosse em seus sonos ou
sonhos. Nos tempos, pareceu ao rei Evalet de seu sobrinho, o filho de Nasiens,
de sua linhagem, tivesse corrido um lago de tal maneira que do lago saíram nove
rios, de que os legítimos eram de uma grandeza e de uma profundidade; então o
último era de largura e profundidade maiores que todos os outros. E era tão
fluente e tão forte que era maravilha. E era turvo no começo e a densidade como
um lamaçal, e no meio era mais alto e limpo, e no final era de outra forma.
Quando era dele como se fosse tão claro quanto no começo, e era tão doce de
beber que lá ninguém conseguia saciar-se e era o último das nove águas de que
vos digo.

Depois observou o rei e viu um homem vir do céu, e ele trouxe um significado
e uma parábola de Nosso Senhor. E pois que veio ao lago, então lavou lá dentro
suas mãos e seu pé,  e em tudo fez assim. E pois que veio ao último, então
lavou o rosto e todo o seu corpo. Isto viu o rei Morderas em seu sono. Como
devo te\footnote{ Nova alteração do pronome de tratamento, para a modalidade
informal.}  mostrar o significado que era: o sobrinho do rei, que
saiu do lago, era o filho de Nasiens, assim que Nosso Senhor o enviou do Reino
do Céu por causa de expulsar e matar os incrédulos. Era seguramente um servo de
Deus, que sabia também o curso das estrelas e o ser dos planetas e do
firmamento tão bem, ou melhor, do que fez o filho de José. E por causa de que
era um homem tão sábio, então ele veio perante ti estando com estrelas, e era o
primeiro rei cristão que lá conservou o reino dos escoceses, onde havia
seguramente um lago e no qual podemos perceber todos os sinais e todo pedaço de
divindade. Do lago fluíam nove rios, que eram nove pessoas de homens que vieram
dele; não assim que vieram com injustiça, quando vieram de retas linhagens, um
do outro. Deles então são sete reis e dois cavaleiros. O primeiro rei que lá
veio da linhagem era chamado Narpus, e era um nobre homem e amava assim muito a
Santa Igreja. O outro era chamado Nasiens, em igualdade a seu ancestral. Nele
se albergava Nosso Senhor Deus tão naturalmente que não se achava em seus
tempos homem mais nobre. O terceiro rei era chamado Eluen, o Grande, que
preferiria ter estado morto a ter feito mal contra seu Criador. O quarto era
chamado Elans, era um homem nobre e verdadeiro e temia Nosso Senhor acima de
todas as coisas e nunca irou seu Deus. O quinto era chamado Gavens, que era
bom, mais verdadeiro e piedoso que nenhum homem, e nunca em seu ser irou a
Nosso Senhor. Ele saiu desta terra e rumou para a Gália [..] e veio morar na
terra e tomou por esposa a filha do rei dessa terra. Era um homem nobre, como
ouviste, pois achaste no nascedouro o corpo de teu ancestral. Dele veio o rei
Ban, teu pai, que também era um nobre homem e de vida santa. Quando algumas
pessoas consideram que o lamento de sua terra o chamou à morte; porém ele não
tinha rezado por alguns dias, e todos os dias que viveu, pediu a Deus
que o deixasse morrer quando lhe pedisse. Então Deus lhe mostrou que o tinha
ouvido, quando assim como ele desejou a morte, então faleceu seu corpo, e sua
alma achou a vida. Estas sete pessoas que te nomeei, que lá são o início da tua
linhagem, são os setes reis que vieram a Morderas em seu sono; e eram os sete
rios, que lá fluíam do lago, que o rei Morderas viu em seu sono. E em todos os
sete Nosso Senhor lavou suas mãos e seu pé. Então preciso te dizer quem são os
dois cavaleiros que lá estavam em sua companhia. O mais velho deles que o
seguiu, é falado que desceu deles [..] quando tu vieste do rei Ban, que lá era
o último dos sete reis. Antes de estarem uns junto aos outros à tua frente,
então falaram: ‘Pai do Reino do Céu, vem ver-nos e coloca-nos em tua casa’.
Nisto que falaram, ‘vem nos ver’, então te faziam companhia em sua companhia e
pediram a Nosso Senhor que quisesse tomar a eles e a ti, por causa de que eram
o princípio e a raiz de ti. E porque falaram ‘Dá a cada homem o que eles
mereceram’, então podes perceber que nunca ganharam entre si senão justiça. E
por causa do amor que tinham por ti, não queriam pedir outra coisa a Nosso
Senhor Deus senão desse a cada qual o que tinha merecido. Pois que isto estava
prometido, então te pareceu que viesse do céu um homem com grande companhia. E
pois que tinha falado do mais velho dos cavaleiros, e tinha lhe dito a fala,
que bem te lembras, faz bem em tomar sobre ti como aqueles que lá estavam
falavam para ti e à frente de ti, quando és um significado disto, de que esta
coisa é dita. Então veio ao cavaleiro jovem, que nasceu de ti e da filha do
rico Rei Pescador. Assim ele veio de ti e ficou como igual a um leão; é para
saber que ele se assentava sobre todos os sábios da cavalaria terrena, como que
ninguém o pode igualar nem em força nem em poder. E lhe deu asas, por causa da
vontade de que ninguém fosse tão rápido e tão pequeno, que ninguém pudesse vir
tão alto nem com nobreza nem com outras coisas como ele fez, e falou: ‘Amado
filho, então podes voar sobre todo o mundo e sobre toda a cavalaria terrena’. E
aquele de pronto principiou a voar, e se lhe tornaram as asas maravilhosamente
grandes, que todo o mundo ficou coberto com elas. E tudo que ouvis\footnote{
Nova alteração pronominal no interior de uma mesma fala, passando para a
modalidade formal.}  aconteceu agora a Galaat, o bom cavaleiro, que
lá é vosso filho. Quando ele é de vida tão santa que é maravilha. E por causa
de que veio tão alto que ninguém pode vir junto dele, podemos bem falar que
Nosso Senhor lhe deu asas, e junto a ele podemos bem entender o nono lago, que
o rei Morderans viu em seu sono, que lá era mais amplo e largo que todos os
outros. Então te\footnote{ O registro retorna, neste momento, ao pronome
informal.}  disse quem são os sete reis, que viste em teu sono e
quem o cavaleiro era, que lá foi empurrado de sua companhia, e quem lá era o
último a quem Nosso Senhor Deus deu Sua graça, que o fez voar sobre todos os
outros”. “Senhor,” falou Lancelot, “que me dizeis que o bom
cavaleiro seja meu filho, muito me assusta”. “Não deveis estar assustado,”
falou o bom homem, “nem deveis tomar isto por nenhuma maravilha, quando bem
sabes\footnote{ Nova alteração pronominal no interior de uma mesma fala,
passando para a modalidade formal. }que dormiste junto à filha do
rico Rei Pescador. E a mesma deu à luz [um filho de] Lancelot, isso se te disse fartamente.
E esse Galaat, que ganhaste junto à mesma donzela, é o cavaleiro que se sentou,
em Pentecostes, no Assento Perigoso, e é o cavaleiro que procuras. Então te
disse dele e te o dei a reconhecer, pelo que não quero que tu te animes a
duelar com ele, quando bem podes fazê-lo cair em pecado mortal por perder teu
corpo. Quando onde te animares a lutar com ele, lá poderíeis bem saber que
seria uma coisa feita contigo, pois a tua nobreza pode igualar a dele”.

 “Senhor,” falou Lancelot, “a coisa é para mim uma grande consolação, o que me
dissestes. Quando me parece, porque Ele impôs que tal fruto veio de mim, aquele
que lá é tão nobre não deve permitir que seu pai, como ele é, fosse à perdição,
e deve simplesmente pedir a Nosso Senhor, noite e dia, que Ele o tire da vida
má, onde por tanto tempo esteve”. “Vou te dizer”, falou o bom homem, “o que é.
Dos pecados mortais carrega o pai seu fardo e o filho, os seus. E o filho nunca
ganha parte nos pecados do pai, nem o pai nos do filho, pois a cada homem, como
ele mereceu, conforme isto deve lhe ser pago. E por causa disto não deves ter
nenhuma esperança por teu filho, senão altamente em Nosso Senhor Deus. Quando
se procuras por auxílio, Ele te ajuda e te redime além de toda a necessidade”.
Então falou Lancelot: “É então que nenhum homem, sem Cristo, pode me apiedar
nem ajudar, assim vos peço que Ele bem me ajude e que Ele não queira me deixar
cair nas mãos do Inimigo e que eu Lhe possa dar de novo o tesouro que Ele me
deseja, que é a alma minha no Dia do Juízo. Quando Ele fala para os maus: ‘Ide
daqui, vós malditos, para o fogo eterno’, e fala para os bons: ‘Vinde cá, vós
benditos herdeiros de Meu Pai e vós benditos filhos, e ide à alegria que nunca
toma fim’”. 

Falaram longamente um com o outro, o bom homem e Lancelot. E pois que era tempo
em que se devia comer, então saíram da capela e sentaram-se na casa do bom
homem e comeram pão e cerveja. E pois que tinham comido, e bom homem
chamou Lancelot a deitar-se na grama, como não tinha preparado outra cama. E
então adormeceu de pronto, quando estava cansado e esgotado. E não se voltou
tanto para a volúpia do mundo quanto tinha feito antes; quando se teria voltado
para ela, não teria nunca conseguido adormecer por causa da vontade da terra,
que lá era tão dura e por causa da camisa do senhor, que lá era áspera e
pinicante em sua carne. Quando foi trazido para que este desconforto e esta
dureza bem o agradassem, que não soubesse de nenhuma coisa que o pudesse
agradar mais, e por causa disto não o apenou nenhum desconforto que tinha. À 
noite então repousou na casa do bom homem. Pois que se fez dia, então se
levantou e fez sua oração. E pois que o bom homem tinha cantado, Lancelot
tomou suas armas e montou seu cavalo e o recomendou a Deus. E o bom homem lhe
pediu muito que ele se conservasse no que tinha começado. Ele falou: “Eu o
faço, se Deus me incorporar saúde”. E cavalgou em frente através da floresta o
dia todo, de maneira que não parou em vereda nem via, e refletiu muito pela sua
lama, e foi-lhe grande padecimento o grande mal que tinha feito, e por causa do
qual estava exilado da alta companhia, que tinha visto em seu sono. E era uma
coisa que ele muito temia, que caiu em dúvida. Quando por causa de que colocou
toda a sua coisa em Jesus Cristo, então pensou ainda voltar para o sítio de
onde tinha sido enxotado, e fazer companhia àqueles de quem ele vinha. E pois
que tinha cavalgado até o meio-dia, então veio a um plano, que lá estava no
ermo. E viu um grande castelo estar à sua frente, com muros e com túmulos, e à
frente do castelo ficava um gramado, e havia tendas armadas de panos de seda e
cores maravilhosas, bem cem. E à frente das tendas estavam bem cem cavaleiros e
alguns grandes corcéis, e tinham começado um grande torneio e em muito
maravilhoso. E alguns estavam cobertos com armas brancas e os outros com
pretas, e nenhuma outra diferença tinham entre si. Aqueles que lá tinham as
armas brancas mantinham-se junto ao ermo, e os outros, junto ao castelo. E
tinham já começado o torneio à vez maravilhosamente, e lá havia muitos
cavaleiros golpeados ao chão, que era maravilha. Então contemplou o torneio
longamente, por tanto tempo que lhe pareceu que os de junto do castelo tiveram
o maior dissabor e perderam o campo de batalha e tinham, porém, mais cavaleiros
do que os outros tinham. Pois que o viu, voltou-se para eles, como se aí viesse
para ajudar com seu poder. Ele afundou sua lança e deixou correr seu cavalo e
pinicou o primeiro tão duramente, que caiu por terra com seu cavalo. E seguiu
em obstáculo e pinicou outra vez e quebrou sua lança, pelo que caiu. Então
conduziu sua mão à sua espada e começou grandemente dar golpes para baixo e para
cima pelo torneio, como aquele que era de grande nobreza, e fez tanto em curtas
horas que todos que o viam deram-lhe honra e o prêmio do torneio. E porém não
conseguiu superar os que estavam contra ele, quando eram tão sofríveis e
dolentes que era maravilha. Ele os golpeou, e eles a ele, muito e quase, como
aquele que golpeava uma madeira podre. Porém não o fizeram tão duramente quanto
queriam os golpes que lhe davam, quando tomaram terra sobre si por toda a via.
Ele ficou tão cansado que não conseguia manter sua espada, quando estava
maravilhosamente cansado e esgotado, que considerou que conseguiria portar
nenhum trabalho nem arma. Então o pegaram com violência e o conduziram ao ermo
e o assentaram lá, e todos os seus companheiros estavam sobre o chão. E aquele
que lá conduziam Lancelot, disseram-lhe: “Lancelot, tanto fizemos que sois
dos nossos e sois em nossa prisão. Deve ser, se quereis sair, que façais nossa
vontade”. E ele assegura e se foi deles e os deixou no ermo, e rumor a uma
outra via que não aquela que tinha percorrido. E pois que veio longe deles, que
o tinham pego, então pensou que hoje estaria envergonhado, e não estaria tão
envergonhado e não teria vindo a torneio nenhum, se ele teria o prêmio e não
teria sido pego em nenhum torneio. 

Pois que o pensou, então começou a ter arrependimento fora de medida e falou que
bem via que era mais pecador que nenhum outro, quando seus pecados e suas más
aventuras lhe tomaram a visão dos olhos e o poder do corpo. É bem uma coisa de
tentar, da visão do Santo Graal, que não conseguiu ver. Do poder do corpo, isso
tinha bem tentado, quando nunca mais veio entre tanto povo como tinha sido no
torneio, que pôde ficar cansado e assim esgotado, quando os fez todos saírem do
lugar, fosse-lhe bem ou mal. Assim irado e assim sem coragem cavalgou por tanto
tempo até que lhe sobreveio a noite em um vale, que era grande e profundo. E
pois que viu que não conseguia vir para abrigo, então se sentou sob uma grande
macieira, retirou a sela e o arreio de seu cavalo, e retirou seu elmo e retirou
sua coifa e deitou-se sobre a grama. E de pronto adormeceu, quando estava muito
cansado e esgotado, mais que esteve por muito tempo. 

Pois que estava adormecido, então lhe pareceu que do céu vinha um homem que lá
bem podia igualar um bom homem, e veio retamente como um homem irado e lhe
disse: “Hei, homem desleal e de fé débil, por que causa voltaste tua vontade
tão levemente ao teu inimigo mortal? Não te protegeste, ele te fez cair no
charco de onde ninguém regressa”. Pois que o tinha falado, então desapareceu da
maneira que Lancelot não soube de onde teria vindo. Então ele estava
insatisfeito demais com esta fala. Entretanto, não despertou até o dia, quando
o dia já brilhava. Então se levantou e faz uma cruz à frente de sua testa e se
encomenda a Nosso Senhor e olha ao redor de si, e não vê seu cavalo. Então o
procura por tanto tempo até que o achou, então colocou a sela e montou. E como
estava pronto e porque queria prosseguir caminho, então viu do lado direito do
caminho que junto a um tiro de besta próximo sentava-se uma enclausurada, que
se reputava a melhor mulher de toda a terra. Pois que o viu, então falou que
seguramente era um maldito e que seu pecado o obstava de todas as boas obras.
Quando lá estava, que ele pela noite não veio no tempo certo, quando bem teria
ido para lá de dia e teria vivido de conselho de sua vida. Então rumou para lá
e sentou-se à porta e atou seu cavalo a uma árvore e tirou seu escudo e sua
espada e colocou tudo à sua frente. E pois que adentrou, havia sobre o altar os
paramentos para o padre vestir-se. E à frente do altar estava um capelão, e
estava de joelhos e falava sua oração. E não durou muito até que tomasse as
armas de Nosso Senhor e as vestisse, e começou a alçar missa da régia Mãe de
Deus. E pois que tinha cantado e destrajado, então chamou a enclausurada que lá
tinha uma pequena janelinha que ela via o altar,\footnote{ \textit{Sic} no próprio texto
medieval. Em alemão contemporâneo, como em português, deveria haver uma
preposição, \textit{durch}.}  Lancelot, por
causa de que ela pensou que ele seria um cavaleiro de aventura e que fazia
necessidade de conselho. E ele veio a ela, e ela lhe pergunta quem ele seria e
de que terra e o que ele procurava. E ele lhe diz palavra por palavra o que ela
lhe tinha perguntado e também lhe contou a aventura do torneio, onde tinha
estado, e como os de arma branca o pegaram e a fala que lhe foi dita. Depois
ele lhe contou o que veio no sono. E pois que tinha lhe contado todo o seu ser,
então ele lhe pediu que lhe desse conselho o melhor que ela pudesse. E ela
falou para ele: “Hei Lancelot, em primeiro vos venceu a cavalaria celestial,
pois éreis o mais aventuroso cavaleiro do mundo. Não deve vos ter maravilha se
vos vêm muitas aventuras, quando não pode ir da mesma maneira por toda a via.
Entretanto do torneio, de que falais, devo vos dizer o significado, quando algo
de  muito especial vos aconteceu, seja dormindo ou acordado, que dizeis que não
é outra coisa senão significado de Nosso Senhor Jesus Cristo. Entretanto, sem
erro e sem engano, esse torneio é da cavalaria terrena, quando tem mais
significado do que considerais. Doravante vos digo por que causa o torneio foi
feito: para se ver quem lá tinha mais cavalaria, se Elias, filho do rei Pellis,
ou Enlugustis, filho do rei Helen, e por causa de que se confessaram uns entre
os outros, então fez Elias que os seus se cobrissem com armas brancas e os
outros com pretas. E pois que vieram uns com os outros, então os pretos foram
vencidos, ainda assim vós os ajudastes e tinham também mais povo que os outros.
Então vos devo dizer o significado destas coisas.

Então no dia de Pentecostes começaram os cavaleiros de ordem e os cavaleiros
espirituais um torneio uns com os outros. Pois é dito que começaram uns com os
outros a demanda. Os que lá estão em pecado mortal, são os cavaleiros de ordem;
e os cavaleiros espirituais, esses são os verdadeiros cavaleiros e as nobres
pessoas, que lá não estão maculadas com pecado mortal, que começaram a demanda
do Santo Graal. Era o torneio, que os cavaleiros de ordem principiaram, que lá
tinham a terra nos olhos e no coração, que começaram um torneio contra os
cavaleiros espirituais. E sua cobertura era preta, como os que lá estavam
cobertos com malditos pecados negros. Os espirituais estavam cobertos de
branco, isto é falado de pureza e castidade, que nada maculava com negridão.
Pois que o torneiro foi principiado, então vistes pecadores e pessoas nobres.
Então te\footnote{ Mais uma vez se dá a alteração pronominal em um mesmo
discurso.}  pareceu que os pecadores eram superados, e por causa
disto tomaste o partido dos pecadores, é falado que duraste em pecado mortal, e
te voltaste para ele e te puseste contra as boas pessoas, pois tu querias
espetar contra Galaat, teu filho, no que ele espetou abaixo teu cavalo, e
também Parsifal. 

Pois que tinhas estado um bom momento no torneio e estavas tão cansado que não
conseguias ajudar-te, as boas pessoas te pegaram e te conduziram à floresta.
Pois que tu então, um dia, vieste à demanda, e que o Santo Graal te apareceu,
então te encontraste tão horrendo e tão impuro de pecados que consideraste que
nunca poderias portar arma nenhuma. É falado que tu te vês tão horrendo e tão
impuro que não consideraste que um dia poderias te tornar cavaleiro e servo de
Nosso Senhor Deus. Porém assim te pegaram, os que te viram no caminho de Nosso
Senhor, que lá está cheio de louvor e verdejante como o ermo estava, e te
aconselharam o que era útil para a tua alma e para ti. Quando estavas perante
eles, é falado que tu não te voltas para os pecados sempre, como antes tinhas
feito. Entretanto, tão logo te pareceu a mais asquerosa alegria do mundo e a
maior cortesia, que costumavas fazer, então te começou o arrependimento, que
não tinhas superado a todos; disto Nosso Senhor pode ter ficado irado contigo.
Então lhe veio à frente em teu sono e pois que estava a dizer-te que duraste na
verdade má e na fé débil, e que o Inimigo te fez cair no charco profundo, que é
o inferno, se não te proteges. Então te disse o significado do torneio e o
significado do sonho, por causa da vontade de que não saias do caminho da
verdade por meio de deleites passados e caias em pecado. Quando de que te
perdeste muito contra o teu Criador, sabe, se fizeres contra Ele o que deves
fazer, Ele te deixaria cair em tormento que não passa, isto é no inferno”. 
Então ele falou: “Senhora, dissestes-me tanto e o bom homem, para o
qual falei, que, estivesse eu algo em pecado mortal, dever-se-ia ralhar comigo
mais que com qualquer outro pecador”. “Deus dê, por meio de Sua misericórdia”,
falou ela, “que nunca mais caiais em pecado”.\footnote{ Mais uma vez se dá a
alteração pronominal em um mesmo discurso.} E então diz ela
de outra parte: “Lancelot, este ermo é muito grande e muito errôneo, e alguém
pode bem rumar por dois dias que nunca acha casa nem abrigo; por causa disto
dizei-me, se vistes casa. Se não tendes, eu vos dou daquela que Deus nos
aconselhou”. E ele responde que não abocanhou ontem nem hoje comida. E ela fez
trazer-lhe pão e água, e ele foi à casa do capelão e tomou aquilo que Deus lhe
acrescia. Pois que tinha comido, então se apartou de lá e encomendou a mulher a
Deus e cavalgou o dia todo até à noite. À noite deitou-se sobre uma penha alta
e maravilhosa, sem companhia de todas as pessoas, a não ser de Deus, e esteve
um longo momento da noite em sua oração e dormiu também um longo momento. De
manhã, pois que viu o dia brilhar, e então faz uma cruz à frente de sua testa e
caiu sobre seus joelhos contra o oriente e falou sua oração, como tinha feito
no outro dia. E depois veio a seu cavalo e colou-lhe a sela e lhe pôs o arreio
e cavalgou adiante, como tinha feito em outros tempos. E cavalgou por tanto
tempo até que veio a um vale, que era profundo e à vez valioso para se ver, e o
vale estava entre duas penhas maravilhosas. Pois que veio ao vale, então
começou a pensar muito duramente. Então olhou à sua frente e viu uma água, que
se chamava Marthose, que estava fechada por dois lados. Pois que o viu, então
não soube o que fazer, quando viu no meio da água,  quando era profunda e
perigosa para atravessar. E porém colocou sua esperança em Deus, entregou todos
os pensamentos e falou que deveria atravessar com o auxílio de Deus. Nisto que
estava nestes pensamentos, então lhe veio avante uma maravilhosa aventura,
quando viu sair da água um cavaleiro, armado com armas pretas, e sentava-se
sobre um cavalo, que era grande e preto. Pois que ele viu o senhor Lancelot,
então afundou sua lança para ele e não falou a ele e espetou o cavalo tão
fortemente que o senhor Lancelot não conseguiu vê-lo em bom momento. Pois que
viu seu cavalo morto abaixo de si, então se levantou de novo e não estava
triste, pois era a vontade de Deus, que nunca olhasse para ele, quando veio à
frente dele assim armado como estava. E pois que tinha voltado à água, então
não conseguiu ver como pôde atravessar. Então permaneceu calmo e tirou seu elmo
e seu escudo e sua espada e falou que queria esperar lá por tanto tempo até que
Nosso Senhor enviasse auxílio.

Assim estava Lancelot obstado por três lados: por um lado pela água, pelo outro
lado pela penha, pelo terceiro lado com o ermo. Então não conseguiu perceber
que não podia sair por nenhum lado. Quando escalou o penhasco e então teve
fome, assim não encontrou ninguém que lhe penitenciasse a fome, se não fosse
que Nosso Senhor Deus viesse em seu auxílio. E foi ao ermo, e era tão errante e
insabido que talvez pudesse se perder e lá precisar permanecer por tanto tempo
até que alguém o ajudasse. E foi à água, assim não soube como poderia sair de
lá sem grande perigo, quando é funda e larga assim que ele não conseguia
fundar-se com os pés. Estas três coisas o fizeram ficar nas margens e ele fez
sua oração a Nosso Senhor, que Ele com Sua misericórdia viesse ajudá-lo e
consolá-lo e o aconselhasse que ele não viesse em tentação do Inimigo, por meio
de engodo do diabo não fosse trazido à tentação e à dúvida. Então se cala a
fala sobre ele e vem para o senhor Gawin.

Aqui nos fala a aventura que,  pois que o senhor Gawin estava separado
de seus companheiros, que ele cavalgou alguns dias, que ele nunca conseguiu
achar aventura que  fosse para se contar. E assim fizeram também os outros
companheiros, quando nunca acharam a décima parte das aventuras que antes
costumavam. Por causa disto, entretanto, não foram à demanda. O senhor Gawin
cavalgou de Pentecostes até o dia de Santa Maria Madalena sem aventura, que
seja de contar. Então se maravilhou muito, quando na demanda do Santo Graal
pensou que a aventura seria grande e maravilhosa e considerou achar mais breve
que em outros sítios.

Um dia ocorreu que veio cavalgando a seu encontro Hector de Mares sozinho. Então
se confessaram de pronto tão logo quanto se viram. Então estava à vez contentes
e o senhor Gawin pergunta de seu ser. E ele falou que estaria saudável e
fresco, Deus seja louvado, e que não encontrou nenhuma aventura em nenhum sítio
a que viesse. “Em verdade”, falou o senhor Gawin, “disto me quero queixar a
vós, quando desde que parti de Camelot, nunca encontrei nenhuma aventura. Assim
são sei como sucedeu, quando por andar através de terra estranha e distante e
por cavalgar dia e noite, isto não permaneceu. Quando vos digo seguramente,
como meu bom companheiro, que nunca encontrei aventura, quando matei dez, que
eram nobres cavaleiros. Nem achei nunca aventura que me agradasse, senão duas”.
E Hector começou a ansiar pela maravilha que ouviu. “Então me dizei,” falou o
senhor Gawin, “nunca achastes desde então nenhum de nossos companheiros?”
“Sim,” falou Hector, “encontrei desde então em cinquenta dias mais que vinte e
aquele sozinho, e não havia nenhum que não reclamasse que não encontrou nenhuma
aventura”. Então falou o senhor Gawin: “Eu ouço maravilha, e não ouvistes falar
do senhor Lancelot?” “Seguramente não”, falou ele, “não encontrei ninguém que
me conseguisse dizer nada dele, tão certo como se tivesse afundado em um
precipício. E por causa disso estou por demais insatisfeito por ele e temo que
agora esteja preso, e Galaat, Parsifal e Bohort. Pois não ouvi desde então
falar deles, eu considero que estejam perdidos, por causa de que não são
sabidos”. “Então Deus os escolte”, falou Gawin, “em que sítio estiverem,
quando seguramente se falharam na demanda do Santo Graal, então os outros nunca
a encontrarão; e creio que bem devem vir a ela, quando são os mais nobres da
demanda”. Pois que tinham falado um com o outro por longo tempo, então falou
Hector que tinham por longo tempo cavalgado e nada encontraram: “Então
deixai-nos cavalgar um com o outro, contemplar se a sorte nos concede aventura,
mais que a cada um sozinho”. Então falou ele: “Bem adiante, e deixa-nos rumar
um com o outro, que Deus lá nos escolte à cidade, que achemos alguma coisa que
estamos procurando”. “Senhor,” falou Hector, “aqui, de onde eu vim para cá, não
encontro nada, nem de onde viestes cá”. Então falou ele, que poderia bem ser,
“pois eu louvo que tomemos outro caminho do que percorremos”. E ele falou que
bem o louvava. E Hector tomou outra via, que lá ia cruzando próximo de onde
estavam. E eles deixaram a grande via e cavalgaram afora por um caminho próximo
e cavalgaram o dia inteiro, que nunca acharam aventura, que lhes foi muito
lamentável. Um dia aconteceu que eles cavalgaram através de um grande ermo e
maravilhoso, que nunca encontraram homem ou mulher. À noite aconteceu que eles
encontraram, entre dois rochedos em uma montanha, uma capela e estava
destruída, como lhes pareceu, quando lá não morava ninguém dentro. Pois que lá
vieram, então desmontaram e afastaram seus escudos e lanças e se deixaram sair
do muro. Então tiraram dos cavalos a sela e os arreios e os deixaram pastar na
montanha. Então desataram suas espadas e as puseram junto de si. Depois foram
ao altar para falar sua oração, como bons cristão costumam fazer. E
pois que o tinham feito, então se foram sentar em um assento, que lá estava ao
lado, e falou um para o outro sobre algumas coisas. Mas de comida e de bebida
nunca tiveram nenhuma conversa, por causa da vontade de que bem sabiam que lá
nada havia. E já estava muito escuro, quando não havia nem vela nem lampião que
lá queimasse. E pois que vigiaram um tempo, então adormeceram, um aqui, outro
ali.

Pois que estavam adormecidos, então lhes sucedeu, a cada um, um sonho
maravilhoso, que não se deve esquecer, quando se deve contá-los, quando têm
significado suficientemente grande. O que sucedeu ao senhor Gawin era um grande
gramado cheio de ervas e flores. No gramado havia uma manjedoura, onde comiam
de outra parte cem touros. E os touros eram todos corteses, sem três, dos quais
três um não era nem malhado nem sem manchas, quando ele tinha um sinal de uma
mancha. E os outros dois eram a valer tão brancos não poderiam parecer mais
belos. Os três touros estavam atados com os pescoços com uma forte corda. Então
falaram todos os touros: “Vamo-nos daqui procurar melhor pastagem do que esta”.
E os touros foram de lá e foram sobre o caminho e não sobre o gramado, e
permaneceram muito longamente lá. E pois que retornaram, então muitos lhes
faltavam, e os que retornaram estavam tão magros e tão cansados que quase não
conseguiam se manter. Dos três sem manchas, lá voltou um e os outros dois
permaneceram. E pois que voltaram para a manjedoura, então principiou uma
maravilhosa rasgação,\footnote{ O termo que figura no texto original é
\textit{reyßung}, que se reporta ao verbo\textit{ reyßen},  cujo significado 
é \textit{rasgar}. Portanto,
preferimos traduzir o vocábulo medieval pelo neologismo informal “rasgação”,
por sua vinculação à cultura oral. Consigne-se, no entanto, que Hans-Hugo
Steinhoff prefere o termo “briga”.}  tal que
toda a comida se foi, e uns e outros precisaram sair dali. 

Assim sucedeu ao senhor Gawin. Mas a Hector sucedeu um outro, muito desigual,
quando lhe pareceu que ele e Lancelot saíram de um assento e montaram sobre
dois grandes corcéis e falaram: “Procuramos aquilo que nunca encontramos”. E de
pronto apartaram-se dali e cavalgaram alguns dias e por tanto tempo até que
Gawin caiu de seu cavalo, e um homem o lançou para fora e o desmontou à
vez. E pois que o tinha empurrado fora, então ele fez vestir-lhe uma saia cheia
de espinhos e o assentou em um burro. Pois que ali o tinha assentado, então
cavalga longo tempo até que veio a uma fonte, a mais bela que jamais viu. E
pois que se tinha afundado para beber, então desapareceu a fonte, assim que não
mais a viu. Pois que viu que não a conseguia ter, então volveu para o lugar de
onde veio. E Hector, que com isso não se tinha preocupado, seguiu por tanto
tempo para lá, que veio à casa de um homem rico, que lá tinha cepa e grandes
bodas. Hector chamou à porta e falou: “Abri!” E o senhor veio à frente e falou
para ele: “Senhor cavaleiro, outro albergue deveis procurar que não este,
quando aqui não deve entrar ninguém que seja tão alto a cavalgar como sois”. E
seguiu de lá de pronto, e tão triste como nenhum homem e voltou ao seu assento
que tinha deixado, e estava tão irado que acordou da ira. Então
começou a se virar e a voltar tanto que não conseguiu dormir. E o senhor Gawin
não adormeceu, quando estava desperto por causa de seu sonho. Pois que ouviu
que Hector se virava, então falou para ele: “Senhor, dormis?” “Senhor,” falou
ele, “não, quando agora me acordou um sonho maravilhoso que me veio à frente no
meu sono”. “Seguramente”, falou o senhor Gawin, “assim também posso dizer,
quando me veio um sonho a toda vez maravilhoso, do qual estou desperto, do qual
nunca fico feliz, antes que eu saiba a verdade”. “Com direito vos digo”, falou
Hector, “quando nunca fico feliz, antes de saber a verdade do meu senhor,
senhor Lancelot, meu irmão”.

No que falavam, então viram vir através da porta da capela uma mão, que trazia
uma cabeça que lá estava coberta com um escarlate. Na mão pendia um arreio não
muito rico e havia na mão uma grande vela que lá queimava muito, que foi
perante eles e se colocou em um candelabro, e desapareceu assim que não
souberam de onde vinha. Na hora ouviram uma voz que lhes dizia: “Cavaleiros de
má verdade e fé débil, estas três coisas, que vistes, faltam-vos, e por causa
disto não podeis vir à aventura do Santo Graal”. E pois que ouviram
esta fala, então ficaram temerosos. E pois que longamente se tinham calado,
então falou o senhor Gawin em primeiro lugar e falou: “Senhor Hector,
entendestes essa fala?” “Seguramente, senhor,” falou ele, “eu não, mas bem a
peguei”. Então falou o senhor Gawin: “Nesta noite vimos tanto que me parece que
o melhor que fazemos é procurar um bom homem, um eremita, que nos diga o
significado do que ouvimos e vimos. E depois que ele nos der conselho, nós
fazemos; quando assim desempenhamos nosso tempo com perda como até aqui
fizemos”. E Hector falou que lhe parecia bem. Assim ainda depois, quando
estavam despertos, nunca adormeceram, quando cada um deles muito pensou no que
tinha visto em seu sono.

Pois que veio o dia, então foram contemplar onde os seus cavalos estariam, e os
procuraram por tanto tempo até que os encontraram, e os prepararam e tomaram
suas armas e montaram e se apartaram da montanha. E pois que vieram ao vale,
então os encontrou um servo, que cavalga um cavalo e estava sem companhia.
Então eles o saudaram e ele ou saudou de volta. “Caro amigo,” falou o senhor
Gawin, “conseguis nos indicar algum enclausurado ou enclausurada perto daqui?”
“Sim, senhor,” falou o servo. Então lhes mostrou uma via à mão direita e falou:
``este caminho vos conduz a uma clausura, que está em uma pequena montanha; mas
nenhum cavalo consegue lá subir, quando ele vai; lá encontrais um enclausurado
que é o melhor homem e de vida mais santa como ele é nesta terra”. “Então
te\footnote{ O texto original  novamente altera os pronomes de tratamento, aqui
indicando, provavelmente, a hierarquia social dos falantes, vale afirmar, dos
cavaleiros em relação ao servo. Nos outros casos, os pronomes se alternam entre
falantes de um mesmo estrato social.}  encomendamos a Deus”, falou o
senhor Gawin, “quando muito bem nos serviste para nossa vontade, da fala que
nos disseste”.  O servo cavalgou para um lado, e eles seguiram para o
outro. E pois que se tinham longamente deslocado, então um cavaleiro os
encontrou na floresta, bem armado, que os chamou ao fecho tão longe quanto os
viu. “Em nome de Deus,” falou o senhor Gawin, “desde que rumei de vós de
Camelot, nunca encontrei nenhum que me chamasse a espetar, e porque este o
chama, assim deve tê-lo”. “Senhor,” falou Hector, “deixai-me ir para lá, se for
vossa vontade”. “Eu não o faço: é que ele me espeta, então não me é mal que
rumeis para longe de mim”. Na hora ele afunda a lança e toma seu escudo à sua
frente, e deixou-se correr para o cavaleiro. E ele lhe veio com tão grande
poder quanto ele pôde com seu cavalo. E espetaram um ao outro  tão duramente
que o escudo foi levado e a coifa se quebrou, e se feriram muito duramente, um
mais que o outro. O senhor Gawin estava ferido do lado esquerdo, quando não era
muito. E o cavaleiro estava muito ferido, quando se via a lança sair pelo outro
lado, e ambos despencaram. E com a queda quebrou-se a lança, tal que o
cavaleiro lá permaneceu caído e não conseguiu se erguer da terra. 

Pois que o senhor Gawin viu-se caído sobre a terra, então se levantou bem logo e
rápido e tomou sua espada e seu escudo à sua frente e fez certo, porque queria
lhe mostrar a maior nobreza que jamais poderia, como nele havia suficiente.
Pois que viu que o cavaleiro não se levantava, então bem pensou que ele estaria
ferido de morte. Então falou: “Ah, Senhor, eu estou morto, sabei-o
seguramente, e por causa disso fazei o que vos peço”. Ele falou que o faria de
bom grado, se conseguisse fazê-lo de alguma maneira. “Senhor,” falou ele, “eu
vos peço que me conduzis a um convento que eu sei, que não faz longe daqui e
que é perto daqui, e me façais como se deve fazer a um cavaleiro”. “Senhor,”
falou o senhor Gawin, “eu não sei aqui perto nenhuma casa de Deus”. “Ah,
Senhor, erguei-me ao vosso cavalo e eu vos dirijo a um convento que eu sei, que
não é longe daqui”. Então o senhor Gawin o coloca à sua frente sobre seu cavalo
e deu a Hector seu escudo para conduzir e o agarrou por causa de que o corpo
não caísse. E o cavaleiro dirige o cavalo a uma abadia, que lá estava próxima
em um vale.

Pois que vieram à porta, então bateram tanto que os que estavam dentro os
ouviram e vieram, e se os deixou entrar e se os recebeu bem. Então assentaram o
cavaleiro ferido e lhe fizeram o melhor que conseguiram. E tão logo eles o
dirigiram para a cama, então pediu que se lhe trouxesse seu Criador. E então
começou a gritar muito seriamente e dirigiu suas mãos juntas e fez sua
confissão e deixou-os todos ouvir, que lá estavam, os pecados de que se sabia
culpado contra seu Criador e pediu graças gritando interiormente. Pois que o
falou o que sabia por pensamento, então o padre lhe deu seu Criador, e ele o
tomou humildemente. Pois que usou o corpo de Nosso Senhor, então pediu ao
senhor Gawin que lhe tirasse a lança do peito. E ele lhe pergunta quem ele
seria e de qual terra. “Senhor,” falou ele, “eu sou da corte do rei Arthur e
companheiro da Távola Redonda e sou chamado Ywan e filho do rei Urgins, e vim
com meus outros companheiros à demanda do Santo Graal. Quando veio por força de
meu pecado e imposição de Nosso Senhor que me matastes, eu vos perdoo
altamente, como Deus vos deve fazer”.

Pois que o senhor Gawin ouviu isto, então falou muito triste e iradamente: “Ah,
Deus, quão grandes são minhas aventuras, ah, Ywan, como me lamento por vós!”
“Senhor,” falou ele, “quem sois vós?” “Eu sou”, ele falou, “Gawin, sobrinho do
rei Arthur”. “Assim não lamento que fui morto por uma tão nobre cavaleiro como
sois! Por Deus, como fordes à corte, assim saudais por mim todos os nossos
companheiros, que encontrares vivos, quando sei bem que suficientes de vós
morrerão nesta demanda. E dizei-lhes por causa da irmandade que há entre mim e
eles, que sempre me recordem em suas orações e que peçam a Nosso Senhor Deus
que se apiede da minha alma”. Então começaram a gritar, o senhor Gawin e
Hector. E então tomou o ferro da lança, que Ywan tinha espetado através de seu
coração, tirou-lhe, que Ywan se repuxava da grande dor que tinha, e assim de
pronto apartou-se a alma do corpo sob a mão de Hector. Então lhe fizeram
sepultar para valer e ricamente em um lençol de seda, que os irmãos lá deram do
convento, pois souberam que ele era filho de um rei. Então lhe fizeram
um tal serviço como se costuma fazer aos mortos e sepultados à frente do alto
altar e lhe fizeram pôr por cima um caixão e fizeram nele escrever seu nome e o
daquele que o tinha matado. Então seguiram daqui, o senhor Gawin e Hector,
tristes e irados da aventura que lhes sucedeu, quando bem viram que era
justamente um acidente. E cavalgaram por tanto tempo até que vieram à montanha
abaixo, à clausura. E pois que para lá vieram, então ataram ambos seus cavalos
a duas árvores. Então foram por um atalho estreito, que lá ia para uma outra
clausura, e acharam a montanha tão árdua para se subir que era maravilha e
ficaram muito esgotados antes que viessem lá. E pois que tinham subido, então
viram a clausura onde o bom homem morava dentro, que lá era chamado Nasiens. E
era uma pequena casa e uma pequena capela. Pois vieram lá e viram em um jardim,
que estava junto à capela, um velho nobre quebrar urtigas para que comesse,
como aquele que não teve outra refeição por muitos tempos. Tão logo quanto ele
os viu, tão bem armados, então pensou que eles seriam os cavaleiros errantes
que lá vieram à demanda do santo Graal, de que há muito tinha sabido. Então
veio até eles e os saudou. E volveram para ele e o saudaram, e ele os saudou de
volta e falou: “Caros Senhores, que aventura vos portou para cá?” “Senhor,”
falou o senhor Gawin, “o grande desejo que tivemos de cavalgar até vós e nos
significar do que estamos em dúvida”. Pois que ele ouviu o senhor Gawin assim
falar, então pensou que ele seria muito sábio de coisas terrenas. Então falou
para ele: “Senhor, não vos falho de nenhuma coisa que sei”, e os tomou e os
dirigiu ambos à capela e lhes pergunta quem eles seriam. E se nomearam e se
fizeram confessar, assim que bem soubesse quem seria cada um. Então lhes
pergunta para que lhe digam de que eles estariam em dúvida, e ele queria
aconselhá-los conforme seu poder. E o senhor Gawin lhe diz de pronto.

“Senhor, aconteceu-me ontem e a meus companheiros, que cavalgávamos através de
um ermo sem aventura o dia inteiro, nem ninguém nos encontrou, nem homem nem
mulher, por tanto tempo que encontramos sobre uma montanha uma capela. Então
rumamos para lá, quando ficaríamos melhor lá que, pois, no ermo. E pois que
para lá viemos, então nos desarmamos e adormecemos, então veio-me à frente uma
maravilhosa aventura”, e lhe contou. Pois que a tinha contado, também Hector
lhe contou a sua. Então lhe contou da mão que tinham visto acordados e da fala
que lhes foi dita, E pois que lhe tinham contado tudo isso, então pediram a ele
que lhes dissesse o significado, quando sem grande significado não lhes teria
vindo à frente em seu sono. Pois que o bom homem ouviu tudo por que eles vieram
até ele, então respondeu: “Senhor Gawin, caro senhor, no gramado, onde vistes
que a manjedoura estava, por isto devemos entender a Távola Redonda. Quando
como na manjedoura estão as árvores para estábulo, que lá separam os lugares,
assim é na Távola Redonda distinto, que se separam as cadeiras umas das outras.
Pelo gramado devemos entender humildade e paciência, que por toda a via são
fortes em vós. E por causa de que a humildade não pode ser vencida nem a
paciência, então a Távola Redonda ficou confirmada, pois a cavalaria é desde
então instruída por causa da doçura e por causa da irmandade que lá havia entre
eles, não pode ser vencida. E por causa disto se diz que foi confirmada na
humildade e na paciência. Na manjedoura comiam, de outra parte, cem
touros. Eles comiam, porém não estavam no gramado: quando se tivesse lá estado,
seus corações teriam ficado na humildade e na paciência. Os touros eram
corteses e vaidosos à vez, sem três. Pelos touros deves\footnote{ Novamente a
alteração pronominal em um mesmo discurso.}  entender os
companheiros da Távola Redonda, que lá por causa da falta de castidade e
através da cortesia caíram em pecado mortal tão duramente que seus pecados não
podem permanecer em alguém. Quando parecem externa e internamente assim que são
maculados, horrendos e maus como os touros eram, exceto os dois que lá eram
brancos e belos e o terceiro, que lá tinha tido um sinal de uma mancha. Os dois
que lá eram brancos e belos significam Galaat e Parsifal, que lá são muito
mais belos e mais brancos que qualquer outro, quando são repletos de poder em
todas as virtudes e cheios de pureza; depois não se pode encontrar nenhum que
não tenha uma mancha. O terceiro que lá tem um sinal de uma mancha  era Bohort,
que em algum momento errou em sua castidade. Quando desde então tão bem fez em
sua pureza que todo o mau ato está em tudo perdoado. Os três touros, que lá
estavam atados com os pescoços, são os três cavaleiros nos quais a pureza é tão
fortemente enraizada, que não têm poder para erguer suas cabeças, isto é tanto
dito que eles não podem pecar nem nenhum pecado pode adentrá-los. Os touros
falaram: ‘Vamos procurar melhor pastagem do que seja esta’. Isto é que os
cavaleiros falaram do dia de Pentecostes: ‘Nós queremos ir à demanda do Santo
Graal, assim seremos satisfeitos da honra do mundo e das iguarias que o Santo
Graal nos envia àqueles que sentam à Távola Redonda do santo Graal, que é a boa
pastagem. Queremos deixar esta e rumar para lá’. Então se apartaram da corte,
então seguiram sem confissão como deveriam e não se colocaram no serviço de
Nosso Senhor nem em paciência, que lá são provados com o gramado, quando
seguiram sobre o campo, que é sobre o caminho onde não cresce flor ou fruto, é
para saber, no caminho do inferno. É no caminho onde todas as coisas estão
perdidas que lá não são reconfortantes. Pois que voltaram para cá, então muitos
lhes faltavam, que é falado que eles não retornaram, quando de vós morreu uma
boa parte. E os que lá voltaram estavam tão magros e tão cansados que bem não
conseguiam ficar de pé. É para entender: aqueles, que para lá voltaram, devem
assim estar preocupados com pecados mortais, e um deve ter matado o outro, que
não devem ter nenhum membro que consigam mover. É para entender: não devem ter
em si nenhuma virtude que possa conservar o homem, isto é à frente da queda no
inferno, e devem ser mantidos com toda a mácula e todo o pecado mortal. Os
outros três sem manchas, um deve voltar e os outros dois devem permanecer. É
para entender que desses três cavaleiros um deve voltar à corte, e não por causa
da refeição, mas para dar mostra da boa pastagem, que aqueles perderam, que lá
estão com pecados mortais. E os outros dois permaneceram, quando encontram
tanta doçura na refeição do Santo Graal, que não nos conhecem. A última fala”,
falou ele, “de Nosso Senhor,  eu não vos digo, quando seria uma coisa de que
nenhum bem vos aconteceria, quando se pudesse vos inverter”. Então falou o
senhor Gawin: assim quero bem carecer disso, pois que é vossa vontade, e deve
assim se fazer simplesmente. Quando bem me informastes do que eu estava em
dúvida, quando vedes claramente a verdade de meu sono”. 

Então falou o bom homem para Hector e lhe disse: “Hector, parece-vos que vós e
Lancelot vos levantastes de uma poltrona, que lá significava maestria ou
suserania. A poltrona, sobre a qual vos sentáveis, caracteriza o grande amor e
a grande honra que se vos fez na Távola Redonda, que deixastes pois que vos
apartastes da corte do rei Arthur. Sentáveis-vos sobre dois grandes corcéis,
isto é, em grande coragem e cortesia, que são dois cavalos do Inimigo. E depois
bem falais: ‘nós procuramos o que nunca encontramos’, isto é o Santo Graal, a
coisa secreta de Nosso Senhor, que nunca vos é provada, quando não sois dignos
de contemplá-la. Pois que estavam separados um do outro, Lancelot cavalgou por
tanto tempo que caiu de seu cavalo, é para entender que ele se afastou da
cortesia e caiu em humildade, para a qual o trouxe Nosso Senhor Jesus Cristo.
Também o moveu de pecado, assim que ele se viu nu de boas virtudes, que um
cristão deve ter, e ele pediu graça. Então Nosso Senhor Deus o vestiu
de novo. Com o que? Com paciência e humildade. Essa foi a saia que Ele lhe deu,
que estava cheia de espinhos. Depois o assentou sobre um burro, que é a
humildade. Foi claramente uma coisa que Nosso Senhor o dirigiu, pois que veio à
cidade de Jerusalém, que lá era um rei e o rei tinha toda a riqueza em sua mão.
Porém não queria vir sobre nenhum corcel nem sobre nenhum palafrém, quando veio
sobre o animal mais simplório e sobre o mais grosseiro, que é sobre o burro,
pela vontade de que o pobre e o rico tomem um exemplo disto. Assim vistes
Lancelot cavalgar em vosso sonho sobre um burro. Pois que ele tinha cavalgado
por um bom tempo, então veio a uma fonte, a mais bela que jamais viu, e se
sentou e quis beber. E pois que se tinha curvado, então desapareceu a fonte,
que não a viu e a conseguiu ter. Então voltou a sentar-se em sua poltrona, da
qual tinha vindo. A fonte está no gramado que não se pode esgotá-la, quantas
pessoas vierem a ela, quando é o Santo Graal,  que é a graça do Espírito Santo.
A fonte é a doce chuva, a doce fala do Evangelho, onde o coração do verdadeiro
penitente encontra a grande doçura, quando quem deseja seu sabor, deseja sempre
mais no futuro: é a graça do Santo Graal. Quando tanto quanto é mais largo e
fundo, assim mais permanece lá. E por causa disso por direito deve ser chamado
como fonte.  

Pois que ele veio à fonte, então se assentou. É para entender: quando ele veio
do Santo Graal, então deve se sentar e não deve se considerar um homem, por
causa de ter caído em pecado. Como se curvou, é falado: como ele se sentou
sobre seu joelho para beber e que ele foi recebido e preenchido com grande
graça, então se perdeu a fonte, que é o Santo Graal. Quando ele perdeu a visão
dos olhos perante o Santo Graal por causa de ser permitido ver a primeira
impureza, e perdeu o poder do corpo. É para entender que ele assim serviu ao
Inimigo, e a vingança dura quatorze dias, que ele não deve comer nem beber nem
falar nem mover pé nem mão nem nenhum membro que ele tem. Quando bem deve
parecer-lhe que por toda via estivesse em tão boa coisa quanto a que estava
quando perdeu a visão dos olhos. Depois deve bem dizer uma parte daquilo que
viu. Então se separa das terras e ruma para Camelot, e vós, e deve por toda a
via cavalgar os grandes corcéis, que é tão falado que deveis por toda a via
viver em pecado mortal e em cortesia e inveja. E deveis de muitas maneiras
rumar errantes, aqui e ali, por tanto tempo até que vindes à casa do rico Rei
Pescador, que os verdadeiros cavaleiros devem ter suas bodas do alto achado que
devem ter encontrado. E quando para lá fordes e lá considerardes entrar, o rei
vos deve dizer que não quer se acercar mais de nenhum homem que seja tomado tão
alto como vós sois, isto é assim para entender: aquele que lá estiver em pecado
mortal e em cortesia. E quando o ouvires, então volveis para Kamelot,\footnote{
O texto original apresenta a variante \textit{Kamelot} nesta passagem, ao invés
de \textit{Camelot}.} e pouco deveis tr criado de útil nesta demanda. 

 Então vos disse e contei uma parte daquilo que deve vos suceder. Então
precisais saber da mão que vistes perante vós aqui caminhar e lá carregar uma
vela e um arreio. Então vos diz a mão que essas três coisas vos faltam. Pela da
mão que vistes devemos entender pureza e pelo vermelho a graça do Espírito
Santo, de que toda a pureza é abraçada. E aquele que lá tem pureza em si é
quente e fonte da graça de Nosso Senhor. Pelo arreio deveis entender
abstinência, quando como o homem guia e dirige um cavalo para onde quer com o
arreio, por certo é da abstinência que não pode cair em pecado mortal nem ir
contra sua vontade, que não seja em boa obra. E por causa da vela que trouxe em
sua mão deveis entender a verdade do Evangelho, que é Jesus Cristo, que lá dá
visão a todos os que se lançam para fora do pecado e vêm para o caminho de
Nosso Senhor Jesus Cristo. Pois que foi visto que a verdade e a abstinência
vieram perante vós na capela, é falado que Nosso Senhor Deus veio à Sua casa em
Sua capela, que Ele não tinha construído por causa da vontade de que a verdade
fosse pregada. E pois que Ele vos viu, então Se levantou por causa do lugar que
Ele tinha separado com vosso repouso. E pois que foi adiante, então falou:
‘Vós, cavaleiros cheios de impuras verdades e má fé, estas três coisas vos
faltam: pureza, verdade, abstinência; e por causa disto não podeis vir à
aventura do Santo Graal’. Então entendestes o significado da mão”. 
Então falou o senhor Gawin: “Em verdade, bem nos contastes, quando
bem vejo claramente.

Então vos peço que nos digais por que causa não encontramos tanta aventura como
costumávamos”. “Eu vos digo”, falou o bom homem, “como é. As aventuras que então
sucedem são o significado e a demonstração do Santo Graal, e os sinais do Santo
Graal não aparecem para nenhum pecador, nem para nenhum homem abraçado por
pecados, por isso não vos aparecem, quando sois por demais infiéis e grandes
pecadores. Por causa disso deveis considerar que esta é aventura do Santo
Graal, não para matar pessoas e assassinar cavaleiros, quando são de coisas
espirituais que lá são maiores e melhores o suficiente”. “Senhor,”
falou o senhor Gawin, “por causa desta fala que me dizeis, parece-me, tanto
quanto estamos em pecados mortais, que em vão viajamos nesta demanda, quando
nós não a conseguiremos”. Então falou o bom homem: “Em verdade, dizeis o
correto, quando sois suficientes os que nunca ganharão senão vergonha”.
“Senhor,” falou Hector, “é que vos seguimos, voltamos para Camelot”. “Isto
aconselho,” falou o bom homem, “ainda pois vos digo bem,  tanto quanto estais
em pecados mortais, assim nunca fareis nenhuma coisa de que ganheis”. Pois que
falou esta fala, então se apartaram de lá. E quando estavam longe dele adiante,
então chamou o bom homem: “Senhor Gawin,” falou ele, “há muito que te\footnote{
Mais uma vez a alteração de pronomes de tratamento, aqui do modo formal para o
informal.}  tornaste cavaleiro e nunca desde então serviste ao teu
Criador, que não fosse muito pouco. Tu és a velha árvore, que em si nunca tem
folhagem nem fruto. Ainda reflete que para Nosso Senhor ficam a ovelha e a
alma, desde que o Inimigo tem as flores e os frutos”. “Senhor,” falou
o senhor Gawin, “tivesse eu a hora para falar convosco, eu falaria mais
longamente convosco, quando meus companheiros me esperam lá abaixo no rochedo,
por causa disso também preciso ir, quando quero voltar para vós tão logo eu
possa, quando tenho grande desejo de falar convosco em muito segredo”. Então se
separaram um do outro. Então desceram a montanha e vieram até seus cavalos e
montaram e cavalgaram até à noite e vieram até a casa de um homem da floresta
que os alberga e lhes faz muito bem. No outro dia seguiram de lá e cavalgaram
seu caminho e cavalgaram por longo tempo sem encontrar aventura que fosse de se
contar. Então se cala sobre eles e se volta para Bohort de Gan\u{u}e. 

\chapter{Uma luta de irmãos}

\textsc{Aqui nos diz} a aventura que, pois que Bohort estava separado de Lancelot, assim
como nos diz a fala, que ele cavalga até as nonas horas. Então seguiu
ele um bom homem de boa idade, vestido com roupas espirituais, e ele cavalga um
burro. Ele não tem consigo nem escudeiro nem servo, nem nenhuma companhia.
Então Bohort o saúda e fala: “Senhor, Deus vos escolte!” E ele o contempla e
confessa que ele dos cavaleiros errantes era um. E lhe respondeu que Deus o
pagasse. E então lhe pergunta Bohort de onde ele vinha assim sozinho. “Eu
venho”, falou ele, “e vi um doente, um de meus servos, que me costumava ir em
minha empreitada”. E ele falou: “Senhor, quem sois vós e aonde rumais?” “Eu
sou”, ele falou, “um cavaleiro errante e vim a uma demanda que eu queria de bom
grado que Nosso Senhor me soubesse, quando é a mais alta demanda que jamais foi
assumida. É a demanda do Santo Graal, onde deve ter tanta honra aquele que
possa conduzi-la ao fim, que o coração de nenhuma pessoa mortal não a consiga
contar nem imaginar. “Seguramente”, falou o bom homem, “dizeis certo, quando
grande honra ele deve ter, isto não é maravilha, quando deve ser o mais nobre e
verdadeiro servo de toda a demanda. Não vem à demanda aquele que lá é
impuro e enganado como os infiéis são, eles vêm então em melhora de suas vidas,
quando é o serviço de Nosso Senhor. Então vede como sois tolo, quando bem
sabeis e ouviste fartamente dizer que ninguém veio a seu Criador, que não tenha
vindo pela porta da pureza, que é com confissão; que ninguém pode se tornar
puro quando com alta confissão, quando com a confissão se expulsa o Inimigo.
Quando o cavaleiro ou o homem, quem ele seja, que lá faz pecado mortal, recebe
o Inimigo e está nele. Quando não consegue expulsá-lo, tenha estado com ele dez
ou vinte anos, confesse então primeiro seus pecados. Quando com a confissão
joga de si o Inimigo e o empurra fora de seu corpo e abriga um outro, de que
ele tem grande honra, que é Jesus Cristo, o verdadeiro Salvador. Então está
preparado para a nobre cavalaria, que é do corpo. Então Ele te
obrigou e renovou muito claramente do que Ele fosse. Quando para Ele está
pronta a refeição do Santo Graal, que lá é um repouso da alma e uma emancipação
do corpo, de que Ele fruiu e com a qual por tanto tempo conteve o povo de
Israel no deserto. E Ele é suave perante eles, quando lhes prometeu apenas
ouro, onde costumavam tomar chumbo. Quando assim a refeição terrena se
converteu em celestial, assim deve ser certo que aqueles que foram terrenos até
esta hora, é para entender: os que até esta hora foram pecadores serão
convertidos de coisas terrenas em obra celeste e deixam seus pecados e sua
impureza e vêm à confissão e ao reconhecimento e são cavaleiros de Nosso Senhor
e portam Seu escudo, que é paciência e humildade. Quando tal escudo
Ele portou contra Seu Inimigo onde Ele venceu na Cruz, pois que dirigiu a morte
por causa de redimir Seus cavaleiros da morte dos infernos e da escuridão em
que estavam. Através da porta que se chama confissão, de outra maneira ninguém
consegue vir a Nosso Senhor Jesus Cristo e precisa estar nesta demanda e
converte o ser de cada um e transforma de novo a refeição que lhe é
transformada. E quem quiser entrar através de outra porta, é para entender que
precisa se trabalhar muito e precisa também primeiro se confessar, de outra
maneira não descobre nunca nenhuma coisa que busca. Também não vem nem para
saborear nem para provar desta refeição, que lhe foi prometida, nem acontece
nenhuma outra coisa. Quando por causa de que devem sentar-se nos lugares dos
cavaleiros celestiais, e eles não são deles, é que devem se conter dos
companheiros da demanda, e eles não são dela. Quando são impuros e maus, e
devem cair um em falta de castidade, outros em adultério. Com isto servem ao
diabo, que lhes dá a paga que ele costuma dar, que é vergonha e desonra, de que
são preenchidos antes de saírem da demanda.

Senhor cavaleiro, isto tudo vos disse por causa de que viestes à demanda do
Santo Graal, quando vos não vos aconselho de maneira nenhuma que trabalheis
mais nessa demanda, não sois como é reto e deveríeis ser por causa do direito”. 
“Nessa demanda”, falou Bohort, “parece-me em direito que me
dizeis o que devem ser todos os companheiros, nenhum vos é de agrado. Quando
fora de dúvida, parece-me que estais em tão alto serviço como é este, que é o
serviço de Nosso Senhor Jesus Cristo. Quando ninguém deve entrar, que não tenha
se confessado, quando quem de outro modo entrasse, não creio que se lhe pudesse
ir bem, que seria uma maravilha de tão grande maravilha como é esta”. “Dizeis
certo”, falou o bom homem. Então Bohort lhe pergunta se ele era padre. “Sou
sim”, falou o bom homem. “Então vos exorto  em nome do santo amor que me deis
conselho, como o bom homem a seu filho; isto é, o pecador que vem à confissão.
Quando o padre está em lugar de Nosso Senhor Jesus Cristo, que lá é pai de
todos aqueles que n’Ele creem. Por causa disto vos peço, caro Senhor, assim me
deis conselho assim para o proveito de minha alma e para a honra da cavalaria”. 
“Em verdade”, falou o bom homem, “vós me chamais uma grande coisa; e
se eu nisto vos faltasse e depois caísseis em pecado mortal ou em dúvida,
poderíeis, no Dia do Juízo Final, atribuir-me perante o rosto de Deus, por
causa disso devo vos aconselhar o melhor que posso”. Então lhe
perguntou como se chamava. E ele respondeu que se chamava Bohort de Gan\u{u}e e
era filho do rei Bohort e sobrinho de meu senhor,  o senhor Lancelot
do Lago. O bom homem ouviu esta fala,  então falou: “Seguramente Bohort, é que
a fala do Evangelho é a vós conservada, deveis ser um bom cavaleiro e
verdadeiro. Quando como Nosso Senhor falou ‘a boa árvore carrega bons frutos’,
e vós sois o fruto de muito boa árvore, quando vosso pai, o rei Bohort, foi o
melhor cavaleiro que já vi e era muito misericordioso e humilde. E a mãe, a
rainha, era uma das melhores mulheres que já vi em longo tempo. Os dois eram
uma única árvore e única carne com justa honra, e porque sois o fruto, deveis
simplesmente ser bom, porque a árvore, da qual saístes, era boa”. 

“Senhor,” falou Bohort, “quando o homem vem de árvores más, é para dizer de mau
pai e má mãe, e ele se converte de amargura em doçura tão logo ele receba a
santa cristandade e a santa fé, por causa disso me parece que não vai conforme
o pai e conforme a mãe que ele seja bom ou mau, quando é conforme o coração do
homem. Quando o coração do homem é como uma vela em um navio, que o vento
conduz para onde quer, para ter ou para perder”. Não é a vela, senão o mestre
que a mantém e a domina e a faz virar para qual fim queira. Quando o que de bom
ele faz com ela, vem da graça do Espírito Santo, e o que com ela faz de mau,
vem do engodo do Inimigo”. 

Falaram o suficiente sobre estas coisas entre eles dois, por tanto tempo que
viram a casa de um eremita. O bom homem seguiu para lá e falou a Bohort que o
seguisse, quando queria lá se albergar, “e pois amanhã cedo quero vos dizer em
segredo daquilo de que me pedistes conselho”. Bohort o seguiu por isto de muito
bom grado. E pois que estavam lá, então desmontaram e encontraram lá dentro um
aluno que lá tirou a sela e o arreio do cavalo de Bohort e o ajudou a
desarmar-se. E pois que estava desarmado, então falou o bom homem se queria ir
ouvir as vésperas, e ele respondeu “sim” e foi com ele à capela. E então aquele
principiou as vésperas e pois que tinha cantado, então chamou a pôr a mesa e
lhes deu água e pão e falou: ``Com tal refeição devem os cavaleiros celestes
alimentar seu corpo, quando refeições muito saborosas trazem o homem à falta de
castidade e aos pecados mortais. Assim Deus me ajude, se vos parece
que quereis fazer algo por causa de mim, eu queria vos pedir”. E Bohort
perguntou o que seria. “É uma coisa que deve vos ajudar para a alma e deve bem
conservar-vos o corpo suficientemente”. E ele lhe promete que queria fazê-lo.
“Grande graça,” falou o bom homem, “sabeis o que me prometestes? Que não
devereis provar nenhuma outra refeição além de água e pão, até que venhais à
Távola do Santo Graal”. “E o que sabeis,” falou Bohort, “se venho
lá?” “Eu bem sei que deveis lá ir três companheiros da Távola Redonda.” “Assim
vos devo prometer como um bom cavaleiro que o vou manter, até que me sente à
mesa da qual falais.” E o bom homem lhe agradece desta abstinência,
que quisesse fazê-lo pelo amor do verdadeiro Deus.

À noite Bohort deitou-se sobre a grama verde, que o aluno tinha cortado junto à
capela. E de manhã, tão logo se fez dia, então se levantou, e então veio o bom
homem até ele e falou: “Senhor, tomais aqui uma saia branca, que deveis vestir
no lugar de uma camisa, que deve ser um sinal da penitência e uma mortificação
da carne”. E ele tirou suas vestes e sua camisa e a colocou em tal
confissão como ele o chamava. E depois vestiu por cima uma saia escarlate. E
depois ele faz uma cruz à frente de sua testa e foi à capela até o bom homem e
fez-lhe sua confissão de todos os pecados de que se sabia culpado contra seu
Criador. Então o achou de tão boa vida que o maravilhou, e achou que ele nunca
fez nenhum erro em nenhuma falta de castidade, sem ser ao tempo em que ele
ganhou Heliam, o Branco. E desde então está obrigado a agradecer muito a Nosso
Senhor Deus. Pois que a ele o bom homem tinha absolvido e lhe deu tal
penitência como lhe pareceu que ele precisasse, então lhe pediu Bohort que ele
lhe desse seu Criador, que assim estivesse certo do fim a que iria, quando não
sabia se deveria morrer nesta demanda ou não. E o bom homem lhe pediu que
ficasse até que cantasse missa. E ele falou que o queria fazer. Então começou o
bom homem a rezar. E quando tinha cantado, então se vestiu e principiou missa.
E pois que tinha dado bênção, então tomou o Corpo de Nosso Senhor e acenou para
Bohort que viesse à frente, e ele o fez e ajoelhou-se à frente do altar. E pois
que o fez, então falou o bom homem para Bohort: “Vês\footnote{ Mais uma
costumeira alteração pronominal.}  o que aqui seguro?” “Sim,”
falou ele, “eu vejo que segurais meu Criador e meu Redentor em igualdade ao
pão; no modo não o vejo, quando meus olhos são tão escuros que não podem ver a
coisa espiritual, e não me deixam ver de outra forma e me tomam a verdadeira
igualdade; quando de outro modo não devo duvidar que seja verdadeira carne e
verdadeiro sangue e verdadeiro homem”. Então principiou a chorar tão
duramente. Então falou o bom homem: “Então é simplesmente que O sirvas todo o
teu tempo em que viveres”. “Senhor,” falou Bohort, ``tanto tempo quanto eu viver,
não devo estar de outra maneira senão Seu servo e não quero nunca sair de Seu
mandamento”. Então o bom homem lhe deu Nosso Senhor, e ele O recebeu
com grande devoção e estava tão contente e tão bem animado que era maravilha. E
pois que o tinha usado, e ajoelhado por tanto tempo quanto lhe aprouve, então
foi até o bom homem e lhe diz que queria de lá se apartar, quando lá tinha
estado por longo tempo. E o bom homem falou que ele podia se separar de lá
quando quisesse. Quando estava armado como os cavaleiros celestes devem estar e
tão bem alertado contra o Inimigo, que ninguém poderia estar melhor. E então
foi para suas armas e as tomou. E pois que estava armado, então se apartou de
lá e encomendou o bom homem a Nosso Senhor. E lhe pediu que por ele rezasse
quando viesse perante o Santo Graal. E Bohort lhe pediu por Deus que por ele
rezasse, que não caísse em pecado mortal por tentação do Inimigo. E o bom homem
respondeu a ele que sempre queria pensar nele de todas as maneiras que pudesse.
De pronto Bohort se apartou de lá e cavalgou o dia inteiro até as nonas horas.
Quando era depois das nonas, então viu de novo montanha no ar e viu um grande
pássaro voar sobre uma árvore, que era velha, sem folhagem e sem frutos. E pois
que tinha voado bem alto, então se sentou sobre a árvore, e lá em cima ele
tinha muitos pequenos passarinhos, e estavam todos mortos. E pois que ele se
sentou sobre eles e os encontrou sem vida, então se golpeou com seu bico bem no
meio, atravessando seu peito, que o sangue jorrou para fora. E tão logo
sentiram o sangue quente, então voltaram à vida, e ele morreu entre eles. E
assim receberam o início de sua vida com o sangue do grande pássaro.
Pois que Bohort viu a aventura, então muito se maravilhou como podia ser,
quando não sabia que coisas lhe podiam acontecer deste significado. Quando bem
lhe pareceu que seria um significado maravilhoso. Então contemplou por longo
momento se o grande pássaro não se levantava de novo, e isto não podia ser,
quando estava morto. E pois que o viu, pôs-se de novo em seu caminho e cavalgou
até depois das vésperas. À noite aconteceu que, como a aventura o conduzia, que
ele veio a uma torre alta e robusta, onde desejou abrigar-se, e se lhe deu de
bom grado. E quando estava desarmado, então se o conduziu a um grande salão,
que lá dentro encontraram as mulheres do castelo, que era bela e jovem e estava
honradamente vestida.\footnote{ O texto, abruptamente, adjetiva
“as mulheres” no singular, de tal modo que, na tradução de Hans-Hugo Steinhoff,
a expressão \textit{die frauwen von der burg}  é
traduzida pelo vocábulo “castelã” no singular. Em nossa
tradução, optamos por conservar a inconsistência do texto original.} E
pois que ela viu Bohort entrar, então o chamou que fosse bem-vindo, e ela o
saúda como uma mulher. Ele o agradece com grande alegria. Depois ela o fez
sentar-se junto de si e lhe fez maravilhosamente grande honra e alegria.

Pois que era tempo de comer, então fez Bohort sentar-se junto dela e os que lá
dentro estavam trouxeram grandes quantidades de carne e as assentaram sobre a
mesa. E quando ele o viu, então pensou que não deveria comê-lo. Então ele
chamou um servo e disse para ele que lhe trouxesse água, e ele o fez e lhe
trouxe em um copo de prata, e o assentou à frente dele. E quando a mulher o
viu, então falou: “Senhor, não vos satisfaz esta comida que se trouxe à vossa
frente?” “Senhora,” falou ele, “bem, porém não como esta noite senão aquilo
que vedes”. E então ela deixou a conversa como aquele que não se atrevia a
fazer o que lhe fosse mau. Pois que os do castelo tinham comido e as mesas
estavam tiradas, então se levantaram e foram para a janela do palácio, então
Bohort viu, próximo, as mulheres. Então entrou um servo, que então falou para a
senhora: “Senhora, vai-vos muito mal, vossa irmã ganhou vosso castelo, e todos
os que lá estavam por vossa graça são prisioneiros. E vos comunica que não vos
quer deixar senão um pedaço ruim de todas as vossas terras, se até amanhã não tiverdes
achado um cavaleiro que lute por vós contra Priaden, o Negro, que é seu
senhor”. Pois que a mulher ouviu esta fala, então começou a colocar grande
lamento e falou: “Ah Senhor Deus, por que causa não me autorizas a conservar
para mim nenhuma terra, pois que devo me tornar deserdada e sem direito”. Pois
que Bohort o ouviu, então perguntou às mulheres o que seria. Então ela falou:
“É a maior maravilha do mundo”. “Dizei-me”, falou ele, “o que seja!” “Em
verdade, Senhor,” falou ela, “de bom grado:

 Quando o rei Amans tinha a terra em sua mão e em seu poder, e ainda mais quando
seja, então tinha amor a uma donzela, que lá é muito mais bela que eu, e
deu-lhe todo o poder de sua terra e de seus homens à vez. Tanto tempo quanto
ficou junto dele, ela trouxe mau costume, que não estava dentro de nenhuma
justiça, e fez tantas grandes injustiças em muitos cantos. Quando o rei o viu,
então a empurrou para fora de sua terra e me deu em meu poder tudo que ela
tinha. Mas tão logo estava morto, então ela começou a me guerrear, de que ele
tomou desde então uma grande parte da minha terra e obrigou a ela muitos de
meus homens. E ainda daquilo que fez, assim não me deixa com paz, quando fala
que quer, à vez, deserdar-me. E ela começou tão bem que não me deixou senão
esta torre, que não me fica se eu não achar um campeão, que amanhã lute por mim
contra Priaden, o Negro, que lá por sua vontade encampou a luta”. “Pois me
dizei quem seja Priaden.” “É”, falou ela, “o mais cruel campeão desta terra e é
de grande nobreza”. “Vossa luta”, falou ele, “deve ser amanhã cedo?” “Sim”,
falou ela. “Então podeis comunicá-lo”, falou ele, “que encontrastes um
cavaleiro que deve lutar contra ele e que deveis ter a terra pois o rei Amans
vos deu, e que ela não deve ter sua fruição, pois seu senhor a precipitou para
fora”. Quando a mulher ouviu esta fala, não ficou pouco contente, quando falou
por causa da alegria que tinha: “Senhor, bem vistes hoje cá dentro, quando me
fizestes tão grande alegria por causa desta promessa. Pois Deus vos dê força e
poder, que possais vencer esta luta, quando verdadeiramente tudo é meu por
direito, quando outra coisa não desejo que seja”. E ele a consola muito e lhe
diz que não tema perder seu direito por tanto tempo quanto ele esteja saudável
e nobre. E comunicou à sua irmão, que seu cavaleiro amanhã estaria pronto a
fazer tudo o que os cavaleiros da terra quisessem que ele devesse fazer. Assim
ela traiu que a luta ficaria encomendada até a outra manhã cedo. 

À noite Bohort faz grande alegria e grande economia. E a mulher fez preparar-lhe
uma cama bela e rica. E quando era tempo de ir dormir, então ela o leva a uma
câmara, que era grande e bela. E pois que lá veio e olhou a cama que se lhe
tinha feito, então os fez todos sair, e eles o fizeram, e apagaram as velas por
fim. Depois ele se deita sobre a terra e coloca um pano sobre sua cabeça e faz
sua oração, que Deus por Sua misericórdia quisesse lhe vir em auxílio contra o
cavaleiro que ele deveria combater tão verdadeiramente quanto o faz, por causa
da justiça e por causa da lealdade, a trazer e empurrar abaixo a injustiça.
Pois que tinha feito sua oração, então adormeceu. E tão logo estava adormecido,
então lhe pareceu que vinham à sua frente dois pássaros, e um era branco como
um cisne e na grandeza de um cisne, e o outro era maravilhosamente preto e não
era de grande idade. E ele o contemplou, quando lhe pareceu ser um corvo, e ele
era muito belo da negrura que tinha. O pássaro branco veio a ele e falou:
``Queres me servir, eu te dou toda a altura do mundo e te faço tão belo e tão
branco quanto eu”. E ele lhe pergunta quem ele era. “Não vês quem eu sou, assim
branco e assim belo e ainda suficientemente mais do que consideras?” E com
isto, foi-se embora. E de pronto veio o pássaro preto e lhe diz: “Deve ser que
me sirvas amanhã cedo e não me deves odiar por causa disto, que sou preto. Sabe
que melhor é minha negrura que outra branquidão”. E foram-se pelo caminho,
assim que ele não viu nem um nem o outro. Depois deste sonho, veio-lhe outro à
frente, muito maravilhoso, quando lhe pareceu que ele encontrou uma casa, que
era bela e grande e se igualava bem a uma capela. E quando para lá foi, lá
encontrou um homem velho sentado sobre uma poltrona, e tinha à mão esquerda, 
estando longe, uma madeira, que estava podre e cheia de vermes, tão doente que
mal podia manter-se reta. E do lado direito tinha duas flores de lis; uma flor
se fazia próxima da outra e queria tomar-lhe sua branquidão. E o bom homem as
separou, tal que uma não movesse a outra. E não ficou muito tempo que de cada
uma saíssem flores, trazendo à vez muitos frutos.

Pois que isto aconteceu, o bom homem falou para Bohort: “Bohort, não seria um
tolo aquele que deixasse essas flores perecerem e viesse em auxílio da madeira
podre, para que não caísse por terra?” “Senhor,” falou ele, “seguramente sim,
quando me parece que a madeira para nada é útil. E essas flores são muito mais
belas do que eu considerava”. “Então te protege, é que vês a aventura vir, que
não deixes essas flores perecerem e venhas em socorro dessa madeira, quando é
que ela significa um fogo grande por demais, elas podem na hora se perder”. E
ele falou que deveria pensar nisso, se isso lhe sucedesse.
Assim lhe vieram à noite os dois sonhos à frente, que lhe fizeram muita
maravilha, quando ele não conseguia pensar no que eles poderiam significar.
Quando se levanta, então faz uma cruz à frente de sua testa e muito se
encomenda a Nosso Senhor. E rezou tanto tempo até que se fez luz e belo dia.
Então veio até ele a donzela da casa e o saudou, e ele a pagou e que Deus lhe
desse alegria. E depois ela lá o escolta à capela e ele ouve missa e o serviço
de Nosso Senhor do dia. E vem um pouco antes das primas horas, então ele saiu
da capela para o salão com grande companhia de cavaleiros e escudeiros que a
mulher tem, a quem a mulher tinha comunicado para ver a luta. E quando ele veio
ao palácio, então falou a mulher que ele comesse um pouco, antes que se
armasse. E ele falou que não queria comer até que tivesse trazido a luta ao
fim. “Assim não é senão para vos trazer vossas armas e vos preparar, quando
consideramos que Priaden esteja agora armado na luta”. Então ele se preparou,
assim que não lhe faltasse. Então se sentou sobre seu cavalo e falou para a
mulher que ela e sua companhia o escoltassem ao campo, onde a luta deveria ser.
E ela e sua companhia sentaram-se de pronto e seguiram de lá e o escoltaram a
um gramado, que ficava em um fundo. E eles viram no vale muitas pessoas, que lá
esperavam Bohort e a mulher, pela vontade de que ele deveria lutar. Então
cavalgaram descendo a montanha. E quando vieram ao plano e as duas mulheres se
viram, então veio uma contra a outra. Então falou a mulher, por cuja causa
Bohort deve lutar: “Senhora, eu me queixo a vós com direito, quando o rei Amans
me deu sua terra, que não deveis ter, como aquela que foi deserdada com a boca
do rei”. “E eu sou aquela que nunca foi deserdada e o quer provar a vosso
cavaleiro, se ele ousar abalar o que é meu”. E pois que ela viu que não podia
de outra forma escapar, então falou para Bohort: “O que vos parece da mulher?”
“Parece-me”, falou ele, “que ela vos guerreia por injustiça e com falsidade, e
falsos são todos aqueles que a ajudam, e ouvi disto muito de vós e de outros,
que bem sei que ela não tem direito. E se um cavaleiro quiser falar que ela tem
direito, estou pronto que o faça hoje, neste dia, mentiroso”. Então saltou
Priaden à frente e falou que não premiava sua verdade com nenhum vintém, quando
estivesse pronto para que reclamasse a donzela. “E eu estou pronto”, falou
Bohort, “para que, perante esta donzela, que para cá me conduziu, eu lute
contra vós, que ela deve ter esta terra, desde que o rei a ela deu, e a outra
donzela a deve deixar com direito”. Então se separaram, um aqui, outro lá, os
que estavam lá sobre o plano, e os dois cavaleiros cavalgaram um do outro.
Depois se deixaram correr um para o outro e se encontraram tão duramente que os
escudos fenderam e as coifas quebraram e as lanças se estilhaçaram. Então se
empurraram com os peitos tão duramente que caíram por terra sob o pé do cavalo,
e levantaram-se rápido o suficiente, como aqueles que lá eram de grande
nobreza. E então sacaram a espada e cobriram-se com seus escudos, e um deu ao
outro grandes golpes, que lhe doeram, e quebraram os escudos e golpearam em
cima e em baixo, e fizeram voar grande pedaço fora da terra e quebraram a coifa
sobre os ombros e jorraram muito sangue. E Bohort encontrou muito mais nobreza
do que considerava; quando bem sabe que estava em boa pretensão e verdadeira,
era uma coisa que muito o consolava e o fez sofrer que o cavaleiro golpeasse
sobre ele. E ele se cobriu e o deixou trabalhar consigo próprio. Pois que lá
viu que o cavaleiro vinha em grande poder, então avançou sobre ele como se no
dia nunca tivesse golpeado nenhum golpe e lhe deu grande empurrão. Então o
trouxe lá em tão curta hora que ele não teve poder de se demover, tantos golpes
tinha recebido. Quando Bohort o viu tão cansado, então o golpeou ainda mais, e
aquele foi cambaleante para cá e para lá, que despencou por terra. E Bohort o
agarra com o pescoço e o puxou para si e lhe trouxe o elmo da cabeça e o
golpeou com o cabo na espada sobre a cabeça, que o sangue de lá saltou para
fora, e os anéis da coifa lhe foram dentro. E ele falou que o mataria se não se
desse por caído na luta, e fez justiça quando quis lhe arrancar o pescoço. E
aquele o viu erguer o braço sobre sua cabeça, então teve preocupação de morrer
e pediu graça. E falou: “Ah, cavaleiro livre, por Deus apiedai-vos de mim e não
me matai, eu vos prometo que nunca quero guerrear contra a donzela tanto tempo
quanto eu viva”. E Bohort o deixou de pronto. Quando a velha
mulher viu que seu campeão estava vencido, então fugiu do plano, como aquela
que considerou que estivesse perdida. E Bohort veio de pronto a todos os que lá
estavam sobre o plano, que lá tinham terra dela, e falou que os queria
expulsar, se não lhe quisessem prestar homenagem.  Lá estavam muitos
homens que prestaram homenagem à donzela. Aqueles que não o quiserem fazer
serão mortos e deserdados; isto aconteceu por causa da nobreza de Bohort, que a
senhora voltasse ao domínio. E pois que a paz foi falada, que o inimigo da
donzela não se atreveu a erguer sua cabeça, então Bohort seguiu de lá seu
caminho e cavalga através da floresta, pensando no que tinha visto em seu sono.
Pois que deseja muito vir à cidade, pois gostaria de ouvir o significado disso.
À noite Bohort deitou na casa de uma viúva, que muito bem o albergou e estava
muito feliz de sua vinda, pois que o confessou. No outro dia, tão logo foi dia,
então seguiu de lá e se pôs acima na alta via do ermo. Pois que tinha caminhado
até o meio-dia, então lhe aconteceu uma aventura à vez maravilhosa, quando
vieram a seu encontro, entre dois caminhos, dois cavaleiros, que lá traziam
conduzido seu irmão nu em sua roupa de baixo, sobre um corcel grande e forte, e
lhe tinham atado as mãos à frente do peito. E cada qual tinha sua mão repleta
de espinhos afiados, com que eles lhe batiam tão duramente que o sangue jorrava
mais que em cem pontas das costas abaixo. E ele não falou nunca palavra, como
aquele que lá era de grande coração, quando padece tudo o que lhe fazem,
justamente como se não sentisse nada.

Nisto que ele queria vir em seu auxílio, então Bohort olhou o outro lado e viu
um cavaleiro armado, que lá conduzia uma donzela com violência, e queria
conduzi-la ao mais denso da floresta. E ela clamou com voz alta: “Senhora Santa
Maria, protege-me!” E quando viu Bohort cavalgar sozinho, então se virou para
lá, quando pensou que seria um dos cavaleiros da demanda. E clamou para ele
tanto quanto pôde: “Ah, cavaleiro, eu vos reclamo por Aquele por quem és
protegido, em cujo serviço te puseste, que me ajudes que eu não seja conduzida
pelo caminho pelo cavaleiro que me tem por violência”. Quando Bohort entendeu
aquela que assim altamente lhe reclamava, então ficou triste que não soube o
que ele deveria fazer. Se deixasse seu irmão seguir adiante, aqueles que o
levavam talvez o matassem, que ele nunca mais o veria com saúde nem nobreza. “E
se não ajudar a donzela, assim sua inocência lhe será tomada, e assim ganha a
vergonha por causa disso”. Então ergueu seus olhos contra os Céus e falou todo
gritando: “Querido e doce Pai Jesus Cristo, o verdadeiro homem de que sou,
protege-me o meu irmão de maneira que eles não o matem. E venho por causa da
Tua misericórdia em auxílio da donzela, que ela não seja trazida à vergonha,
quando me parece que o cavaleiro queira tomar sua inocência.” Então se volta
para lá aonde o cavaleiro seguia e clamou por ele: “Senhor cavaleiro, deixai
seguir a donzela, senão estais morto!” Quando ouviu esta fala, então assentou
a donzela, e ele estava armado com todas as armas, sem lança. Então pôs seu
escudo à sua frente e sacou sua espada e voltou-se para Bohort. E Bohort o
espetou tão duramente, que o espetou através do escudo e da coifa, e aquele
desabou por terra. Depois veio Bohort à donzela e falou: “Parece-me que
estivestes preocupada por causa do cavaleiro; é do vosso agrado que eu vos faça
mais?” “Amigo,” falou ela, “desde que me reclamastes e me protegestes minha
honra, que não fui envergonhada, então vos peço que me conduzis para lá, onde o
cavaleiro me tomou”. E ele falou que o faria de bom grado, e então tomou o
cavalo do cavaleiro e a montou e a conduziu como ela lhe mostrava. E quando
veio um pouco longe, então ela falou: “Senhor cavaleiro, tende grande
agradecimento porque me ajudastes; quando se ele tivesse tomado minha
inocência, cem homens precisariam ser mortos por causa disso, que então
permanecem vivos”. E ele pergunta quem seria o cavaleiro. “Seguramente”, falou
ela, é um parente, e não sei por qual cilada do Inimigo ele estava tão ardente,
que me tomou secretamente na casa de meu pai e me conduziu a esta floresta para
me tomar minha inocência. E se o tivesse feito, seria morto de pecado e
vergonha do meu corpo e eu enganaria toda a minha vida”. 

 Nisto que assim falara, então viram aproximarem-se doze cavaleiros, todos
armados, que lá procuravam a donzela no ermo. E quando eles a viram, tiveram
tão grande alegria que foi maravilha, quando ela clamou que agradecessem ao
cavaleiro e o tomassem com eles, quando estaria enganada, não fossem Deus e seu
homem! Então o tomaram pelo arreio e falaram para ele: “Senhor, deveis vir
conosco, quando assim deve ser, pois tanto nos servistes, que não
conseguiríamos vos agradecer plenamente”. “Caros Senhores,” falou ele, “eu não
venho convosco de maneira alguma, quando tanto tenho a fazer em outro lugar,
que não posso permanecer. E não o tomai por mal, sabei que de bom grade
seguiria convosco, quando a companhia me é lá tão grande e a perda tão grande e
tão impadecível, que ninguém senão Deus sozinho me poderia fazer retornar”.
Pois que eles ouviram isto, então não queriam mais importuná-lo e o
encomendaram a Deus. E a donzela lhe pediu tão bem que ele a
viesse ver, tão logo tivesse hora, e mostrou onde poderia encontrá-la. E ele
falou que o queria fazer, se a aventura lá o trouxesse. Então se separou deles,
e os cavaleiros conduziram  a donzela com eles. E Bohort segue para onde tinha
visto Leonel, seu irmão, e olhou tão longe quanto podia ver e cobiçou se
conseguia escutar. E pois que não ouviu coisa nenhuma, de que pudesse ter
esperança de seu irmão, então se pôs em seu caminho, onde o tinha visto para cá
ser conduzido. E quando tinha cavalgado um bom tempo, então seguiu um homem,
vestido com roupas de uma pessoa comum, e cavalga um cavalo, mais preto que um
mouro. 

Quando escutou que Bohort seguia de todo atrás dele, então clamou por ele e
falou: “Senhor cavaleiro, o que procurais?” “Senhor,” falou ele, “eu procuro
eu irmão, que agora vi que dois cavaleiros lhe batiam”. Então falou o homem:
“Para que não vos porteis erradamente e para que não caiais em dúvida, quero
vos dizer o que sei sobre isso”. Pois que Bohort entendeu esta fala, então
pensou, todo de pronto, que os dois cavaleiros o tivessem matado. Então começou
a colocar grande lamento, e quando conseguiu falar, então falou: “Ah, Senhor,
ele está morto, então me mostrai o corpo, assim o faço sepultar e lhe faço tal
honra, como se costuma fazer ao filho de um rei. Quando seguramente ele era
filho de um homem nobre e de uma mulher nobre”. “Então olha ao
teu\footnote{ Novamente a contumaz alteração pronominal.}  redor e
vê-o.” Então viu jazer sobre a terra um corpo, que parecia estar recentemente
morto. Ele o contemplou e o confessou, como lhe pareceu que seria seu irmão.
Então teve tão grande remorso que não conseguiu ficar em pé, quando caiu por
terra e veio de dentro de si mesmo e jazeu lá desmaiado por longo tempo. E
quando se levantou, então falou: “Ah, amado irmão, quem vos fez isto?
Seguramente, nunca ficarei contente, se Aquele que vem em socorro do pecador
não me consolar nesta tristeza. E porque assim é, amado irmão, que a sociedade
entre nós dois está separada, Aquele que tomei por companheiro deve ser meu
guia e meu protetor em todas as minhas necessidades, quando doravante não tenho
mais no que pensar senão no valor de minha alma, porque estais separado da
vida”. Pois que o tinha falado, então tomou o corpo e o ergueu à sela
justamente como não precisasse se fatigar, como lhe pareceu. Depois falou para
aquele que estava junto dele: “Senhor, por Deus, dizei-me, há em qualquer lugar
por aqui uma capela, onde eu possa sepultar este cavaleiro?” “Sim,” falou ele,
“aqui há uma capela à frente de uma torre, onde ele pode ser sepultado”.
“Senhor, por Deus,” falou Bohort, “guiai-me lá!” “Eu muito vos peço,” falou o
bom homem, “vinde atrás de mim”. E Bohort montou seu corcel e seguiu à sua
frente, como lhe pareceu, o cadáver de seu irmão. Então não seguiram longe, que
viram à sua frente uma torre alta e robusta, à frente estava uma velha casa em
igualdade a uma capela. Então se distanciaram da porta por onde se entrava.
Então foram até ela e colocaram o cadáver sobre uma padiola, que lá estava no
meio da casa. Então olhou para cima e para baixo, quando não viu nem vinho, nem
água, nem cruz nem nenhum sinal verdadeiro de Nosso Senhor Jesus Cristo. “Então
o deixai estar bem aqui e deixai-nos ir à torre até amanhã, que eu volto para
cá para fazer o serviço de Nosso Senhor ao vosso irmão.” “Como é, pois?”, falou
Bohort, “Sois padre?” Então ele falou: “Sim”. “Então vos peço”, falou Bohort,
“que me digais a verdade de meu sonho e de outras coisas que eu duvido”. “Então
dizei a mim”, falou ele. E lhe contou do pássaro, que ele tinha visto na
floresta, e depois lhe diz dos pássaros, que um era branco, o outro preto, e da
madeira podre e das flores brancas. “Eu devo dizer-vos uma parte e amanhã a
outra parte: o pássaro que lá vos veio da maneira de um pássaro que se igualava
a um cisne, significa uma donzela que te\footnote{ Novamente a contumaz
alteração pronominal. } deve ter amor e por muito tempo te
considerou e deve brevemente vir a ti e deve te pedir que tu queiras ser seu
amado e que queiras dormir junto dela. E tu não a queres seguir, isto significa
que deves recusá-la, então ela segue seu caminho e morre de remorso, é que isto
não se consterna. O pássaro preto significa um grande compadecimento que deves
ter para recusá-la, quando por causa do amor que tens por Deus e ainda nem pelo
que tens por ti, não a recusaste, quando o fizeste por causa de que se te
considera puro e casto, para merecer o elogio e a nobre alegria do mundo. E
desta castidade te vem ainda tão grande mal, que Lancelot, teu sobrinho, deve
morrer por causa disto. Quando o amigo da donzela o deve matar, e eles morrem
de remorso, que lhe recusaste o teu amor. E por causa disto se pode bem te
dizer que tu sejas um assassino de uma destas duas coisas, assim que foste do
teu irmão, a quem bem poderias vir em auxílio, que ele permanecesse junto à
vida, se tivesses querido, que o deixaste seguir caminho, e seguiste o caminho
para ajudar a donzela, que não te dizia respeito. Então contempla qual seria a
maior pena, se ela tivesse perdido sua inocência ou que teu irmão foi
assassinado, que lá era um dos melhores cavaleiros do mundo. Seguramente, para
mim seria preferível que todas as donzelas no mundo perdessem sua inocência a
que ele fosse assassinado”. 

Quando Bohort ouviu que aquele junto a quem ele considerava encontrar grande,
bom conselho, o culpava por causa de que ele tinha ajudado a donzela, não soube
responder. E ele lhe pergunta: “Ouviste o significado do teu sonho?” “Sim,
senhor”, falou ele. “Bohort, fique junto a ti de Lancelot, teu sobrinho;
quando quiseres, então podes bem redimi-lo da morte, e se quiseres, podes
matá-lo. Então está contigo, qual deles tu queres que aconteça”. “Em
verdade”, falou Bohort, “não sei nenhuma coisa que fizesse mais logo, antes que
matasse meu senhor Lancelot”. “Isto bem se deve ver de pronto”, falou o outro,
e depois o acompanha à torre. E quando lá entrou, então encontrou cavaleiros e
mulheres e donzelas, que lhe falaram todos: “Bohort, sede bem-vindo”. E o
conduziram ao salão e o desarmaram. E quando estava de corpo desnudo, então lhe
trouxeram um rico manto, forrado com arminho, e o lançaram para ele e o
assentaram sobre uma bela cama e o consolaram todos bastante, e começaram a
fazer-lhe alegria, tanto que o fizeram em parte esquecer seu remorso. No que o
consolavam e lhe faziam alegria, assim veio cá uma donzela, tão bela e
maravilhosa, que parecia ter consigo toda a beleza do mundo, e tinha também as
mais belas roupas do mundo. “Senhor,” falou um cavaleiro, “toda aqui está a
donzela a quem pertencemos, a mais bela e rica do mundo, que sempre vos
considerou, e vos esperou por longo tempo e nunca mais quis ter outro amado
senão vós”. E pois que entendeu esta coisa, então ficou todo temeroso. E quando
a viu vir, então a saudou, e ela o saudou de volta, e sentaram-se um ao lado do
outro e abriram de muitas coisas e tanto que ela lhe pediu que quisesse ser seu
amado, quando ela o amava sobre todos os homens do reino da terra. E se ele
quisesse lhe dar seu amor, então assim ela quereria fazê-lo mais rico que
nenhum outro em sua linhagem. Quando Bohort ouviu esta coisa, então
ficou muito triste como aquele que não quer perder de maneira nenhuma
sua pureza, e ele não sabia o que deveria responder. E ela falou: “O
que é contigo, Bohort, não queres fazer?” “Senhora,” falou ele, “não há nenhum
homem tão rico no mundo, por causa de quem eu o fizesse, e não se me deve
cansar na medida em que agora estou; quando meu irmão ali dentro jaz morto, que
hoje foi assassinado, e não sei como”. “Ah, Bohort, não pensai\footnote{
Novamente a contumaz alteração pronominal. } sobre isto, precisais
fazer tudo aquilo que eu vos peço. E sabei, não te tivesse eu amor mais que
nenhuma outra mulher ganhou nenhum homem, eu não os peço, quando não é
costumeiro que as mulheres peçam os homens em primeiro, por mais que lhe tenham
amor. Quando minha grande esperança, que por toda a via tive por vós, trouxe-me
para cá o coração, que eu preciso dizer o que por longo tempo ocultei. Por
causa disto eu vos peço, caro amigo amigável, que queirais fazer o que vos
peço: que esta noite queirais dormir junto a mim.” E ele falou que não o faria
de nenhuma maneira. E pois que ela viu isto, então a assolou tão grande
remorso, como ele considerou, que lhe pareceu que ela chorava e a assolou tão
grande lamento; quando tudo o que ela fez não lhe foi útil. E quando viu que
não conseguia vencê-lo de nenhum modo, então falou: ``Bohort, trouxeste-me aqui,
e por causa da recusa morro toda de pronto por vós”. Então o tomou com a mão e
o conduziu até a porta do palácio e lhe diz: “Ponde-vos aqui e vede como
morro!” “Em verdade”, falou ele, “eu o vejo assim”. E ela pediu aos que lá
dentro estavam que o mantivessem todo lá. E eles falaram que o fariam, e de
pronto escalaram as ameias da torre, e conduziu consigo bem duzentas donzelas.
E pois que haviam escalado, então falou uma, não a senhora: “Ah Senhor,
apiedai-vos de nós todas e fazei pela minha senhora! Seguramente, não lhe
façais isto, que nos deixamos todas cair da torre antes que nossa senhora,
quando sua morte não podemos ver de maneira nenhuma. Quando
seguramente, não nos deixais morrer por causa de tal coisa pequena, assim
nenhum cavaleiro fez tão grande deslealdade”. E ele as contemplou e considerou
seguramente que a mulher era uma dama nobre, e isto o apiedou muito à vez. E
porém, não estava de outro modo aconselhado, que lhe fosse preferível que todas
elas perdessem sua alma que ele, a dele. E ele falou que não terminaria isto,
nem por causa de sua morte, nem por causa de sua vida. E então
elas se deixaram cair da torre sobre a terra. E quando ele viu isto, ficou
aterrado. Então ergueu sua mão e persignou-se. Então ele ouviu de pronto ao
redor de si um grande estampido e gritaria que lhe pareceu que todos os
inimigos do inferno estariam ao seu redor. Sem dúvida, estavam alguns bem ao
seu redor. Então olhou à volta de si e não viu nem torre nem a mulher que lhe
pediu pelo amor, nem nenhuma coisa que antes tinha visto, senão sua arma
sozinha, que tinha para lá trazido. E pois que não viu a casa, então considerou
ter deixado seu irmão morto. 

Pois que viu, então pensou que era obra do demônio, que lhe tinha feito o
engodo, que o queria conduzir à perdição do corpo para perder a alma, quando
pela graça de Nosso Senhor, assim lhe tinha fugido. Então ergueu suas mãos ao
céu e falou: “Amado Pai Jesus Cristo, bendito sois que me destes força e poder
para lutar contra Teu Inimigo e me deixaste vencer a Tua batalha”. Então foi
para lá onde considerava encontrar seu irmão morto. Então não o encontrou,
então ficou muito mais feliz que antes, quando então ele pensou bem que não
estaria morto e que seria um engodo o que ele tinha visto. Então veio às suas
armas e armou-se e separou-se do plano, quando pensou que seria uma morada do
Inimigo. E quando tinha cavalgado um longo tempo, então ouviu soar um sino à
mão esquerda. Então ficou muito feliz desta aventura e seguiu caminho. E não
esperou muito tempo, que encontrou um convento trancado com bons muros, e era
de monges, que eram brancos. E ele veio à porta e bateu muito tempo à porta:
deixou-se-o entrar. E quando o viram armado, então bem pensaram que seria um
companheiro da Távola Redonda. Então o desarmaram de pronto e o guiaram a uma
câmara e lhe fizeram tudo o que podiam para bem. E ele falou para um bom homem,
que ele bem pensou que seria um padre: “Senhor, por Deus, conduzi-me a um irmão
aqui dentro, que lá é homem de Deus e o mais santo. Quando hoje me veio à
frente uma maravilhosa aventura, da qual eu quero de bom grado tomar conselho
dele e de Deus”. “Senhor cavaleiro,” falou ele, “deveis ir, conforme nosso
conselho, para o abade, quando ele é o mais nobre homem do mundo”. “Senhor, por
Deus, conduzi-me lá!” E ele falou que o faria de bom grado.

Então o guiou o irmão a uma capela, onde o abade estava dentro; pois que lhe
tinha mostrado, então se voltou. E Bohort seguiu em frente e o saudou, e tende
para ele e lhe pergunta de onde ele seria. E Bohort respondeu que fosse um
cavaleiro de aventura. Depois lhe conta o que lhe sucedeu no dia. E quando lhe
tinha contado tudo, então o bom homem lhe diz: “Senhor cavaleiro, não sei quem
sois, quando deveis saber que não considerava que nenhum cavaleiro de vosso ser
fosse tão forte em graça de Nosso Senhor como sois. Não dissestes de vossas
coisas, de que esta noite não consegui vos informar, quando é por demais tarde,
quando assim deveis ir repousar, quando estais cansado; e amanhã de manhã, 
então devo vos aconselhar o melhor que eu consigo”. Então se separou dele e
encomendou o abade a Deus; e ele permanece e pensou bem naquilo que ele lhe
tinha dito. E ele encomendou aos irmãos que lhe fizessem bem, quando fosse um
nobre, mais do que se considerava. À noite Bohort foi servido muito
mais ricamente do que lhe era agradável. Preparou-se-lhe carne e peixe, e deles
nunca morde. Quando ele tomou pão e água e comeu tanto quanto lhe era
necessário e não comeu mais nada, como aquele que lá de maneira nenhuma queria
ter quebrado sua penitência. Pela manhã, tão logo ouviu primas e missa, o
abade, que não o tinha esquecido, veio a ele e falou que Deus lhe desse boa
manhã. E Bohort lhe diz assim de volta. Depois o conduz para um fim distante
dos outros, à frente de um altar e falou que lhe contaria o que lhe tinha vindo
à frente na demanda do Santo Graal. Então lhe conta palavra por palavra o que
veio à frente dele em seu sono e também desperto e lhe pediu que ele lhe
dissesse este significado daquilo. E então pensou um pouco e falou que queria
de bom grado dizer-lhe. 

“Bohort, pois que recebestes o Alto Mestre, é para entender: pois que tomastes o
corpo de Nosso Senhor, então lhe disse Ele no caminho para saberes que
deveríeis encontrar o grande achado, que é uma aventura dos verdadeiros
cavaleiros de Jesus Cristo, que lá estão na demanda. Então não tínheis
cavalgado muito, que Ele vos veio à frente à maneira de um pássaro e vos provou
o martírio e o tormento que ele segue por vossa causa, e quer vos dizer como o
vedes. Quando o pássaro veio sobre a árvore sem folhas e sem frutos, então Ele
começou a ver seus pássaros e viu que nenhum estava vivo. Então se colocou
junto deles e começou a golpear-se no meio de seu peito com seu bico tanto
tempo que o sangue fluiu de lá, e ele morreu bem lá. E do sangue viveram todos
os jovens, que vistes. Então quero vos dizer o significado: o pássaro significa
nosso criador, que à Sua igualdade faz o homem. E pois que ele foi empurrado
para fora do Paraíso por causa de seu pecado, então veio ao reino da terra,
onde encontrou a morte, quando lá não estava a vida. A árvore sem folhas e sem
frutos significa claramente o mundo, no qual, a este tempo, nada havia senão
pecado e miséria e sofrimento. Os jovens significam o nascimento humano, que lá
estava tão perdido que todos seguiram para o inferno, os bons tão bem quanto os
maus, e estavam à toda vez em amargor. Quando o Filho de Deus viu isto, então
se tornou homem e subiu à árvore, que era à Sua cruz. Então foi golpeado com o
bico, foi que lá ficou espetado com a ponta da lança do lado direito tanto que
o sangue se precipitou para fora. E do sangue foram vivificados Seus jovens,
que fizeram a Sua vontade, quando os resgatou do inferno, que todos estavam lá
dentro, onde ainda nenhuma vida há. 

Esta bondade, que Deus fez ao mundo, também Ele provou a outros pecadores. A vós
ele veio prová-lo à maneira de um pássaro, por causa de que não mais temais
morrer por Ele do que Ele fez por vós. Depois vos guiou para a mulher, a quem o
rei Amans deu sua terra para conservar. Pelo rei Amans deves\footnote{
Novamente a contumaz alteração pronominal. } entender Nosso Senhor
Jesus Cristo, que lá é mais amado, e mais se pode encontrar junto a Ele doçura
e misericórdia do que se poderia em qualquer homem terreno. Então os outros a
guerrearam tanto que foi empurrada de junto da terra; então Ele faz uma luta e
os vence. Então devo vos\footnote{ O texto retorna ao emprego
do pronome formal.}  dizer o que o significado de Nosso Senhor
demonstrou, que Ele perturbou Seu sangue por vossa causa. E vós de pronto
também assumistes uma luta. Por Sua vontade foi que o fizestes à donzela.
Quando por ela devemos entender a Santa Igreja, que lá conserva a Santa
Cristandade em suas retas verdades, que lá é o reino da terra e o doce abrigo
de Nosso Senhor Jesus Cristo. Por causa da outra mulher, que lá foi deserdada,
e a guerreava com outros, esta é a Antiga Aliança, o inimigo, que lá por todos
os caminhos guerreia contra a Santa Igreja e contra a Santa Fé.

Pois que a donzela vos tinha contado o direito pelo qual a velha mulher a
guerreava, então também assumistes a luta, assim como devíeis. Quando fostes um
cavaleiro de Jesus Cristo, por causa disto viestes ao justo, para defender a
Santa Igreja, vista à maneira de uma mulher entristecida, e desaconselhada e
irada por causa de se a ter deserdado por injustiça. Ela não veio a vós vista
com roupas alegres, senão vos veio vista em roupas iradas, que era em roupas
pretas. Ela vos apareceu triste e negra por causa da ira que seus filhos lhe
fizeram, que são cavaleiros pecadores que devem ser os filhos, e são seus
filhos adotivos; e devem protegê-la como sua mãe; mas eles não o fizeram,
quando a entristeceram dia e noite. E por causa disto ela vos veio
vista em igualdade a uma mulher irada, que muito te\footnote{ Novamente a
contumaz alteração pronominal. } apiedaste. Pelo pássaro negro deves
entender Jesus Cristo, que então falou: ‘Eu sou negro, quando sou belo, e sabe
que muito mais bela é minha negritude que outro branco’. Junto ao pássaro branco,
que lá estava como um cisne, deves entender o Inimigo, e eu devo vos dizer
junto dele: o cisne é branco por fora e preto por dentro, que se iguala ao
hipócrita, que lá é belo por fora e branco, e pensado pelas pessoas que seja
servo de Nosso Senhor, e é por dentro tão odiável e tão impuro  de impureza e
de pecados que engana perversamente o mundo. O pássaro te veio à frente em teu
sono, assim te fez também desperto. Sabes onde que o Inimigo veio a ti à
maneira de um homem espiritual, que te disse que assassinaste teu irmão? Então
te mentiu, quando teu irmão não está morto, quando ele vive! E ele te diz isso
por causa de que te traia e te conduzia mal para a desesperança e para a falta
de castidade, e assim teria te trazido a pecado mortal, por causa do qual
terias falhado na aventura do Santo Graal.  

Então te contei o que o pássaro branco e o pássaro preto e quem a mulher eram,
por causa de quem lutaste e contra quem era. Então preciso te dizer o que seria
a madeira podre e as flores. A madeira podre sem força significa o teu irmão
Leonel, que lá não tem em si nenhum bem para Nosso Senhor, que o queria manter
reto. A podridão significa os grande pecados, que ele tem em si; por causa
disso se deve chamá-lo uma madeira podre e devorada por vermes. E as duas
flores, que lá estavam à mão direita, significam duas donzelas; é o cavaleiro
uma que lhe fez ferida, e a outra, à qual viestes\footnote{ Novamente a
contumaz alteração pronominal. } em auxílio. Uma flor se faz próxima
junto à outra, que era o cavaleiro que queria ter com violência e queria
tomar-lhe a inocência, e vós a reclamastes. É falado que Nosso Senhor não
queria que ela perdesse sua castidade, quando Ele vos guiou para lá, que vós as
separásseis uma da outra e conservásseis a cada qual sua pureza. E Ele vos
disse, Bohort, que ele bem seria um que deixaria essas flores se perderem e
viria em socorro dessa madeira podre. Então vos mandou, e o fizestes, de que à
vez Ele vos disse grande agradecimento. Quando vistes vosso irmão, que os dois
cavaleiros guiavam, e vistes a donzela, que o cavaleiro guiava. Ela vos pediu
tão docemente, então fostes enganado por causa da verdade fraterna e oração, e
deixastes para trás amor natural por causa do amor de Jesus Cristo, e seguiste
a ajudar a donzela. E deixastes vosso irmão em inquietação,
quando Aquele, em cujo serviço vos colocastes, estava em vosso caminho. E disto
então veio um grande sinal por causa do amor, que demonstrastes pelo Rei do
Reino dos Céus, que de pronto caíram os cavaleiros mortos. Então ele se soltou
e tomou as armas de um e se armou e montou seu cavalo e seguiu à busca dos
outros. E desta aventura deveis saber brevemente a verdade. 

Então vede que das flores vieram folhas e frutos. Isto significa que do
cavaleiro torna-se e vem grande linhagem, de que ainda devem vir pessoas nobres
e verdadeiros cavaleiros, que se podem ter por frutos. E assim se faz, pois, da
donzela. E se tivesse vindo que tivesse perdido em tão horrível pecado sua
inocência, assim a ira de Nosso Senhor teria ido sobre ambos, assim que
estariam danados com morte súbita. E assim estariam perdidos com corpo e com
alma. E isto vós conservastes; e por causa disto se deve ter-vos por um servo
de Deus, leal e bom. E, assim Deus me ajude, não seria vosso serviço tão alta
aventura, nunca vos sucederia que resgatásseis das pessoas de Nosso Senhor, o
corpo do sofrimento terreno e a alma do padecimento.

 Então vos contei o significado das aventuras, que vos vieram à frente nesta
demanda do Santo Graal”. “Senhor,” falou Bohort, “dizeis
verdade, bem me significastes, e melhorarei todos os dias que eu viver”. “Então
vos peço”, falou o bom homem, “que rogueis por mim, quando, assim Deus me
ajude, eu penso que Ele vos ouve mais breve que a mim”. E Bohort então se cala
e se envergonhou por causa de o abade o ter por um bom homem. Pois que tinham
conversado por um longo tempo um com o outro, então Bohort se apartou de lá, e
encomendou o abade a Deus. E quando estava armado, pôs-se em seu caminho e
cavalga até que veio a um castelo, que se chamava Thunburg, que lá estava em um
descampado. Pois que veio ter ao castelo, um servo o encontra, que logo foi ter
à floresta, e veio perante ele e lhe perguntou se sabia de notícia nova. “Sim,”
disse ele, “amanhã deve, bem lá do castelo, ser um torneio maravilhoso”. “De
que pessoas?”, falou Bohort. “Dos condes da terra e da senhora do castelo”,
falou ele. Quando Bohort ouviu esta notícia, então pensou que queria
permanecer, quando não poderia ser que encontrasse alguns companheiros da
demanda. Quando poderia ver algum que lhe dissesse novas de seu irmão, ou seu
irmão simplesmente estivesse lá ou em algum lugar próximo, e ele soubesse
notícias dele, se estava com saúde. Então se voltou para a clausura, que viu
estar em um dos fins do arbusto. Quando lá veio, então encontrou Leonel, seu
irmão, que lá se sentava desarmado na entrada da capela e lá estava albergado,
por causa de que queria estar no torneio no outro dia, que lá deveria ser
golpeado no gramado. E pois que viu seu irmão, teve tão grande alegria que
ninguém a conseguia contar. Então pulou de seu cavalo e falou: “Amado irmão, de
onde vindes?”

Quando Leonel ouviu esta fala, então o reconheceu e não se calou e falou:
“Bohort, não se vos trouxe que eu depois de pouco não fui assassinado, quando
os dois cavaleiros me conduziam para golpear, e deixastes-me conduzir, que
nunca me ajudastes. Quando seguistes a ajudar a donzela, que o cavaleiro
conduzia, e deixastes-me em necessidade de morte, que nunca um cavaleiro fez a
seu irmão tão grande deslealdade. E por causa da falta, assim não vos asseguro
nada além da morte, quando bem a merecestes. Então vos protegei de mim, pois
não deveis esperar outra coisa senão a morte, em qual sítio eu venha, tão logo
esteja armado”.

Pois que Bohort ouviu esta fala, então ficou muito abalado que seu irmão
estivesse irado com ele. Então caiu sobre seus joelhos perante ele e lhe pediu
por graça, com as mãos dirigidas, e lhe pediu que o quisesse perdoar. E ele
falou que não podia ser, quando o queria matar, Deus o ajudasse que o vencesse.
E por causa de que não mais queria ouvi-lo, então foi à casa do eremita aonde
tinha conduzido suas armas, e as tomou e armou-se logo. E quando estava armado,
então veio ao seu cavalo e o montou e falou para Bohort: “Protegei-vos de mim!
Assim Deus me ajude, eu o derroto, não vos faço outra coisa senão o que se deve
fazer a um cavaleiro desleal. Quando seguramente sois o cavaleiro mais desleal
que procedeu de homens tão nobres como era o rei Bohort, que ganhou a mim e a
vós. Então montai vosso cavalo, então estais bem mais seguro. Se não o
fazes, eu o mato assim a pé como estais. Assim a vergonha é minha e o dano é
vosso; quando a vergonha não me causa dano, quando prefiro muito ser admoestado
que não vos envergonhar, como simplesmente deveis ser. E vós o merecestes!” 

Pois que Bohort viu que precisava lutar contra seu irmão, então não soube o que
deveria fazer, quando não estava aconselhado de nenhuma maneira que devesse
lutar. E porém, por causa de que estivesse seguro, então montou seu cavalo,
então ainda queria tentar para que conseguisse encontrar graça. Então caiu
sobre seus joelhos, sobre a terra, à frente dos pés do cavalo de seu irmão e
grita de todo o coração e fala: “Amado irmão, apiedai-vos de mim e perdoai-me
esta falta, e não me mateis, e lembrai-vos do grande amor que deve ser entre
mim e vós”. Isto tudo não ajudou ao que Bohort podia dizer, então Leonel não se
deteve, quando o diabo o tinha inflamado tanto que queria assassinar seu irmão. 
E Bohort estava todo sobre seus joelhos e lhe pediu por graça com as
mãos erguidas. Pois que Leonel viu que ele não fazia outra coisa e que
ele não se levantava, então seguiu à frente e o encontrou tão duramente com o
peito de seu cavalo, que o empurrou atrás de si sobre a terra. E com a queda
que ele fez, então ficou muito ferido, e seguiu por cima de seu corpo com o
cavalo tanto que ele o prensou. E Bohort desmaiou da dor que recebeu. E pois
que Leonel o trouxe a que ele não se conseguisse ajudar, então desmontou seu
cavalo e queria cortar-lhe a cabeça fora. Pois que ele estava distante a
desaninhar-lhe o elmo da cabeça, então veio cavalgando o bom homem, que lá era
muito velho e bem tinha ouvido as falas que lá tinham acontecido. E pois que
viu que Leonel queria cortar a Bohort sua cabeça, então se deixou cair sobre
ele e falou: “Ah, nobre cavaleiro, tem de ti mesmo graça e de teu irmão, quando
se o matasses, então morrerias de pecado, e seria uma grande vergonha dele fora
de medida, quando é o mais nobre cavaleiro do mundo”. “Assim Deus me
ajude”, falou Leonel, “se não fugirdes dele, eu vos mato e por causa disto de
nada adianta a ele”. “Em verdade,” falou o bom homem, “é-me preferível que me
mates, quando não seria tão grande vergonha de minha morte como da dele, e por
causa disto prefiro morrer a ele”. E colocou-se por cima dele, tanto tempo
quanto era, e o enlaçou pelos ombros. E quando Leonel o viu, então sacou sua
espada e golpeou o bom homem tão duramente que lhe cortou para trás o pescoço,
e ele se esticou, quando o tormento da morte o obrigou. Pois que o tinha feito,
então ainda não tinha esquecido sua vontade perversa, quando tomou seu irmão
pelo elmo e o desaninhou, para cortar-lhe fora sua cabeça, e o teria matado em
curta hora. Então veio da graça de Deus que lá veio Galogrevant, um cavaleiro
da corte do rei Arthur e era também companheiro da Távola Redonda.
Quando viu o bom homem morto, isto o maravilhou muito, o que era. Depois olhou
à sua frente e viu que Leonel queria matar seu irmão e agora tinha soltado o
elmo. E ele reconheceu Bohorten,\footnote{ Nesta passagem, o nome do cavaleiro é
grafado como Bohorten, mesmo que não se trate de uma flexão do acusativo.}  
a quem tinha tanto amor. Então saltou sobre a terra e agarrou
Leonel com os ombros e o jogou para a terra e falou para ele: “O que é isto,
quereis assassinar vosso irmão? Estais sem juízo? Ele é, porém, o melhor
cavaleiro que se pode encontrar. Assim me ajude Deus, nenhum homem que é nobre
permite isso”. “Como é pois,” falou Leonel, “quereis ajudá-lo? Seguramente,
tomai-vos sua defesa, e o deixo ficar e agarro-me convosco”. E ele o contemplou
e afastou-se todo dele. “O que dizeis, Leonel, é sério que quereis matá-lo?”
“Matá-lo eu quero”, falou ele, “quando por causa de vossa vontade nem de nenhum
outro, fica isto sem se cumprir, eu o mato, quando fez tantas faltas a mim que
bem se culpou para a morte”. Então foi por cima dele e cortou-lhe sua cabeça.
Então se pôs Galogravant\footnote{ Nesta passagem, o nome é grafado desta forma.} 
entre os dois e falou que, se fosse tão mais ousado que lhe
pusesse a mão, ele se poria solidamente entre os dois.

Pois que Leonel ouviu isto, então lhe perguntou quem ele seria. E ele se tomou.
E quando o reconheceu, então tomou seu escudo e o contradisse e seguiu por cima
dele e deu-lhe um golpe tão duro quanto jamais pôde. E pois que o viu vir com o
golpe, então correu ao seu escudo e à sua espada, e ele era um cavaleiro forte
e de grande poder e defendeu-se, à vez, nobremente. E o golpe durou tanto que
Bohort estava levantado tão doente que não considerou que, em um mês, não
conseguiria ganhar qualquer poder, que Deus o ajudasse, pois. Quando viu que
Galogravant lutava com seu irmão, então não soube o que deveria fazer, quando
se Galogravant matasse seu irmão à sua frente, nunca ficaria feliz. E se seu
irmão matasse Galogravant, então a vergonha seria sua, quando bem sabia que ele
nunca teria começado a lutar senão por sua causa. E à volta estava por demais
insatisfeito e se teria de bom grado apartado, se tivesse podido; quando tanto
lhe doía que ele mesmo não pudesse se defender, nem ajudar ao outro. E quando
tinha por longo tempo assistido, então viu que Galogravant caiu na luta, quando
Leonel era de maior cavalaria e muito viril e lhe teria quebrado seu escudo e
seu elmo e o teria ali trazido, que não estivesse mais seguro que da morte.
Quando tinha perdido tanto de seu sangue, que era maravilha que pudesse ficar
de pé. E quando assim cair, então temeu morrer, então olhou ao redor de si e
viu que Bohort se tinha alinhado. Então falou para Bohort: “Ah, Bohort, por que
causa não vindes em meu socorro e me lançais fora da necessidade da morte, para
onde vim por causa de vos ajudar? Quando estáveis muito mais próximo da morte.
E então vedes seguramente que eu morro, todo o mundo devia ralhar convosco”.
“Seguramente, não sois de nenhum uso, nem ninguém pode vos ajudar, e vos mato
aos dois com esta espada.” Quando Bohort ouviu isso, então não estava seguro, e
teve à vez grande tristeza e pediu a Nosso Senhor que Se apiedasse da alma;
quando por causa de coisa tão pequena, nunca morreu nenhum nobre homem. E
Galogrevant o chamou: “Ah, Bohort, deixais-me morrer assim? É de vosso agrado
que eu morra, então a morte é-me do agrado, quando não se pode morrer por homem
mais nobre”. Então Leonel o golpeou com a espada, que lhe fez saltar o
elmo da cabeça. Pois que achou sua cabeça nua e descoberta e viu que ele não
queria vir de lá, então falou: “Amado Pai Jesus Cristo, se impuseres que eu me
coloque em teu serviço não tão dignamente como devo, apieda-te de minha alma de
maneira que esta dor, que meu corpo padece por causa do bem e da esmola que eu
quis fazer, deve ser a minha penitência e conservação da minha alma”.

Nisto que falou esta fala, então Leonel o golpeou tão duramente que o matou, e
ele caiu sobre a terra, e o corpo esticou-se da dor que ele padecia. Pois que
Leonel o matou, então não se deixou satisfazer, quando correu para seu irmão e
lhe deu um tão grande golpe, que ele se abalou. E porque a humildade nele
estava tão naturalmente arraigada, pediu-lhe por Deus que o perdoasse, “quando
viesse, amado irmão, que eu vos matasse ou vós a mim, deveríamos morrer de
pecado”. “Jamais Deus me ajuda”, falou Leonel, “se eu vos perdoar, eu vos mato,
é que vos derroto; quando não vos adianta que eu não fui morto”. Então Bohort
sacou sua espada e falou gritando: “Amado Pai Jesus Cristo, isto não me deve
ser observado por pecado, é que protejo meu corpo contra meu irmão”. Então
alçou sua espada contra a montanha e nisto que quis matar, então ouviu uma voz
que lá falou: “Bohort, foge, não o golpeia, senão o matas!” Então desceu no
meio deles um brilho de fogo em igualdade a um relâmpago, que lá vinha do céu,
e dele saiu uma chama tão maravilhosa e tão brilhante, que ambos seus escudos
foram chamuscados, e eles estavam tão assustados que ambos caíram por terra e
jazeram longo tempo desmaiados. Então se contemplaram assim
duramente e viram que a terra entre ambos estava bem vermelha do fogo que lá
esteve. Quando, então, Bohort viu isso e que seu irmão não tinha nenhuma dor,
então alçou suas mãos contra o céu e louvou a Deus com bom coração. 

Então ouviu uma voz que falou: “Levanta, Bohort, e vai daí e não mantém mais
companhia a teu irmão, quando deves cavalgar na direção do mar, e não
permanecer em nenhum sítio antes que lá venhas, quando Parsifal te aguarda bem
lá”. Pois que Bohort ouviu esta fala, então caiu sobre seus joelhos e
colocou suas mãos juntas contra o céu e falou: “Bendito sois Tu, que me chamais
para Teu serviço”. Então veio para Leonel, que ainda lá jazia tanto tempo
quanto era, e falou para ele: “Amado irmão, aqui fizestes mal do cavaleiro
vosso companheiro que matastes e deste eremita. Por Deus, não segui daqui antes
de os corpos estarem sepultados, e seja-lhes feita tão grande honra como se deve
simplesmente lhes fazer”. E Leonel falou: “O que quereis fazer, não quereis
ficar tanto tempo até que sejam sepultados?” “Não,” falou Bohort, “eu quero
seguir para o mar onde Parsifal me espera, como me deixou entender a voz
celestial”. Pois que tinha falado esta fala, então saltou acima e fez uma cruz
em sua testa e pediu a Nosso Senhor que o quisesse conduzir. Então foi ao seu
cavalo e colocou a sela e o arreou. E pois que estava pronto, então montou e
seguiu de lá e veio à noite a um convento, onde ficou albergado. E quando
estava adormecido, então ouviu uma voz que lhe dizia: “Levanta, Bohort, e
prepara-te”,  e ele se preparou, por causa de que não se ficasse alerta
de que ele queria cavalgar àquela hora. E então foi olhar todo ao redor se
poderia sair, tanto tempo até que encontrou atrás do muro um pedaço quebrado
onde ele tinha bom caminho. Então veio ao seu cavalo e montou e cavalgou para o
buraco e seguiu através dele. Assim se apartou de lá, que ninguém de
apercebesse dele, e cavalgou tanto até que veio ao mar, e lá encontrou à margem
um navio, coberto com samítico branco. Então se afastou e encomendou-se a Nosso
Senhor Jesus Cristo. E tão logo lá tinha entrado, então viu que o navio seguia
da margem, e o vento batia na vela, que estava no navio, que lhe pareceu que
voava no ar. Quando viu que lhe tinha sido esquecido colocar lá o seu cavalo,
então deixou ser e olhou ao redor no navio, quando não viu senão a noite, e a
noite estava escura e sombria, e por causa disto não conseguia ver bem. Então
veio a bordo do navio e lá se ajoelhou e pediu a Nosso Senhor que para lá o
conduzisse, onde podia conservar sua alma. E pois que fez sua oração, então
adormeceu até o dia.

Quando estava desperto, então olhou no navio e viu um cavaleiro, armado com
todas as suas armas, sem o elmo, que estava defronte dele. E pois que o tinha
visto, então reconheceu que era Parsifal. Todo de pronto ele o agarrou e teve,
por sua causa, grande alegria. E Parsifal de todo se afastou dele, que o viu
diante de si, quando não soube como tinha vindo ali. Então lhe pergunta quem
ele seria. “Como é isto,” falou Bohort, “não me conheceis?” “Seguramente eu
não”, falou Parsifal, “e maravilha-me muito como para cá viestes, pois Deus
vos transportou aqui”. E Bohort começou a rir desta fala e tirou seu elmo,
então ele o contemplou e o reconheceu. Então não seria fácil dizer a alegria
que tiveram entre eles. E Bohort começou a contar-lhe como teria vindo ao navio
e por causa de qual coisa. E Parsifal contou-lhe de novo as aventuras que se
lhe tinham sucedido, no penhasco em que tinha estado, que o diabo lhe tinha
vindo à maneira de uma mulher, que o trouxe até o pecado mortal. Assim estavam
os dois um com o outro. E então deixa a fala de falar deles e volta para o bom
cavaleiro Galaat.

\chapter{A irmã de Parsifal}

E\textsc{ntão nos diz} a aventura que, pois que o bom cavaleiro estava separado de
Parsifal, pois que o tinha resgatado dos vinte cavaleiros que o tinham
capturado, que se pôs na alta via do ermo e seguiu errante por alguns dias, um
à sua frente, outro atrás de si, assim como a aventura o conduzia. E
encontrou lá no meio algumas aventuras, que trouxe ao fim, que aqui não se
conta, quando lá seria muito de se conseguir, que cada um quisesse contar
especialmente. Pois que o bom cavaleiro tinha cavalgado por um bom tempo
através do reino de Logres em todos os sítios que ouviu chamar que a aventura
lá estivesse, então seguiu de lá e cavalgou para o mar. Então veio que ele
cavalgou à frente de um castelo, onde havia um maravilhoso torneio. Quando os
de dentro tinham, de verdade, feito tanto que fugiram em direção ao castelo,
quando os de fora eram muito melhores cavaleiros que os outros. 

Quando Galaat viu que os de dentro estavam derrubados, tanto que se os matasse
na frente do castelo, então volveu para eles e pensou que queria ajudá-los.
Então afundou sua lança e tomou o cavalo com as esporas e picou o primeiro que
cavalgou contra ele tão duramente que o fez abalar-se, depois conduziu sua mão
à sua espada como aquele que bem conseguia com isto ajudar. E pôs-se lá, onde
era de todo denso e começou a abater cavaleiro e corcel, que ninguém o viu que
o tomasse por um nobre. E o senhor Gawin, que ali veio para o torneio, e
Hector, que ajudaram os de fora. E tão logo viram o escudo branco com a cruz
vermelha, então falou um para o outro: “Vede lá o bom cavaleiro, ele seria um
portão que perdurasse, quando ninguém pode resistir à sua espada”. Nisto que o
falaram, então veio Galaat irrompendo contra o senhor Gawin, como para lá o
trouxe a aventura. Então o golpeou tão duramente que lhe fendeu o elmo e a
viseira. E o senhor Gawin, que lá considerou estar morto do golpe, caiu do
cavalo, e Galaat não pôde conter seu golpe e encontrou o corcel e o rompeu
através dos ombros, tanto que o matou embaixo do senhor Gawin.

Quando Hector viu que o senhor Gawin estava a pé, então se pôs abaixo dele por
causa de que o queria proteger e amar como seu sobrinho. E ele seguiu para cima
e para baixo e fez tanto em curta hora que os de dentro ficaram recobrados, os
que antes estavam derrubados. E não se deteve de golpear e de espetar, até que
os de fora estivessem derrubados com reta força. E eles fugiram para lá, onde
julgavam em verdade conservar-se, e ele os caçou por um bom tempo. Pois que viu
que nenhum deles queria voltar, então seguiu caminho tão calmamente que lá
ninguém conseguiu dizer para onde tinha rumado. Então conduziu de ambos os
lados o louvor e o prêmio do torneio. E o senhor Gawin estava tão dolorido do
golpe, que ele lhe tinha dado, que nunca pensou em vir de lá vivo. E falou para
Hector, que estava à sua frente: “Em minha verdade, tornou-se então verdade a
fala que por último me foi dita no dia de Pentecostes do pilar e da espada, à
qual eu tinha lançado mão, que eu ainda deveria receber um golpe da espada
antes que o ano saísse, que eu não queria ser assim golpeado ao redor de um
castelo. Em minha verdade, é a mesma espada com que o cavaleiro me golpeou.
Então posso bem falar que a coisa me veio como me foi prometido”. “Senhor,”
falou Hector, “o cavaleiro vos feriu como dizeis?” “Seguramente”, falou o
senhor Gawin, “sim, ele, assim que não posso vir sem grande preocupação, Deus
queira então me ajudar”. “E o que fazemos?”, falou Hector. “Pois me parece que
nossa demanda foi derrubada, pois que estais assim ferido”. “Senhor,” falou
ele, “a vossa não está derrubada nem conduzida como a minha; tanto quanto Deus
quiser, sigo-vos de perto”. 

Nisto que assim falaram, então se reuniram todos lá, os cavaleiros do castelo, e
eles reconheceram o senhor Gawin. E quando ficaram alertas de que estava assim
ferido, então ficaram muito irados pela maior parte, quando sem dúvida ele era
mais amado entre os estranhos que nenhum outro homem seria no mundo. Então o
tomaram e o portaram ao castelo e o desarmaram e o colocaram em uma câmara
secreta e longe das pessoas. E depois mandaram chamar um médico e o fizeram
contemplar suas feridas e lhe perguntaram se o poderia curar. E ele lhes
prometeu que queria ajudá-lo e curá-lo em uma lua, de maneira que pudesse
cavalgar e conduzir arma. E o senhor Gawin lhe prometeu que conseguiria
fazê-lo, ele queria dar-lhe assim tanto que seria rico seu dia de vida. E ele
falou que estivesse seguro, quando queria fazê-lo como lhe tinha prometido. E
assim permanece o senhor Gawin lá e Hector junto dele, quando não queria
afastar-se dele até que estivesse convalescido.

Assim cavalga o bom cavaleiro, pois que vinha do torneio, assim como a aventura
o conduzia, que veio à noite bem duas milhas junto de Korbeyn. Então veio que
pernoitou à frente de uma clausura, e desmontou e chamou à frente da porta ao
enclausurado tanto que o deixou entrar. Quando, pois, ele viu que era um dos
cavaleiros aventurosos, então falou o bom homem que ele era bem-vindo. E ele
pediu que o abrigasse e o fez desarmar. E quando estava desarmado, então lhe
fez dar de comer daquilo que Deus tinha aconselhado, e ele o tomou de bom
grado, quando ele durante todo o dia não tinha comido. Pois que tinha comido,
então foi dormir sobre uma trouxa de feno que lá estava, muito perfumada. 

Quando estavam dormindo, então veio uma donzela, que lá bateu à porta e chamou
Galaat. E o nobre foi até a porta o perguntou o que seria então e o que queria
lá dentro àquela hora. “Senhor,” falou ela, “eu quero falar com o cavaleiro que
está aí dentro, quando preciso reunir-me com ele”. E o nobre o acordou e falou:
“Senhor cavaleiro, uma donzela vos quer ter e falar convosco, quando muito bem
ela pode, como me parece”. E Galaat levantou-se e veio a ela e lhe perguntou o
que ela queria. “Galaat,” falou ela, “quero que vos armeis e monteis vosso
cavalo e me sigais, e eu vos digo que devo vos mostrar a mais alta aventura que
nenhum cavaleiro já viu”. Quando Galaat ouviu esta notícia, então foi às suas
armas e armou-se. Pois que tinha conduzido a sela sobre seu cavalo, então
montou e encomendou o eremita a Deus e falou para a donzela: “Então podeis
seguir aonde vos pareça bem, e eu vos sigo, a que sítio fordes”. E ela rumou tão
logo pôde com seu palafrém, e ele a seguiu de perto; e tinham cavalgado tanto
tempo que começou o dia. E pois que o dia estava belo e claro, então vieram a
um ermo que lá ia até o mar, e era chamado Celikle. Então cavalgaram a alta via
o dia inteiro, em medida que nunca comeram bocado nem beberam. 

À noite, depois da hora da ceia, vieram a um castelo, que lá estava em um vale e
era muito bem guarnecido com todas as coisas, e estava fechado com águas
correntes e com muros altos e fortes e com covas fundas, altas e grandes. E
quando os de dentro os viram, então começaram a falar todos: “Sede bem-vinda,
Senhora!” E receberam com grande alegria como a que era sua senhora. E falou
que eles fizessem alegria ao cavaleiro que com ela tinha vindo, “quando é o
mais nobre cavaleiro que já portou arma”. E correram para ele e o desarmaram
tão logo ele estava desmontado. E ele falou para a donzela: “Senhora, devemos
permanecer aqui dentro esta noite?” “Não,” falou ela, “tão logo tenhamos
comido e dormido um pouco, assim devemos cavalgar”. Então foram comer e depois
foram dormir. Tão logo tinham dormido o primeiro sono, então ela chamou Galaat:
“Senhor, levantai!”, falou ela. E ele o fez. E os do castelo trouxeram velas,
para que ele se visse armar. E ele montou seu cavalo. E a senhora tomou uma
caixa, muito bela e rica, e a colocou à sua frente. E pois que estava montada,
então se apartaram do castelo e rumaram tanto até que vieram ao mar. E pois que
lá tinham vindo, então acharam o navio onde Bohort e Parsifal estavam dentro e
a esperavam no canto do navio. E não se calaram quietos, quando chamaram de
longe: “Ah, Galaat, como vos esperamos por tanto tempo! Então vos temos,
louvado seja Deus, e vinde cá à frente, quando aqui não há outra coisa além de
que queremos rumar para a alta aventura, que Deus nos preparou”. E pois que os
ouviu, então lhes perguntou quem eles eram, que por tanto tempo o esperaram;
então pergunta à donzela se deveria desmontar. “Senhor,” falou ela, “sim, e
deixai bem aqui vosso cavalo, quando também devo deixar aqui o meu”; então se
afastaram. E ele tirou a sela e o arreio de seu cavalo e também do cavalo da
donzela. Então ele faz uma cruz à frente de sua testa e encomendou-se a Deus e
foi ao navio e a donzela com ele. E os dois companheiros os receberam como com
grande alegria como conseguiram. Então o navio começou a rumar muito rápido
através do mar, quando o vento soprou bastante a vela. E rumaram tão longe em
curta hora, que não viram nenhuma terra nem longe nem perto. E por fim
reconheceram-se e choraram os três da alegria que tinham de que se tinham
encontrado.

Então Bohort retirou seu elmo e Galaat, o seu, e também sua espada; mas sua
coifa não quis tirar. E pois que viu o navio tão deslumbrante na parte de
dentro como na parte de fora, então perguntou a seus dois companheiros, de onde
viria o navio deslumbrante. E Bohort falou que não sabia. E Parsifal lhes
contou que sabia daquilo e disse-lhe\footnote{ Nesta passagem, o pronome no
dativo  torna-se singular.}  como lhe aconteceu no
penhasco, e como o bom homem, que lhe parecia ser um padre, tinha-o feito ali
entrar. “E disse-me que eu não deveria permanecer muito tempo, que deveria
ter-vos em minha companhia; quando desta donzela não me diz nada”.
“Seguramente, aqui dentro eu nunca teria vindo, não o tivesse feito esta
donzela”, falou Galaat, “que para cá me dirigiu. Por causa disso bem quero
falar que vim mais por causa dela que por minha causa; quando por esta via não
vinha nunca mais, e de vós, dois companheiros, e esperava nunca mais ouvir
falar em sítio tão estranho como este”. E então se calaram todos calmos. E
então contaram um ao outro de suas aventuras tanto que Bohort falou para
Galaat: “Senhor, seria então meu senhor, senhor Lancelot, aqui vosso pai,
assim me parece que não nos aproveita”. E Galaat diz: “Não pode ser, pois não é
a vontade de Deus”. 

Nas falas ficaram tanto tempo que eram as nonas horas. Então bem puderam a um
bom traço do reino de Logres, quando o navio tinha seguido o dia inteiro e à
noite a toda a corrida. E então aterraram entre dois rochedos, em uma ilhota,
que era muito selvagem e tão escondida, que era maravilha; e sem dúvida, era um
braço do mar. E pois que lá tinham aterrado, viram um outro navio sobre uma
penha, aonde não conseguiriam vir senão a pé. “Caros Senhores, no navio está a
aventura”, falou a donzela, “por cuja causa que Nosso Senhor nos trouxe juntos,
e lá dentro deveis ir”. E eles falaram que o fariam de bom grado. Então saíram
do navio e também portaram a donzela para fora. E pois que vieram junto ao
navio, então o acharam muito mais rico que aquele de que tinham saído. Quando
muito se maravilharam que não viram ninguém dentro. Então foram para mais
perto, se viam alguma coisa. E olharam a bordo do navio e viram letras escritas
na parte de dentro, que lá falavam à vez uma fala estranha e assustadora a
todos que quisessem entrar. E estava falado desta maneira:

“Escuta, homem que quer entrar aqui, sejas quem for, protege-te que sejas
completamente verdadeiro, pois eu não sou senão fiel. Por causa disto, assim te
protege que sejas por alguma coisa maculado, quando sou puro e completamente
fiel, e por causa disto te protege antes que entres, que não sejas maculado.
Pois tão logo te lances fora da fé verdadeira, então te lanço para fora de tal
maneira que não tenhas de mim nenhum socorro, quando te ordeno com isto, em
qual sítio és agarrado em descrença, ainda que sejas pouco agarrado”.

Pois que viram as letras e as reconheceram, então um contemplou o outro. Então
falou a donzela para Parsifal: “Sabeis quem eu sou?” “Seguramente”, falou
ele, “não, eu não vos vejo nunca mais do que me parece”. “Sabei”, falou ela,
“que sou vossa irmã e uma filha do rei Pelehen.\footnote{ Provável grafia
alternativa para o rei Peles, constando do documento original.} E
sabeis por que me fiz por vós reconhecer? Por causa de que me façais mais
crença do que devo vos dizer. Eu vos digo”, falou ela, “em primeiro como uma
coisa que mais amo, que não sois totalmente crente em Deus, que no navio não
entreis de maneira nenhuma, pois sabei por verdadeiro que de pronto
parecereis. Quando o navio é uma coisa tão alta que, quem lá é maculado com
coisas más, esse não deve permanecer senão em inquietação”.
Pois que Parsifal o ouviu, então a contemplou e refletiu tanto que reconheceu
que ela era sua irmã. Então ele lhe faz grande alegria e falou: “Pois amada
irmã, eu devo lá entrar, e sabeis por quê? Por causa disto: é que sou
incrédulo, que pereço como um infiel, e é que sou cheio de verdade e assim como
um cavaleiro deve ser, que eu seja então conservado”. “Então entrai
livremente”, falou ela, “que Deus vos deve ser um auxiliador e um protetor!”
Nisto que ela assim falou, Galaat, que lá era o mais digno, alçou sua mão e
persignou-se e entrou, e pois que estava dentro, então começou a olhar de um
fim a outro. E a donzela entrou depois, e faz uma cruz à sua frente. E quando
Bohort viu isso, então não se obstou, quando também virou para dentro. E quando
bem tinham visto abaixo e acima, então falaram que não consideravam que no mar
ou fora do mar houve um navio tão belo, nem tão rico quanto lhes parecia ser.
Pois que tinham procurado por toda a parte, então viram na parte do meio do
navio um pano muito rico, espraiado em igualdade a uma cortina, e abaixo uma
cama muito deslumbrante e rica com grande maravilha.

Galaat foi ao pano e o ergueu e viu a mais bela cama que já tinha visto. Quando
a cama era bela e rica, e estava à cabeceira uma coroa de ouro e, nos pés,
jazia uma espada, que era bela e clara, e no meio da cama, afastada bem meio pé
da bainha. A espada era feita por tal mão, e o cabo era de uma pedra, que tinha
nela todas as cores que se podem encontrar no reino da terra. E ainda tinha
outras piedades que eram melhores, quando cada cor tem nela uma virtude. E a
aventura nos diz que a pedra era de um rochedo em serpente, que mais duram na
Calcedônia que em outras terras, e a serpente é chamada Papalides. O
cabo era feito de dois animais, o primeiro animal era o \textit{linckus}.
Quando a pessoa que lá tem dele uma costela ou uma perna e a traz junto a ele,
não pode se preocupar que ganhe grande calor.

Das naturezas era um lado, e o outro era de uma perna de peixe. Não era grande
demais, e está na água que lá se chama Eufrates e em nenhuma mais, e o peixe é
chamado \textit{Archenons}. E suas costelas são da natureza assim: se tem
alguém um, tanto tempo quanto o tenha tido, não pensa em nenhuma coisa, nem
alegrias nem tristezas que teve, senão na coisa por cuja causa ele o tomou. Das
virtudes eram as duas coisas que lá estavam no cabo da espada. Estavam cobertas
com um pano vermelho, que era muito rico, cheio de letras que lá falavam: “Eu
estou maravilhado para ver e para reconhecer, quando nenhum homem conseguiu de
envolver, fossem suas mãos de qualquer grandeza, nem nunca o fará, senão apenas
um. E ele deve superar com sua obra manual aqueles que antes dele estavam e
todos os que devem vir depois dele”. 

E assim falavam as letras do cabo. E tão logo eles as tinham lido, que lá
suficientemente conseguiram a escrita, então um contemplou o outro, e falaram:
“Aqui se pode ver maravilha”. “Em nome de Deus”, falou Parsifal, “eu devo
contemplar se poderia envolver a espada”. Então lançou sua mão sobre ela, e não
conseguiu envolver o cabo. “Em verdade,” falou ele, “então bem creio que estas
letras dizem verdade”. Então Bohort lançou sua mão, e não conseguiu à vez
envolver. Pois que viram isso, então falaram para Galaat: “Senhor, tentai
também vós nessa espada, quando bem sabemos que deveis executar esta aventura
em que nós faltamos”. E ele falou que não queria tentar, quando veria
suficientemente maior maravilha do que já viu. Depois ele contemplou a lâmina
da espada, que lá estava retirada tanto ouvistes e viu letras, que eram
vermelhas. Elas falavam: “Nenhum seja tão ousado que me retire da
bainha, que deve fazer então mais simplesmente que nenhum outro e muito mais
ousadamente. E aquele que me retirar de outro modo, deve bem saber que não pode
sair da morte ou de ferimento. E esta coisa foi agora provada muitas vezes”.
Quando Galaat viu esta coisa, então falou: “Em verdade, quero ter retirado esta
espada, e desde que a proibição é tão grande, assim lanço a ela minha mão”.
Assim falaram também Parsifal e Bohort. “Caros Senhores,” falou a
donzela, “sabei que o retirar é a todos proibido que não ele, e eu devo vos
dizer como veio, não é longo.

 É verdade”, falou a donzela, “que o navio aterrou no reino de
Lůgůße,\footnote{Hans-Hugo Steinhoff traduz este nome por \textit{Logres}, porém
a prudência de permitir ao leitor uma conclusão própria aconselha a conservá-lo
como consta do documento original.}  e ao tempo lá havia grande
guerra entre o rei Lambral, que lá era pai do rei que se chamava Mahamen, e o
rei Ülant, que lá foi por todos os seus dias pagão. Quando então se tinha
tornado cristão, tanto que se o tinha por um dos homens mais nobres que se
achavam no mundo. Um dia aconteceu que o rei Lambrans\footnote{ Grafia
alternativa a \textit{Lambral}, constante do próprio documento original.
 } e o rei Ulans\footnote{ Grafia alternativa ao nome \textit{Ülant},
também encontrada no texto medieval.}  vieram juntos ao mar onde o
navio tinha rumado para a terra. Então sucedeu que o rei Ulans era vindo para o
mais irritado e sua gente lhe foi morta. Então se temeu muito que lá devesse
morrer. Então veio ao navio e saltou dentro. E pois que tinha encontrado a
espada, então a retirou e saltou fora do navio e encontrou o rei Lambral; era o
mais leal homem do mundo e onde\footnote{ O documento medieval apresente o
pronome \textit{da, }que significa “onde”. Por rigor de tradução, manteve-se o
pronome, ao invés da locação pronominal correta no português contemporâneo “em
que” ou “no qual”.}  Nosso Senhor Deus mais tinha parte. E quando o
rei Ulans viu Lambrel\footnote{ Outra variante para o nome \textit{Ülant},
apresentada no documento medieval.},  então ergueu a espada e o
golpeou em cima, sobre o elmo, tão duramente que o partiu e seu cavalo até por
sobre a terra. E tal foi o primeiro golpe da espada, e aconteceu no reino de
Logres. Então veio tão grande morte e tão grande perecimento nos dois reinos,
que nunca depois a terra conseguiu pagar aos trabalhadores seu trabalho. Quando
desde então nunca cresceu na mesma nenhum grão, nem as árvores portaram nenhum
fruto, nem nas águas foram encontrados peixes como outrora. E por causa disso
se chamou a terra dos dois reinos a Terra Esgotada, por causa de que por causa
cruel golpe, estava assim queimada e destruída.\footnote{ O texto original já
apresenta tal construção frasal pleonástica.}  

Quando o rei Ulans viu assim muito cortante, pensou que queria voltar a portá-la
em sua bainha. E quando ele veio ao navio e entrou e enfiou a espada em sua
bainha e tão logo o tinha feito, então caiu à frente da cama e estava morto.
Assim foi provado que essa espada, que ninguém retiraria, que não morresse por
causa disto ou fosse ferido. Então permaneceu o corpo do mesmo rei à frente
dessa cama tanto tempo até que uma donzela o lançou fora. Quando lá nenhum
homem foi tão ousado que ousasse lá entrar por causa da proibição que as letras
junto ao bordo faziam”. “Em minha verdade”, falou Galaat, “foi
uma bela aventura, e eu bem creio que assim aconteça, e não o duvido, que essa
espada seja suficientemente maravilhosa e mais que nenhuma outra”. E então
caminhou à frente, para que a retirasse. Então falou a donzela: “Deixai estar
um breve momento, até que tenhamos visto esta maravilha”. E então começaram a
contemplar a bainha, quando não conseguiam saber do que poderia ser feita, que
não fosse feita de pele de serpente. E não por causa disto, eles viram que ela
era, porém, vermelha como de pétalas de rosas vermelhas, e sobre ela estavam
escritas letras, umas de ouro, outras de prata. Quando então adveio que
deveriam contemplar a articulação da espada, não havia nenhum deles que não se
maravilhasse mais do que antes tinham feito. Quando viram que a articulação não
se equiparava a nenhuma igual, quando eram de tão débeis naturezas e de tão
horrendas coisas, retamente como se de obra de cânhamo, e eram também tão
pretos que lhes pareceu, como parece, que não conseguiam segurar a espada lá
junto. E as letras da bainha diziam: “Aquele que deve me portar, deve ser muito
mais astuto que nenhum outro, que me porte pura como simplesmente me deve
portar. Quando não devo estar em nenhum sítio onde esteja a impureza ou o
pecado devam estar; quem para lá me portar ou colocar, deve bem saber que deve
ser o primeiro a quem isso vale. Quando me proteger e me mantiver pura, então
pode seguramente ir sem preocupação, quando o corpo, em cujo lado eu pendo, não
pode ser maculado tanto tempo quanto esteja afivelado com a fivela da espada,
onde eu pendo. Nem ninguém seja tão ousado que desfaça essa articulação por
causa de nenhuma coisa, quando não é imaginado com nenhuma pessoa que então
viva nem deva vir disto; quando não devem ser desfeitas senão pelas mãos de uma
donzela que lá seja filha de um rei e de uma rainha. Ela pode fazer uma troca e
pode pôr outra em seu lugar, a mais adorável que saiba no mundo, e isso ela
deve fazer neste sítio. E então a donzela precisa todos os dias, por que viva,
ser uma pura moça em vontade e em obras. E se adviesse que ela perdesse sua
inocência e sua pureza, assim ela pode estar certa de que morre como em
perversa morte como nenhuma mulher pode morrer. E a donzela deve tomar a espada
com seu nome e a mim, com o meu; ainda nunca, antes que isso aconteça, ninguém
deve ser chamado ousado”. 

 Pois que tinham lido as letras, então começaram a rir e falaram que seria
maravilha de ver, de ouvir e de saber. “Senhor,” falou Parsifal, “volve em
torno da espada!” E ele volveu ao redor dela pelo outro lado. E pois que
estava virada ao contrário, então viram que era mais vermelha que sangue. E
sobre ela havia letras, que falavam: “Aquele que, acima de tudo, louvar-me,
deve encontrar-me acima de tudo para ralhar, e para ele devo ser a melhor de
todas, para ele devo ser de todo a mais irritada. E isso não deve acontecer
senão em uma hora, quando assim deve ser com poder”. Quando a donzela viu isso,
então falou para Parsifal: “Amado irmão, estas duas coisas aconteceram muitas
vezes. Eu devo vos dizer quando isso aconteceu e com quais pessoas, por causa
de que ninguém deve temer-se de tomar essa espada, que seja digno dela.

 Aconteceu bem quatorze anos depois do martírio de Nosso Senhor Jesus Cristo que
Nasiens, cunhado do rei Morderans, foi guiado em uma nuvem mais que trinta dias
de distância através de um mandamento de Nosso Senhor, em uma ilha na terra em
direção do ocidente, que lá era chamada Tornoant. Pois que lá veio, lá
encontrou esse navio junto a um penhasco. E quando entrou nele, e tinha
encontrado essa espada e essa cama, então as contemplou por longo tempo e lhe
aprouve assim muito de ver que era maravilha, e porém não teve ele a astúcia,
que ousasse tirá-la. Então caiu em grande desejo e a teria tido de bom grado e
permaneceu bem lá oito dias no navio sem comer e sem beber, que não fosse muito
pouco.

No nono dia, então adveio que um grande e maravilhoso vento o tomou e o fez
rumar da ilha de Tornoant e o guiou a uma outra em direção contrária ao
oriente, bem longe daquela. E então aterrou justamente à frente de um penhasco,
e quando veio à terra, então encontrou na ilha um gigante, o maior de todos e o
mais maravilhoso do mundo, que lhe gritou: ‘Tu deves estar morto!’ E este
temeu morrer, e olhou a seu redor, quando não viu nada de nada, com o que se
pudesse proteger. Então correu para a espada e a tirou da bainha. E pois que
ele viu isto, então louvou assim muito que não queria louvar mais nenhuma
coisa. Então começou a clamar contra a montanha; quando com o primeiro
estremecimento veio que se quebrou ao meio em duas. Então ele falou que a coisa
que ele teria muito louvado em primeiro lugar, que deveria então muito ralhar,
e com direito, quando o tinha precipitado em grandes necessidades. Depois assim
a coloca de novo sobre a cama e foi lutar com o gigante e o matou. Depois veio
de novo ao navio. E como o vento se tinha batido na vela, então seguiu ele
tanto tempo com aventura no mar que o encontrou um outro navio, que lá era do
rei Morderans, que lá tinha voltado muito e estava incomodado no rochedo do
bordo perigoso. 

Quando um viu o outro, então se alegraram os dois muito à vez, justamente como
aqueles que se tinham amor, com grande amor. Então perguntaram eles um ao outro
de suas aventuras e tanto que Nasiens falou: ‘Senhor, eu esqueço o que me
falais das aventuras do mundo. Quando desde que não mais me vistes, assim vos
digo que me sucedeu uma das mais maravilhosas aventuras do mundo, que nenhuma
pessoa ainda pôde antepor’. Então lhe contou o que aconteceu com a espada e como
ela lhe foi quebrada em grandes necessidades, em que ele considerava matar o
gigante. ‘Em minha verdade’, falou ele, ‘dizeis-me maravilha! O que fizestes
com a espada depois?’ ‘Senhor,’ falou Nasiens, ‘eu a coloco de novo lá dentro
de onde a tomei, e podeis ir ver, se for de vosso agrado, quando está lá
dentro’. Então caminhou o rei Morderans ao navio e foi para a cama. E quando
viu os pedaços, então os louvou mais que nenhuma coisa que já viu e falou que
isto não teria acontecido por causa da maldade da espada, senão por causa de
muito significado ou por causa do pecado de Nasiens. E tomou os dois pedaços e
os colocou juntos. E tão logo os dois pedaços vieram um junto ao outro, então
saltou a espada junta de novo tão levemente como estava quebrada em dois. E
quando ele viu isso, então começou a rir e falou: ‘Seguramente, é um milagre da
graça de Nosso Senhor Jesus Cristo, quando a quebrou e a expiou muito mais leve
do que alguém poderia considerar’. Então ele enfiou a espada de novo na bainha
e a colocou dentro onde agora vedes. E de pronto ouviram uma voz que para eles
falou: “Saí deste navio e ide ao outro, quando por causa do temor de que não
caiais em pecado; e se fôsseis encontrados em pecado enquanto estais aqui
dentro, não poderão sair da inquietação do pecado’. Bem lá saíram do navio e
caminharam para o outro. E quando Nasiens saiu de um para o outro, então foi
golpeado com uma espada entre seus ombros tão duramente que caiu para trás de
si. E com a queda que ele fez, falou: ‘Ah, Deus, como estou ferido!’ Então
ouviu uma voz que falou para ele: ‘Isto é pela falta que tu fizeste com a
espada, quando não eras digno de retirá-la. Então te protege melhor uma outra
hora que não vás de novo contra teu Criador’. Na maneira assim veio essa fala
que aqui está escrita: ‘Aquele que mais me louve, deve mais que tudo encontrar-se
a ralhar em grandes necessidades’”. “Em nome de Deus,” falou
Galaat, “isto nos mostrastes bem. Então nos dizei a outra coisa que aconteceu”. “De
bom grado”, falou a donzela.

“É verdade”, falou ela, “que o rei Barlans, que se chamava rei Mahagine, tanto
quanto quis cavalgar, assim amava muito a santa cristandade e honrava as
pessoas pobres mais do que nenhuma pessoa fazia, e ele era de tão boa vida que
não se achava seu igual em todo o mundo. Um dia foi ele caçar em um de seus
arbustos, que lá se estendia até o mar. E assim perdeu seu cão e sua presa que
tinha prendido e seus cavaleiros até que ficou um só. E quando viu que
tinha perdido sua companhia, então não soube o que deveria fazer e cavalgou tão
profundamente no ermo que não conseguia dele sair, quando não tinha
conhecimento do caminho. Então se puseram a caminho, ele e seu cavaleiro, e
cavalgaram tanto tempo até que vieram a uma margem do mar que se estendia até a
Irlanda. E pois que para lá tinha vindo, então ele encontrou o navio onde agora
estamos dentro. Então ele veio a bordo e encontrou estas letras que agora
vimos. E quando as viu, então nunca se assustou, como aquele que se soubesse
não culpado contra Nosso Senhor Jesus Cristo, e ele era repleto de todas as
virtudes que jamais nenhum cavaleiro sobre a terra pôde ter. Então foi
sozinho ao navio, quando o cavaleiro, que estava com ele, não se atreveu que
entrasse lá com ele. Quando viu a espada, então a tirou da bainha, tanto quanto
pôde ver; quando antes do tempo não viu com tudo da lâmina. Quando tinha à vez
a retirado sem resistência, quando veio para lá rumando um brilho, com que ele
foi espetado em ambos seus joelhos, tão duramente que é bem visível, e desde
então nunca esteve saudável e nunca ficou saudável, até que viésseis a ele.
Então ficou ferido por causa da ousadia que ele fez, e ele é vosso ancestral. E
por causa da vingança, então se fala que seja mal àquele a quem deveria ser
bem, quando ele era uma dos melhores cavaleiros e mais nobres homens que jamais
se pôde encontrar”. 

“Em nome de Deus”, eles falaram, “donzela, disseste-nos tanto que não se deve
deixar de tomar essa espada por causa das letras que sobre ela estavam
escritas”. Então contemplaram a cama e viram que estava cheia de madeira. E no
meio da cama, à frente, estava um fuso, que estava acoplado através da madeira,
que lá estava na parte mais anterior da cama, no meio, e estava novamente bem
escondida. Então havia um outro fuso, enfiado do outro lado da cama, e um fuso
estava tão longe do outro quanto era larga a cama. E sobre os fusos assim um
estava colocado, que estava acompanhando os outros dois, e o fuso inferior era
tão branco como uma neve, e o posterior era mais vermelho que nenhum sangue, e
aquele que lá estava mais acima era ainda mais verde que uma grama. E dessas
três cores estavam esses três fusos sobre a cama. E sabei que essas três cores
naturais eram especialmente fundidas, quando lá não tinham sido colocadas por
meio de nenhum homem mortal. Por causa de que muita gente podia ouvir, que
puderam tê-las por mentira, fez-se que não se pudessem saber como poderia ser,
e por causa disso a fala se volta um pouco de seu reto caminho e de suas
matérias, para falar os três fusos que lá eram das três cores. 

Então nos diz a aventura do Santo Graal bem aqui que aconteceu que a primeira
pecadora, que lá foi a primeira mulher, tinha tomado conselho ao Inimigo
mortal, que era o Diabo, que lá, à hora, começou a invejar a linhagem humana e
a matar. E ele tinha feito engolir do pecado mortal, que era de vaidade e
cortesia, por cuja causa que foi lançado fora do Paraíso e empurrado da grande
alegria do Céu. E ele os trouxe, com sua vontade desleal, a que fizessem
quebrar o fruto mortal, e da mesma árvore tinham tomado um galhinho, como denso
de camadas, que o galho ficou junto ao fruto que se quebrou. E tão logo ela lhe
fez provar o que tinha conseguido e insuflado, e o empurrou à terra com suas
mãos, da maneira como o trouxe do galho como ouvistes, então aconteceu que o
galho permanece. Tão logo tinham comido do fruto mortal, que simplesmente deve
ser chamado mortal, quando por meio dele veio a morte sobre esses dois e depois
para os outros, e então se transformou todo o seu ser, e viram que eram de
carne e estavam nus, os que antes não eram senão uma coisa espiritual, ainda
que tivessem corpo. Porém assim não estima a aventura que fossem, à vez,
espirituais. Quando uma coisa lá é formada de coisas tão débeis como de terra, não
pode se tornar pura. Quando lá foram forjados como uma coisa espiritual para
viver por toda a via, teria sido que se tivessem mantido sem pecado. E pois que
depois olharam, então se viram nus, então cobriram suas vergonhas e
envergonharam-se, um perante o outro. Temeram agora tanto por seu erro e
cobriram cada qual suas vergonhas com suas duas mãos.

Eva mantinha em sua mão o galhinho e nunca quis deixá-lo, nem antes, nem depois.
Quando Aquele que sabe todos os pensamentos foi alertado de que eles tinham
pecado, então veio até eles e falou primeiro a Adão, quando era certo que ele
era mais para se culpar que sua mulher. Quando ela era de compleição muito mais
débil que ele, justamente como aquela que foi formada de uma costela do homem,
e era certo que fosse a ele obediente e não ele a ela, e por causa disso chamou
primeiro a Adão. E pois que tinha falado esta fala: “Deves comer teu pão em
trabalho e em suor”, então não queria largar sua mulher, que seria participante
do martírio, assim como tinha mostrado a via para o erro, e falou para ela: “Em
tristeza e dor deves ganhar tua criança”. Depois jogou a ambos para fora do
Paraíso, que a Escritura chama de Paraíso das alegrias. E pois que estavam
fora, então ainda tinha Eva o galhinho em sua mão, que ainda àquele tempo nunca
contemplou. Quando então se contemplou e ao galhinho, então ficou pensativa por
causa de que ainda o via verde, como não tinha sido arrancado há muito tempo,
então ela bem sabia que o galho, que ela tinha arrancado, era uma coisa de sua
perdição. Então ela falou que em igualdade a tão grande padecimento, que lhes
veio por meio da árvore, então deveria manter o galho tanto quanto jamais
pudesse e também da maneira que o pudesse ver farto, para um reconhecimento de
seu azar.

Então pensou Eva que ela não tinha nem caixa nem escrínio, onde pudesse
conservá-lo, quando ao mesmo tempo não tinha uma tal coisa. Então o enfiou à
terra e falou que o queria ver assim farto e suficiente. O galho, que lá estava
enfiado na terra, cresceu e permaneceu pela vontade de Nosso Senhor. O galho,
que a primeira pecadora trouxe do Paraíso, era de grande significado, quando
nisto que ela o portou em sua mão, que significa uma grande felicidade,
justamente como deveriam dizer aos que viessem depois deles, quando era ainda
uma virgem. E o galho significa justamente como ela lhes falou: “Então não vos
volteis para que fomos expulsos de nossa herança, quando não a perdemos de todo
o modo; vede bem aqui o símbolo de que em algum momento ainda devemos
ganhá-la”. E ele que lá quis perguntar no Livro, por qual causa o homem não
trouxe o galho para fora do Paraíso, quando o homem é uma coisa muito mais
elevada que uma mulher, ao que ele respondeu e falou que o portar do galho
pertence não ao homem, senão à mulher. E isso é: a mulher o portou
simplesmente, quando por meio dela a vida foi perdida, e por meio da mulher a
vida deve novamente ser encontrada. E era um significado que por meio da Virgem
Maria a herança deve ser ganha de novo, que lá estava perdida até então.

Depois nos diz a aventura do galho que lá estava empurrado à terra, e fala que
ele se tornou muito grande em curta hora. E pois que se tornou grande, então era
branco no tronco e nos galhos e nas folhas, e isto é um significado da castidade
e uma virtude, e o corpo era conservado puro e a alma branca e era pura em
todas as virtudes. Isto significa: a que o trouxe era uma virgem casta. Ao
tempo em que Adão foi expulso do Paraíso, então eram ambos castos e puros.
Porém sabei que a castidade e a virgindade não são iguais em virtudes, quando
há uma grande diferença entre elas. Quando a virgindade não se pode igualar à
castidade, e vos digo por qual causa. A virgindade é uma virtude, que todos não
têm, que tenham dormido com outro, com sociedade carnal. E castidade é coisa
muito mais elevada e virtuosa, quando ninguém a pode ter, seja mulher ou homem,
conforme tenha pensamento e vontade para a falta de castidade. E a castidade
ela ainda tinha quando foi empurrada do Paraíso e fora da grande delicadeza que
havia lá dentro. E ao tempo em que plantaram o galho, então não tinham quebrado
a castidade. Quando depois mandou Deus que ele teria que se forjar com sua
mulher, isto é, que ele deveria dormir junto a ela, assim como a natureza
deseja, isto é, que o homem deve estar junto de sua mulher.

Então ela tinha perdido a castidade; depois, à frente, pois que ele teve
sociedade carnal com ela, tanto que aconteceu, conforme isso, que ele a
reconheceu que ela tinha descanso em baixo da árvore. E Adão começou a vê-la e
começou a se queixar de seu arrependimento. Então começaram os dois a gritar,
um por causa do outro. Então falou Eva que não seria maravilha que lá tivessem
conhecimento de seu padecimento, quando a árvore o tinha em si, quando ninguém
podia ser feliz lá embaixo, e por mais feliz o que lá abaixo viesse, ele
deveria ficar triste. “E com bom direito, assim ele está triste, quando é a
lembrança.” E tão logo ele o tinha falado, então falou uma voz para ele: ``Ah,
vós desgraçados, por que vos condenastes à morte, um de vós não deve distribuir
o outro, quando é mais vida que morte”. Assim falou a voz, que muito os
consolou. E por causa disso a chamaram, doravante, de Árvore da Vida, e por
causa da grande alegria que dela tiveram, então plantaram dela muitas das
árvores. Quando tão logo arrancavam um galho, então o colocavam na terra, e de
pronto crescia, e todos os dias recebiam da doçura da árvore. E ela cresceu
todo o dia e ficou mais bela que nenhuma outra. Então veio que se sentaram
embaixo dela, como antes tinham feito. E aconteceu que um dia se sentavam lá
embaixo, e a aventura nos diz que era uma sexta-feira.

Pois que se tinham por longo tempo sentado um ao lado do outro, então ouviram
uma voz, que lhes falou e mandou-lhes que deveriam ter criação um com o outro.
Então estavam tão grandemente envergonhados, que não queriam ir a uma obra tão
vergonhosa. Quando o homem se envergonhou tanto quanto a mulher, e eles não
sabiam como transpassar o conselho de Nosso Senhor. E o primeiro mandamento os
segurou, e eles começaram a olhar-se muito envergonhados. Então Nosso Senhor
viu sua vergonha, e isso o apiedou, quando por causa disto, de que seu pecado
não podia ser ignorado, e Sua vontade era assim que dos dois queria alimentar a
linhagem humana, para preencher o décimo coro dos anjos, que do Céu foi
empurrado por causa de sua cortesia, e por causa disso assim lhes enviou grande
consolo para sua vergonha. E então se maravilharam muito como a escuridão logo
podia vir em direção deles. 

Então um chamou ao outro e tocaram-se sem ver. E por causa disso, que todas as
coisas foram feitas conforme a vontade de Nosso Senhor, por causa disso deve
ser que se tivessem forjado um com o outro, assim como o verdadeiro Pai lhes
tinha ordenado. E pois que o tinham feito, então tinham feito nova semente, de
que seu grande pecado foi um pouco mitigado. E Adão o tinha concebido e Eva,
sua mulher, a Abel, o Justo, que serviu primeiro a seu Criador com boa devoção,
que Lhe deu lealmente seu dízimo.

Assim foi Abel, o Justo, recebido sob a árvore na sexta-feira, como ouvistes.
Então a escuridão seguiu seu caminho e eles se viram como antes. Então
perceberam que Nosso Senhor tinha feito aquilo por causa de cobrir sua
vergonha, pelo que estavam muito contentes. E bem de pronto aconteceu uma coisa
maravilhosa, que a árvore que lá era antes branca, ficou tão verde como uma
grama, e todas as que dela foram plantadas, todas se tornaram verdes na madeira
e nas folhas.

 Assim ficou a árvore transformada de branco em verde, e as de lá vieram dela,
nunca se transformaram de sua primeira cor, nem nunca se viu a cor verde em
duas, quando apenas em uma; e ela ficou coberta com cores verdes em cima e
embaixo. E do tempo em diante, começou a florescer e a portar frutos. E a que
se transformou da cor branca em verde, isto significa que a castidade tinha ido
embora daqueles que a tinham plantado. E as flores e os frutos significam que
neles estava o fruto e a semente, e que por toda a via deveria estar em serviço
de Nosso Senhor, que é em bom pensamento e em bom amor quanto ao seu Criador. E
que a criatura, que ficou concebida embaixo da árvore, deve permanecer casta e
pura no amor. E os frutos significam que ele deve demonstrar santidade e todo o
bem nas coisas terrenas. Assim esteve a árvore por longo tempo em
cores verdes e todas que dela vieram, até ao tempo em que Abel cresceu. E ele
era tão bom em relação ao seu Criador e o amava tanto que lhe dava seu dízimo e
sua promessa de todas as mais belas coisas e as melhores que tinha. E Caim seu
irmão não fez assim, quando tomou tudo o mais leve e o pior que tinha e
sacrificou ao seu Criador. E disso veio que Nosso Senhor deu também uma boa
coisa àquele que lhe oferta o bom e o melhor. Quando tinha subido ao sítio, em
que costumavam queimar seus sacrifícios, assim como Nosso Senhor lhes tinha
mandado, sua fumaça foi diretamente para o Céu, e a de Caim, seu irmão, não foi
desta maneira, quando se espalhou sobre o campo e era preta e horrenda e malcheirosa,
e a fumaça que era do sacrifício de Abel era branca e bem cheirosa.

Quando Caim viu que Abel era, no seu serviço, muito melhor que ele e que Nosso
Senhor tomava muito melhor em agradecimento o sacrifício de seu irmão, que ao
seu, então isso o cansou muito, e caiu em um grande ódio de seu irmão e tanto
que ele ficou odiado fora de medida. E então ele começou a pensar como poderia
se vingar, assim como ele falou para si mesmo que queria matá-lo e em outra
coisa não conseguia pensar, que pudesse se vingar. Então portou Caim por muito
tempo ódio de seu irmão em seu coração, mas não fez o mesmo para que se pudesse
provar que pensava algum mal. E o ódio ficou tão escondido que Abel foi um dia
ao campo um pouco longe da árvore. E o dia estava muito quente, e o sol brilha
muito claro, tanto que Abel não conseguia suportar este calor. Então foi
sentar-se sob a árvore e começou a dormir, e seu irmão o tinha espiado e
demorou tanto até que o viu deitar sob a árvore. Então veio e considerou
matá-lo tão secretamente que isso não pudesse ser alertado. E Abel bem o ouviu
vir, e por causa disso se olhou, e pois que se olhou, então se ergueu e o
saudou, quando lhe tinha muito amor e falou: “Sede bem-vindo, caro irmão!” E
ele o saudou de volta e o fez sentar-se junto de si. Então tomou uma faca curva
e o espetou dentre seus peitos. Então tomou Abel a morte das mãos de seu
desleal irmão, e no mesmo sítio onde fora recebido na sexta-feira, à mesma hora
foi morto. Quando Abel padeceu a morte, então não havia mais que três homens
sobre o reino da terra, e isto significava a morte da verdadeira cruz, e Caim
significa Judas, por causa do qual ele padeceu a morte, e Abel, Nosso Senhor.
Justamente como Caim saúda Abel, seu irmão, assim Judas saúda seu Senhor e
agora tinha propiciado Sua morte.

Assim bem se igualaram as duas mortes, porém não em elevação, senão em
significado. E justamente como Caim assassinou seu irmão na sexta-feira, assim
fez Judas ao seu Criador na sexta-feira, e não com suas mãos, senão com sua
língua. Assim Caim significa Judas, quando não podia encontrar coisa por causa
da qual devesse odiar seu irmão, e ele tinha causa sem direito, quando por
causa de que não tinha visto nenhuma maldade nele. Quando é costume de todas as
pessoas más lutar contra as pessoas boas, por causa de que as odeiam. Judas,
que era um mau sujeito, tivesse ele sabido tanta maldade em Nosso Senhor quanto
nele mesmo, não o teria odiado assim tanto; teria sido uma coisa que lhe teria
tido mais amor. Da traição que Caim fez a seu irmão Abel, fala Nosso Senhor no
Saltério por meio de Davi, o rei, que lá falou uma coisa maravilhosa; por causa
de que não sabia como a tinha falado, e fala justamente como se a falasse para
Caim: “Pensaste e falaste maldade contra teu irmão, e contra o filho de tua mãe
colocaste tua traição e a tua maldade. E isso fizeste, e Eu me calei quieto, e
por causa disso então achaste que eu te seria igual, por causa de que não te
falei: ‘Porém devo te castigar e devo te fazer isso cruelmente’”. 

 Esta vingança foi dita, antes que Davi tivesse profetizado, então Nosso Senhor
veio a Caim e falou: “Caim, onde está teu irmão?”, e ele lhe respondeu como
aquele que se sabia culpado da traição, e tinha agora coberto seu irmão com as
folhas da árvore, por causa de que não fosse encontrado: “Senhor, não sei dele,
sou então protetor de meu irmão?” E Nosso Senhor falou para ele: “O que é isto
que fizeste, a voz do sangue de Abel se me queixa de ti! E por causa disso, de
que fizeste isso, assim deves ser maldito sobre a terra. E a terra deve ser
maldita nas obras que fizeres, por causa de que receberam o sangue de teu
irmão, que tu perturbaste”.

Assim Nosso Senhor amaldiçoou a terra, quando não amaldiçoou a árvore, sob a
qual Abel foi morto, nem as que dela vieram, nem as que depois cresceram sobre
o reino da terra. E da árvore aconteceu um milagre, quando tão logo Abel tinha
padecido a morte sob a árvore, então ela perdeu a cor verde e ficou vermelha.
Foi um testemunho do sangue que lá foi perturbado. Depois não se pôde plantar
nenhuma dela, quando todas as que daí se plantavam feneciam de imediato. Ela
cresceu e ficou mais bela, assim que se tornou a mais bela árvore que desde
então se viu. Por longo tempo permaneceu na cor e na beleza, como ouvistes. Nem
nunca, desde então, ficou podre ou seca, sem que portasse nenhum fruto desde o
tempo em que o sangue de Abel foi perturbado lá embaixo. Quando as outras, que
dela tinham vindo, floresciam e portavam frutos. E assim ficou desta maneira
até que o mundo estava tão crescido e que tinha se tornado muito, e eles o
tinham por um grande monumento, os que vieram de Adão e de Eva, e eles o
honraram assim muito. E uns contavam aos outros de linhagem em linhagem, como a
primeira mulher as tinha plantado. E daí vieram em curto tempo os velhos e os
jovens, e vieram de longe para dela se consolarem, quando estavam em
padecimento próprio, quando era chamada Árvore da Vida, e lhes fez depois
alegres pensamentos. E a árvore cresceu e aumentou, assim fizeram todas as
outras que dela vieram, que então eram brancas. E no mundo não havia ninguém
tão ousado que tivesse arrancado um galho de lá. 

Da árvore aconteceu ainda uma outra maravilha. Quando então Nosso Senhor Deus
enviou a água sobre todo o reino da terra e que o mundo, que era tão mau, ficou
todo arruinado, e os frutos das árvores e as moradias precisaram tão duramente
se justificar, que desde então nunca ganhariam tão bom sabor, quando depois
todas as coisas se tinham transformado em um amargor. Quando os frutos da
Árvore da Vida não tinham se modificado, nem a cor que tinham antes. Tanto
tempo permaneceram as árvores na maneira, que Salomão, filho do rei Davi,
reinava, e ele era tão sábio de toda a boa arte, que nenhum coração humano
conseguia pensar. E ele conhecia todas as forças das pedras e as virtudes de
todas as ervas, e ele sabia o percurso das estrelas tão bem que ninguém, sem
ser Deus, sabia melhor. E, porém, toda a sua grande sabedoria não foi contra a
esperteza de sua mulher, quando ela o enganou o suficiente e fartamente quando
tinha colocado isso e seu sentido. E não se deve ter isso por maravilha, quando
sem dúvida, quando uma mulher quer nisto colocar seu sentido e seu coração,
ninguém sobre todo o reino da terra conseguiria se defender dela, e isto não
principiou em nós, senão em nossa primeira mãe, Eva. 

Quando Salomão viu que não conseguia se defender de sua mulher, então se
maravilhou de onde aquilo podia ter-lhe vindo, e estava consigo muito irado,
quando nunca se atreveu a fazer algo sobre isso. Então falou ao seu livro, que
se chama \textit{Parábolas}: “Eu rumei ao redor do mundo e fui lá através do
meio de tal maneira, como se não pudesse solicitar o coração de nenhum homem, e
nele todo não consigo encontrar uma boa mulher”. Esta fala falou Salomão por
causa da ira que tinha de sua mulher, contra a qual ele nada conseguiu fazer.
E, porém, o tenta ele em alguma coisa, quando não podia ser. Quando o viu,
então começou a perguntar a si mesmo por que causa uma mulher ira um homem de
tão bom grado. A esta pergunta respondeu uma voz para ele e para tudo aquilo
que ele pensou e falou para ele: “Salomão, o sentido das mulheres vem para o
homem em engano. Não te volvas, quando lá deve ainda tornar uma mulher, de que
muita alegria deve vir ao homem, cem vezes mais do que seja esse engano. E a
mulher deve vir da tua linhagem, sabe-o!” 

Quando Salomão ouviu essa fala, então se tomou por um tolo, que tivesse ralhado
com sua mulher. E então começou a pensar em coisas, que lá acontecem acordadas
ou dormentes, contemplar se conseguia saber a verdade e o fim de sua linhagem;
e procura tanto por isto que o Espírito Santo lhe mostrou o futuro da real
Virgem Maria e uma parte do que deveria vir depois. E pois que percebeu esta
notícia, então perguntou se seria o fim de sua linhagem. “Não,” falou uma voz,
“um homem casto deve ser o fim de tua linhagem, e ele deve ser tão melhor que
José, teu cunhado, como a virgem deve ser melhor que tua mulher. Então te fiz
seguro daquilo em que demoravas em grande dúvida”. 

Pois que Salomão ouviu esta fala, então falou: “Então estou muito contente que
minha linhagem deva tomar fim em tão elevada cavalaria”. E pois que
Salomão, que por longo tempo tinha estado em dúvida, soube da verdade de seu
futuro, depois pensou muito e procura, quando não conseguia ver como que ele
poderia avisar a alguém o que depois de um longo tempo deveria acontecer e que
ele não conseguia saber nenhuma notícia daquilo. E sua esposa pensou que ele
estivesse escondendo alguma coisa,\footnote{ A expressão que consta do texto
medieval é \textit{under henden}, sob as mãos, que corresponde a uma expressão
idiomática ainda presente ao alemão contemporâneo, \textit{unter den Händen},
que significa ocultar algo.}  que ele não conseguiria vir até o
fim. E ela lhe teve amor, porém, não como amor como algumas mulheres têm por
seus senhores. E ela era, porém muito sábia e não queria lhe perguntar de
pronto e demorou tanto tempo até que, uma noite, ele estava feliz, o que ela na
mesma hora viu nele. Então ela lhe pediu que lhe quisesse dizer daquilo que ela
lhe perguntava. E ela falou que o faria de bom grado, como aquele que daquilo
não se protege. Então falou: “Senhor, pensastes muito nesta semana e também na
outra, que vossos pensamentos nunca ficaram em longo tempo. Por causa disso
assim sei bem que pensastes em coisas de que não conseguistes vir a um fim. E
por causa disso assim bem queria de bom grado saber, quando não há no mundo
coisa nenhuma tão grande, que eu quisesse que fosse ao seu fim, da grande
sabedoria que está em mim”. 

 Quando Salomão ouviu isto, então pensou bem que o coração de nenhuma pessoa
sobre o reino da terra deveria dar ali conselho para essas coisas, como ela
deveria fazer. Pois que a tinha encontrado em tão grande sabedoria, que ele não
considerou que houvesse no mundo pessoa mais sábia. Por causa disso lhe veio em
seu ser que queria dizer a ela todo o seu pensamento. E então disse a ela toda
a verdade. E pois que lhe tinha dito, então ela refletiu consigo por um tempo e
lhe respondeu assim: “Como é isto,” falou ela, “fostes então enganado como
podeis fazer alerta ao cavaleiro, que soubestes a verdade dele?” “Sim,” falou
ele, “quando não consigo ver como isso poderia suceder, quando lá ainda há
muito tempo, que isso me admira”. “Em verdade,” falou ela, “porque não o
sabeis, assim vos devo eu ensinar; quando me dizei primeiro, quanto tempo vos
parece que ainda seja até lá”. Então ele falou: “Eu penso que ainda duzentos
anos ou mais estejam aí”. “Então vos digo”, falou ela, “o que deveis fazer:
fazei fazer um navio da melhor madeira e da mais verdadeira que se consiga
encontrar e que seja de tal modo que nunca consiga apodrecer, nem por causa da
água, nem por causa do fogo ou outra coisa”. E ele falou que o faria de bom
grado. 

No outro dia mandou o rei Salomão a todos os camareiros de sua terra que lhe
fizessem o navio mais maravilhoso que jamais foi visto e de tal madeira que lá
não pudesse apodrecer. E eles falaram que o deveriam fazer como ele o tinha
ordenado. E eles tinham procurado a madeira e a árvore do mastro. Então falou a
mulher de Salomão: “Senhor, porque deve assim ser que o cavaleiro deve vencer
todos os cavaleiros de cavalaria que foram antes dele e os que devem vir
depois, assim seria grande honra que vós lhe prepareis algo de armas que bem
superem todas as outras armas, assim como ele deve superar toda a outra
cavalaria”. Então ele falou que não sabia nenhum lugar de onde a tomar
assim como ele faria de bom grado. “E eu devo vos indicá-la”, falou ela, “no
templo que fizestes para a honra de Nosso Senhor. Lá está a espada que era do
rei Davi, vosso pai, que a mais afiada e a mais bela que jamais foi dada de
nenhum cavaleiro. Tomai-a e retirai a maçã e o cabo assim que tenhais a lâmina
nua. Quando bem reconheceis o poder das ervas e a virtude das pedras e a
matéria de todas as coisas sobre o reino da terra, deveis fazer na espada uma
maçã de uma rica pedra, assim tão habilmente feita junta, que ninguém sobre o
reino da terra depois de vós possa confessar\footnote{ Hans-Hugo Steinhoff
traduz \textit{bekennen}  por \textit{unterscheiden},
mas, em nossa tradução, optou-se pela fidelidade ao texto
original, devendo a presente nota elidir eventuais dificuldades de leitura e
compreensão.}  um do outro, e cada qual que a tem deve pensar que
seja tudo uma coisa só. Depois fazei nela um cabo, que não haja nenhum melhor
sobre o reino da terra. Depois fazei a bainha tão maravilhosa como a espada é
em sua dignidade. Quando o tenhais feito, então devo nela fazer a asa como me
parece bem”. E ele faz tudo o que ela lhe disse, especialmente a maçã, na qual
não fez mais que uma pedra, que era de todas as cores que se poderiam descobrir
sobre o reino da terra. E faz nela um cabo, assim como ela lhe tinha mostrado,
no outro fim. 

E pois que o navio estava feito e foi colocado no mar, a mulher fez lá dentro
fazer uma cama, que era grande e maravilhosa, e fez sobre ela jazerem ricas
cobertas bordadas. A cama, ela era grande e bela, e à cabeceira assim colocou o
rei sua coroa e a cobriu com um pano branco de seda. Então tinha dado a espada
à sua mulher, que deveria nela fazer a asa, e ele falou: “Trazei cá a espada,
assim a coloco aos pés da cama”. Então o viu e viu que a asa estava feita de
obra. Então deve ter ficado bastante irado. Então ela falou: “Senhor, eu não
tinha nenhuma outra coisa que lá fosse digna para lá”. “O que devemos, então,
fazer-lhe?”, falou ele. “Deveis assim a deixar, quando não nos pertence para
que nela o façamos, e eu não sei quando isso deve ser”. Então deixaram a espada
como estava. Depois fizeram cobrir o navio com lençóis de seda, que não se
poderia ter preocupação de que apodrecesse. E pois que o tinham feito, a mulher
contemplou a cama e falou: “Ainda falta aqui algo”, e saiu do navio e tomou
consigo dois camareiros e veio à árvore, sob a qual Abel fora assassinado, e
falou para eles: abatei-me a madeira tanto que eu possa dela fazer um fuso”.
“Ah, Senhora, não o fazemos; não sabeis que é a árvore que nossa primeira mãe
plantou?” “Deve ser que o façais, de outro modo vos faço perecer”. Eles
falaram que o fariam, porque eram obrigados a isso, e preferiam fazer mal a
deixarem-se perecer. Então começaram a bater na árvore. E não tinham batido por
muito tempo, até que os dois desmaiaram e viram claramente lágrimas de sangue
saírem, tão vermelhas como rosas vermelhas. E então quiseram parar com seu
abate; e ela os fez de novo principiar, fosse de seu agrado ou tormento. E
cortaram dela tanto que dela tiveram suficiente para um fuso. E depois ela fez
também tanto tomar da uma das árvores verdes, que dela tinham vindo, e também
de uma das árvores brancas, que era externa e internamente branca.

Pois que ela tinha de três madeiras e de três cores, então ela veio ao navio, e
entrou e falou: “Deveis fazer-me destes três, primeiro, três fusos, e que um
seja colocado de lado na cama e o outro, ao contrário, do outro lado e o
terceiro deite em cima, assim que esteja acoplado aos dois”. E eles os fizeram
como ela mandou e os colocaram bem ali. E desde então nunca nenhum se
transformou em cor, tanto tempo quanto durou o navio. E quando o tinham feito,
então Salomão contemplou o navio e falou para a sua mulher: “Fizeste maravilha!
E estivesse o mundo inteiro aqui, não poderia perceber este significado deste
navio, a não ser que Deus mesmo lhes dissesse, nem tu mesma, como o fizeste.
Então o cavaleiro não deve saber que eu não percebi nenhuma notícia dele, Nosso
Senhor então ajude a dar conselho a isso”. “Então deixai se tornar”, falou ela,
“quando deveis puramente ouvir outras notícias disso, mais do que considerais”.


À noite deitou-se o rei Salomão à frente do navio com pequena companhia. E
quando ficou adormecido, então lhe pareceu que do Céu veio um homem com grande
companhia, e entrou no navio. E pois que lá tinha entrado, então lhe pareceu
que um anjo trazia em  uma caldeira de vinho de cor de prata e caiou o navio.
Depois veio à espada e escreve letras nela. Pois que o tinha feito, então
escreve letras na maça e no cabo. E depois veio ao bordo do navio e escreve
letras nele. E quando ele o tinha feito, então se foi deitar sobre a cama. E
depois Salomão não soube aonde ele veio, e ele desapareceu, ele e sua companhia. 
No outro dia, tão logo Salomão estava desperto, então veio ao navio e
encontrou letras escritas no bordo, que lá falavam: ``Ouve, homem, que quer em
mim entrar: protege-te que não entres, se não fores totalmente verdadeiro,
quando não sou senão verdadeiro! E tão logo te lançaste fora da reta fé, eu te
lanço fora de maneira que não tenhas contenção nem socorro; quando te deixo
perecer qual tempo sejas agarrado em descrença”.

Quando Salomão viu essas letras, então estava muito temeroso, que não ousou lá
entrar, quando se moveu para trás. E o navio se tinha agora empurrado ao mar, e
tão logo rumou de lá, que em curta hora ele tinha perdido a vista do navio.
Então se sentou sobre a margem abaixo e começou a pensar nestas coisas. E então
baixou uma voz, que falou: “Salomão, o último cavaleiro de tua linhagem deve
repousar na cama que fizeste fazer, e deve bem perceber mais de ti”.
Desta fala Salomão ficou muito contente, e ele acorda sua mulher e todos que
estavam junto dele e contou-lhes a aventura. E fez para que soubessem aos
pátrios e aos forasteiros, como sua mulher tinha trazido ao fim o que ele não
conseguia dar nenhum conselho. E por causa deste direito que este livro vos
contou, assim nos diz a aventura por que causa o navio foi feito e como os
fusos eram, de cor natural, brancos, verdes e vermelhos, sem fusão. E bem aqui
a aventura o deixa e fala de outras coisas.

Bem aqui nos diz a aventura que por um longo tempo contemplaram os três
companheiros a cama e os fusos, e tanto que confessaram que os fusos eram de
cores naturais, sem fusão. Então se maravilharam muito, quando não sabiam como
podia ser. E quando tinham contemplado por muito tempo, então ergueram o lençol
e contemplaram a coroa de ouro e lá, junto a um alforje, que era à vez rico,
como pensaram, e o abriram e lá dentro encontraram uma carta. E quando os
outros viram isso, então falaram: “Se Deus quiser, esta carta nos deve dizer
deste navio e de onde ele seja vindo e quem o fez de primeiro”. E então
começaram a ler a carta, tanto que ela lhes diz as maneiras dos fusos e do
navio, como a aventura nos contou antes. Não houve lá nenhum que não chorou,
quando ela os fez pensar em altas linhagens. Quando Parsifal os tinha
informado sobre as maneiras dos fusos e do navio, então falou Galaat: “Caros
Senhores, então precisamos ir procurar a donzela que tire esta asa e lá faça
outra, quando de outra forma ninguém pode retirar a espada”. Então falaram que
não conseguiriam saber onde podiam achá-la, “porém vamos de bom grado, porque
tem de ser”.

Quando a irmã de Parsifal ouviu isso, então falou para eles: “Senhores, disto
não dizei, quando se Deus quiser, antes que venhamos daqui, assim devem as asas
estar tão ricas e tão belas como lá pertencem”. Então a donzela abriu um
escrínio que ela tinha e tomou de dentro uma asa, feita de seda e de ouro e
cabelo, tão rica, e nela os cabelos eram tão claros e tão iluminados, que se
podiam reconhecer os fios de ouro. E depois lá estavam duas fivelas de ouro,
que dificilmente se poderiam encontrar iguais a elas em nenhum fim.

“Caros Senhores,” falou ela, “vede aqui a asa que lá deve estar. Sabei”, falou
ela, “que as faço das coisas que estão acima de mim, que teria preferido, que
era de meu cabelo. Que eu tivesse amor não era maravilha, quando no dia santo
de Pentecostes, em que fostes feito cavaleiro”, falou ela para Galaat, “então
eu tinha a cabeça mais bela de todas, como nenhuma mulher no mundo tinha.
Quando tão logo eu soube que a aventura me estava preparada e que a preciso
fazer, então me fiz aparar habilidosamente e faço as asas daí, assim como
vedes”. “Em Deus”, falou Bohort, “por causa disso assim deveis ser bem-vinda
para nós e para Deus, quando nos ajudastes a sair da inquietação em que
teríamos entrado, não fosse esta nova notícia”. E então ela foi até a espada e
tirou a asa de obra e a outra de lá, tão deslumbrante e habilmente, como se o
tivesse feito todos os dias de sua vida.

Pois que o tinha feito, então falou para os companheiros: “Sabeis como a espada
se chama?” Eles falaram: “Não, deveis nomeá-la a nós, as letras o dizem”.
“Sabei”, falou ela, “que essa espada é chamada a espada da Asa Estranha, e a
bainha é chamada a Devoção de Sangue. Quando ninguém que tem seu sentido junto
a ela, que veja um lado da bainha, que lá foi feita da árvore da vida, deixa de
pensar na morte de Abel e em seu sangue”. Pois que ela o tinha dito, então
falou para Galaat: “Então vos pedimos em nome de Nosso Senhor Jesus Cristo e
por causa de que toda a cavalaria seja preenchida, que afiveleis a espada com a
asa estranha, que muito se cobiçou no reino de Logres, que os apóstolos de
Nosso Senhor nunca desejaram tanto. Quando por causa dessa espada, bem
consideram que as maravilhas do Santo Graal ficam atrás e as aventuras, que lá
são perigosas e que lhes vêm todos os dias”. “Então me deixai”, falou
ele, “fazer à espada, primeiro, o que lhe é devido, quando ninguém a deve ter,
que possa agarrá-la em torno da maça, que bem podeis ver que é minha”. Então
falaram que seria verdade. E então ele a tomou com a mão. Então lhe foi tão bem
com o agarrar ao redor, que um dedo ia o suficiente sobre o outro.

Quando os companheiros viram isso, então falaram para Galaat: “Senhor, bem
sabemos que é vossa, por causa disso não pode mais ser falado nisso, precisais
afivelá-la em vossa cintura”. Então ele a tirou da bainha, e então a viu tão
bela e tão clara que se poderia ter visto nela. Então Galaat a enfiou de volta
à bainha, e agora lhe tomou a donzela do lado que ele tinha afivelado e
afivelou-lhe o outro com a asa ao redor. Pois que lhe tinha feito pender, então
ela falou: “Seguramente, então não me prejudica quando eu morra. Quando agora
me tenho por uma das mais castas e melhores donzelas que estejam no mundo e que
lá fez cavaleiro o homem mais nobre do mundo; quando bem sabeis que não fostes
por direito cavaleiro, quando não tínheis a espada que por vossa causa foi
trazida a esta terra”. “Donzela, tanto fizestes”, falou ele, “que doravante por
toda a via devo ser vosso cavaleiro, e grande agradecimento deveis ter daquele
com quem falastes”. “Então podemos ir”, falou ela, “daqui para nosso
cometimento”. Então saíram do navio bem de pronto e vieram ao rochedo. Então
Parsifal falou para Galaat: “Seguramente, Senhor, nunca haverá o dia em que eu
não louve a Nosso Senhor que estive junto a uma tão grande aventura como foi
essa. E ela é maravilhosa o suficiente, mais do que jamais vi alguma”.

E pois que tinham vindo ao seu navio, então entraram, e o vento batia na vela,
tanto que logo estavam longe dali. E pois que aconteceu que a noite lhes
sobreveio, então começou um a perguntar ao outro se estariam um pouco perto da
terra. E cada qual deles falou que não sabia. À noite se deitaram no mar, sem
nunca bocado de comida ou bebida. Então veio que no outro dia aportaram à
frente de um castelo, que era chamado Kartaloch e ficava na marca da Escócia. E
pois que lá tinham aportado,  então louvaram Nosso Senhor daquilo que Ele os
tinha dirigido em paz para a aventura da espada e de novo à terra. E então
foram ao castelo e vieram até a porta. Então falou a donzela: “Senhores,
veio-nos mal que tenhamos aportado bem aqui. Quando se for alertado que somos
da corte do rei Arthur, cair-se-á sobre nós, quando, à vez, aqui se odeia muito
o rei Arthur”. “Pois não vos temais, donzela,” falou Bohort, “quando Aquele que
nos ajudou do penedo deve nos resgatar, quando Ele quer que cá venhamos”.

Nisto que assim falaram, então veio um servo que lá falou: “Senhores,
cavaleiros, quem sois?” E eles falaram que seriam da corte do rei Arthur. Então
falou o servo: “Valha-me um corceleiro, viestes cá mal”. Então se volta o servo
na direção do forte. Então ouviram de pronto soar uma corneta, que bem se podia
ouvir sobre todo o castelo. E então veio uma donzela até eles e lhes pergunta
quem seriam, e eles lhe disseram. “Ah,” falou ela, “por Deus, se pudésseis
fazê-lo, retornai, quando assim Deus me ajude, viestes para vossa morte, por
causa disso vos aconselho que retorneis, antes que venham os do castelo”. Então
falaram que não queriam retornar. “Então quereis”, falou ela, “morrer?” “Pois
nos deixai tornar”, falaram eles, “Aquele, em cujo serviço estamos, deve bem
nos dirigir”. Nesta fala assim viram vir cá  através da alameda bem cem
cavaleiros armados, que falaram todos para eles: “Dai-vos prisioneiros, ou nós
vos matamos!” Eles falaram que não o fariam. “Pois deixai para cá correr”,
falaram. Então deixaram correr os do castelo seus cavalos e aqueles que pouco
se temiam, ainda que eles fossem mais. Uns estavam a pé, outros a cavalo. E
Parsifal golpeou a um tão duramente, que ele se abalou por terra. Então tomou
o cavalo e o montou. Assim também Galaat tinha feito. E tão logo estavam a
cavalo, então começaram a golpear e também deram a Bohort um cavalo. E quando
os outros viram que lá tão mal estavam, então volveram para trás de si e
fugiram, e esses o caçaram até a parte superior do castelo.

Pois que vieram ao salão, lá encontraram cavaleiros e servos por causa da
gritaria. E pois que os três companheiros, que estavam golpeados depois dos
outros com cavalos, viram que eles se armavam, então rumaram por cima deles com
espadas erguidas e os golpearam e os dobraram abaixo. Quando por fim precisaram
virar as costas, quando Galaat faz maravilha e mata deles tantos que não
consideraram que ele fosse humano. Quando consideraram que seriam os inimigos
do inferno, que lá tinham vindo para os expulsar. E por fim, pois que viram que
não podiam atacá-los, então fugiram para fora dos portões e das janelas, e uma
parte quebrou os pescoços, os outros as pernas, e os terceiros, as mãos. 

Quando os três companheiros viram o palácio solteiro, então viram os que lá
estavam, que tinham matado. Então se tiveram por pecadores destas obras e
falaram que tinham feito mal, que tivessem assassinado tanta gente.
“Seguramente”, falou Bohort, “eu não creio que Deus lhes tivesse amor, quando,
se Deus lhes tivesse amor, então nunca teriam ficado abaixo como estão. Quando
foram más pessoas e incrédulas e fizeram talvez tanto erro contra Nosso Senhor,
que Ele não mais quis que vivessem. Por causa disso Ele nos enviou para cá, que
deveríamos expulsá-los”. “Não dizeis suficiente,” falou Galaat, “fizeram contra
Nosso Senhor, a vingança não era nossa para tomar, senão daquele que lá
permanece tanto tempo que o pecador se converte e se redime. E por causa disso
assim vos digo que nunca fico feliz, antes que eu saiba a verdade destas obras
que fizemos”. 

Nisto que assim falavam, então saiu um bom homem de uma câmara, e era padre, e
estava vestido com trajes brancos e trazia um corpo de Nosso Senhor em um
cálice. E quando viu os mortos jazerem no salão, então ficou bastante assustado
e se moveu para trás, como aquele que lá não soubesse o que devia fazer, pois
que viu tanta gente morta. E Galaat, que lá bem tinha visto o que ele trazia,
tirou seu elmo à frente dele e soube bem que o padre tinha temido. Então fez
seus companheiros esperarem e veio ao bom homem e falou: “Senhor, por que causa
vos movestes para trás, não podeis nos temer!” “Quem sois vós?”, falou o bom
homem. E ele falou que eles seriam da corte do rei Arthur.  

Quando o bom homem tinha entendido esta notícia, então ficou muito contente e
sentou-se e falou para Galaat que lhe dissesse como os cavaleiros tinham sido
todos assassinados. E ele lhe contou como os três companheiros da demanda
tinham sido lançados pelos ares por eles “e o azar caiu sobre esses, como bem
se pode ver”. E pois que o viu, então o padre falou: “Vós, Senhores, sabeis que
fizeram a melhor obra que jamais fizeram os cavaleiros! E deveis viver tanto
quanto dure o mundo, assim não creio que vós pudésseis fazer maior esmola que
essa. Por causa disso sei bem que Deus para cá vos enviou para fazer essas
obras. Quando lá não havia nenhuma gente em todo o mundo que odiasse tanto
Nosso Senhor como estes três irmãos fizeram, que tiveram dentro este castelo. E
por causa de sua grande infidelidade, então tinham invertido tanto esses desse
castelo, que eram mais maus que pagãos e não fizeram nada senão o que era
contra Nosso Senhor e contra a Santa Igreja”. “Senhor,” falou Galaat,
“arrepende-me muito que eu tenha estado junto quando foram assassinados, quando
eram cristãos”. “Nunca deixai que vos arrependa, seguramente, que
assassinastes, disto vos diz Deus grande agradecimento. Quando não eram
cristãos, senão que eram as pessoas mais desleais que jamais vi, e devo vos
dizer como eu o sei. Neste castelo em que estamos, cá era senhor o conde
Ernons, disso já faz um ano, e ele tinha três filhos, suficientemente bons
cavaleiros às armas, e uma filha, a mais bela desta terra. E os três irmãos
amavam a irmã com louco amor, assim que ficaram acesos, que tomaram sua
inocência. E por causa de que ela se reclamou disto contra o pai, pois a
mataram. E quando o pai viu esta maldade, então quis expulsá-los de perto dele.
E disto não padeceram e agarraram seu pai e o deixaram prisioneiro, e o feriram
muito. E o teriam matado, se não tivesse um seu irmão feito, que o protegeu. E
depois começaram eles a fazer toda a maldade do mundo, quando assassinaram
pastores e alunos, monges e abades e fizeram lançar abaixo duas capelas que
estavam no castelo. E fizeram tanta maldade que é maravilha que durante muito
tempo não se tenham afundado. Quando hoje cedo aconteceu que seu pai, que lá
dentro está, está moribundo, assim eu considero, pediu-me que eu viesse a ele
assim armado como me vedes, e vim de muito bom grado como para aquele que
sempre me teve amor. Quando assim que para cá vim, eles me fizeram tanta
vergonha que os pagãos não me tinham feito tanto. E eu o padeço de bom grado
por causa do Senhor a quem fizeram isso por vergonha. E quando eu vim à prisão,
onde dentro estava o conde, e que eu lhe tinha contado a desonra que eles me
fizeram, então ele me respondeu e falou: ‘Não volvais para lá, quando minha
vergonha e a vossa deve ser vingada com três servos de Nosso Senhor Jesus
Cristo. Quando assim me ofertou o Alto Senhor’. E por causa desta fala bem
podeis ver e saber que Nosso Senhor não se irou por causa disto que fizestes.
Quando sabei seguramente que Nosso Senhor vos enviou para assassiná-los, e
deveis ainda hoje visivelmente ver sinais além dos que já vimos”. Então o bom 
senhor começou a chorar tão lamentosamente e Galaat com ele.
“Senhor,” falou ele, “por longo tempo vos esperamos e desejamos, porém o temos,
Deus seja louvado!” “Por causa de Deus”, falou o conde, “tomai-me entre vosso
exército, para que meu corpo descanse um pouco e para que minha alma fique mais
leve, que dela o corpo seja mantido por um tão nobre homem e cavaleiro como
sois”. E ele fez o que lhe pediu de bom grado. E quando tinha deitado o conde
sobre seu peito, então se deixou afundar justamente como aquele que se prepara
para a morte e falou: “Amado Pai do Reino dos Céus, em Tuas mãos encomendo meu
espírito e minha alma”. Então cai amolecido todo à vez e permanece desta
maneira tanto tempo que eles consideraram que ele estivesse morto; então ele
foi falado. Pois que tinha assim ficado, então falou: “Senhor Galaat, isto
te\footnote{ Mais uma vez a alteração no pronome de tratamento.} 
pede o grande senhor, que hoje o vingastes em seus inimigos, que todas as
hostes celestes disso se alegram. Então deves rumar para o rei Maghame o mais
breve que puderes, para que ele fique saudável, tão logo lá venhas, e deveis
vos alçar àquela direção o mais breve que consigais ou possais”. 

Com isso se calou e não falou mais. Então de pronto lhe seguiu a alma do corpo,
e quando os do castelo viram isso, que o conde estava morto, então ficaram
muito tristes, quando eles lhe tinham muito amor. Então fizeram enviar a todos
os homem que lá próximo moravam, e sepultaram o corpo por terra na clausura de
um eremita tão ricamente como simplesmente se deve fazer a um tal homem. No
outro dia então se apartaram de lá os três companheiros e vieram ao ermo
deserto, e a irmã de Parsifal seguia com eles. Então vieram ao ermo, então
viram o veado branco, que os quatro leões conduziam, que Parsifal mais viu.
“Galaat,” falou Parsifal, “então podeis ter visto maravilha, quando nunca vi
maravilhosa aventura, e eu creio seguramente que os leões protegem o veado. É
uma coisa pela qual nunca fico feliz, antes que saiba a verdade”. “Em nome de
Deus”, falou Galaat, “assim também desejo saber. Então nos deixai seguir atrás,
até que possamos saber a verdade e sua morada, quando bem imagino que seja uma
aventura de Deus”. E os outros o seguiram de bom grado. Então seguiram atrás do
veado até que vieram a um vale. Lá viram à sua frente uma clausura, lá dentro
morava um bom velho, e o veado lá entrou e os leões também com ele. E os que os
seguiam voltaram-se para a direção da capela e viram o bom homem vestido com as
armas de Nosso Senhor, que ele deveria começar a missa do Espírito Santo. E
quando os companheiros viram isso, então falaram que bem para lá tinham vindo.
Então ouviram a missa que o bom homem cantava. E quando chegou a calmaria da
missa, então se maravilharam os três companheiros mais do que tinham feito.
Quando viram, como lhes pareceu, que o veado se transformou em uma pessoa e
sentou-se acima, sobre o altar, em uma poltrona, que à vez era bela e rica. E
viram que os leões se transformavam, um ficou como uma pessoa, outro como uma
águia, o terceiro permaneceu leão, e o quarto, como um boi. Assim se
transformaram os quatro leões e bem puderam ter voado, se fosse da vontade de
Deus. Então tomaram a poltrona, em que o veado se sentava, dois aos pés e os
outros dois às cabeças. E eles seguiram para uma janela de vidro afora em uma
medida que o vidro nunca ficou destruído ou quebrado. E quando seguiram
caminho, então ouviram uma voz que falou: “Desta maneira assim veio o Filho de
Deus veio à pura Virgem Maria, que sua castidade e pureza nunca foram
destruídas”.

Pois que ouviram esta fala, então caíram por terra tanto tempo quanto estavam,
quando a voz lhes tinha dado tão grande clareza e tão grande alarido, que a
capela ficou com grande adornamento. E pois que eles voltaram à sua força e ao
seu poder, então viram que o padre se tinha agora despido como aquele que tinha
executado a missa. Então vieram até ele e lhe pediram que lhes quisesse dizer o
significado daquilo que tinham visto. “O que vistes então?”, falou ele. “Vimos
um veado se transformando em forma humana, e assim se transformaram os leões
também em outra forma”. 

Quando o bom homem ouviu isso, então falou para eles: “Ah, Caros Senhores,
precisais ser-me bem-vindos! Então sei bem por aquilo que me dizeis, que sois
as pessoas mais nobres e os verdadeiros cavaleiros que lá devem levar a fim a
demanda do Santo Graal e que lá padecem o trabalho e o tormento. Quando sois
aqueles a quem o Senhor Deus mostrou as coisas celestiais e escondidas, e Ele
vos deixou então ver apenas uma parte. Quando nisto que o veado foi formado na
forma de uma pessoa, não foi na tua forma de uma pessoa mortal, isso ele mostra
que superou na Cruz, que estava coberto com cobertura humana, com que ele
superou, com o morrer, a eterna morte e nos trouxe de volta à vida. E isso pode
bem assinalar o veado. Quando justamente assim como o veado rejuvenesce nisto
que ele deixa seus cornos e uma parte de seu cabelo, justamente assim veio
Nosso Senhor da morte à vida, quando deixou o corpo terreno, que é a carne
mortal, que Ele tinha tomado no corpo da pura Virgem Maria. E por causa de que
a pura donzela nunca fez pecado, por causa disso vos apareceu em forma de um
veado branco sem máculas. E pelos quatro leões deveis entender os quatro
evangelistas, que lá descreveram uma parte das obras de Nosso Senhor Jesus
Cristo, que era atuante no meio de nós  tanto tempo quanto foi terreno. Então
sabei que nunca nenhum cavaleiro conseguiu mais saber o que isso significa, nem
a verdade. E isso o Bom Mestre, o mais alto, nestas terras e em algumas terras
deixa verem as boas e santas pessoas e os bons cavaleiros à maneira de um veado
e em tal companhia de quatro leões, para que os que viram isso tomem daí um
exemplo. Quando sabei por certo que doravante ninguém deve ver isso em
igualdade, por mais nenhum tempo”. 

Pois que ouviram esta fala, então choraram de alegrias, que daquilo tinham, e
agradeceram a Nosso Senhor daquilo que Ele lhes tinha mostrado visivelmente.
Então ficaram o dia inteiro junto ao bom homem. E quando, no outro dia, tinham
ouvido missa, e pois que deviam sair de lá, então Parsifal tomou a espada, que
foi de Galaat, e falou que doravante mais a queria portar, e deixou sua espada
na casa do bom homem. Quando estavam de lá apartados, e tinham cavalgado até o
meio-dia, então se aproximaram de um castelo, que se chamava Gyech, que era bom
e forte. E não se voltaram para aquela direção, quando seu caminho não os
trouxe para lá. E pois que vieram do portão maior, então viram um cavaleiro,
que veio até eles cavalgando, que falou: “Senhores, a donzela que convosco
segue é uma virgem pura? “Em verdade,” falou Bohort, “uma donzela e uma pura
virgem ela é, isso deveis saber”. Então deitou o braço e a agarrou com o arreio
e falou: “Valha-me um pequeno corcel, não me podeis ir embora, deixastes então
aqui o costume deste castelo!” 

Quando Parsifal viu o cavaleiro que mantinha sua irmã, então isso o incomodou
muito à vez e ele falou: “Senhor cavaleiro, não sois sábio com a fala! Quando
donzelas virgens, aonde venham, estão fora de todo costume, e principalmente
uma donzela nobre como é esta, filha de um rei”. No instante em que falaram,
assim saíram do castelo dez cavaleiros armados, e com eles veio uma donzela,
que tinha uma tigela de prata em sua mão. E eles falaram: “Precisa ser com
violência que a donzela, que conduzis, dê-nos o costume deste castelo”. E
Galaat perguntou qual costume seria aquele. “Senhor,” falou um cavaleiro, “cada
donzela que passe por esse castelo precisa dar esta tigela cheia de sangue de
sua mão direita, de que não pode ser liberada”. “Deus lhe dê desgraça”, falou
Galaat, “àquele que fez este descostume além de cavaleiro, quando é mau e
interminável. E, como Deus me ajudar, quanto a esta donzela falhastes, quando,
tanto tempo quanto eu esteja saudável, e ela me siga, não será vosso o que
desejais”. “Assim Deus me ajude”, falou Parsifal, “ser-me-ia preferível que eu
fosse assassinado”. “E eu também”, falou Bohort. “Em verdade, assim deveis
morrer”, falou o cavaleiro, “e fostes os melhores cavaleiros do mundo”.

 Então deixaram correr uns para os outros. Então adveio que os três companheiros
derrubaram os dez cavaleiros, antes que suas lanças quebrassem. Depois assim
tomaram sua espada e os dobraram abaixo como se fossem touros. E eles os teriam
assassinado muito facilmente, quando os que estavam no castelo quiseram sair
com sessenta cavaleiros, que estavam todos armados, para ajudá-los. E diante
deles veio um velho que lá falou para os três companheiros: “Ah, caros
senhores, apiedai-vos de vós mesmos e não vos façais matar! Seria grande pena,
pois vós sois por demais nobres pessoas e verdadeiros cavaleiros. Por causa
disso assim vos queremos pedir que nos deis o que vos desejamos”.
“Seguramente”, falou Galaat, “falais à toa, quando nunca recebereis tanto tempo
quanto ela me crer”. “Como é isto,” falou aquele, “quereis então morrer?”
“Ainda não é vir tão longe,” falou Galaat, “preferimos morrer a querermos
padecer a maldade que buscais junto de nós”. 

Então começou a alçar-se briga grande e maravilhosa, e eles tinham cercado os
três companheiros de todos os lados. Quando Galaat, que tinha a espada com a
asa estranha, golpeou pelos lados direito e esquerdo e assassinou tudo o que
ele agarrou e faz tal maravilha, que quem a visse, não consideraria que fosse
uma pessoa mortal. E conduziu tudo à sua frente, que nunca se volveu para trás,
e seguiu por sobre seus inimigos, e também o ajudou muito que seus companheiros
o ajudassem pelos dois lados, que ninguém conseguia vir à frente deles. 

Ao tempo, perdurou a luta até as nonas horas, que os três companheiros não se
torceram nem evacuaram o lugar. E durou tanto tempo até que se fez noite, assim
que com força precisaram deixar o lugar, assim que os do castelo falaram que
precisariam negociar a briga. Então veio o bom homem até os três cavaleiros que
antes lhes tinha falado e falou: “Senhores, nós vos pedimos amigavelmente que
queirais esta noite vos abrigar conosco, e nós vos prometemos que pela manhã
cedo vos deixamos voltar na medida em que agora estais. E sabeis por que causa
eu vo-lo digo? Quando bem sei que, tão logo saibais a verdade destas coisas,
que vos deve ser do agrado e que devereis nos dar o que vos desejamos”. Então
falou a donzela: “Senhores, fazei-o porque eles vos pedem!” Então eles
seguiram isso e deram para aqueles sua verdade. Então entraram com um no
castelo. Nunca houve tão grande alegria quanto a que os do castelo faziam aos
três companheiros, e os fizeram sentar e desarmar. E pois que se tinham
sentado, então perguntaram o costume do castelo e como ele se teria principiado
e por que causa. E um do castelo falou para eles: “Isto bem vos queremos
dizer.

É verdadeiro”, falou ele, “que aqui dentro está uma donzela a quem todos nós
obedecemos, e esta terra e este castelo são dela e muitos. Então aconteceu, faz
bem dois anos, que ela caiu em um vício por vontade de Deus. E pois que ela
tinha deitado por longo tempo, então contemplamos que vício ela tinha. Então
vimos que ela tinha lepra. Então mandamos chamar todos os médicos longe e
perto. Então nos diz um sábio: poderíamos ter uma tigela cheia do sangue de uma
donzela, que ainda fosse casta em vontade e em obras, e que fosse filha de um
rei e fosse irmã de Parsifal, que também é puro, assim se poderia curar a
donzela.

Quando ouvimos isso, então fizemos um costume de que nunca de nenhuma donzela
que viesse por aqui ainda pura virgem, deixaríamos de ter uma tigela de seu
sangue. Então ficamos hoje nas portas do castelo, que pudemos manter todos que
viessem à frente delas. Então ouvistes como o costume do castelo foi
principiado, e o encontrastes assim, pelo que podeis fazer o que quiserdes”.
Então clamou a donzela para os três companheiros e falou: “Senhores, então bem
vedes que a donzela está viciada, que bem a quero curar, se eu quero. E eu
quero, então ela não pode convalescer, por isso quero ajudá-la”. “Em nome de
Deus,” falou Galaat, “fazei isso no que sois jovem, assim não podeis vos sair,
que devereis morrer”. “Em minha verdade”, falou ela, “é que preciso morrer para
que eu a cure, para mim é uma honra, e devo também fazer uma parte por vossa
causa, e uma parte por causa dela. Se voltares de manhã uns para os outros como
hoje estivestes, assim não pode ser, será maior perda que da minha morte. E por
causa disso assim vos digo que quero fazer isso porque eles desejam; assim esta
guerra está julgada. E eu vos peço por Deus que nisto me sigais”. Disso ficaram
muito tristes.

Então clamou a donzela aos do castelo e falou: ``Estai felizes que vossa luta de
manhã está suspensa, e eu vos prometo que devo me deixar da maneira que as
donzelas se deixam”. Quando aqueles ouviram isso, então muito lhe agradeceram.
Então começou a alçar-se alegria lá dentro muito mais do que tinham feito antes.
Então os serviram o melhor que conseguiam. À noite foi bem servido aos
companheiros, e ainda se lhes teria ofertado melhor, se tivessem querido. No
outro dia, pois que tinham ouvido missa, então veio a donzela ao palácio e
falou que se lhe trouxesse a donzela que lá estava viciosa, que deve
convalescer de seu sangue. E eles falaram que o fariam de bom grado, e se a
trouxe. E quando os companheiros a viram, então se maravilharam muito como ele
poderia viver no padecimento que ela tinha. Quando a viram vir, então se
levantaram perante ela e a fizeram sentar-se junto deles. E ela falou para a
donzela que ela lhe daria o que tinha prometido. E ela falou que o faria de bom
grado. Então a donzela chamou que se lhe trouxesse a tigela, e se lhe trouxe. E
ela moveu seu braço direito para frente e fez golpear uma veia, e agora o
sangue saltou para frente. E então ela faz uma cruz à sua frente e se encomenda
a Deus e fala para a donzela: “Senhora, eu me dei à morte por causa de vos
curar, por Deus, rogai por mim!”

Nisto que ela assim falou, então se lhe ficou o coração puro do sangue que ela
perdia, e a tigela estava cheia. Então os companheiros correram para lá e a
seguraram. E pois que tinha ficado muito tempo em desmaio, que podia falar,
então falou para Parsifal: “Por causa de curar esta donzela, eu morro. Por
causa disso assim vos peço, tão logo eu esteja morta, assim me deitai em um
navio na primeira margem que está lá ao lado, e deixai-me seguir como a
aventura me guiar. E eu vos digo, tão logo vindes à cidade de Saras, para lá
deveis rumar atrás do Santo Graal, que me encontrais então na margem, abaixo da
torre. Então fazei tanto por causa de mim e por causa de meu amor, que fazei
sepultar meu corpo no palácio espiritual. E sabei por qual razão eu vos peço
por isso? Por causa de que Galaat e vós deveis ficar, isso eu sei!”

Pois que Parsifal ouviu isso, então ele lhe promete isso e grita e falou que o
faria de bom grado. E ela falou: “Separai-vos pela manhã uns dos outros e cada
qual de vós siga um caminho até o tempo em que a aventura vos traga novamente
um ao outro na casa do rei Mahames,\footnote{ O texto apresenta
esta grafia alternativa ao nome do rei.}  quando assim o quer o
Alto Senhor e vos pede comigo”. E eles falaram que o queriam fazer de bom
grado. Então ela pediu que se lhe fizesse vir o corpo de Nosso Senhor. Então se
mandou chamar um enclausurado, que não morava longe de lá, em uma floresta. E
ele não permaneceu muito tempo e veio perante a donzela. E quando ela o viu
vir, então ergueu sua mão na direção de seu Criador e o recebeu com grande
devoção. E depois de pronto, lá se separou deste reino da terra, do que esses
companheiros ficaram muito tristes, assim que não conseguiam se consolar
facilmente.

No mesmo dia a senhora ficou saudável. Quando tão logo ela foi lavada com o
sangue da santa donzela, então ela ficou purificada da lepra, e sua carne
voltou à grande beleza, que antes foi impura e horrenda de ver. Disso ficaram
muito contentes os três companheiros e também os do castelo. Depois fizeram à
donzela o que ela tinha desejado, e fizeram-lhe do corpo o que dele se deve
fazer, e depois assim se a embalsamou tão ricamente como se ela tivesse sido
uma imperatriz. Depois eles tomaram um navio e o recobriram com um pano muito
rico e fizeram lá dentro fazer uma cama muito bela. Pois que tinham preparado o
navio tão ricamente como conseguiram, então colocaram a donzela lá dentro e
empurraram o navio ao mar. E Bohort falou para Parsifal que lhe seria muito
lamentável que não tivessem colocado uma carta junto dela, que lá pudesse
significar toda a sua vida e como tinha morrido e também todas as aventuras que
ela tinha ajudado a trazer ao fim. Por causa de que se abordasse em terras
estranhas, que se soubesse quem ela era. “Eu vos digo”, falou Parsifal, “que
eu coloquei uma à sua cabeça, que lá mostra todos os seus pais e como ela
morreu e toda a aventura que foi junto dela, no caso de ser encontrada em
terras estranhas, que se saiba bem quem ela é”. E Galaath\footnote{ Neste
momento, o nome do cavaleiro surge escrito com a letra H ao final. 
} falou que o teria feito de muito bom grado, quando quem pudesse encontrar seu
corpo, que lhe fizesse maior honra do que ele próprio faria, porque ele sabe
quem ela era e como sua vida tinha sido.

Tanto quanto os no castelo puderam ver o navio, assim ficaram à margem, e
choraram muito, a maior parte deles, quando ela tinha mostrado grande nobreza
que se tivesse entregado à morte por causa de curar uma donzela estranha. E
falaram que nunca nenhuma donzela teria feito mais. E quando não mais
conseguiam ver o navio, então foram ao castelo. E então falaram os
companheiros que não queriam lá entrar, por causa da donzela que eles lá
tinham perdido. E ficaram do lado de fora e falaram para os do castelo que lhes
trouxessem suas armas; e o fizeram de pronto. 

Pois que os três companheiros estavam montados e queriam se pôr a caminho, então
viram que tudo estava escuro no campo e as nuvens se tinham transformado. Então
seguiram na direção de uma capela, que ficava junto à via, e entraram e
deixaram seus cavalos lá em frente, em uma casinha. E então viram que o tempo
tinha se fortalecido e começou a trovejar e relampejar, e as pedras de granizo
caíam ao redor do castelo, justamente como se fosse chuva. O dia inteiro durou
o mau tempo tão grande e tão maravilhoso ao redor do castelo, que os muros do
castelo bem pela metade caíram abaixo. Disto ficaram muito assustados, quando
não tinham pensado que em uma noite o castelo podia ser tão arruinado com um
mau tempo, como agora viam lá fora.

Pois que chegavam perto as nonas horas e o tempo se tinha colocado abaixo, então
os companheiros viram à sua frente um cavaleiro cavalgar, que estava muito
ferido no corpo e fugia pela via e falava por algumas vezes: “Ah, amado Senhor
Deus, vem em meu auxílio, que então faz necessidade”. E depois dele vinha um
outro cavaleiro rumando, que grita para ele de longe: “Estais morto, não podeis
vos proteger!” E ele ergueu suas mãos na direção do Céu e falou: “Amado Senhor
Deus, vem em meu auxílio e não me deixa morrer em tão grandes necessidades e
padecimentos como esse é”. Pois que os companheiros ouviram que ele clamava ao
Nosso Senhor Deus, então isso muito os apiedou, e Galaat falou que queria vir
em seu auxílio. “Senhor,” falou Bohort, “quero fazê-lo, quando não é
necessidade que rumeis para lá por causa de um cavaleiro”. E isso foi do agrado
de Galaat porque ele o queria fazer. E Bohort veio ao seu cavalo e montou e
falou para seus companheiros: “Caros Senhores, se é que eu não volte, assim não
deixai vossa demanda, quando pela manhã cedo vos ponhais cada qual em sua via
tanto tempo até que Nosso Senhor Deus nos ajude que venhamos de novo uns aos
outros na casa do rei Mahames”. E falaram que ele seguisse em escolta de Deus,
quando queriam ambos de manhã separar-se. E Bohort seguiu de lá e seguiu atrás
do cavaleiro que assim clamava perante o Nosso Senhor por socorro. Bem aqui a
aventura se deixa e diz dos dois companheiros.

Então nos diz a aventura que a noite inteira estiveram os dois companheiros na
capela, Galaat e Parsifal, e pediram a Nosso Senhor que quisesse proteger
Bohort, em que cidade ele viesse. Pela manhã, quando já era dia e o
tempo estava posto e o ar estava belo e claro, então montaram em seus cavalos e
seguiram na direção do castelo, contemplar como tinha ido lá dentro. E pois que
vieram à porta, então o encontraram todo queimado, e os muros estavam
destruídos. Então entraram. E quando estavam dentro, então se maravilharam
muito mais do que antes tinham feito, quando não viram nem encontraram lá
dentro nenhuma pessoa, e eles visitaram em cima e em baixo, e falaram que seria
grande pena. E quando vieram ao mais alto palácio, lá encontraram os muros
invertidos e as paredes destruídas, e encontraram os cavaleiros mortos, um
aqui, outro lá, como se Deus os tivesse arruinado com o temporal. Pois que os
companheiros viram isso, então falaram que seria uma vingança espiritual, e
nunca teria acontecido, se não fosse por causa de que a ira de Nosso Senhor
para lá tivesse rumado.

Nisto que assim falavam, então ouviram uma voz que lá falava para eles: “Esta
vingança é do sangue da santa donzela, que aqui foi perturbado, por causa de
que uma má pecadora convalescesse”. E pois que ouviram isso, então
falaram que a vingança fora realmente maravilhosa, e falaram que faz justamente
tolice aquele que faz contra a vontade de Deus, nem por causa da vida, nem por
causa da morte.

Pois que por longo tempo tinham ido em torno do castelo, então encontraram,
junto a uma capela, um cemitério cheio de cabeças quebradas e cheio de ervas
verdes. E ele estava todo cheio de belos caixões, que bem podiam ser quarenta.
E bem lá ele tão rico e tão deslumbrante, que lhes pareceu que nunca nenhum
tempo teria vindo ali, como também era. Quando lá dentro jaziam os corpos que
por causa da vontade da senhora estavam mortos.

Então vieram ao cemitério, assim a cavalo como estavam, e cavalgaram para as
sepulturas e encontraram sobre algumas o nome daqueles que lá jaziam. Então
leram por longo tempo até que lá encontraram doze donzelas, que eram todas
filhas de reis e de alta linhagem. E quando viram isso, então falaram: “Este
era um mau costume que estes do castelo tinham, quando rebaixaram algumas
linhagens por causa das mulheres que lá estão mortas”. Pois que os dois ficaram
lá até as primas horas, então seguiram de lá até o ermo. E quando lá entraram,
então falou Galaat para Parsifal: “Hoje é o dia em que devemos nos separar,
por causa disso vos encomendo a Deus”. Parsifal falou: “Que precisa nos ajudar,
que precisamos vos encontrar em curto tempo! Quando não encontrei nunca uma
pessoa cuja companhia me fosse tanto do agrado como a vossa, e por causa disso
tanto me entristece a separação muito mais do que considerais; quando assim
precisa ser, porque é a vontade de Deus”. Então tiraram seus elmos e se
beijaram na separação. Assim se separaram os companheiros no ermo, que os da
terra chamavam Ube, e cada qual seguiu seu caminho. Bem aqui se deixa
a aventura de dizer deles e volta para o meu senhor, Senhor Lancelot do Lago,
quando por muito tempo não disse dele. 

\chapter{Lancelot em Corbenit}

\textsc{Bem aqui} nos diz a aventura que, pois que Lancelot veio à água de Markoßen
e que se viu fechado por três mãos de coisas, que não o faziam muito feliz. 
Quando por um lado o ermo era grande e intratável, pelo outro lado havia
dois penhascos, que eram muito grandes e altos, e pelo terceiro lado havia a
água profunda e escura. Essas três coisas o trouxeram a que ele falasse que não
queria estar lá dentro e queria bem lá esperar pela graça de Nosso Senhor, e
permaneceu desta maneira até que foi noite. Pois que era no tempo em que o dia
e a noite deviam se separar, então Lancelot tirou suas armas e se deitou e se
encomendou a Nosso Senhor, que não se esquecesse dele e viesse em seu auxílio.
E pois que tinha falado isto, então adormeceu de maneira que mais pensou em
Nosso Senhor que em coisas terrenas. E quando estava adormecido, então veio uma
voz até ele e falou: “Lancelot, levanta e vai ao primeiro navio que
encontrares!” E quando ele ouviu isso, então despertou e abriu os
olhos e viu tão grande claridade ao seu redor que considerou que fosse dia. Mas ela
não permaneceu muito tempo, quando desapareceu, assim que ele não soubesse de
onde ela tinha vindo. Então ele ergueu sua mão e se persignou e se encomendou
a Nosso Senhor, e tomou suas armas e se prepara e viu à margem um navio sem
vela e sem remos lá parado, e ele entrou. Então lhe pareceu que ele cheirava a
todos os bons aromas que havia no mundo. Então ficou cem vezes mais contente do
que antes estava. Então lhe pareceu que ele tinha tudo o que sempre desejou.
Por isso louvou Nosso Senhor e caiu sobre seus joelhos e falou: “Amado Pai,
Senhor Jesus Cristo, eu não sei de onde isto pode vir, se não viesse de Vós!
Quando vejo meu coração então em tão grande alegria que não sei se no reino da
terra ou no paraíso terrestre”. Então se colocou a bordo do navio e adormeceu
de grandes alegrias. A noite inteira adormeceu ele tão bem e tão em paz que não
lhe pareceu que ele fosse como costumava ser. Pela manhã, quando ele despertou,
então olhou todo ao seu redor e viu no meio do navio uma cama muito bela, e lá
dentro jazia uma donzela, que estava morta, e dela não se via mais que a face.
E pois que ele a viu, então se levantou e persignou-se e agradece a Nosso
Senhor da companhia que Ele lhe tinha adicionado. Então foi junto dela como
aquele que tivesse sabido de onde ela era. Então ele contemplou acima e abaixo
e tanto que viu debaixo de sua cabeça uma carta. Então tomou a carta e a abriu
e lá dentro achou escrito: “Esta donzela foi irmã de Parsifal de Gales, e foi
por toda a via e todo o tempo uma virgem pura em vontade e em obras. Foi aquela
que trocou a asa da espada estranha que Galaat, filho de Lancelot, agora
porta”.

Depois ele encontrou lá dentro toda a sua vida e como ela tinha morrido. Pois
que Lancelot ouviu que Galaat, Bohort e Parsifal a colocaram lá dentro como
ela estava no navio, por causa do mandamento da voz celestial, então ficou mais
feliz que antes, de que sabia a verdade; quando estava bem feliz de que Galaat,
Parsifal e Bohort estavam juntos. Então ele coloca a carta de volta e vai a
bordo do navio e pede a Nosso Senhor que Ele, antes que esta demanda tome um
fim, devesse lhe presentear a graça que devesse ver Galaat e que então o
conhecesse. Nisto que Lancelot estava em sua oração, então viu o navio aportar
perto de um rochedo, que era grande e perto de uma pequena capela. E à frente
da porta viu um velho, que era grisalho. E quando veio junto dele, então o
saudou. E o bom homem o saudou de volta e levantou-se de lá onde estava sentado
e veio ao navio e sentou-se sobre um monte de terras e pergunta a Lancelot que
aventura o tinha ali trazido. E Lancelot lhe conta a verdade de seu ser e como
a sorte o tinha trazido para ali, aonde nunca veio, como lhe pareceu.

Então lhe pergunta o bom homem quem ele era, e ele lhe diz. E quando ele ouviu
que era Lancelott\footnote{ Nesta altura do texto original, o nome do
cavaleiro aparece grafado com dois tês.}  do Lago, então teve nele
grande maravilha como ele viera ao navio e quem estivera lá dentro com ele.
“Senhor,” falou Lancelot, “vinde cá e contemplai, se vos for do agrado!” E
ele foi ao navio e encontrou a donzela e a carta. E pois que a tinha lido de um
fim ao outro, e pois que ouviu que ela também falava da espada com a asa
estranha, então falou: “Ah, Lancelot, não considero viver tanto que saiba o
nome desta espada. Então podes bem falar que infelizmente estás porque não estiveste
nesta alta aventura onde estes três bons cavaleiros estiveram, que eu algumas
vezes considerei que eles não seriam tão bons quanto tu. Quando então é uma
coisa evidente que eles são pessoas mais nobres e cavaleiros mais verdadeiros
do que foste perante Nosso Senhor Deus. Quando, porém, como tiveres sido no
tempo que está à frente, eu bem creio que, se quiseres doravante te proteger de
pecados mortais, e que não faças nenhuma coisa que seria contra o teu Criador,
ainda podes encontrar graça e piedade sem vacilo, Aquele que agora clamou por
ti no caminho da verdade. Quando então me diz como vieste a esse
navio”. E ele lhe contou. E o bom homem lhe respondeu gritando: “Lancelot,
sabe que Deus te mostrou grande bem, que Ele te colocou na companhia de uma tão
alta donzela. Então te protege que sejas casto doravante em vontade e em obras,
assim que tua castidade se iguale à pureza dela; assim pode a vossa segunda
companhia durar.\footnote{ Mais uma alteração de pronome de tratamento no mesmo
discurso, passando para a modalidade formal.}  E pede ao Nosso
Senhor com bom coração e Lhe promete também que não fazes nenhuma coisa com que
penses em pecar contra teu Criador. Então segue tua via, quando não tens que
ficar e, se Deus quiser, deves puramente vir à casa aonde por longo tempo
desejaste vir”. “Caro Senhor,” falou Lancelot, “permaneceis cá?” “Sim,” falou
ele, “assim deve ser seguramente”.

Nisto que falavam um para o outro, então se bateu o vento no navio e o fez rumar
do penhasco. Pois que viram que um devia separar-se do outro, então um
encomendou o outro a Deus, e o bom homem quis ir para sua morada. Quando antes
lá veio, então falou para Lancelot: “Servo de Nosso Senhor, nunca te esqueças
de mim, quando pede a Galaat, o verdadeiro cavaleiro, que deves puramente ter
junto de ti, que ele rogue a Nosso Senhor que Ele, por causa de Sua
misericórdia, queira Se apiedar de mim”. Assim clamou o bom homem para
Lancelot, que lá ficou muito contente das notícias que tinha ouvido, que
Galaat brevemente deveria estar em sua companhia. Então veio ele a bordo do
navio e fez sua oração sobre seus joelhos, que Nosso Senhor devesse escoltá-lo
à cidade, onde pudesse fazer o que Lhe era agradável. 

Então esteve Lancelot no navio um mês ou mais, que nunca saiu de lá. E quem
perguntava de que ele vivia, quando não tinha nenhuma refeição no navio, a esse
respondia a aventura que o Alto Senhor, que saciou o povo de Israel com o pão
celestial no deserto e que lá fez sair água da penha e jorrar, que bebessem,
conservava esse de tal maneira, que todas as manhãs, quando ele tinha feito sua
oração e executado e que tinha pedido a Nosso Senhor que não o esquecesse e que
lhe enviasse seu pão, como um pai por direito deve fazer ao seu filho ou sua
criança.

Todo o tempo em que Lancelot tinha feito sua oração, então se encontrou tão
satisfeito que lhe pareceu que tinha comido de todas as iguarias do mundo. Pois
que viajou por longo tempo no mar, que nunca saiu de lá, então aconteceu em um
tempo, que ele aportou à frente de uma floresta, à noite. Então ele ouviu um
cavaleiro vir a cavalo, que fazia barulho muito grande através do ermo. Pois
que veio à frente da floresta, então viu o navio e desmontou de seu cavalo e
lhe tirou a sela e o arreio e o deixou ir aonde quisesse. Depois veio ao navio
e entrou assim armado como estava.

Quando Lancelot viu o cavaleiro, então não correu para suas armas, quando bem
pensou que fosse a promessa que o bom homem lhe tinha feito, de Galaat, que
ele, antes de muito tempo, devia estar junto a ele. Então se levantou e falou:
“Senhor cavaleiro, sede bem-vindo por Deus!” E quando o ouviu falar, então lhe
responde como aquele que não considerava que alguém estivesse no navio: “Deus
vos dê boa sorte e por causa de Deus, que possa ser, assim me dizei quem sois,
quando desejo sabê-lo muito por direito!” E ele falou que era chamado
Lancelot do Lago. “É verdade?”, falou ele. “Então sede bem-vindo por Deus e
por mim! Assim para mim um pequeno corcel, eu estava mais desejoso de vós que
de todos do mundo, e eu o faço simplesmente, quando fostes um princípio da
minha vida”. Então o cavaleiro retirou seu elmo e colocou no navio. E Lancelot
lhe pergunta: “Ah Galaat, sois vós?” “Senhor,” falou ele, “sim!” E pois que
ele entendeu isto, então correu para lá com os braços abertos, e um começou a
beijar o outro e a fazer grande alegria, e deles um pergunta ao outro de seu
ser. E então conta deles um para o outro suas aventuras, assim como elas lhe
sucederam desde o tempo em que se separaram da corte do rei Arthur. E ficaram
tanto tempo no falar que o dia brilha belo e claro. E pois que o sol tinha
nascido, que podiam ver-se e reconhecer-se, então a alegria começou a se lhes
tornar grande e maravilhosa.

Quando Galaat viu a donzela que jazia no navio, então bem a confessou como
aquele que a tinha visto em outros tempos. E ele perguntou a Lancelot se ele
sabia quem a donzela era. “Sim,” falou ele, “quando a carta, que estava junto
dela, diz claramente; e dizei-me, por Deus, terminastes a aventura da espada
com a asa estranha?” “Sim, Senhor,” falou ele, “vede, cá está”. E pois que
Lancelot a viu, então bem pensou que o era. Então ele a tomou e começou a
beijar a maça e o cabo e a bainha. Então ele pediu a Galaat que lhe quisesse
dizer como a tinha encontrado e em qual sítio. E então lhe contou a matéria do
navio, que a mulher de Salomão tinha feito fazer, e dos três fusos, como Eva, a
primeira mãe, tinha plantado a árvore, de que os fusos eram de cores naturais,
branco, verde e vermelho. E pois que lhe tinha dito a matéria do navio, então
falou Lancelot que nunca a nenhum cavaleiro tinha sucedido tão alta aventura
como lhe tinha sucedido, louvado por isto seja Deus sempre. 

No navio permaneceram um junto ao outro bem meio ano ou mais, na medida em que
nenhum deles se omitiu de servir ao seu Criador com bom coração. E muito
fartamente aportaram em ilhas estranhas e longe de pessoas, que não acharam
outra coisa senão animais selvagens; pois encontraram alguma aventura
maravilhosa a cujo fim vieram assim com sua nobreza, assim com a graça do
Espírito Santo, que os ajudou todo o tempo, de que a aventura do Santo Graal
não diz. Quando seria muito para dizer, aquele que devesse tudo contar, que
aconteceu no tempo. 

Depois da Páscoa, assim todas as coisas começaram a verdejar e que os pássaros
nas florestas alçaram seus doces cânticos, por causa do doce tempo que
principiava, e que todas as mais coisas começavam a alegrar-se mais que em
outros tempos, por causa do tempo aconteceu que em um meio-dia eles aportaram
perto de uma floresta à frente de uma cruz de pedra. Então viram sair do ermo
um cavaleiro, que cavalgava muito ricamente, armado com armas brancas, e
conduzia pela mão direita um cavalo branco. E quando ele viu o navio junto à
terra, então cavalga para lá e saúda os dois cavaleiros dos caminhos do Alto
Mestre e falou para Galaat: “Cavaleiro, ficastes tempo suficiente junto a vosso
pai, saí do navio e montai neste cavalo e rumai para onde a aventura vos
conduza e procurai a aventura do reino de Logres e a conduzi ao fim!” Pois que
ouviu isso, então veio para seu pai e o beijou muito amigavelmente e chora e
falou para ele: “Querido pai, Deus vos abençoe, não sei se vos vejo algum dia
mais”. E então começaram os dois a chorar de coração.

Nisto que Galaat saiu do navio, e montou em seu cavalo, então veio uma voz entre
eles dois: “Cada um de vós bem pensai, quando um de vós não vir o outro até o
dia do Juízo Final, então Nosso Senhor deve dar a cada um o que mereceu”. 
Então começaram os dois a gritar. Pois que Lancelot entendeu isso, então
falou para Galaat chorando: “Amado filho, quando assim é que eu nunca mais te
possa ver, e precise me separar de ti e não devo te ver mais, assim pede ao
Nosso Senhor por mim, que Ele não me deixe apartar de Seu serviço, e que Ele me
proteja, que eu seja Seu servo mundano e espiritual”. E Galaat respondeu para
ele: “Nenhum oração vos é tão boa como a vossa, e por causa disso assim pensai
por vós mesmo”. Assim se separou um do outro, e Galaat seguiu para o ermo, e o
vento batia no navio, tanto que Lancelot muito depressa veio da margem. Assim
Lancelot estava só no navio, sem que a donzela lá estivesse. Então seguiu ele
bem dois meses totalmente no mar, de maneira que não dormiu muito. Quando ele
despertou e pediu a Nosso Senhor chorando que pudesse vir ao sítio onde pudesse
ficar sabendo algo do Santo Graal. 

À noite, depois da meia-noite, ele aportou à frente de um castelo, que era muito
rico, belo e forte. Quando atrás no castelo ficava uma porta, que saía por
sobre a água, e estava por toda a via, de noite e de dia, aberta, quando do
lado não tinham nenhuma inquietação. Quando lá ficavam por toda a via dois
leões que lá protegiam, um atrás, outro na frente, assim que ninguém podia
entrar, ele deveria passar pelos dois leões, se quisesse ir até a porta.

No tempo em que Lancelot lá aportou, então brilha a luz justamente clara e
bela, assim que se podia ver longe o suficiente. E então ele ouviu uma voz que
falou para ele: ``Sai do navio e vai ao castelo, lá deves encontrar uma parte do
que procuras”. Pois que ouviu isto, então correu de pronto para suas
armas e as tomou e não deixou nenhuma coisa lá dentro que tivesse lá trazido
consigo. Quando veio aqui fora à frente da porta, então encontrou os dois
leões, então bem pensou que não poderia entrar lá sem lutar. Tão logo
Lancelot tinha sacado sua espada, então olhou contra a montanha e viu vir uma
mão de fogo, que o golpeou tão duramente que a espada lhe caiu fora da mão. E
ouviu uma voz que lhe falou: “Ah, homem de fé débil, por causa de que crês mais
nas tuas mãos que no teu Criador? És muito desgraçado que considerais que
Aquele em cujo serviço te destes não possa melhor te ajudar que tuas armas”. 

Lancelot ficou assim muito assustado desta fala e da mão que o tinha golpeado,
que caiu por terra tão esticado e ficou tão invertido que não soube se seria
dia ou noite. Quando depois de um longo instante se ergueu e falou: “Amado Pai
Jesus Cristo, eu vos agradeço de que me quisestes castigar de meu erro. Então
vejo bem que me tendes por vosso servo, porque me mostrastes sinais de minha
desgraça”. Então ele tomou sua espada e a enfiou na bainha e falou que por sua
causa não mais a queria tirar. Então ele se deu à misericórdia de Deus e falou:
“É Sua vontade que eu morra, assim é, porém, uma conservação de minha alma”.
Então ele faz uma cruz à frente de sua testa e veio aos leões. E eles se
sentaram quando o viram vir e não fizeram nenhuma igualdade de que quisessem
lhe fazer padecimento. Então foi entre eles, que eles nunca o tocaram, e ele se
pôs na alameda principal e foi contra o monte na direção do castelo, por tanto
tempo até que veio ao mais alto castelo. E os do castelo estavam agora todos
dormindo. Então não encontrou ninguém lá dentro que assim lhe dissesse onde ele
estava. E ele veio às escadas e subiu contra o monte, até que veio ao salão
principal, assim armado como ele estava. E pois que lá veio, então olhou para
frente e para trás, e não viu nem homem nem mulher; disto ele muito se
maravilhou, quando não pensou encontrar um tão belo palácio sem pessoas. Então
ele seguiu em frente e pensou que em algum momento deveria encontrar gente que
lhe dissesse onde ele tinha aportado. Quando ele não sabia de si, o que o
inquietava. Por tanto tempo foi Lancelot até que ele veio a uma câmara, e a
porta estava fechada. Então tomou ele sua espada e contemplou se ela podia
abrir, e não o conseguiu fazer. Então ele ouviu uma voz, que falava de coisas
espirituais, como lhe pareceu que falasse: “Paz, louvor e honra estejam
convosco, Pai do Reino dos Céus!”

 Quando Lancelot ouviu esta fala, então se lhe estremeceu seu coração, e caiu
sobre seu joelho à frente da porta, quando bem pensou que o Santo Graal
estivesse lá dentro. Então falou chorando: “Amado Pai, Senhor Jesus Cristo, se
fiz alguma coisa que Te fora do agrado, amado Senhor, por causa do bem que está
em Ti, não me injuria e me deixa ver uma parte daquilo que eu procuro”. Quando
tão logo ele o tinha falado, então olhou à sua frente e viu que a porta da
câmara estava aberta. E nisto que ela se abriu, então saiu uma claridade tão
grande, bem como o sol. E quando ele viu isto, então ficou tão justamente feliz
e tinha tão grande desejo que poderia saber de onde viria a claridade que era
tão grande que disto tudo se lhe esqueceu. Então ele veio para os portões à
frente da câmara e quis entrar. Então falou uma voz: “Foge, Lancelot, e não
entra aí, quando não o deves fazer, e se aí entrares, vais te arrepender”. 

Quando Lancelot o ouviu, então foi para trás muito triste como aquele que teria
entrado de muito bom grado, quando porém assim conteve a si próprio. Então viu
ele sobre uma távola o Santo Graal, recoberto com um samítico vermelho. E viu
também ao redor e ao redor anjos por toda a parte, que serviam ao Santo Graal,
de maneira que lá seguravam turíbulos de prata e velas. Os outros seguravam
cruzes e a aparelhagem do altar, e deles não havia nenhum que não o servisse
com algo. E à frente do Santo Graal sentava-se um velho, vestido como um padre.
E pois que ele estava na tranquilidade da missa e devia erguer o corpo de Nosso
Senhor, então pareceu a Lancelot que sob as mãos do bom homem estariam três
pessoas, e as duas colocavam o mais jovem entre as mãos do bom homem. Então ele
o ergueu alto e fez justamente como se quisesse cair. E Lancelot muito se
maravilhou, e lhe pareceu que ele estivesse por demais carregado pelas três
pessoas que ele segurava. Então lhe pareceu que ele deveria cair para trás. E
quando ele viu isto, então quis vir-lhe em auxílio, quando a ele pareceu bem
que ninguém lhe queria vir em auxílio, que junto dele estavam. Então lhe teria
de muito bom grado vindo em socorro, que não lhe pareceu que lhe era proibido. 

Então ele veio aos portões e falou: “Amado Pai Jesus Cristo, não me transformeis
em que eu vá em auxílio do bom homem quando lhe é necessário”. Então andou para
lá e foi para a távola prateada. E pois que veio lá junto dela, então sentiu um
vento tão quente que lhe pareceu que seria agarrado com fogo, e o golpeou para
a direção dos portões, assim que lhe pareceu que o tinha queimado. Então não 
tinha mais poder, que viesse adiante, como aquele que estava
justamente invertido, e tinha perdido todo o poder do corpo, assim em ouvir
como em ver. Então sentiu muitas mãos que o carregavam e o agarravam para cima
e para baixo e o empurraram à frente da capela e o deixaram jazer bem lá. 

No outro dia, pois que era dia e que os do castelo estavam levantados, então
encontraram Lancelot deitado à frente dos portões da câmara. Então muito se
maravilharam com quem ele pudesse ser. Então clamaram muito para ele, que ele
se levantasse, quando ele não fez como se os ouvisse. E pois que viram isto,
então falaram que ele estaria morto, e então o desarmaram e o contemplaram
abaixo e acima, se ele estaria vivo. E então acharam que ele não estava morto,
quando ele não tinha nenhum poder para se levantar, nem conseguia falar, quando
ele estava exatamente como um torrão da terra. E então o tomaram por todos os
lados e o portaram a uma câmara e o deitaram em uma cama muito rica longe das
pessoas. E o tomaram verdadeiramente, e ficaram por toda a vida à sua frente e
foram fartamente até ele, contemplando se ele podia falar alguma coisa, quando
ele fez igual a que se nunca tivesse falado palavra. Então lhe apalparam o
pulso e as veias e falaram que seria uma maravilha deste cavaleiro que
estivesse vivo e não pudesse falar. Os outros falaram que não sabiam de onde
aquilo poderia vir, que não fosse em algum momento uma vingança ou um sinal de
Nosso Senhor. O dia inteiro ficaram os do castelo à frente de
Lancelot, e no outro, e no terceiro e no quarto dia. E esses falaram que ele
estaria morto, os outros falaram que ele estaria vivo. “Em nome de Deus,” falou
um velho, que lá conseguia o suficiente na medicina, “eu vos digo seguramente
que ele não está morto, quando ele está tão cheio de vida como um de vós. E por
causa disso assim louvo que se o proteja até que Nosso Senhor lhe devolva sua
saúde, que em algum momento teve, e então seremos bem cientes de quem ele é.
Seguramente, eu bem acredito que ele foi um dos bons cavaleiros do mundo e
ainda deve ser, se Deus quiser! Quando ele não deve ter preocupação da morte,
como me parece, quando eu não falo que ele possa ainda por muito tempo estar
vicioso da maneira como agora está”. 

Assim falou o bom homem de Lancelot, que era muito sábio, quando ele falou
pouco alguma coisa que não fosse verdadeira, assim como ele a tinha sabido.
Assim o protegeram bem catorze dias e catorze noites, que não mordeu nunca de
comer ou de beber, nem nunca palavra saiu de sua boca, nem se moveu mão nem pé,
nem parecia que ele estivesse vivo. E se queixavam muito por ele todos e
falavam: “Ah, Deus, é da maior pena a deste cavaleiro, que lá tão nobre parece
e era tão belo; então desta maneira Deus o colocou nesta prisão”.

E assim falavam os do castelo fartamente e algumas vezes de Lancelot e não
conseguiam contemplá-lo muito que o conseguissem reconhecer. E porém lá havia
alguns cavaleiros que o tinham visto fartamente, que bem deveriam tê-lo
reconhecido. Desta maneira jazeu Lancelot bem lá bem catorze dias, que não
esperavam outra coisa senão sua morte. E quando se veio ao décimo-quinto dia,
então aconteceu bem ao meio-dia que ele abriu os olhos. E quando viu as
pessoas, então formou tão grande lamento e falou: “Ah, Deus, por que causa me
despertastes tão logo, quando eu estava mais contente do que fico assim. Ah,
Senhor Jesus Cristo, que boas obras poderia aquele que lá soubesse as grandes
maravilhas dos vossos segredos, pois que minha visão foi paralisada com tão
grande impureza deste mundo!”

Pois que aqueles que estavam junto de Lancelot ouviram essa fala, então ficaram
muito contentes e lhe perguntaram o que ele tinha visto. “Eu vi”, falou ele,
“tão grande maravilha que minha boca não a conseguiria contar, nem meu coração
conseguiria pensar quão grande alegria é. Quando não foi coisa mundana, quando
foi espiritual, e não tivesse sido meu grande pecado, eu teria ainda visto
mais, quando perdi a visão de meus olhos e o poder do meu corpo, por causa da
grande deslealdade que Deus em mim encontrou”. 

Então Lancelot falou para os que lá estavam: “Amados Senhores, eu me maravilho
muito que seja aqui encontrado, quando não me parece como tenha sido colocado
cá”. Então eles lhe disseram tudo o que dele tinham visto e como tinham estado
junto dele bem catorze dias, que não sabiam se ele estaria vivo ou morto. E
pois que ouviu isto, então começou a pensar por causa do significado que tinha
assim colocado na medida e por tanto tempo que refletiu que bem tinha servida
ao Inimigo por catorze anos. Por cauda disso Nosso Senhor o tinha lançado em
penitência, que ele tinha perdido o poder de seu corpo bem por catorze dias. 
E eles lhe perguntaram como lhe tinha acontecido. E ele respondeu: “Nobre
e saudável, da graça de Deus, quando por Deus, dizei-me onde estou”. Eles
falaram que ele estaria no castelo de Corbenit. 

Então veio uma donzela, que lhe trouxe roupas de linho novas e frescas, mas ele
não quis vesti-las, quando tomou a cota de ferro. E quando aqueles que estavam
junto dele viram isto, “Senhor cavaleiro,” falaram, “bem podeis deixar a cota
de ferro, quando trouxestes vossa demanda ao fim. E não trabalheis doravante
mais para procurar o Santo Graal, quando não o encontrareis mais do que vistes.
Então deve Deus nos enviar os puros senhores que mais devem ver”. Por causa
disso Lancelot não quis deixar a cota de ferro e a vestiu e colocou por cima
as roupas de linho. Depois lhe vieram as vistas daqueles do castelo e tomaram
por grande maravilha o que Deus tinha feito a ele. Então não o contemplaram por
muito tempo, reconheceram-no e falaram: “Ah, Senhor, Senhor Lancelot, sois
vós?” E ele falou: “Sim”. Então começou grande alegria e a alçar-se
maravilhosamente, então foram as notícias ao palácio, que o rei Pellis ficou
ciente. Então um cavaleiro lhe diz: “Senhor, eu posso vos dizer maravilha”. “De
quem?”, falou o rei. “Seguramente, o cavaleiro que por tanto tempo ficou aqui
dentro, exatamente como se estivesse morto, agora se levantou e está saudável,
e é Lancelot do Lago”.

Pois que o rei entendeu isto, então ficou muito contente e veio vê-lo muito
depressa. E quando Lancelot o viu, então se levantou perante ele e ficou muito
contente. E o rei lhe diz notícias de sua bela filha que lá estava morta, que
lá tinha ganhado Galaat. Isso foi muito lamentável para Lancelot. Quando no
quinto dia depois, pois que estavam sentados para a refeição, então tinha o
Santo Graal preenchido a távola agora tão deslumbrante e tão maravilhosamente
que ninguém no reino da terra podia pensar nem dizer. 

Nisto que lá dentro se comeu, aconteceu uma aventura que eles consideraram muito
maravilhosa, quando viram que as portas do palácio se fecharam visivelmente,
sem que ninguém pusesse nenhuma mão. Então eles ficaram muito assustados. E um
cavaleiro, armado com todas as armas, montava um grande cavalo branco, que veio
à frente das portas e começou a clamar: Deixai-me entrar!” E os do castelo não
o queriam deixar entrar. E ele clamou por tanto tempo até que o rei se levantou
da távola e veio a uma janela do palácio e o viu e falou: “Senhor cavaleiro,
não podeis aqui entrar, quando ninguém que cavalgou tão alto como vós pode
entrar, porque o Santo Graal está aqui. Por causa disso assim segui a vossa
via, quando seguramente não sois dos companheiros da demanda, quando sois
daqueles que abandonaram o serviço de Nosso Senhor Jesus Cristo, e fostes ao
serviço do Diabo”. Pois que o cavaleiro ouviu isto, então ficou muito triste e
não soube o que deveria fazer e retornou. E o rei clamou para ele e falou:
“Senhor cavaleiro, porque para cá viestes, então me dizei, quem sois?”
“Senhor,” falou ele, “eu sou do reino de Logres, e sou chamado Hector de Mares
e sou irmão do senhor Lancelot do Lago”. “Em nome de Deus,” falou o rei,
“então bem vos reconheço, então estou muito mais triste do que estava antes, e
lamento pelo vosso irmão que está cá dentro”. 

Pois que Hector ouviu que seu irmão estava lá dentro, a pessoa que ele mais
temia no mundo pelo grande amor que por ele carregava, então falou: “Ah Deus,
então cresce minha vergonha! Então não ouso nunca vir à frente dele desde que
faltei naquilo em que os verdadeiros cavaleiros e as pessoas nobres não faltam.
Seguramente, o bom homem na clausura me disse a verdade, que falou comigo e com
o senhor Gawin e me diz o significado de meu sonho”. Então seguiu através do
castelo tão logo quanto conseguia andar a cavalo. E pois que os do castelo o
viram fugir tão logo, então clamaram por ele e lhe falaram mal e maldisseram a
hora em que ele nasceu e ralharam com ele e clamaram: “Seu mau cavaleiro e
incrédulo!”  E ele teve tão grande arrependimento que bem queria ter estado
morto, e fugiu até que veio à frente do castelo e de pronto seguiu ao ermo que
lá era todo farto. Então veio o rei Pellis para Lancelot e lhe diz as notícias
de seu irmão, de que ele também ficou triste, que não sabia o que fazer. Porém
não conseguia esconder, os do castelo testaram bem que as lágrimas lhe saíam
dos olhos abaixo pelo rosto inteiro. Por causa disto o rei lamentou que lhe
tivesse dito essa notícia.

Pois que se tinha comido, então falou Lancelot que se lhe fizesse trazer suas
armas, quando queria rumar para o reino de Logres, quando nunca tinha vindo
para lá em um ano. “Senhor,” falou o rei, “quero vos pedir que me perdoais que
eu vos trouxe notícias de vosso irmão”. E ele falou que queria de bom grado
perdoar. Então mandou o rei Pellis que se lhe trouxessem suas armas; então se
as trouxeram, e ele as tomou. E quando estava armado, então o rei lhe fez
trazer um bom cavalo e falou que ele montasse. E ele o fez e tomou licença do
rei e se apartou de lá e cavalga uma grande distância diária através da terra
estranha. E sucedeu uma noite que ele se albergasse em um mosteiro branco, onde
os irmãos lhe fizeram grande honra porque ele era um dos cavaleiros
aventurosos. De manhã, pois que tinha ouvido missa, e que devia sair
do mosteiro, então viu pela direção do lado direito uma sepultura, que foi
feita recentemente, como lhe pareceu. Então foi para lá contemplar o que era. E
quando foi lá junto dela, então viu que era tão bela que lhe pareceu que lá embaixo
jazia um grande senhor. Lá estavam letras que lá falavam: “Aqui jaz o
rei Brandemer de Gorre, que o senhor Gawin assassinou”. Então ele ficou muito
triste porque ouviu isso, quando lhe tinha muito amor, e tivesse sido um outro
que não o senhor Gawin, deveria ter estado morto. Então falou “ah Deus”, e
gritou muito e falou que seria grande vergonha, pois, da corte do rei Arthur e
ainda de alguns cavaleiros de terras estranhas.

 Durante o dia Lancelot permaneceu lá dentro muito triste por causa do rei, do
nobre senhor, que lhe tinha feito alguma honra. Pela manhã, quando
estava armado, então encomendou os irmãos a Deus e montou seu cavalo e seguiu
por tanto tempo até que, com aventura, veio até as sepulturas onde as espadas
estavam estiradas. E tão logo ele viu as sepulturas e a aventura, então seguiu adentro
com seu cavalo e contemplou os túmulos. E então se apartou de lá e rumou tanto
tempo até que veio à corte do rei Arthur, em que um e os outros lhe fizeram
grande honra, tão logo eles o viram. Quando ansiavam muito por sua vinda e dos
outros companheiros, de quem poucos vieram. E os que lá retornaram, não tinham
conseguido senão que tivessem grande vergonha. Aqui, porém, deixa-se a
fala de dizer deles e volta para Galaat, que era filho de Lancelot. 

\chapter{O Santo Graal}

\textsc{Então nos diz} a aventura que, quando Galaat se separou de Lancelot, que ele
cavalga alguns dias assim como a aventura o guiava, um tempo à sua frente, um
tempo para trás, assim tanto tempo até que veio ao convento onde jazia o rei
Morderans. E pois que ouviu mais notícias do rei, que esperava pelo bom
cavaleiro, então pensou em vê-lo. No outro dia, pois que tinha ouvido missa,
então ele foi aonde o rei estava. E pois que para lá veio, o rei, que por longo
tempo lá tinha perdido a visão e o poder do corpo, então aconteceu que se
endireitou, tão logo ele sentiu o cavaleiro, que começava a aproximar dele, e
falou para Galaath: “Servo de Nosso Senhor, verdadeiro cavaleiro, cuja vinda
esperei por tanto tempo, ajuda-me e deixa-me repousar sobre teu peito, assim
que eu possa morrer entre teus braços castos. Quando sobre todos os cavaleiros
sois assim casto e puro, assim como as flores dos lírios, em que a castidade é
significada, que lá é muito mais branca e bela que todas as outras. És um lírio
de castidade, és uma reta rosa e uma flor de boas virtudes que lá tem calor do
fogo. Quando o fogo do Espírito Santo verteu-se em ti e está apanhado em ti;
quando minha carne, que lá estava toda morta e arruinada, tornou-se toda vivaz
e em boas virtudes”.

Pois que Galaat tinha ouvido esta fala, então se sentou junto à cabeça do rei e
o envolveu e o deitou sobre seu peito, por causa de que era necessário ao bom
homem que descansasse. E o rei se afundou para ele e o envolveu sobre os ombros
e começou a pressioná-lo e falou: “Amado Pai Jesus Cristo, então tenho minha
vontade! Então Te peço que venhas e me tomeis, quando nesta alegria, que por
tanto tempo desejei, não é senão rosas e lírios”. 

Tão logo ele tinha feito essa oração a Nosso Senhor, então foi uma coisa visível
que Nosso Senhor ouviu sua oração, quando todo de pronto lá Lhe deu a alma, a
quem ele tinha servido, e faleceu nos braços de Galaat. E quando os no convento
ficaram cientes, então vieram até o corpo e acharam que as feridas, que ele por
longo tempo tinha tido, estavam todas curadas, e tomaram isso por um grande
milagre. Então fizeram ao corpo o que lhe era devido, como simplesmente se deve
fazer a um rei, e o sepultaram ali. E Galaat fica lá dois dias, e no
terceiro dia então se aparta de lá e cavalga tanto tempo até que veio a um
grande ermo, onde encontrou uma fonte, que fervia, de fato, muito. E tão logo
ele tinha empurrado a mão lá para dentro, então voou o calor, por causa de que
ele não tinha ganho nenhum calor de falta de castidade. E isso os da terra
tomaram por um grande milagre. E depois ela perdeu seu nome e ficou chamada
Fonte de Galaat. 

Pois que tinha trazido ao fim a aventura, então veio a Gorre. E veio ao convento
onde Lancelot tinha estado, onde ele tinha encontrado a sepultura do rei
Loßente, filho de José de Arimateia, e no túmulo de Simeão, onde tinha faltado.
E pois que lá entrou, então olhou nas covas que estavam sob o mosteiro. E pois
que ele viu que queimava tão maravilhosamente, então perguntou aos irmãos o que
seria aquilo. “Senhor,” eles falaram, “é uma aventura maravilhosa que não pode
ser levada ao fim, a não ser que o faça aquele que lá deve superar, com bem e
com cavalaria, todos os companheiros da Távola Redonda”. “Eu queria”, falou
ele, “que me escoltásseis às portas por onde se entra, se vos for do agrado”.
Então falaram que o queriam fazer de bom grado. Então o guiaram até as portas
da cova, e ele entrou. E tão logo ele veio para junto da tumba, então se
puseram o fogo terrível e a chama, que por alguns dias tinha sido grande e
maravilhosa, por causa da vinda daquele que não tinha em si nenhum mau calor. E
pois que veio à tumba, então a ergueu contra o monte e viu lá dentro jazer o
corpo de Simeão. E tão logo o fogo se pôs, então ele ouviu uma voz que lá
falou: “Galaat, Galaat, deveis simplesmente louvar e honrar a Deus daquilo que
Ele vos deu tão grande graça, que ele, por causa da boa vida que tendes, pode
deixar o tormento terreno e colocá-las na alegria do Reino dos Céus.\footnote{
O texto original mostra-se, nesta altura, muito confuso, porém procuramos
manter a tradução fiel ao documento. A tradução de Hans-Hugo Steinhoff
apresenta outra versão, com propósito de esclarecimento do excerto: “Galaat,
Galaat, tendes todo o motivo para louvar e elogiar a Deus, pois Ele vos
presenteou muita Graça, que vós, graças à vida pia que levais, podeis apagar o
tormento terreno e colocar a alma dele na alegria do Reino do Céu”.} 
Eu sou Simeão, um de vossos velhos, e estive bem trezentos anos e sessenta e
quatro anos nesse grande calor que vistes, por causa de um pecado que uma vez
fiz contra José de Arimateia. E com o tormento que eu padeci, assim estaria
danado e perdido, quando a graça do Espírito Santo, que lá mais atuou em vós
que a cavalaria terrena, viu-me com a misericórdia, por causa da grande
humildade que há em vós. Assim Ele me arrancou através de Sua graça do tormento
terreno e colocou-me na alegria do Reino dos Céus, somente por causa da graça
de que cá viestes”.

Os do convento que tinham vindo lá para perto tão logo a chama estava apagada,
então bem ouviram a fala, e a tomaram por uma grande maravilha. E Galaat
retirou o corpo do túmulo e foi ao mosteiro. E pois que o tinha feito, então o
tomaram os que lá estavam e o sepultaram como se deve sepultar um cavaleiro,
quando ele tinha sido um cavaleiro, e lhe fizeram tal serviço como se lhe deve
fazer e o sepultaram à frente do alto altar. E pois que tinham feito
isto, então vieram a Galaat e lhe fizeram tão grande honra como jamais lhe
puderam fazer. Então perguntaram a ele de onde ele seria e de que
pessoas, e ele lhes disse a verdade.

Pela manhã, pois que Galaat tinha ouvido missa, então Galaat separou-se de lá e
encomendou os irmãos a Deus. E põe-se em seu caminho e cavalga desta maneira
cinco anos, antes de vir à casa do rei Mahames. E todos os cinco anos Parsifal
lhe fez companhia para onde seguisse. Ao tempo eles tinham trazido ao fim todas
as aventuras no reino de Logres, que não se viu acontecer muito mais, que não
fossem sinais de Nosso Senhor Deus. E em nenhum fim a que jamais vieram, não
foram vencidos, quantas pessoas lá estivessem, nem que se temessem nem tivessem
inquietação. 

Um dia aconteceu que eles deviam sair de um ermo, que era muito grande e
maravilhoso. E Bohort os encontrou próximo a uma via, e ele cavalga sozinho. E
quando eles o reconheceram, não havia dúvida de que estavam felizes, quando
tinham rumado por longo tempo, que tinham ansiado por ele. Eles o saudaram e
falaram que ele deve ter boa sorte e honra, e ele também lhes respondeu assim.
Então lhe perguntaram de seu ser, e ele lhes diz a verdade e como ele tinha,
desde então, rumado e falou que seriam bem cinco anos que ele nunca ficou
quatro noites lá onde antes tinha ficado, nem em abrigos, mas em ermos
estranhos, onde ele teria estado mais de cem vezes morto, pois que a graça do
Espírito Santo o tinha consolado todo o tempo. “E vós encontrastes o que
estamos procurando desde então?”, falou Parsifal. “Seguramente”, disse ele,
“não, quando bem sei que não vamos ainda nos separar um do outro, que não
tenhamos levado a fim aquilo pelo qual esta demanda foi assumida”. “Nisto Deus
precisa nos ajudar”, falou Galaat, “quando, como Deus me ajude, eu não sei de
nenhuma coisa que mais me agrade que vós terdes vindo, quando amo vossa
companhia e vos desejo de todo o meu coração”. 

Assim vieram juntos os três companheiros que a aventura tinha separado um do
outro. Então cavalgaram por longo tempo, assim que um dia vieram ao castelo de
Corbemon.\footnote{ Na tradução de Hans-Hugo Steinhoff, o nome Corbemon já
aparece, diretamente, como Corbenic. Entretanto, optamos, mais uma vez, pela
manutenção dos nomes que figuram no texto medieval.} E pois que lá
vieram, e que o rei os reconheceu, então a alegria ficou grande e maravilhosa,
quando então bem sabiam que deveriam ficar cientes, com este acontecimento, da
aventura do castelo, que por muito tempo tinham esperado. E as notícias subiram
e desceram tanto que todos os do castelo ficam cientes, e todos eles vieram
vê-los. E o rei Pellis grita para Galaat, e assim também fizeram os outros,
quando o tinham visto criança pequena. E quando estavam desarmados, Elias, o
filho do rei Pellis, trouxe-lhe à frente a espada que lá estava quebrada, de
que a aventura vos tinha dito em outros tempos,  com que José foi
golpeado através de sua perna. E pois que ele tinha sacado a espada de sua
bainha, e que ele lhes tinha contado como ela foi quebrada, então Bohort a
tomou com a mão, contemplando se conseguiria ser colocada junta de novo, mas
isso não podia ser. Pois que viu que tinha faltado, então a deu a Parsifal e
falou: “Senhor, tentai esta aventura, se a conseguis levar ao fim”. “De bom
grado”, falou ele e tomou o pedaço e os colocou juntos; quando não conseguiam
saltar juntos. E pois que ele viu isto, então falou: “Senhor, nessa aventura
faltamos. Então deveis vós tentá-la e assim se faltardes, então bem creio que
ela nunca será levada ao fim com nenhum homem mortal”. Então tomou Galaat a
espada e encaixou um quebro no outro. De pronto saltaram os pedaços um para o
outro tão maravilhosamente que nenhum homem no mundo era aquele que poderia
reconhecer os quebros que antes foram e que já tinham sido quebrados.

Quando os companheiros viram isto, então falaram Nosso Senhor lhes tinha
mostrado um bom começo. Então bem acreditaram que facilmente deveriam levar ao
fim as outras aventuras, porque aqui lhes tinha ido com algo bom. E quando os
do castelo viram a aventura da espada consumada, então ficaram muito
maravilhosamente contentes, e então a deram a Bohort e falaram que não poderia
ser melhor alocada, quando ele era muito um nobre e um bom cavaleiro.

 Pois que eram as vésperas, então o céu começou a nublar-se e ficar escuro e a
mudar. E o vento batia-se grande e maravilhoso e seguia através do palácio, e
estava bem cheio de calor, que a maior parte queimou e chamuscou seu cabelo, e
muitos também desmaiaram de grande inquietação que tinham. E então ouviram uma
voz que lhes falou: “Os que lá não devem sentar-se à távola de Jesus Cristo,
que sigam sua via; quando então devem ser saciados os verdadeiros cavaleiros da
refeição celestial”. Pois que eles ouviram isto, então saíram todos,
exceto o rei Pellis, que lá muito era um nobre homem e era de vida santa, e
Elias, seu filho, e uma donzela casta, que lá era a sobrinha do rei, que lá era
a mais santa e mais nobre que se poderia encontrar. E junto a esses três
ficaram os três companheiros, a contemplar que sinais Nosso Senhor queria
fazer-lhes. Pois que estavam lá há um tempo, então viram vir bem nove
cavaleiros aos portões, estando armados, que tiraram seu elmo e suas armas. E
vieram a Galaat e inclinaram-se para ele e falaram: “Senhor, nós muito nos
apressamos e ansiamos, por causa de que estivéssemos junto a vós para esta ceia
em que o Alto Mestre deve ser repartido”. E ele lhes respondeu que eles eram
bem-vindos, quando vieram bem a tempo. Então se sentaram no meio do palácio, e
Galaat perguntou-lhes quem eles eram. E os três falaram que eram da Gália, e os
outros três falaram que eram da Dinamarca. 

E nisto que eles falam, então viram sair de um câmara uma cama de madeira, e
quatro donzelas a portavam. E na cama jazia um bom homem igual a um homem
doente, e tinha uma coroa dourada sobre sua cabeça. E pois que estavam no meio
do palácio,  então o assentaram e seguiram sua via. E ele ergueu sua cabeça e
falou para Galaat: “Senhor, deveis ser bem-vindo, quando muito ansiei por
ver-vos e muito esperei por vossa vinda, na medida e no tormento em que um
outro não conseguiria ter padecido. Quando se Deus quiser, então veio o tempo
em que essas coisas devem ser resolvidas, e eu devo apartar-me deste reino da
terra assim como me foi prometido há muito tempo”. No instante em que assim
falaram, então ouviram uma voz que lá falou: “Os que lá não foram companheiros
do Santo Graal, que saiam, quando não é de direito que permaneçais aqui mais
tempo”. Tão logo a fala foi falada, então saíram o rei Pellis e Elias
e a donzela. E quando o palácio estava solteiro daqueles que lá não eram
companheiros da demanda, então pareceu àqueles que lá dentro permaneceram que
do Céu veio um homem, vestido em igualdade a um bispo, e tinha um bastão
curvado\footnote{No texto medieval, o termo “curvado”  
figura como adjetivo de “bastão”. Convém assinalar que
Hans-Hugo Steinhoff optou, em sua adaptação para o alemão contemporâneo, por
introduzir o termo composto \textit{Krummstab}, que se traduz por “báculo”. Em
nossa tradução, mantivemos a disposição sintática do texto medieval.} 
na mão e tinha uma mitra sobre a cabeça. E quatro anjos o traziam em uma
poltrona muito rica e o assentaram junto ao Santo Graal. 

Aquele que lá foi trazido em igualdade a um bispo tinha letras em sua testa, que
lá falavam: “Vede cá José, o primeiro bispo da Cristandade, o mesmo que Nosso
Senhor sacramentou na cidade de Saras no palácio espiritual”. E os cavaleiros,
que viram isto, bem reconheceram as letras, quando muito se maravilharam do que
poderia ser. Quando José, de que falavam as letras, foi separado deste mundo há
mais de trezentos anos. E ele falou para eles: “Ah, cavaleiros de Deus, servos
de Nosso Senhor Jesus Cristo, não vos maravilheis que me vejais à vossa frente,
como eu estou perante o Santo Vaso. Quando assim como eu o servi no reino da
terra, assim o sirvo no instante em que sou espiritual”. 

Pois que ele o tinha falado, então foi na direção da távola, que lá é de prata,
e ficou lá perto sobre seus joelhos perante o altar. E pois que tinha estado
assim por longo tempo, então ouviu a porta abrir-se, e olhou naquela direção, e
assim também fizeram todos os outros, e viram saírem os anjos que tinham
trazido José. E dois deles traziam duas velas e o outro, um pano de samítico
vermelho, e o quarto trazia uma lança, que sangrava tanto que as gotas corriam
a contra vale\footnote{ Convém ressaltar que, em nosso intuito de preservar ao
máximo a riqueza de estilo do escrito medieval, traduzimos, com alguma
literalidade, as expressões \textit{wiedder berg}   e \textit{wiedder tale},  que exaram,
respectivamente, os sentidos de “para cima” e “para baixo”.} a um
cibório que ele tinha na mão. E os dois colocaram suas velas sobre a távola e o
terceiro, o pano à frente do Santo Vaso; e o quarto segurava a lança estendida
sobre o Santo Vaso, assim que o sangue que fluía a contra vale correu para o
Santo Vaso. Tão logo eles o tinham feito, então José se levantou e ergueu a
lança um pouco mais alto quanto ao vaso e a cobriu com o pano de samítico, que
era muito rico. 

Então José fez igual a que se preparasse para a missa. E pois que estava um
momento naquilo, então agarrou o Santo Vaso e tomou uma hóstia, que lá estava
feita em igualdade a um pãozinho. E com isto que ele deveria erguê-la, então
desceu do Céu uma figura igual a uma criança. E tinha um rosto tão vermelho e
tão flamejante quanto fogo e entrou no pão, assim que os no palácio
visivelmente viram que o pão tinha uma forma de uma pessoa. E pois que José o
tinha longamente segurado, então o coloca no Vaso. 

Pois que José tinha feito o que pertencia ao serviço da missa, então veio a
Galaat e o beijou e falou que beijava seu irmão, e ele o fez. E pois que o
tinha feito, então falou para eles: “Servos de Jesus Cristo, quando vos
trabalhastes em tormento por causa de que pudésseis ver uma parte do milagre do
Santo Graal, então vos coloqueis à frente desta távola, assim sereis preenchido
da mais alta refeição e do melhor que nenhum cavaleiro jamais conseguiu e da
mão mesma de Nosso Senhor. E bem podeis falar que vos trabalhastes bem, quando
hoje deveis tomar a alta paga que nenhum cavaleiro tomou”.

Pois que José o tinha dito, então desapareceu entre eles, que não souberam aonde
tinha vindo. Então se sentaram de pronto à távola com grande temor e choraram
muito de coração, assim que seus rostos ficaram vermelhos. Então os
companheiros viram no Vaso um homem, como se estivesse nu, e as mãos lhe
estavam sangrentas, e o corpo e os pés. E ele falou para eles: “Cavaleiros e
Meus fiéis servos e Meus bons filhos, que da vida mortal se tornaram
espirituais e que por tanto tempo Me procuraram, que nunca posso Me esconder de
vós, precisais ver uma parte de meus segredos. Quando tanto fizestes que sois
desejados à Minha távola, em que nunca nenhum mais cavaleiro comeu desde o
tempo em que José de Arimateia aí ceou. Quando os outros tiveram assim como
mereciam, é para saber que os cavaleiros daqui e ainda de muitas outras terras
foram saciados da graça do Espírito Santo e do Santo Vaso. Quando não se
sentaram junto a Mim mesmo, assim como vós estais sentados. Então tomai e
recebei a Alta Refeição pela qual por tanto tempo ansiastes, e por cuja causa
tanto vos trabalhastes”.

Então tomou Ele mesmo o Santo Vaso, e veio a Galaat, e ele se ajoelhou. Então
lhe deu seu Criador, e ele O recebeu com as mãos dobradas juntas. E assim
fizeram todos os três, e deles não houve nenhum ao qual não parecesse que se
lhe dava o pedaço na boca em igualdade ao pão. Pois que eles todos tinham
recebido a Alta Refeição, que lhes pareceu tão doce e tão maravilhosa, que lhes
pareceu que seria a melhor coisa que se poderia imaginar, Aquele que os tinha
preenchido, que falou para Galaat: “Filho, tão puro e tão bom como uma
verdadeira pessoa pode ser, sabes o que tenho sob minhas mãos?” “Não,” falou
ele, “dizei-me pois”. “É a tigela de que Jesus comeu o cordeiro na Quinta-feira
da Paixão com seus jovens. É a tigela que lá serviu todos que, para
agradecimento, lá estavam em Meu serviço. É a tigela que nunca nenhum crente
viu, sem que lhe servisse até agradecer. E por causa de que ela tinha servido a
todas as pessoas, então ela deve simplesmente ser chamada de Santo Graal. 

Viste aquilo por que tanto tempo ansiastes ver. E ainda não o viste
completamente como ainda deves fazer. E sabes onde isto deve ser? Na cidade de
Saras, no palácio espiritual. E por causa disto deves partir daqui e fazer
companhia ao Santo Vaso, que à noite deve se separar do reino de Logres, de
maneira que nunca mais deve ser visto lá, nem doravante deve acontecer nenhuma
aventura. E sabes por qual causa se afasta daqui? Por causa de que não é
honrado como deve, nem é servido como deve de direito, por aqueles desta terra.
Quando se irritaram, por mais que tenham sido bem repletos da graça do Santo
Vaso. E por causa disso quero que vás de manhã cedo até o mar, e lá deves
encontrar o navio, dentro do qual tomas a espada com a asa estranha. E por
causa de que não rumes sozinho, quero que tomes contigo Parsifal e Bohort, e
ninguém mais. E também por causa de que não quero que te apartes desta terra,
primeiro fizeste saudável o rei Mahagine, então quero que tomes do sangue desta
lança e untes a perna dele. Quando é uma coisa com que ele deve convalescer, e
não de outro modo”.

“Ah, Senhor, por que causa não os deixais todos seguirem comigo?” “Por causa de
que não o coloco! Quando por direito assim como vós vos sentastes comigo à
távola do Santo Graal, e vós sois doze, como doze eram os apóstolos, e Eu sou o
décimo-terceiro sobre vós, quando devo ser vosso mestre e vosso pastor. E
justamente assim como eu os dividi e os fiz ir pregar em toda terra, bem assim
vos separo, um aqui, outro lá, e deveis todos morrer neste serviço, exceto um
de vós”. Então lhes deu Sua bênção e seguiu de lá de maneira que eles não
souberam aonde ele vinha, quando O viram seguir na direção do Céu. E Galaat 
veio até a lança, que lá estava colocada sobre o Santo Vaso,
sobre a távola, e tomou do sangue e veio ao rei e untou-lhe a perna através da
qual estava espetado. Então se vestiu de pronto e saltou da cama saudável e
nobre. Então ele agradeceu a Nosso Senhor de que o tinha feito saudável tão
suavemente e alimentado. E depois viveu ainda longo tempo, quando isto não foi
no mundo, quando ele se deu de pronto em um convento com monges brancos. E
Nosso Senhor fez desde então alguns sinais por causa de Sua vontade, de que a
aventura aqui não diz, quando disso haveria muito a dizer. 

Bem por volta da meia-noite, quando tinham por muito tempo orado a Nosso Senhor,
que Ele por Sua misericórdia os quisesse conduzir para a conservação de suas
almas aonde viessem, então veio uma voz que falou: “Meus filhos, e não meus
filhos adotivos, meus amigos, e não meus inimigos, ide daqui e rumai para onde
achais o melhor a fazer e justamente como a aventura vos conduza”.
Pois que ouviram isto, então respondem todos com uma voz: “Pai do Reino do Céu,
bendito és Tu que nos tomas por Teus filhos e por Teus amigos. Então bem vemos
que não perdemos em nosso tormento e em nosso trabalho”. Então saíram do
palácio e foram à corte e encontraram armas e cavalos e montaram
habilidosamente tão logo estavam armados. Pois que saíram do palácio, então
perguntaram e falaram um para os outros de onde eram, para que um reconhecesse
os outros. E acharam que os três eram da Gália, e Claudins, filho do rei
Claudas, era um deles; e os outros, de onde fossem, eram nobres o suficiente e
de alta linhagem.

Pois que chegou uma despedida, então se beijaram como irmãos e choraram muito,
de coração, e falaram para Galaat: “Senhor, sabei que nunca fomos tão felizes
como ao sabermos que deveríamos ter-vos por companheiro, e nunca ficamos tão
tristes como estamos de que tenhamos de nos separar de vós tão logo; quando bem
vemos que esta separação Nosso Senhor quer e por causa disto devemos nos
separar sem arrependimento”. Então falou Galaat: “Caros Senhores, eu tive
vossa companhia de tão bom grado como tivestes a minha, quando bem vedes que
não pode ser que um de nós faça companhia ao outro. E por causa disto assim vos
encomendo a Deus e vos peço: vinde à corte do rei Arthur, que me saudeis meu
senhor Lancelot, meu pai, e todos os companheiros da Távola Redonda”. E eles
falaram que viriam para lá, que não queriam se esquecer dele. Então se
separaram um dos outros e seguiram sua via. E Galaat voltou para trás, ele e
seus companheiros, e cavalgaram, que vieram ao mar no quarto dia, e teriam lá
vindo mais cedo, quando não conheciam das vias.

Pois que vieram ao mar, lá encontraram o navio onde encontraram, dentro, a
espada com a asa estranha e viram também as letras no bordo do navio, que lá
falavam que ninguém deveria entrar, que não acreditasse totalmente em Deus. E
pois que vieram a bordo, e olharam lá dentro, então viram que no meio da cama,
que foi feita no navio, estava a távola de prata que eles tinham deixado na
casa do rico rei Mahagine. E o Santo Graal estava sobre a távola, coberto com
um samítico vermelho, que estava feito em igualdade a um pano. Pois que os
companheiros viram a aventura, então mostraram eles uns aos outros e falaram que
bem lhes tinha vindo que lhes deveria fazer companhia, pelo que ansiaram ver
por todo o tempo, assim até onde devessem permanecer. Então fez cada
qual uma cruz à sua frente e encomendaram-se a Deus e entraram no navio. E tão
logo lá entraram, então veio um grande vento e bateu na vela tão
maravilhosamente que fez o navio partir das margens e o lançou em alto mar.
Então ele começou a rumar muito rápido, quando o vento o impele. 

Desta maneira seguiram por muito tempo no mar, que nunca souberam de onde vinham
e para onde rumavam. Todo o tempo em que Galaat esteve de pé e foi dormir,
pedia a Nosso Senhor que, quando ele pedia e desejava a morte, que Ele lhe
quisesse enviá-la. E fez isto por tanto tempo, noite e dia, que a voz celestial
falou: “Galaat, não tenhais dúvida, quando Nosso Senhor dever fazer-te a tua
vontade, daquilo por que Lhe pediste. E quando desejas a morte, assim deves
tê-la no corpo, e a alma deve ter a vida e a alegria que não passa”. A
oração que Galaat tão fartamente tinha feito, Parsifal bem a tinha ouvido, e o
teve muito maravilhosamente porque ele fez isto. “Devo dizer-vos,” falou
Galaat: “na noite em que vimos uma parte do milagre do Santo Graal, que Nosso
Senhor nos mostrou através de Sua misericórdia, no que eu vi as coisas ocultas
que lá não estão descobertas para cada qual, apenas para o servo de Jesus
Cristo, nisto que vi essas coisas, que o coração de nenhuma pessoa pode
imaginar, então estava meu coração em tal desejo, quando estivesse apartado do
mundo, eu bem sei que nunca nenhuma pessoa teria morrido em tal alegria como eu
teria feito. Quando foi tão grande companhia de anjos junto a mim, que eu teria
partido desta vida terrena para a alegria celestial dos anjos e dos mártires e
dos amados amigos de Nosso Senhor. E por causa de que vi que ainda devo vir a
tão grande volúpia e alegria ou em maior do que estava, por causa disso faço
esta oração. E tão logo Nosso Senhor me coloque em tão grande alegria, assim
quero pedir nessa oração e assim esperar apartar-me deste reino da terra”. 

Assim Galaat comunicou a Parsifal como ele queria morrer, assim como a voz
celestial tinha dito. E da maneira como eu vos disse, assim o reino de Logres
perdeu o Santo Graal por causa de seus pecados, que muitas vezes o tinha
repleto. E assim como Nosso Senhor lhes tinha enviado Galaat e
José e, aos outros, os que deles vieram, por causa de seu bem, justamente assim
o tomou dos maus por causa de sua maldade, que neles encontrou. E por causa
disso assim se pode ver visivelmente que os maus o perderam por causa de sua
maldade, que os bons tinham conservado por muito tempo com seus bens. 

Por longo tempo permaneceram os companheiros, tanto que um dia falaram para
Galaat: “Senhor, na cama que vos foi feita assim como as letras falam nunca
dormis. E por causa de que assim o deveis fazer, quando esta carta fala que
deveis descansar sobre essa cama”. E ele falou que queria lá repousar. E
deitou-se lá e dormiu por um longo tempo. E pois que estava acordado, então
olhou à sua frente e viu o portão de Saras. Então veio uma voz que falou para
eles: “Saí do navio e tomai convosco as três de prata e as portai à cidade
assim como está, e não as coloqueis no chão antes que venhais ao palácio
espiritual, onde Nosso Senhor Deus em primeiro sacramentou José de Arimateia
como bispo”. 

Nisto que quiseram tirar as távolas do navio, então olharam a água abaixo e
viram vir o navio onde há muito tempo tinham colocado a irmã de Parsifal. E
pois que viram isto, então falou um deles para o outro: “A donzela bem nos
manteve sua promessa, que até agora nos serviu”. Então tomaram as
távolas e as tiraram do navio. Bohort e Parsifal a tomaram mais pela frente, e
Galaat a tomou por trás.\footnote{ Importa ressaltar, para correto
acompanhamento do texto, que o original apresenta variação de referência a uma
távola e a três távolas na presente passagem.} Então foram na
direção da estrada. E pois que vieram até a porta, então Galaat estava muito
cansado. E ele viu um homem entre as portas, que lá ia com muletas e esperava
as pessoas que lhe costumavam fazer bem por causa da vontade de Deus. E quando
veio para junto dele, então lhe clamou: “Bom homem, vem cá e ajuda-me até que
tenhamos carregado estas távolas ao palácio”. “Ah,” falou ele, “por Deus, o que
dizeis, já faz bem dez anos que nunca consegui ir sem auxílio”. “Isto não
esclarece,” falou ele, “levanta-te, e não teme por ti, estás convalescido”. 

Pois que Galaat o tinha falado, esse viu se poderia ir. Então se achou tão
saudável e tão forte como se nunca tivesse sofrido dor. Então correu em direção
à távola e a tomou por um lado, em frente a Galaat. E quando foi à cidade, então
falou para todos aqueles que o encontravam o milagre que Deus lhe tinha feito.
Pois que subiram ao palácio, então viram a poltrona que Nosso Senhor fez por
causa de que José nela se sentasse. E todo de pronto perceberam o rei
e os da cidade o grande milagre e vieram ver o homem paralítico que foi
recentemente tornado saudável.

Pois que os companheiros tinham feito o que lhes foi ordenado, então foram ao
navio onde a irmã de Parsifal jazia, e a sepultaram tão ricamente como se deve
simplesmente fazer ao filho de um rei. Quando o rei da cidade viu os três
companheiros, então lhes perguntou de onde seriam e o que foi que tinham
trazido sobre a távola de prata. E lhe disseram a verdade e os milagres do
Santo Graal e o poder que Deus lá tinha dado. E ele ficou descrente e mau como
aquele que veio da linhagem dos pagãos, e não retinha o que eles lhe diziam e
falavam, como se fossem maus traidores, e esperou até que tiraram suas armas. 
Então ele os fez agarrar e colocar na prisão e os manteve assim por um ano, que
de lá nunca saíram. E foi-lhes, porém, tão bem que, tão logo estavam na prisão,
Nosso Senhor, que não se tinha esquecido deles, enviou-lhes para sua frente o
Santo Graal por causa de que lhes fizesse companhia, de cuja graça eles foram
saciados todos os dias e repletos tanto tempo quanto estiveram na prisão do
rei.

No tempo de um ano, então adveio que Galaat queixou-se perante Nosso Senhor e
falou: “Senhor, parece-me que estive o suficiente neste mundo. Se é vossa
vontade, assim me tomai brevemente”. No mesmo dia em que o rei Estorant
deitou-se viciado da dor da morte, então os fez vir perante ele e pediu-lhes
pela graça de que os tinha mantido injustamente. E eles o perdoaram.
Então ele morreu de pronto. 

E pois que estava sepultado, então os da cidade ficaram muito lamentosos, quando
não sabiam quem deveriam fazer rei e aconselharam-se por longo tempo. E nisto
que estavam em conselho, então ouviram uma voz que lá falou para eles: “Tomei o
mais jovem dos três companheiros, que deve bem vos proteger e aconselhar por
tanto tempo quanto estiver junto a vós”. E fizeram o mandamento da voz
e tomaram Galaat e o fizeram seu senhor, fosse de seu agrado ou lamento. E
colocaram-lhe a coroa sobre a cabeça, que lhe foi de grande lamento; quando
então ele viu que isso precisava ser, então lhes permitiu, de outra maneira o
teriam assassinado. Pois que Galaat veio a que devesse conservar terra, então
fez abrir, sobre a mesa de prata, uma arca de ouro e de rica pedra, que lá
cobria o Santo Vaso. E pela manhã, tão logo se levantou, então veio perante o
Santo Vaso e também seus companheiros e falaram sua oração para Nosso Senhor
Deus com bom coração. 

Quando se veio, ao tempo de um ano, no mesmo dia em que foi coroado, então se
levantaram o rei Galaat e seus companheiros muito contentes. E quando vieram ao
palácio, que se chamava de palácio espiritual, então viram à frente do Santo
Vaso estar um bom homem, vestido justamente como um bispo. E esse se assentava
sobre os joelhos e se batia à frente de seu coração. E tinha ao redor de si uma
multidão de anjos, justamente como se fosse Deus mesmo. E pois que por muito
tempo tinha ficado sobre seus joelhos, então se levantou e começou a falar
missa da real Mãe de Deus. E quando adveio o silêncio da missa, então fez a
pátena do cálice e clamou: “Galaat, vem cá à frente, servo de Deus Nosso Senhor
Jesus Cristo, deves contemplar aquilo por que tanto tempo ansiavas ver”. E ele
veio aqui à frente e viu o Santo Vaso. Então começou a palpitar muito
maravilhosamente, quando aquele que lá era mortal viu as coisas espirituais.
Quando Galaat, o rei, viu isso, então ergueu suas mãos perante o Céu e falou:
“Senhor, eu te agradeço que me completaste meu anseio, quando então vejo
justamente, visivelmente, o que nenhuma língua consegue contar, nem nenhum
coração o pode imaginar. Aqui vejo as grandes adesões, e as grandes honra e
dignidade, bem aqui vejo milagre sobre todos os milagres. E porque é assim,
amado Pai do Reino dos Céus, que me preencheste meu desejo, por causa disso
assim me dai aquilo pelo que sempre ansiei. Então vos peço\footnote{ Mais uma
alteração pronominal de tratamento.} que no ponto e na alegria em
que estou agora internamente, queirais tomar-me desta vida terrena e queirais
conduzir-me à companhia celestial”.

Tão logo Galaat tinha feito esta oração a Nosso Senhor, o bom homem, que estava
à frente do altar em igualdade a um padre, tomou o corpo de Nosso Senhor sobre
a mesa e o deu ao rei Galaat, e ele o recebeu muito humildemente com grande
devoção. Então falou o bom homem: “Sabes quem sou eu?” “Eu não, dizei-me
então!” “Então sabe que sou José, que lá era filho de José de Arimateia, e que
Nosso Senhor para cá enviou para fazer-te companhia. E sabes por que causa Ele
me enviou mais que qualquer outro? Por causa de que foste igual a mim em duas
coisas, nisto que viste os milagres do Santo Graal como eu fiz, e nisto que
foste casto como eu fui. Por causa disso é bem de direito que um homem casto,
puro, venha prestar companhia ao outro”. Pois que tinha falado esta
fala, então foi o rei Galaat a Parsifal e a Bohort e os beijou e falou para
Bohort: “Saudai por mim meu pai tão logo o vejais”. Então voltou Galaat para a
távola e sentou-se junto dela, sobre um joelho. E não ficou assim muito tempo,
que caiu morto sobre as lajes do salão, quando a alma lhe estava agora fora do
corpo, e os anjos que conduziam as almas pelo caminho com grandes alegrias e
louvavam muito a Nosso Senhor Deus.

Tão logo Galaat estava falecido, aconteceu bem então um grande milagre, quando
os dois companheiros viram visivelmente uma mão, que lá veio do Céu, quando não
viram de que corpo era a mão. E ela veio diretamente para o vaso e o tomou e o
conduziu para o Céu, de maneira que nunca desde então se conseguiu assim
esperar por aquele que pudesse falar que tivesse desde então visto o Santo
Graal. 

Quando Parsifal e Bohort viram que Galaat estava morto, ficaram tão tristes que
não podiam ter estado mais tristes. E se não fossem tão boas pessoas e de tão
boa vida, podiam bem logo ter caído em uma descrença, por causa da grande
tristeza que tinham, e a maior parte da terra, que estavam à vez muito tristes.
Então foi feita uma sepultura muito rica, e tão logo ele estava sepultado,
então Parsifal se pôs em uma clausura fora da cidade e vestiu roupas
espirituais. E Bohort estava junto dele, quando não transformou nunca suas
roupas do mundo, quando o fez por causa de que ainda desejava vir à corte do
rei Arthur.

Um ano e dois meses vive Parsifal na clausura, e depois se separa deste mundo.
E Bohort o fez sepultar junto de sua irmã e junto ao rei Galaat no palácio
espiritual. Pois que Bohort viu que ficou sozinho em terra tão longínqua,
justamente como se estivesse na Babilônia, então se apartou de lá e seguiu até
o mar assim armado e veio a um navio; e foi-lhe tão bem que em curto tempo veio
ao reino de Logres. E pois que estava na terra, então cavalga tanto tempo até
que veio perante Camlot\footnote{ Grafia alternativa a “Camelot”, presente ao
texto original.},  onde estava o rei. Então nunca a nenhum homem
foi feita tamanha alegria, como se lhe fez, quando consideravam tê-lo perdido,
quando estivera por muito tempo fora da terra.

Pois que se tinham assentado na corte, o rei Artur fez vir à frente os escribas,
que lá costumavam descrever as aventuras dos cavaleiros da corte do rei Arthur.
E quando Bohort tinha contado as aventuras do Santo Graal, da maneira como as
tinha visto, e foram descritas e conservadas na abadia de Salisbúria. Disto o
Mestre Gatiers começou a fazer o livro do Santo Graal do latim para o gaélico,
por vontade do rei Henrique, seu senhor, a quem ele tinha muito amor. Disto
assim se cala a aventura e não diz mais das aventuras do Santo Graal nem
daqueles que a levaram ao fim. 



