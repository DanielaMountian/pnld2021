
\chapter[Introdução, por Sylvie Debs]{Introdução}

\epigraph{A literatura popular existe em outros países, mas nenhuma é
tão relevante quanto a do Nordeste [\ldots{}]. Aqui, no Nordeste, ela
resiste e se transforma cada vez mais.}{Raymond Cantel\footnotemark}
\footnotetext{ Em \textit{La littérature populaire brésilienne}, p. 16.}

Patativa do Assaré, cujo verdadeiro nome é Antônio Gonçalves da Silva, nascido
no dia 5 de março de 1909 na Serra de Santana, pequena propriedade rural da
prefeitura de Assaré, ao sul do estado do Ceará, inclui-se na linhagem dos
cantadores sertanejos, continuando essa tradição. Oriundo de um meio muito
modesto, descobre a literatura por meio dos folhetos\footnote{ Folhetos: poemas
narrativos divulgados sob a forma de livretos de oito, dezesseis e, mais
raramente, trinta e duas páginas. Cf. Julie Cavignac, ``Mémoires en
mirior'', p. 50.} de cordel\footnote{
Cordel vem do provençal \textit{corde}; literatura de cordel: nome dado pelos
pesquisadores dos folhetos de feira. Cf. Julie Cavignac, op. cit., p. 49.} e
dos cantadores, repentistas e violeiros\footnote{ Cantadores de rua e de feiras
acompanhados de uma viola de dez cordas, improvisadores de versos que dão
conselhos em alexandrinos, hexassílabos ou pentassílabos e contam histórias em
sextilhas, quadras ou sonetos. Cf. Véronique Mortaigne, “Poètes-reporters et
menteurs professionnels”.} do Nordeste. Casado, pai de nove filhos, dedicou sua
vida ao trabalho nos campos de Assaré. No dia 23 de março de
1995, o presidente Fernando Henrique Cardoso rendeu uma homenagem pública ao
poeta popular, atualmente cego, conferindo-lhe a medalha José de Alencar quando
foi a Fortaleza para a celebração de seu octogésimo sexto aniversário.\footnote{
Arlen Medina, “Presidente condecora Patativa e recebe pedido pela refinaria”.}
Nessa ocasião, foi lançado o disco \textit{Patativa do Assaré: 85 anos de poesia}.
Patativa do Assaré, figura emblemática da poesia oral, tradicional e popular,
graças à sua memória impressionante, recitou versos que celebram as grandezas e
as misérias do sertão, e cantou, acompanhado por Raimundo Fagner, entre outros, o
célebre “Vaca estrela e Boi fubá” (símbolo da aflição do sertanejo diante das
amarguras do destino e da rudeza de sua exploração), que contribuíra para sua
notoriedade nacional nos anos 1970. A justaposição deliberada de alguns
elementos de uma sucinta biografia põe em perspectiva a denominação “mestre da
poesia popular”, conferida pelo ensaísta e cineasta Rosemberg Cariry, que
contribuiu largamente para a divulgação de sua obra.\footnote{ Rosemberg Cariry,
“Patativa do Assaré, um mestre da poesia popular”.} Assim, por meio da evocação
do itinerário pessoal do poeta e da análise de seus textos mais representativos,
propomo-nos a apresentar as características essenciais da poesia popular,
examinada aqui em uma dimensão mais larga, aquela da cultura popular nordestina.

Primeiro ponto de amarração de nosso estudo, o trabalho que Raymond Cantel --
primeiro pesquisador francês a se interessar pelo cordel -- conduziu durante
longos anos para a descoberta, o conhecimento, o estudo e a conservação da
literatura de cordel. Ele percorreu regularmente o Brasil a partir de 1959 para
recolher textos de repentistas, o que lhe valeu o título de “embaixador
itinerante”, outorgado pelos repentistas da Bahia.\footnote{ Annick Moreau,
“Introdução”, em Raymond Cantel, op. cit., p. 14.} Segundo ponto, a
aproximação de culturas populares proposta por Jean-Claude Passeron, que tenta
ir além da atitude relativista (até mesmo populista), assim como da atitude
legitimista (até mesmo miserabilista).\footnote{ Claude Grignon \& Jean-Claude
Passeron\textit{, A propos des cultures populaires}, p.~183.} Terceiro ponto,
nosso encontro pessoal com o poeta, em Assaré, que nos concedeu diversas
entrevistas e nos proporcionou a ocasião de assistir às suas improvisações.

\section{Uma aproximação da poesia popular}

A denominação “poesia popular” foi muitas vezes associada a um certo número de
representações negativas que a situam ao lado da literatura menor, em oposição à
Literatura. As conotações mais correntes que lhe são conferidas são aquelas da
simplicidade dos temas abordados e das ideias tratadas, facilidade de
versificação e banalidade das rimas, ingenuidade dos sentimentos expressos,
falta de originalidade e criatividade, pobreza de vocabulário, riqueza
estilística limitada, simbólica indigente.\footnote{Idem, p. 17.} É
nesses termos que Arthur Rimbaud (1854--1891) confessa seu interesse pela arte
popular:

\begin{quote}
Eu amava as pinturas idiotas, estofos sobre portais, cenários, lonas
de saltimbancos, tabuletas, estampas coloridas populares; a literatura fora de
moda, latim de igreja, livros eróticos sem ortografia, romances de nossas avós,
contos de fadas, livrinhos infantis, óperas velhas, estribilhos piegas, ritmos
ingênuos.\footnote{ Arthur Rimbaud\textit{, Uma temporada no inferno \&
Iluminações, alquimia do verbo}, p.~63.}
\end{quote}

\noindent Essa concepção se inscreve numa
tradição romântica que compara o povo e a expressão artística e popular a uma
imagem errônea, visto que idealizada: a imagem de um povo bom, bonachão,
trabalhador e virtuoso. De sua parte, o escritor e filósofo alemão J.G.~Herder
(1774--1803), um dos teóricos do movimento romântico ``Sturm und Drang'' (Tempestade e ímpeto), havia
defendido, tanto de um ponto de vista filosófico como literário, uma concepção
da história segundo a qual os diferentes tipos de civilizações e de culturas
seriam a expressão da alma popular, opondo ao ideal clássico -- resultado do
respeito a regras claramente enunciadas e reverente aos modelos da Antiguidade
greco-romana -- o gênio popular, expressão natural e espontânea. A poesia
popular, segundo ele, é “a obra anônima do Homem Natural, irmão histórico do Bom
Selvagem: ela é a ‘Naturpoesia’\,”.\footnote{ Claude Roy, \textit{Trésor de la
poésie populaire}, p. 8.} Nessa ideia, já estava presente a aproximação
proposta por Montaigne (1553--1592), persuadido de que o povo era capaz de se
exprimir espontaneamente:

\begin{quote}
A poesia natural e puramente natural possui
ingenuidade e graça, por onde ela se compara à principal beleza da poesia
perfeita segundo a arte: como se vê em vilarejos da Gasconha e nas canções que
se nos relatam sobre nações que não possuem conhecimento de ciência alguma,
tampouco de escrita”.\footnote{ Idem, p. 8.}
\end{quote}

Em outros termos, a poesia popular existiria ao largo de toda aprendizagem ou
respeito às regras acadêmicas e apresentaria êxitos dignos de serem
reconhecidos.

No contexto nordestino, é preciso recordar que a poesia popular inscreve-se na
tradição oral dessa região do interior: um de seus principais agentes, o
cantador, proveniente do meio rural e em geral analfabeto, improvisa ou narra,
graças à sua memória prodigiosa, 

\begin{quote}
a história dos homens famosos da região, os
acontecimentos maiores, as aventuras de caçadas e de derrubas de touros,
enfrentando os adversários nos desafios que duram horas e noites inteiras, numa
exibição assombrosa de imaginação, brilho e singularidade na cultura
tradicional.\footnote{ Luís da Câmara Cascudo, \textit{Dicionário do folclore
brasileiro}, p. 237.}
\end{quote}

\noindent A versificação utilizada, em geral a sextilha
hexassilábica ou a décima heptassilábica de rimas contínuas,\footnote{ Raymond
Cantel, op. cit., pp. 49 e 97.} parece ser mais a expressão de uma
técnica de memorização que a expressão de uma forma poética erudita, a serviço
da transmissão de um “saber simbólico: ciência, cultura popular, tradição”.
\footnote{ Julie Cavignac, op. cit., p. 57.} Daí, a própria escansão dos
poemas é muitas vezes surpreendente pela falta de preocupação expressiva:

\begin{quote}
Nenhuma preocupação de desenho melódico, de música bonita. Monotonia. Pobreza.
Ingenuidade. Primitivismo. Uniformidade\ldots{} Não se guarda a música de
colcheias, martelos e ligeiras. A única obrigação é respeitar o ritmo do verso.
\footnote{ Luís da Câmara Cascudo, op. cit., p. 237.}
\end{quote}

\noindent A declamação se
atém ao essencial: a narrativa dos acontecimentos.

A convivência com os chamados textos de poesia clássica, assim como a leitura da
obra de Patativa do Assaré, permitem pôr em perspectiva essa primeira aproximação
e interrogar a conformidade dessas conotações evocadas precedentemente. Sem
dúvida, conviria debruçar-se mais adiante sobre as temáticas abordadas para
perceber que sob essa aparente ingenuidade esconde-se uma profunda experiência
da vida cotidiana que confere uma dimensão simbólica determinante à sua obra.
Com efeito, como ressalta Claude Roy, 

\begin{quote}
o que nos toca do nosso folclore não é
ele ser a obra ``de quem não sabe'', mas, ao contrário, nascer do sofrimento e da
alegria, da malícia e do coração daqueles que sabem muito bem. Eles sabem o que
é ter fome ou dor de amor, ir à guerra quando não se queria ou trabalhar com a
última das forças. E estes encontram muito exatamente, ao longo do tempo,
palavras insubstituíveis para manifestar sua dor ou sua felicidade, para embalar
suas mágoas ou exprimir sua cólera.\footnote{ Claude Roy, op. cit., p.~17.}
\end{quote}

\noindent Restituindo-se a obra de Patativa do Assaré ao contexto sertanejo,
considerando a influência das tradições dos trovadores, dos repentistas, dos
violeiros e da literatura de cordel,\footnote{ Julie Cavignac, \textit{Figures et
personnages de la culture nordestine dans la littérature de cordel au
Brésil,} em Caravelle, \textit{Cadernos do mundo hispânico e luso"-brasileiro},
p. 69: “es una literatura de pobres para pobres sobre todo [\ldots{}]”.} é
forçoso reconhecer na voz do poeta popular o eco dos sofrimentos, das alegrias e
das desgraças da população nordestina do sertão:

\begin{quote}
Poesia telúrica, colhida da
terra, dos roçados, como se estivesse apanhando feijão, arroz, algodão, ou
quebrando milho e arrancando batata e mandioca. Sua inspiração não é fruto de
estudos. Ela germina dentro de si como a semente nas entranhas da
terra.\footnote{ Pe. Antônio Vieira, “Patativa do Assaré”, em \textit{Ispinho e
Fulô}, p. \textsc{vii}.}
\end{quote}

Testemunha então de um modo de vida, mas também reivindicação de valores
próprios, elaboração de uma identidade. Por isso, ele é apresentado como o
“verdadeiro, autêntico e legítimo intérprete do sertão”.\footnote{ Plácido Cidade
Nuvens, \textit{Patativa e o universo fascinante do sertão}, p. 15.} Com
efeito, uma das dimensões mais marcantes da obra de Patativa do Assaré é a
preocupação de descrever a vida cotidiana do sertão e, com esse testemunho,
protestar o reconhecimento da dignidade, da integridade e da modéstia do
camponês sertanejo, em oposição à arrogância do cidadão urbano ou do brasileiro
do Sul. Parece que a afirmação de sua própria identidade passa mais
frequentemente pelo confronto com o outro, como chama atenção o título da
compilação: \textit{Cante lá que eu canto cá}. Esta última, composta a partir de uma
seleção de textos feita pelo próprio autor com a intenção de definir suas
preferências literárias, traz o seguinte subtítulo: ``Filosofia de um trovador
nordestino''. É, portanto, referindo-nos de uma só vez ao conjunto dos poemas
publicados e à vida de Patativa do Assaré que tentaremos depreender as
características próprias da sua obra.

\section{Referências biográficas}

\subsection{Uma infância sertaneja}

Segundo filho de um agricultor pobre da região do Cariri, havendo perdido muito
jovem a visão de um dos olhos em consequência de uma doença, órfão de pai aos
oito anos, Antônio Gonçalves da Silva é, naturalmente, conduzido a ajudar sua
mãe e sua família participando do trabalho no campo, meio de subsistência
tradicional para os habitantes dessa região. Escolarizado durante seis meses
quando tinha 12 anos, ele reconhece que seu mestre, embora extremamente
atencioso e generoso, era precariamente letrado e não sabia ensinar a pontuação.
É assim que ele aprende a ler sem ponto nem vírgula, como se o ritmo das
palavras fosse dado unicamente pela voz. Essa estranha aprendizagem, em
realidade, é apenas a expressão profunda da oralidade que caracteriza a cultura
popular e a tradição dos poetas-repórteres.\footnote{ Véronique Mortaigne, op.
cit.} Como a maior distração do jovem Antônio, desde seu retorno dos campos,
era ler ou escutar seu irmão mais velho ler folhetos da literatura de cordel,
ele descobriu muito cedo sua vocação poética e iniciou, ao contato desta
literatura, a composição de versos:

\begin{quote}
De treze a quatorze anos comecei a fazer
versinhos que serviam de graça para os serranos, pois os sentidos de tais versos
eram o seguinte: brincadeiras\footnote{ Ver jogos e divertimentos poéticos em
Luís da Câmara Cascudo, op. cit., p. 188.} de noite de São João,
testamento do Judas,\footnote{ Sátira de personagens locais célebres, o sábado
Santo. Cf. Luís da Câmara Cascudo, op. cit., p. 493.} ataque aos
preguiçosos que deixavam o mato estragar os plantios da roça etc.\footnote{
Patativa do Assaré, “Autobiografia”, em \textit{Inspiração nordestina}, p. 13.}
\end{quote}

\noindent Aos 16 anos, adquire uma viola de dez cordas e decide fazer improvisações
segundo a tradição sertaneja dos violeiros, tratando de todos os assuntos
concernentes à sua experiência profissional, segundo o modelo motivo-glosa.
Põe-se a cantar por prazer, com a esperança de ser convidado para as festas --
comemoração de santos, casamentos -- e participar, assim, da vida local: 

\begin{quote}
A poesia sempre foi e ainda está sendo a maior distração da minha vida. O meu
fraco é fazer verso e recitar para os admiradores, porém nunca escrevo meus
versos. Eu os componho na roça, ao manejar a ferramenta agrícola e os guardo na
memória, por mais extenso que seja.\footnote{ Idem, ibidem.}
\end{quote}

\noindent confessa ele.
Assim, se continuou a entregar-se às improvisações pelo prazer, a poesia que ele
destina à transcrição está intimamente ligada ao ritmo do trabalho cotidiano,
acompanhando os gestos dos trabalhos do campo e composta mentalmente ao longo
dos anos, servindo-se de capacidades impressionantes de memorização.

\subsection{Um poeta itinerante}

Aos vinte anos, por ocasião de uma visita ao vilarejo de um primo materno,
encantado pelas improvisações de Antônio, pediu autorização à sua mãe para
seguir com ele para o estado do Pará, propondo-se, de sua parte, a auxiliar nas
necessidades do jovem e consentir que este retornasse a seu lar sempre que
quisesse. Foi nessa ocasião que conheceu o escritor cearense José Carvalho de
Brito, que lhe consagrou um capítulo em seu livro intitulado \textit{O matuto cearense e
o caboclo do Pará}. Além disso, este publica os primeiros textos de Antônio
Gonçalves da Silva no \textit{Correio do Ceará}, no qual colaborava. Esses textos foram
acompanhados de comentários nos quais José Carvalho de Brito comparava a poesia
espontânea de Antônio Gonçalves da Silva à pureza do canto da patativa, pássaro
do Nordeste. Foi assim que nasceu o pseudônimo de Patativa.\footnote{ Rosemberg
Cariry, “Patativa do Assaré: sua poesia, sua vida”, p. 34.} E, para
distingui-lo de outros improvisadores, acrescia-se o topônimo de sua vila natal:
Assaré. Patativa do Assaré empreendeu então uma viagem a Belém, e em seguida a
Macapá, onde ficou dois meses. Julgando a vida relativamente insípida, e não
apreciando o fato de deslocar-se sistematicamente por barco para ir de uma casa
à outra, decidiu retornar a Belém, onde continuou suas improvisações em
companhia de outros poetas como Francisco Chapa, Antônio Merêncio e Rufino
Galvão. Ao termo de cinco meses, não resistindo mais aos ataques de saudade,
decidiu voltar a viver no Ceará.

\subsection{A consagração oficial}

Em seu retorno, José Carvalho de Brito entregou-lhe uma carta de recomendação
para obter uma audiência com a dra.~Henriqueta Galeno, filha do poeta Juvenal
Galeno. Ele foi recebido com honras dignas de um “poeta de classe, um poeta de
cultura, um poeta erudito”\footnote{ Idem, p. 34.} e improvisou, em seu
salão, acompanhado de sua viola. De volta a Assaré, retomou os trabalhos do
campo, aos quais dedicou o resto de sua vida. O latinista José Arraes de
Alencar, que o ouvira improvisar na Rádio Araripe, pergunta-lhe por que não
publicava seus textos, tão “dignos de atenção e próprios de
divulgação”.\footnote{ Idem, p. 35.} Patativa do Assaré argumentou que
não era mais que um pobre agricultor e que não dispunha, portanto, de meios para
publicar sua obra. José Arraes de Alencar propõe-lhe uma solução: ele se
encarregaria das negociações com o editor Borçoi, no Rio de Janeiro, e Patativa
do Assaré lhe reembolsaria os custos da impressão com o produto da venda dos
livros. É assim que surge a sua primeira compilação, \textit{Inspiração nordestina}, em
1956. No prefácio, José Arraes de Alencar sublinha as qualidades particulares
dos poetas nordestinos:

\begin{quote}
Nada arranca aos rapsodos nordestinos a admirável
espontaneidade, que é um milagre da inteligência, um inexplicável poder do
espírito, faculdade portentosa daqueles homens simples e incultos, de cuja boca
prorrompem, em turbilhões, os mais inspirados versos, as trovas mais dolentes e
sentimentais, ou épicas estrofes, que entusiasmam e arrebatam.\footnote{ José
Arraes de Alencar, “Prefácio”, em \textit{Inspiração nordestina}, p. 9.}
\end{quote}

\noindent Superado seu primeiro receio de não estar em condições de lhe reembolsar, Patativa
do Assaré aceitou. O sucesso da antologia permitiu-lhe uma segunda edição em
1966, enriquecida de novos textos: \textit{Cantos de Patativa}. Nessa ocasião, ele passou
quatro meses no Rio de Janeiro; entretanto, a venda de seus livros ocorreu
essencialmente no Ceará.

\subsection{A divulgação da obra}

Em 1970, o professor José de Figueiredo Filho publicou uma nova coletânea de
poemas acompanhada de seus comentários: o \textit{Patativa do Assaré}. Em 1978, a partir
da iniciativa do professor Plácido Cidade Nuvens (que trabalha na Fundação do
Padre Ibiapina, cuja missão é preservar e divulgar a cultura popular do Cariri),
a compilação \textit{Cante lá que eu canto cá} -- considerada até hoje a compilação da
maturidade -- foi publicada pela editora Vozes. Em 1988, surge uma nova
antologia de textos de Patativa do Assaré, intitulada \textit{Ispinho e Fulô}, sob a
direção de Rosemberg Cariry. Essa edição compreende uma seleção de textos
publicados nos folhetos, jornais, revistas ou discos, produtos de numerosos
recitais feitos pelo país. Mais recentemente, na ocasião de seu 86º aniversário,
a Secretaria de Cultura do Estado do Ceará publicou uma coletânea de textos em
homenagem ao poeta, \textit{Aqui tem coisa}, que salienta sua originalidade, sua
ancoragem na oralidade, graças à prática da improvisação e à técnica de desafios
poéticos:

\begin{quote}
Métrica, ritmo e rima fluem com a naturalidade com que enuncia seu
canto, o que ele diz é transcrito para o papel, mas continua fiel aos códigos de
transmissão oral. É como se ele estivesse em permanente peleja, não contra um
rival de ofício, que ninguém chegaria à sua estatura, mas com a própria poesia.
Ele é o seu opositor e o seu duplo. A oralidade não seria decorrente de sua
cegueira, não que ele também não retome uma tradição que passa por Homero,
Aderaldo e Borges.\footnote{ Gilmar de Carvalho, “Prefácio”, em \textit{Aqui tem
coisa,} p. 8.}
\end{quote}

\noindent Assim, Patativa do Assaré, como mestre da poesia oral, nunca
tentou publicar um texto com seus próprios meios, mas foi sempre publicado pelos
admiradores de sua obra. Da mesma forma, ele continua a ser solicitado tanto
pelos amadores como pelos especialistas da cultura popular, não somente
brasileiros mas também estrangeiros, que se interessam ao mesmo tempo pelo
processo de criação e pela transmissão dessa tradição nordestina.

\section{Apresentação da obra}

\subsection{Patativa do Assaré, um poeta da oralidade}

Na condição de herdeiro da tradição nordestina, os primeiros esboços da obra de
Patativa do Assaré, improvisações e encomendas, conforme ressaltamos, são
marcados pelo aspecto lúdico e comemorativo. Poemas de circunstância, associados
aos acontecimentos sociais, religiosos, em relação direta com o presente, únicos
e efêmeros: festas de santos, casamentos, aniversários. Poesia improvisada a
partir de um esboço tradicional, poesia repetitiva por suas formas e temas,
personalizada em função de seu destinatário. Poesia declamada ou cantada, ela
participa plenamente da vida da comunidade:

\begin{quote}
age falando, cantando,
representando, dançando no meio do povo, nos terreiros das fazendas, nos pátios
das igrejas nas noites de ``novena'', nas festas tradicionais do ciclo do gado,
nos bailes do fim das safras de açúcar, nas salinas festas dos ``padroeiros'',
potirum, ajudas, bebidas nos barracões amazônicos, espera da ``Missa do Galo''; ao
ar livre, solta, álacre, sacudida, ao alcance de todas as críticas de uma
assistência que entende letra e música, todas as gradações e mudanças do
folguedo.\footnote{ Luís da Câmara Cascudo, \textit{Literatura oral no Brasil},
p. 27.}
\end{quote}

\noindent Convém ressaltar que Patativa do Assaré, entregando-se sempre a esse
gênero de improvisações, tem uma parte importante da obra que não foi nem será
nunca transcrita. Esse aspecto efêmero e circunstancial é, com efeito, uma das
características da poesia oral tradicional.

Quando se descobre a transcrição dos poemas de Patativa do Assaré, o primeiro
elemento determinante da oralidade da obra é o recurso sistemático do emprego de
uma língua falada, que retoma o estilo e a pronúncia popular, a saber, a
utilização do que José Arraes de Alencar definiu como ``língua cabocla'':

\begin{quote}
a linguagem sertaneja, de tonalidade própria, fértil em metafonias e metáteses,
avessa aos esdrúxulos, com frequente abrandamento ou amolecimento e vocalização
de consoantes e grupos consonantais, com a eliminação das letras e fonemas
finais.\footnote{ José Arraes de Alencar, op. cit., p. 11.}
\end{quote}

\noindent Os primeiros versos de “Coisas do meu sertão” são assim transcritos:

\begin{verse}
Seu dotô que é da cidade \\*
Tem diproma e posição \\
E estudou derne minino \\*
Sem perdê uma lição 
\end{verse}

\noindent por

\begin{verse}
Senhor Doutor que é da cidade\\*
Tem diploma e posição\\
E estudou desde menino\\*
Sem perder uma lição
\end{verse}

A marca oral e regional era tão intrínseca à primeira compilação que foi
publicada com um Elucidário que propunha três esclarecimentos diferentes ao
leitor: uma simples restituição fonética (\textit{biête} por bilhete ou \textit{muié} por mulher),
uma correspondência referencial (cão por diabo) e uma explicação denotativa
(tipoia: rede pequena, rede velha). A necessidade desse Elucidário é
justificada pela observação de José Arraes de Alencar:

\begin{quote}
a linguagem cabocla -- o
linguajar da rude gente sertaneja, tão crivado de erros, de mutilações e
acréscimos, de permutas e transposições, que os vocábulos, com frequência, se
desfiguram completamente, sendo imprescindível um elucidário para o leitor não
habituado a essas formas bárbaras e, ao mesmo tempo, refeitas de típico e
singular sabor.\footnote{ Idem, ibidem.}
\end{quote}

\noindent Essas marcas da
oralidade confirmam a origem rural do poeta e reforçam o caráter sertanejo do
universo descrito. O registro de língua utilizado, a alteração das palavras e o
vocabulário regional conferem a esses textos todo o sabor e a originalidade da
língua do interior das terras, do sertão.

Uma outra marca significativa dessa oralidade é a forte presença, muitas vezes
desde o título, da função conativa da linguagem: interpelação do ouvinte como
\textit{Cante lá que eu canto cá}, interrogações como “Você se lembra?”, “Seu Dotô me
conhece?”, destinação como “Ao leitor”, “Aos poetas clássicos”, “À minha esposa
Belinha”. Da mesma forma, os primeiros versos de seus poemas instauram,
geralmente, o ritual discursivo, seja como forma de indagação: “Querem saber
quem eu sou?” (\textit{Aqui tem coisa}, p. 63); seja sob forma de oração: “Quero que me dê licença/
para uma história contá.” (\textit{Cante lá que eu canto cá}, p. 47); seja por uma saudação: “Boa noite, home e menino/ e muié dêste lugá.” (\textit{Inspiração nordestina}, p. 27); 
seja ainda por uma ordem: “Vem cá,
Maria Gulora,/ escuta, que eu vou agora/ uma coisa te contá.” (Idem, p. 47). Enfim,
a invocação do interlocutor abre diversos poemas: as formas mais utilizadas são
“Seu Moço” (Idem, pp. 19, 51, 99) e “Seu Dotô” (Idem, pp. 60, 66 e 69). Encontram-se
variantes sob a forma de “Meu filho querido” (Idem, p. 132), “Meu amigo” (Idem, p.
209), “Minha gente” (Idem, p. 206), “Sinhô Dotô”
(Idem, p. 203).

A relação de vizinhança está sublinhada pelo emprego do tom familiar: \textit{meu}, que
indica igualmente o enraizamento do poeta em seu meio. Esses termos de
endereçamento traduzem ao mesmo tempo o respeito de uma hierarquia social
estrita, em uma sociedade onde a taxa de analfabetismo é elevada. O poeta, como
personagem familiar, é originário do mesmo meio, dirigindo-se em pé de igualdade
a seus interlocutores, seja ao mais rico, ao mais poderoso ou ao mais diplomado,
pedindo licença para contar uma história simples à sua maneira -- último
elemento enfim, todavia essencial, o próprio poeta Patativa do Assaré. Não
havendo jamais escrito texto algum e dotado de uma notável capacidade de
memorização (é capaz de recitar qualquer uma de suas composições, qualquer seja
sua antiguidade), ele continua a praticar a improvisação em todas as
circunstâncias:

\begin{quote}
A agilidade do improviso, o inesgotável repertório de
situações, as respostas instantâneas às sugestões recebidas acentuam o
repentista à capela [\ldots{}]. Métrica, ritmo e rima fluem com a naturalidade
com que enuncia seu canto. O que ele diz é transcrito para o papel, mas continua
fiel aos códigos da transmissão oral.\footnote{ Gilmar de Carvalho, op.
cit., p. 8.}
\end{quote}

\noindent É frequente que o poeta, após perguntar o nome e algumas
informações sobre as pessoas que vêm vê-lo, improvise um pequeno poema no qual
traça um retrato de seu visitante, apesar de sua cegueira. Muito atento durante
as discussões, sua habilidade lhe permite apoderar-se da personalidade de seu
interlocutor. A voz permanece para ele como instrumento privilegiado do
conhecimento e da comunicação.

\subsection{Patativa do Assaré, um poeta popular}

Fiel à tradição dos poetas de cordel, ele mesmo autor de cordéis, Patativa do
Assaré compõe uma poesia essencialmente narrativa, que testemunha a história
cotidiana do sertanejo e torna-se, de qualquer maneira, “o mediador encarregado
de traduzir o mundo exterior aos sertartejos”.\footnote{ Julie Cavignac, op.
cit., p. 59.} Essa obra, “nascida no seio do povo, aplaudida e amada por esse
mesmo povo”,\footnote{ Rosemberg Cariry, “Patativa do Assaré\ldots{}”, p. 1.}
coloca-se ao lado das referências literárias do Nordeste como \textit{A bagaceira}, \textit{Pedra
bonita}, \textit{Vidas secas}, \textit{O quinze}, \textit{Grande
sertão: veredas},\footnote{ Obras de José Américo de
Almeida, José Lins do Rego, Graciliano Ramos, Raquel de Queiroz e Guimarães
Rosa, respectivamente.} na medida em que o autor contribui para a elaboração de uma imagem da
identidade nordestina e de representações simbólicas que nos permitem
compreender melhor os valores fundamentais do sertanejo através das personagens
encenadas”.\footnote{ Plácido Cidade Nuvens, op. cit., pp. 33--36.} Se a
origem social do poeta e a origem social de seu público são determinantes para
qualificar essa poesia como popular, é preciso igualmente considerar outros
critérios que permitam caracterizar sua obra: os assuntos tratados, a função do
poeta e a filosofia empregada. Ao longo da leitura de títulos de cordéis
recentemente editados pela \textsc{urca},\footnote{ \textsc{urca}: Universidade Regional do Cariri,
situada em Juazeiro do Norte, um dos mais importantes centros de impressão da
literatura de cordel.} constata-se a presença de numerosos temas habitualmente
abordados na literatura popular nordestina: o ciclo religioso e o messianismo, a
tradição épica, a descrição da vida do Nordeste com seus flagelos, caatinga,
inundações, secas, migrações: “Saudação ao Juazeiro do Norte”, “História de
Aladim e a lâmpada maravilhosa”, “ABC do Nordeste flagelado”, “A triste
partida”, “Emigração”\ldots{}\footnote{ Raymond Cantel, op. cit. Ver o artigo
“La Littérature populaire du Nordeste brésilienne”, pp. 79-105.} Uma
leitura mais abrangente da obra descobre também a presença de personagens
tradicionais do sertão: o vaqueiro, o caboclo, o roceiro, o caçador, o mendigo,
sem esquecer os animais familiares, como o cavalo, o boi e o cachorro. É preciso
ressaltar, enfim, a grande variedade de personagens que habitam os poemas e que
são nomeados de forma tradicional e popular, seja por referência ao pai (Zé
Geraldo), à mãe (Zé de Ana), ou à atividade profissional (Ciça do Barro
Cru).\footnote{ Plácido Cidade Nuvens, op. cit., p. 70.} Entretanto,
nem as narrativas das aventuras de um desses habitantes do sertão (“Brosogó,
Militão e o Diabo”, “As façanhas de João Mole”, “Vicença e Sofia”, ou “O castigo
de mamãe”), tampouco a descrição das dificuldades encontradas pelo sertanejo são
jamais apresentadas fora de uma preocupação educativa: divertindo o ouvinte ou o
leitor, o poeta tem por tarefa instruí-lo, transmitindo valores morais. Do ponto
de vista da função determinada para a poesia popular, encontramos paradoxalmente
um dos componentes do ideal clássico: “agradar e instruir”. Quanto à estrutura
dos textos, eles estão muito próximos do modelo da fábula: conduzem o leitor à
abertura, narram, formulam a moral no desfecho. Com efeito, os primeiros versos
focalizam as intenções do autor ou os valores morais que ele se propõe a
transmitir aos receptores, como na abertura de “As façanhas de João Mole”:

\begin{verse}
Neste pequenino drama\\*
O caro leitor verá\\
Que dentro de cada homem\\
Um pouco de ação está\\
E um só homem sem coragem\\*
No nosso mundo não há.
\end{verse}

Essa vontade didática está claramente afirmada, na medida em que os cordéis
geralmente terminam com uma evocação direta do leitor e uma lembrança da lição
que convém extrair da história escutada. A última estrofe do cordel citado
acima encerra-se nestes termos:

\begin{verse}
Agora, caro leitor,\\*
Não desaprove o que digo\\
Todo homem tem coragem\\
O rico, o pobre e o mendigo\\
No ponto da hora H\\
Insulte um, e verá\\*
O mais feroz inimigo.
\end{verse}


Os valores morais aos quais se refere Patativa do Assaré não são fundados sobre
os princípios teóricos; são ou simples heranças de gerações anteriores, ou fruto
direto de uma experiência vivida. Sua concepção do mundo e sua relação com o
outro repousam sobre uma crença que se poderia qualificar de humanista ou de
cristã, e que corresponde, além disso, à uma realidade cultural
nordestina.\footnote{ Plácido Cidade Nuvens, op. cit., p. 23.} Assim, a
abertura de “Brosogó, Militão e o Diabo” afirma como ponto de partida os
seguintes valores:

\begin{verse}
O melhor da nossa vida\\*
É paz, amor e união\\
E em cada semelhante\\*
A gente vê um irmão.
\end{verse}

Raymond Cantel já havia, por sua vez, sublinhado largamente as intenções
moralistas da literatura popular nordestina:

\begin{quote}
Os sentimentos tradicionais, a
família e o amor do próximo são celebrados, mas trata-se, antes de tudo, de
ensinar ao sertanejo, sempre distraindo-o, que se ele não souber resistir aos
impulsos de seu temperamento, ele terá de suportar as consequências.\footnote{
Raymond Cantel, op. cit., p. 23.}
\end{quote}

\noindent Patativa do Assaré explica a origem
de certas composições por essas mesmas razões: melhor que punir um de seus netos
desobedientes ou um menino da vizinhança que lhe havia enganado para roubá-lo,
ele optou por recorrer à poesia, com o duplo objetivo de expor publicamente
aquele que cometeu uma falta (punição que ele julga mais eficaz do que um acerto
de contas cara a cara) e ensinando-o, ao mesmo tempo, o perdão e a boa conduta
(“Incelência das Cuinhas”).\footnote{ Rosemberg Cariry, “Patativa do Assaré: sua
poesia, sua vida”, p. 43.} Essa atitude de sabedoria popular constitui um
ensinamento moral prático que toma suas referências no cotidiano.

É assim que Patativa do Assaré preenche sua função de educador tanto junto às
crianças consideradas por ele um elemento fundamental -- “A criança, para mim, é
a maior riqueza do mundo”\footnote{ Idem, p. 45.} --, como junto aos
seus compatriotas sertanejos:

\begin{quote}
Ele [o poeta] deve empregar a sua lira em
benefício do povo, em favor do bem comum. Ele deve empregar a sua poesia numa
política em favor do bem comum, uma política que requer os direitos humanos e
defende o direito de cada um.\footnote{ Idem, pp. 52 e 104.}
\end{quote}

\noindent Em um contexto de miséria e analfabetismo largamente propagado, em outros termos, em
meio à ausência de estruturas educativas de base, o poeta popular desempenha um
papel importante no despertar da consciência cívica e política. Patativa do
Assaré afirma sua solidariedade com a luta dos sertanejos pelo reconhecimento de
seus direitos e com a reivindicação de uma reforma agrária que lhes permitiria
ter um nível de vida mais digno: “A temática social que domina sua poesia está
assentada em aspirações universais de justiça e igualdade, sem qualquer
refinamento ideológico”.\footnote{ Cláudio Cerri, “Canto da terra”, p. 49.}

Agricultor, ele denuncia a morosidade dos políticos que jamais tentaram eliminar
a seca, flagelo maior do Nordeste, que é a origem das constantes migrações de
sertanejos:

\begin{quote}
A seca pertence ao império da natureza, mas pode ser resolvida pelo
homem. Em países de clima igual ou pior que o nosso, o problema de abastecimento
de água foi superado. A diferença aqui é que os donos do poder não se interessam
pela solução. Eles vivem do problema.\footnote{ Idem, ibidem.}
\end{quote}

\noindent declara Patativa do Assaré. Na coletânea \textit{Cante lá que eu canto cá}, confere uma posição
preponderante à questão da terra, e numerosos poemas evocam essa realidade
dramática: “O poeta da roça”, “Eu e o sertão”, “É coisa do meu sertão”, “Vida
sertaneja”, “Caboclo roceiro”, “Cabocla da minha terra”, “No terreiro da
choupana”, “A terra é natura”, “O retrato do sertão”, “Serra de Santana ”,
“Minha serra”, “Coisas do meu sertão”, “ABC do Nordeste flagelado”. O poeta, com
efeito, ergue, em “Terreiro da choupana”, não somente um atestado amargo da
realidade cotidiana:

\begin{verse}
Minha vida é uma guerra\\*
E duro o meu sofrimento\\
Sem tê um parmo de terra:\\
Eu não sei como sustento\\*
A minha grande famia [\ldots{}]
\end{verse}

\noindent mas reivindica a necessidade de uma reforma agrária:

\begin{verse}
A bem do nosso progresso\\*
Quero o apoio do Congresso\\
Sobre uma reforma agrária\\
Que venha por sua vez\\
Libertar o camponês\\*
Da situação precária.
\end{verse}


Ao defender, assim, a principal reivindicação dos habitantes do sertão, ele
torna-se verdadeiramente a voz do Nordeste e o símbolo de um processo de
reconhecimento dos direitos fundamentais: “Em todas as grandes lutas sociais e
políticas do Ceará, Patativa disse: presente”.\footnote{ Rosemberg Cariry,
“Patativa do Assaré\ldots{}”, p. \textsc{ii}.} Esse comprometimento faz com que um certo
número de poemas -- como “Triste partida”, “Lição do Pinto”, “Vaca Estrela e Boi
Fubá” -- se tornassem emblemas do povo nordestino, atestando a importância do
sucesso que ele alcançou junto aos sertanejos. Com efeito, Patativa do Assaré
passou de uma poesia sentimental e lírica para uma poesia de protesto: “uma
poesia que pede reforma agrária, reclama contra o abandono do nordestino, contra
o sistema de meação vigente no campo, contra a seca”.\footnote{ J.M.~Andrade,
“Voz do sertão”, p. 80.}

\subsection{Patativa do Assaré, uma identidade sertaneja}

É verdade que não somente a língua, os personagens e o cotidiano descrito
pertencem ao mundo rural sertanejo que Patativa do Assaré viu nascer e viver,
mas também as aspirações sociais, as reivindicações políticas e econômicas. O
combate que ele conduz é aquele do “caboclo roceiro, do camponês sertanejo, da
classe matuta”.\footnote{ Plácido Cidade Nuvens, op. cit., p. 52.} Com
efeito, o elemento mais tocante da identidade sertaneja é essa evocação
constante de uma vida extremamente difícil, de uma terra particularmente hostil,
de um universo encerrado sobre si mesmo. Patativa do Assaré testemunha de forma
direta em “Cante lá que eu canto cá”: 

\begin{verse}
Cá no sertão eu infrento\\*
A fome, a dó e a misera.\\
Pra sê poeta divera\\*
Precisa tê sofrimento 
\end{verse}

\noindent ou ainda em “Coisas do meu sertão”:

\begin{verse}
Pois aqui vive o matuto\\*
De ferramenta na mão.\\
A sua comida é sempre\\*
Mio, farinha e fejão
\end{verse}

As numerosas expressões colhidas por Plácido Cidade Nuvens em seu estudo
\textit{Universo fascinante do sertão} -- expressões referentes a um cotidiano brutal,
massacrante, absurdo, asfixiante -- traduzem essa luta constante do sertanejo:

\begin{quote}
vida apertada, lida pesada, sina tirana, grande labutação, vida de cativo,
correr estreito, tormento do triste agregado, vida mesquinha, rojão seguro, gaio
duro, situação crua, quebradeira, horrível peleja, aperreio, grande canseira,
meu cativeiro, constante lida, batalha danada, verdadeiro inferno, situação
mesquinha. Todas estas denominações refletem o abandono, o isolamento, a extrema
penúria. Manifestam a tenacidade, a obstinação, a resistência do
sertanejo.\footnote{ Plácido Cidade Nuvens, op. cit., p. 53.}
\end{quote}

A coragem, a paciência, a resistência à fadiga aparecem como atributos
fundamentais dos sertanejos. A poesia cabocla, feita de suor, de fome e de
fadiga, e nascida dessa miséria, reivindica sua diferença face à poesia de
salão, em “O poeta da roça”:

\begin{verse}
Meu verso rasteiro, singelo e sem graça, \\*
Não entra na praça, no rico salão, \\
Meu verso só entra no campo e na roça \\*
Nas pobre paiça, da serra ao sertão 
\end{verse}

Uma das figuras recorrentes dessa afirmação de identidade é a oposição: o
sertanejo se determina essencialmente pela diferença. O poema inaugural de sua
obra escrita, “Ao leitô” (\textit{Inspiração nordestina}, p. 15), avisa ao leitor que ele vai descobrir uma
poesia marcada pela deficiência; a ladainha das negações e das restrições
sublinha estilisticamente essa confissão:

\begin{verse}
Não vá percurá neste livro singelo\\*
Os cantos mais belo da lira vaidosa,\\
Nem brio de estrela, nem moça encantada,\\*
Nem ninho de fada, nem cheiro de rosa. 
\end{verse}

Em \textit{Cante lá que eu canto cá}, o poeta sertanejo salienta, sempre por negações
anafóricas, a pobreza que o condena ao duro trabalho da terra (“Vida
sertaneja”):

\begin{verse}
Sou matuto sertanejo\\*
Daquele matuto pobre\\
Que não tem gado nem queijo,\\*
Nem ouro, prata nem cobre
\end{verse}

Igualmente, o sistema de negações parece ser a pedra angular de uma percepção
desvalorizada de si. Patativa do Assaré tece, paralelamente a isso, uma rede
semântica de conotação negativa. No poema “O poeta da roça”, apresenta-se como
“cantô de mão grossa,/ poeta das brenha,/ não tenho sabença,/ meu verso rasteiro,/
singelo e sem graça” (\textit{Inspiração nordestina}, p. 16). Em “Seu Dotô me conhece?”, ele se define como
“o mendigo sem sossego,/ desgraçado, aquele roceiro/ sem camisa e sem dinheiro”
(Idem, p. 69). Em “No meu sertão”, Patativa do Assaré salienta sua falta de
educação “Inducação eu não tenho” (Idem, p. 75), em “Aos poetas clássicos”, ele
recorda sua origem humilde: “Sou um caboclo roceiro,/ sem letra e sem instrução”
(\textit{Cante lá que eu canto cá}, p. 19). Em “O retrato do sertão”, ele recorda que é “poeta de mão calosa,/
[\ldots{}] que não conhece cinema,/ teatro, nem futebol” (Idem, p. 238). Em
“Emigrante nordestino no Sul do país”, ele define seus compatriotas como
“vagando constantemente,/ sem roupa, sem lar, sem pão”. Toda descrição, toda
desvalorização se faz sempre em referência ao cidadão urbano, ao letrado, ao
rico, ao Sul.

Patativa do Assaré propõe uma visão dicotômica do mundo tanto sobre o plano
espacial (sertão / cidade ; Nordeste / Sul) como sobre o plano temporal (passado
/ presente). Na coletânea \textit{Cante lá que eu canto cá}, essa oposição espacial
anunciada desde o título se traduz por uma constante recordação das diferenças
de identidade. A oposição mundo urbano / mundo rural está construída a partir de
diferenças socioculturais e do sistema de valores: educação e saber contra
analfabetismo e ignorância; dinheiro e bem-estar contra pobreza e sofrimento;
hipocrisia e vaidade contra honestidade e modéstia. Patativa do Assaré rejeita o
“poeta niversitaro,/ poeta de cademia/ de rico vocabularo/ cheio de mitologia”
(“Aos poetas clássicos”), a quem ele recomenda cantar a cidade que é sua, porque
ele teve “inducação,/ aprendeu munta ciença,/ mas das coisa do sertão/ não tem boa
esperiênça” (“Cante lá que eu canto cá”). Ao ensino livresco, ele opõe o ensino
prático: “Aqui Deus me ensinou tudo,/ sem de livro precisá” ou a experiência do
sertão (“O poeta da roça”, “Eu e o sertão”, “É coisa do meu sertão”, “Vida
sertaneja”, “Seu Dotô me conhece?”, “O vaqueiro”).

Assim como faz com o ensinamento moral, as tomadas de posição de Patativa do
Assaré são fundadas sobre a experiência: aquele que não conheceu o sertão na
carne, dele não pode falar. A única legitimidade admissível é a de pertencer a
seu povo (“Aos poetas clássicos”):

\begin{verse}
Na minha pobre linguage \\*
A minha lira servage \\
Canto que a minha arma sente \\
E o meu coração incerra, \\
As coisa de minha terra \\*
E a vida da minha gente 
\end{verse}

Ao dinheiro, opõe a felicidade, assim, em “Ser feliz” ele ressalta que a
felicidade “nasceu na simplicidade/ sem ouro, sem lar nem pão”. Opõe os bens
materiais à riqueza interior: “Dentro da minha pobreza,/ eu tinha grande riqueza”
(“A morte de Nana”), e fustiga aqueles que são escravos dos bens materiais em
detrimento do respeito aos valores humanos (“A escrava do dinheiro”). Com
efeito, o sertanejo confere uma importância maior à qualidade das relações
humanas (“Vida sertaneja”): 

\begin{verse}
O que mais estima e qué, \\*
é a paz, a honra e o brio, \\
o carinho de seus fio \\*
e a bondade da muié
\end{verse}

Este olhar sobre o mundo, numa perspectiva espacial, recupera também uma
oposição passado / presente; tradição / modernidade. A situação do sertanejo --
obrigado a abandonar sua terra em função da seca, a ir em direção às cidades do
litoral, ou então em direção às cidades do Sul --, é uma posição delicada, na
medida em que ele passa sem transição de um mundo rural à escala humana a um
mundo urbano onde impera o anonimato. O encontro desses dois universos é, não
raro, doloroso e acompanhado de uma volta aos valores tradicionais. As cidades,
o progresso e a técnica são acusados de veicular os piores males da civilização:
“Mas a civilização faz coisa/ que eu acho ruim” (“O puxadô de roda”). O Sul, em
particular, é tido como a sede da corrupção como em “Emigrante nordestino no Sul
do país”: 

\begin{verse}
Nos centros desconhecidos \\*
Depressa vê corrompidos \\
Os seus filhos inocentes, \\
Na populosa cidade \\
De tanta imoralidade \\*
E costumes diferentes
\end{verse}

Assim, o universo descrito por Patativa do Assaré é percebido como um espelho da
realidade. O aspecto quase documental de sua poesia foi salientado por um certo
número de críticos, entre os quais Luzanira Rego, segundo a qual sua obra

\begin{quote}
reflete em seus poemas todo o mundo visionário e fantasmagórico do caboclo
nordestino, pintando, em ácidas estrofes, a realidade de uma região onde o homem
e a terra se unem pela força do mesmo abandono.\footnote{ Luzanira Rego,
“Patativa do Assaré, poeta das injustiças e do sertão”.}
\end{quote}


\section{Conclusão}

O que faz a força e o sabor da poesia de Patativa do Assaré é, sem dúvida, esse
vínculo indestrutível entre o poeta, o sertão e o público. O canto só pode
nascer da repetição do cotidiano, com seu labor, suas alegrias e sofrimentos. O
canto só pode ser plenamente compreendido por aqueles que comungam desse
cotidiano e dessas mesmas experiências. A afeição com que é tratado pelos
habitantes do sertão que vêm visitá-lo e que pedem para recitar o seu poema
preferido; o sucesso que ele encontra durante suas excursões, notadamente junto
às comunidades sertanejas do Sul, e os cordéis escritos em sua homenagem são
prova irrefutável de que se tornou, por sua vez, um personagem-chave do panteão
nordestino. Patativa do Assaré é um poeta popular que, mesmo se no início
cantou o sertão de forma essencialmente nostálgica e lírica, tomou consciência
das possibilidades de mudança e do impacto que a sua voz podia ter. Embora
recebido pelos responsáveis políticos e honrado por sua obra, não cessa de lhes
recordar a realidade de onde ele extraiu a sua principal fonte de inspiração.
Uma de suas maiores preocupações é um futuro melhor para as gerações que virão.
Esse objetivo não pode ser alcançado sem passar por uma melhor educação, e
Patativa do Assaré vê no livro o seu auxiliar indispensável, como em “Ao meu
afilhado Cainã”: 

\begin{verse}
É por meio da leitura \\*
Que poderá a criatura \\
Na vida desenvolver, \\
O livro é companheiro \\
Mais fiel e verdadeiro \\*
Que nos ajuda a vencer 
\end{verse}

É notável que aquele que representa hoje a tradição oral da forma mais
monumental sonhe em continuar sua ação por meio da tradição escrita: sinal dos
tempos, evolução das tradições? Pesquisadores e universitários têm lamentado, há
alguns anos, o fim da literatura de cordel,\footnote{ Julie Cavignac, “Mémoires
en miroir”, p. 58.} avaliando que esse modo de transmissão de conhecimentos não
resistirá mais diante dos novos meios de comunicação. Talvez fosse preciso
formular de forma diferente o problema, diante do lugar ocupado por Patativa do
Assaré: herdeiro de uma forte tradição, logrou transformar seu papel e sua
mensagem. O que é, sem nenhuma dúvida, o objeto de uma evolução. É a função do
poeta popular, e não sua arte propriamente dita.

\begin{flushright}
\textit{Sylvie Debs}\\
Université Robert Schuman -- Estrasburgo
\end{flushright}


\bigskip

%\begin{bibliografia}[Bibliografia de Patativa do Assaré]
%
%\tit{}\textit{Inspiração nordestina, Cantos de Patativa}, 2ª ed.~ampliada,
%Rio de Janeiro, 1967.
%
%\tit{}\textit{Cante lá que eu canto cá, filosofia de um trovador 
%nordestino}. Petrópolis: Vozes, 1978.
%
%\tit{}\textit{Ispinho e fulô}. Fortaleza: Secretaria de Cultura, Turismo
%e Desporto, Imprensa Oficial do Ceará, 1988.
%
%\tit{}\textit{Aqui tem coisa}. Fortaleza: Multigraf, Secretaria 
%da Cultura e Desporto do Estado do Ceará, 1994. 
%
%\tit{}\textit{Cordéis Juazeiro do Norte}. \textsc{urca}, Universidade Regional
%do Cariri.
%
%\tit{}\textit{Balceiro}\ (Patativa e outros poetas de Assaré). Fortaleza: 
%Secretaria de Cultura e Desporto, 1991.
%
%\end{bibliografia}
%
%\bigskip

\pagebreak

\section{Bibliografia}

\subsection{Ensaios críticos}

\begin{bibliografia}[]

\tit{Filho}, Figueiredo. \textit{Patativa do Assaré, Novos poemas 
comentados}. 1970.

\tit{Nuvens}, Plácido Cidade. \textit{Patativa e o universo fascinante
do sertão}. Fortaleza: Fundação Edson Queiroz, 1995.

\end{bibliografia}

\subsection{Estudos críticos}

\begin{bibliografia}[]

\tit{Alencar}, José Arraes de. “Prefácio”, em \textit{Inspiração
Nordestina}, 2ª ed. Rio de Janeiro: 1967.

\tit{Alencar}, F. S. ``Patativa do Assaré, poeta compassivo'', 
em \textit{Cante lá que eu canto cá}. Rio de Janeiro: Vozes, 
1978.

\tit{Cariry}, Rosemberg. “A poesia popular está viva”, em 
\textit{Balceiro}. Fortaleza: 1991, pp.~9--11. 

\titidem. “Patativa do Assaré, um mestre da poesia popular”,
em \textit{Ispinho e Fulô}. Fortaleza: 1988, pp.~\textsc{i--vi}.

\tit{Cariry}, Rosemberg \& \textsc{barroso}, Oswald. “Patativa do 
Assaré, sua poesia, sua vida”, entrevista em \textit{Cultura 
insubmissa}. Fortaleza: Nação Cariri, 1982.

\tit{Cavalcante}, Raimundo. “Testemunho poético de um 
tempo”, em \textit{Aqui tem coisa}, 2ª ed. Fortaleza: \textsc{uece/
rvc}, 1995. 

\tit{Carvalho}, Gilmar de. “Prefácio”, em \textit{Aqui tem coisa}. 
Fortaleza: Multigraf, Secretaria da Cultura 
e Desporto do Estado do Ceará, 1994. 

\tit{Nuvens}, Plácido Cidade. “Patativa do Assaré, poeta 
social”, em \textit{Cante lá que eu canto cá}. Rio de Janeiro: 
Vozes, 1978.

\tit{Vieira}, Antônio. “Patativa do Assaré”, em \textit{Ispinho e
Fulô}. Fortaleza: 1988, pp.~\textsc{vii--xi}.

\end{bibliografia}

%\subsection{Artigos de imprensa}
%
%\begin{bibliografia}[]
%
%\tit{Cariry}, Rosemberg. “Patativa do Assaré, um lavrador, um 
%poeta do seu povo”, em \textit{O Povo}. Fortaleza: 20.out.1977.
%
%\tit{Carvalho}, Eleuda de. “Retrato 5x4 do poeta Patativa”, em
%\textit{O Povo}. Fortaleza: 25.mar.1995.
%
%\tit{Carvalho}, Gilmar de. “Quem é cego: Patativa ou nós?”, 
%em \textit{O Povo}. Fortaleza: 25.mar. 1995.
%
%\tit{Cerri}, Claudio. “Canto da terra”, \textit{Globo rural}.
%Rio de Janeiro: setembro de 1994.
%
%\tit{Medina}, Arlen. “Presidente condecora Patativa e 
%recebe pedido pela refinaria”, em \textit{O Povo}. Fortaleza:
%24.mar.1995.
%
%\tit{Neto}, Lira. “Gosto de ser Cabra-da-Peste”, em \textit{O Povo}: Fortaleza,
%25.mar.1995.
%
%\tit{Nuvens}, Plácido Cidade. “Patativa é poesia em estado 
%puro”, em \textit{O Povo}. Fortaleza: 25.mar.95.
%
%\tit{Rego}, Luzanira. “Patativa do Assaré, poeta das injustiças
%e do sertão”, em \textit{Diário de Pernambuco}. Recife: 3.out.1978.
%
%\tit{Vicelmo}, Antônio. “Patativa e o universo fascinante
%do Sertão”, em \textit{Diário do Nordeste}. Fortaleza: 22.out.1995.
%
%\end{bibliografia}
%
%\subsection{Filmes}
%
%\begin{bibliografia}[]
%
%\tit{Cariry}, Rosemberg. \textit{Patativa de Assaré, um poeta
%camponês}, documentário de curta-metragem. Fortaleza, Brasil: 1979.
%
%\titidem. \textit{Patativa do Assaré, um poeta do povo}, documentário
%de curta-metragem. Fortaleza, Brasil: 1984.
%
%\end{bibliografia}

\subsection{Bibliografia geral}

\begin{bibliografia}[]

\tit{Cantel}, Raymond. \textit{La littérature populaire brésilienne}. 
Poitiers: Centre de Recherches Latino-americaines, 1993.

\tit{Cascudo}, Luís da Câmara. \textit{Dicionário do folclore
brasileiro}. Rio de Janeiro: Ediouro, 1954. 

\tit{Cavignac}, Julie. “Figures et personnages de la culture 
nordestine dans la littérature de cordel au Brésil”, 
em \textit{Cadernos do mundo hispânico e luso-brasileiro}, n.~56. 
Baroja: 1991.

\titidem. “Mémoires en miroir”. \textit{Cahiers du Brésil
contemporain}, n.~9. 1990.

\tit{Grignon}, Claude \& \textsc{passeron}, Jean-Claude. “A propos 
des cultures populaires”. \textit{Cadernos do Cercon}, n.~9. 
Marselha: \textsc{cnrs}, Universidade de Nice, \textsc{ehess}, abr.~1985. 

\tit{Mortaigne}, Véronique. “Poètes-reporters et menteurs
professionnels”, em \textit{Le monde}, 30 dez.~1995.

\tit{Rimbaud}, Arthur. \textit{Uma temporada no inferno \&
Iluminações}, trad.~Lêdo Ivo, 3ª ed. Rio de Janeiro: Francisco Alves,
1981.

\tit{Roy}, Claude. \textit{Trésor de la poésie populaire}. Paris:
Seghers, s.d.

\end{bibliografia}


