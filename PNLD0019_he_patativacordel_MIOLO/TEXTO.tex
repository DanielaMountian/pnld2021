\SVN $Id: TEXTO.tex 7365 2010-09-02 20:54:51Z oliveira $

\chapter[História de Aladim e a lâmpada maravilhosa]{História de Aladim\break e a lâmpada maravilhosa}

\begin{verse}
Na cidade de Bagdá\\*
Quando ela antigamente\\*
Era a cidade mais rica\\*
Das terras do Oriente\\*
Deu-se um caso fabuloso\\*
Que apavorou muita gente

Nessa cidade morava\\*
Uma viúva de bem\\*
Paciente e muito pobre\\*
Não possuía um vintém\\*
Dentro da sua choupana\\*
Sem falar mal de ninguém

Vivia bem satisfeita\\*
Nessa pobreza sem fim\\*
Tendo só um filho único\\*
Com o nome de Aladim\\*
Que apesar de ser travesso\\*
Ninguém lhe achava ruim

Aquele belo garoto\\*
Tinha um leal coração\\*
Mas fugia do trabalho\\*
Buscando a vadiação\\*
Era a mãe que trabalhava\\*
Para fornecer-lhe o pão

Aladim não trabalhava\\*
Seu emprego era brincar\\*
E a sua mãe empregada\\*
Em uma roca a fiar\\*
Atrás de ganhar o pão\\*
Para o filho sustentar

Aladim um certo dia\\*
Pensando na sua vida\\*
Achou que estava fazendo\\*
Uma existência perdida\\*
De causar muito desgosto\\*
À sua mamãe querida

Chegando-se a ela disse\\*
Um tanto contrariado:\\*
Mamãe, perdoe os desgostos\\*
Que eu já tenho lhe causado\\*
Garanto que hoje em diante\\*
Hei de viver empregado

Saiu em busca da praça\\*
Atrás de colocação\\*
Agradou muito a viúva\\*
Aquela resolução\\*
Deu-lhe naquele momento\\*
A sua santa bênção

Da África tinha chegado\\*
Por aquele mesmo ano\\*
Um velho misterioso\\*
De aspecto desumano\\*
A quem o povo chamava:\\*
“O feiticeiro africano”

Era um grande necromante\\*
Que de tudo conhecia\\*
Com o segredo da arte\\*
De sua feitiçaria\\*
Viu que perto de Bagdá\\*
Um grande tesouro havia

Mediante a vara mágica\\*
De força prodigiosa\\*
Soube ele que ali havia\\*
Uma gruta misteriosa\\*
Onde se achava escondida\\*
Uma lâmpada maravilhosa

Esta lâmpada tinha um gênio \\*
Que obedecia a ela\\*
Aparecia vexado\\*
Quando se apertava nela\\*
Pronto para obedecer\\*
A quem fosse dono dela

Porém a lâmpada da gruta\\*
Não havia quem tirasse\\*
Só se fosse uma pessoa\\*
Que o segredo ignorasse\\*
O velho andava à procura\\*
Dum homem que o ajudasse

O feiticeiro africano\\*
Com Aladim se encontrou\\*
Dos modos do rapazinho\\*
Ele muito se agradou\\*
Aproximando-se dele\\*
Desta forma lhe falou:

Se quiser eu lhe protejo\\*
Faça o favor de me ouvir\\*
Lhe farei feliz e rico\\*
Se com gosto me servir\\*
Não sofrerá mais pobreza\\*
Durante enquanto existir

Aladim lhe respondeu:\\*
Disponha do seu criado\\*
Estarei às suas ordens\\*
Pra fazer qualquer mandado\\*
Contanto que o senhor\\*
Me deixe recompensado

Inda disse o africano:\\*
Tenha confiança em mim;\\*
Foi depressa em uma loja\\*
Trouxe uma peça de brim\\*
E de especial fazenda\\*
Trajou-se o moço Aladim

Se afastaram da cidade\\*
Cada qual mais satisfeito\\*
Até que ambos chegaram\\*
Num vale bastante estreito\\*
Foi nesse dito lugar\\*
Onde parou o sujeito

O vale era situado\\*
Entre dois morros sem fim\\*
Ali o velho parou\\*
E olhando pra Aladim\\*
Com frases autoritárias\\*
Lhe foi ordenando assim:

Junta os raminhos secos\\*
Que naquela árvore tem\\*
E ponha fogo nos mesmos\\*
Não temas, escuta bem\\*
Irás agora ver coisa\\*
Jamais vista por ninguém

Aladim queimando os ramos\\*
O velho se aproximou\\*
Sua vara de condão\\*
Pelo ar ele vibrou\\*
E ali certos perfumes\\*
Sobre o fogo derramou

O moço ficou imóvel\\*
Prestando bem atenção\\*
Quando naquele momento\\*
Ouviu-se um grande trovão\\*
Começando a terra abrir-se\\*
Na mesma ocasião

Aladim morto de medo\\*
Olhava assombradamente\\*
O pobre que até ali\\*
Ainda estava inocente\\*
Notou que dum feiticeiro\\*
Achava-se dependente

Ali tratou de correr\\*
Deixando o seu camarada\\*
Mas de surpresa o velho\\*
Deu-lhe uma grande mancada\\*
Que o mesmo caiu por terra\\*
Sem dar sentido de nada

O feiticeiro gritou-lhe:\\*
Miserável criatura!\\*
Tu não me juraste fé?\\*
Não quebres a tua jura\\*
Tira aquela grande pedra\\*
Ali daquela abertura

Na pedra havia ligado\\*
Um anel de ferro grosso\\*
No qual pegou com as mãos\\*
O obediente moço\\*
Que com o menor impulso\\*
Moveu o grande colosso

Quando ele tombou a pedra\\*
De onde ela estava encerrada\\*
Viu-se uma cova medonha\\*
Aonde tinha uma escada\\*
Aladim daquilo tudo\\*
Não compreendia nada

O feiticeiro ordenou-lhe:\\*
Agora tens de descer\\*
Já que me juraste fé\\*
Tu tens que me obedecer:\\*
E pegou a ensinar-lhe\\*
Como devia ser

Primeiro tu passarás\\*
Por três salas asseadas\\*
Que de preciosidades\\*
Acham-se todas ornadas\\*
E por um belo clarão\\*
Todas três iluminadas

Passe por todas depressa\\*
Sejas correto, Aladim\\*
E bem na terceira sala\\*
Quando chegares no fim\\*
Verás uma porta ao lado\\*
Que entra para um jardim

No centro deste jardim\\*
Encontrarás com certeza\\*
Um belo pórtico de mármore\\*
Feito com muita beleza\\*
Em um nicho desse pórtico\\*
Tem uma lâmpada acesa

Segura nela de leve\\*
Depois de ter apagado\\*
Derrama dela o azeite\\*
E voltarás bem apressado\\*
Fazendo como lhe digo\\*
Serás bem recompensado

Para maior segurança\\*
Leva este anel contigo\\*
Que é de grande vantagem\\*
Pois com verdade lhe digo\\*
Ele tem o poder mágico\\*
De afugentar o perigo

O moço no mesmo instante\\*
Botou o anel no dedo\\*
E desceu pela escada\\*
Nada lhe causava medo\\*
Curioso por saber\\*
Aquele grande segredo

Passou por tudo depressa\\*
Como lhe fora ordenado\\*
Quando ele entrou no jardim\\*
Ficou muito admirado\\*
Olhando a grande beleza\\*
Daquele sítio encantado

A passarada cantava\\*
Tudo ali mostrava um riso,\\*
Aladim ficou pasmado\\*
Pensando que de improviso\\*
Por um sonho havia sido\\*
Transportado ao Paraíso

Encontrou naquele sítio\\*
Toda sorte de fruteiras\\*
O vento soprava brando\\*
Entre as esbeltas mangueiras\\*
Sussurrando mansamente\\*
Nos galhos das laranjeiras

Bem no centro do jardim\\*
O tal nicho ele encontrou\\*
Do qual tirando a lâmpada\\*
Com cuidado a apagou\\*
E derramando o azeite,\\*
Na algibeira guardou

E saiu pelo jardim\\*
Andando mui distraído\\*
Do feiticeiro africano\\*
Estava quase esquecido\\*
Ligando pouca importância\\*
Do que tinha prometido

Das fruteiras do jardim\\*
Quando ele um fruto tirava\\*
De repente o mesmo fruto\\*
Em vidro se transformava\\*
Mas ele não se importando\\*
Na algibeira os guardava

Quando encheu as algibeiras\\*
De frutos que ali achou\\*
E de mais preciosidades\\*
Que no jardim encontrou\\*
Para a porta de entrada\\*
Alegremente voltou

O velho sem paciência\\*
Olhou da boca da fenda\\*
Chamando por Aladim\\*
Gritando com voz horrenda:\\*
Anda depressa, maldito\\*
Dá-me logo essa encomenda!

Aladim não respondeu\\*
Ouviu e ficou calado\\*
Do feiticeiro africano\\*
Queria ficar vingado\\*
Devido à grande pancada\\*
Que este havia lhe dado

Tornou o velho a gritar:\\*
Dá-me a lâmpada, Aladim\\*
O moço lhe respondeu:\\*
Mestre, não se vexe assim\\*
Descei para examinardes\\*
As belezas do jardim

O feiticeiro africano\\*
Vendo que estava logrado\\*
Fechou a boca da gruta\\*
Raivoso desesperado\\*
Deixando o pobre Aladim\\*
Na caverna sepultado

Proferindo maldições\\*
O velho se retirou\\*
E naquele mesmo dia\\*
Para África embarcou.\\*
Vamos tratar de Aladim\\*
Que na caverna ficou

O pobre Aladim ficou\\*
Em triste situação\\*
A cova que até ali\\*
Se achava em belo clarão\\*
Tornou-se um ermo esquisito\\*
De medonha escuridão

Três dias ele passou\\*
Errando na cova escura\\*
Sem achar naquele ermo\\*
Uma pequena abertura\\*
Sofrendo com paciência\\*
Sua cruel desventura

No fim do terceiro dia\\*
Quase não podendo andar\\*
Veio-lhe naquela hora\\*
A lembrança de rezar\\*
Ali mesmo ajoelhou-se\\*
E começou a orar

No momento em que ele\\*
Fazia a sua oração\\*
Casualmente moveu-se\\*
Na presente ocasião\\*
O anel que ele trazia\\*
No dedo mínimo da mão

Em mover o dito anel\\*
Que tinha no dedo seu\\*
De uma bela claridade\\*
Aquela gruta se encheu\\*
E uma visão horrenda\\*
Junto dele apareceu

Essa visão era um gênio\\*
Que o anel obedecia\\*
Se movendo ele no dedo\\*
O gênio aparecia\\*
E tudo que se mandasse\\*
Ele depressa fazia

A visão chegou e disse:\\*
Menino, não tenhas medo\\*
Que eu sou o gênio escravo\\*
Do anel que tens no dedo\\*
O que mandar-me que eu faça\\*
Eu te farei muito cedo

Aladim aproveitando\\*
Aquela oportunidade\\*
Disse ao gênio: sendo assim\\*
Leva-me para a cidade\\*
À casa de minha mãe\\*
É esta a minha vontade

Apenas o moço tinha\\*
Tais frases pronunciado\\*
Pra cidade de Bagdá\\*
Foi depressa transportado\\*
Em casa de sua mãe\\*
Chegou bastante cansado

Por causa dos sofrimentos\\*
Que na caverna passava\\*
Chegou mais morto que vivo\\*
Porque com fome se achava\\*
Pois já fazia três dias\\*
Que ele não se alimentava

Sua mãe bastante aflita\\*
Vendo o seu sofrimento\\*
Foi buscar a toda pressa\\*
Um pouquinho de alimento\\*
Ele saciando a fome\\*
Cobrou de novo o alento

Depois que o moço Aladim\\*
Tinha comido e bebido\\*
Adquirindo de novo\\*
O seu esforço perdido\\*
Contou logo a sua mãe\\*
O que tinha acontecido

A sua mãe no princípio\\*
Quase nada acreditou\\*
Mas quando os frutos de vidro\\*
E a lâmpada ele mostrou\\*
Da sua história toda\\*
Ela nada duvidou

Aladim disse: eu agora\\*
Vou estas coisas vender\\*
Para ver se algum dinheiro\\*
Com elas posso obter\\*
E assim mais alguns dias\\*
Teremos de que viver

Disse a velha: para isso\\*
Não acharás comprador\\*
Somente esta lâmpada velha\\*
Pode ter algum valor\\*
Vou limpá-la para dar-lhe\\*
Um preço superior

Assim que a viúva estava\\*
Esse trabalho fazendo\\*
Apareceu de repente\\*
Um gênio alto e horrendo\\*
Que com um gesto espantoso\\*
Foi desta forma dizendo:

Eu sou o gênio da lâmpada\\*
Pois vivo sujeito a ela\\*
Disposto a obedecer\\*
A quantos pegarem nela\\*
O que quiseres que eu faça\\*
Farei com toda cautela

A viúva ouvindo isto\\*
Deu-lhe logo um passamento\\*
Aladim pegou na lâmpada\\*
Aproveitando o momento\\*
Para exigir do gênio\\*
O que tinha no pensamento

Disse Aladim para o gênio:\\*
Se tu vens me obedecer\\*
Se és escravo da lâmpada\\*
Mostra-me o teu poder\\*
Trazendo-me alguma coisa\\*
Com que possa me manter

Quando o moço declarou\\*
Qual era o desejo seu\\*
O gênio escutando tudo\\*
Uma palavra não deu\\*
Numa nuvem de fumaça\\*
Dali desapareceu

Desapareceu o gênio\\*
Depois voltou pressuroso\\*
Com uma porção de pratos\\*
De metal mui precioso\\*
Que vinham todos compostos\\*
De alimento saboroso

Trazia mais outros vasos\\*
Cheios de vinho excelente\\*
Antes porém que Aladim\\*
Lhe visse mais claramente\\*
Pousou tudo sobre a mesa\\*
E fugiu rapidamente

O moço chamou a velha\\*
Com gesto amoroso e ledo\\*
Dizendo: mamãe, acorda\\*
Nada aqui lhe causa medo\\*
Que foi descoberto agora\\*
Da lâmpada o grande segredo

Quando a velha despertou\\*
Sentiu medonha surpresa\\*
Aladim lhe contou tudo\\*
E para maior surpresa\\*
Ela viu em sua sala\\*
A comida sobre a mesa

Dali em diante Aladim\\*
Quando uma coisa queria\\*
Apertava a sua lâmpada\\*
E o gênio aparecia\\*
O que o moço ordenasse\\*
Ele depressa fazia

Assim viveu alguns anos\\*
Em completa liberdade\\*
Gozando com sua mãe\\*
Perfeita felicidade\\*
Procurando fazer parte\\*
Da alta sociedade

Passando um dia em frente\\*
Do palácio do sultão\\*
Viu uma linda princesa\\*
De uma rara perfeição\\*
Aladim sentiu por ela\\*
A mais ardente paixão

Tinha o nome de Clarice\\*
Esta linda criatura\\*
Olhos meigos atraentes\\*
Lábios cor-de-rosa pura\\*
Era o retrato de Vênus\\*
A deusa da formosura

Quando ele chegou em casa\\*
Um pouco impressionado\\*
Contou logo a sua mãe\\*
O que havia passado\\*
Que pela bela Clarice\\*
Achava-se apaixonado

A mãe pensou que seu filho\\*
Tinha perdido o juízo\\*
Aconselhou-o ternamente\\*
Com o mais doce sorriso\\*
Temendo aquela impressão\\*
Causar algum prejuízo

Ele então lhe respondeu:\\*
Eu não estou a brincar\\*
Quero que vá ao sultão\\*
A fim de solicitar\\*
A sua querida filha\\*
Para comigo casar

Leve os frutos que colhi\\*
Lá no jardim encantado\\*
Pois eles não são de vidro\\*
Como tínhamos pensado\\*
E sim, pedras preciosas\\*
Dum valor mais sublimado

Entrega ao nobre sultão\\*
Este sublime presente\\*
Ele vendo este tesouro\\*
Ficará muito contente\\*
E assim o que desejo\\*
Obterei facilmente

Dentro de um vaso de ouro\\*
O qual chamava atenção\\*
A velha botou as pedras\\*
Com a maior precaução\\*
E logo seguiu em busca\\*
Do palácio do sultão

Chegou ela no palácio\\*
Decentemente trajada\\*
Porém só no fim do dia\\*
Lhe concederam entrada\\*
No quartel do soberano\\*
Entrou bastante acanhada

Foi até o pé do trono\\*
Chegando ali se prostrou\\*
E levantou-se somente\\*
Quando o sultão lhe ordenou\\*
E com palavras severas\\*
Depressa lhe perguntou:

Aqui neste meu palácio\\*
O que vieste buscar?\\*
A matrona respondeu-lhe:\\*
Venho vos solicitar\\*
A vossa filha Clarice\\*
Pra com meu filho casar

O sultão ouvindo aquilo\\*
Riu-se indiferentemente\\*
Mas quando ela entregou-lhe\\*
O magnífico presente\\*
Quase que morre de espanto\\*
Não coube em si de contente

Disse muito admirado:\\*
Agora sei com certeza\\*
Que teu filho é um senhor\\*
Duma raríssima grandeza\\*
Que possa dispor assim\\*
De tão sublime riqueza

Volta para tua casa\\*
Vai a ele prevenir\\*
Que com gosto o casamento\\*
Eu poderei consentir\\*
Se mandar-me com urgência\\*
O que eu vou exigir

Se por vinte escravos pretos\\*
E vinte brancas de cor\\*
Me mandar quarenta vasos\\*
Dum esmerado primor\\*
Todos cheios de ouro e pedras\\*
Do mais sublime valor

Assim que o moço Aladim\\*
Recebeu aquele recado\\*
Agradeceu a viúva\\*
Falando com muito agrado\\*
Pelo serviço de amor\\*
Que ela havia lhe prestado

Dizia ele contente:\\*
Muito obrigado, senhora\\*
Com a princesa Clarice\\*
Eu hei de casar-me agora\\*
O que o sultão deseja\\*
Arranjarei sem demora

Às doze horas da noite\\*
Levou a lâmpada na mão\\*
Pra uma extensa praça\\*
De prolongada amplidão\\*
Que ficava bem na frente \\*
Do palácio do sultão

Chegando ali apertou\\*
A lâmpada maravilhosa\\*
Quando chegou de repente\\*
A visão misteriosa\\*
Perguntando o que queria\\*
Com sua voz estrondosa

O moço Aladim fitando\\*
A poderosa visão\\*
A qual nunca lhe faltava\\*
Com a sua proteção\\*
Foi logo lhe declarando\\*
Os desejos do sultão

Quero que com muita pressa\\*
Desempenhe este mandado\\*
Quero também nesta praça\\*
Um palácio edificado\\*
Como nenhum sobre a terra\\*
Já tenha sido encontrado

Apenas o moço tinha\\*
Tais frases pronunciado\\*
Viu diante dos seus olhos\\*
Um palácio edificado\\*
Mais belo e bonito ainda\\*
Do que tinha imaginado

Ele entrou naquele prédio\\*
Com grande admiração\\*
Viu os quarenta escravos\\*
Que estavam de prontidão\\*
Com as preciosidades\\*
Exigidas do sultão

O sultão no outro dia\\*
Contemplava da janela\\*
Aquele grande edifício\\*
De arquitetura tão bela\\*
Dizendo nunca ter visto\\*
Obra semelhante àquela

Assustou-se quando soube\\*
Que o grande personagem\\*
Dono daquele palácio\\*
De admirável vantagem\\*
Estava se preparando\\*
Pra lhe render homenagem

Mandou logo convidar\\*
Os ministros da cidade\\*
E formou no seu palácio\\*
Grandíssima sociedade\\*
Pra receber Aladim\\*
Que vinha com brevidade

Acompanhado da velha\\*
Mais tarde ele apareceu\\*
Com o maior regozijo\\*
O sultão os recebeu\\*
De palmas, bravos e vivas\\*
Todo o palácio se encheu

Aladim assim que foi\\*
No palácio introduzido\\*
Mandou logo os seus criados\\*
Que ali os tinha trazido\\*
Entregarem os presentes\\*
Que o sultão tinha exigido

Na tarde do mesmo dia\\*
Celebrou-se o casamento\\*
No palácio do sultão\\*
Um soberbo ajuntamento\\*
Felicitava os noivos\\*
Cheio de contentamento

Casou-se o moço Aladim\\*
Com a princesa Clarice\\*
Gozando do casto amor\\*
Daquela flor de meiguice\\*
Sem pensar que a desventura\\*
Contra ele inda surgisse

Mas enquanto ele gozava\\*
A união conjugal\\*
O feiticeiro africano\\*
Em sua terra natal\\*
Tramava grandes maldades\\*
A fim de fazer o mal

Este velho um certo dia\\*
Mandou seus gênios no ar\\*
Pra da morte de Aladim\\*
Melhor se certificar\\*
Saber se a lâmpada mágica\\*
Inda estava em seu lugar

Os gênios quando voltaram\\*
Disseram ao velho então\\*
Que Aladim em Bagdá\\*
Era o genro do sultão\\*
E ocupava na cidade\\*
Uma alta posição

Disseram mais que Aladim\\*
Belo palácio habitava\\*
Ao lado de sua esposa\\*
A quem ternamente amava\\*
E a lâmpada maravilhosa\\*
Em seu poder se encontrava

O velho sabendo disso\\*
Muito raivoso ficou\\*
Em direção de Bagdá\\*
Com raiva se encaminhou\\*
Em traje dum ambulante\\*
Naquela cidade entrou

Certo dia em que o sultão\\*
E Aladim foram à caça\\*
O feiticeiro chegou\\*
Bem disfarçado na praça\\*
Com cesto de lâmpadas novas\\*
Pra fazer sua trapaça

O maldito macumbeiro\\*
Soube seu plano formar\\*
Bem na frente do palácio\\*
Começou ele a gritar:\\*
Quem é que tem lâmpada velha\\*
Para por nova trocar?

A princesa que se achava\\*
Na janela debruçada\\*
Lembrou-se de uma lâmpada\\*
Já velha inutilizada\\*
No quarto de seu marido\\*
Onde se achava guardada

Correu às pressas buscá-la\\*
E disse a uma criada:\\*
Leve-a ao pobre tolo\\*
Pra ser por outra trocada\\*
Que parece gostar muito\\*
De lâmpada inutilizada

A princesa não sabia\\*
Que lâmpada era aquela\\*
E não podia estimar\\*
O valor que tinha ela\\*
Porque seu marido nunca\\*
Tinha lhe falado nela

A negra levou a lâmpada\\*
Ao feiticeiro entregou\\*
Ele foi logo apertando-a\\*
Ali o gênio chegou\\*
Dizendo com voz tremenda:\\*
Às tuas ordens estou

O malfazejo africano\\*
Foi logo dizendo assim:\\*
Quero ver este palácio\\*
Com a mulher de Aladim\\*
E todos os seus escravos\\*
Na África, no meu jardim

Ouviu-se um forte trovão\\*
Que a cidade estremeceu\\*
No mesmo instante o palácio\\*
Do seu lugar se moveu\\*
E girando pelo espaço\\*
Logo desapareceu

Quando o sultão e o genro\\*
Ambos voltaram da caça\\*
Acharam o povo todo\\*
Chorando a grande desgraça\\*
Que há poucas horas havia\\*
Se dado naquela praça

Disse o sultão a seu genro:\\*
Veja se pode arranjar\\*
Com que o palácio volte\\*
Com tudo a este lugar\\*
Ou dentro em pouquinhos dias\\*
Eu te mando degolar

E botou o pobrezinho\\*
Em uma triste enxovia\\*
Onde não tinha licença\\*
De ver o clarão do dia\\*
Mas o verdadeiro herói\\*
Com paciência sofria

O infeliz Aladim\\*
Quando se achou na prisão\\*
Veio-lhe grande desejo\\*
De fazer uma oração\\*
Pedindo a seu Deus Alá\\*
Sua grande proteção

Quando ele estava rezando\\*
Suplicando humildemente\\*
O anel que tinha no dedo\\*
Moveu repentinamente\\*
Ali a prisão se encheu\\*
De um clarão resplandecente

Essa grande claridade\\*
Deu-lhe um prazer sem fim\\*
Depois veio uma visão\\*
Que foi lhe dizendo assim:\\*
Sou o escravo do anel\\*
O que desejas de mim?

Disse Aladim: se vieste\\*
Me livrar da cruel morte\\*
Nesta tal situação\\*
De ti exijo um transporte\\*
No lugar onde estiver\\*
A minha boa consorte

O gênio agarrando ele\\*
Rapidamente o levou\\*
No salão do seu palácio\\*
Na África ele se achou\\*
A sua querida esposa\\*
Chorosamente encontrou

Disse ela: meu bom esposo\\*
Meu sofrimento é sem par\\*
O feiticeiro africano\\*
Quer comigo se casar\\*
Oh! Se eu não consentir\\*
Ele manda me matar!

Disse Aladim: se console\\*
Vou aventurar a sorte\\*
Pois trago comigo um vaso\\*
Contendo um veneno forte\\*
Que o ente que bebê-lo\\*
Não escapará da morte

Encha com este veneno\\*
O copo do feiticeiro\\*
Que o velho há de bebê-lo\\*
Por seu excelente cheiro\\*
E assim veremos logo\\*
O seu dia derradeiro

Por trás de uma cortina\\*
Oculto Aladim ficou\\*
Depois dum quarto de hora\\*
O feiticeiro chegou\\*
Era a hora do jantar\\*
Da mesa se aproximou

Durante aquele jantar\\*
A mulher ofereceu\\*
O copo ao velho africano\\*
Ele com gosto bebeu\\*
Depois caiu no sofá\\*
No mesmo instante morreu

O velho caindo morto\\*
Aladim correu ligeiro\\*
Tirou-lhe do bolso a lâmpada\\*
E disse mui prazenteiro:\\*
Hoje o feitiço virou\\*
Por cima do feiticeiro

Então apertando a lâmpada\\*
Viu logo o gênio surgir\\*
E transportar o palácio\\*
Sem ele abalo sentir\\*
Em sua terra natal\\*
Foi de novo residir

Dali em diante Aladim\\*
Bem descansado viveu\\*
Depois de dois ou três anos\\*
O velho sultão morreu\\*
Foi ele o único herdeiro\\*
De tudo que era seu

Ele ficou sendo dono\\*
Da riqueza fabulosa\\*
Gozando da companhia\\*
Da sua esposa bondosa\\*
Sem precisar do auxílio\\*
Da lâmpada maravilhosa

Aquele velho africano\\*
Na sua feitiçaria\\*
Trabalhava com cuidado\\*
Outro não lhe competia\\*
Nada valeu no seu mal\\*
Indo habitar afinal\\*
O gelo da terra fria

Porque a negra ambição\\*
Atrasa sem compaixão\\*
Traz consigo a maldição\\*
Amamentando a desgraça\\*
Tira o dinheiro do nobre\\*
Ilude a gana do pobre\\*
Vê do indigente o cobre\\*
A ambição onde passa
\end{verse}

\chapter[O padre Henrique e o dragão da maldade]{O padre Henrique\break e o dragão da maldade}

\begin{verse}
Sou um poeta do mato\\*
Vivo afastado dos meios\\*
Minha rude lira canta\\*
Casos bonitos e feios\\*
Eu canto meus sentimentos\\*
E os sentimentos alheios

Sou caboclo nordestino\\*
Tenho mão calosa e grossa,\\*
A minha vida tem sido\\*
Da choupana para roça,\\*
Sou amigo da família\\*
Da mais humilde palhoça

Canto da mata frondosa\\*
A sua imensa beleza,\\*
Onde vemos os sinais\\*
Do pincel da natureza,\\*
E quando é preciso eu canto\\*
A mágoa, a dor e a tristeza

Canto a noite de são João\\*
Com toda sua alegria,\\*
Sua latada de folha\\*
Repleta de fantasia\\*
E canto o pobre que chora\\*
Pelo pão de cada dia

Canto o crepúsculo da tarde\\*
E o clarão da linda aurora,\\*
Canto aquilo que me alegra\\*
E aquilo que me apavora\\*
E canto os injustiçados\\*
Que vagam no mundo afora

E, por falar de injustiça\\*
Traidora da boa sorte\\*
Eu conto ao leitor um fato\\*
De uma bárbara morte\\*
Que seu deu em Pernambuco\\*
Famoso Leão do Norte

Primeiro peço a Jesus\\*
Uma santa inspiração\\*
Para escrever estes versos\\*
Sem me afastar da razão\\*
Contando uma triste cena\\*
Que faz cortar coração

Falar contra as injustiças\\*
Foi sempre um dever sagrado\\*
Este exemplo precioso\\*
Cristo deixou registrado\\*
Por ser reto e justiceiro\\*
Foi no madeiro cravado

Por defender os humildes\\*
Sofreu as mais cruéis dores\\*
E ainda hoje nós vemos\\*
Muitos dos seus seguidores\\*
Morrerem barbaramente\\*
Pelas mãos dos malfeitores

Vou contar neste folheto\\*
Com amor e piedade\\*
Cujo título encerra\\*
A mais penosa verdade:\\*
O padre Antonio Henrique\\*
E o dragão da maldade

O padre Antonio Henrique\\*
Muito jovem e inteligente\\*
A 27 de maio\\*
Foi morto barbaramente,\\*
No ano 69\\*
Da nossa era presente

Padre Henrique tinha apenas\\*
29 anos de idade,\\*
Dedicou sua vida aos jovens\\*
Pregando a santa verdade\\*
Admirava a quem visse\\*
A sua fraternidade

Tinha três anos de padre\\*
Depois que ele se ordenou\\*
Pregava a mesma missão\\*
Que Jesus Cristo pregou\\*
E foi por esse motivo\\*
Que o dragão lhe assassinou

Surgiu contra padre Henrique\\*
Uma fúria desmedida\\*
Ameaçando a Igreja\\*
Porque estava decidida\\*
Conscientizando os jovens\\*
Sobre os problemas da vida

Naquele tempo o Recife\\*
Grande bonita cidade\\*
Se achava contaminada\\*
Pelo dragão da maldade,\\*
A rancorosa mentira\\*
Lutando contra a verdade

Nesse clima de tristeza\\*
Os dias iam passando\\*
Porém nosso padre Henrique\\*
Sempre a verdade explicando\\*
E ameaças contra a Igreja\\*
Chegavam de vez em quando

Por causa do seu trabalho\\*
Que só o que é bom almeja\\*
O espírito da maldade\\*
Que tudo estraga e fareja\\*
Fez tristes acusações\\*
Contra d.~Hélder e a Igreja

Com o fim de atemorizar\\*
O apóstolo do bem\\*
Chegavam cartas anônimas\\*
Com insulto e com desdém,\\*
Porém quem confia em Deus\\*
Jamais temeu a ninguém

Anônimos telefonemas\\*
Com assuntos de terror\\*
Chegavam constantemente\\*
Cheios de ódio e rancor\\*
Contra padre Henrique, o amigo\\*
Da paz, da fé e do amor

Os ditos telefonemas\\*
Faziam declaração\\*
De matar trinta pessoas\\*
Sem ter dó nem compaixão\\*
Que tivessem com d.~Hélder\\*
Amizade ou ligação

Veja bem leitor amigo\\*
Quanto é triste esta verdade\\*
O que defende os humildes\\*
Mostrando a luz da verdade\\*
Vai depressa perseguido\\*
Pelo dragão da maldade

Mas o ministro de Deus\\*
Possui o santo dever\\*
De estar ao lado dos fracos,\\*
Suas causas defender\\*
Não é só salvar a alma\\*
Também precisa comer

Os poderosos não devem\\*
Oprimir de mais a mais,\\*
A justiça é para todos\\*
Vamos lutar pela paz\\*
Ante os direitos humanos\\*
Todos nós somos iguais

A Igreja de Jesus\\*
Nos oferece orações\\*
Mas também precisa dar\\*
Aos humildes instruções \\*
Para que possam fazer\\*
Suas reivindicações

Veja meu caro leitor,\\*
A maldade o quanto é:\\*
O padre Henrique ensinava\\*
Cheio de esperança e fé,\\*
Aquelas mesmas verdades\\*
De Jesus de Nazaré

E foi por esse motivo\\*
Que surgiu a reação,\\*
Foi o instinto infernal\\*
Com a fúria do dragão,\\*
Que matou o nosso guia\\*
De maior estimação

A 27 de maio,\\*
O santo mês de Maria\\*
No ano 69\\*
A natureza gemia\\*
Por ver o corpo de um padre\\*
Morto sobre a terra fria

Naquele dia de luto\\*
Tudo se achava mudado,\\*
Parece até que o Recife\\*
Se mostrava envergonhado\\*
Por ver que um triste segredo\\*
Estava a ser revelado

Rádio, \textsc{tv} e jornais,\\*
Nada ali noticiaram\\*
Porque as autoridades\\*
Estas verdades calaram\\*
E o padre Antonio Henrique\\*
Morto no mato encontraram

Estava o corpo do padre\\*
De faca e bala furado,\\*
Também mostrava ter sido\\*
Pelo pescoço amarrado\\*
Provando que antes da morte\\*
Foi bastante judiado

No mato estava seu corpo\\*
Em situação precária:\\*
Na região do lugar\\*
Cidade Universitária\\*
Foi morto barbaramente\\*
Pela fera sanguinária

Por aquele mesmo tempo\\*
Muitos atos agravantes\\*
Surgiram lá no Recife\\*
Contra os jovens estudantes\\*
Que devem ser no futuro\\*
Da pátria representantes

Invadiram o Diretório\\*
Estudantil, um recinto\\*
Universidade Católica\\*
De Pernambuco e, não minto,\\*
Foi atingido por bala\\*
O estudante Cândido Pinto

Foi sequestrado e foi preso\\*
O estudante Cajá\\*
O encerramento no cárcere\\*
Passou um ano por lá\\*
Meu Deus! A democracia\\*
Deste país onde está?

Cajá o dito estudante\\*
Pessoa boa e benquista,\\*
Pra viver com os pequenos\\*
Deixou de ser carreirista\\*
E por isto o mesmo foi\\*
Taxado de comunista

Será que ser comunista\\*
É dar ao fraco instrução,\\*
Defendendo os seus direitos\\*
Dentro da justa razão,\\*
Tirando a pobreza ingênua\\*
Das trevas da opressão?

Será que ser comunista\\*
É mostrar certeiros planos\\*
Para que o povo não viva\\*
Envolvido nos enganos\\*
E possa se defender\\*
Do jogo dos desumanos?

Será que ser comunista\\*
É saber sentir as dores\\*
Da classe dos operários,\\*
Também dos agricultores\\*
Procurando amenizar\\*
Horrores e mais horrores?

Tudo isto, leitor, é truque\\*
De gente sem coração\\*
Que, com o fim de trazer\\*
Os pobres na sujeição,\\*
Da palavra comunismo\\*
Inventa um bicho-papão

Porém a Igreja dos pobres\\*
Fiel se comprometeu,\\*
Cada um tem o direito\\*
De defender o que é seu,\\*
Para quem segue Jesus\\*
Nunca falta um Cirineu

Mostrando a mesma verdade\\*
De Jesus na Palestina\\*
O movimento se estende\\*
Contra a opressão que domina\\*
Sobre os nossos irmãos pobres\\*
De toda América Latina

Quando Jesus Cristo andou\\*
Pregando sua missão\\*
Falou sobre a igualdade,\\*
Fraternidade e união,\\*
Não pode haver injustiças\\*
Na sua religião

Por este motivo a Igreja\\*
Nova posição tomou\\*
Dentro da América Latina\\*
A coisa agora mudou,\\*
O bom cristão sempre faz\\*
Aquilo que Deus mandou

É justo por excelência\\*
O Autor da Criação,\\*
Devemos amar a Deus\\*
Por direito e gratidão,\\*
Cada um tem o dever\\*
De defender seu irmão

Por isto, os nossos pastores\\*
Trilham penosas estradas\\*
Observando de Cristo\\*
Suas palavras sagradas,\\*
Trabalhando em benefício\\*
Das classes desamparadas

Declarando dessa forma\\*
A santa luz da verdade\\*
Para que haja entre todos\\*
Amor e fraternidade\\*
E boa organização\\*
Dentro da sociedade

Pois vemos o estudante\\*
Pelo poder perseguido,\\*
Operário, agricultor,\\*
O nosso índio querido\\*
E o negro? Pobre coitado!\\*
É o mais desprotegido

Vendo a medonha opressão\\*
Que vem do instinto profano,\\*
Me vem à mente o que disse\\*
O grande bardo baiano\\*
O Poeta dos Escravos\\*
Apelando ao Soberano

Senhor Deus dos desgraçados\\*
Dizei-me vós, Senhor Deus,\\*
Se é mentira, se é verdade\\*
Tanto horror perante os céus

Meu caro leitor desculpe\\*
Esta falta que cometo,\\*
Me desviando do assunto\\*
Da história que lhe remeto,\\*
O caso do padre Henrique,\\*
Motivo deste folheto

Se me desviei do ritmo,\\*
Não queira se aborrecer,\\*
É porque as outras coisas\\*
Eu queria lhe dizer,\\*
Pois tudo que ficou dito,\\*
Você precisa saber

\pagebreak

Mas, agora lhe prometo\\*
Com bastante exatidão,\\*
Terminar para o amigo\\*
Esta triste narração\\*
Contando tudo direito\\*
Sem sair da oração

Eu disse ao caro leitor\\*
Que foi no mato encontrado\\*
Nosso padre Antonio Henrique\\*
De faca e bala furado,\\*
Agora conto direito\\*
Como ele foi sepultado

Na igreja do Espinheiro\\*
Foi o povo aglomerado\\*
E ao cemitério da Várzea\\*
Foi pelos fiéis levado\\*
O corpo do padre Henrique\\*
Que morreu martirizado

Enquanto o cortejo fúnebre\\*
Ia levando o caixão\\*
Este estribilho se ouvia\\*
Pela voz da multidão:\\*
“Prova de amor maior não há\\*
Que doar a vida pelo irmão”

Treze quilômetros a pé\\*
Levaram o corpo seu\\*
Lamentando lagrimosos\\*
O caso que aconteceu\\*
A morte de um jovem padre\\*
Que pelos jovens morreu

Ia naquele caixão\\*
Quem grande exemplo deixou\\*
Em defesa dos oprimidos\\*
A sua vida entregou\\*
E foi receber no céu\\*
O que na terra ganhou

O corpo ia acompanhado\\*
Em forma de procissão\\*
Com as vozes dos fiéis\\*
Ecoando na amplidão:\\*
“Prova de amor maior não há\\*
Que doar a vida pelo irmão”

A vida do padre Henrique\\*
Vamos guardar na memória\\*
Ele morreu pelo povo,\\*
É bonita a sua história\\*
E foi receber no céu\\*
Sua coroa de glória

Pensando no triste caso\\*
Entristeço e me comovo,\\*
O que muitos já disseram\\*
Eu disse e digo de novo\\*
O padre Henrique é um mártir\\*
Que morreu pelo seu povo

Prezado amigo leitor\\*
Esta dor é minha e sua\\*
De ver morrer padre Henrique\\*
De morte tirana e crua\\*
Porém a Igreja dos pobres\\*
Sua luta continua

Quem da igreja do Espinheiro\\*
Santa casa de oração\\*
Ao cemitério da Várzea\\*
Palmilhar aquele chão\\*
A 27 de maio\\*
Sentirá recordação

Do corpo de um padre jovem\\*
Conduzido em um caixão\\*
E parece ouvir uns versos\\*
Com sonora entoação\\*
“Prova de amor maior não há\\*
Que doar a vida pelo irmão”

\end{verse}

\chapter[Emigração e as consequências]{Emigração\break e as consequências}

\begin{verse}
Neste estilo popular\\*
Nos meus singelos versinhos,\\*
O leitor vai encontrar\\*
Em vez de rosas espinhos\\*
Na minha penosa lida\\*
Conheço do mar da vida\\*
As temerosas tormentas\\*
Eu sou o poeta da roça\\*
Tenho mão calosa e grossa\\*
Do cabo das ferramentas

Por força da natureza\\*
Sou poeta nordestino\\*
Porém só conto a pobreza\\*
Do meu mundo pequenino\\*
Eu não sei contar as glórias\\*
Nem também conto as vitórias\\*
Do herói com seu brasão\\*
Nem o mar com suas águas\\*
Só sei contar minhas mágoas\\*
E as mágoas do meu irmão

De contar a desventura\\*
Tenho sobrada razão\\*
Pois vivo de agricultura\\*
Sou camponês do sertão\\*
Sou um caboclo roceiro\\*
Eu trabalho o dia inteiro\\*
Exposto ao frio e ao calor\\*
Sofrendo a lida pesada\\*
Puxando o cabo da enxada\\*
Sem arado e sem trator

Nesta batalha danada,\\*
Correndo pra lá e pra cá\\*
Tenho a pele bronzeada\\*
Do sol do meu Ceará\\*
Mas o grande sofrimento\\*
Que abala o meu sentimento\\*
Que a providência me deu\\*
É saber que há desgraçados\\*
Por esse mundo jogados\\*
Sofrendo mais do que eu

É saber que há muita gente\\*
Padecendo privação\\*
Vagando constantemente\\*
Sem roupa, sem lar, sem pão,\\*
É saber que há inocentes\\*
Infelizes indigentes\\*
Que por esse mundo vão\\*
Seguindo errados caminhos\\*
Sem ter da mãe os carinhos\\*
Nem do pai a proteção

Leitor, a verdade assino\\*
É sacrifício de morte\\*
O do pobre nordestino\\*
Desprotegido da sorte\\*
Como bardo popular\\*
No meu modo de falar\\*
Nesta referência séria\\*
Muito desgostoso fico\\*
Por ver num país tão rico\\*
Campear tanta miséria

Quando há inverno abundante\\*
No meu Nordeste querido\\*
Fica o pobre em um instante\\*
Do sofrimento esquecido\\*
Tudo é graça, paz e riso\\*
Reina um verde paraíso\\*
Por vale, serra e sertão\\*
Porém não havendo inverno\\*
Reina um verdadeiro inferno\\*
De dor e de confusão

Fica tudo transformado\\*
Sofre o velho e sofre o novo\\*
Falta pasto para o gado\\*
E alimento para o povo\\*
E um drama de tristeza\\*
Parece que a natureza\\*
Trata a tudo com rigor\\*
Com esta situação\\*
O desumano patrão\\*
Despede o seu morador

Vendo o flagelo horroroso\\*
Vendo o grande desacato\\*
Infiel e impiedoso\\*
Aquele patrão ingrato\\*
Como quem declara guerra\\*
Expulsa da sua terra\\*
Seu morador camponês\\*
O coitado flagelado\\*
Seu inditoso agregado\\*
Que tanto favor lhe fez

Sem a virtude da chuva\\*
O povo fica a vagar\\*
Como a formiga saúva\\*
Sem folha para cortar\\*
E com a dor que o consome\\*
Obrigado pela fome\\*
E a situação mesquinha\\*
Vai um grupo flagelado\\*
Para atacar o mercado\\*
Da cidade mais vizinha

Com grande necessidade\\*
Sem rancor e sem malícia\\*
Entra a turma na cidade\\*
E sem temer a polícia\\*
Vai falar com o prefeito\\*
E se este não der um jeito\\*
Agora o jeito que tem\\*
É os coitados famintos\\*
Invadirem os recintos\\*
Da feira e do armazém

A fome é o maior martírio\\*
Que pode haver neste mundo,\\*
Ela provoca delírio\\*
E sofrimento profundo\\*
Tira o prazer e a razão\\*
Quem quiser ver a feição\\*
Da cara da mãe da peste,\\*
Na pobreza permaneça,\\*
Seja agregado e padeça\\*
Uma seca no Nordeste

Por causa desta inclemência\\*
Viajam pelas estradas\\*
Na mais cruel indigência\\*
Famílias abandonadas\\*
Deixando o céu lindo e azul\\*
Algumas vão para o Sul\\*
E outras para o Maranhão\\*
Cada qual com sua cruz\\*
Se valendo de Jesus\\*
E do padre Cícero Romão

Nestes medonhos consternos\\*
Sem meios para a viagem\\*
Muitas vezes os governos\\*
Para o Sul dão a passagem\\*
E a faminta legião\\*
Deixando o caro torrão,\\*
Entre suspiros e ais,\\*
O martírio inda mais cresce\\*
Porque quem fica padece\\*
E quem parte sofre mais

O carro corre apressado\\*
E lá no Sul faz “desejo”\\*
Deixando desabrigado\\*
O flagelado cortejo\\*
Que procurando socorro\\*
Uns vão viver pelo morro\\*
Um padecer sem desconte\\*
Outros pobres infelizes\\*
Se abrigam pelas marquises\\*
Outros debaixo da ponte

Rompendo mil empecilhos,\\*
Nisto tudo o que é pior\\*
É que o pai tem oito filhos\\*
E cada qual o menor\\*
Aquele homem sem sossego\\*
Mesmo arranjando um emprego\\*
Nada pode resolver\\*
Sempre na penúria está\\*
Pois o seu ganho não dá\\*
Para a família viver

Assim mesmo, neste estado\\*
O bom nordestino quer\\*
Estar sempre rodeado\\*
Por seus filhos e a mulher\\*
Quanto mais aumenta a dor\\*
Também cresce o seu amor\\*
Por sua prole adorada\\*
Da qual é grande cativo\\*
Pois é ela o lenitivo\\*
De sua vida cansada

A pobre esposa chorosa\\*
Naquele estranho ambiente\\*
Recorda muito saudosa\\*
Sua terra e sua gente\\*
Relembra o tempo de outrora,\\*
Lamenta, suspira e chora\\*
Com a alma dolorida\\*
Além da necessidade\\*
Padece a roxa saudade\\*
De sua terra querida

Para um pequeno barraco\\*
Já saíram da marquise\\*
Mas cada qual o mais fraco\\*
Padecendo a mesma crise,\\*
Porque o pequeno salário\\*
Não dá para o necessário\\*
Da sua manutenção\\*
E além disto falta roupa\\*
E sobre sacos de estopa\\*
Todos dormindo no chão

Naquele ambiente estranho\\*
Continua a indigência\\*
Rigor de todo tamanho\\*
Sem ninguém dar assistência\\*
Aquela família triste\\*
Ninguém vê, ninguém assiste\\*
Com alimento e com veste,\\*
Que além da situação\\*
Padece a recordação\\*
Das coisas do seu Nordeste

Meu leitor, não tenha enfado\\*
Vamos ver mais adiante\\*
Quanto é triste o resultado\\*
Do nordestino emigrante\\*
Quero provar-lhe a carência\\*
O desgosto e a inclemência\\*
Que sofre o pobre infeliz\\*
Que deixa a terra onde mora\\*
E vai procurar melhora\\*
Lá pelo Sul do país

O pobre no seu emprego\\*
Seguindo penosos trilhos\\*
Seu prazer é o aconchego\\*
De sua esposa e seus filhos\\*
Naquele triste penar\\*
Vai outro emprego arranjar\\*
Na fábrica ou no armazém\\*
À procura de melhora\\*
Até que a sua senhora\\*
Tem um emprego também

Se por um lado melhora\\*
Aumentando mais o pão\\*
Por outro lado piora\\*
A triste situação\\*
Pois os garotos ficando\\*
E a vida continuando\\*
Sem os cuidados dos pais\\*
Sozinhos naquele abrigo\\*
Se expõem ao grande perigo\\*
Da vida dos marginais

Eles ficando sozinhos\\*
Logo fazem amizade\\*
Em outros bairros vizinhos\\*
Com garotos da cidade\\*
Infelizes criaturas\\*
Que procuram aventuras\\*
No mais triste padecer\\*
Crianças abandonadas\\*
Que vagam desesperadas\\*
Atrás de sobreviver

Esses pobres delinquentes,\\*
Os infelizes meninos,\\*
Atraem os inocentes\\*
Flagelados nordestinos\\*
E estes com as relações,\\*
Vão recebendo instruções,\\*
Com aqueles aprendendo\\*
E assim, mal acompanhados,\\*
Em breve aqueles coitados\\*
Vão algum furto fazendo

São crianças desvalidas\\*
Que os pais não lhe dão sustento,\\*
As mães desaparecidas\\*
Talvez no mesmo tormento\\*
Não há quem conheça o dono\\*
Desses filhos do abandono,\\*
Que sem temerem perigos,\\*
Vão esmolando, furtando\\*
E às vezes até tomando\\*
O dinheiro dos mendigos

Os pais voltam dos trabalhos\\*
Cansados mas destemidos\\*
E encontram os seus pirralhos\\*
No barraco recolhidos,\\*
O pai dizendo gracejo\\*
Dá em cada qual um beijo\\*
Com amorosos acenos;\\*
Cedo do barraco sai\\*
Não sabe como é que vai\\*
A vida dos seus pequenos

No dia seguinte os filhos\\*
Fazem a mesma viagem\\*
Nos seus costumeiros trilhos\\*
Na mesma camaradagem\\*
Com os mesmos companheiros\\*
Aqueles aventureiros\\*
Que na maior anarquia\\*
Sem terem o que comer\\*
Vão rapinagem fazer\\*
Para o pão de cada dia

Sem já ter feito o seu teste\\*
Em um inditoso dia\\*
Um garoto do Nordeste\\*
Entra em uma padaria\\*
E já com água na boca\\*
E necessidade louca\\*
Se encostando no balcão\\*
Faz mesmo sem ter coragem\\*
A primeira traquinagem\\*
Dali carregando um pão

Volta bastante apressado\\*
O pobre inexperiente\\*
Olhando desconfiado\\*
Para trás e para frente\\*
Mas naquele mesmo instante\\*
Vai apanhado em flagrante\\*
Na porta da padaria\\*
Indo o pequeno indigente\\*
Logo rigorosamente\\*
Levado à delegacia

É aquela a vez primeira\\*
Que o garoto preso vai\\*
Faz a maior berradeira\\*
Grita por mãe e por pai\\*
Mas outros garotos presos\\*
Que já não ficam surpresos\\*
Com história de prisão\\*
Consolam o pequenino\\*
Dando instrução ao menino\\*
Da marginalização

Depois que aquela criança\\*
Da prisão tem liberdade;\\*
Na mesma vida se lança\\*
Pelas ruas da cidade\\*
E assim vai continuando\\*
Aliada ao mesmo bando\\*
Forçados pela indigência\\*
Pra criança abandonada\\*
Prisão não resolve nada\\*
O remédio é assistência

Quem examina descobre\\*
Que é sorte muito infeliz\\*
A do nordestino pobre\\*
Lá pelo Sul do país\\*
A sua filha querida\\*
Às vezes vai iludida\\*
Pelo monstro sedutor\\*
E devido à ingenuidade\\*
Finda fazendo a vontade\\*
Do monstro devorador

Foge do rancho dos pais\\*
E vai vagar pelo mundo\\*
Padecendo muito mais\\*
Nas garras do vagabundo\\*
O pobre pai revoltado\\*
Fica desmoralizado\\*
Com a alma dolorida\\*
Para o homem nordestino\\*
O brio é um dom divino\\*
A honra é a própria vida

Aquele pai fica cheio\\*
De revolta e de rancor\\*
Mas não pode achar um meio\\*
De encontrar o malfeitor\\*
Porém se casualmente\\*
Encontrar o insolente\\*
Lhe dará fatal destino\\*
Pois foi sempre esse o papel\\*
E a justiça mais fiel\\*
Do caboclo nordestino

Leitor, veja o grande azar\\*
Do nordestino emigrante\\*
Que anda atrás de melhorar\\*
Da sua terra distante\\*
Nos centros desconhecidos\\*
Depressa vê corrompidos\\*
Os seus filhos inocentes\\*
Na populosa cidade\\*
De tanta imoralidade\\*
E costumes diferentes

A sua filha querida\\*
Vai pra uma iludição\\*
Padecer prostituída\\*
Na vala da perdição\\*
E além da grande desgraça\\*
Das privações que ela passa\\*
Que lhe atrasa e lhe inflama\\*
Sabe que é preso em flagrante\\*
Por coisa insignificante\\*
Seu filho a quem tanto ama

Para que maior prisão\\*
Do que um pobre sofrer\\*
Privação e humilhação\\*
Sem ter com que se manter?\\*
Para que prisão maior\\*
Do que derramar o suor\\*
Em um estado precário\\*
Na mais penosa atitude\\*
Minando a própria saúde\\*
Por um pequeno salário?

Será que o açoite, as algemas\\*
E um quarto da detenção\\*
Vão resolver o problema\\*
Da triste situação?\\*
Não há prisão mais incrível\\*
Mais feia, triste e horrível\\*
Mais dura e mais humilhante\\*
Do que a de um desgraçado\\*
Pelo mundo desprezado\\*
E do seu berço tão distante

O garoto tem barriga,\\*
Também precisa comer\\*
E a cruel fome lhe obriga\\*
A rapinagem fazer\\*
Se ninguém a ele ajuda\\*
O itinerário não muda\\*
Os miseráveis infantes\\*
Que vivem abandonados\\*
Terão tristes resultados\\*
Serão homens assaltantes

Meu divino redentor\\*
Que pregou na Palestina\\*
Harmonia, paz e amor\\*
Na vossa santa doutrina\\*
Pela vossa mãe querida\\*
Que é sempre compadecida\\*
Carinhosa, terna e boa\\*
Olhai para os pequeninos\\*
Para os pobres nordestinos\\*
Que vivem no mundo à toa

Meu bom Jesus Nazareno\\*
Pela vossa majestade\\*
Fazei que cada pequeno\\*
Que vaga pela cidade\\*
Tenha boa proteção\\*
Tenha em vez de uma prisão\\*
Aquele medonho inferno\\*
Que revolta e desconsola\\*
Bom conforto e boa escola\\*
Um lápis e o caderno
\end{verse}

\chapter{Brosogó, Militão e o diabo}

\begin{verse}
O melhor da nossa vida\\*
É paz, amor e união\\*
E em cada semelhante\\*
A gente ver um irmão\\*
E apresentar para todos\\*
O papel de gratidão

Quem faz um grande favor\\*
Mesmo desinteressado\\*
Por onde quer que ele ande\\*
Leva um tesouro guardado\\*
E um dia sem esperar \\*
Será bem recompensado

Em um dos nossos estados\\*
Do Nordeste brasileiro\\*
Nasceu Chico Brosogó\\*
Era ele um miçangueiro\\*
Que é o mesmo camelô\\*
Lá no Rio de Janeiro

O Brosogó era ingênuo\\*
Não tinha filosofia\\*
Mas tinha de honestidade\\*
A maior sabedoria\\*
Sempre vendendo ambulante\\*
A sua mercadoria

Em uma destas viagens\\*
Numa certa região\\*
Foi vender mercadoria\\*
Na famosa habitação\\*
De um fazendeiro malvado\\*
Por nome de Militão

O ricaço Militão\\*
Vivia a questionar\\*
Toda sorte de trapaça\\*
Era capaz de inventar\\*
Vendo assim desta maneira\\*
Sua riqueza aumentar

Brosogó naquele prédio\\*
Não apurou um tostão\\*
E como na mesma casa\\*
Não lhe ofereceram pão\\*
Comprou meia dúzia de ovos\\*
Para sua refeição

Quando a meia dúzia de ovos\\*
O Brosogó foi pagar\\*
Faltou dinheiro miúdo\\*
Para a paga efetuar\\*
E ele entregou uma nota\\*
Para o Militão trocar

O rico disse: Eu não troco,\\*
Vá com a mercadoria\\*
Qualquer tempo você vem\\*
Me pagar esta quantia\\*
Mas peço que seja exato\\*
E aqui me apareça um dia

Brosogó agradeceu\\*
E achou o papel importante,\\*
Sem saber que o Militão\\*
Estava naquele instante\\*
Semeando uma semente\\*
Para colher mais adiante

Voltou muito satisfeito\\*
Na sua vida pensando\\*
Sempre arranjando fregueses\\*
No lugar que ia passando\\*
Vendo sua boa sorte\\*
Melhorar de quando em quando

Brosogó no seu comércio\\*
Tinha bons conhecimentos\\*
Possuía com os lucros\\*
Daqueles seus movimentos\\*
Além de casas e terrenos\\*
Meia dúzia de jumentos

De ano em ano ele fazia\\*
Naquele seu patrimônio\\*
Festejo religioso\\*
No dia de santo Antônio\\*
Por ser o aniversário\\*
Do seu feliz matrimônio

No festejo oferecia\\*
Vela para são João\\*
Santo Ambrósio, santo Antônio\\*
São Cosme e são Damião\\*
Para ele qualquer santo\\*
Dava a mesma proteção

Vela para santa Inês\\*
E para santa Luzia\\*
São Jorge e são Benedito\\*
São José e santa Maria\\*
Até que chegava à última\\*
Das velas que possuía

Um certo dia voltando\\*
Aquele bom sertanejo\\*
Da viagem lucrativa\\*
Com muito amor e desejo\\*
Trouxe uma carga de velas\\*
Para queimar no festejo

A casa naquela noite\\*
Estava um belíssimo encanto\\*
Se via velas acesas\\*
Brilhando por todo canto\\*
Porém sobraram três velas\\*
Por faltar nome de santo

Era linda a luminária\\*
O quadro resplandescente\\*
E o caboclo Brosogó\\*
Procurava impaciente\\*
Mas nem um nome de santo\\*
Chegava na sua mente

Disse consigo: o Diabo\\*
Merece vela também\\*
Se ele nunca me tentou\\*
Para ofender a ninguém\\*
Com certeza me respeita\\*
Está me fazendo o bem

Se eu fui um menino bom\\*
Fui também um bom rapaz\\*
E hoje sou pai de família\\*
Gozando da mesma paz\\*
Vou queimar estas três velas\\*
Em tenção do Satanás

Tudo aquilo Brosogó\\*
Fez com naturalidade\\*
Como o justo que apresenta\\*
Amor e fraternidade\\*
E as virtudes preciosas\\*
De um coração sem maldade

Certo dia ele fazendo\\*
Severa reflexão\\*
Um exame rigoroso\\*
Sobre a sua obrigação\\*
Lhe veio na mente os ovos\\*
Que devia ao Militão

Viajou muito apressado\\*
No seu jumento baixeiro\\*
Sempre atravessando rio\\*
E transpondo tabuleiro\\*
Chegou no segundo dia\\*
Na casa do trapaceiro

Foi chegando e desmontando\\*
E logo que deu bom dia\\*
Falou para o coronel\\*
Com bastante cortesia:\\*
Venho aqui pagar a conta\\*
Que fiquei devendo um dia

O Militão muito sério\\*
Falou para Brosogó:\\*
Para pagar esta dívida\\*
Você vai ficar no pó,\\*
Mesmo que tenha recurso\\*
Fica pobre como Jó

Me preste bem atenção\\*
E ouça bem as razões minhas:\\*
Aqueles ovos no choco\\*
Iam tirar seis pintinhas\\*
Mais tarde as mesmas seriam\\*
Meia dúzia de galinhas

As seis galinhas botando,\\*
Veja a soma o quanto dá\\*
São quatrocentos e oitenta\\*
Ninguém me reprovará\\*
Pois a galinha aqui põe\\*
De oito ovos pra lá

Preste atenção Brosogó\\*
Sei que você não censura\\*
Veja que grande vantagem\\*
Veja que grande fartura\\*
E veja o meu resultado\\*
Só na primeira postura

Das quatrocentas e oitenta\\*
Podia a gente tirar\\*
Dos mesmos cento e cinquenta\\*
Para no choco aplicar\\*
Pois basta só vinte e cinco\\*
Que é pra o ovo não gorar

Os trezentos e cinquenta\\*
Que era a sobra eu vendia\\*
Depressa, sem ter demora\\*
Por uma boa quantia\\*
Aqui, procurando ovos\\*
Temos grande freguesia

Dos cento e cinquenta ovos\\*
Sairiam com despacho\\*
Cento e cinquenta pintinhas\\*
Pois tenho certeza e acho\\*
Que aqui no nosso terreiro\\*
Não se cria pinto macho

Também não há prejuízo\\*
Posso falar pra você\\*
Que maracajá e raposa\\*
Aqui a gente não vê\\*
Também não há cobra preta\\*
Gavião, nem saruê

Aqui de certas moléstias\\*
A galinha nunca morre\\*
Porque logo à medicina\\*
Com urgência se recorre\\*
Se o gogó se manifesta\\*
A empregada socorre

Veja bem, seu Brosogó\\*
O quanto eu posso ganhar\\*
Em um ano e sete meses\\*
Que passou sem me pagar,\\*
A conta é de tal maneira\\*
Que eu mesmo não sei somar

Vou chamar um matemático\\*
Pra fazer o orçamento,\\*
Embora você não faça\\*
De uma vez o pagamento\\*
Mesmo com mercadoria,\\*
Terreno, casa e jumento

Porém tenha paciência\\*
Não precisa se queixar,\\*
Você acaba o que tem,\\*
Mas vem comigo morar\\*
E aqui, parceladamente,\\*
Acaba de me pagar

E se achar que estou falando\\*
Contra sua natureza,\\*
Procure um advogado\\*
Pra fazer sua defesa,\\*
Que o meu eu já tenho e conto\\*
A vitória com certeza

Meu advogado é\\*
Um doutor de posição\\*
Pertence à minha política\\*
E nunca perdeu questão\\*
E é candidato a prefeito\\*
Para a futura eleição

O coronel Militão\\*
Com orgulho e petulância\\*
Deixou o pobre Brosogó\\*
Na mais dura circunstância\\*
Aproveitando do mesmo\\*
Sua grande ignorância

Quinze dias foi o prazo\\*
Para o Brosogó voltar\\*
Presente ao advogado\\*
Um documento assinar\\*
E tudo que possuía\\*
Ao Militão entregar

O pobre voltou bem triste\\*
Pensando, a dizer consigo:\\*
Eu durante a minha vida\\*
Sempre fui um grande amigo,\\*
Qual será o meu pecado\\*
Para tão grande castigo?

Quando ia pensando assim\\*
Avistou um cavaleiro\\*
Bem montado e bem trajado\\*
Na sombra de um juazeiro\\*
O qual com modos fraternos\\*
Perguntou ao miçangueiro:

Que tristeza é esta?\\*
Que você tem, Brosogó?\\*
O seu semblante apresenta\\*
Aflição, pesar e dó,\\*
Eu estou ao seu dispor,\\*
Você não sofrerá só

Brosogó lhe contou tudo\\*
E disse por sua vez\\*
Que o coronel Militão\\*
O trato com ele fez\\*
Para às dez horas do dia\\*
Na data quinze do mês

E disse o desconhecido:\\*
Não tenha má impressão\\*
No dia quinze eu irei\\*
Resolver esta questão\\*
Lhe defender da trapaça\\*
Do ricaço Militão

Brosogó foi para casa\\*
Alegre sem timidez,\\*
O que o homem lhe pediu\\*
Ele satisfeito fez\\*
E foi cumprir seu trato\\*
No dia quinze do mês

Quando chegou encontrou\\*
Todo povo aglomerado\\*
Ele entrando deu bom dia\\*
E falou bem animado\\*
Dizendo que também tinha\\*
Achado um advogado

Marcou o relógio dez horas\\*
E sem o doutor chegar\\*
Brosogó entristeceu\\*
Silencioso a pensar\\*
E o povo do Militão\\*
Do coitado a criticar

Os puxa-sacos do rico\\*
Com ares de mangação\\*
Diziam: o miçangueiro\\*
Vai-se arrasar na questão\\*
Brosogó vai pagar caro\\*
Os ovos de Militão

Estavam pilheriando\\*
Quando se ouviu um tropel\\*
Era um senhor elegante\\*
Montado no seu corcel\\*
Exibindo em um dos dedos\\*
O anel de bacharel

Chegando disse aos ouvintes:\\*
Fui no trato interrompido\\*
Para cozinhar feijão\\*
Porque muito tem chovido\\*
E o meu pai em seu roçado\\*
Só planta feijão cozido

Antes que o desconhecido\\*
Com razão se desculpasse\\*
Gritou o outro advogado:\\*
Não desonre a nossa classe\\*
Com essa grande mentira!\\*
Feijão cozido não nasce

Respondeu o cavaleiro:\\*
Esta mentira eu compus\\*
Para fazer a defesa\\*
É ela um foco de luz\\*
Porque o ovo cozinhado\\*
Sabemos que não produz

Assim que o desconhecido\\*
Fez esta declaração\\*
Houve um silêncio na sala\\*
Foi grande a decepção\\*
Para o povo da política\\*
Do coronel Militão

Onde a verdade aparece\\*
A mentira é destruída\\*
Foi assim desta maneira\\*
Que a questão foi resolvida\\*
E o candidato político\\*
Ficou de crista caída

Mentira contra mentira\\*
Na reunião se deu\\*
E foi por este motivo\\*
Que a verdade apareceu\\*
Somente o preço dos ovos\\*
O Militão recebeu

Brosogó agradecendo\\*
O favor que recebia\\*
Respondeu o cavaleiro:\\*
Eu era quem lhe devia\\*
O valor daquelas velas\\*
Que me ofereceu um dia

Eu sou o Diabo a quem todos\\*
Chamam de monstro ruim\\*
E só você neste mundo\\*
Teve a bondade sem fim\\*
De um dia queimar três velas\\*
Oferecidas a mim

Quando disse estas palavras\\*
No mesmo instante saiu\\*
Adiante deu um pipoco\\*
E pelo espaço sumiu\\*
Porém pipoco baixinho\\*
Que o Brosogó não ouviu

Caro leitor nesta estrofe\\*
Não queira zombar de mim\\*
Ninguém ouviu o estouro\\*
Mas juro que foi assim\\*
Pois toda história do diabo\\*
Tem um pipoco no fim

Sertanejo, este folheto\\*
Eu quero lhe oferecer,\\*
Leia o mesmo com cuidado\\*
E saiba compreender,\\*
Encerra muita mentira\\*
Mas tem muito o que aprender

Bom leitor, tenha cuidado,\\*
Vivem ainda entre nós\\*
Milhares de Militões\\*
Com o instinto feroz\\*
Com traçadas e mentiras\\*
Perseguindo os Brosogós

\end{verse}

\chapter{ABC do Nordeste flagelado}

\begin{verse}
A -- Ai como é duro viver\\*
Nos estados do Nordeste\\*
Quando o nosso Pai Celeste\\*
Não manda a nuvem chover,\\*
É bem triste a gente ver\\*
Findar o mês de janeiro\\*
Depois findar fevereiro\\*
E março também passar\\*
Sem o inverno começar\\*
No Nordeste brasileiro

B -- Berra o gado impaciente\\*
Reclamando o verde pasto,\\*
Desfigurado e arrasto\\*
Com o olhar de penitente\\*
O fazendeiro, descrente\\*
Um jeito não pode dar\\*
O sol ardente a queimar\\*
E o vento forte soprando,\\*
A gente fica pensando\\*
Que o mundo vai se acabar

C -- Caminhando pelo espaço\\*
Como os trapos de um lençol,\\*
Pras bandas do pôr do sol\\*
As nuvens vão em fracasso;\\*
Aqui e ali um pedaço\\*
Vagando\ldots{} sempre vagando\\*
Quem estiver reparando\\*
Faz logo a comparação\\*
De umas pastas de algodão\\*
Que o vento vai carregando

D -- De manhã, bem de manhã\\*
Vem da montanha um agouro\\*
De gargalhada e de choro\\*
Da feia e triste cauã,\\*
Um bando de ribançã\\*
Pelo espaço a se perder\\*
Pra de fome não morrer\\*
Vai atrás de outro lugar\\*
E ali só há de voltar\\*
Um dia quando chover

E -- Em tudo se vê mudança\\*
Quem repara vê até\\*
Que o camaleão que é\\*
Verde da cor de esperança\\*
Com o flagelo que avança\\*
Muda logo de feição\\*
O verde camaleão\\*
Perde a sua cor bonita\\*
Fica de forma esquisita\\*
Que causa admiração

F -- Fogo o prazer da floresta\\*
O bonito sabiá,\\*
Quando flagelo não há\\*
Cantando se manifesta\\*
Durante o inverno faz festa\\*
Gorjeando por esporte\\*
Mas não chovendo é sem sorte\\*
Fica sem graça e calado\\*
O cantor mais afamado\\*
Dos passarinhos do Norte

G -- Geme de dor, se aquebranta\\*
E dali desaparece\\*
O sabiá só parece\\*
Que com a seca se encanta\\*
Se outro pássaro canta\\*
O coitado não responde;\\*
Ele vai não sei pra onde,\\*
Pois quando o inverno não vem\\*
Com o desgosto que tem\\*
O pobrezinho se esconde

H -- Horroroso, feio e mau\\*
De lá de dentro das grotas\\*
Manda suas feias notas\\*
O tristonho bacurau\\*
Canta o João corta-pau\\*
O seu poema numério;\\*
É muito triste o mistério\\*
De uma seca no sertão\\*
A gente tem impressão\\*
Que o mundo é um cemitério

I -- Ilusão, prazer, amor\\*
A gente sente fugir,\\*
Tudo parece carpir\\*
Tristeza, saudade e dor\\*
Nas horas de mais calor\\*
Se escuta pra todo lado\\*
O toque desafinado\\*
Da gaita da siriema\\*
Acompanhando o cinema\\*
No Nordeste flagelado

J -- Já falei sobre a desgraça\\*
Dos animais do Nordeste;\\*
Com a seca vem a peste\\*
E a vida fica sem graça,\\*
Quanto mais dias se passa\\*
Mais a dor se multiplica\\*
A mata que já foi rica,\\*
De tristeza geme e chora\\*
Preciso dizer agora\\*
O povo como é que fica

L -- Lamenta desconsolado\\*
O coitado camponês\\*
Porque tanto esforço fez,\\*
Mas não lucrou seu roçado\\*
Num banco velho, sentado\\*
Olhando o filho inocente\\*
E a mulher bem paciente,\\*
Cozinha lá no fogão\\*
O derradeiro feijão\\*
Que ele guardou pra semente

M -- Minha boa companheira,\\*
Diz ele, vamos embora,\\*
E depressa, sem demora\\*
Vende a sua cartucheira,\\*
Vende a faca, a roçadeira,\\*
Machado, foice e facão;\\*
Vende a pobre habitação,\\*
Galinha, cabra e suíno\\*
E viajam sem destino\\*
Em cima de um caminhão

N -- Naquele duro transporte\\*
Sai aquela pobre gente\\*
Aguentando paciente\\*
O rigor da triste sorte\\*
Levando a saudade forte\\*
De seu povo e seu lugar\\*
Sem nem um outro falar\\*
Vão pensando em sua vida\\*
Deixando a terra querida\\*
Para nunca mais voltar

O -- Outro tem opinião\\*
De deixar mãe, deixar pai,\\*
Porém para o Sul não vai\\*
Procura outra direção,\\*
Vai bater no Maranhão\\*
Onde nunca falta inverno;\\*
Putro com grande consterno\\*
Deixa o casebre e a mobília\\*
E leva sua família\\*
Pra construção do governo

P -- Porém lá na construção\\*
O seu viver é grosseiro\\*
Trabalhando o dia inteiro\\*
De picareta na mão\\*
Pra sua manutenção\\*
Chegando dia marcado\\*
Em vez do seu ordenado\\*
Dentro da repartição\\*
Recebe triste ração\\*
Farinha e feijão furado

Q -- Quem quer ver o sofrimento\\*
Quando há seca no sertão\\*
Procura uma construção\\*
E entra no fornecimento\\*
Pois, dentro dele, o alimento\\*
Que o pobre tem a comer\\*
a barriga pode encher,\\*
Porém falta a substância\\*
E com esta circunstância\\*
Começa o povo a morrer

R -- Raquítica, pálida e doente\\*
Fica a pobre criatura\\*
E a boca da sepultura\\*
Vai engolindo o inocente,\\*
Meu Jesus! Meu Pai Clemente\\*
que da humanidade é dono\\*
Desça do seu alto trono,\\*
Da sua corte celeste\\*
E venha ver seu Nordeste\\*
Como ele está no bandono

S -- Sofre o casado e o solteiro\\*
Sofre o velho, sofre o moço\\*
Não tem janta nem almoço\\*
Não tem roupa nem dinheiro\\*
Também sofre o fazendeiro\\*
Que de rico perde o nome,\\*
O desgosto lhe consome\\*
Vendo o urubu esfomeado\\*
Puxando a pele do gado\\*
Que morreu de sede e fome

T -- Tudo sofre e não resiste\\*
Este fardo tão pesado,\\*
No Nordeste flagelado\\*
Em tudo a tristeza existe,\\*
Mas a tristeza mais triste\\*
Que faz tudo entristecer\\*
É a mãe, chorosa a gemer\\*
Lágrimas dos olhos correndo,\\*
Vendo seu filho dizendo:\\*
Mamãe, eu quero comer!

U -- Um é ver, outro é contar\\*
Quem for reparar de perto\\*
Aquele mundo deserto\\*
Dá vontade de chorar,\\*
Ali só fica a teimar\\*
O juazeiro copado,\\*
O resto é tudo pelado\\*
Da chapada ao tabuleiro,\\*
Onde o famoso vaqueiro\\*
Cantava tangendo o gado

V -- Vivendo em grande maltrato,\\*
A abelha zumbindo voa\\*
Sem direção, sempre à toa\\*
Por causa do desacato\\*
À procura de um regato,\\*
De um jardim ou de um pomar\\*
Sem um momento parar,\\*
Vagando constantemente,\\*
Sem encontrar, a inocente,\\*
Uma flor para pousar

X -- Xexéu, pássaro que mora\\*
Na grande árvore copada,\\*
Vendo a floresta arrasada,\\*
Bate as asas, vai embora;\\*
Somente o sagui demora,\\*
Pulando a fazer careta,\\*
Na mata tingida e preta\\*
Tudo é aflição e pranto;\\*
Só por milagre de um santo\\*
Se encontra uma borboleta

Z -- Zangado contra o sertão\\*
Dardeja o sol inclemente,\\*
Cada dia mais ardente\\*
Tostando a face do chão;\\*
E, mostrando compaixão\\*
Lá do infinito estrelado,\\*
Pura, limpa, sem pecado\\*
De noite a lua derrama\\*
um banho de luz no drama\\*
Do Nordeste flagelado

Posso dizer que cantei\\*
Aquilo que observei;\\*
Tenho certeza que dei\\*
Aprovada relação\\*
Tudo é tristeza e amargura,\\*
Indigência e desventura,\\*
Veja, leitor, quanto é dura\\*
A seca no meu sertão\\*

\end{verse}

