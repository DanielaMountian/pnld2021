\chapter{Vida e obra de Patativa do Assaré}

\section{Sobre o autor}

\noindent{}Antonio Gonçalves da Silva, o Patativa do Assaré, foi um dos mais importantes
poetas brasileiros. Nascido no Cariri, na Serra de Santana, próximo a Assaré,
no Ceará, a 5 de março de 1909, desde menino fazia versos e os apresentava a
quem quisesse ouvir. Só em 1956 seus poemas apareceriam em livro, com a edição
pela Borsoi, do Rio de Janeiro, do belo \textit{Inspiração nordestina}. O
sucesso de grande público, contudo, viria pouco depois, em 1964, com a gravação
em disco de ``A triste partida'', poema musicado pelo Rei do Baião, Luiz
Gonzaga. Poeta de genuína inspiração popular, Patativa do Assaré tornou"-se
sinônimo de poesia popular no país, tendo lançado, em sua longa vida de quase
um século, uma dezena de livros e discos com seus poemas, além de inúmeros
folhetos avulsos de cordel. De sua obra, destacam"-se os livros
\textit{Inspiração nordestina} (1956), \textit{Cante lá que eu canto cá}
(1978), \textit{Ispinho e fulô} (1988), \textit{Balceiro} (1991), \textit{Aqui
tem coisa} (1994) e \textit{Digo e não peço segredo} (2001); e os discos
\textit{Poemas e canções} (1979), \textit{A terra é naturá} (1981) e
\textit{Canto nordestino} (1989). Doutor \textit{honoris causa} em diversas
universidades, objeto de teses, filmes e peças, a voz da ave canora que lhe deu
nome, a patativa, ainda será ouvida por muitos e muitos anos em qualquer canto
do Brasil. Patativa morreu aos 93 anos em sua casa em Assaré.

Além de poeta extremamente interessante nos causos, histórias e fatos de sua terra, Patativa do Assaré também cantou muito sobre a política e a situação de seu povo, engajando"-se em debates como a reforma agrária.
Nas palavras do próprio poeta:

\begin{quote}
Eu sou um caboclo roceiro que, como poeta, canto sempre a vida do
povo. O meu problema é cantar a vida do povo, o sofrimento do meu
Nordeste, principalmente daqueles que não têm terra, porque o ano
presente, esse ano que está se findando, não foi uma seca, podemos dizer
que não foi a seca. Lá pelo interior, mesmo no município de Assaré, lá
no Assaré, tem duas frentes de serviço, com muita gente. Mas naquela
frente de serviço nós podemos observar que é só dos desgraçados que
não possuem terra. Os camponeses que possuem terra não sofrem estas
consequências e não precisam recorrer ao trabalho de emergência, como
os agregados e esses outros desgraçados trabalham na terra dos patrões.
E é isso que eu mais sinto: é ver um homem que tanto trabalha, pai de
família e não possui um palmo de terra. É por isso que é preciso que
haja um meio da reforma agrária chegar, uma reforma agrária que chegue para o povo que não tem terra.

E esta luta pela reforma agrária e pelo sindicato dos camponeses, mas o
verdadeiro sindicato conduzido pelos próprios camponeses, procurando,
reivindicando os seus direitos, é preciso que continue até chegar o
tempo do camponês sofrer menos do que vem sofrendo. Precisa fazer como
eu digo nos meus versos ``Lição do Pinto'', pois o pinto sai do ovo
porque trabalha. Ele belisca a casca do ovo, rompe e sai. É assim que o
povo também deve fazer, unido sempre, trabalhando.\footnote{Depoimento concedido a Rosemberg
Cariry, no Crato, em 1979.}
\end{quote}

\section{Sobre a obra}

\emph{Ispinho e Fulô} reúne poemas que recontam a trajetória de Patativa
do Assaré, passando por diferentes temas e versificações, apresentando
as principais características da poesia popular -- poesia essa que,
muitas vezes, é deixada de lado por se distanciar do erudito. No
entanto, Patativa do Assaré contribuiu ativamente para a construção e
divulgação da identidade nordestina, utilizando, em seus poemas, imagens
marcantes da tradição popular e uma grande variedade de imagens que
simbolizam o nordeste brasileiro.

A Literatura de Cordel sofreu, estruturalmente, diversas modificações
com o passar dos anos, por se tratar de uma linguagem oral que foi sendo
transformada, também, em escrita. No início, os repentistas não tinham
compromisso com número de versos ou métrica, entretanto a rima sempre
esteve presente nos poemas -- instrumento utilizado para favorecer a
memorização e facilitar a articulação dos repentistas. Entretanto, a
simplicidade não está atrelada apenas à oralidade, mas também ao alcance
social que uma linguagem acessível pode fornecer.

Em \emph{Ispinho e Fulô}, publicado em 1988, estão reunidos setenta e oito poemas dos mais
diversos tipos de métricas e rimas, representando a expressão do canto
do poeta e desvencilhando-se da forma de saber erudita.

Inserida no contexto sertanejo, a obra de Patativa do Assaré é influenciada pela tradição medieval, como a arte dos trovadores, que viria a influir na formação da poesia de cordel, marcada pela figura dos repentistas, violeiros e cantadores.
Cantando os sofrimentos e desgraças, mas também as alegrias da população nordestina do sertão, é forçoso reconhecer Patativa do Assaré como um dos maiores representantes da literatura de cordel. Nas palavras de Sylvie Debs, estudiosa da obra do poeta:

\begin{quote}
Testemunha então de um modo de vida, mas também reivindicação de valores
próprios, elaboração de uma identidade. Por isso, ele é apresentado como o
“verdadeiro, autêntico e legítimo intérprete do sertão”.\footnote{ Plácido Cidade
Nuvens, \textit{Patativa e o universo fascinante do sertão}, p. 15.} Com
efeito, uma das dimensões mais marcantes da obra de Patativa do Assaré é a
preocupação de descrever a vida cotidiana do sertão e, com esse testemunho,
protestar o reconhecimento da dignidade, da integridade e da modéstia do
camponês sertanejo, em oposição à arrogância do cidadão urbano ou do brasileiro
do Sul. Parece que a afirmação de sua própria identidade passa mais
frequentemente pelo confronto com o outro, como chama atenção o título da
compilação: \textit{Cante lá que eu canto cá}. Esta última, composta a partir de uma
seleção de textos feita pelo próprio autor com a intenção de definir suas
preferências literárias, traz o seguinte subtítulo: Filosofia de um trovador
nordestino. É, portanto, referindo"-nos de uma só vez ao conjunto dos poemas
publicados e à vida de Patativa do Assaré que tentaremos depreender as
características próprias da sua obra.\footnote{\textsc{debs}, Sylvie. ``Introduçã''. In: \textsc{assaré}, Patativa do. \textit{Cordel na escola}. São Paulo: Hedra, 2000, p.\,14.}
\end{quote}

Nas primeiras obras do poeta, fruto de improvisações e encomendas, ressaltam"-se os traços lúdicos e comemorativos. São poemas de circunstâncias, ligados aos acontecimentos sociais e religiosos, como festas de santos, casamentos, aniversários e outros acontecimentos com relação direta com o presente.
É uma poesia, portanto, que está fortemente inserida na vida da comunidade e dela participa.

Como ressalta Sylvie Debs, a marca da oralidade, inserida nesse contexto comunitário, era tão forte que as primeiras compilações de sua poesia saíram com alguns aparatos para orientar o leitor nesse novo vocabulário: ''A marca oral e regional era tão intrínseca à primeira compilação que foi
publicada com um “Elucidário” que propunha três esclarecimentos diferentes ao
leitor: uma simples restituição fonética (\textit{biête} por bilhete ou \textit{muié} por mulher), uma correspondência referencial (cão por diabo) e uma explicação denotativa (tipoia: rede pequena, rede velha)''.\footnote{Ibid., p.\,15.}

Outra marca significativa da oralidade, ainda na análise de Debs, é a forte presença da função conotativa da linguagem:

\begin{quote}
interpelação do ouvinte como
\textit{Cante lá que eu canto cá}, interrogações como “Você se lembra?”, “Seu Dotô me
conhece?”, destinação como “Ao leitor”, “Aos poetas clássicos”, “À minha esposa
Belinha”. Da mesma forma, os primeiros versos de seus poemas instauram,
geralmente, o ritual discursivo, seja como forma de indagação: “Querem saber
quem eu sou?” (\textit{Aqui tem coisa}, p. 63); seja sob forma de oração: “Quero que me dê licença
para uma história contá.” (\textit{Cante lá que eu canto cá}, p. 47); seja por uma saudação: “Boa noite, home
e menino e muié dêste lugá.” (\textit{Inspiração nordestina}, p. 27); seja ainda por uma ordem: “Vem cá,
Maria Gulora, escuta, que eu vou agora uma coisa te contá.” (\textit{Idem}, p. 47). Enfim,
a invocação do interlocutor abre diversos poemas: as formas mais utilizadas são
“Seu Moço” (\textit{Idem}, pp. 19, 51, 99) e “Seu Dotô” (\textit{Idem}, pp. 60, 66 e 69). Encontram-se
variantes sob a forma de “Meu filho querido” (\textit{Idem}, p. 132), “Meu amigo” (\textit{Idem}, p.
209), “Minha gente” (\textit{Idem}, p. 206), “Sinhô Dotô”
(\textit{Idem}, p. 203).\footnote{Ibidem.}
\end{quote}

Através dessa referencialidade a pessoas e circunstâncias de seu entorno, percebe"-se na poesia de Patativa do Assaré o forte tom familiar e a relação de vizinhança que está implicada em seus poemas, marcas de um autor enraizado ao seu meio:

\begin{quote}
Esses termos de
endereçamento traduzem ao mesmo tempo o respeito de uma hierarquia social
estrita, em uma sociedade onde a taxa de analfabetismo é elevada. O poeta, como
personagem familiar, é originário do mesmo meio, dirigindo-se em pé de igualdade
a seus interlocutores, seja ao mais rico, ao mais poderoso ou ao mais diplomado,
pedindo licença para contar uma história simples à sua maneira -- último
elemento enfim, todavia essencial, o próprio poeta Patativa do Assaré. Não
havendo jamais escrito texto algum e dotado de uma notável capacidade de
memorização (é capaz de recitar qualquer uma de suas composições, qualquer seja
sua antiguidade), ele continua a praticar a improvisação em todas as
circunstâncias.\footnote{Ibidem.}
\end{quote}

Em um poema como “Padre Henrique e o dragão da maldade”, por exemplo, percebe"-se claramente o lastro de Patativa aos acontecimentos contemporâneos do poeta e de sua gente. A obra narra um acontecimento factual, ocorrido em 1969 na cidade do Recife. Um padre foi brutalmente
assassinado pelos agentes da repressão do Governo Federal, que teriam
procurado com a violência atingir outro integrante histórico da política
brasileira, o arcebispo Dom Hélder Câmara. A partir do poema, Patativa do Assaré reflete um acontecimento contemporâneo, referenciado em aspectos do seu tempo histórico, como o período de repressão da Ditadura Militar brasileira.

Outro reflexo da inserção do poeta em seu meio é a quantidade de histórias e lendas que
o poeta vai buscar na tradição popular para compor seus poemas.
É o caso, para ficarmos apenas em um exemplo, do poema “Brosogó, Militão e o diabo”, que pode ser considerado um “causo”, ou seja, uma história protagonizada por personagens típicos e que em geral traz uma lição, uma moral.
No poema temos Brosogó, um simples e decente
vendedor ambulante, que, um dia, entra na casa de um coronel chamado Militão. Sem
conseguir vender nada, quer comprar meia dúzia de ovos de Militão. Como o
coronel não tem troco, diz a Brosogó que leve os ovos e volte depois para
pagar. Brosogó prospera e conquista muitas posses. Um ano e sete meses depois,
volta para pagar a dívida e é enganado por Militão, que faz as contas dizendo
que os ovos teriam sido chocados e que as galinhas nascidas poriam novos ovos e
teriam novas crias, e assim por diante. Brosogó fica desiludido, mas é ajudado
pelo Diabo, a quem teria acendido velas em um dia em que não lembrava de mais
nenhum santo a quem agradecer.


\section{Sobre o gênero}

Para uma primeira definição de poesia enquanto gênero literário, poder"-se"-ia recorrer à definição do professor Domingos Paschoal Cegalla, para quem ``poesia é a linguagem subjetiva, carregada de emoção e sentimento, com ritmo melódico constante, bela e indefinível como o mundo interior do poeta visa a um efeito estético''.\footnote{\textsc{cegalla}, Domingos Paschoal. \textit{Novíssima Gramática da Língua Portuguesa}. São Paulo: Companhia Editora Nacional, 2008, p.\,640}

Aprofundando um pouco essa definição, o crítico Antonio Candido expande a definição de poesia ao diferenciá"-la do verso.
Para o crítico, a poesia enquanto ato criador do artista independe da forma métrica do verso, que passa a ser apenas um dos registros possíveis do poético:

\begin{quote}
A poesia não se confunde necessariamente com o verso, muito menos com o verso metrificado. Pode haver poesia em prosa e poesia em verso livre. [\ldots]
Pode ser feita em verso muita coisa que não é poesia.\footnote{\textsc{candido}, Antonio. \textit{O estudo analítico do poema}. São Paulo: Terceira leitura, 1993, p.\,13--14.}
\end{quote}

Delineada, de forma breve e geral, a forma poética, pode"-se pensar agora em seus três gêneros básicos: lírico, épico e dramático.
Para o crítico Anatol Rosenfeld, a lírica é o gênero mais subjetivo, no qual uma voz central exprime um estado de alma traduzido em orações poéticas.
Seria a expressão de emoções e experiências vividas, ``a plasmação imediata das vivências intensas de um Eu no encontro com o mundo, sem que se interponham eventos distendidos no tempo (como na Épica e na Dramática)''.\footnote{\textsc{rosenfeld}, Anatol. \textit{O teatro épico}. São Paulo: Perspectiva, 2006, p.\,22.}

Devido a essa característica central da lírica, a expressão de um estado emocional, Rosenfeld considera que o eu"-lírico, nesse gênero, não se delineia enquanto um personagem. Embora possa evocar personagens e narrar acontecimentos, a lírica entendida enquanto gênero puro afasta"-se sobremaneira da apreensão objetiva do mundo, que não existe independente da subjetividade intensa que o apreende e exprime. Assim, na lírica prevalece a fusão entre o sujeito e o objeto, que serve mais a realçar os estados profundos de alma do poeta.
Sobre os aspectos formais do gênero, Rosenfeld nota:

\begin{quote}
À intensidade expressiva, à concentração e ao caráter ``'imediato'' do poema lírico, associa"-se, como traço estilístico importante, o uso do ritmo e da musicalidade das palavras e dos versos. De tal modo se realça o valor da aura conotativa do verbo que este muitas vezes chega a ter uma função mais sonora que lógico"-denotativa. A isso se liga a preponderância da voz do presente que indica a ausência de distância, geralmente associada ao pretérito. Este caráter do imediato, que se manifesta na voz do presente, não é, porém, o de uma atualidade que se processa e distende através do tempo (como na Dramática) mas de um momento ``eterno''.\footnote{Ibidem, p.\,23.}
\end{quote}

No caso específico da poesia de cordel, dizem os especialistas, é uma poesia escrita para
ser lida, enquanto o repente ou o desafio é a poesia feita oralmente,
que mais tarde pode ser registrada por escrito. Essa divisão é muito
esquemática. Por exemplo, o cordel, mesmo sendo escrito e impresso para
ser lido, costumava ser lido em voz alta e desfrutado por outros
ouvintes além do leitor. A poesia popular, praticada principalmente no
Nordeste do Brasil, tem muita influência da linguagem oral, aproveita
muito da língua coloquial praticada nas ruas e na comunicação
cotidiana. 

Naturalmente, portanto, pode-se considerar a poesia narrativa do cordel
uma forma de poesia mais compartilhada e desfrutada coletivamente, o
que lhe dá também uma grande ressonância social. Muitos dos temas do
cordel são originários das tradições populares e eruditas da Europa
medieval e moderna. Encontramos temas retirados das novelas de
cavalaria medievais e das narrativas bíblicas. Como no caso de
``Brosogó, Militão e o Diabo''. Ao lado destes temas
mais literários, encontram-se os temas locais, quase sempre narrados na
forma de crônicas de coisas realmente acontecidas, como em
``Eu e o padre Nonato'' e “Padre Henrique e o dragão da maldade”.
E os poemas que misturam"-se com acontecimentos políticos do país
e proclamam uma nova ordem social, caso de ``Carta do padre Antonio Vieira ao Patativa do Assaré'', ``Reforma agrária'', ``Inleição direta 1984'', entre outros.
Também há as histórias fantásticas, que
se valem das tradições semirreligiosas, ligadas à experiência com o
mundo espiritual. 

Os grandes poemas de cordel são perfeitamente metrificados e rimados. A
métrica e a rima são recursos que favorecem a memorização e
tradicionalmente se costuma dizer que são resquícios de uma cultura
oral, na qual toda a tradição e sabedoria são sabidas de cor. 


\subsection{O sertão geográfico e cultural}

O sertão tem mitos culturais próprios. Contemporaneamente, o sertão
evoca principalmente o sofrimento resignado daqueles que padecem a
falta de chuva e de boas safras na lavoura. Evoca a experiência
histórica de uma região empobrecida, embora tenha sido geradora de
riquezas, como o cacau e a cana-de-açúcar, ambos bens muito valiosos. 

O sertão formou também o seu imaginário por meio de grandes
personalidades e uma pujante expressão artística. Além do cordel, o
sertão viu nascer ritmos tão importantes quanto o forró e o baião.
Produziu artistas tão expressivos quanto Luiz Gonzaga, grande cantor da
vida do sertanejo em canções como ``Asa
branca''. Um escultor como Mestre Vitalino criou toda
uma tradição de representação da vida e dos hábitos sertanejos em
miniaturas de barro. A gravura popular, que sempre acompanha os
folhetos de cordel, também floresceu em diversos pontos e ficou mais
famosa em Juazeiro do Norte, no Ceará, e em Caruaru, no estado de
Pernambuco. 

Dentre os grande mitos do sertão, está certamente o do cangaço com seu
líder histórico, mas também mítico, Virgulino Ferreira, o Lampião. Até
hoje as opiniões se dividem: para alguns foi um grande homem, para
outros um bandido impiedoso. Neste volume, o leitor vai encontrar 
um cordel dedicado ao mito do cangaço (``Combate e morte de `Lampião'\,''). 

Uma figura muito presente na cultura nordestina é o Padre Cícero Romão,
considerado beato pela Igreja Católica. Consta que teria feito milagres
e dedicado sua vida aos pobres. 

\subsection{Variação linguística}

A linguística moderna usa o termo
``idioleto'' para marcar grupos
distintos no interior de uma língua. Um idioleto pode ser a fala
peculiar de uma região, de um grupo étnico ou de uma dada profissão. 

Uma das grandes forças da poesia popular do Nordeste se origina em sua
forma muito própria de falar, com um ritmo muito diferente dos falares
do sul, e também muito diferentes entre si, pois percebe-se a diferença
entre os falares de um baiano, um cearense e um pernambucano, por
exemplo.

Além desse aspecto rítmico, quase sempre também há palavras peculiares a
certas regiões. 

\begin{bibliohedra}

\tit{DIEGUES JÚNIOR}, Daniel. \textit{Literatura popular em verso}. Estudos. Belo Horizonte: Itatiaia, 1986. 

\tit{MARCO}, Haurélio. \textit{Breve história da literatura de cordel}. São Paulo: Claridade, 2010.

\tit{TAVARES}, Braulio. \textit{Contando histórias em versos. Poesia e romanceiro popular 
no Brasil}. São Paulo: 34, 2005.

\titidem. \textit{Os martelos de trupizupe}. Natal: Edições Engenho de Arte, 2004


\end{bibliohedra}