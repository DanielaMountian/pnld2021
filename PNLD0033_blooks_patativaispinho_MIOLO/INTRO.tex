\chapter{Patativa do Assaré -- Depoimento}

Eu nasci no dia 5 de março de 1909, no lugar denominado Serra de
Santana, que fica no interior do Estado do Ceará, pertencendo ainda a
região do Cariri. A Serra de Santana esta distante da cidade de Assaré
18km. O meu pai, um pobre agricultor, Pedro Gonçalves da Silva, e mi-
nha mãe, Maria Pereira da Silva. Deste casal nasceram cinco filhos:
José, Antônio, Joaquim, Pedro e Maria. Eu sou o segundo filho, o
Antônio Gonçalves Silva. Quando meu pai morreu, eu fiquei apenas com 9
anos de idade. Meu pai morreu muito moço. E eu, ao lado dos meus irmãos
e da minha mãe, tivemos que enfrentar a vida de pobre agricultor, no
diminuto terreno que meu pai deixou como herança. Na idade de 12 anos
eu freqüentei uma escola lá mesmo no campo, onde vivia e onde ainda
estou vivendo. Nesta escola o professor era muito atrasado, embora muito
bom, muito cuidadoso, mas o coitado não conhecia nem sequer a pontuação.
Eu aprendi apenas a ler, sem ponto de português, sem vírgula, sem ponto,
sem nada, mas como sempre a minha maior distração foi a poesia e a
leitura, quando eu tinha tempo, chegava da roça, ao meio-dia ou à noite,
a minha distração era ler, ler e ouvir outro ler para mim, o meu irmão
mais velho, José. Ele lia sempre os folhetos de cordel e foi daí de
onde surgiu a minha inspiração para fazer poesia. Eu comecei a fazer
verso com 12 anos de idade. E continuei sempre na vida de agricultor e
ali entre meus irmãos e ao lado da minha mãe. Com 16 anos, eu comprei
uma viola e comecei a cantar de improviso. Naquele tempo, 16 anos, eu já improvisava, mesmo glosando, sem ser ao pé
da viola. Comprei a viola e que comecei a cantar também, não fazendo
profissão. Eu cantava assim por esporte, atendendo convite especial,
renovação de santo, casamento que não ia haver dança, também aniversá-
rios de pessoas amigas. O certo que eu só cantava ao som da viola
atendendo convite especial.

Com 20 anos de idade, um primo legítimo da minha mãe, um negociante que
morava no Pará, veio visitar a família que aos 15 anos havia saído do
Assaré, então foi à casa da minha mãe e me ouviu cantar ao som da viola.
Ficou encantado e maravilhado com os meus improvisos e pediu
carinhosamente à minha mãe para que deixasse eu ir com ele ao Pará, que
custearia todas as despesas e ela não tivesse cuidado que eu voltaria
quando quisesse. Então, a minha mãe, muito chorosa, pela amizade e
atenção que tinha ao primo, consentiu que eu fosse. Eu viajei ao Pará,
eu tinha 20 anos naquele tempo. Viajei com tio, chegando lá ele me
apresentou ao escritor Cearense José Carvalho de Brito, autor do livro
``O Matuto Cearense e o Caboclo do Pará'', em cujo volume eu tenho um
capítulo. José Carvalho me recebeu com a maior atenção e me pediu uns
versos para publicar no ``Correio do Ceará''. E ele era redator do
``Correio do Ceará''. Ele colaborava no ``Correio do Ceará''. Então, no
final dos versos, ele faz a apreciação dele, fazendo uma referência
sobre meus versos e disse que a espontaneidade da minha poesia tinha
semelhança, se assemelhava ao canto sonoro da patativa do Nordeste, a
nossa patativa aqui do Ceará. E então o jornal circulou, daquele tempo
para cá, eu já com 20 anos, foi que começaram a me chamar Patativa.
Posso dizer que foi José Carvalho Brito que pôs esse apelido que o povo
hoje conhece, esta alcunha. Patativa do Assaré. Depois começou a surgir
outro Patativa por aí afora também fazendo versos, cantando ao som da
viola, e o povo, para distinguir, quando se falava de uma poesia que
povo gostava, perguntava logo: é o do Patativa do Assaré? Só quero se
for do Patativa do Assaré, sendo do Patativa do Assaré eu quero. Então
começaram a me tratar Patativa do Assaré. E com muito direito, porque
Assaré é a minha terra, a minha cidade. Sim, como eu ia dizendo, lá em
Belém do Pará eu desci para Macapá, onde morava outro primo legítimo
da minha mãe. Lá eu passei um dia, mas achei a vida trancada, uma vida
insípida, uma vida sem distração, eu só podia sair de dentro de uma casa
se levado por outra pessoa, porque lá a gente sai de dentro de casa já é
na canoa, desce da porta não é escada, sai de dentro da canoa e então
vai para outra casa, que é tudo alagado. Então eu não agüentei e
passei apenas dois meses lá. Voltei a Belém do Pará, para casa do outro
tio, daí fui às colônias do Pará, cantei com os cantadores das colônias,
Francisco Chagas, Antônio Merêncio, Rufino Galvão, mas a saudade danada
não me deixou demorar mais no Pará. Passei apenas 5 meses e tantos dias
e voltei ao Ceará.

De volta ao Ceará, José Carvalho de Brito, que era muito amigo de Dra.
Henriqueta Galeno, filha do afamado poeta Juvenal Galeno, me deu uma
carta de recomendação para a Dra. Henriqueta Galeno. Eu, chegando
aqui, entreguei a carta, ela leu e me recebeu no salão de Juvenal
Galeno, como ela sempre recebeu um poeta de classe, um poeta de cultura,
um poeta erudito. Ali fiz alguns improvisos, cantei ao som da viola,
porque eu trazia minha viola. Então eu voltei novamente ao Assaré, 5
meses e tantos dias. Cheguei lá, recomecei a minha vida de roceiro,
sempre trabalhando e sem nunca mais viajar, porém sempre fazendo
versos. E já tinha uma farta bagagem de produções, quando o latinista
José Arrais de Alencar, vindo do Rio de Janeiro, onde ele morava,
visitar a D. Silvinha, a sua mãe, ouviu o programa na rádio Araripe,
onde eu estava recitando verso.
Perguntou de quem era, quem era aquela pessoa que recitava versos tão
dignos de atenção e próprios de divulgação. Aí disseram a ele que era um
caboclo, um roceiro, um agricultor. Então ele mandou me chamar. Eu fui
à presença dele, ele ficou muito satisfeito, recitei muita poesia para
ele, pois a minha bagagem, que dava um volume, eu tinha toda na mente,
toda guardada na memória, e ele perguntou: porque você não publica
essa poesia, esta coisa tão admirável que você tem, tantos versos
próprios de divulgação! Eu respondi: doutor, porque eu não posso, eu sou
pobre, sou roceiro, nem sequer nunca pensei em publicar alguma coisa.
Ele disse: pois você vai publicar o seu livro. Você, eu publico seu
livro e você pagará o impresso com a venda do próprio livro. Então eu
respondi: doutor, e se o livro não tiver sorte, como é que acontece?
Então ele disse: você é um vencido, não tem coragem. E com certeza a sua
honestidade é grande. Ficou com medo de ficar devendo algum di-
nheiro? Não, não acontecerá isso. E se assim acontecer, você não ficará
devendo um vintém a seu ninguém. Você não tá pedindo para ninguém
publicar seu livro. E na presença estava o Dr. Moacir Mota, que era
gerente do Banco do Brasil na cidade do Crato, filho do saudoso Leonardo
Mota, poeta, e se ofereceu para datilografar as minhas produções, sem me
cobrar um vintém. E assim fez. A cópia foi datilografada na cidade de
Crato pelo Dr. Moacir Mota, foi remetida para o Rio de Janeiro, lá o
Dr. José Arrais de Alencar, esse latinista, homem de profundo
conhecimento, publicou meu livro na editora Borçoi e remeteu para o
Banco do Brasil, foi guardado no Banco do Brasil, de onde eu tirava os
volumes e vendia aí pelo Assaré, no meio do meu conhecimento. Fizemos
lançamento também no Crato, o certo é que eu paguei com facilidade o
impresso desse livro. No ano de 66, o mesmo livro foi editado, a 2a
edição, com o aumento de um livrozinho que eu tinha, com o título
\emph{Cantos de Patativa}. \emph{Cantos de Patativa} era um livrozinho que
eu tinha, um livro inédito, com o qual eu ampliei o ``Inspiração
Nordestina'', na sua 2a edição, na mesma editora Borçoi, no Rio de
Janeiro onde estive 4 meses. Lá no Rio de Janeiro, quando saiu a
impressão do meu livro, eu tive um dos prazeres maiores da minha vida. É
que lá, eu sem conhecimento para a venda do meu livro, e o Borçoi, o
dono da editora, publicou fazendo o mesmo negócio, para eu pagar com a
venda do próprio livro, eu dando apenas uma entrada de meu. Aí então eu
sabendo que no Ceará era onde eu poderia vender com facilidade, fui à
presença do Dr. Borçoi e falei para ele: digo, doutor, eu venho aqui
tratar de negócios com o senhor. É que o meu livro aqui eu não posso
vender com facilidade, não tenho conhecimento, sou muito tímido, sou
muito pessimista e eu vou voltar ao Ceará, que lá eu vendo e então
enviarei o dinheiro. Venha para que a gente assine aqui um documento,
uma promissória de tudo e ele respondeu: poeta, tem quatro meses que
você está aqui no Rio, eu já estou lhe conhecendo, já conheço assim, fi-
quei conhecendo a sua índole, a sua honestidade, sua capacidade. Olha,
volta lá para o teu Ceará, com os teus livros, eu apenas te dou este
cartão do banco para onde você vai remeter o dinheiro e pode me pagar
parceladamente e eu estou confiando e sei que recebo o dinheiro. Ora, eu
voltei muito satisfeito dele me confiar e, se eu tinha desejo de pagar
com brevidade, ainda mais me cresceu esse desejo de fazer isso com a
maior facilidade.

No ano de 70, o Prof. J. de Figueiredo Filho publicou um livro, esse
livre eu não posso dizer que ele é meu, porque o comentarista do livro
é o J. de Figueiredo Filho. A poesia é toda minha, mas o livro foi
apresentado por ele, que é: ``O Patativa do Assaré''. Ele mesmo se
explica e diz: o livro é meu? Não, o livro não é meu. O livro é do poeta
Patativa. Eu sou apenas o comentarista do livro, sou apenas
o apresentador. Então o ``Patativa do Assaré'' já foi esgotado, o
``Inspiração Nordestina'' foi também esgotado, eu só tenho publicado os
meus livros por iniciativa dos homens de cultura, como agora mesmo o
``Cante Lá que Eu Canto Cá''. O ``Cante Lá que Eu Canto Cá'' foi
iniciativa do homem de letras, o prof. Plácido Cidade Nuvens.

Ele veio a mim e disse: olhe, vamos publicar o seu livro, o livro, um
novo livro. Eu disse: você pode? Pode, porque a fundação Pe. Ibiapina
está aqui para trabalhar e apresentar aquilo que de melhor tem na região
e eu não vejo outra coisa melhor do que a sua capacidade de fazer
versos, essa sua cultura popular, esse seu pensamento de penetrar em
todos os assuntos sociais e cantar a vida do povo. E nós vamos publicar
o seu livro. Eu mesmo serei o portador para me entender com a Editora
Vozes, faço negócio e então vamos publicar o seu livro. Eu faço isso
não é interesse de você ganhar dinheiro porque o poeta, aqui no Brasil,
ele não ganha dinheiro, mas ele é a riqueza da divulgação. É um
documentário que eu quero deixar aqui na fundação Pe. Ibiapina e esse
documentário ficará não só aqui como em outros lugares. E assim foi, que
a minha vida tem sido assim. Tido isso sem eu deixar meu trabalho na
roça. Eu nunca de mim próprio procurei voluntariamente publicar um
livro. São os apreciadores, os interessados pela cultura popular que me
procuram, pois até mesmo da Inglaterra veio o Dr. Collin à minha casa,
passou aí 3 dias, conversou muito comigo, é um escritor que já escreveu,
já tem livros publicados, como ele tem um livro de título Gente da
Gente, que é sobre os índios da Guiana Inglesa. Recebi uma carta desse
escritor, Dr. Collin, lá de Londres, pedindo licença para traduzir o meu
livro ``Cante Lá que Eu Canto Cá'' em língua inglesa. Eu disse a ele que
sim. E ele disse: olha, Patativa, o apresentador e o tradutor que sou
eu, não quero ganhar um centavo nesse trabalho. Será todo seu.
E você querendo poderá oferecer à Fundação Pe. Ibiapina. Pois bem, isto
aqui é uma história, é uma parte da história da minha vida. Tudo isso eu
tenho feito sem deixar o meu trabalho de roça lá na Serra de Santana,
lugar onde eu nasci, tenho vivido e hei de viver o resto da minha
vida, porque nunca me habituei à vida da cidade, sempre o meu mundo
foi a minha poesia e a minha família e aonde eu quero viver o resto da
minha vida. E ali mesmo eu hei de morrer, se Deus quiser, um dia feliz.

Voltando do Pará e demorando uns dias aqui em Fortaleza, fui parar no
Assaré. Lá recomecei minha vida de agricultor, nos meus 21 anos de
idade. Quando cheguei aos 25 anos de idade eu casei com uma serrana, uma
rapadeira de mandioca, uma cabocla que eu já conhecia desde menina,
que é a Belarmina Gonçalves Cidrão, conhecida por Belinha. Aí comecei,
recomecei, continuei a minha vida de casado, me senti muito feliz e essa
felicidade ainda hoje continua. Sou pai de 7 filhos, 4 homens e 3
mulheres: Afonso, Pedro, Geraldo, João Batista, Lúcia, Inês e Miriam.
É esta a minha família, é esse meu mundo que me sinto feliz e vivo entre
eles e é por isso que eu quero estar semprte na Serra de Santana, pois é
onde está toda essa minha família, todos continuando na vida do velho
Patativa, tratando da agricultura, naquela vida pobre do camponês, a
minha espossa muito paciente, muito trabalhadora, muito carinhosa e
graças a Deus já estou com 70 anos, mas minha felicidade sempre
continua, porque a felicidade para mim não é possuir dinheiro, não é ser
um fazendeiro, não é esse estado financeiro muito fraco. A felicidade
consiste em a pessoa viver dentro da harmonia com todos e principalmente
com seus familiares. E é por isso que me sinto muito feliz.

Em 1973, sendo convidado aqui para o sesquicentenário de Fortaleza, no
mês de agosto, tive a infelicidade de ser acidentado. Ia atravessando a
Av. Duques de Caxias, fui colhido por um carro e quando recobrei os sentidos eu já estava em cima
da cama de operação, no hospital, e então foi uma infelicidade para mim.
Bota gesso, tira gesso, e ali passei 11 meses e não recuperei. Então
resolvi ir ao Rio de Janeiro, pois eu tenho parentes e amigos. Dr. Mário
Dias Alencar, que é meu parente e é filho de Assaré, mandou me buscar
para o Rio de Janeiro. Ele não é ortopedista, ele é operador de outras
coisas, viu? Mas me pôs lá no Hospital
S. Francisco de Assis, onde um professor de ortopedia operou minha
perna, pelejou, ainda houve duas operações, mas lá já cheguei retardado
e fui obrigado a pôr um aparelho ortopédico, com o auxílio do mesmo é
que eu vivo me locomovendo e ando, vou por onde quero. Queriam amputar
minha perna, mas eu me danei, não deixei. Não deixei, não, eu não queria
minha perna cortada, não. E o médico teimou comigo e disse: você não vai
agüentar que é um aparelho ortopédico, dói muito e talvez até ainda tire
ele para mandar amputar a perna. Então fiz a seguinte pergunta:
doutor, há perigo de infeccionar? A perna vai infeccionar com esse
aparelho ortopédico? Ele disse: não, não infecciona, não. Dói é muito
pra que você possa se acostumar. Eu digo: ah, doutor, pois eu já sei
que me acostumo. Eu sou é cabeça do mato, acostumado a levar pancada de
pau quando estou brocado, coice de animais, quanta coisa tem. Eu já tô
acostumado com o embate da vida. Aí então botou o aparelho ortopédico,
doeu por mim, parente, amigo, o diabo a sete, mas me acostumei e hoje
estou andando para onde quero, embora com dificuldade, mas não me dói.
Em compensação eu não relembro aquele verso, mas pelo menos meu acidente
foi no dia 13 de agosto e eu não tenho superstição. E o povo sempre
comentava: mas Patativa, além de ser no mês de agosto, ainda mais no dia
13. Você foi muito feliz, porque este mês, não sei o quê\ldots{} Então fiz
este soneto:

\begin{verse}
Foi a 13 de agosto que um transporte\\
Me colheu quebrou a minha perna\\
E ainda hoje padeço o duro corte\\
Que me aflige, me atrasa e me consterna.

Diz alguém que esta data é quem governa\\
Os desastres, nos dando triste sorte\\
Apesar da ciência tão moderna\\
Nossa estrela se apaga e não tem norte.

Mesmo sofrendo a minha sorte crua,\\
Não direi nunca que esta culpa é tua\\
13 de agosto de 73.

Porém, tratado com desdém será\\
E a classe ingênua não perdoará\\
Porque te chama de agourento mês.
\end{verse}

Eu sou um caboclo roceiro que, como poeta, canto sempre a vida do
povo. O meu problema é cantar a vida do povo, o sofrimento do meu
Nordeste, principalmente daqueles que não têm terra, porque o ano
presente, esse ano que está se findando, não foi uma seca, podemos dizer
que não foi a seca. Lá pelo interior, mesmo no município de Assaré, lá
no Assaré, tem duas frentes de serviço, com muita gente. Mas naquela
frente de serviço nós podemos observar que é só dos desgraçados que
não possuem terra. Os camponeses que possuem terra não sofrem estas
conseqüências e não precisam recorrer ao trabalho de emergência, como
os agregados e esses outros desgraçados trabalham na terra dos patrões.
E é isso que eu mais sinto: é ver um homem que tanto trabalha, pai de
família e não possui um palmo de terra. É por isso que é preciso que
haja um meio da reforma agrária chegar, uma reforma agrária que chegue para o povo que não tem terra. Por isso eu digo neste meu soneto
``Reforma Agrária'':

\begin{verse}
Pobre agregado, força de gigante,\\
escuta, amigo, o que te digo agora.\\
Depois da treva vem a linda aurora\\
e a tua estrela surgirá brilhante.

Pensando em ti eu vivo a todo instante\\
minh'alma triste, desolada chora\\
quando eu te vejo pelo mundo afora\\
vagando incerto qual judeu errante.

Para saíres da fatal fadiga\\
do invisível jugo que cruel te obriga\\
a padecer a situação precária,

lutai altivo, corajoso e esperto\\
pois só verás o teu país liberto\\
se conseguires a reforma agrária.
\end{verse}

E esta luta pela reforma agrária e pelo sindicato dos camponeses, mas o
verdadeiro sindicato conduzido pelos próprios camponeses, procurando,
reivindicando os seus direitos, é preciso que continue até chegar o
tempo do camponês sofrer menos do que vem sofrendo. Precisa fazer como
eu digo nos meus versos ``Lição do Pinto'', pois o pinto sai do ovo
porque trabalha. Ele belisca a casca do ovo, rompe e sai. É assim que o
povo também deve fazer, unido sempre, trabalhando.

\hfill{}Depoimento concedido a Rosemberg

\hfill{}Cariry, no Crato, em 1979.