\mbox{}
\thispagestyle{empty}
\vfill
\epigraph{Fica saber: o chão deste mundo é o teto de um mundo mais por baixo.}{\textsc{mia couto}}

\chapter{Prefácio}

\begin{flushright}
Rafael Haddock-Lobo
\end{flushright}

O extrapolar dos muros erguidos pela intelectualidade acadêmica não é
apenas uma forma de a filosofia saltar rumo às ruas e, com isso, ganhar
vida; é também um convite feito às ruas para adentrarem o ambiente
acadêmico e trazer toda sorte de contaminações que são fundamentais para
uma experiência de pensamento. É assim que espectros e odores vagueiam e
nos levam a passear, com o livro do Claudio, pelas ruas do Rio de
Janeiro.

A escrita, que é poética, melódica e desenvolve-se em três partes (são
três estados de espírito, marcados por três diferentes lugares de
escrita e diferentes amores), num crescendo que parece orquestrado,
constrói uma experiência olfativa como raramente se vê em filosofia
(sempre marcada pelo privilégio da visão e da audição).

Essa escrita de e com os narizes, como uma indicação da necessidade do
faro para o pensamento, nos leva a percorrer, com Claudio e com seus
fantasmas, as ruas, vielas, cortiços, quilombos e questões filosóficas
---- que nos encaminham desde um problema sobre as condições de
possibilidade (ou sobre os \textit{a priori} históricos) para
compreendermos a medicina social na história da nossa cidade, o Rio de
Janeiro, até o indizível e o inefável, através do qual podemos ver que a
escrita que aqui se tece é, também, uma escrita de si; e que, portanto,
essa jornada que Claudio empreende com a ajuda de Michel Foucault, Luiz
Antonio Simas, Luiz Rufino, cabocla do Castelo e Pai Manoel é também um
encontro de sua voz filosófica.

Preciso ainda ressaltar que, neste livro, os filósofos são postos
(necessariamente) como coadjuvantes. Seus personagens principais são os
``objetos'' de reflexão, que, com tal protagonismo, abandonam o estatuto
de objetos, tornando-se eles ---- os doentes, os moradores dos cortiços,
os corpos putrefatos, os vadios, os curandeiros ----, através de Claudio,
os sujeitos escritores do livro. Além disso, a tese que conduz o leitor,
apesar de historicamente delimitada, é atualíssima e mostra que tudo o
que se disse sobre os cortiços permanece o mesmo com relação às favelas,
e que o mesmo higienismo contra o preto e o pobre apresenta-se tanto na
violência dos discursos de elite como na ``pobreza limpinha'' dos
subúrbios: ambos desejam se distanciar da vagabundagem ---- fato este que,
aliás, faz com que certa parte da população constantemente eleja membros
da elite como seus ``representantes''!

Em tempos pandêmicos como o que atualmente vivemos ---- uma exceção que
não é exceção, pois sempre foi norma de Estado ---- é possível ver de modo
bem mais claro a ``estrutura'' do higienismo, que foi base das diversas
tentativas de erigir uma Nação: quando constatamos, por exemplo, o
aumento da violência policial, das mortes de cidadãos pretos e pardos,
dos ataques às favelas; a supressão de informações sobre a violência
policial;a subnotificação das mortes e contaminações pelo Covid-19; o
incentivo ao uso de medicamentos não comprovados e que podem aumentar
ainda mais o número de mortes de pessoas em situação vulnerável\ldots{}
Tudo isso apenas mostra de modo hiperbólico a necessidade de o Poder não
apenas decidir quem deve viver e quem deve morrer mas, como Claudio
explora neste livro, empreender uma política de morte direcionada a
certos indivíduos.

Por isso, mais ainda, o que o texto de Claudio nos aponta é a urgência
de, contra o atual modelo de democracia representativa, pensarmos uma
política dos bandos, herdeira dos quilombismos, na qual todos esses
excluídos fedorentos (próximos do que Preciado chama de multidões
``\textit{queer}'', no texto homônimo),\footnote{Beatriz Preciado,
  ``Multidões \textit{queer}: notas para uma política dos `anormais'\,'',
  \textit{Revista Estudos Feministas}, vol.~19, n.~1, 2011, p. 11-20.}
essa aglomeração tumultuosa que des-ordena a ordem estabelecida, vadios,
viados, pretos, pobres, macumbeiros, travestis e todos aqueles que as
narinas ricas e embranquecidas não suportam, que todos esses devemos nos
juntar com o intuito de marcar outro tempo.

Este livro é uma obra fundamental para se compreender o momento em que
estamos, pelo menos a maioria de nós, buscando respeitar uma quarentena
e compreender que ferramentas temos para pensar a pandemia. Mas não
apenas: é um livro \textit{para a pandemia e além}. Pois apresenta uma das
poucas políticas positivas e possíveis de resistência fora da esfera da
colonialidade.

Essa escrita ---- que é tanto espiritual como política ---- é, portanto,
também uma oferenda: malandros, povo de rua, marinheiros, pretos
curandeiros, caboclos são aqui exaltados. Este livro faz parte do que
tenho chamado de ``giro macumbístico'' da filosofia brasileira ---- uma
filosofia que, nas palavras de Simas e Rufino, precisa empreender a
terreirização dos territórios filosóficos por meio de uma epistemologia
e de uma política macumbeiras.

Como um golpe de duas faces, como o Oxé de Aganju, que traz sua justiça
entre trovões e erupções vulcânicas, \textit{Mármore e Barbárie} atinge ao
mesmo tempo o objetivo de empreender uma descrição precisa dos
acontecimentos históricos e o de abrir novos campos de batalha e de
mandinga, mostrando que a escrita filosófica, como poesia e feitiço,
luta ao escrever e escreve ao lutar. Uma escrita, portanto, que marca e
é marcada pela espectralidade à qual o autor tanto deseja fazer justiça
e que poderia ser um manifesto em favor disso tudo que não se vê, pelo
apagamento, mas que se sente, que se cheira e que, justamente por isso,
precisa ser escrito.

Por isso, não posso deixar de rememorar o dia e o momento da defesa da
tese que deu origem a este livro e que aconteceu no dia 13 de maio de
2019. Sim! O livro é resultado de uma tese em filosofia, que trata de
doença, de cura, de liberdade, e que foi defendida no mentiroso dia da
assim chamada abolição da escravidão. Salvando Pai Cipriano, Pai Manoel
e todas as linhas de cura, a defesa da tese se fez marcar pela lembrança
de que, apesar da mentira encenada pela Princesa Isabel, a umbanda elege
esse dia para celebrar os pretos velhos.

Como memória da Revolta das Carrancas ---- ocorrida em 13 de maio de 1833
e que antecede em mais de 50 anos a assinatura da lei áurea ----, o dia da
defesa pública deste trabalho fez lembrar que a luta é ainda tão
necessária quanto nos séculos passados e que pretos e pretas velhas nos
lembram disso quando entoam seus pontos, para um assombro nosso que deve
ser cotidiano. É por essa razão que ``Vovó não quer casca de coco no
terreiro\dots{} Casca de coco faz lembrar dos tempos do cativeiro''.

As cantigas que firmam ponto nos terreiros da umbanda carioca são marcas
dessa história, que é de ontem e de hoje. Marcas da violência policial
contra pretos e pobres que, antes ou depois da abolição da escravidão,
são cantadas por Pai Joaquim de Angola:

\begin{verse}
No dia treze de maio tava tocando meu tambor\\
Pai Joaquim estava dançando quando a polícia chegou.\\
Entra preto, branco não entra, se entrar pau vai levar\\
Esse nêgo é meu, vem cá! Esse nêgo é meu, vem cá!\\
\end{verse}

No dia 13 de maio, no dia da abolição, a polícia entra na festa dos
pretos para resgatar um escravizado fugido. Isso foi antes da assinatura
da lei? Isso é o que provoca a revolta das carrancas? Isso foi ontem em
algum canavial?

Só sabemos que esse problema é a razão de nossa luta ---- política,
intelectual, epistemológica ----, e que, em torno deste livro, unidos
nessas linhas de mandinga e de batalha, precisamos fazer da filosofia um
lugar de resistência. Creio ser esta a tarefa filosófica deste livro,
tarefa à qual me junto a fim de firmarmos um terreno no qual tantas
outras vozes possam ser ouvidas, outras histórias possam ser contadas e
outros cheiros possam ser experimentados, pois, ao contrário do que
conta nossa história oficial, ``O preto velho é um nego feiticeiro. Se
não fosse o preto velho, não acabava o cativeiro''.

\chapter{Apresentação}\label{apresentauxe7uxe3o}

No ano de 1876, o senador e ex-diretor da Faculdade de Medicina do Rio
de Janeiro, Dr.~José Martins da Cruz Jobim, escreve aos
jornais\footnote{José Martins da Cruz Jobim, ``A febre amarela e o
  Sr.~Barão de Lavradio'', \textit{Diário do Rio de Janeiro. Rio de
  Janeiro}, 8 de Abril de 1876, p.~2.} condenando as medidas de saúde
pública adotadas pela Junta Central de Higiene Pública, então presidida
pelo Dr.~Pereira Rego, o Barão de Lavradio. Jobim cita as publicações do
higienista francês Motard, especialmente o \textit{Tratado de Higiene
Geral}.\footnote{Adolphe Motard pensa o contágio através do que ele
  denomina ``teoria dos vírus patológicos'', segundo a qual,Cf.. quando
  um órgão é portador de uma matéria virulenta, o vírus, à maneira das
  plantas, germina uma grande quantidade de vírus. Esse vírus é então
  transmitido, pelas vias excretórias, a um organismo são, causando o
  contágio. Ver Adolphe Motard, \textit{Traité d'Hygiène Générale ---- Tome
  Second} (Paris, J. B. Baillière et Fils, 1868), p.~529.} Julga
inflados os contos gastos em limpezas e pântanos, segundo a suposição
barata de querer combater ``moinhos de vento''. Que ``extravagância é
essa de dizer que o contágio vem direto das lamas e imundícies para de
lá reverberar vigoroso e entrar nos corpos humanos e matá-los?'' Ora,
nos ``cortiços não bastará que estejam muitas pessoas juntas e mal
arejadas para lhes entrar o mal pela respiração''? Poucos meses depois a
Câmara decide não conceder mais licenças para a construção de cortiços,
``casinhas ou (\ldots{}) nomes equivalentes, no perímetro da
cidade''\footnote{Rio de Janeiro, Câmara Municipal, \textit{Código de
  Posturas da Ilustríssima Câmara Municipal do Rio de Janeiro e Editais
  da mesma Cidade}, Posturas de 1º de Setembro de 1876.} do Rio de
Janeiro. E nos projetos urbanísticos da Comissão de Melhoramentos da
Cidade --- que elaborará dois relatórios (1875-6) que inspirarão em
alguma medida a execução da reforma Passos na primeira década do \textsc{xx} ---
constarão, além do dessecamento de pântanos e terrenos alagadiços, um
alargamento de ruas que exigiria a retificação de quarteirões ocupados
pela população pobre, como resposta ao perigo da concentração de
estalagens e cortiços no eixo e nas imediações da cidade: ``A principal
causa da insalubridade das casas em nosso país'', dizem os engenheiros
da Comissão, ``reside no péssimo sistema de sua distribuição interna
(\dots{}), são as nossas habitações desprovidas dos meios de ventilação e de
renovação de ar nos quartos de dormir''.\footnote{Sonia Gomes Pereira,
  \textit{A Reforma Urbana de Pereira Passos e a Construção da Identidade
  Carioca} (Rio de Janeiro, UFRJ-EBA, 1998), p.~135.}

O referido \textit{Tratado de Higiene Geral}, de Adolphe Motard, foi
publicado em 1868. Ele é uma voz entre um conjunto de livros e artigos
científicos dedicados a um tema profícuo, no Brasil, sobretudo a partir
da segunda metade do \textsc{\textsc{xix}}: a Higiene. Motard nos oferece, além de lições
de profilaxia, anatomia e fisiologia, um amplo arcabouço médico-teórico
dedicado ao clima, à geografia médica e física em geral, à nutrição, às
habitações, aos banhos, roupas e ginástica, às condições higiênicas
atribuídas a diferentes formas de trabalho, à \textit{higiene das
necessidades morais} --- ou seja, às consequências higiênicas boas e más
resultantes das principais instituições sociais, o casamento, o
celibato, a educação, os alienados, a prisão etc. Árdua tarefa de fazer
repassar uma a uma todas as funções humanas para reportá-las, segundo
seu autor, às determinações higiênicas que as modificam.

Assim, a finalidade da higiene geral deve ser a satisfação das
necessidades físicas e morais, na medida de o que melhor convém ao
desenvolvimento individual e social.\footnote{Cf. Adolphe Motard,
  \textit{Traité d'Hygiène Générale --- Tome Premier} (Paris, J. B.
  Baillière et Fils, 1868), p.~4.} A Higiene, segundo o Motard, um
código que a ``natureza do homem'' revelou pela decisão do instinto,
chegará enfim à sua etapa filosófica. Ocupando aí um posto entre as
ciências positivas, transcenderá os ramos tradicionais da medicina e,
impulsionado pelo entusiasmo do higienista, consagrar-se-á a si própria
uma Moral: Mais hábil do que as ciências físicas, ela discerniu antes de
tudo nossas condições de existência; mais poderosa do que as
legislaturas nacionais, ela impõe leis que, no limite, faz gravar
diariamente pelos infortúnios da doença, da morte, da iminência da
finitude.

\begin{quote}
Não há nenhuma moral que nos ensine tão bem a sabedoria e, se é verdade
que é ela quem melhor nos concede a expressão da felicidade através da
virtude, podemos nos perguntar se a moral difere da higiene. Porque as
leis mais constantes, as mais úteis, as mais apropriadas a todas as
necessidades da nossa civilização devem ser extraídas da higiene geral.
Não é a experiência do momento que é preciso consultar, mas a
experiência de todos os séculos; as condições de existência e da
felicidade de um homem serão melhor deduzidas consultando as condições
de existência e da felicidade de um povo. Com efeito, os princípios
higiênicos do bem-estar físico e moral do homem resultam de uma série de
modificadores gerais, cujas influências individuais não podem ser tão
percebidas senão pela comparação de sociedades humanas observadas em
todos os tempos e em todos os lugares.\footnote{\textit{Ibidem}, p.~3.
  Tradução nossa.}
\end{quote}

\asterisc

O presente livro, \textit{Mármore e Barbárie}, trata antes de tudo do
corpo --- do corpo como ``uma grande razão, uma multiplicidade com um só
sentido, uma guerra e uma paz, um rebanho e um pastor''.\footnote{Friedrich
  Nietzsche, \textit{Assim falou Zaratustra: um livro para todos e para
  ninguém} {[}1883{]}, trad. Mário da Silva, 13ª Ed. (Rio de Janeiro,
  Civilização Brasileira, 2005), p.~60.} Trata do corpo enviesado pela
vontade de verdade de uma certa medicina, do corpo instaurado como
objeto em campos móveis de correlação de força; do corpo-higiênico como
produto de dispositivos de conjunto bem determinados; corpo tributário
daqueles que, como Motard, velaram pelo bem-estar físico e moral de um
certo conceito de humanidade. Por fim, o corpo da cidade imperial, se
por ela entendemos um meio histórico que a partir de um momento tal
passou a ser integrado por uma polivalência tática de práticas e
discursos de ordem médico-higienista.

Como se fez do corpo alvo de um saber-poder normativo que passa uma a
uma as funções humanas, para reportá-las aos valores das influências
higiênicas aí dispostas? Até onde se fez sentir os efeitos de um
dispositivo que reclama autoridade sobre a problematização, a
objetivação e a terapêutica dos comportamentos ou condutas em relação às
suas formas difusas de habitar, de ocupar as ruas, às suas formas de
amar, à sua cultura do asseio etc.? Ademais, o que possibilitou a um
conjunto de instituições públicas elaborar um saber científico --- muitas
vezes cindido por embates multiformes --- e um domínio de ação --- nem
sempre cerceado exclusivamente pelo poder médico oficial --- que
organizou táticas de governo e intervenção na forçosa materialidade
histórica da cidade de São Sebastião? As condições históricas que
determinaram o aparecimento de algo como um dispositivo
médico-higienista no Brasil não foram caracterizadas pelo linear
aperfeiçoamento do espírito humano, também não foram efeito natural da
engenhosidade e progresso acumulativo do conhecimento científico. Como
não escorregarmos no aconchego da linearidade histórica? A Corte
Imperial foi alvo de catástrofes epidêmicas mais ou menos notáveis ao
longo do século \textsc{xix}. Essas marcações históricas, que identificamos nas
epidemias de febre amarela excepcionalmente graves nas décadas de 1850 e
1870, definiram descontinuidades em termos de funcionamento do poder e
descontinuidades em termos de políticas de saúde adotadas. Definiram,
mais profundamente, transformações táticas acopladas à constituição de
objetividades, se por objetividades não entendermos somente a abertura
de um domínio de visibilidades a ser conhecido, mas o regime de silêncio
que jaz aí ensombrecido. Para cada campo de visibilidade alocado, um
jogo de luz e sombra a contornar o escuso, o marginal, a barbárie, o
negro, o imigrante, o outro.

Uma cidade, uma habitação, um corpo, cada elemento dá testemunho de o
que efetivamente se é pela hierarquia dos valores em que se dispõem os
impulsos íntimos da sua história. Um corpo é uma encruzilhada histórica
de relações de força. O corpo ademais objetivado pelo dispositivo
médico-higienista compreende, filosoficamente, a qual real ou imaginário
deve estar condicionado um sujeito para se tornar agente legítimo deste
ou daquele tipo de conhecimento. Do desenvolvimento mútuo e das relações
de troca entre procedimentos de objetivação e subjetivação se origina o
que podemos chamar de \textit{jogos} ou \textit{regimes de verdade}: ou
seja, não a descoberta do verdadeiro pela síntese de contraditórios, mas
as regras segundo as quais foi possível submeter legitimamente o real à
demarcação do verdadeiro e do falso. Regras, reitero, que não se
presentificam em um conjunto uníssono de representações implicadas em um
paradigma científico. O que elas fazem é permitir que se deixe subsistir
uma multiplicidade de discursos sobre procedimentos de prevenção ou
terapêutica da epidemia, cada qual empenhado em sua polêmica
singularidade, cada qual investido nas correlações de força que animam
os jogos.

\asterisc

Gostaríamos que este livro funcionasse tanto como uma história da
percepção da cidade colonial como fenômeno patológico quanto como uma
genealogia do corpo objetivado pelo dispositivo médico-higienista --- o
que encontramos na superfície do texto é uma narrativa desse momento em
que a epidemia de febre amarela se dispôs como experiência possível para
um saber-poder médico. Interessa-nos a abertura de um campo de
experiência em que sujeito e objeto são ambos constituídos
simultaneamente, mas que, por sua vez, não cessam, eles próprios, de
modificar o campo de experiência referido. Justificamo-nos com a
hipótese de que a história do dispositivo médico-higienista se confunde
com a própria história da experiência da epidemia em nossa sociedade,
uma vez que as descontinuidades que caracterizam as alternâncias táticas
do dispositivo são as descontinuidades que marcam a experiência da
epidemia.

Como passamos da peste como constituição meteorológica da enfermidade
para a experiência da epidemia como efeito da insalubridade pública? De
que forma a condenação do estado sanitário da cidade é acrescido pela
tese de que as habitações coletivas são imorais e antiestéticas? Em que
medida o aburguesamento da rua no entresséculos passou por uma
condenação dos hábitos coloniais, empreendendo não só a vertiginosa
remodelação do perímetro urbano do Rio de Janeiro, mas, sobretudo, uma
normatização higienista do corpo? É mais ou menos o \textit{a priori}
histórico de cada uma dessas três questões que conduzem a ordem dos
capítulos do livro.


\chapter{A febre amarela em 1850: do corpo-microcosmo ao cadáver como
questão de saúde
pública}
\hedramarkboth{A febre amarela em 1850}{}

Antes de 1832 --- ano de fundação, sob influência de médicos franceses
que residiam na Corte, de Escolas e Faculdades de Medicina em Rio e
Bahia ---, a prática da medicina, quando existia, era delegada a físicos
portugueses ou brasileiros formados em Coimbra. De modo que, enquanto os
fenômenos pestilenciais reinavam no nordeste açucareiro nos séculos \textsc{xvii}
ou \textsc{xviii}, era esta medicina, de raízes neo-hipocráticas, a convidada a
prestar socorros. Interessa-nos, neste primeiro momento, o estatuto do
corpo adoecido para esse tipo de medicina. Mas interessa, sobretudo, o
regime de verdade que sustenta a necessária simpatia metafísica entre
castigo da peste e conjunção dos astros.

Na segunda parte, discutiremos a primeira epidemia de febre amarela na
Corte do Rio de Janeiro (1849-1850), que causou, em uma população de
266.000 habitantes, 90.658 amarelentos e ceifou 4.160 vidas. Na ocasião,
o Império se apressou em criar instituições de saúde particularmente
responsáveis pelo combate às epidemias. A partir de então, os
higienistas já não mais seriam convocados a agir apenas em caráter
excepcional. O campo de experiência da febre amarela, a partir da grande
epidemia de 1850, estaria em vias de se institucionalizar --- o que
significa que as estratégias de poder passariam a ser exercidas por um
agenciamento regular do estado sanitário voltadas a minimizar efeitos
desencadeados pelo surto epidêmico. No curso da epidemia de 1850, o
território passava a assumir momentaneamente a constituição epidêmica:
era a cidade na sua territorialidade que carecia de ser sangrada. Porém,
o efetivo processo de despatologização ou desinfecção da cidade
careceria ainda de esperar algumas décadas para entrar em marcha. Por
ora, o dispositivo médico-higienista atuará em regime mais combativo do
que propriamente preventivo. Proliferarão técnicas de quarentena --- o
estado de emergência ---, entendidas como procedimentos francamente
adequados a uma experiência da epidemia associada ao perfil sanitário
que um território assume momentaneamente por ocasião da grande
mortandade --- o que pôde oferecer motivo suficiente para que a
``essência perniciosa'' se instalasse. Paralelamente, o dispositivo
médico-higienista procurará estabelecer de forma pioneira o controle
sobre uma nova cultura fúnebre, primeiro ao transformar o cadáver em
fenômeno repugnante e a morte, consequentemente, em fenômeno insólito.
Em seguida, reorganizando consideravelmente a repartição urbana dos
espaços de isolamento e quadriculamento da morte.

\section*{O corpo-microcosmo e a medicina das
semelhanças}

\epigraph{Octavio paz escreveu: ``A palavra pão,
tocada pela palavra dia, torna-se efetivamente um astro; e o
sol, por sua vez, torna-se alimento luminoso.''
Paul de Man escreveu: ``Ninguém em seu perfeito juízo
ficará à espera de que as uvas em sua videira amadureçam sob 
a luminosidade da palavra dia.''}{\textit{O livro das semelhanças}, \textsc{ana martins marques}}

Conta-se que no ano de 1685 certa moléstia epidêmica grassou o Recife e
logo se foi ateando no povo, passou à cidade de Olinda e seu recôncavo,
continuou ``com alguma pausa, mas com tal intenção e força que era o
mesmo adoecer que em breves dias acabar, lançando pela boca copioso
sangue''.\footnote{Cândido Barata Ribeiro, \textit{Quais as medidas
  sanitárias que devem ser aconselhadas para impedir o desenvolvimento e
  propagação da febre amarela na cidade do Rio de Janeiro?}, tese
  apresentada à Faculdade de Medicina do Rio de Janeiro como primeira
  prova de concurso de Lente Substituto a um lugar vago, na seção de
  Ciências Médicas (Rio de Janeiro, Typographia do Direito, 1877),
  p.~13.} Era febre ordinariamente do gênero dos sinocos podres,
epidêmica --- descreve Joaquim Ferreira da Rosa, então físico da
Capitania: ``pois tem causado tal mortandade, que em seus princípios
quase o deixou deserto, morrendo com muita brevidade; (\ldots{}) ainda
que o número de mortos neste Recife neste contágio não passe muito de
duas mil pessoas''.\footnote{Joaquem Ferreira da Rosa, \textit{Tratado
  único da constituição pestilencial de Pernambuco} (Lisboa, Oficina de
  Miguel Manescal, 1694), p.~4-5.} Alcançou a Bahia, iniciando seu
rastro de defuntos em abril de 1686, e fez durar o efeito dos venenos
influxos até 1695. Em Pernambuco, ceifou discriminadamente o governador
da Capitania, Fernão Cabral (descendente de Álvares Cabral) e um seu
filho. Não menos mortífera foi a manifestação na Bahia, sede do governo
geral: segundo Rocha Pitta, em sua \textit{História da América
Portuguesa}, os primeiros feridos do achaque foram dois homens que
``jantando em casa de uma mulher meretriz, morreram em 24 horas; caso
que a fez ausentar, por se lhe arguir que em um prato de mel lhes
disfarçara o azíbar do veneno; mas pelos sintomas e sinais, com foi
ferindo o \textit{contágio}, se conheceu que dele faleceram''.\footnote{Sebastião
  da Rocha Pitta, \textit{História da America Portugueza} {[}1730{]},
  citado em Cândido Barata Ribeiro, \textit{Quais as medidas
  sanitárias\ldots{}}, 1877\textit{,} p.~13.} Logo mais cinco
desembargadores, um tenente-general, o arcebispo do Brasil, além do
governador geral Matias da Cunha. Dentre os jesuítas mais de
cem.\footnote{Cf. Licurgo Santos Filho, \textit{História geral da medicina
  brasileira} (São Paulo, HUCITEC / EDUSP, 1977), p.~172.} ``Foram logo
adoecendo e acabando tantas pessoas, que se contavam os mortos pelos
enfermos. Houve dia em que caíram duzentos, e não escaparam
dois''.\footnote{Cândido Barata Ribeiro, \textit{Quais as medidas
  sanitárias\ldots{}}, 1877, p.~13.} E em tão grande miséria e
consternação de espírito que destoou do aspecto afetivo da cidade, do
simpático telhado de quatro águas e das treliças portuguesas típicas,
porque uma vez abatidos os pastores da lei, também a redentora
autoridade das mesmas, das divinas como das humanas. Em alguns o
sentimento do desamparo foi tanto mais eficiente que, comovidos no
argumento de que nenhuma medicina traria remédio, abandonaram a cidade,
homens e mulheres, largavam parentes e animais e tramavam nos sertões ou
lugar alheio à pestilência. Assim já havia descrito Giovanni Boccaccio
em 1348, quando testemunhou o comportamento da praga em Florença: ``como
se a cólera de Deus estivesse destinada não a castigar a iniquidade dos
homens com aquela peste, onde eles estivessem, e sim a oprimir,
comovido, somente os que teimassem em ficar dentro dos muros de sua
cidade''.\footnote{Giovanni Boccacio, \textit{Decamerão} {[}1353{]}, trad.
  Torrieri Guimarães (São Paulo, Abril Cultural, 1981), p.~14.} Outros
não se precipitavam pelos arrabaldes de travessas acidentadas, tampouco
abriam as varandas ou metiam os narizes pelas brechas das gelosias, o
que faziam (ao menos nas casas) era acender incensos ou ter em mãos
ervas aromáticas, ``como rosas, sândalos, tragacanto, benjoim,
rosmaninho, alecrim, estoraque, mirra, almíscar, como ensina Zacuto
Lusitano: porque com estas coisas se faz o ar mais puro, e
cheiroso''.\footnote{Joaquem Ferreira da Rosa, \textit{Tratado
  único\ldots{}}, 1694, p.~36-7.} Isso quando não traziam às narinas
exaustivamente aquelas flores ou raras especiarias, como se estimassem
ser ótima coisa conformar o cérebro com perfumes para amortecer o vapor
dos defuntos ou o ar impregnado de vinagre.

Ao que tudo indica, no curso da década em que perduraram, os sinais da
constituição pestilencial foram infiltrando continente adentro --- é o
que traz Odair Franco, em sua \textit{História da febre amarela no
Brasil}, de 1969. Na velha igreja de São Cosme e São Damião, em
Igarassu-PE, o historiador encontrou um painel com uma legenda que
guarda a memória do socorro dos padroeiros:

\begin{quote}
Um dos especiais favores que tem recebido esta freguesia de Igarassu dos
seus padroeiros São Cosme e São Damião, foi o de a defenderem da peste a
que chamaram males que infestaram a todo Pernambuco, e duraram muitos
anos, começando em 1685 e ainda que passaram a Goiânia e outras
freguesias adiante, só a toda esta Igarassu deixaram intata, porque se
bem 2 ou 3 pessoas os trouxeram do Recife nelas se findaram sem passar a
outra, o que tudo é notório (\dots{}).\footnote{Odair Franco, \textit{História
  da febre amarela no Brasil} (Rio de Janeiro, Ministério da Saúde ---
  Dep. Nacional de endemias rurais, 1969), p.~21.}
\end{quote}

Quanto às causas, é ainda Joaquim Ferreira da Rosa quem diz no
\textit{Tratado único da constituição pestilencial de Pernambuco}:

\begin{quote}
Tendo nós já dado notícia que o ar se pode influenciar pelos Astros
(quaisquer que sejam) e principalmente pelos eclipses do Sol e da Lua,
podemos entender que não faltaram estas causas: pois no ano de 1685, a
dez de Dezembro (conforme Argolo) houve eclipse da Lua às seis horas
para sete neste hemisfério, estando a Lua na cabeça do Dragão no Signo
de Gêmeos, e o Sol na cauda do Dragão no Signo de Sagitário, e conjunção
com Mercúrio, e oposição com a Lua. Precedeu algum tempo antes outro
eclipse do Sol, a quem um insigne Matemático Padre da Companhia Valentim
Estancel chamava Aranha do Sol; e conforme a calculação, e juízo, que
formou dos movimentos dos Planetas, além de outros infortúneos,
prognosticava doenças. E em um Tratado manuscrito diz nesta forma:
Durarão os efeitos de seus venenos influxos (se a Divina Misericórdia
não se compadecer de suas criaturas) até o ano de 1691. Oxalá não passem
a mais anos nossas calamidades.\footnote{Joaquem Ferreira da Rosa,
  \textit{Tratado único\ldots{}}, 1694, p.~11}
\end{quote}

Uma coisa nos interessa particularmente no tratado de Rosa: um sistema
global de correspondências entre entes singulares --- terra e céu,
corpo-microcosmo e macrocosmo, epidemias e órbitas celestes ---, que vêm
se alojar em uma sorte de metafísica das semelhanças. A relação de
conjunto que tais semelhantes comunicam entre si se apresenta na forma
de uma estrutura circular: o céu envolve a terra, o cosmo determina o
corpo, as esferas concêntricas entram em um jogo de espelhos com o
espaço envolvido, de modo que já não nos será possível assinalar, dos
mútuos reflexos que percorrem o espaço, quem será o primeiro, quem age
sobre quem, de quem é o primado na ordem das simpatias. Será notável a
riqueza da trama semântica da semelhança. Ela pareceu ter organizado as
figuras do saber, ou mesmo definiu as formações discursivas do
conhecimento médico luso-brasileiro --- pelo menos até meados do \textsc{xviii}.
Interrogamos o lugar que ocupa o corpo-microcosmo nessa forma de saber
médico, de que modo este ente constelado, que é o homem, sustenta-se sob
a condição de ter sido criado à imagem e semelhança da ordem do mundo.
Ou, mais especificamente, como foi possível a uma cultura pensar o
corpo-microcosmo esmorecido por uma doença, se esta designa apenas perda
momentânea do equilíbrio entre o firmamento --- onde cintilam estrelas
visíveis --- e o firmamento íntimo do corpo constelado? Sob os termos da
análise histórico-filosófica foucaultiana, equivaleria dizer: como
pensar o pensamento de um horizonte histórico soterrado por outras
formas arqueológicas sucessivas do saber?

Face erguida entre as faces das coisas, o corpo-microcosmo pertence a um
horizonte das semelhanças que é, no fundo, um ``espaço de irradiação.
Por todos os lados, o homem é por ele envolvido; mas esse mesmo homem,
inversamente, transmite as semelhanças que recebe do mundo. Ele é o
grande fulcro das proporções --- o centro onde as relações vêm se apoiar
e donde são novamente refletidas''.\footnote{Michel Foucault, \textit{As
  palavras e as coisas} {[}1966{]}, trad. Salma T. Muchail, 9ª ed.~(São
  Paulo, Martins Fontes, 2007), p.~31.} Retenhamos por enquanto que a
possibilidade do adoecimento, na perspectiva do \textit{Tratado único da
constituição pestilencial de Pernambuco}, teria a ver com esse lapso
reversível da natureza, quando o ocaso deixa de exercer justiça às
proporções que encadeiam destino humano e conjunção dos astros.

A concepção do corpo-microcosmo pareceu ter desempenhado um papel
construtor no saber médico luso-brasileiro dos séculos \textsc{xvii} e \textsc{xviii}. A
ela está ligado o conceito de clima. (É preciso desembrulhar os fios com
cuidado e partir do simples em direção ao complexo para que se prossiga
às condições de emergência do problema.) O clima era tão somente a
mudança do aspecto do céu, o aspecto que o céu assume, gradativamente,
do Equador ao Polo. Os planetas, os astros, os corpos celestes guardarão
boas influências enquanto conservam certa forma e figura; e por diversa
posição ou ordem, resultam infelizes efeitos que se comunicam mediante o
ar. Por sua vez o ar recebia inquinamento ou sordície, ou qualidade
contagiosa, dos eclipses do Sol e da Lua ou quaisquer outros aspectos
infaustos das Estrelas e Planetas: ``a conjunção de Saturno e Marte, por
exemplo, no signo de Aquário, relacionava-se para muitos com o surto de
várias pestilências particularmente perigosas''.\footnote{Sergio Buarque
  de Holanda, \textit{Visão do Paraíso: os motivos edênicos no
  descobrimento e colonização do Brasil} {[}1959{]} (São Paulo,
  Brasiliense, 2000), p.~328.}

O próprio Fracastoro --- que bem ou mal costuma surgir alinhado como
precursor das teorias modernas da transmissão --- reúne registro dos
sinais cósmicos que anunciam o perigo do contágio presente.\footnote{N.
  da E.: Girolamo Fracastoro (c.1478-1553), professor de medicina da
  Universidade de Pádua e médico-chefe do Papa Paolo III, autor de
  \textit{De contagione et contagiosis morbis} (Sobre o contágio e doenças
  contagiosas), de 1546.} Quando no cimo do céu algum desses astros
chamados planetas setentrionais ou austrais procuram entrar em
conjunção, sabe-se que forçosamente nessa região se fazem grandes
mudanças no entorno da terra: grandes umidades como efeito de vapores
que exalam da terra e das águas; em seguida as secas que causam
incêndios para desfazer a nuvem de vapores. O que significa que mudanças
no aspecto celeste trazem consigo putrefação ou corrupção do ar, que são
as causas do contágio à distância nas febres pestilentas. Mas o recurso
à influência dos astros, como expediente para a prática médica, permite
ainda que se façam previsões de um contágio por vir. Lembrava, o mesmo
Fracastoro, que em 1546, os astrônomos teriam anunciado com anos de
antecedência o aparecimento da sífilis.

\begin{quote}
Se a conjunção dos astros se faz sob as maiores estrelas que chamamos
fixas, então vocês podem prever algum contágio notável. Há certos
aspectos de planetas aos quais os astrônomos atribuem estes presságios,
e que não se deve negligenciar absolutamente (\dots{}). O ar também pode nos
dizer sinais: tais como os numerosos e frequentes incêndios que aparecem
na região mais elevada chamada zênite, como as estrelas, os cometas, os
meteoros, e outros fenômenos do mesmo gênero que fazem putrefações em
torno da terra.\footnote{Girolamo Fracastoro, \textit{La contagion, les
  maladies contagieuses et leur traitement} {[}1546{]} (Paris, Société
  d'éditions scientifiques, 1893), p.~72-4.}
\end{quote}

A medicina portuguesa não era exatamente empírica --- as raízes
hipocráticas que interligavam filosofia e medicina permaneciam intactas
nos séculos \textsc{xvii} e \textsc{xviii}. O domínio do Latim, da \textit{Física} de
Aristóteles, do \textit{Tetrabiblos} de Ptolomeu, dos trabalhos de Avicena
construíam o currículo na Universidade de Coimbra, em comunhão com a
leitura e comentário dos textos de Hipócrates e Galeno. O médico --- cuja
formação, prognósticos e terapêuticas estão costurados pela continuidade
entre causas meteorológicas e o corpo-microcosmo --- não arrasta o mundo
pelos cabelos para torná-lo mais racional e higiênico, mas faz
restituí-lo ao jogo primevo de semelhanças, como quem fecha um mapa de
modo a dobrar o mundo sobre si, para que entre terra e céu se estabeleça
o jogo de espelhos --- dois espelhos que se enxergam um no outro, um
mundo constituindo cadeia consigo, o corpo humano visto como o espelho
dos céus. Em uma espécie de metafísica das semelhanças, a linguagem está
assentada no mundo e o mundo retém uma sintaxe que dele fez parte desde
idades remotas. ``Trata-se da teoria da simpatia universal, intuição
vitalista do determinismo universal, que dá seu sentido à teoria
geográfica dos meios''.\footnote{Georges Canguilhem, \textit{O
  conhecimento da vida} {[}1952{]}, trad. Vera L. A. Ribeiro (Rio de
  Janeiro, Forense Universitária, 2012), p.~63.}

O reconhecimento das semelhanças que preveem pestes e cataclismos dirige
o prognóstico, e este auxilia o médico em uma descoberta que é da ordem
da simpatia das coisas entre si. Para tal, as coisas humanas, as esferas
celestes, os entes sublunares serão dotados de uma linguagem sem
historicidade. As coisas falam uma língua que se dispõe a narrar a
sintaxe que as liga originariamente, e é pelo manejo dessa sintaxe que o
médico discerne como agir com correção em sua arte. Não há que se temer
a penúria ou a escassez ou o vazio dos signos quando se trata de uma
sociedade que de uma ponta a outra expõe seus signos ou códigos de
registro, não podendo haver algo que escape, não podendo porventura
subsistir o não assinalável em relação à própria sociedade.
Foucault\footnote{Cf. Michel Foucault, \textit{As palavras e as coisas}
  {[}1966{]}, 2007, p. 40.} aponta nessa direção quando, pensando a
Renascença, diz que a natureza das coisas, sua coexistência, o
encadeamento que as vincula e pelo que se comunicam não seria diferente
de sua semelhança. Mas entre realidade e imagem projetada, como é
possível --- seria possível? --- enumerar um duplo domínio, já que a sua
razão de ser é justamente uma espécie de conterraneidade?

A semelhança não é precisamente um código genético das coisas, as coisas
mesmas são modos de marcação dotados de uma potência e fluidez
extraordinárias, o que nos leva ao raciocínio de que ela não se presta a
categorias no formato significante-conteúdo. A semelhança é sim uma
dobra do ser. Ela é, em uma única figura, as leis de afinidade, o
domínio das marcas e o conteúdo assinalado. Paracelso compara tal
``duplicação fundamental do mundo à imagem de dois gêmeos `que se
assemelham perfeitamente, sem que seja possível a ninguém dizer qual
deles trouxe ao outro sua similitude'\,''.\footnote{Citado em idem,
  p.~27.} Por exemplo, o homem há de ser, como o firmamento, um
``constelado de astros'':

\begin{quote}
Seu céu interior pode ser autônomo e repousar somente em si mesmo, sob a
condição, porém, de que, por sua sabedoria, que é também saber, ele se
torne semelhante à ordem do mundo, a retome em si e faça assim
equilibrar no seu firmamento interno onde cintilam as estrelas
invisíveis. Então, essa sabedoria do espelho envolverá, em troca, o
mundo onde estava colocada; seu grande elo girará até o fundo do céu e
mais além; o homem descobrirá que contém ``as estrelas no interior de si
mesmo (\dots{}), e que assim carrega o firmamento com todas as suas
influências''.\footnote{Paracelso, \textit{Liber Paramirum} {[}1531{]},
  trad. francesa Grillot de Givry (Paris, 1913), p.~3, citado em Michel
  Foucault, \textit{As palavras e as coisas} {[}1966{]}, 2007, p.~28.}
\end{quote}

Talvez daí a preferência por partos durante a lua cheia e a lua nova, em
detrimento das luas crescente e minguante, que são as piores luas. Ou a
poda preferida na lua nova, ao passo que a poda durante a lua crescente
mirra os frutos.\footnote{Cf. Ptolomeu, \textit{Tetrabiblos} {[}c.~séc.
  I{]}, trad. inglesa J. M. Ashmand (London, W. Foulsham \& CO., 1917),
  p.~3. Tradução nossa.} Daí o plano piloto das povoações que originam
as cidades romanas, orientadas por duas fartas avenidas principais que
se cruzam, o \textit{cardo} e o \textit{decumanus}, respectivamente: ``duas
linhas traçadas pelo \textit{littus} do fundador, de norte a sul e a de
leste a oeste, que serviam como referência para o plano futuro da rede
urbana. (\dots{}) Nestas o agrupamento ordenado pretende apenas reproduzir
na terra a própria ordem cósmica''.\footnote{Sérgio Buarque de Holanda,
  \textit{Raízes do Brasil} {[}1936{]}, 26ª ed. (São Paulo, Cia. das
  Letras, 1995), p.~97.}

Talvez não se possa falar em princípios fundamentais para um código da
medicina luso-brasileira no período colonial. Mas tampouco é absurda a
proposta de fixar Hipócrates como o grande instaurador de
discursividade, espécie de lei que anima a possibilidade infinita dos
discursos que a ele pedem fiança e dele se valem como herói e veículo.
Hipócrates --- não precisamente o autor como esquema de inteligibilidade
que reside soberano sobre um \textit{corpus} filosófico, mas o regime de
discursividade a ele associado --- como um fornecedor fundamental de um
modo de ser do discurso médico e da epistemologia médica até, ao menos,
meados do Setecentos. Quando Romão Mosia Reinhipo escreve seu
\textit{Tratado único das Bexigas e Sarampo} (1683) e apresenta os meios
práticos para enfrentar as enfermidades que faziam estragos em Recife,
adverte quanto às razões e porquês de os cometas exercerem mais efeitos
na América, produzindo Bexigas, que em outros Reinos do mundo.

\begin{quote}
A razão parece fácil, e é: que como as Bexigas nascem da ebulição, ou
fervor do sangue, e o clima do nosso Brasil seja naturalmente quente, e
úmido, mais capaz para estes fervores, junto com o incêndio do Cometa,
faz mais os seus efeitos nesta América com estes fervores, produzindo
Bexigas, e Sarampos, do que nos outros Reinos, onde produzem guerras, e
outros efeitos semelhantes. (\ldots{}) Advertência, que razão há para
que nos outros Reinos haja todos os anos Sarampos, e Bexigas, daquelas,
que chamam esporádicas, que são as que vêm por causa interna, e
compreendem a pouca gente, e não por influxo celeste, a que a chamamos
Epidêmicas, que são as que comumente vêm ao Brasil, e ofendem a
todos.\footnote{Romão Mosia Reinhipo, \textit{Trattado Unico das bexigas e
  sarampo, oferecido a D. João de Sousa} (Lisboa, na oficina de João
  Galrão, 1683), p.~12.}
\end{quote}

O \textit{Tratado} versará não só nos termos de uma lição de meteorologia,
das modificações atmosféricas em seu encadeamento com um influxo celeste
capaz de liberar efeitos destrutivos seja no curso da peste, seja no
destino dos tronos. Nessa versão tardia da prática hipocrática a
percepção do processo mórbido baliza o traçado onde aparece inserido o
raio de visibilidade médica. Não é só uma concepção não ontológica e sim
dinâmica da doença, é, ademais, a consideração dinâmica e totalizante de
um corpo dotado de todos os meios para a cura.

\begin{quote}
Eu responderia que, como o clima do Brasil é tão cálido, e tão úmido, e
os poros dos corpos, que nele habitam, andem sempre abertos, suando e
tressuando; nesta evacuação do suor, gastam alguma porção do humor, que
lhe podia servir de matéria para a Bexiga, e para o contágio, o que não
acontece nos outros Reinos, porque como andam com os poros da carne mais
fechados, não gastam, como no Brasil, aquela porção de humor, que lhes
servem para padecerem lá todos os anos as Bexigas, e no Brasil só uma
influência dos astros, e um agente tão poderoso, como é um Cometa, as
faz produzir epidemicamente, ofendendo a tantas criaturas.\footnote{\textit{Ibidem,}
  p.~13.}
\end{quote}

Quando, indo a praias e águas alheias, um Rosa ou um Reinhipo atracam no
porto de Recife, eles chegam inexperientes sobre como as estações do ano
deformam ou conservam tal ou qual região, desconhecem como se manifesta
o inverno, se é seco e boreal, se a primavera é chuvosa, austral.
Naturalmente não discernem que em cidades com esses atributos o verão
procura ser propício às febres e produz oftalmias e disenterias em boa
escala. Ignoram igualmente, diria Hipócrates, que se o Recife estiver
voltado para os ventos quentes --- que ocorrem entre nascente e ocaso
hibernais do Sol --- e resguardado dos ventos vindos das Ursas, as águas
serão um pouco salgadas, quentes no verão e frias no inverno, e
naturalmente nos homens com mais de 50 anos os fluxos que sobrevêm do
cérebro tornar-se-ão na maioria das vezes hemiplégicas. Quer dizer,
aquele médico que chega em terra estrangeira precisa estar atento a seus
mistérios: ``à posição dela, a como está assentada, e aos ventos e aos
nascentes do sol; pois não podem ter a mesma propriedade a cidade que
está voltada para o bóreas e a que se volta para o noto, nem a que se
volta para o sol que se ergue e a que se volta para o sol se
pondo''.\footnote{Hipócrates, \textit{Textos hipocráticos: o doente, o
  médico e a doença} {[}c.~séc IV a.C.{]}, trad. Henrique Cairus e
  Wilson Ribeiro (Rio de Janeiro, Ed. FIOCRUZ, 2005), p.~94 (\textit{Ares,
  águas e lugares} §I.1.)} O tratado dos \textit{Ares, Águas e Lugares} se
vale da observação dos fenômenos da atmosfera (ou causas meteorológicas)
para julgar em quais elos desencadeia-se a produção de doenças. Da
intempérie dos dias, das estações, dos anos, provêm doenças epidêmicas.
``Igualmente Galeno observa que desde que as estações sejam bem
regradas, não há nem peste, nem epidemia, mas somente doenças que
dependem da dieta''.\footnote{Cf. Louis Lépecq de La Cloture,
  \textit{Observations sur les maladies epidémiques} (Paris, De
  l'imprimerie de Vincent, 1770), p.~XCVII. Tradução nossa.}

A Cosmografia no Seiscentos é a descrição universal do mundo, seu
domínio compreende os quatro elementos --- a Terra, a Água, o Ar, o Fogo
---, o Sol, a Lua e todas as Estrelas, e tudo o que está cercado e
coberto pelo céu. Primeiramente, traça os círculos de que dispõe a
esfera celeste. ``A região celeste (a qual os filósofos chamam quinta
essência), é de uma substância invariável, sem mudança ou alteração, e é
dividida em dez esferas ou círculos''.\footnote{Pierre Apian, \textit{La
  Cosmographie} (Paris, par Vivant Gaultherot, 1551), p.~9. Tradução
  nossa.} Pela distinção das esferas se extrai a medida e a distância
entre os lugares, a diversidade dos dias e noites, as quatro partes do
mundo, o movimento, a nascente, o poente das estrelas, a proporção dos
climas e demais caracteres da dimensão sublunar. Seus astrônomos
dividiram a largura da Terra em sete fatias, outros em nove, cada uma
dessas recebendo o nome de Clima. Clima é um espaço da terra
compreendido entre duas linhas paralelas, ``o qual vemos mudar a cada
meia-hora, para que o dia vá se tornando cada vez mais desigual.
(\ldots{}) Tanto mais um clima está numericamente distante do
equinocial, tanto mais as unidades de meia-hora avançam no curso do dia
de doze horas''.\footnote{\textit{Ibidem}, p.~9.} Nessas condições, o
primeiro Clima se chama \textit{dia meroës}, porque \textit{dia} em grego
significa ``por'' e Meroe é uma cidade da África, e esse Clima passa por
essa cidade. O segundo: \textit{dia Syenes}, porque o meio dele passa por
Siena, uma cidade do Egito sob o trópico de Câncer. O terceiro Clima
chama-se ``por Alexandria'' e assim por diante. O que é notável e exige
atenção não é a facilidade de algum tipo de passagem permeável entre
região celeste e mundo elementar, como se fosse necessário supor alguma
causa física, por exemplo, um meio etéreo ou uma matéria sutil
cartesiana que faz papel de veículo ou comunica, de uma esfera à outra,
a influência mórbida. O que se dá entre mundo celeste e elementar é
muito mais a disposição de uma rede complexa de semelhanças alojada em
uma língua natural. A semelhança funcionou como condição de
possibilidade para uma medicina luso-brasileira que, ao seu modo
peculiar, serviu-se de Hipócrates. Daí os climas, as causas
meteorológicas e os fenômenos atmosféricos tenderem a ser não
primeiramente tema ou fonte indefinida de saber, mas quem sabe um
possível solo homogêneo de diferenças emergentes e identidades
ordenáveis.

No ano de 1685, segundo o \textit{Tratado único da constituição
pestilencial de Pernambuco}, um eclipse da Lua sucedeu um eclipse Solar.
Três anos antes Reinhipo descrevia como um agente tão poderoso como um
Cometa produz as Bexigas em certa constituição, em função do jogo de
semelhanças entre os astros e a realidade sublunar. O bexiguento, um
amarelento, é um constelado de astros, ele carregará o firmamento
consigo em todas as suas influências. Onde entra o médico? Ao médico é
aconselhável assistir ao doente e não prescrever drogas de acordo com
seu gosto pessoal, muito menos fazer incisão antes que se passem os dez
dias. O médico observa o levante dos astros, ele profetiza as pestes tal
como as antenas meteorológicas modernas preveem tempestades. Seu raio de
ação é sempre estar à espreita do ``conhecimento das mudanças das
estações, e dos nascimentos e ocasos dos astros, e de como cada um deles
ocorre'', sabendo ``de antemão como será o ano''.\footnote{Hipócrates,
  \textit{Textos hipocráticos}, 2005, p.~95.} É preciso estar atento para
obter bom êxito na arte hipocrática, ser um conhecedor prévio das
ocasiões oportunas ou inapropriadas para a saúde. Com efeito, pode um
médico julgar serem esses temas ``muito estratosféricos'', mas deve
assumir como sua a opinião de ``que a astronomia tem lugar na medicina,
e não um lugar pequeno, mas realmente grande; pois as cavidades mudam
nos homens de acordo com as estações do ano''.\footnote{\textit{Ibidem},
  p.~95.} A semelhança entre corpos humanos e astros será ``base dos
estudos fisiognomônicos, que propunham interpretar o corpo e o
comportamento humano fundamentado nas `assinaturas' deixadas pelos
corpos celestes''\footnote{Jean Luiz Nevez Abreu, \textit{Nos domínios do
  corpo: o saber médico luso-brasileiro no século \textsc{xviii}} (Rio de
  Janeiro, Ed. FIOCRUZ, 2011), p.~59.} desde o marco fundamental da
Criação.

\begin{quote}
A teoria hipocrática baseava-se na correspondência isomórfica entre a
ordem do cosmo e o equilíbrio do organismo, que se exprimia em um poder
natural de correção de desordens: a \textit{vis medicatrix naturae}. Para
a fisiologia humoral, portanto, a ideia de doença se expressava como
desequilíbrio do organismo em face da ordem da \textit{physis}. Buscando
reconstituir o equilíbrio humoral rompido pela desarmonia entre o meio
interno e o ambiente envolvente, a terapêutica --- basicamente expectante
--- dependia tanto do médico como da sujeição do paciente ao processo de
cura.\footnote{Flavio Coelho Edler, \textit{A Medicina no Brasil Imperial:
  clima, parasitas e patologia tropical} (Rio de Janeiro, Ed. FIOCRUZ,
  2011), p.~30.}
\end{quote}

O corpo humano não é unidade. Diz-se, pois, que a natureza humana não é
alguma entidade monista, não é ar, nem fogo, nem água ou quaisquer
unidades exclusivas da física antiga. O homem não é uma unidade, pois se
assim o fosse não haveria de sofrer e não cairia enfermo. Mas quando as
quatro substâncias do corpo (o sangue, o fleuma, a bile amarela e bile
negra) entram em desarranjo, quando os humores, contra a natureza do
corpo humano, esfriam-se, esquentam-se, secam ou umedecem, eis que se
sucedem doenças. E as doenças não são outra coisa senão uma desarmonia
ou colapso na ordem dos humores, seja no tocante às proporções,
propriedades ou quantidades. É tarefa do médico fazer vigília, ``pôr-se
em oposição à constituição das doenças, às características físicas, às
estações e às idades, e relaxar o que estiver tenso, retesar o que
estiver relaxado''.\footnote{Hipócrates, \textit{Textos hipocráticos},
  p.~46 (\textit{Da natureza do homem} §9).} Por exemplo: ``as doenças que
engendram repleção, a evacuação as cura; as doenças que surgem pela
evacuação, a repleção as cura; as que são oriundas do exercício, a pausa
cura; e as que são geradas pela inércia, cura-as o
exercício''.\footnote{\textit{Ibidem}, p.~46.} Ora, as doenças ``provêm
umas das dietas, outras do ar, o qual inspiramos para viver''\footnote{\textit{Ibidem,}
  p.~46.}, porque quando muitos são ao mesmo tempo tomados por uma só
doença, quando se instaura uma \textit{epidemia}, deve-se atribuir a causa
ao ar, porque é o que há de mais comum, aquilo de que todos nos servimos
em comunhão. E mesmo no instante em que se instaura a carnificina da
peste, o bom médico vem a ser filósofo virtuoso e seguro de que é a
natureza quem sozinha cura as doenças. Pois o que faz o médico versado
em Hipócrates no contexto da epidemia? Ele é uma farmácia de regimes de
vida, um balcão de comércio de sabedoria: é aconselhável não ``mudar as
suas dietas (\dots{}); estar atento ao corpo que emagrece e se enfraquece ao
máximo (\dots{}); porque se muda rapidamente a dieta, o elemento mais novo
torna-se um perigo no corpo; mas é preciso manter as dietas como
estavam, quando parecem em nada prejudicar''.\footnote{\textit{Ibidem,}
  p.~47}

Corpo aberto com poros e cavidades que fazem transbordar e jorrar
humores em porções. Um corpo enquanto ponto de partida para a produção e
fluxos de humores, ponto de chegada para a interceptação de venenos,
vapores e qualidades comunicadas dos Astros, lugar de discórdia. Como
lugar de interseção e ponto de corte dessas qualidades, ele é sua
condição de hiância, um sempre estar em estado de mudança com tendências
ao equilíbrio. O corpo-microcosmo é um corpo de ordem diversa quando
temos em mãos o seu sistema de interações dinâmicas com a atmosfera.
Porque, tratando-se das epidemias, são justamente as intempéries
climáticas as responsáveis por romper o equilíbrio humoral e instaurar
alguma desarmonia entre meio interno e esferas envolventes.

O \textit{Tetrabiblos} de Ptolomeu já interrogava ser ou não possível
desenvolver um prognóstico da disposição do corpo humano pelo
conhecimento acurado da qualidade das estações.\footnote{``O que
  impediria esse pesquisador cuidadoso de compreender também a qualidade
  geral da idiossincrasia de cada um dos homens (por exemplo, qual é o
  seu tipo de corpo e qual é o seu tipo de alma), com base no seu
  ambiente de nascimento? E de compreender também os eventos de cada
  momento, tendo em vista que, por um lado, um tipo de ambiente é
  proporcional a um tipo de temperamento e pode contribuir para a saúde,
  e, por outro lado, outro tipo de ambiente é desproporcional e
  contribui para a adversidade? Portanto, através desses e de
  semelhantes argumentos, pode-se compreender que esse tipo de
  conhecimento é possível.'' Ptolomeu, \textit{Tetrabiblos,} p.~6 da
  versão inglesa, tradução de Marcus Reis, \textit{Caderno de História e
  Filosofia da Ciência}, Campinas, série 4, vol.~1, n.~2, jul.-dez.
  2015, p.~315.} ``Através da configuração celestial, (\ldots{}) o tipo
corpóreo e a capacidade mental da qual a pessoa é dotada desde o berço
podem ser anunciados''.\footnote{Ptolomeu, \textit{Tetrabiblos,} p.~4-5.
  Tradução Nossa.} Brás Luís de Abreu, em seu \textit{Portugal Médico}
(1726), dizia que os homens dotados de compleição saturnina tinham com
frequência a estatura do corpo ``grossa, avultada e grave, mas com
alguma improporção a respeito das partes que a compõe'': o rosto é
vertical e gordo, a cabeça em um arredondamento estabanado, os olhos
negros e góticos, ``centralmente dispostos, um maior que o outro'', ``o
nariz grande, descarnado e agudo''.\footnote{Citado em Jean L. N. Abreu,
  \textit{Nos domínios do corpo}, 2011, p.~57.}

A noção de \textit{medicina das semelhanças}, que forjamos a fim de
caracterizar o registro tardio da prática hipocrática em Portugal,
baseia-se em um duplo valor, que até aqui quisemos atribuir na forma de
conceitos: primeiramente, certa percepção historicamente situada da
peste, o reconhecimento de que a peste estaria associada a mudanças do
aspecto do céu. Logo, era competência do médico o manejo da astronomia,
a fim de manter claro o jogo das semelhanças que envolve surtos
epidêmicos e influxos celestes. Papel, portanto, importante esse dos
fenômenos atmosféricos, ou da posição dos astros, para uma medicina que
reclama a tarefa de antecipar, na forma de presságios, as pestes, pelo
recurso à astronomia. Em seguida, a noção de corpo-microcosmo, se por um
malabarismo conceitual nos for possível dizer, em relação ao microcosmo,
que a menção a alguma escala ou proporção não é o mais fundamental. O
corpo-microcosmo, pensamos, funciona como zona de indistinção ou
conflito entre um corpo interno ou autônomo e a ordem do mundo. A
consideração dinâmica, diferencial e totalizante de um corpo, para quem
a doença significaria um colapso reversível da ordem dos humores, está
ligada à própria dinamicidade da doença. O colapso, por sua vez, no caso
das epidemias, tinha a ver com a qualidade das interações dinâmicas do
corpo com as causas meteorológicas. O que nos leva a pensar que os dois
valores invocados na forma de conceitos não deixam de ser como duas
faces de uma só moeda.

O procedimento cartesiano tende a preparar terreno para outra ordem de
coisas,\footnote{Não nos deve causar surpresa o fato de que Descartes e
  Hobbes sejam contemporâneos da publicação de boa parte dos Tratados de
  medicina luso-brasileira aos quais fizemos menção. Pelo menos dois
  motivos justificam o fato de os médicos formados em Coimbra não
  participarem, até talvez início do \textsc{xviii}, dos circuitos de
  transformações nos estudos de anatomia em outras regiões da Europa. Em
  primeiro lugar, um sistema de conhecimento esgotado na interpretação e
  no comentário dos cânones, o que manteve à margem dos currículos as
  experimentações anatômicas animadas pelo mecanismo. O currículo do
  ensino de medicina na Universidade de Coimbra foi durante muito tempo
  tributário, quase que exclusivamente, da leitura e comentário dos
  autores da Antiguidade --- principalmente Hipócrates, Galeno e os
  comentadores árabes, como Avicena. O comentário foi técnica pedagógica
  por excelência na Escolástica e, por essa via, existia como uma
  tecnologia de controle dos discursos. Ele, mais fundamentalmente,
  mantinha aceso e em movimento um sistema de conhecimento que não teria
  tanto a ver com demonstração ou classificação e sim com o interpretar.
  E porque o objetivável do comentário de Hipócrates seria um programa
  de semelhanças entre o texto primeiro (que quase assumia \textit{status}
  de transcendental) e o infinito das interpretações, não era intenção
  do ensino da medicina instaurar novos saberes, mas preparar
  profissionais competentes para atuar segundo o conhecimento, a memória
  e a proliferação do já dito. ``O comentário'', diz Foucault, ``se
  assemelha indefinidamente ao que ele comenta e que jamais pode
  enunciar (\dots{}), assim a tarefa infinita do comentário se assegura na
  promessa de um texto efetivamente escrito, que um dia a interpretação
  revelará por inteiro.'' Cf. Michel Foucault, \textit{As palavras e as
  coisas} {[}1966{]}, 2007, p.~57-8. O segundo motivo para o
  aparecimento tardio dos valores que já haviam ativado a modernidade
  filosófico-científica em França e Inglaterra talvez tenha sido a
  Igreja. A impressão e a circulação, em Portugal, dos tratados de
  medicina nos séculos \textsc{xvii} e \textsc{xviii} exigiam como pré-requisito uma
  listagem de Licenças do Santo Ofício, que serviam para assegurar que o
  livro não continha matéria que contrariasse a fé e os bons costumes
  católicos. ``Em Portugal, não só a medicina teria ficado alheia à
  maior parte dessas renovações, como também as demais ciências
  continuaram a se fundamentar nos princípios aristotélicos e na
  tradição escolástica (\dots{}). Esses autores e comentários de suas obras
  eram obrigatórios nos cursos, submetidos a uma concepção sacral e
  teológica do saber. Segundo os princípios do Tomismo, vigentes na
  cultura ibérica até o século \textsc{xviii}, a inteligência primeira que tudo
  ordenava era Deus. Sendo assim, as ações humanas transcendiam o
  próprio homem. Em razão dessa premissa, as ciências naturais não
  encontraram um espaço de autonomia, pois as leis relativas a esse
  mundo sublunar não poderiam explicar o supralunar, não autorizando
  `uma superposição epistemológica das ciências físicas e naturais sobre
  a teologia'. (\ldots{}) Apesar da relativa `abertura' dos jesuítas no
  restante da Europa, que propunham conciliar as doutrinas aristotélicas
  com a ciência nos moldes do cartesianismo, o ensino em Portugal não
  incorporou tais mudanças. Em 1746, o reitor do Colégio de Artes
  determinava por meio de um edital a proibição em ensinar e defender as
  `opiniões recebidas ou inúteis (\ldots{}) como são as de Renato
  Descartes, Gassendi, Newton (\ldots{}) ou quaisquer outras conclusões,
  opostos ao sistema de Aristóteles'\,''. Cf. Jean L. N. Abreu,
  \textit{Nos domínios do corpo}, 2011, p.~18-9.} pelo menos no plano
filosófico. Hobbes publicava o \textit{Leviatã} (1651) em Londres, sete
anos após Descartes ter publicado os \textit{Princípios de Filosofia}.
Figura na introdução do livro de Hobbes: ``Pois o que é o coração, senão
uma mola; e os nervos, senão outras tantas cordas; e as juntas, senão
outras tantas rodas, imprimindo movimento ao corpo inteiro, tal como foi
projetado pelo Artífice?''\footnote{Thomas Hobbes, \textit{Leviatã}
  {[}1651{]}, trad. João Monteiro e Maria da Silva, 2ª ed.~(São Paulo,
  Abril Cultural, 1979), p.~5.} Sabe-se como Descartes compara o coração
a um relógio, afirmando que o sangue circula no corpo como os
contrapesos movem engrenagens. Também as árvores produzem frutos como os
relógios indicam as horas. O que vemos nascer aos poucos é a imagem do
corpo extensivo. Trata-se da assimilação do funcionamento do corpo à
tecnologia do dispositivo mecânico, substituição do corpo constelado
pelo jogo de ligações mecânicas;\footnote{Cf. Canguilhem, Georges
  Canguilhem, \textit{O conhecimento da vida} {[}1952{]}, 2012,
  p.~122-123.} substituição do corpo-microcosmo pela \textit{res}
\textit{extensa} integrada pela coisa pensante; substituição da semelhança
entre corpo-microcosmo e conjunção de astros por uma analogia entre
espantalhos autômatos e corpo humano. É como se em Descartes o real
repetisse certa estrutura mecânica, e essa tese é favorecida pela
unicidade da matéria e pela identificação da matéria à extensão: os
movimentos dos órgãos estão encadeados como engrenagens de um corpo
autômato porque as regras da Mecânica são as regras da Filosofia Natural
e as coisas artificiais não diferem das naturais.

Canguilhem sugere ser essa compreensão da analogia entre máquina e
organismo algo inseparável do próprio \textit{cogito}: ``A distinção
radical da alma e do corpo, do pensamento e da extensão, acarreta a
afirmação da unidade substancial da matéria, seja qual for a forma
afetada por ela''.\footnote{\textit{Ibidem}, 2012a, p.118.} Em outras
palavras, um binômio metafísico requer a positividade de cada um dos
atributos que compõem a natureza do homem. O homem é composto de
\textit{cogito} e da extensão do corpo.

Era antes o coração humano o sol da esfera humana, os olhos as estrelas,
``as vistas meteoros, as iras raios, os mugidos trovões, os flatos
ventos; as lágrimas chuveiros, as palpitações terremotos, e tempestades
as aflições''.\footnote{Jean L. N. Abreu, \textit{Nos domínios do corpo},
  2011, p.~53.} Pela imagem dos autômatos, os corpos, privados de
qualidades humorais, distinguem-se agora unicamente pela figura
geométrica que podem assumir, ``tudo quanto pode ser atribuído ao corpo
pressupõe a extensão e não passa de dependência do que é
extenso''.\footnote{René Descartes, \textit{Princípios de Filosofia}
  {[}1644{]}, trad. João Gama (Lisboa, Ed. 70, 2006), p.~46.} Quebra-se
a ordem da \textit{phýsis} como função do corpo.

\begin{quote}
Relacionada ao caráter analítico que se imprime à racionalidade
científica moderna há a proliferação, a partir de 1650, dos termos
derivados do vocábulo ``órgão'' --- organização, organizado, orgânico,
organismo ---, o que mostra a tentativa de filósofos e médicos no sentido
de encontrar uma ordenação capaz de explicar a vida. O organismo passou
a representar uma \textit{ordem de relações entre as partes de um todo, um
mecanismo}. O ser vivo, a partir de então, foi compreendido por meio do
desvendamento do seu funcionamento (\dots{}). A ordem dos seres vivos passou
a ser concebida como a de uma máquina e pensada mediante as leis da
mecânica. Não por acaso foi nessa época que Harvey explicou o
funcionamento da circulação sanguínea, comparando o coração a uma bomba
hidráulica e analisando-o em termos de volume e fluxo.\footnote{Dina
  Czeresnia, \textit{Do contágio à transmissão: ciência e cultura na
  gênese do conhecimento epidemiológico} (Rio de Janeiro, Ed. FIOCRUZ,
  1997), p.~23.}
\end{quote}

Então volumes atomizados, nascerão em pouco mais de um século os
indivíduos divorciados do sistema de interações cósmicas.\footnote{Já
  nas histórias naturais do século \textsc{xviii}, no domínio discursivo aí
  compreendido, Foucault reconhecerá o conceito de organização atribuído
  aos entes, um modo de composição de indivíduos complexos a partir de
  materiais mais elementares, coisa que produzirá um corte no espaço
  taxinômico. Ao agregado dos sólidos brutos, à justaposição da matéria
  bruta, opunha-se a composição dos sólidos organizados enredados em um
  número infinito de partes orgânicas. Uma mutação no sistema de
  pensamento da história natural ``acarreta uma consequência maior: a
  radicalidade da divisão entre orgânico e inorgânico. No quadro dos
  seres que a história natural desdobrava, (\ldots{}) a oposição entre o
  orgânico e o inorgânico torna-se fundamental.'' Michel Foucault,
  \textit{As palavras e as coisas} {[}1966{]}, 2007, p.~318.} E o homem
tornar-se-á, um dia, organismo. Alguns aspectos do racionalismo serão,
entretanto, ligeiramente convertidos em obstáculo epistemológico, para o
surgimento de uma medicina moderna como ciência anatomoclínica. Entre
esse mecanicismo cartesiano e o nascimento da anatomia patológica mudam
aprioristicamente as formas fundamentais de espacialização do corpo como
\textit{organismo}. A ênfase migrará, aos poucos, da doença para o doente,
da medicina das classificações das espécies patológicas para a
objetivação biológica do corpo. É uma espécie de ensaio para uma
revolução copernicana acessória. (E quando o fenômeno patológico chegar
um dia a ser quantificado como variação mórbida do funcionamento normal
do organismo doente, surgirá o problema da vida, da vida não como valor,
mas como norma. Porque a noção de vida será drasticamente colonizada
pela noção de bem-viver, ou seja, pela \textit{saúde}. E uma ciência do
bem-viver não é outra senão a Higiene.) Voltaremos ao fio dessa
discussão no terceiro capítulo, e lá será preciso situar de antemão o
nascimento dessa experiência outra da doença dentro de uma profunda
transformação arqueológica. Para que a Higiene possa se constituir
positivamente ela dependerá de uma medicina que disponha de um olhar
anatomoclínico, diferente do ``olhar cartesiano''.

\begin{quote}
Segundo Descartes e Malebranche, ver era perceber (e até nas espécies
mais concretas da experiência: prática da anatomia no caso de Descartes,
observações microscópicas no caso de Malebranche); mas tratava-se de,
sem despojar a percepção de seu corpo sensível, torná-la transparente
para o exercício do espírito: a luz, anterior a todo olhar, era o
elemento da idealidade, o indeterminável lugar de origem em que as
coisas eram adequadas à sua essência e a forma segundo a qual estas a
ela se reuniam através da geometria dos corpos; atingida sua perfeição,
o ato de ver se reabsorvia na figura sem curva, nem duração, da
luz.\footnote{Michel Foucault, \textit{O Nascimento da Clínica}
  {[}1963{]}, trad. Roberto Machado (Rio de Janeiro, Forense
  Universitária, 1977), p. XII.}
\end{quote}

Porém, logo as formas da racionalidade médica penetrarão a maravilhosa
espessura da percepção, e o leito do doente, convertido em experiência
de laboratório, se abrirá sob o comando de outro regime de
visibilidades. Aí, nesse contexto, a relação entre palavras e coisas
mudará de figura, e o olhar assumirá o poder de trazer à luz
objetividades até então neutralizadas. O surgimento do \textit{organismo}
e a biopolítica do corpo medicalizado nos interessam tão somente como
acesso para tratarmos a higiene geral como ciência do bem-viver, como
ciência, a rigor, da saúde moral e física de populações policiadas. Mas
até lá tentemos recobrar nossos primeiros caminhos, delimitemos as
relações entre ``experiência da epidemia no período pré-higienista'' e a
sucessão dos surtos de febre amarela que varreram a antiga capital do
Império, a partir de 1850.

\section*{Constituições epidêmicas e patologização do
cadáver}

\epigraph{Pode ser que o veneno teatro lançado no corpo social o desagregue, como
diz Santo Agostinho, mas à maneira de uma peste, um flagelo vingador,
uma epidemia salvadora na qual épocas crédulas quiseram ver o dedo de
Deus e que nada mais é senão a aplicação de uma lei da natureza pela
qual todo gesto é compensado por outro gesto e toda ação por uma reação
(\ldots{}) O teatro, assim com ao peste, é uma crise que se resolve pela
morte ou pela cura. E a peste é um mal superior por ser uma crise
completa, não sobrando nada depois dela a não ser a morte ou a
purificação.}{\textit{O teatro e a peste}, \textsc{antonin artaud}}


Passou a noite tranquilo. Porém de manhã inquietou-se impertinente e um
tanto febril, que terminou a febre por suor. Fizeram-se fricções sobre
as costas e a face interna dos braços com sulfato de quinina dissolvido
em água, óleo de vitriolo e vinagre.\footnote{Cf. ``Quinina'', em Pedro
  Luiz Napoleão Chernoviz, \textit{Diccionario de Medicina Popular} ---
  \textit{Volume Terceiro}, 2ª ed.~(Rio de Janeiro, Eduardo \& Henrique
  Laemmert, 1851), p.~335-6.} Às cinco da tarde, a febre decresceu. Das
seis para as sete, novo acesso e, às sete e meia, quando ia para o
banho, antes que se despisse sofre um ataque de convulsões e perda dos
sentidos. Às oito e meia, as convulsões se sucedem com pequeno intervalo
de tempo.

Onze da noite: permanece em estado febril. Diminuição sensível da
congestão cerebral, após evacuação provocada pelo efeito purgante do
medicamento administrado pelo Dr.~Joaquim Candido Soares de Meirelles.
Recobra os sentidos ainda em estado de torpor: volta-se procurando as
posições habituais em que costuma dormir. ``Sua alteza está em perigo de
vida''.\footnote{``Parte Official. Boletins'', \textit{Jornal do
  Commercio}, Rio de Janeiro, 11 de Janeiro de 1850, p.~1.}

Meia-noite: está em melhor estado, dormiu cerca de 40 minutos, os
fenômenos de congestão dissipam-se progressivamente. Segue o estado
esperançoso.

Uma da manhã, quando estava tranquilo e a febre decrescendo, um novo
ataque de convulsões ainda mais violento o assaltou. À vista do
redobramento do mal, as esperanças lisonjeiras que os entretinham da
salvação do Príncipe começam a desaparecer. É conveniente a presença do
padre da capela da fazenda, revestido de vestes e símbolos litúrgicos,
seguido pelo acólito.

O acesso que termina sua preciosa existência acontece às 4:20 do 10 de
janeiro de 1850. Sua Alteza Imperial O Príncipe D. Pedro Affonso
desfaleceu na Imperial Quinta de Santa Cruz, enquanto ali gozava a
família um verão afetado pelo rigor da seca e um calor abrasador.
Ostentou-se o segundo filho do Imperador como venturosa garantia de
continuidade do trono (o primogênito morrera em 1847). A família
deixaria aos poucos de veranear em Santa Cruz. A princesa Isabel e o
próprio Imperador ficariam doentes naquele verão de 1850. O casal
imperial não teve mais filhos.

No 15 de fevereiro de 1850, a ordem do dia na Câmara traz à tribuna uma
série de juízos desencontrados sobre a febre reinante na Corte Imperial.
Existem pomos de discórdias constantes entre deputados, médicos
diplomados e charlatões das boticas clandestinas: a qualidade contagiosa
ou não da moléstia, o dispêndio político, o humor popular associado à
confirmação da possibilidade de contágio, a quarentena e seus prejuízos
comerciais etc. Mas a epidemia de febre carecia de reputação. Teria sido
inédita na cidade do Rio de Janeiro, e mesmo as razões que a enquadram
ou liberam do gênero epidêmico são rodeadas pelo descrédito da
legislatura. O deputado Cruz Jobim, presidente da Academia Imperial de
Medicina, pede urgência para que se imprima um projeto de resolução
sobre saúde pública. Urge regularizar em todo o Império e quanto antes
um serviço sanitário dos portos e povoações. Será preciso superintender
a polícia médica e a higiene pública no que for relativo à limpeza das
povoações, das valas, dos aquedutos e matadouros, ao abuso das bebidas
alcóolicas, à prostituição, à extinção dos mangues ou quaisquer focos de
infecção permanente ou temporária. O projeto faz coro pela fiscalização
da prática médica, a expansão do serviço de \textit{cowpox}, um modelo
geral para os atestados de óbito a fim de que se produzam estatísticas
com mapas necrológicos. Porém, a Assembleia não contabiliza a criação
dos conselhos de saúde pública com a mesma pressa com que se avaliarão
nos tempos de guerra os gastos com balas, pólvora e saca-trapos. Fala-se
em transitar ou passar pelos trâmites, e o projeto empaca, engaveta-se
por ora. Não é oportuno --- são as vésperas de uma sequência de
cataclismos que irão desencadear na Corte empestada a incrível
mortandade do ano de 1850. A peste não despertara ainda seus sentimentos
escatológicos, é cedo para que os padres ressuscitem o vertiginoso
desatino do mundo, o terror público sob prenúncios de fim dos tempos não
havia chegado às paróquias. Enquanto ela silencia --- porque a princípio
a peste caminhava devagar e ``com passo certo, quase de uma casa para
outra, de uma travessa para outra, e nas casas e nas travessas atacando
uma pessoa após outra''\footnote{Roberto Lallement,
  \textit{Observações acêrca da epidemia da febre-amarela no ano de 1850,
  no Rio de Janeiro}. (Rio de Janeiro, Tip. Imp. e Const. de J.
  Villaneuve \& Comp., 1851), citado em Odair Franco, \textit{História da
  Febre Amarela no Brasil} (Rio de Janeiro, Ministério da Saúde/Dep.
  Nac. de Endemias Rurais, 1969), p.~38.} --- o deputado e médico Paula
Cândido, futuro presidente da Junta Central de Higiene Pública, toma a
palavra sobre um objeto que de perto nos toca:

\begin{quote}
Senhores, como eu entendo que a epidemia atual não é coisa que nos deva
surpreender porquanto os caracteres que até agora ela tem apresentado
não são assustadores, dando a comparação entre o número dos afetados, e
dos mortos, resultado muito favorável, acho-me habilitado para dizer que
a epidemia atual é sem dúvida da natureza dessa afecção chamada febre
amarela, mas como acontece na epidemia da cholera morbus, como
acontece durante a influência da peste nos países afetados destes
flagelos, acontece também com a epidemia atual; na grande maioria dos
casos os doentes se restabelecem (\dots{}). A ser pois esta afecção a febre
amarela, será isto motivo para causar terror? Não, porque se a
febre amarela é mortal em muitos casos, também é mortal o cancro, a
tísica, a pleurisia, e outras moléstias, entretanto que elas não causam
terror. Eu quero dizer com este argumento que a proporção dos mortos,
embora a moléstia seja por sua natureza grave, não deve a ninguém
atemorizar; esta proporção é muito pequena. (\dots{}) Eu tenho a
persuasão, não ouso dizer a convicção, de que a epidemia está muito
longe de ser contagiosa, porquanto (\ldots{}) temos nós visto morrer
algum médico no hospital, no Lazareto? Não consta.\footnote{``Câmara
  dos Srs. Deputados. Sessão em 12 de fevereiro de 1850'', \textit{Jornal
  do Commercio}, Rio de Janeiro, 15 de fevereiro de 1850, p. 1-3, grifo
  nosso.}
\end{quote}

No mês de março, a esposa do deputado adoece atingida pela febre e ele
se ausenta da Câmara. Um irmão seu, deputado por Minas Gerais, Antonio
Cândido, também adoece de febre, e morre. Os trabalhos na Câmara
estariam em breve suspensos em função dos estragos realizados. ``A
enfermidade já se tinha emancipado, tinha já principiado seu passeio
terrível pelas ruas''.\footnote{Roberto Lallement,
  \textit{Observações acêrca da epidemia da febre-amarela}, 1851, citado
  emOdair Franco, \textit{História da Febre Amarela no Brasil}, 1969,
  p.~38.}

Nesse mesmo 1850, do outro lado do Atlântico, dois médicos franceses
travam um duelo na \textit{Revue médicale française et étrangère, journal
des progrès de la médicine hippocratique} pela verdade etiológica da
febre amarela, doença que o Dr.~Sigaud, médico do Imperador,
classificara como ``patologia intertropical''.\footnote{Joseph François
  Xavier Sigaud, \textit{Du Climat et des Maladies du Brésil} (Paris,
  Fortin, Masson et Cie, Libraires, 1844), p.~215. Tradução nossa.} A
par da eclosão da epidemia no Brasil, um desses senhores, o
Dr.~Dourand-Fardel, traz a seguinte consideração:

\begin{quote}
Eu gostaria de investigar se, do ponto de vista das grandes epidemias, a
peste não é efetivamente para o ocidente o que o tifo é para os climas
temperados e frios, e o que a febre amarela é nos climas intertropicais,
isto é, a forma especial que de acordo com o clima, engendra
espontaneamente influências análogas, de corrupção ou de indiferença nos
seres organizados, e consequentemente doenças infecciosas.\footnote{Maxime
  Durand-Fardel, ``Des maladies contagieuses et infectieuses. A propos
  d'un mémoire de M. Audouard'', \textit{Revue médicale française et
  étrangère, journal des progrès de la médicine hippocratique}, t. II (
  Paris, 1850), p.~653, tradução nossa.}
\end{quote}

As falas de Paula Candido e Dourand-Fardel apontam para a repetição de
certa experiência da epidemia, solo comum (que ajuda a esclarecer o
porquê de o contágio ser questão inofensiva) que permitirá em seu
fechamento a conjunção de um princípio geográfico e certa etiologia
miasmática da doença: ``as manifestações febris no caso das chamadas
febres essenciais (remitentes, intermitentes e perniciosas) não possuíam
uma base anatômica clara e suas lesões eram secundárias e sintomáticas,
o que sugeriria uma etiologia especial''.\footnote{Flavio C. Edler,
  \textit{A Medicina no Brasil Imperial: clima, parasitas e patologia
  tropical}, 2011, p.~94.} Mas há um tema mais decisivo em jogo
a respeito da epidemia, pedra de toque para algo que supomos importante.
A epidemia, para a patologia de então --- a epidemia, pensamos, e não o
sentido tardio atribuído à palavra e ao conceito ---, não é o estado que
uma forma particular de doença atinge quando ganha proporções, também
não é modo de ser de uma doença ou modo autônomo de percepção de tal ou
qual enfermidade. Em \textit{O nascimento da clínica} Foucault argumenta
como o suporte dessa experiência da epidemia não é um tipo específico de
doença, ``mas um núcleo de circunstâncias. A essência da epidemia não é
a peste ou o catarro; é Marselha em 1721, é Bicêtre em 1780; é Ruão em
1769''.\footnote{Michel Foucault, \textit{O Nascimento da Clínica}
  {[}1963{]}, 1977, p.~26.} Emendamos: é o Rio de Janeiro de 1850?
Primeiramente a febre não carrega o sentido de sintoma, ela não é
sinônimo da pirexia (estado mórbido de elevação anormal da temperatura
do corpo). Era a febre epidêmica não um signo clínico que remetia a
alguma doença indeterminada. A febre não era um sintoma principal, não
era evidência próxima da essência patológica, era algo pelo qual se
reconhecia uma série de doenças epidêmicas que recebiam o nome de
``febre''.\footnote{Cf. \textit{ibidem}, p.~183.}

Uma doença --- para o neo-hipocratismo do século \textsc{xviii} --- era o esforço
ativado pela natureza para evacuar uma matéria mórbida. Se essa matéria
deriva em parte de certas partículas do ar que não são análogas aos
humores, que se insinuam sobre o corpo e se misturam com o sangue
corrompendo-o, no que diz respeito especialmente às \textit{febres},
deve-se interpretá-la como um recurso ``para separar do sangue as
partículas que o infectam, e evacuá-las pelo curso do ventre, pelas
erupções ou outras vias''.\footnote{Thomas Sydenham, \textit{Médicine
  Pratique} (Paris, chez Théophile Barrois le jeune, 1784), p.~3, n.~4,
  tradução nossa.} Entretanto, caso a febre, ou doença específica, ocupe
um lugar no rol de uma dada circunstância sanitária, adquire nessa
passagem a essência da morbidez epidêmica. Na temporada que marca o
fenômeno epidêmico, a epidemia ganha status de uma entidade nosológica
com uma história própria, de maneira que o lugar de origem da verdade de
uma doença haverá de ser a ``constituição epidêmica'' na sua condição de
acontecimento datado. Ou seja, a verdade da enfermidade epidêmica não
pertence a lesões ou agentes microbióticos. A produção das diversas
febres, uma vez que epidêmicas, não é atribuída a índices mórbidos
alojados nos órgãos e tecidos; ela tampouco se confunde com causas
atmosféricas ou com o atributo de a muitos encomendar a sepultura.

No curso da epidemia, era o território que assumia a constituição
epidêmica. Era a cidade, em sua forçosa materialidade, mas apenas sob a
perspectiva de ser um território dinâmico, território, e não tanto o
\textit{ethos} do indivíduo sitiado, que emerge como objeto da disciplina
médica. Para que haja epidemia, encomenda-se que mate a muitos, sendo
dos mortos a maior parte, mas, para que a essência perniciosa esteja
instalada, basta que o território tenha assumido naquele ano um
determinado perfil sanitário.

Em 1852, quando das idas e vindas dos surtos de febre amarela que se
arrastarão pelo menos até 1857, o procônsul britânico John J. C.
Westwood certifica à Vossa Majestade da ineficiência das práticas de
segregação, alvos permanentes do desespero da classe mercantil. Uma
economia de estrutura agrária e exportadora, orgulhosa do salto
desenvolvimentista ensaiado na época áurea dos saquaremas, não sairia
ilesa após assaltos periódicos de epidemias. O ano de 1850 é marcado não
só pela Lei Eusébio de Queirós, que proíbe tráfico de africanos
escravizados no Império, ou pela regulamentação do acesso à terra e o
fim do regime de posses na letra da Lei de Terras: nesse ano terá também
início a primeira linha regular de vapores para Grã-Bretanha partindo do
Brasil, privilégio entregue à inglesa \textit{Royal Mail Steam Packet
Company}. Ora, somente no ano da grande epidemia, a Inglaterra remetera
3.000.000 de libras de produtos manufaturados para o Império. O café é a
cultura brasileira que melhor navega o Atlântico desde a década de 1840,
coisa que não se explica apenas pelo consumidor inglês. A maior parte
das exportações de café escoava para os Estados Unidos. É, por sua vez,
de Nova Orleans, em 30 de setembro de 1849, que chegara à Bahia o brigue
com o gérmen da epidemia, conforme exaustivamente notificado pela
imprensa. A febre amarela manterá de sobreaviso a gestão de equilíbrio
da balança comercial por longa data. Mais tarde, no ano de 1872, boatos
de que uma nova epidemia de febre amarela reinava em terras imperiais
acabaria por gerar não só embargos comerciais, mas verdadeiros embaraços
diplomáticos com Argentina e Uruguai. Via de regra o estado de
quarentena nunca deixou de ameaçar interesses de negociantes britânicos,
traficantes de escravizados e latifundiários nacionais, estes quase
sempre atolados em dívidas com casas bancárias estrangeiras que
forneciam crédito para custeio da lavoura de café e cana.\footnote{Cf.
  Raimundo Faoro, \textit{Os donos do poder: formação do patropato
  polótico brasileiro vol.~2} {[}1958{]}, 10ª Ed. (São Paulo, Globo,
  2000), p.~3-43.} Daí a vontade do legislativo, nem sempre realizada,
pela suavização de práticas policiais de quarentena no porto a partir
desse período. A presença da polícia na zona portuária é comercialmente
indesejada. É então de se esperar que o comentário do procônsul John J.
C. Westwood não esteja livre de suspeitas, visto o papel da Inglaterra
no monopólio do mercado mundial e, especialmente, na restruturação de um
novo pacto colonial\footnote{``A \textit{preeminência britânica} motivaria
  o comentário de Sérgio Teixeira de Macedo, ministro brasileiro na
  Grã-Bretanha, em 1854, de que `o comércio entre dois países é
  movimentado pelo capital inglês, em navios ingleses e por firmas
  inglesas. Os lucros, (\ldots{}) os juros sobre o capital, (\ldots{}) o
  pagamento dos prêmios de seguros, as comissões e os dividendos
  provindos das operações financeiras, tudo é carreado para o bolso dos
  ingleses'.'' Ilmar Rohloff de Mattos, \textit{O tempo saquarema} (São
  Paulo, HUCITEC, 1987), p.~16.} --- dentro da ordem capitalista
oitocentista --- com a economia brasileira. No relatório do consulado
publicado pela Comissão Geral de Saúde, Westwood argumenta que qualquer
que venha a ser a natureza da febre amarela, ela ``é local e endêmica na
sua origem''. Ou seja, não há argumento que sustente o juízo de que a
epidemia fora importada: ``julgamos que ora é opinião geral e unânime de
acordo com o que se tem demonstrado'' que \textit{quarentenas} e
\textit{cordões sanitários} não oferecem proteção real contra a introdução
e o desenvolvimento da febre amarela. Por quê? Pois ``as condições que
influem na localização da febre amarela são sabidas, definidas e em
grande parte removíveis'' --- e como se parafraseasse Paula Cândido:
essas condições ``são substancialmente as mesmas que as causas locais da
Cólera e de todas as outras moléstias epidêmicas''.\footnote{BR RJAGCRJ
  8.3.7 Fundo Câmara Municipal --- Série Higiene Pública (Higiene e Saúde
  Pública / Avisos / 1850-1854), p.~660-2. Grifo nosso.} O que nos
permite então traçar esse paralelismo entre as causas locais da Cólera,
as causas locais do tifo nos climas temperados, o diagrama da febre
amarela nos climas intertropicais?

Há casos regulares das doenças epidêmicas acompanhados dos mesmos
sintomas e fenômenos em cada organismo --- essas são epidemias que, ``por
um instinto secreto da Natureza, a exemplo de certos pássaros e de
certas plantas, acompanham tempos particulares do ano''.\footnote{Thomas
  Sydenham, \textit{Médicine Pratique}, 1784, p.~\textsc{xxiii}, n.11. Tradução
  nossa.} Há, do contrário, surtos esporádicos de epidemias de
disenteria, câmaras de sangue, diarreia ou tifo que parecem simular vez
em quando a providência de uma praga bíblica. Falar em ``estado
sanitário'' local era falar das epidemias que grassaram uma província,
ou seja, quando um órgão do governo emitia um relatório anual sobre o
estado sanitário de alguma província do Império, o que se fazia era
notificar a respeito de quais moléstias adquiriram caráter epidêmico
naquele ano, o número das vítimas, o grau de intensidade do achaque
etc.\footnote{Ver os relatórios sobre o estado sanitário das Províncias
  do Maranhão, Pará, Bahia, Pernambuco e Sergipe durante o ano de 1856,
  escritos pelas respectivas Comissões d'Higiene à Junta Central
  d'Higiene Pública e reproduzidos pelo presidente da Junta, Dr.
  Francisco Paula Candido, 15 de Abril de 1857. (Arquivo Nacional. MAÇO
  IS 4-24, Série Saúde --- Higiene e Saúde Pública --- Instituto Oswaldo
  Cruz).} O ``estado sanitário'' de uma cidade não correspondia à causa
ou condição propícia para a moléstia, mas à própria constituição de uma
epidemia que variava, obedecendo ou não a ciclos. Dentro desse contexto,
até mesmo uma cifra estatística da ``salubridade'' de um país como o
Brasil se fazia calcular de acordo apenas com o número proporcional de
doentes e mortos comparado ao índice de doentes e mortos de outro país
tomado como unidade de medida.\footnote{Cf. Flavio C. Edler, \textit{A
  Medicina no Brasil Imperial}, 2011, p. 66.}

Nessa dinâmica se insere um importante nome da medicina do séc. \textsc{xviii}, o
``Hipócrates Inglês''\footnote{José Pereira Rego, \textit{Historia e
  Descripção da Febre Amarella Epidemica que grassou no Rio de Janeiro
  em 1850} (Rio de Janeiro, Typographia de F. de Paula Brito, 1851),
  p.~66.} como bem o diz Dr.~José Pereira Rego: Thomas Sydenham.
Foucault acomoda Sydenham como fundador do pensamento da medicina
classificatória\footnote{Cf. Michel Foucault, \textit{O Nascimento da
  Clínica} {[}1963{]}, 1977, p. 23.} no século \textsc{xviii}, mas é ele também
talvez o pilar desse sistema de pensamento que alguns historiadores da
ciência identificam como a ``topografia médica''. Só que apesar da
alcunha de discípulo de Hipócrates, ele parece ter medido os limites
pessoais do seu hipocratismo. Sydenham empenhou-se em observar por um
tempo, embora inutilmente, as constituições meteorológicas para deduzir
constituições médico-epidêmicas. Não podendo encontrar nas primeiras a
causa completa das segundas, buscou nas qualidades ocultas e secretas do
ar um \textit{divinum quid} capaz de estabelecer isso que quis chamar de
\textit{constituição geral} ou \textit{constituições anuais,} uma vez que a
cada ano se produz particularmente uma ou outra epidemia como resultado
da alteração secreta ou degeneração das qualidades do ar. Diferentemente
do pensamento médico hipocrático, atribui-se à constituição geral ``o
poder de dominar todas as outras doenças imprimindo-lhes seu gênio
particular, de sorte que, durante o reino de uma constituição epidêmica
inflamatória, todas as doenças assumirão esse tipo''.\footnote{Adolphe
  Motard, \textit{Traité d'Hygiène Générale --- Tome Second} (Paris, J. B.
  Baillière et Fils, 1868), p.~527-8. Tradução nossa.} Por exemplo, uma
pneumonia poderá tornar-se, nesse sistema, ``inflamatória, biliosa,
pútrida, e reclamar tratamentos diferentes. Esse mesmo gênio epidêmico
poderá assim criar, em cada detalhe, doenças especiais, que imprimem
então seus caracteres em todas as doenças sazonais''.\footnote{\textit{Ibidem,}
  p.~527-8}

No vocabulário taxonômico que é caro a Sydenham, as doenças epidêmicas
formam uma \textit{família}, que se divide em duas \textit{classes} em
função dos equinócios: doenças de primavera e de outono. A \textit{cholera
morbus}, por exemplo, é uma \textit{espécie} que integra as doenças
epidêmicas de outono, porque inaugura seu domínio no mês de agosto e
dura o mês. Mas particularmente, no tocante às febres,

\begin{quote}
(\ldots{}) a maior parte das que são \textit{contínuas} não possuem nenhum
nome particular, uma vez que dependem da constituição geral. Os nomes
que as distinguem entre si são tomados de alguma alteração considerável
no sangue ou de algum sintoma mais evidente. É só neste sentido que são
chamadas \textit{pútridas, malignas} etc. Mas já que ordinariamente cada
constituição, além das febres que provoca, tende a provocar ao mesmo
tempo outra doença mais epidêmica e de maior consequência, tais como a
peste, a varíola e a disenteria, eu não sei por que essas febres não
derivariam seus nomes da constituição que as faz eclodir, e sim uma
alteração qualquer do sangue ou sintoma particular que pode ser
encontrados igualmente em febres de uma outra espécie.\footnote{Thomas
  Sydenham, \textit{Médicine Pratique}, 1784, p.~8, n.~13. Grifo e
  tradução nossos.}
\end{quote}

As epidemias não estão ligadas a causas mórbidas produzidas pelo
organismo. É coisa evidente, dirá Sydenham, que todo homem que for para
as regiões onde reina uma febre epidêmica será atacado ao fim de alguns
dias, goze ele ou não da mais perfeita saúde do mundo.\footnote{Cf.
  Thomas Sydenham, \textit{Médicine Pratique}, 1784, p.~11, n.19. Tradução
  nossa.} A febre epidêmica pertence à lógica das constituições
variáveis que qualificam cada ano em sua inconfundível especificidade.
Uma constituição geral epidêmica não tem origem no calor, no frio ou na
umidade, mas depende de ``mudanças ocultas e inexplicáveis dentro das
entranhas da terra. O ar se torna infectado de perniciosas exalações que
causam esta ou aquela enfermidade''.\footnote{Thomas Sydenham, \textit{The
  Works --- Vol. 1}, translated from the latin ed.~by R. G. Latham
  (London, printed for the Sydenham Society, 1848), p.~33-4, §5.} Essa
sorte de doença que reina durante uma constituição do ar específica e em
nenhuma outra é justamente a das doenças denominadas ``epidêmicas''.
Elas derivam de alguma alteração secreta e inexplicável do ar que
infecta o sangue. Isso significa que a epidemia não depende de alguma
qualidade particular do sangue e dos humores, senão ao longo do tempo
que dura o contágio, se por contágio entendermos a ação através da qual
o ar infectado imprime uma qualidade perniciosa no sangue e nos humores.
As doenças vêm em parte de certas partículas do ar que não são análogas
aos humores e que se insinuam sobre o corpo, e se misturam com o sangue
infectando-o e corrompendo-o.

Após os ciclos dos anos, no curso dos quais a influência de uma
constituição geral reina em uma cidade, um novo espaço-tempo se instaura
e sob nova superfície emerge uma nova constituição geral. ``Cada qual e
todas estas constituições gerais assumem a melancólica característica de
alguma forma adequada e peculiar da febre; forma que em nenhum outro
período será igual''.\footnote{\textit{Ibidem,} p.~33-4 (§5)}

Thomas Sydenham converge em sua \textit{Medicina Prática} duas tradições
do pensamento de origem diversa: em alguma medida a medicina grega
hipocrática (para quem os fatores cósmicos são de primeira importância
na arte de curar); e as Histórias Naturais do século \textsc{xviii}, cuja
cientificidade é pautada pela reprodução do imperativo cartesiano da
medida e da classificação. Importante é que também, entre os
naturalistas anteriores a Cuvier e Darwin --- para invocarmos um Buffon
com símbolo ---, a noção de clima traria consigo um componente
cosmológico que, por sua influência, estremeceria o equilíbrio e a
individuação das espécies. Em Buffon, a má influência do meio\footnote{Cf.
  Claudio Medeiros, \textit{O devir do conceito de ``meio'' entre os
  séculos \textsc{xvii} e \textsc{xix}, segundo a História das Ciências de Georges
  Canguilhem}, Mestrado em Filosofia (São Paulo, PUC-SP, 2014), p.
  106-38.} sobre o organismo desdobra-se na ação desordenada do clima,
da alimentação e dos costumes sobre as moléculas orgânicas, naquilo que
deveria ser, do contrário, a manutenção da espécie e da hierarquia dos
viventes regida pela obediência ao ``molde interior'' (algo como um
princípio de individuação da espécie). Logo, os caracteres degenerados
pela influência do meio, ``sendo em seguida perpetuadas pela geração,
tornaram-se caracteres gerais e constantes, através dos quais nós
reconhecemos as raças e mesmo as diferentes nações que compõe o gênero
humano''.\footnote{Georges Louis Leclerc de Buffon, \textit{Histoire
  Naturelle, générale et particuliere, avec la description du Cabinet du
  Roy, Tome Quatorzième} (Paris, de l'Imprimerie Royale, 1749), p.~316.
  Tradução nossa.} Essa noção negativa e diferencial do clima, própria
do séc. \textsc{xviii}, não serviu à necessidade de definição em termos genéticos
das adaptações e convergências. Ela pleiteou explicar a cor da pele dos
povos ameríndios (Buffon); as paixões do coração, a moral, e, portanto,
o espírito geral de uma nação como elemento determinante das leis e
estilos de governo (Montesquieu); as pestes ou as doenças epidêmicas
pela introdução do conceito de \textit{aclimatação} (Boudin).

Naquela primeira quinzena de fevereiro de 1850, como a peste, para
ferir, não fazia discriminação social --- como diz Lallement ---, e
``exercendo assim o socialismo mais genuíno'' contabilizava entre 80 e
90 vítimas por dia,\footnote{Cf. Odair Franco, \textit{História da Febre
  Amarela no Brasil}, 1969, p. 39-40.} o Ministro dos Negócios do
Império, Visconde de Mont'Alegre, nomeou dez médicos dentre os mais
distintos da Corte\footnote{Compondo a mesa perfilavam personagens como
  José Pereira Rêgo, José Francisco Xavier Sigaud, Roberto Jorge Haddock
  Lôbo e José Maria de Noronha Feital.} para a criação de uma Comissão
Central de Saúde Pública. Em caráter de emergência o governo financia
uma Comissão que frequentemente se reúne para deliberar medidas
necessárias a fim que se evite a propagação do mal e em busca de meios
para remediá-lo. Imediatamente jornais publicam os ``Conselhos às
Famílias sobre o comportamento que devem observar durante a epidemia'':

\begin{quote}
Para tranquilizar o espírito do povo, a comissão declara que a febre
amarela, que principia a reinar epidemicamente nesta cidade, acomete de
preferência as pessoas recém-chegadas de países estrangeiros,
marinheiros e outros indivíduos não aclimatados ou não habituados às
influências de temperatura e outras especiais ao clima do nosso país:
que ela se desenvolve a bordo dos navios e em terra, nos lugares onde
costumam reunir-se e pernoitarem marinheiros, como se observa em certas
casas da Rua da Misericórdia e praia de D. Manoel; que nas pessoas
nacionais e estrangeiras já aclimatadas residentes nesses distritos a
febre apresenta-se benigna e pouco caracterizada.\footnote{``PARTE
  OFICIAL. Ministério do Império. Conselhos às Famílias sobre o
  comportamento que devem observar durante a epidemia'', \textit{Jornal do
  Commercio}, Rio de Janeiro, 15 de fevereiro de 1850, p. 1.}
\end{quote}

Semelhante às plantas encontradas em todos os lugares do mundo, ao passo
que em algumas zonas circunscritas existe uma flora nacional que convive
ao largo das ervas que crescem de maneira endêmica, as doenças do homem
são, elas também, ou disseminadas sobre toda a superfície da terra, ou
ligadas a zonas e localidades. É preciso destacar: a distribuição
geográfica da doença pretendida por Boudin em seu \textit{Tratado de
Geografia e de Estatística Médicas e das Doenças Endêmicas} (1857)
difere razoavelmente da representação sazonal da doença em Sydenham. A
imagem sazonal ilustra a intermitência de certas epidemias: ``a exemplo
de certos pássaros e de certas plantas, acompanham tempos particulares
do ano''.\footnote{Thomas Sydenham, \textit{Médicine Pratique}, 1784,
  p.~\textsc{xxiii}, n.11.} Não se trata de dizer que o conceito de
``aclimatação'', às custas da miopia de um neo-hipocratismo, preexistia
nos trabalhos de Sydenham. É evidente que, quando determina que os
estrangeiros não aclimatados são mais sensíveis à epidemia, a Comissão
Central de Saúde Pública inova justo pela tênue descontinuidade entre a
``aclimatação'' com o acento na patologização de um fator topográfico
passível de intervenção (o relevo, o clima, a urbanização insuficiente)
e a ``aclimatação'' em sentido diverso, de caráter antropológico e
normativo, com ênfase na raça que congrega em si as circunstâncias que
predeterminam a frequência ou raridade da manifestação de uma doença.
Entre as duas acepções do conceito de aclimatação, insiste, com maior ou
menor complexidade, o mais bruto colonialismo. Para Boudin, que era
médico de uma Armada francesa ocupada com a burocracia colonial, a
aclimatação do homem era um problema do domínio do projeto imperial,
envolvendo, por exemplo, a escolha das raças recrutadas para as tropas
que serviriam em campos de batalha longe da metrópole; a aclimatação era
igualmente tema de economia política, uma vez que forneceria ao
parlamentar base experimental adequada para as instituições de
quarentena.

Há tipos de raças que parecem se adaptar maravilhosamente bem às
mudanças do clima, enquanto outras suportam a duras penas os menores
deslocamentos, o bastante para que o médico e militar francês acrescente
``que o negro jamais consegue se aclimatar fisicamente e a perpetuar sua
raça fora dos trópicos, esta mudança de clima parece trazer graves danos
a suas faculdades intelectuais''.\footnote{Jean Christian Marc BOUDIN,
  \textit{Traité de géographie et de statistique médicales et des maladies
  endémiques} (Paris, J.-B. Baillière et Fils, 1857), p.~\textsc{xxxvii}.
  Tradução nossa.} De que maneira a distribuição geográfica das doenças
passa a ser de interesse da higiene pública é tema que exigirá nosso
cuidado adiante. O que cabe antecipar é que a transplantação do programa
higienista das metrópoles europeias para um país da periferia do
capitalismo se dá por uma descontinuidade histórica entre os fatores
comumente considerados produtores de epidemias. Aos ares, às águas e
lugares serão incrementados alguns aspectos urbanísticos, sociais e,
tardiamente, higiênico-normativos. Segundo Edler, quando ``a obra de
Boudin veio a público, a estatística já se consolidara como o principal
instrumento metodológico da saúde pública francesa, provendo fortes
evidências em favor da teoria social de causação das
doenças''.\footnote{Flavio C. Edler, \textit{A Medicina no Brasil
  Imperial}, 2011, p.~64-5.} Basta por ora que os juízos da Comissão
Central de Saúde Pública sobre a predileção da epidemia pelos
estrangeiros não aclimatados e o conceito de constituição geral
epidêmica, colhido de Sydenham e enriquecido por alguns hábeis médicos
da Corte, lancem luz sobre os episódios de 1850. Roberto Lallement
descreveu assim a propagação da epidemia:

\begin{quote}
Como um raio no céu azul, caia em geral a febre-amarela, sobre o povo.
Quando os marinheiros estavam carregando os seus navios, quando os
negociantes iam à Praça do Comércio, quando os oficiais seguiam seu
trabalho e os pretos puxavam suas carroças e levavam o café, pelas ruas,
neste instante mesmo, de repente, aparecia uma horripilação, mais ou
menos forte, um frio e a febre se manifestava.\footnote{Roberto
  Lallement, \textit{Observações acêrca da epidemia da
  febre-amarela}, 1851, citado em Odair Franco, \textit{História da Febre
  Amarela no Brasil}, 1969, p.~38-9.}
\end{quote}

Veio-nos a peste de presente por uma barca dinamarquesa de nome
\textit{Navarre}, que rápida paragem fizera na província da Bahia e que
aportou à nossa baía no 3 de dezembro de 1849. Nada constando sobre o
risco oferecido pelos tripulantes, ou pelos vapores de miasmas que
transpiravam nas madeiras podres do calabouço dos navios (exalações
emanadas nos lugares da decomposição de corpos orgânicos, as quais
desenvolvem certos gases\footnote{Cf. ``Camara dos Srs. Deputados.
  Sessão em 12 de fevereiro de 1850'', \textit{Jornal do Commercio}, Rio
  de Janeiro, 15 de fevereiro de 1850, p.~1-3.}), teve a embarcação
livre prática no porto. Assim que o consignatário da barca viu a
enfermidade que grassava a bordo, estremeceu e tratou às pressas de
vendê-la, e a tripulação dispersou-se. Alguns marinheiros passaram para
outros navios, alguns correram para terra e foram morar em uma
\textit{public house} mantida por um francês de nome Frank, na Rua da
Misericórdia. Os que moravam próximo das praias em geral, mormente
daquelas que ficam vizinhas dos ancoradouros, e bem assim aqueles que
residiam nas ruas da Misericórdia, S. José, Direita e becos adjacentes,
foram as vítimas prediletas do achaque. O Hotel de Neptuno, na Rua da
Misericórdia, a tal \textit{public house} de Jack, defronte da de Neptuno,
o Hotel da Califórnia, na Rua Fresca, e a casa de New York no Beco do
Cotovelo eram alguns desses estabelecimentos, junto à orla da baía, que
careciam da boa fama junto à autoridade policial. A ruidosa clientela de
marinheiros de navios de guerra e mercantes, cada qual a se servir de
idioma próprio, consumia em semelhantes casas o duplo serviço de
inferninho e pernoite. Francisco Gonçalves Martins, presidente da Junta
de Higiene Pública no ano de 1852, solicitará auxílio policial ``contra
a existências de certas casas de hospedaria'', cujos hóspedes, ``os
quais entregando-se a repetidas orgias, saem dali afetados da febre
amarela, e uma grande parte deles é vítimas de tais excessos e do mal
que em semelhantes casas parece estar localizado''.\footnote{BR RJAGCRJ
  8.3.7 Fundo Câmara Municipal --- Série Higiene Pública (Higiene e Saúde
  Pública / Avisos / 1850-1854), p.~587.} Desse sítio teriam remetido o
primeiro doente para a Santa Casa, um marinheiro dinamarquês, logo
seguido de outros russos:

\begin{quote}
A 28 de dezembro, quando o médico alemão Roberto Cristiano Bertoldo
Lallement fazia a visita habitual na enfermaria dos estrangeiros no
Hospital da Santa Casa, sua atenção voltou-se para dois doentes, os
marinheiros Anderson e Enquist, que estavam febris, ictéricos, vomitando
um líquido escuro; tinham soluços, oligúria. Um morreu à noite; o outro,
no dia seguinte. O sueco Anderson, ex-tripulante do ``Navarre'', morava
na hospedaria de um tal Frank; o finlandês Enquist, que viera no brigue
russo ``Wolga'', hospedara-se numa casa da ladeira do Castelo, que
ficava atrás daquela hospedaria, e era freqüentador da estalagem de
Frank.\footnote{Odair Franco, \textit{História da Febre Amarela no
  Brasil}, 1969, p.~35.}
\end{quote}

Segundo a opinião da Comissão Geral de Saúde Pública, a moléstia
apresentava duas divisões bem distintas, ora atacando nacionais e
aclimados, ora os recém-chegados. No primeiro caso era de natureza
benigna e de diagnóstico pouco preciso, no segundo era bastante grave.
Consta que a moléstia fixava-se particularmente nos centros nervosos e
no fígado --- ordinariamente o órgão que mais sofre. Que os nacionais
poderiam vir a sentir calafrios, dores de cabeça, ou sobre os olhos,
tonteiras, dores contusivas pelo corpo, costas, lombos ou cadeiras,
fraqueza geral, inapetência, dores pelo ventre, pulso cheio e duro.
``Que estes sintomas duram de 12 a 60 horas, (\ldots{}) sucedendo-lhes o
restabelecimento do doente com apenas falta de apetite e algum
abatimento do corpo''.\footnote{``Publicações a pedido. A febre
  reinante'', \textit{Jornal do Commercio}, Rio de Janeiro, 7 de março de
  1850, p.~2.} Conforme, os doentes que no primeiro perfil da moléstia
se apresentavam eram imediatamente sangrados, ``em seguida tomavam óleo
de rícino e uma infusão de flores de borragem com duas oitavas de
acetato de amônia; adicionando sinapismos ou banhos de pés com mostarda.
Nos casos mais simples limitava-me ao óleo e bebida
sudorífica''.\footnote{``Estatística dos doentes da febre amarela que se
  trataram nas enfermarias a cargo do 1º cirurgião do hospital da
  marinha até 31 de março'', \textit{Jornal do Commercio}, Rio de Janeiro,
  8 de maio de 1850, p. 2.} Nos estrangeiros que davam entrada no
Hospital da Marinha sob o segundo aspecto da moléstia, ou quando no
hospital passavam a esse estado, os vômitos verde escuros, cor de café
ou pretos, a boca pastosa ou amarga, as náuseas, a falta de secreção de
urinas, a pele seca e quente, as hemorragias, a língua seca, o aspecto
tifoide e ``a icterícia etc., mostram-se comumente nelas, mas não em
todos os casos, faltando várias vezes a icterícia, que de ordinário se
estabelece em o segundo período, e que muitas vezes só se desenvolve
depois da morte''.\footnote{``Publicações a pedido. A febre epidêmica
  reinante é o tifo americano, ou a febre amarela'', \textit{Jornal do
  Commercio}, Rio de Janeiro, 29 de março de 1850, p.~3.} O Dr.~José
Maria de Noronha Feital --- integrante da Comissão e 1º cirurgião do
Hospital da Marinha --- aconselhava, para o tratamento dos que chegavam
em estado terminal, empregar ``as limonadas muriática ou sulfúrica
geladas, o sulfato de quinina interna e externamente, o cozimento
antifebril de Lewis, os banhos tépidos ou frios e os sinapismos; tendo
rara vez lançado mão das ventosas, das sanguessugas ao ânus e dos
cáusticos''.\footnote{``Estatística dos doentes da febre amarela que se
  trataram nas enfermarias a cargo do 1º cirurgião do hospital da
  marinha até 31 de março'', \textit{Jornal do Commercio}, Rio de Janeiro,
  8 de maio de 1850, p. 2.}

No 14 de fevereiro, o Visconde de Mont'Alegre, a mando do Imperador (que
contava apenas 25 anos), remete à Câmara um artigo com
providências\footnote{AGCRJ Códice 43.3.26 --- Fundo Câmara Municipal ---
  Série epidemias (Febre Amarela --- Medidas Higiênicas --- Portaria do
  Ministro do Império Visconde de Monte Alegre, etc. --- 1850), folhas
  1-5.} para evitar a entrada e o reingresso do mal.
Institucionalizava-se o sequestro dos afetados --- esse é o regime de
\textit{quarentena}.

Para ``prevenir e atalhar o progresso da febre amarela'', todos os
navios considerados focos de infecção são ``colocados em lugar afastado,
e a sotavento da cidade, conservando entre si a maior distância
possível''; serão forçados a subir barra afora, a fim de serem
descarregados, lavados e fumegados nas ilhas para isso destinadas. Um
novo lazareto seria construído na Ilha do Bom Jesus (atualmente
integrada à do Fundão) para atender aos fins ditados. Os doentes a bordo
são obrigados a recolherem-se nos referidos lazaretos e não só os
marinheiros, como ``todos os outros moradores no Porto desta Cidade,
serão visitados duas vezes ao dia pelos Médicos (\ldots{}) que
observarão o estado de asseio, e de arejamento, e darão destino aos
doentes que encontrarem''. Além da Santa Casa, do Hospital da Marinha e
de uma enfermaria a ser criada na Rua da Misericórdia, seriam
estabelecidos mais dois lazaretos: o da Rua do Livramento e outro de
``extraordinária mortalidade'', estabelecido na Gamboa pelo Dr.~Peixoto,
do qual se ouviu dizer ``geralmente que quantos entravam para aquele
lazareto de lá iam para o cemitério''.\footnote{``Senado. Sessão de 17
  de abril e 1850'', \textit{Jornal do Commercio}, Rio de Janeiro, 19 de
  abril de 1850, p.~1.} Em todos os aposentos das casas dos doentes da
epidemia reinante serão feitas ``fumigações cloruretadas; e aquele em
que houver permanecido o doente, será mais que todos lavado, caiado e
fumegado''.

A prática de controle da epidemia, entregue nas mãos de uma polícia
médica especializada, passará por uma terapêutica do território. Se a
atmosfera foi infectada, urgente é transportar os doentes dos navios
para lazaretos nas áreas extremas da cidade e expô-los aos ares salubres
das ilhas. Anular momentaneamente a constituição geral epidêmica no
território se faz acendendo fogueiras de lenha, alcatrão e aroeira nas
praias e sobre as sepulturas dos infelizes, encomendando meios adequados
e espaços exclusivos para a sepultura dos infelizes, disparando tiros de
canhão para purificar os ares, incinerando roupas, móveis e pertences
das vítimas da epidemia, lançando cal virgem sobre o chão das casas,
baixando normas de asseio corporal e temperança alimentar, proibindo
amancebamentos públicos para que não se desperte a doença pela via dos
abusos venéreos, para que não se deboche demais da ira divina.

Atormentada era a impressão que encontrava o navio que tentasse aportar:
as fogueiras nas praias e os fios de fumaça subindo dos arrabaldes, a
cidade febril sob o sol de fevereiro; o efeito do creosote, da
terebintina, a ação enérgica das águas de Labarraque, que, ao serem
aplicados para frear a decomposição dos miasmas, ardiam as vistas já
lacrimejadas. ``Qual é a família que não vê assustada escoarem-se os
dias, acreditando sempre que o dia seguinte pode talvez ser de luto?
Quantos não veem os seus últimos recursos exaustos, e choram na
impossibilidade de acudirem às mais urgentes necessidades de suas
famílias?''\footnote{``Comunicado'', \textit{Diário do Rio de Janeiro},
  Rio de Janeiro, 24 de abril de 1850, p.~2.} O aspecto de descalabro
geral, a sensação de indolência nos transeuntes, quando não as ruas
desertas, pois o Visconde orientava que os mendigos fossem recolhidos,
que os exercícios militares fossem suspensos, que quaisquer obras que
remexessem as entranhas do solo fossem interrompidas.

Geralmente --- conforme relato de Ribeyrolles em \textit{Brazil
Pittoresco}, de 1859 ---, a cidade ``envenenada pelas infiltrações e
engulhos de suas valas, guarda ainda dentro das casas, e por carregar
através das ruas, outras pestilências''.\footnote{Charles Ribeyrolles,
  \textit{Brazil Pittoresco --- Tomo II} (Rio de Janeiro, Typographia
  Nacional, 1859), p.~43-4.} Não havia poços na Corte de então, e sim
barris; as carroças passavam em certas horas, e o tonel exalando as
águas servidas e as matérias fecais tomava o caminho das praias.
``Quanto ao resto\dots{} lá vai indo até o mar à cabeça dos negros, como um
cesto de laranjas. (\ldots{}) A este pormenor de edilidade chama-se o
serviço de \textit{tigres}. Arreda-se a gente de noite, quando esses
tristes obreiros da labutação imunda se prolongam pelas
ruas''.\footnote{\textit{Ibidem}, p.~43-4.} Capistrano de Abreu em sua
história do Brasil condena o fato do enterro dos ``cadáveres nas
igrejas. Só a pouca população explica a ausência de epidemias. Da
higiene pública incumbiam-se as águas da chuva, os raios do sol e os
diligentes urubus''.\footnote{Capistrano de Abreu, \textit{Capítulos de
  história colonial, 1500-1800} {[}1907{]}, 7ª ed.~(São Paulo,
  Publifolha, 2000), p.~240.} E enquanto a água e os esgotos eram
entregues à iniciativa particular, conclui Ribeyrolles, semeavam os
\textit{tigres} ``a cada passo a vingança, e mais tarde, no encalço do
infecto, chegam as exalações que trazem a morte, febres, tifos e pestes.
Os \textit{tigres} tem seu cortejo!''

O Visconde estabelece paliativos em proveito do asseio público. Ordena
que praias, praças, ruas e cocheiras sejam diariamente limpas das
imundícias. Que os tais pretos seminus, rígidos e firmes, sob os pesados
fardos em seus crânios, fizessem seus despejos ``ao mar o mais longe das
praias que for possível; fazendo-se para isto, o quanto antes, em
diferentes pontos do litoral, pontes estreitas mas de suficiente
extensão''.\footnote{AGCRJ Códice 43.3.26 --- Fundo Câmara Municipal ---
  Série epidemias (Febre Amarela --- Medidas Higiênicas --- Portaria do
  Ministro do Império Visconde de Monte Alegre etc. --- 1850).}
Ordena-se, por fim, que as cadeias fossem alternadamente esvaziadas para
serem consertadas, fazendo asfaltar o solo, caiá-las, lavá-las e
fumegá-las repetidas vezes; e os presos obrigados a lavarem-se a miúdo e
a mudarem roupas, fornecendo-se uma muda aos pobres.

Pobres torravam seus fundos para arcar com as modestas honrarias
fúnebres, e a resolução para muitos corpos pardos atacados da febre eram
valas coletivas --- quando não amanheciam cheirando nas vielas.
Morria-se, e morria-se às claras. O chefe da intendência de polícia
aciona a Câmara pedindo que se mande fixar o preço dos caixões, dos
artigos para enterros, do carreto dos corpos, já que havia quem
especulasse com a dor popular, exigindo o cocheiro dos tílburis fúnebres
uma taxa além da quantia previamente combinada.

O Teatro de S. Pedro, o Teatro São Januário, os demais teatros da
cidade, tudo cheirava à peste --- não havia espetáculo, por melhor que
fosse, que lhe fizesse concorrência. Uma infeliz companhia artística
italiana encontrou quase toda a morte no flagelo. Os abastados da
alfândega, o comissariado, a classe política, a burguesia urbana subiam
para a Tijuca e Petrópolis. Os que cá embaixo penavam cruzavam com
cadáveres nas ruas ou com o cheiro da cera queimando das procissões.
Subtraindo os anos epidêmicos --- em que seria quase inacessível ao pobre
receber algum tipo de sacramento antes de falecer --- o bom católico
costumava morrer assistido por algum vigário, geralmente em seu leito,
geralmente segurando uma vela piedosa e sussurrando nomes do Cristo e da
Virgem. Mas no contexto de 1850, na falta de padres, o cotidiano dos
mortos contados aos milhares seria diferente. ``O pastor fluminense,
ferido do mal comum, jaz no seu leito; o seu vigário geral, acompanhado
do clero das paróquias, o substitui carregando a Imagem do Crucificado
nas numerosas procissões de penitência; o povo aterrado grita
misericórdia''.\footnote{``Correspondência. Febre Amarela'',
  \textit{Diário do Rio de Janeiro}, Rio de Janeiro, 19 de abril de 1850,
  p.~2.} As ordens religiosas convocam preces públicas por três dias
sucessivos e, no fim dessas, uma grave procissão de penitência, à qual
concorriam corporações, confrarias e irmandades, conduzindo em ardor sua
imagem de devoção particular.

\begin{quote}
Era para admirar a concorrência do povo que, em cardumes, logo cedo se
vinha apinhar no templo em todos os dias de preces, regressando muitos
nos dois dias últimos por já não caberem na igreja, apesar de ser
desmedidamente grande, e a maior sem dúvida dessa cidade. (\dots{}) Todos,
com os pés descalços, para mais de quatro mil pessoas sem exageração,
caminhavam, segundo suas antiguidades e hierarquias com o mais profundo
silêncio, indo atrás das alas o clero da cidade de Albados, e com
estelões roxos, e coroas de cordas sobre amictos nas cabeças, e o
capitulante no meio, trajado na mesma forma, mas com estola roxa
pendente.\footnote{``S. João D'El Rei, 17, 18, 19, 20 e 21 de abril'',
  \textit{Jornal do Commercio}, Rio de Janeiro, 8 de maio de 1850, p.~2.}
\end{quote}

A procissão infundiu tristeza e compunção nos ânimos, e todos, ferindo
contritamente os peitos, disciplinando os ombros com veemênica, vertendo
lágrimas de dor e sentimento, temiam o juízo escatológico: ``É o anjo da
morte que Deus enviou a esta cidade, é o enviado da justiça de Deus, que
pairando há dois meses sobre esta população, abaixa o dedo e aponta hoje
sobre estas casas, amanhã sobre aquelas, e os seus moradores caem mortos
ou feridos..''.\footnote{``Correspondência. Febre Amarela'' \textit{Diário
  do Rio de Janeiro}, Rio de Janeiro, 19 de abril de 1850, p.~2.} Mas o
chefe de polícia solicita o controle da consagração religiosa pelo toque
de recolher. As frequentes procissões, da maneira como ocorriam,
demorando-se as multidões nas ruas da cidade desgraçada e viciando o ar
das igrejas, ``não podem ser agradáveis à Infinita Bondade porque eles
tendem pelas leis da natureza a provocar e desenvolver a horrível
enfermidade; o fervor religioso do povo pode e deve ser modificado pela
prudência, segundo os mesmos sãos princípios religiosos''.\footnote{``Comunicados'',
  \textit{Diário do Rio de Janeiro}, Rio de Janeiro, 23 de março de 1850,
  p.~2.}

No espetáculo lúgubre que por alguns meses a cidade estreou não se ouviu
mais o dobre das igrejas, e o povo em luto não tardou em esvaziar
velórios empestados, não armavam mais as portas com safenas, não desciam
dos sobrados. E já que cemitérios eram predominantemente para
protestantes, pagãos, pobres --- e não para quem fosse da religião
oficial e pertencesse à nobreza rural ou à burguesia patriarcal ---,
começavam a rarear igrejas, conventos e capelas particulares para o rito
e sepultamento católicos. No antigo templo cristão, enterra-se pelo solo
e, dependendo da origem do morto, da irmandade religiosa à qual se
pagava donativos, enterra-se ``pelas paredes, debaixo dos altares, por
cima deles, por detrás dos oratórios. (\ldots{}) \textit{Recheio de igreja
é defunto}''.\footnote{Luís Edmundo, \textit{O Rio de Janeiro no tempo dos
  vice-reis --- 1763-1808}, (Brasília, Senado Federal --- Conselho
  Editorial, 2000), p.~81.} Escreveu-se recentemente sobre a cultura
fúnebre brasileira\footnote{Cf. João José Reis, \textit{A morte é
  uma festa: Ritos fúnebres e revolta popular no Brasil do século \textsc{xix}}
  (São Paulo, Companhia das Letras, 1991); e também idem, ``O cotidiano
  da morte no Brasil oitocentista'', em Luiz Felipe de Alencastro,
  \textit{História da vida privada no Brasil/Império} (São Paulo, Cia. das
  Letras, 1997).} que a década de 1850 foi importante para que o Governo
Imperial decidisse sancionar leis e decretos que estabelecessem
cemitérios nos subúrbios da Corte, não apenas para indigentes e
escravizados, mas para mortos em geral. Havia décadas a medicina
acadêmica vinha alardeando os efeitos mórbidos causados pelos cadáveres,
alertando sobre a tarefa de neutralização de suas exalações pútridas com
projetos para cemitérios salubres isolados da rotina urbana. Antes de
1850, no destino reservado para os sepultamentos,

\begin{quote}
os cadáveres ali se atiram a montes em um grande valado; são mal
cobertos de terra e ainda pior socadas as camadas que neles lançam.
Resulta passarem para o ar as matérias gaseificadas dos corpos em
decomposição. Quando os valados abrem, ainda se não acha completada esta
decomposição; os ossos saem ainda pegados pelos ligamentos e a
putrilagem dos outros tecidos brandos sai com lama nas enxadas, lançando
uma prodigiosa quantidade de corpúsculos e emanações pútridas.\footnote{``Relatório
  da Comissão de Salubridade geral da Sociedade de Medicina do Rio de
  Janeiro sobre as causas da infecção da atmosfera da corte, aprovado
  pela mesma Sociedade em 17 de dezembro de 1831'', \textit{Semanário de
  Saúde Pública}, 1832, p.~12, citado em Roberto Machado, \textit{Danação
  da Norma: a medicina social e constituição da psiquiatria no Brasil}
  (Rio de Janeiro, Graal, 1978), p.~289.}
\end{quote}

Tradicionalmente, o que se poderia chamar de cemitério foi um lote
agregado ao edifício hospitalar. Havia o Hospital da Santa Casa de
Misericórdia, mais aparentado a uma instituição filantrópica de
assistência material e espiritual preparada para o pobre, não tanto para
que se fizesse medicalizar, mas para que o pobre tivesse um leito onde
morrer. Em meados do \textsc{xix}, quinta parte dos que para ali se dirigiam
anualmente, na expectativa de alívio para sua moléstia, eram de lá
prontamente direcionados ao túmulo. O hospital é o lugar que restou para
morrer, onde a religião encontrava termo para o indigente, recolhia o
marinheiro estrangeiro acometido de febre, recebia o escravizado
inválido da mão do senhor. Logo, função de transição entre vida e morte,
função de algum precário registro obituário, função de salvação pela
oferta de algum sacramento e, por fim, esta tarefa bastante urgente que
é a encomenda dos meios e destino adequado para o enterro. Para a
sensibilidade popular a experiência do hospital era já como a antecâmara
do túmulo e o túmulo não era outro senão a vala coletiva do tamanho de
nove palmos de largura. Disso surgirão conflitos incitados por forças
policiais, na hora de se fazer arrastar coercitivamente amarelentos para
enfermarias e lazaretos criados por ocasião da epidemia de 1850. Quanto
ao histórico de sepultamentos, havia esse terreno junto ao Morro do
Castelo nos fundos do Hospital Geral da Prata de Santa Luzia,
pertencente à Santa Casa, que durou algum tempo como cemitério. Somente
na década de 1830, quase 30 mil enterros em valas quase à flor da terra.
Em 1830 seria desativado o Cemitério dos Pretos Novos, no Valongo,
especialmente destinado a africanos capturados que, já em terra, ``eram
mortos'' (antes de serem traficados para a lavoura) e ali depositados a
um palmo do chão. Em 1840 o cemitério anexo ao Hospital também viria a
ser desativado.

\begin{quote}
Algumas irmandades religiosas e ordens terceiras, que inumavam os
segmentos mais altos da sociedade nos terrenos adjacentes às igrejas,
ocupavam-se, em alguns casos, do enterro de escravos, porém era bastante
comum seus corpos serem apenas jogados à beira dos caminhos e de praias.
(\ldots{}) Em meados do século \textsc{xix}, a situação do Hospital (\ldots{})
tornara-se insustentável com o cemitério ao seu lado. Os médicos
protestavam violentamente contra a proximidade dos cadáveres, tanto de
mendigos quanto de irmãos da Misericórdia, em número crescente,
comprometendo a salubridade não só do hospital, como também da própria
cidade.\footnote{Tania Andrade Lima, ``De morcegos e caveiras a cruzes e
  livros: a representação da morte nos cemitérios cariocas do século \textsc{xix}
  (estudo de identidade e mobilidade sociais)'', \textit{Anais do Museu
  Paulista}, N. Ser. Vol. 2., São Paulo, Jan/dez. 1994, p.~92.}
\end{quote}

Na década de 1850, junto a viúvos, órfãos, padres e cadáveres, soma-se
ao elenco da morte o nariz do higienista. O caráter escatológico do
regime de práticas funerárias da Corte, expresso na utilização, em
lápides tumulares, de ``signos macabros, mórbidos e sombrios, como
caveiras, morcegos, corujas, serpentes, entre outros que remetem à
consumação dos tempos'',\footnote{Tania Andrade Lima, ``Humores e
  odores: ordem corporal e ordem social no Rio de Janeiro, século \textsc{xix}'',
  \textit{História, Ciências, Saúde --- Manguinhos}, Vol. II (3). Nov.~1995
  --- Fev. 1996, p.~44-5.} a visibilidade de um cadáver chorado por dias
a fio, a prática do luto, tudo isso será objeto de uma inquietação e
suspeita. Uma vez que a encomendação da alma não isenta o corpo morto
das suas imundícies e dos efeitos da putrefação, \textit{ao medo do
inferno soma-se o medo do morto}.

A epidemia de 1850 não reservará apenas lugar distinto para a sepultura:
há algo de fundamental no cotidiano da morte que sofrerá uma mudança
sensível e isso, presumimos, na medida da frequência com que epidemias
mais ou menos avassaladoras varrem a cidade. Tem-se certo número de
pequenos pânicos alimentados pela exposição de cadáveres e pela maneira
como os médicos avaliam nos jornais o efeito nocivo dos odores. Mas o
que tomou de assalto a vida urbana, a ponto de conseguir mover um pouco
o oportunismo de uma classe médica em busca da credibilidade popular,
foi um cio de heroísmo dos higienistas movido por um entusiasmo
possibilitado pela abertura na instância oficial do poder.

Entusiasmo médico-higienista, para não dizer policial, atrelado
primeiramente a uma nova cultura de controle do registro dos óbitos. Nas
\textit{Constituições primeiras do arcebispado da Bahia} --- legislação
eclesiástica do início do \textsc{xviii}, que serviu de base para a implementação
da doutrina católica até o período imperial ---, nenhum defunto poderia
ser enterrado sem antes ser encomendado pelo pároco. Era a
``oficialização da entrega do corpo à Igreja, sendo também uma forma de
o sacerdote garantir o recebimento dos emolumentos pagos pelos parentes
vivos por ocasião dos óbitos, além de ter o controle sobre o registro
das mortes ocorridas em sua paróquia''.\footnote{Claudia Rodrigues e
  Maria da Conceição Franco, ``O Corpo morto e o corpo do morto entre a
  Colônia e o Império'', em Mary Del Priore (org.), \textit{História do
  Corpo no Brasil} (São Paulo, Ed. Unesp, 2011), p.~172.} Quando, em
1850 e 1855, reinam epidemicamente entre nós a febre amarela e a
\textit{cholera morbus}, respectivamente, o Governo Imperial manda
estabelecer, em diferentes pontos da capital --- por proposta da Comissão
Central de Saúde Pública ---, ``postos médicos criados e sustentados pela
Polícia nos quais eram encontrados, em horas determinadas, os médicos
verificadores de óbitos que não se negavam a socorrer também as pessoas
que as procuravam''.\footnote{Arquivo Nacional. MAÇO IS 4-27, Série
  Saúde --- Higiene e Saúde Pública --- Instituto Oswaldo Cruz, sem
  paginação\textit{.}} Esse postos da Polícia com médicos verificadores de
óbitos seriam extintos quando suprimidas as epidemias que reclamavam sua
criação, mas é a partir dessas primeiras decisões que se aperfeiçoram,
ao longo da segunda metade do \textsc{xix}, as estatísticas patológicas e
mortuárias da Corte. Veremos como esses relatórios estatísticos perderão
o caráter de excepcionalidade ligado à irrupção de epidemias,
inscrevendo-se na cultura de competências da Junta de Higiene.\footnote{Um
  médico empregado pelo Ministério percorrerá os hospitais militares,
  religiosos e filantrópicos, coletando as informações sobre o número
  dos óbitos, tal como as causas da morte e do perfil do defunto, de
  modo a compor mensalmente um boletim estatístico entregue à mesa do
  presidente da Junta. A partir de epidemia de 1873 os boletins de
  mortalidade passam a ser encomendados quinzenalmente, e não mais
  mensalmente, como de costume.} Muda-se, portanto, de mãos, da Igreja
para as autoridades policiais, e daqui para uma instituição de saúde
pública responsável pela elaboração de um atestado, sem o qual, pelas
disposições das posturas municipais, ``nenhum corpo pode ser dado à
sepultura''.\footnote{Arquivo Nacional. MAÇO IS 4-27, Série Saúde ---
  Higiene e Saúde Pública --- Instituto Oswaldo Cruz, sem paginação.} De
maneira que, por mais frouxos que sejam ainda os reais efeitos de poder
desta instituição higienista, parecem ter sido tanto o território quanto
o corpo do cadáver --- e não propriamente o corpo higiênico, mas sim o
corpo do cadáver, singularizado de acordo com a qualidade da moléstia,
sua duração e a origem do infeliz --- seus objetos de investimento
pioneiro.

O fato da reorganização de uma matriz normativa de controle obituário em
torno de uma instância política nova não será descolado da efetivação de
certas relações estratificadas que compõem práticas e saberes médicos.
Mas esse complexo de práticas e saberes desenvolve não simplesmente um
novo regime de visibilidade reportado à presença/ausência do cadáver.
Insistimos que a lenta transformação do cotidiano da morte é garantida,
promovida, forjada, por um processo de patologização do cadáver, e por
um processo de patologização dos territórios sujeitos à ação do
princípio miasmático. É pela emergência do ``cadáver'' como um corpo
matriz de uma nova constelação de problematizações, até então
evanescentes ou latentes, que o dispositivo médico-higienista tentará
esticar seus cordões sanitários dos períodos caracterizados por
``constituições epidêmicas'' para outros aspectos mais cotidianos, menos
espetaculares, das formas de morrer.

Mas há um segundo aspecto do entusiasmo policialesco que novamente
aponta para a contribuição do dispositivo médico-higienista na
transformação da cultura fúnebre no Império. Em seu \textit{Os exercícios
da arte de curar no Rio de Janeiro (1828 a 1855)}, a historiadora Tânia
Pimenta narra algumas denúncias contra os excessos de poder ocorridos na
tutela dos cadáveres, por ocasião da epidemia de cólera de 1855:

\begin{quote}
Hipólito de Assis Araújo compartilhou o seu drama com os leitores do
jornal ao enviar uma carta em que narrava os constrangimentos a que fora
submetido. Segundo Hipólito, depois da morte de sua mulher por cólera, a
família foi compelida a deixar a casa para que esta fosse fumigada,
estragando alguns pertences. Para piorar a situação, o enterro ocorreu
apenas três horas e meia depois da morte e nem os familiares puderam
acompanhar o funeral --- chocante para os costumes da época. (\ldots{})
Provavelmente, dramas semelhantes foram vividos por muitos outros que,
no entanto, sem acesso às folhas em circulação, permaneceram incógnitos
sob denúncias genéricas da oposição. Podemos inferir essa situação a
partir da correspondência entre o ministro Coutto Ferraz e o provedor da
Santa Casa, responsável pelos cemitérios públicos da cidade. O último
tranquilizava o primeiro dizendo que desde o aparecimento da cólera, os
corpos das vítimas das epidemias eram conduzidos diretamente para os
``campos santos''. A Misericórdia também deveria seguir a recomendação
do presidente da Junta de fumigar a cama dos falecidos com ácido
sulfuroso, destruir as roupas usadas e envolver o cadáver em cal,
cloreto para então ``depositá-lo convenientemente''.\footnote{Tânia
  Salgado Pimenta, \textit{O exercício da arte de curar no Rio de Janeiro
  (1828 a 1855)}, tese de doutorado (Campinas, UNICAMP, 2003),
  p.~209-10.}
\end{quote}

Somente para efeito de comparação, remeto a um exemplo, um testemunho de
um rito fúnebre, que fora editado em Portugal no século \textsc{xvii}. Trata-se
de um relato originalmente impresso em Roma em um pequeno volume de
1689, \textit{Relação verdadeira da última enfermidade e morte de N.
Santíssimo Padre Inocêncio XI}, que dá notícias de um Papa que sofria de
febres fortíssimas e chagas distribuídas pelo corpo em seus últimos 58
dias de vida. Diz o escrito que logo assim que o corpo do Pontífice
expirou, e na presença de um amplo secto de Cardeais, padres
penitenciários, clérigos, jesuítas, Superiores de três Ordens etc.
entraram os cirurgiões para abrir o cadáver a ser embalsamado e
encontraram pedras nos ``interiores membros'' do Pontífice. Declara, aí,
o narrador: não posso deixar de lembrar que enquanto estavam abrindo o
Santo Cadáver, todos aqueles que estavam presentes ``procuravam adquirir
alguma pequena parte para guardarem como relíquia'', e os da alta esfera
``molhavam os lenços com o seu sangue por devoção, e os que se ocupavam
naquela função não se davam mãos a molhar, mas a ensopar os lenços
beijando-os e venerando-os com grandíssimo sentimento e saudade''. Na
tradição cristã, os pedaços do corpo eram relíquias que atualizavam a
presença do santo entre os fiéis. Naturalmente, a excepcionalidade de
uma liturgia fúnebre dedicada a um chefe de Estado dá margem para
pensarmos se tratar de uma comparação sem lugar. Há, no entanto,
passagens da biografia de Rosa Egipcíaca de Vera Cruz --- religiosa
católica rejeitada pela Igreja e levada como herege à Inquisição no
século \textsc{xviii} --- que afirmam que, ainda em vida, ``suas companheiras de
jornada religiosa recolheram cabelos, água de banho e saliva da santa,
por acreditarem poder preparar poções milagrosas de cura e até mesmo
afastar o diabo utilizando partes do corpo dela''.\footnote{Anderson
  J.M. Oliveira, ``Corpo e santidade na América Portuguesa'', em Mary
  Del Priore (org.), \textit{História do Corpo no Brasil}, 2011, p.~67.}
Não eram excepcionais os valores atribuídos ao corpo santificado. Era
uma sociedade que enquanto aguardava o sepultamento dos seus santos
podia permitir aos devotos não só tocar as feridas e disputar partes do
hábito que vestia seus santos, mas embeber lenços de sangue e beijá-los
como sinal de devoção.

A serenidade no tratamento do corpo de um Pontífice difere da
repugnância diante do corpo atacado por moléstia epidêmica, em
princípio, por aquilo que a epidemia de febre amarela foi capaz de
produzir, que é tomar de assalto a ``conivência pacífica'' com os
avanços cotidianos da morte para transformá-la, a morte, e de modo
insólito, em um fenômeno súbito e repugnante em grandes proporções. Mas
esse é apenas um dos aspectos do problema. Segundo João José Reis --- em
artigo sobre o cotidiano da morte no Brasil em meados do Oitocentos ---
essa conivência proporcionava certo ideal da boa morte. A morte ideal, a
boa morte,

\begin{quote}
não devia ser uma morte solitária, privada. Ela se encontrava mais
integrada ao cotidiano extradoméstico da vida, desenhando uma fronteira
tênue entre o privado e o público. Quando o fim se aproximava, o doente
não se isolava num quarto hospitalar, mas esperava a morte em casa, na
cama em que dormira presidindo a própria morte diante de pessoas que
circulavam incessantemente em torno de seu leito (\dots{}). Reuniam-se
familiares, padres, rezadeiras, conhecidos e desconhecidos. Era como em
Portugal.\footnote{João José Reis, ``O cotidiano da morte no Brasil
  oitocentista'', 1997, p.104.}
\end{quote}

Outrora era o leito de morte um mecanismo de concessão de perdão divino
e reparação moral. Quitação de dívidas com credores, pagamento de
promessas com santos de devoção pessoal, reconhecimento da prole
extraconjugal eram compromissos de preservação da honra e memória do
futuro defunto. Era preciso aprender a morrer como quem se prepara para
entrar na eternidade, saber ordenar o tempo para as despedidas mundanas,
não esquecer pendentes pecados antigos, fazer a confissão da verdade
sobre si. ``A hora da morte não era momento de mentiras porque, se
ludibriar os que ficavam ainda era possível, não o era fazê-lo com o Pai
Eterno, cujo julgamento seria implacável''.\footnote{\textit{Ibidem},
  p.~104.} A vida mal havia estrebuchado no peito do próprio Brás Cubas
de Machado de Assis e ele já prontamente se posicionava: cá estou ``do
outro lado da vida, posso confessar tudo''. As \textit{Memórias Póstumas}
são escritas ``com pachorra, com a pachorra de um homem já desafrontado
da brevidade do século''.\footnote{Machado de Assis, \textit{Memórias
  Póstumas de Brás Cubas; Dom Casmurro} {[}1881; 1889{]} (São Paulo,
  Abril Cultural, 1978), p.~19, cap. IV.} A salvo da censura pela
enunciação da verdade sobre o poder senhorial que o patrocinou em vida,
o defunto falastrão usará da verdade como procedimento que traz às
claras a ideologia saquarema de Casa-Grande. Portanto, meta final de
dizer a verdade sobre si, meta de dizer a verdade sobre a sociedade de
classes que dá coesão à sua visão de mundo sem que lhe seja requisitado
vincular a verdade dita ao sujeito que a profere. Afinal uma das
vantagens de ser um defunto-autor é não mais poder ser objeto de
escândalo. Do leito de morte Machado extrai o cenário confortável para
que Brás Cubas golfasse livremente seu narcisismo de classe senhorial,
pusesse às claras íntimos privilégios de herdeiro, confessasse amores
ilegítimos com Virgília, exibisse descarada complacência com o nepotismo
e o regime de favores dentro do alto funcionalismo imperial. Não há
nada, afinal, ``tão incomensurável como o desdém dos
finados''.\footnote{\textit{Ibidem}, p.~54, cap. \textsc{xxiv}.} Mas ainda não se
trata de dizer que a epidemia de 1850 encomendara apenas lugar distinto
para a sepultura, ou que simbolizara o desencantamento da morte: o ano
de 1850 rompe ou pelo menos modula diferentemente a conivência pacífica
entre mortos e vivos diante do espetáculo da epidemia.

Carl Schilichthorst, que organizou memórias sobre a vida social e
política no Primeiro Reinado, impressionou-se com o ``excessivo
desperdício nos enterros'' católicos. Cobre-se o caixão de ``veludo
preto, ricamente agaloado de ouro. (\ldots{}) Na igreja, colocam-no
aberto sobre uma eça. As pessoas que o acompanham e qualquer outra que
se ache presente recebem uma vela de cera acesa. Começa, então, o ofício
do corpo presente, (\ldots{}) em excelente acompanhamento vocal e
instrumental''.\footnote{Carl Schilichthorst, \textit{O Rio de Janeiro
  como é (1824-1826)}, trad. Emmy D. G. Barroso (Brasília, Senado
  Federal, 2000), p.~122-3.} Era regra enterrar-se ao cair da noite, sob
efeito da luz de tochas e velas. Terminada a prece, lança-se sobre o
defunto água benta e ``uma medida de cal virgem e fecha-se o esquife,
que é metido num dos nichos abertos nas paredes''\footnote{\textit{Ibidem},
  p.~122-3.}. Salvo situações em que a causa da morte pudesse produzir
contágio, era autorizado aos corpos serem conduzidos embalados em redes
às catacumbas das igrejas, que eram cobertas com tábuas de madeira ou
pedra lioz. Essas catacumbas acolhiam vários cadáveres ao longo dos
anos, e eram abertas e reabertas em função do processo de decomposição
para dar lugar às gerações seguintes.

O livro \textit{Lugares dos mortos na cidade dos vivos}\footnote{Cláudia
  Rodrigues, \textit{Lugares dos mortos na cidade dos vivos: tradições e
  transformações fúnebres no Rio de Janeiro} (Rio de Janeiro: Secretaria
  Municipal de Cultura --- Dept. Geral de Documentação e Informação
  Cultural --- Divisão de Editoração, 1997).} descreve certa tolerância
cotidiana com o odor dos cadáveres no interior das igrejas, ao menos na
cidade de Salvador na passagem do \textsc{xviii} para o \textsc{xix}. É um país onde os
exercícios religiosos formam parte essencial da vida, onde a instituição
católica era não só matriz da experiência oficial do sagrado mas também
do lazer popular, segundo o calendário das festas, batizados, as festas
de matrimônios. Era uma Corte onde --- diz o protestante Schilichthorst
--- ``todos os dias parecem mais ou menos domingos. (\ldots{}) Pela
manhã, inúmeras igrejas abrem as largas portas à piedosa multidão, que
nelas se reúne para rezar. Uma curta Ave Maria marca o fim de cada dia.
Entre as preces matutinas e da tarde, a vida segue sua
rotina''.\footnote{Carl Schilichthorst, \textit{O Rio de Janeiro como é
  (1824-1826)}, p. 105.} Portanto, é de se esperar que seus fiéis
participassem do processo de consumo do cadáver nos demorados velórios,
nas paredes, dentro do assoalho. Bem distribuídos estavam os papéis
entre os que mandavam dizer missa para as almas do purgatório, as
carpideiras, os que pediam donativos para a filantropia dos enterros, as
viúvas de luto e as outras almas ali sufragadas cujos corpos dariam o ar
da presença pelo cheiro. Nesse sentido, talvez fosse possível contrapor
a sensibilidade olfativa dos ``melindrosos modernos'' àquela dos
``católicos piedosos'', como faz o livro de João Reis, \textit{A morte é
uma festa}.\footnote{Exemplo semelhante suscita Alain Corbin enquanto
  mede a influência da medicina neo-hipocrática sobre a vigilância
  atmosférica própria da epidemologia do Antigo Regime francês e das
  técnicas de saber do período napoleônico. Enquanto a medicina clínica
  que se esboça na passagem do \textsc{xviii} para o \textsc{xix} põe em relevo o mórbido
  e as lesões observadas no interior do cadáver, um neo-hipocratismo
  mesclado à herança mecanicista toma por referência para os odores do
  patológico a gama definida pela observação da decomposição pútrida.
  Disso veremos aos poucos fervilhar e brotar espontaneamente uma
  política higienista resultante de um sincretismo médico cuja
  epidemologia residirá não tanto na qualidade dos lugares, na altitude
  ou na natureza dos ventos, mas na cruzada contra miasmas pútridos, na
  valorização de desinfetantes, na fumigação com ervas aromáticas etc. O
  interessante é que desse dispositivo médico-higienista dependerá uma
  vigilância olfativa que nem sempre encontrou correspondência na
  conivência pacífica com um tipo de \textit{visibilidade} dos mortos:
  ``\,`Perseguidas pelas exalações dos cadáveres empilhados no cemitério
  dos Inocentes, as jovens passeiam e conversam; é em meio ao odor
  fétido e cadaveroso que as vemos comprar coisas da moda, fitas\dots{}' As
  meninas da paróquia de Santo Eustáquio ouvem o catecismo sem ficar
  enjoadas com as emanações nauseabundas. O texto redigido pelos curas
  de Paris com a finalidade de se opor ao translado dos mortos traz a
  marca dessa relativa anestesia popular. O fato essencial continua
  sendo que essa tolerância com a `proximidade exasperante' passa
  doravante a ser marcada com o selo da estranheza.'' Alain Corbin,
  \textit{Saberes e odores: o olfato e o imaginário social nos séculos
  \textsc{xviii} e \textsc{xix}} {[}1982{]}, trad. Lygia Watanabe (São Paulo: Cia. das
  Letras, 1987)\textit{,} p.~80.} De certa maneira o ``\,`incômodo
passageiro do mau cheiro dos defuntos' era um ato de fé e porque a dor
da perda amainava na certeza de que os entes queridos jaziam em terra
abençoada, esperando-os para `participar com eles dos mesmos jazigos, e
das mesmas honras'\,''.\footnote{João José Reis, \textit{A morte
  é uma festa}, 1991, p.~268.}

Em 1843, no Rio de Janeiro, uma portaria do ministro do Império era
enviada à Câmara cobrando algum rigor no cumprimento das Posturas que
estipulavam prazo de 18 meses para reabertura das catacumbas. ``Ao
contrário do estabelecido, elas estariam sendo abertas para dar lugar a
novos sepultamentos no intervalo de quatro a cinco meses, (\ldots{}) de
forma que se pedia à municipalidade (\ldots{}) `a fim de pôr termo a tão
escandaloso abuso'\,''.\footnote{Cláudia Rodrigues, \textit{Lugares dos
  mortos na cidade dos vivos}, 1997, p.~93.} Daí até 1850 o que se
espera é a oportunidade política e sanitária para se fazer valer essa
nova sensibilidade, animada pela emergência de uma morte limpa,
invisibilizada, e de um cadáver higiênico. Porque onde antes se
enxergava a doméstica integração entre o teatro da vida e o teatro da
morte, onde vivos e mortos ``faziam companhia uns aos outros nos
velórios em casa, (\ldots{}) atravessavam juntos ruas familiares, os
vivos enterravam os mortos em templos onde estes haviam sido batizados,
tinham casado, confessado, assistido a missas e cometido ações menos
devotas''\footnote{João José Reis, ``O cotidiano da morte no Brasil
  oitocentista'', em L.F. Alencastro, História da vida privada no
  Brasil/Império, 1997, p.~140-1.}; onde antes isso se enxergava, o que
se verá é o cadáver assumindo o signo de doença, de sede da doença, e
não tanto de objeto de culto ou do estímulo para preces e súplicas.

``Crê-se, frequentemente, que foi o cristianismo que ensinou à sociedade
moderna o culto dos mortos'', diz Foucault. A individualização do
cadáver, do caixão e do túmulo aparecem ``por razões não
teológico-religiosas de respeito ao cadáver, mas político-sanitárias de
respeito aos vivos. Para que os vivos estejam ao abrigo da influência
nefasta dos mortos é preciso que os mortos sejam tão bem classificados
quanto os vivos''.\footnote{Michel Foucault, ``O nascimento da medicina
  social'' {[}1974{]}, em \textit{Microfísica do poder}, trad. Roberto
  Machado (Rio de Janeiro, Graal, 1979), p.~89-90.} Não uma ideia
cristã, portanto, mas médica, política. De maneira que começa a
acontecer, um ano após a epidemia de febre amarela, a fundação de
cemitérios públicos nos subúrbios. Em 1851, D. Pedro II funda pelo
Decreto nº 842 os cemitérios públicos de S. Francisco Xavier --- na Ponta
do Caju --- e de S. João Baptista --- Botafogo.\footnote{A proposta para
  cemitérios extramuros e para o serviço de enterros já havia sido
  minuciosamente regulamentada no texto de um decreto do mesmo ano. Cf.
  Decreto nº 583, 5 de Setembro de 1850 (Coleção de Leis do Império ---
  1850, página 273. vol.~1, pt.~1); Decreto nº 796, 14 de Junho
  de 1851 (Coleção de Leis do Império --- 1851, página 138. vol.~1,
  pt.~II); Decreto nº 842, 16 de Outubro de 1851 (Coleção de
  Leis do Império --- 1851, página 314. vol.~1, pt.~II).} O poder
político centralizador submeterá a partir de agora a concessão de
terrenos para cemitérios particulares das Ordens Terceiras e Irmandades
à autorização do governo. Cemitérios deverão ser cercados por muros com
altura de dez palmos, além de uma grade que vede a entrada de animais.
As covas terão sete palmos de profundidade,~deverão ser individualizadas
e numeradas, lançando-se o número no livro dos assentos dos
enterramentos. Será obrigatório o uso do caixão. Valas gerais para
sepultura dos pobres falecidos em hospitais serão separadas das valas
dos negros; ambas terão nove palmos de largura, 14 de profundidade,
comprimento compatível com a qualidade do terreno. Os corpos, cobertos
na medida em que forem depositando uma camada de terra socada --- que não
poderá ter menos de três palmos de altura, de modo que os últimos
cadáveres ficarão pelo menos quatro palmos abaixo do chão.

É de fato provável que o cal virgem aplicado sobre o cadáver e o uso de
defumadores na liturgia católica competissem com os odores dos corpos
dentro das igrejas. Seria um pouco absurdo admitir algo diferente. Mas à
parte o aspecto do infectado --- tanto um bexiguento quanto o Papa
Inocêncio XI traziam chagas sobre a pele ---, o que tornou possível a
sensibilidade olfativa de melindrosos homens de ciência? Talvez não se
trate exatamente de reduzir o acontecimento à aparição de nova
sensibilidade diante da morte, agora colonizada por um discurso secular
alinhado aos progressos do espírito científico, mas de interrogar sobre
esse novo objeto ausente na paisagem mundana e na liturgia fúnebre do
\textsc{xviii}: o cadáver, a visibilidade do cadáver, o cadáver no que pesa ao
perigo que exerce para o estado sanitário da cidade.

Em 1853, na ausência de uma grande epidemia na constituição da cidade, a
Junta oficiava ao chefe de polícia que não fosse sepultado ``cadáver de
indivíduo falecido repentinamente sem que primeiro se procedesse à
autópsia cadavérica; e isto não só para evitar se há impunidade de algum
crime, como também para que a ciência ganhe reconhecimento da lesão
patológica que determina tais mortes repentinas''.\footnote{Cf. Tânia
  Salgado Pimenta, \textit{O exercício da arte de curar no Rio de
  Janeiro}, 2003, p.~157, nota 127.} Os primeiros exemplos documentados
de referências a biópsias no Brasil datam de 1835, ao menos até onde se
pôde atestar. E de fato, ``a expansão sistemática, verificada na Europa
desde o final do século \textsc{xviii}, da biópsia e da inspeção
anatomopatológica, tinha já substituído a leitura de marcas no corpo do
doente pela investigação de lesões no seu corpo, até mesmo depois da
morte''.\footnote{Georges Canguilhem, \textit{Ideologia e racionalidade
  nas ciências da vida} {[}1977{]}, trad. Emília Piedade (Lisboa,
  Edições 70, 1977), p. 69.} Ora, a tarefa reclamada pela Junta de
proceder a biópsias de indivíduos falecidos repentinamente nos faz
presumir que a biópsia cadavérica em circunstâncias de eclosão da febre
amarela pode não ter sido comum por razões mais ou menos previsíveis,
seja pelo perigo que envolve a exposição do cadáver, seja pela diretriz
teórica dos instrumentos de diagnóstico da anatomia patológica. Bichat
diz que, mediante o fato de a anatomia patológica se reportar ao
conhecimento das doenças, as doenças deverão ser divididas em duas
classes: as que afetam o regime geral do corpo e aquelas que atacam um
órgão particular. As primeiras não são absolutamente objeto da anatomia
patológica.

\begin{quote}
Todas as diversas espécies de febres causam uma afecção geral, sem que
por causa delas, na maioria das vezes, algum órgão seja particularmente
lesionado. O conhecimento das doenças gerais difere essencialmente do
conhecimento das doenças orgânicas: para aquelas, a observação é
suficiente; para as últimas, a observação e a abertura dos cadáveres.
Eis o que faz com que o conhecimento das doenças gerais não se baseie em
certos signos desencontrados. Tal é a etiologia das febres e doenças
semelhantes: todas as distinções, classificações, segundo as estações
dos anos, os humores etc., dão evidentemente em círculos viciosos. Sua
nosografia apresenta uma dificuldade extrema.\footnote{Xavier Bichat,
  \textit{Anatomie pathologique}, dernier cours de Xavier Bichat: d'après
  un ms. autographe de P.-A. Béclard avec une Notice sur la vie et les
  travaux de Bichat / par F.-G. Boisseau (Paris, chez J.-B. Baillière,
  Libraire, 1825), p.~1-2. Tradução nossa.}
\end{quote}

A lesão anatômica, para Bichat identificada na observação do cadáver
segundo o enquadramento epistemológico do diagnóstico, é perseguida em
casos da morte repentina, não em casos de doença crônica, ou epidemias
de febre, e sim quando as causas da morte não se oferecem ao esculápio
imediatamente na superfície dos signos do corpo. Mesmo para as gerações
de médicos brasileiros academicamente regidos pela anatomia patológica,
fundada sobre bases dessa primeira grande tradição médica que é a
anatomoclínica francesa, a etiologia das epidemias continuaria ligada,
pelo tempo que durou o impulso neo-hipocrático sobre o pensamento
higienista, ao princípio miasmático. Aqui, o odor que se desprendia do
cadáver ganhara uma carga adicional de periculosidade. Por sua vez, a
inalação do ar viciado, das emanações pútridas nos anfiteatros das aulas
de anatomia, expunha os estudantes a prováveis acidentes.\footnote{É
  ainda Bichat quem o atesta, segundo relato no livro de Alain Corbin:
  ``Observei que com a permanência nos anfiteatros, minhas ventosidades
  adquiriram com frequência um odor exatamente análogo ao que exalam os
  cadáveres em putrefação. Mas vejamos como me assegurei de que é a
  pele, assim como o pulmão, os que absorvem, portanto, as moléculas
  odoríficas. Tapei meu nariz e adaptei minha boca a um tubo bastante
  largo, que atravessando a janela me servia para respirar o ar externo.
  Pois bem, minhas ventosidades, depois de uma hora de permanência em
  uma pequena sala de dissecação ao lado de dois cadáveres muito
  fétidos, apresentaram um odor mais ou menos semelhante ao deles.''
  Alain Corbin, \textit{El perfume o el miasma: el olfato y lo imaginario
  social, siglos \textsc{xviii} y \textsc{xix}} {[}1982{]}, trad. Carlota Lazo (México,
  Fondo de Cultura Económica, 1987, p.~53 {[}ed. bras.: \textit{Saberes e
  odores: o olfato e o imaginário social nos séculos \textsc{xviii} e \textsc{xix}},
  1987{]}.}

Durante as epidemias de febre, por algumas razões, as biópsias
realizadas por médicos verificadores não eram uma prática, conforme
dizíamos, seja pela recorrência de uma combinação dos mesmos sintomas em
uma vasta população de enfermos e, portanto, pela previsibilidade do
diagnóstico, seja pela periculosidade que envolve frequentar o odor
infecto dos cadáveres. E isso é realmente decisivo: não se exercia com
frequência a biópsia de cadáveres vítimas de epidemia pela sua
disposição a produzir o contágio. O contágio não era, ainda, um
instrumento conceitual de peso no regime discursivo do dispositivo
médico-higienista. Na verdade não se procedia à biópsia de cadáveres, em
meados do \textsc{xix}, na ausência de um minucioso processo químico de
desodorização dos gases nauseabundos. Não são incomuns relatos
dramáticos entre os médicos sobre os ``perigos das sepulturas'', ligados
ao poder infeccioso da exposição dos cadáveres. \textit{Perigos das
sepulturas} é o título de uma coletânea de acontecimentos fantásticos
organizados por Vicq d'Azyr, que demonstram como cadáveres e túmulos
atuaram diretamente como focos de gases tóxicos. Phillipe Ariès
reproduz, entre outros casos, o seguinte:

\begin{quote}
Em Nantes, em 1774, durante um enterro em uma igreja, ao deslocar-se um
caixão, um odor fétido exalou-se: ``Quinze dos presentes morreram pouco
tempo depois; as quatro pessoas que haviam removido o caixão foram as
primeiras a morrer e os seis padres presentes à cerimônia por pouco não
pereceram''.\footnote{Philippe Ariès, \textit{História da morte no
  ocidente} {[}1974{]}, trad. Priscila Viana de Siqueira (Rio de
  Janeiro, Nova Fronteira, 2017), p.~161.}
\end{quote}

Na primeira metade do \textsc{xix}, algumas estratégias de desodorização como
alternativa para purificar o espaço urbano foram impulsionadas por uma
verdadeira ``revolução farmacêutica dos cloretos''. Conforme Alain
Corbin, em 1823 o grande toxicólogo Mathieu Orfila deveria praticar
biópsia em um cadáver exumado. A pestilência do cadáver se revelava
espantosa. O farmacêutico Labarraque encontrou no cloreto de cal meio de
deter a marcha da putrefação. A sugestão de Labarraque, aspersão com o
cloreto de cálcio dissolvido em água, produzia ``efeito maravilhoso'' e
o ``odor infecto é instantaneamente destruído''. A morte de Luís \textsc{xviii}
vem confirmar o êxito de Labarraque. O cadáver do rei se encontrava em
estado de podridão, chamaram o farmacêutico, ele empapou um lençol com a
fórmula, cobriu o corpo e conseguiu que desaparecesse o mau
cheiro\textit{.} François Delavau, médico e prefeito após a revolução de
1830, reproduziu a experiência: ordenou desinfetar com água de
Labarraque ``latrinas, urinóis e outros banheiros hediondos da
capital''. A ``água de Labarraque''

\begin{quote}
se converterá rapidamente em instrumento indispensável de todas as
grandes empresas higiênicas. Em 1826, permite desinfetar os operários
que se ocupam da limpeza do esgoto de Amelot\textit{.} É este novo licor
que, em 1830, desodoriza cadáveres dos falecidos de julho. A Revolução
de 1830 marca o triunfo definitivo da água de Labarraque. O Dr.~Troche
borrifa as covas que mandou cavar sob a praça do mercado dos Inocentes e
ante Colunata do Louvre. (\ldots{}) Menos de dois anos mais tarde,
quando estala a cólera morbus, é da capital inteira de que se trata de
desinfetar (\dots{}). O prefeito Gisquet dá ordens para utilizá-la na
limpeza das vitrines dos açougueiros e toucinheiros, e para
``neutralizar'' as "emanações pútridas que escapam das fossas e
trincheiras (\dots{}); manda regar o piso dos mercados, o pavimento das
ruas, os fossos dos boulevares.\footnote{Alain Corbin, \textit{El perfume
  o el miasma} {[}1982{]}, 1987, p.~139.}
\end{quote}

A desodorização de Labarraque permite ressituar um espinhoso problema
que planteia a biópsia, mas não só. Até então, o fedor nauseabundo dos
anfiteatros despertava o temor constante de infecção. Empregado como
desodorante e desinfetante, o licor era aspergido no chão para purificar
o ar das enfermarias, anfiteatros e outros lugares infectos.
Aconselhou-se o uso da composição líquida nos quartos das vítimas da
febre, nas roupas, nos móveis e no cadáver.\footnote{O \textit{Dicionário}
  de Chernoviz aconselha que ``pessoas expostas a emanações paludosas ou
  de substâncias em putrefação'' em geral lavem bem as mãos com ``água
  de Labarraque; o gás cloro, que se acha nela, fixa-se na pele por
  algum tempo e neutraliza as emanações nocivas.'' Pedro L. N.
  Chernoviz, \textit{Diccionario de Medicina Popular --- Volume Primeiro
  A-F} (Paris, A. Roger \& F. Chernoviz, 1890), p.~73).} O óxido de
cálcio, o chamado cal virgem, a água de Labarraque, sucederam-se ao
longo do \textsc{xix} como procedimentos prescritos pela Junta Central de Higiene
Pública, destinados a desinfetar as emanações cadavéricas, as alas das
enfermarias, os calabouços dos navios, os presídios, as matérias fecais
ou qualquer tipo de acumulação de detritos. Os muros urbanos degradados
pela urina, a localização dos matadouros, a assombrosa vigilância
olfativa do ar que se respira na atmosfera do quarto de um enfermo:
naturalizou-se a tutela do dispositivo médico-higienista sobre a
liturgia fúnebre justificada pelos efeitos mórbidos que o cadáver em
putrefação exerceria sobre a economia sanitária da cidade. De um ponto
de vista estritamente teórico, não seria inédito entre nós o
encadeamento entre epidemia, higiene pública e a prática de inumações
intramuros. Em 1846, José Pereira Passos defendeu na Faculdade de
Medicina do Rio de Janeiro a tese \textit{Sobre a influência perniciosa
das inumações praticadas intramuros.} Em 1831, Manuel Maurício Rebouças
obtinha o título de Doutor em Medicina pela Faculdade de Paris
defendendo sua \textit{Dissertação sobre as inumações em geral}. Rebouças
foi aluno de Bayle, Trousseau e Broussais, este ocupante da cadeira de
Patologia e terapêutica médicas. ``É ainda comum no Brasil enterrar os
mortos nas igrejas ou em cemitérios próximos a elas'', de maneira que
diariamente populações inteiras estão expostas a miasmas pútridos.
``Nenhum médico ignora que sepulturas feitas em lugares pouco arejados
são perigosas, e não é à medicina que falta prová-lo. Esclarecer nossos
cidadãos a este respeito é aquilo a que me proponho''.\footnote{Manuel
  Maurício Rebouças, \textit{Dissertation sur les inhumations en géneral
  (leurs resultats fâcheux lorsqu'on les pratique dans les églises et
  dans l'enceinte des villes, et des moyens d'y rémedier par des
  cimetières extra-muro)}, thèse présentée et soutenue à la Faculté de
  Médicine de Paris (Paris, l'imprimerie de Didot le Jeune, 1831),
  p.~31. Tradução nossa.} Por que perigosas? Diz o autor: Buffon
estabelece a existência de um movimento, que se produz nos corpos
retirados do seio da terra, cuja ação ``determina o que atualmente
chamamos de \textit{fermentação pútrida}''.\footnote{\textit{Ibidem}, p.~32.}
A fermentação ocasiona a exalação de moléculas que tornam o ar carregado
de substâncias orgânicas do cadáver em decomposição. Como o ar é
facilmente comprometido pelas moléculas que os cadáveres exalam, suas
impurezas penetram os humores dos vivos, seja pelos pulmões, seja pelos
poros da pele. Mediante a perda da sua pureza natural, os princípios da
putrefação tornam o ar funesto e desencadeiam epidemias.

O mesmo dispositivo que cria a morte limpa e invisibilizada do cadáver
higiênico, o mesmo poder que promove o corpo pós-morte como sede da
epidemia; a mesma medicina que reclama o monopólio sobre os meandros da
morte contribuirá na década seguinte, e de maneira eficaz e bastante
feliz, para que o Rio de Janeiro fosse uma das primeiras capitais a
contratar a instalação de um moderno sistema domiciliar de esgotos. A
companhia inglesa \textit{The Rio de Janeiro City Improvements} ficaria
encarregada pelo governo imperial de implantar uma rede de esgotos na
capital. Portanto, será a partir não só da suspeita quanto às imundícies
do corpo do cadáver, mas do combate às emanações nocivas próprias da
coexistência urbana, que os higienistas, quase 20 anos mais tarde,
poderão aparelhar sua forma de atuar.\footnote{``À medida que se
  agravava o quadro sanitário da cidade, especialmente desde a primeira
  grande epidemia de febre amarela, em 1850, o sistema de esgotos
  tornou-se o principal alvo da campanha movida pelos médicos e, logo,
  por toda a''opinião pública" ilustrada em favor de melhoramentos que
  saneassem a capital do Império. As valas, sumidouros e fossas negras
  causavam a infecção do lençol de água subterrâneo e, segundo as
  concepções médicas ainda predominantes, contaminavam o ar com seus
  pútridos miasmas, propiciando o desenvolvimento da febre tifoide,
  cólera, diarreias infecciosas, febre amarela e de um sem-número de
  moléstias." Jaime Larry Benchimol, \textit{Pereira Passos: um Haussmann
  tropical: a renovação urbana da cidade do Rio de Janeiro no início do
  século \textsc{xx}} (Rio de Janeiro, Dep. Geral de Documentação e Informação
  Cultural --- Divisão de Editoração, 1992), p.~72-3.}

Voltemos, no entanto, ao estado atual do nosso problema. A epidemia de
1850 dizimou nada menos do que 4.160 vidas de uma província de 266.419
habitantes, e tudo indica que essa estimativa oficial foi
consideravelmente subestimada. ``Houve quem falasse em 10 mil, 12 mil,
15 mil vítimas fatais''.\footnote{Sidney Chalhoub, \textit{Cidade febril:
  cortiços e epidemias na Corte Imperial} (São Paulo, Cia. das Letras,
  1996), p.~61.} Em um relatório de 1858, reencontramos o deputado
Dr.~Paula Candido, então Presidente da Junta de Central Higiene Pública,
a comentar que a Capital do Império

\begin{quote}
tem ainda de sofrer os tristes efeitos da epidemia que em Dezembro de
1849 pela primeira vez nos visitou, e depois se hospedou, sem que se
tenha de fato conseguido extinguir-lhe o germe. Em fins de Novembro de
1856 começou a febre amarela, como minuciosamente expus no meu relatório
de 1855, chegou a fazer as duas primeiras vítimas nas tripulações dos
navios; e crescendo gradualmente de intensidade, chegou ao seu apogeu em
março de 1857, e desapareceu na descida de Abril, tendo causado a morte
a mais de 1000 doentes.\footnote{Arquivo Nacional. MAÇO IS 4-24, Série
  Saúde --- Higiene e Saúde Pública --- Instituto Oswaldo Cruz, sem
  paginação.}
\end{quote}

No ápice dos efeitos da epidemia de 1850 a Comissão redige um
Regulamento Sanitário a ser observado, por aviso do dia 6 de março, nas
comissões paroquiais de saúde pública. O que se lerá não difere
essencialmente das medidas que instauraram o regime de quarentena (os
mesmos zelos e cuidados pelo asseio dos espaços públicos, a mesma
atribuição da febre à insalubridade e aos defeitos da topografia), com
exceção para os artigos que preveem um controle estatístico dos mortos,
doentes e da população flutuante. Pede-se às ``autoridades policiais
competentes uma relação circunstanciada do número de indivíduos
indigentes que residir em cada quarteirão, com indicação de seus nomes,
sexo, idade, ocupação, nome da rua e número da casa em que
habitarem''.\footnote{``Regulamento sanitario mandado observar por aviso
  d'esta data nas commissões parochiaes de saude publica, creadas por
  aviso de 14 de fevereiro de 1850'', \textit{Diário do Rio de Janeiro},
  Rio de Janeiro, 6 de Março de 1850, p.~1.} Instaura-se uma organização
semanal da estatística mortuária, onde se deve registrar o nome do
falecido, a enfermidade, sexo, estado, idade, profissão, condição e
residência. Por fim, um esquadrinhamento das freguesias nos distritos
onde fosse conveniente a inspeção das habitações dos doentes, com
fiscalização de prisões, hospitais, estalagens ou quaisquer
estabelecimentos onde se reúnam mais de 20 indivíduos, ``superintendendo
em tudo que for concernente à polícia médica e higiene pública''. Em 14
de setembro de 1850, o Visconde sanciona o decreto nº 598, para que se
execute uma resolução que cria a Junta de Higiene Pública, a quem
competirá, no curso das décadas seguintes, pôr em marcha os
melhoramentos do estado sanitário da Capital e das outras Províncias do
Império (o decreto não é senão uma cópia daquele projeto de lei do
deputado Dr.~J. M. Jobim --- presidente da Academia Imperial de Medicina
---, apresentado na sessão da Câmara no dia 12 de fevereiro).\footnote{``Câmara
  dos Srs. Deputados. Sessão em 12 de fevereiro de 1850'', \textit{Jornal
  do Commercio}, Rio de Janeiro, 15 de fevereiro de 1850, p. 1-3.}
Cria-se de fato uma instituição oficial responsável pela promoção da
higiene após o advento da calamidade febril. Será preciso, nos capítulos
seguintes, afinar os sentidos e valores atribuídos tardiamente ao
conceito de ``higiene'', já que por ora o conceito não está descolado
daquilo que entendemos indistintamente por ``estado sanitário'' ou
``salubridade pública''. Os trabalhos da Junta consistirão em lidar com
tais objetos relativos à salubridade: ``dessecamento de lugares
alagadiços, que se tenham reconhecido insalubres, o estabelecimento de
valas, e canos de despejo, e reparação e limpeza dos existentes, a
multiplicação de depósitos de água para uso, e asseio das Povoações e
outros trabalhos de semelhante natureza''.\footnote{``Decreto nº 598, de
  14 de setembro de 1850'', \textit{Coleção de Leis do Império --- 1850},
  página 299. vol.~1, pt.~1.} Não nos cabe agora medir o grau de
eficiência e o raio de ação efetivo desse braço do governo imperial; o
que interessa notar por ora é que as funções políticas atribuídas à
Junta são bem mais flexíveis do que a letra que rege o regulamento.
Vimos como as tarefas atribuídas à Junta nem de longe se restringem à
promoção do saneamento básico e, portanto, em que medida o que é visto
não é redutível ao enunciável. Visibilidades não se confundem com os
objetos nem com as qualidades vistas, o que complexifica nossa tarefa.
Aquilo que faz ver o raio de ação policial da Junta, o seu campo ou
cercado de visibilidade, atua de maneira estratégica com um decreto ou
um regulamento que lhe confere sanção jurídica.

Ainda no calor do momento epidêmico, exatamente um ano depois, o Dr.
Pereira Rego --- membro da Comissão, futuro presidente da Junta e,
possivelmente, dentre os higienistas brasileiros, aquele que mais
tentará fazer valer o poder normativo do dispositivo --- publica sua
\textit{Historia e Descripção da Febre Amarella Epidemica que grassou no
Rio de Janeiro em 1850}. O livro, um tanto quanto indeciso nos seus
diagnósticos, funciona como uma espécie de constelação de tudo que
poderia ser dito ou visto pela sensibilidade médica brasileira de então
a respeito da epidemia. Sem o saber, Pereira Rego fornece um horizonte
em relação ao que era possível se dizer sobre a experiência da epidemia
na primeira metade do \textsc{xix}. Esse ``campo do possível'', aparentemente
desregrado e disperso, remete a certas regularidades enunciativas cuja
manifestação gêmea são as próprias práticas sanitárias e regimes
institucionalizados de quarentena:

\begin{quote}
A causa eficiente e especial da moléstia, aquela que se pode chamar
essencial, nos é inteiramente desconhecida, como as de todas as
moléstias epidêmicas ou contagiosas, os quais só se deixam apreciar por
seus efeitos sobre o organismo. O que unicamente podemos dizer a tal
respeito, é que ela consiste em um princípio miasmático, \textit{sui
generis}, resultante da decomposição de substâncias orgânicas vegetais e
animais, princípio miasmático para cujo desenvolvimento se exige certo
grau de calor e umidade unido a condições especiais de localidade, como
parece demonstrar a observação.\footnote{José P. Rego, \textit{Historia e
  Descripção da Febre Amarella Epidemica\ldots{}}, 1851, p.~85.}
\end{quote}

Morria-se intoxicado pelos miasmas que se desgarravam de matéria
orgânica em decomposição. E seria imprudente subestimar a ação dos
miasmas. A febre amarela limitava-se à esfera da constituição geral da
epidemia, logo, ``a infecção'', diria um tratado sobre a febre amarela
do início do \textsc{xix}, ``não tem propriedade contagiosa \textit{a
posteriori}''.\footnote{Louis Valentin, \textit{Traité de la fiévre jaune
  d'Amérique} (Paris, Méquignon Libraire, 1803), p.~238. Tradução nossa.}
Além disso, indaga Pereira Rego, já sabemos se o miasma, ``ou essa
substância desconhecida assim denominada, não é suscetível de sofrer
modificações em sua natureza essencial, segundo as circunstâncias
climatéricas e outras a que seja ela submetida?''\footnote{José P. Rego,
  \textit{Historia e Descripção da Febre Amarella Epidemica\ldots{}},
  1851, p.~65-7} E conclui: o homem de ciência que contemplava o estado
aparente de salubridade de que gozávamos no meio desses elementos de
destruição, decerto não podia deixar de enxergar

\begin{quote}
nesse como torpor ou inação dos elementos de destruição que nos rodeavam
um desfecho tanto mais terrível para a humanidade, quanto maior fosse
sua duração, uma vez que condições favoráveis viessem pôr em
conflagração os elementos combustíveis há tanto tempo acumulados,
atendendo a que a reação devida ao rompimento desse como equilíbrio
aparente devia ser igual à força de ação das leis que o
mantinham.\footnote{\textit{Ibidem}, p.~1-2.}
\end{quote}

O médico francês Louis Valentin, que, escrevendo em 1803, tentou
conceitualizar a ação à distância dos miasmas, dizia que, como na peste,
a febre amarela está limitada ao domínio das circunstâncias locais: ``se
os doentes são (\ldots{}) transportados para fora dos navios ou dos
hospitais, e expostos ao ar salubre, eles não comunicam mais sua
infecção''.\footnote{Louis Valentin, \textit{Traité de la fiévre jaune
  d'Amérique}, 1803, p. 150.} Os princípios miasmáticos decorriam de um
modo de ser ou das condições especiais de uma localidade. O que
significa que o material das epidemias está no país e, tal como um fogo
sem combustível, o miasma é ineficaz ou se extingue caso se tomem
medidas harmonizadas em direção a um serviço sanitário do mar e de terra
de tal Cidade comerciante fundada à borda de um estuário. Os miasmas,
portanto, as ``emanações de substâncias animais e vegetais em
putrefação'',\footnote{\textit{Ibidem}, p.~238.} aderiam aos corpos ``por
meio da atração química''\footnote{\textit{Ibidem,} 238.} e causavam
doenças infecciosas em função de uma atmosfera degenerada.

Há, portanto, um ponto cardeal onde devem mirar as medidas, tendo como
urgência atenuar os efeitos das epidemias: por ora, não tanto extrair os
agentes patológicos, mas modular o ``excitador'' que instaura a
constituição geral epidêmica. Agir contra miasmas contidos na própria
economia do território significa ``neutralizar o excitador epidêmico nos
veículos que o importam antes que eles contaminem a povoação, e
destruí-lo ou neutralizá-lo nos primeiros centros que ele invadir no
interior da povoação''.\footnote{Arquivo Nacional. MAÇO IS 4-24, Série
  Saúde --- Higiene e Saúde Pública --- Instituto Oswaldo Cruz, \textit{sem
  numeração.}}

Um artigo remetido da Bahia pelo Dr.~Egas Muniz Barreto Carneiro de
Campos, no dia 26 de janeiro, reproduzia raciocínio semelhante a
respeito das condições em que a febre se alojara em Salvador:

\begin{quote}
A febre amarela manifesta-se de preferência nas cidades populosas,
situadas em costas marítimas mais ou menos úmidas, e raras vezes estende
seus estragos a mais de dez léguas distantes do mar, propagando-se
somente mais longe ao longo de algum rio considerável. Ela prefere os
lugares baixos, e não se pode duvidar que as emanações paludosas, bem
que contribuam para seu aparecimento, contudo não são sua única causa
(\dots{}). Donde se pode concluir que se a atmosfera em que vivemos não
estivesse viciada, o gérmen trazido por esse brigue americano não teria
desenvolvimento algum.\footnote{``Publicações a pedido. A febre
  epidêmica reinante é o tifo americano, ou a febre amarela'',
  \textit{Jornal do Commercio}, Rio de Janeiro, 29 de março de 1850, p.~3.}
\end{quote}

Novamente, a produção das diversas febres, uma vez que epidêmicas, era
atribuída à profusão de substâncias em putrefação que viciavam a
atmosfera de um território. Mas esses elementos só ganharam notável
importância e abertura para políticas públicas 20, 25 anos mais tarde.
Aos poucos, no curso do ano em que agia a epidemia, será o corpo da
cidade que assumirá a constituição geral epidêmica, a cidade, em sua
economia sanitária própria, que carecerá de ser investida por um
processo de desodorização. Daqui toda a legislação dedicada ao asseio
público, à disposição correta dos focos das emanações nocivas, à
promoção da salubridade, ao poder de polícia --- por frouxo que fosse ---
conferido à Junta. Partirá daqui inclusive toda uma cruzada higienista
por décadas afora pelo desmonte dos morros da Cidade Velha, uma
condenação moral da Mata Atlântica justificada por um discurso
facilmente aderente. Um bizarro combate à proximidade dos morros em
relação à Cidade Velha --- o do Castelo (o mais nocivo, porque obstrui a
viração e causava estagnação do ar), o Sto. Antônio, o Fernando Dias,
além daqueles compreendidos entre o S. Bento e S. Diogo, de onde
escoariam as águas das chuvas que se acumulam nos mangues.

Quanto à quarentena, a quarentena como tecnologia não profilática é a
própria materialização jurídica do tratamento neo-hipocrático da doença.
O que fazia a Comissão de Saúde Pública senão assistir policialescamente
ao paciente pedindo-o uma contribuição para o tratamento por meio de sua
adesão a uma dieta de reclusão? A dura alternativa de deixar perecer ao
desamparo os míseros marinheiros enfermos, como se tivessem aportado a
terra de bárbaros, ou de relaxar a quarentena, permitindo a ida de
facultativos a bordo, ou de suspendê-la, não era signo de irrisória
ausência do poder político. A decisão de reunir em um mesmo local
pessoas afetadas e em local próximo, mas separado, as que com elas
tiveram algum tipo de contato não era tomada sem ponderação. No contexto
da epidemia de 1850, a quarentena era uma prática positiva de
transferência do ofício da cura para o processo natural de adaptação do
indivíduo a um clima que não lhe é habitual, adaptação da economia
humana estrangeira ao clima dos trópicos. Portanto, é preciso
interpretá-la como uma prática cientificamente prevista. Era normal que
um regime de \textit{quarentena} quisesse dizer que tanto poderiam ser 40
dias como 40 semanas, ou 40 meses. O que é preciso é que não saiam de lá
senão quando aclimatados, ou seja, curados. Ou mortos. A quarentena será
o cordão sanitário emergencial, que dura enquanto duram as
circunstâncias epidêmicas. Já o único cordão sanitário permanente, que
simbolizou uma conquista do dispositivo médico-higienista, foi a
realização dos cemitérios públicos extramuros.

Por que a época que invisibiliza o cadáver da convivência com as
sepulturas intramuros é a mesma época que integra, epistemologicamente,
a morte à experiência médica que possibilita a anatomia patológica para
os sucessores de Bichat? Sob o risco de deixarmos escapar nossos
objetivos, deixemos esse desenvolvimento ao sólido trabalho
histórico-filosófico de Michel Foucault, já consolidado em livros como
\textit{O Nascimento da Clínica}. Limitemos a fala à hipótese --- não menos
alinhada com Foucault --- de que o indivíduo e a população, a clínica e o
dispositivo médico-higienista, o cadáver e a disciplinarização do
cemitério são dados simultaneamente como objetos de saber e alvos de
intervenção da medicina, graças a certas mudanças das tecnologias de
poder e da experiência da epidemia na nossa sociedade --- tema com o qual
nos ocuparemos no último capítulo. ``A redistribuição dessas duas
medicinas será um fenômeno próprio do século \textsc{xix}. A medicina que se
forma no século \textsc{xviii} é tanto uma medicina do indivíduo quanto da
população''.\footnote{Michel Foucault, \textit{Microfísica do poder},
  1979, p.111.}

Como e quando passamos da peste como constituição meteorológica da
enfermidade para a experiência da epidemia como efeito da insalubridade
pública? Se entendermos a salubridade não exclusivamente como oferta de
serviço domiciliar de esgotos, e sim como um conjunto de circunstâncias
ou como uma base material capaz de assegurar a saúde, então a
salubridade se torna questão de governo no momento em que uma epidemia
de febre amarela foi capaz de transformar, oportunamente, o cadáver em
fenômeno repugnante e perigoso, e a morte em fenômeno insólito, em um
acontecimento súbito de grandes proporções. Quais elementos qualificaram
como insalubre o estado sanitário da Corte Imperial quando exposta a um
grande surto de febre amarela? Uma atmosfera contaminada por emanações
de substâncias animais e vegetais em putrefação, um princípio miasmático
reinante em uma localidade datada. E, afinal, como o dispositivo
médico-higienista responde aos desafios que se apresentam na ocasião da
sua institucionalização?

\begin{enumerate}
\def\labelenumi{\arabic{enumi}.}
\item
  Por intermédio de \textit{normas provisórias e combativas} que integram
  os chamados \textit{estados de quarentena}. A interdição do porto, a
  promoção do asseio da rua, o esquadrinhamento, a fiscalização e a
  desinfecção dos navios, prisões, hospitais, estalagens ou qualquer
  instituição que reúna mais de 20 indivíduos, a instauração de postos
  de vigilância e o toque de recolher, a purificação das casas e o
  translado dos enfermos, enfim, medidas de caráter emergencial e
  passageiro, cujo sentido está em contribuir para que a atmosfera
  degenerada daquele território recue paulatinamente em direção ao seu
  estado anterior. É a tecnologia de policiamento extensivo que faz
  funcionar o estado de quarentena, e o mais importante: é o modelo de
  uma atmosfera desodorizada e desinfetada que irá selar o estado
  sanitário ótimo de um território. Trata-se, \textit{in summa}, de frear
  a degeneração da atmosfera através de um combate a princípios
  miasmáticos decorrentes da incúria dos homens e da natureza.
\item
  Paralelamente, o dispositivo procura realizar de forma
  \textit{preventiva e pioneira o controle} sobre uma nova cultura
  fúnebre, que alterará consideravelmente a repartição urbana dos
  espaços de individualização e o registro obtuário. Do caos dos
  sepultamentos nas igrejas ao quadriculamento das covas do Cemitério do
  Caju --- mas não só isso. O cadáver, ao qual resiste por definição o
  corpo vivo, virtualmente já se faz presente, segundo o calendário que
  acompanha a evolução da epidemia, na própria materialidade do corpo
  enfermo. Da mesma maneira como é preciso manter sob custódia um
  amarelento no lazareto, uma morte limpa e inofensiva deve estar ao seu
  alcance. Desenvolveu-se na experiência da epidemia de febre amarela a
  abertura para um novo investimento contra a doença. Não a partir do
  ser vivo ou dos seus humores ou da dieta, mas contra a degeneração da
  vida, contra os efeitos desencadeados pela decomposição orgânica,
  nesse limite intransponível de uma morte que agora se precipita contra
  a vida. Um detalhe a mais que é preciso ter presente: as forças que
  fazem emergir o cadáver sob o signo do patológico, do perigo, e que
  nos conduzem progressivamente a um ``desencantamento do rito fúnebre''
  se confundem intimamente com as técnicas de desinfecção da
  constituição urbana. Ambas fazem parte de uma tarefa política muito
  mais ampla, que tem como plano de reflexão um meio envolvente e como
  finalidade última a desarticulação dos princípios miasmáticos em seus
  eixos potenciais de emissão.
\end{enumerate}

Como passamos da peste como resultado da variação meteorológica da
enfermidade para essa nova experiência da epidemia como efeito da
constituição geral epidêmica? Foi na direção desse tipo de preocupação
que optamos seguir até agora. Mas essa é somente a primeira etapa para o
desbloqueio de um processo de patologização da cidade movido por certas
práticas e saberes reclamados por um corpo médico em vias de se
institucionalizar. Interessa-nos, a partir de agora, antever em que
medida a cidade anti-higiênica foi uma invenção dos médicos higienistas
ou, mais especificamente, de que modo o dispositivo médico-higienista
contribuiu para que se criasse a imagem da ``cidade colonial retrógrada
e anti-higiênica''? Certamente, nisso tiveram o seu lugar os regimes
históricos de verdade sobre a epidemia instaurados pelas séries de
práticas e saberes médicos que tentamos revisitar.

Vejamos como, com o passar dos anos, o dispositivo médico-higienista
desenvolverá um autêntico projeto de cidade, que agrega ao planejamento
do espaço urbano --- com seu perigoso expoente de dados naturais (focos
de emanação de miasmas) e dados artificiais (as ruas estreitas, a
localização dos hospitais e cemitérios etc.) --- isso que poderemos
denominar ``corpo higiênico''. Há aí uma descontinuidade em termos de
funcionamento de poder. O que era verdadeiramente ``característico da
medicina urbana'', Foucault aponta, ``é a habitação privada não ser
tocada e o pobre (\ldots{}) não ser claramente considerado um elemento
perigoso para a saúde da população''.\footnote{Michel Foucault,
  \textit{Microfísica do poder}, 1979, p.~94.} Há aí uma descontinuidade
não só em termos de tecnologias políticas, mas de objetos de
investimento que revelam por que a afirmação da Higiene na formação
urbana e social da Corte não foi isenta de conflitos e contradições.

\chapter{O nascimento da cidade
anti-higiênica}

A cidade anti-higiênica existiu, mas muito mais como produto ou efeito,
não como origem. Como foi possível algo que não existia concentrar
tamanha densidade ontológica? Como se elaboraram os sólidos contornos
discursivos e mecanismos de poder capazes de fazer emergir sob a
condição de coisa algo cuja clarividência não foi manifesta? A questão
é, portanto, refinando aqui alguns objetivos do capítulo: como foi
possível que se assumisse de forma refletida a ligação, o nexo causal,
entre o desalinho das ruas coloniais (a disposição, a fisionomia das
ruas) e a insalubridade pública (o antiestético, o mau cheiro)? Ou: o
que permitiu que o traçado urbano herdado do período colonial fosse
considerado insalubre pelo discurso higienista da segunda metade do \textsc{xix}?
Propomos que, em princípio, não foram os higienistas que lançariam as
bases de percepção da rua como causa intransigente de uma conveniência
com o mau cheiro da cidade. Interessa-nos de que modo o \textit{a priori}
histórico que ativou a sensibilidade para a condição defeituosa da
cidade de S. Sebastião já estaria sendo preparado para que a prática
higienista objetivasse a cidade como campo potencial de intervenções. A
rigor, a importância preliminar do dispositivo médico-higienista foi ter
institucionalizado, capturado, um ramo de competências em torno da rua
e, depois, do cortiço como formas de problematização quase exclusivas no
contexto das epidemias de febre amarela da década de 1870.

\section*{A cidade que os portugueses construíram na América não é produto mental}

Há duas regularidades na vasta bibliografia de reminiscências de
viajantes suíços, franceses, norte-americanos, alemães de passagem pelo
Rio de Janeiro ao longo do século \textsc{xix} --- ou, no mínimo, duas impressões
comuns do desembarque no Largo do Paço. Um registro visual e
cenográfico, e um segundo registro olfativo. Um audiovisual, outro de
caráter olfativo, os dois reminiscências que descarregam com exuberância
o narcisismo colonialista.\footnote{Sobre o sujeito do discurso
  protoantropológico em questão ver: Eduardo Viveiros de Castro, ``O
  Anti-Narciso: lugar e função da Antropologia no mundo contemporâneo'',
  \textit{Revista Brasileira de Psicanálise}, vol.~44, n.~4, São Paulo,
  2010,
  \textless{}http://pepsic.bvsalud.org/scielo.php?script=sci\_arttext\&pid=S0486-641X2010000400002\textgreater{}.}
Primeiro, a graça pitoresca e o fascínio colorido com que enxergam o
espetáculo da escravidão. Sabe-se com que deleite Freyre se alimentou
nessas fontes,\footnote{Em 1836, o britânico George Gardner emite uma
  opinião digna de nota: ``Não sou defensor da manutenção da escravidão;
  pelo contrário, eu me alegraria em vê-la varrida da face terra --- mas
  não dou ouvidos àqueles que representam o senhor de escravos
  brasileiro como um monstro cruel. Minha experiência entre eles foi
  muito boa, e muito poucos atos de crueldade foram feitos sob minha
  própria observação. (\ldots{}) Eles são de um temperamento lento e
  indolente.'' George Gardner, \textit{Travels in the interior of Brazil}
  (London, Reeve Brothers, 1846), p.~17. Tradução nossa.} e em que
medida \textit{Vida social no Brasil nos meados do século \textsc{xix}} começou a
bordar a antropologia da mestiçagem e o mito de democracia que
\textit{Casa-Grande \& Senzala} selou. ``No Largo do Paço o forasteiro
vê-se envolvido por uma turba multiforme tanto na aparência como nos
hábitos, e tão variegada na compleição e nos costumes como jamais
poderia imaginar'',\footnote{Daniel Parish Kidder, \textit{Reminiscências
  de viagens e permanências no Brasil: Rio de Janeiro e Província de São
  Paulo} {[}1845{]}, trad. Moacir Vasconcelos (Brasília, Senado Federal
  --- Conselho Editorial, 2001), p.~62.} escreve Daniel Kidder. ``Quanto
a mim, nunca os negros se me mostraram sob um aspecto tão
artístico'',\footnote{Jean Louis Rodolphe Agassiz, \textit{Viagem ao
  Brasil 1865-1866} {[}1868{]}, trad. Edgar S. de Mendonça (Brasília,
  Senado Federal / Conselho Editorial, 2000), p.~68.} diz um suíço.
Graça e fascínio: pois pode não parecer que são descrições de
escravizados de ganho descalços levando sacas de café de 80 kg,
africanos transportando cadeirinhas, adolescentes expostas em vitrines
abertas, à venda, como amas de leite. Sim, a escravidão como os joelhos
de um projeto econômico de uma nação esclarecida, a escravidão como
instituição, ela escandaliza. Mas assusta em maior medida a investida
contra a escravidão no Haiti em 1804, a Constituição do Haiti, o
confisco das terras dos colonos, os franceses decapitados, a Revolução
movida por africanos insurretos. Diferente do Haiti é o golpe desferido
contra a escravatura norte-americana em 1863, que, segundo o viajante,
``feriu-a de morte onde quer que ela exista; fato esse que nos parece
consolador e significativo''.\footnote{\textit{Ibidem}, p.~67.} Há um
paradoxo aqui, mas ele é um falso paradoxo.\footnote{Aqui,
  aproximamo-nos em alguma medida da posição de Bosi. Talvez sejam
  perspectivas que se somem acerca desse aparente paradoxo: ``O
  liberalismo econômico não produz \textit{sponte sua,} a liberdade social
  e politica. O comércio franqueado entre as nações amigas, que o
  término do \textit{exclusivo} acarretou, não surtiu mudanças na
  composição da força de trabalho: esta continuava escrava (não por
  inércia, mas pela dinâmica mesma da economia agroexportadora), ao
  passo que a nova prática mercantil pós-colonial se honrava com o nome
  de liberal. Daí resulta a conjunção peculiar ao sistema
  econômico-político brasileiro, e não só brasileiro, durante a primeira
  metade do século \textsc{xix}: liberalismo mais escravismo. A boa consciência
  dos promotores do nosso \textit{laissez-faire} se bastava com as
  franquezas do mercado. Nesse bloco histórico não é de estranhar
  absolutamente que a supressão do tráfico demorasse, como demorou, 25
  anos para efetuar-se ao arrepio de tratados que expressamente o
  proibiam.'' Alfredo Bosi, \textit{Dialética da colonização}, 3ª ed.~(São
  Paulo, Cia. das Letras, 1992), p.~198-9.} Se a todos escandalizam as
estruturas econômicas que sustentam o poder senhorial, já nem tanto os
símbolos e rituais do poder senhorial. Tudo se passa como se o fim
efetivo do sistema escravista não implicasse mexer em privilégios ou
como se a libertação jurídica dos cativos, no fim do \textsc{xix}, trouxesse
acoplada uma emancipação de séculos de domínio moral e estético. Para o
historiador Sidney Chalhoub, ``ao menos até a crise que resultou na lei
de 1871, o Brasil imperial oferecia ao mundo o curioso espetáculo de um
país no qual todos condenavam a escravidão, mas quase ninguém queria dar
um passo para viver sem ela''.\footnote{Sidney Chalhoub, \textit{Machado
  de Assis: historiador} (São Paulo, Cia. das Letras, 2003), p.~141.} As
palavras de ordem do viajante liberal condenam a força produtiva de uma
economia de exportação redutível ao braço de homens-propriedade. O que
não pode haver é a confrontação franca com o sofrimento, pelo contrário,
fetichizam-no e o tornam pitoresco, porque o cativo é, sobretudo, a
negação de mim, ele é um negro, um rastro de animalidade na cultura, o
exótico.

\begin{quote}
Gostais da África? Ide de manhã ao mercado contíguo ao porto. Lá a
encontrareis assentada, acocorada, ondulando e tagarelando, com seu
turbante de cachemira ou de trapos, arrastando a renda, ou os andrajos.
É uma galeria curiosa, estranha, um consórcio de graça e de burlesco; é
o povo de Cham agrupado.\footnote{Charles Ribeyrolles, \textit{Brazil
  Pittoresco --- Tomo II}, 1859, p. 60-1.}
\end{quote}

Em meados do \textsc{xviii}, o depósito e mercado de escravizados deixou de
acontecer na Alfândega porque o governo do Marquês do Lavradio, ao
chegar ao Rio de Janeiro, ``horrorizou-se com o espetáculo degradante
que oferecia o desembarque de escravizados, então efetuado (\ldots{}) em
plena Rua da Direita, a mais importante e central da cidade, onde eram
vendidos praticamente nus e aparentando as misérias''.\footnote{Luís
  Carlos Soares, \textit{O ``Povo de Cam'' na Capital do Brasil: a
  escravidão urbana no Rio de Janeiro do século \textsc{xix}} (Rio de Janeiro,
  FAPERJ / 7Letras, 2007), p.~39.} Criou-se o Mercado do Valongo,
suficientemente distante para que o cenário do atacado de carne humana
não contrastasse as galas das novidades francesas nas vidraças e a
frivolidade desocupada da rua do Ouvidor. A mercadoria de cor
continuaria sendo negociada, mas seria uma venda limpa, nos fundos das
lojas de varejo, alugada a terceiros ou mesmo hipotecada, em casas de
consignação e de leilões nas imediações das ruas do Ourives, Ouvidor e
Alfândega. Não eram poucas, na década de 1860,\footnote{A razão, em fins
  da década de 1860, do encerramento das atividades de notórios
  leiloeiros e de traficantes retalhistas de escravizados, ao menos no
  centro comercial e financeiro do Rio de Janeiro, foi a aprovação da
  lei de 15 de setembro de 1869, que proibiu a venda de escravizados em
  lotes ou em exibição pública.} ``casas de compra e venda de escravos
que combinavam as duas formas de usura, o empréstimo de dinheiros a
juros e os negócios de agências de penhores, sem esquecer que também
atuavam como casas de consignação''.\footnote{Luís Carlos Soares,
  \textit{O ``Povo de Cam'' na Capital do Brasil}, 2007, p.~44.} O que
desaparece das imediações do trapiche da Alfândega não é o negócio de
escravizados, mas um tipo de visibilidade das misérias sofridas durante
o aprisionamento na África e a travessia atlântica. Desaparecem os
currais de bexiga, anemia e escorbuto, e em seu lugar ergue-se uma rede
bancária e financeira mais adaptada à internacionalização da economia
nacional e sem dúvida mais adaptada à capitalização das esferas da vida
social. A intermediação realizada pelos negociantes muda o tráfico de
pessoas. O tráfico se despersonifica, desmaterializa-se. Os traficantes
de outrora são os banqueiros de casaca e cartola de então. No lugar de
escravizados vemos o ganho de rua (comércio ambulante e transporte de
carga), negros de ganho mandados à rua para trabalhar e, no fim da
semana ou do dia, desembolsar uma quantia de dinheiro previamente
estabelecida pelos senhores.\footnote{``O recenseamento de 1872 nos
  fornece dados sobre algumas categorias profissionais, entre as quais
  muitos escravos de ganho podiam ainda ser encontrados. A primeira
  delas era a dos `criados e jornaleiros' e entre eles encontravam-se
  4.972 escravos (4.203 homens e 709 mulheres) da cidade e 873 escravos
  (794 homens e 79 mulheres) das freguesias fora da cidade. Obviamente,
  entre os `criados' não se encontram escravos de ganho, mas, sim,
  escravos domésticos que os senhores alugavam a terceiros, muitos deles
  por intermédio de agências alugadoras. Os escravos de ganho seriam
  aqueles classificados como `jornaleiros', numa alusão ao `jornal'
  (salário) que recebiam, mas é realmente difícil saber o seu número
  exato, pois não existem dados específicos para eles e nem para aqueles
  classificados como `criados'. Outra categoria que empregava muitos
  escravos de ganho era a dos `marítimos', que incluía marinheiros,
  remadores e barqueiros.'' Luís Carlos Soares, \textit{O ``Povo de Cam''
  na Capital do Brasil}, 2007, p.~124.} O modelo predileto de Debret no
Brasil não é a cidade nem o gênero humano, é uma ``turba agitada de
negros carregadores e negras vendedoras de frutas'',\footnote{Jean-Baptiste
  Debret, \textit{Viagem pitoresca e histórica ao Brasil}, T. I, vol.~II
  {[}1834-39{]} (Rio de Janeiro, Martins, 1949), p.~126.} e é com a
mesma didática e o mesmo pincel de botânico que ele representa um
açoite, um calceteiro descalço sob o sol ou uma paquera de galeria. Ao
colonizador, seja um viajante em expedição pelo país, seja o pequeno
burguês cioso da sua criadagem, a liberdade é um dever porque o Brasil
tornou-se o último baluarte da escravidão no mundo civilizado. Menos
vexatória, por outro lado, é a perpetuação do colonizador no
protagonismo político e estético. Privilégios não são negociados, mas
tirados à força. E são menos negociáveis ainda quando, no plano dos
valores, o sistema escravista é invisibilizado na sua essência
extrativista e predatória para assumir as ``sutilezadas'' da
fantasmagoria de uma relação entre partes interessadas. Para o suíço
Agassiz, a indolência da população, as ruas imundas, é compensada por
grupos pitorescos que arejam a vida: ``o efeito pitoresco é tal, pelo
menos aos olhos de um viajante, que todos esses defeitos
desaparecem''.\footnote{Jean Louis Rodolphe Agassiz, \textit{Viagem ao
  Brasil 1865-1866} {[}1868{]}, 2000, p.~67.} A indiferença do
naturalista europeu é a complacência das elites nacionais, estrangeiras
na própria terra. A segunda repetição das impressões dos viajantes é
pontualmente o que queremos, neste momento, avaliar mais de perto.
Escreveu ainda Agassiz, quando esteve na Corte em 1865:

\begin{quote}
O que chama desde logo a atenção no Rio de Janeiro é a negligência e a
incúria. Que contraste quando se pensa na ordem, no asseio, na
regularidade das nossas grandes cidades! Ruas estreitas infalivelmente
cortadas, no centro, por uma vala onde se acumulam imundícies de todo
gênero; esgotos de nenhuma espécie; um aspecto de descalabro geral,
resultante, em parte, sem dúvida, da extrema umidade do clima; uma
expressão uniforme de indolência nos transeuntes: eis o bastante para
causar uma impressão singular a quem acaba de deixar a nossa população
ativa e enérgica.\footnote{\textit{Ibidem,} p.~67}
\end{quote}

Algumas coisas a examinar. Primeiro, vejamos, a menção ao traçado das
ruas se repete em diferentes textos de diferentes épocas. Na década de
1870, o alemão Canstatt é surpreendido na cidade ``com o traçado de suas
ruas, das quais as mais importantes são tão estreitas que, se duas
carruagens se encontram, indo em direções opostas, uma tem que subir na
calçada''.\footnote{Oscar Canstatt, \textit{Brasil: terra e gente, 1871},
  trad. Eduardo de Lima e Castro (Brasília, Senado Federal --- Conselho
  Editorial, 2002), p.~296.} Em geral, o centro urbano não causa boa
impressão ao europeu que o percorre. Excetuando as ruas da Direita e do
Ouvidor, Seidler, já em 1835, denunciava as ruas como sendo em sua maior
parte ``compridas, tortas e estreitas, as casas quase todas baixas,
sujas e edificadas em estilo vulgar, sem levar em conta questões de
gosto e de comodidade da vida social, à feição da vontade do momento e
da urgência''.\footnote{Carl Seidler, \textit{Dez anos no Brasil}
  {[}1835{]}, trad. Bertoldo Kliger (Brasília, Senado Federal ---
  Conselho Editorial, 2003), p. 60.} Portanto, primeira regularidade:
regularidade do diagnóstico negativo das ruas em textos de épocas muito
diferentes. Em segundo lugar, aparição de uma quase natural contiguidade
espacial entre imundície, ruas e mau cheiro, como atesta Agassiz na
citação anterior. Pois bem, haverá associação direta entre as ruas da
Cidade Velha e a ``forma como no Rio, e aliás em todo o Brasil, se exige
tanto do olfato''.\footnote{Oscar Canstatt, \textit{Brasil: terra e gente,
  1871}, 2002, p.~306.} A menção a ruas estreitas e sinuosas é
naturalmente acompanhada de uma sequência de observações sobre a largura
econômica dos terrenos, o acavalamento sem método das casas, as paredes
laterais compartilhadas com vizinhos, a alcova, a crítica do
pé-de-moleque, o péssimo sistema de escoamento, a falta de asseio das
ruas e dos negros, o mau cheiro em geral.

\begin{quote}
Ademais, o calçamento é inclinado, de ambos os lados, para o centro, de
maneira que em lugar de terem duas sarjetas para o escoamento das águas,
como na Europa, este se faz por uma só, no centro. Essa disposição, que
na época das chuvas transforma as ruas em riachos caudalosos, é tão
defeituosa quanto o calçamento, e a consequência natural disso é a
acumulação de toda espécie de detritos, a que se deve, em primeira
linha, o mau cheiro das ruas do Rio.\footnote{\textit{Ibidem,} p.~296.}
\end{quote}

A crítica da rua se desmembra como por sucessão causal na insalubridade
da cidade, no mau cheiro da cidade. Não foge à regra Luís Edmundo,
referindo-se ao Segundo Reinado: ``Rio de Janeiro de ruas estreitas, de
vielas imundas, quase sem árvores para fazer a sombra das
calçadas!''\footnote{Luís Edmundo, \textit{O Rio de Janeiro do meu tempo}
  {[}1938{]} (Brasília, Senado Federal --- Conselho Editorial, 2003),
  p.~26.} O que justifica a valoração negativa das sinuosidades e
tortuosidades das ruas imediatamente acompanhada do diagnóstico sobre a
insalubridade? ``As ruas da cidade são, em geral, muito estreitas e
calçadas com pedras grandes''.\footnote{Daniel Parish Kidder,
  \textit{Reminiscências de viagens e permanências no Brasil} {[}1845{]},
  2001, p.~63.} Até então nenhum juízo chama a atenção. O notável é ser
da natureza desses relatos tanto a relação entre ruas estreitas/mau
cheiro como, aliás, a contradição entre embevecimento perante a paisagem
natural e a repugnância suscitada pela paisagem material-urbana. Em
1836, o naturalista britânico George Gardner atraca no Rio e hipnotizado
recorda que visitou outros lugares celebrados pela beleza e grandeza,
nenhum tendo deixado semelhante impressão na alma. ``A natureza parece
ter depositado aqui todas as suas energias''.\footnote{George Gardner,
  \textit{Travels in the interior of Brazil}, 1846, p.~3. Tradução nossa.}
A bordo, vista do navio pela manhã, ``a cidade tinha a mais imponente
das aparências, mas um contato mais próximo dissimula a ilusão. As ruas
são estreitas e sujas. E com o fedor dos milhares de negros que se
aglomeram (\ldots{}) as primeiras impressões se tornam tudo menos
agradáveis''.\footnote{\textit{Ibidem,} p.~5.} Não é a primeira vez nem a
última que o africano desperta aversão e repulsa no estrangeiro pela
falta de asseio. Higienistas não se esqueceriam de contabilizar em 1850
que o vômito preto atacava de forma mais benigna a população negra, ao
passo que brancos e, mormente, imigrantes não aclimatados morriam em
maior escala. Combinado à pressão britânica pelo fim da importação de
escravizados, isso fez dos calabouços dos navios negreiros objetos de
investimento das práticas de quarentena. Mas não durou o tempo de
promulgação da lei Eusébio de Queirós. No entanto, narrativas que
conferem a autoria do flagelo epidêmico à população negra produtiva não
se restringem a navios nem param por aí. Em Tocqueville, o negro
escravizado é abjeto. ``Suscita aversão, repulsa e desgosto. Animal de
manada, é o símbolo da humanidade castrada e atrofiada, da qual emana
uma exalação envenenada, uma espécie de horror constitutivo''.\footnote{Achille
  Mbembe, \textit{Crítica da Razão Negra} {[}2013{]}, trad. Marta Lança
  (Lisboa, Antígona, 2014), p.~146.} O racismo não termina aqui porque a
natureza tóxica do escravizado se converte no comportamento tóxico do
\textit{tigre} no exercício das atividades de esgotamento da sociedade.
Por exemplo, é decepcionante para a etiqueta de Carl Seidler que
``negros encarregados de transportar das casas para a praia toda sorte
de lixo, por sua vez se revelem demasiado comodistas para levarem o vaso
transbordante em longa caminhada até o mar, e na primeira esquina
despejam toda a porcaria e se vão embora''.\footnote{Carl Seidler,
  \textit{Dez anos no Brasil} {[}1835{]}, 2003, p.~63.}

Aos olhos do proprietário de escravizados, a presença do negro na vida
urbana era uma linha tensionada entre escravatura e autoridades
policiais, sobretudo após o levante dos Malês, em 1835, em Salvador. No
Rio de Janeiro, estamos falando de uma população urbana da qual 48,8\%
correspondia, em 1849, a braços escravizados e alforriados.\footnote{Sonia
  Gomes Pereira, \textit{A Reforma Urbana de Pereira Passos e a Construção
  da Identidade Carioca} (Rio de Janeiro, UFRJ-EBA, 1998), p.~79.}
População habilmente inserida na divisão do trabalho imperial,
transitando na máquina econômica urbana como pedreiros, sapateiros,
quitandeiras, lavadeiras, barbeiros, criados etc., vez ou outra
manejando a mesma fé e a mesma língua materna. A Corte havia vivido os
últimos anos da década de 1840 constantemente sobressaltada com rumores
de levantes de escravizados em fazendas vizinhas --- em Campos, Valença,
Vassouras etc. Mal se podia imaginar o que aconteceria se o espírito dos
insurgentes das áreas de \textit{plantation} contagiasse os mais de 100
mil cativos da capital.\footnote{Cf. Sidney Chalhoub, \textit{Cidade
  febril}, 1996, p.~72-3.} Não surpreende que gerassem sobressaltos, em
meados do \textsc{xix}, os rumores de que a escravatura organizada avançava
insubmissa, a arriscar a vida pela libertação nas áreas rurais da
província. Vivia o poder senhorial sob o medo constante dessa ameaça.
Agia sob o terror da possibilidade de ser esmagado por um levante dos
seus cativos, ou seja, pelas figuras de homens que ele próprio não
reconhece como inteiramente humanos.\footnote{``Em ofício de 12 de
  novembro de 1836, Eusébio de Queiroz solicitou que o juiz de paz do
  segundo distrito da Candelária obtivesse informações sobre `um tal
  Emiliano \textit{suspeito de haitismo}', que estaria na casa de Miguel
  Cerigueiro, na rua da Quitanda.'' Sidney Chalhoub, \textit{Visões de
  liberdade: uma história das últimas décadas da escravidão na Corte}
  (São Paulo, Cia. das Letras, 1990), p.~192. Grifo nosso.} Só que nas
décadas de 1860 e 1870 talvez não fossem os Malês organizados, nem mesmo
os calabouços dos navios negreiros e muito menos a introdução de
africanos da Costa da África que teriam o poder virtual de ameaçar a
estrutura da sociedade imperial. As epidemias de febre amarela
revisitarão a cidade até o fim do século. Sitiar a cidade, lavar as
ruas, contar os corpos, recolher nos lazaretos são medidas de contenção
e atalho. Será preciso prevenir a epidemia mediante projetos de
redefinição da infraestrutura urbana. Paralelamente, apresenta-se uma
agenda política e sanitária para regulamentar a construção das casas, e
nesse contexto a habitação coletiva ocupará em pouco tempo o núcleo
excelente da insalubridade urbana. Como veremos no fim deste capítulo, o
antigo pavor diante da virtualidade de um levante de africanos se
redefinirá na habitação da massa produtiva. Mais do que defender a
sociedade contra a extinção dos valores sistêmicos que naturalizam a
escravidão sob um racismo de Estado, é preciso defendê-la agora dos
riscos implicados na existência desassistida da população produtiva. A
segregação social dos novos imigrantes e a segregação racial dos
africanos parecem crescer na proporção em que deixa de haver tráfico
humano, quer dizer, na medida em que foi se tornando mais ardilosa, mais
tênue, à medida que foi aparecendo nas leis, fortalecendo-se nos
detalhes, nos costumes, no mundo do trabalho, nas formas de habitar, de
portar-se em público e de ocupar a rua. Veremos na sequência como aos
poucos as ruas e a circulação nas ruas serão investidas pelo dispositivo
médico-higienista. E de que maneira não é tanto um deslocamento, mas um
acoplamento de objetivos e tecnologias de poder que nos conduzirá da
crítica da rua colonial à tarefa de controle dos cortiços.

Uma das regularidades na vasta bibliografia de reminiscências de
viajantes europeus de passagem pelo Rio de Janeiro ao longo do \textsc{xix} é a
impressão do mau cheiro das ruas, a forma como foram traçadas as ruas, a
condição das ruas da cidade, os grupos que compõem a paisagem urbana, a
maneira como esse encadeamento de registros exige em demasia do olfato
do estrangeiro. A questão por ora não é quão salubre poderia ser a
cidade, a questão é como foi possível que se assumisse de forma
(aparentemente) irrefletida a ligação entre o desalinho das ruas
coloniais, a disposição das ruas e a cidade anti-higiênica.

Em 1846, o britânico Thomas Ewbank veio ao Rio visitar um irmão,
permanecendo quase um semestre. A estadia resultou no seu \textit{Vida no
Brasil ou diário de uma visita à terra do cacau e da palmeira.} Sobre a
natureza imagética do relato, fervilha a atividade dos escravizados nas
ruas, o ritmo com que detalhes da arquitetura domiciliar são rastreados.
Outra particularidade: Ewbank parece aproximar o Rio de Janeiro Imperial
de algumas das cidades coloniais que Portugal erigiu em Marrocos, África
Meridional e Ásia, ou propriamente do aspecto da cidade ibérica
medieval. A razão dessa identificação se encontra na semelhança entre a
Corte e Portugal anterior à expansão europeia no Atlântico --- quer
dizer, Portugal na época em que não passava de uma colônia periférica do
mundo árabe. Encaminha-se o relato para uma comparação entre Rio de
Janeiro e a cidade árabe. De que nos serve o imaginário da cidade
muçulmana?

Na cidade muçulmana medieval não há propriamente quarteirão: o casario
funde-se em um todo homogêneo e indivisível. No Brasil, o casario em
questão tem beirada de telhas, vergas das portas em arco batido como
quase tudo que é edifício térreo e assobradado do início do \textsc{xix} --- e
como talvez ainda vejamos nos arredores do Campo do Santana. Lotes
profundos, com largura econômica das testadas, luz e ar extraídos das
duas extremidades do edifício. Os sobrados da época --- mais tarde,
munidos de toldos como ainda vemos em centros comerciais de subúrbio ---
estavam em uma proporção tal com a largura das ruas que guarneciam o
pedestre como algum esconderijo do sol. Algumas galerias, ou as varandas
dos edifícios com dois pavimentos, eram ainda fechadas por meio de
painéis com treliça, que se abriam como postigos ou se moviam sobre
dobradiças colocadas na parte superior. ``Atrás destas gelosias'',
lembra Ewbank, ``as mulheres da família eram confinadas, como se
estivéssemos na Turquia ou na Grécia antiga, onde às mulheres não era
permitido sair de casa senão sob circunstâncias excepcionais e tampouco
se mostravam nas janelas''.\footnote{Thomas Ewbank, \textit{Life in
  Brazil, or a Journal of a visit to the land of the cocoa and the palm}
  (New York, Harper \& Brothers publishers, 1856), p.~86. Tradução
  nossa.} Onde o britânico narrou a ausência de hábitos cosmopolitas, o
mouro procurou quietude e recolhimento. ``Viviam voltados para dentro,
em seus terraços, em seus pátios, desconfiados da rua, de que se
protegiam com as suas adufas, gelosias, rótulas e muxarabis. Queriam
inspecioná-las, mas sem serem vistos'',\footnote{Paulo Ferreira Santos,
  \textit{Formação das cidades no Brasil colonial}, 2ª ed.~(Rio de
  Janeiro, Editora da UFRJ/IPHAN, 2008), p. 26.} diz Paulo Santos.
Tratava-se de uma sociedade de balcões, avizinhados pelo agrupamento das
casas laterais e pela estreiteza das ruas que mantinham próximas as
galerias das casas defronte. Foi ali que um estrangeiro, em 1826,
acreditou ser agradável o estilo arquitetônico dos portugueses porque
favorece as relações amistosas entre vizinhos. ``Respira-se o mesmo ar e
sente-se o encanto da vizinhança amiga'', diz \textit{Schlichthorst
(dentre uns poucos que pouparam estas construções de comparações
depreciativas). Do olhar de relance a um cumprimento "}facilmente se
passa para relações mais íntimas e, assim, toda a vizinhança forma uma
espécie de roda familiar, que, conforme a situação e a necessidade, ora
se alarga o quarteirão, ora se reduz às casas mais próximas``.\footnote{Carl
  Schilichthorst, \textit{O Rio de Janeiro como é (1824-1826)}, 2000,
  \textit{p.~71.}} O que comandava o traçado da cidade romana era a rua,
ao passo que na cidade muçulmana é a casa. Não era a rua, porque sinuosa
e não planificada, que orientava o trajeto dos carros. Nisso somos
diferentes de Santiago e Buenos Aires: nem sempre teve o Rio vocação
para o bulevar ou para o rígido traço ortogonal de acampamento militar.
As ruas tornearam os acidentes da topografia primeiro em função da
intimidade espontânea entre as casas. Por isso cultivar vizinhos.
Gostava-se de conversar debruçado nos balcões, nas janelas ou nas
portas, e há''até quem não saia dos salões estufas, e se conserve entre
as luzes e o piano``.\footnote{Charles Ribeyrolles, \textit{Brazil
  Pittoresco --- Tomo II}, 1859, p. 48-9.} Por isso o quietismo da casa
de campo aguçar sentimentos de alerta e incerteza. Por isso não nos é
indiferente quando se diz que as''casas do Rio são em geral baixas,
pequenas, sujas, sem gosto e incômodas``,\footnote{Carl Seidler,
  \textit{Dez anos no Brasil} {[}1835{]}, 2003, p.~62.} ou que''o estilo
das casas particulares é muito simples, em muitos sentidos muito pouco
de acordo com o clima``.\footnote{Oscar Canstatt, \textit{Brasil: terra e
  gente, 1871}, 2002, p.~304.} Poderiam ser simples as casas, mas no
sentido de não serem de fato construídas para serem pitorescas ou
monumentais. Os edifícios eram sensíveis ao papel funcional, seja
concernente à forma ou à localização.''Acidentes de vista e contrastes
de forma e cor resultavam de contornos da terra (da topografia local) e
da engenhosa seleção dos sítios para cada estrutura. A posição do
comando da catedral ou igreja dava uma singular unidade à
cidade``.\footnote{Arthur B. Gallion, \textit{The urban pattern} (Londres
  /Nova York / Toronto, Van Nostrand, 1950), p.~36, citado em Paulo
  Ferreira Santos, \textit{Formação das cidades no Brasil colonial}, 2008,
  p.~23.} Para o viajante britânico, as marcas da presença muçulmana na
cidade imperial eram fortes também nos costumes da população. Ao lado da
baía, não mais de 100 metros dali, Ewbank testemunha a chegada de uma
embarcação pesqueira à região do Mercado de Peixe, e os negros cesteiros
caymmianos mergulham na arrebentação''rivalizando entre si, correndo e
passando uns pelos outros para obter primeiro uma porção da carga; seus
berros, gritos e brigas se assemelham às coisas de Níger".\footnote{Thomas
  Ewbank, \textit{Life in Brazil}, 1856, p.~88.} Níger, os Turcos, o povo
de Cham. A cidade de meados do \textsc{xix} conservou da influência oriental na
Península algo além de gelosias, muxarabis e treliças de casas de poucas
janelas.

A largura mínima da rua sombreava o calor do trópico, e o traçado da
cidade, que ao forasteiro pareceu confuso, labiríntico, era conhecido
pelo cidadão da comunidade como prático, familiar e agradável. Se
preferimos hoje a Rua da Alfândega em seu desalinho, para quem quer
chegar aos trens da Central, às calçadas uniformes da atual Avenida
Presidente Vargas, não é por outro qualquer motivo. Se as freguesias de
Sacramento e Candelária não morriam de amores pela linha reta não era
``desleixo'',\footnote{``É verdade que o esquema retangular não deixava
  de manifestar-se --- no próprio Rio de Janeiro já surge um esboço ---
  quando encontrava poucos empecilhos naturais. Seria ilusório, contudo,
  supor que sua presença resultasse da atração pelas formas fixas e
  preestabelecidas, que exprimem uma enérgica vontade construtora,
  quando o certo é que procedem, em sua generalidade, dos princípios
  racionais e estéticos de simetria que o Renascimento instaurou,
  inspirando-se nas ideias da Antiguidade. Seja como for, o traçado
  geométrico jamais pôde alcançar, entre nós, a importância que veio a
  ter em terras da Coroa de Castela: não raro o desenvolvimento ulterior
  dos centros urbanos repeliu aqui esse esquema inicial para obedecer
  antes às sugestões topográficas. (\ldots{}) A rotina e não a razão
  abstrata foi o princípio que norteou os portugueses, nesta como em
  tantas outras expressões de sua atividade colonizadora. Preferiam agir
  por experiências sucessivas, nem sempre coordenadas umas às outras, a
  traçar de antemão um plano para segui-lo até o fim. Raros os
  estabelecimentos fundados por eles no Brasil que não tenham mudado
  uma, duas ou mais vezes de sítio, e a presença da clássica vila velha
  ao lado de certos centros urbanos de origem colonial é persistente
  testemunho dessa atitude tateante e perdulária. (\ldots{}) \textit{A
  cidade que os portugueses construíram na América não é produto
  mental}, não chega a contradizer o quadro da natureza, e sua silhueta
  se enlaça na linha da paisagem. Nenhum rigor, nenhum método, nenhuma
  previdência, sempre esse significativo abandono que exprime a palavra
  `desleixo' --- palavra que o escritor Aubrey Bell considerou tão
  tipicamente portuguesa como `saudade' e que, no entender, implica
  menos falta de energia do que uma íntima convicção de que `não vale a
  pena\dots{}'\,'' Sérgio Buarque de Holanda, \textit{Raízes do Brasil}
  {[}1936{]}, 1995, p.~109-110.} como ilustrava Sérgio Buarque ao
elaborar sua ``leitura liberal do mito racional freyriano''.\footnote{Jessé
  Souza, \textit{A tolice da inteligência brasileira: ou como o país se
  deixa manipular pela elite} (São Paulo, LeYa, 2015), p.~42.} Foi
porque houve dias em que a rebeldia caprichosa da paisagem atlântica não
exigiu corretivos que fossem muito além da resolução do solo alagadiço.
Não foi então falta de trabalho: a Cidade, quando era somente uma entre
outras vilas velhas portuguesas, adequou-se à topografia em função da
experiência e do expediente dos mestres de obra. Por exemplo, a Avenida
Mem de Sá e a Rua do Riachuelo, na antiga freguesia de Sto. Antônio, são
o entroncamento de duas cidades do Rio de Janeiro rivais entre si. A
primeira fez parte do quadro do ideário reformista que converge para o
Plano de Melhoramentos da Cidade implementado a partir da prefeitura de
Pereira Passos (1902-1906). Unindo o Passeio Público à Rua Frei Caneca,
a avenida foi traçada de antemão na prancheta do engenheiro em uma linha
reta na diagonal que só ganhará o plano regular às custas do arrasamento
do Morro do Senado. Desmonta-se uma Cidade para que outra rotina passe
por cima impondo a necessidade da sua presença. A Rua do Riachuelo é
mais antiga, é a antiga Matacavalos, farto cenário de narrativas que vão
de românticos a Machado. Ela abre o vale por entre os Morros de Sta.
Teresa e o Morro de Sto. Antônio, em seguida vai alisando as costas do
Senado, de modo a se formar em uma linha curva que contorna
carinhosamente o morro. Ambas largam dos Arcos da Lapa e caem na Frei
Caneca: mas depois de arrasados os Morros de Sto. Antônio e do Senado, a
rua do Riachuelo segue envergada para a direita, sem que a teimosia do
paralelismo com a Mem de Sá pese como dívida. É que elas não são filhas
dos mesmos valores, nas duas acepções do termo.

Paulo Santos, nesse imprescindível \textit{Formação de cidades no Brasil
colonial}, cita \textit{The Culture of Cities} (1938), do historiador
Lewis Mumford:

\begin{quote}
Graças à sua faculdade persistente de adaptar-se ao lugar e às
necessidades práticas, a cidade medieval apresentava estes exemplos
multiformes de individualismo: o homem que fazia o plano tirava proveito
do irregular, do acidentado e do inesperado (\ldots{}) e não era
contrário à simetria e à regularidade, quando se podia fazer o traçado
num só plano de terra virgem, tal como ocorria nas cidades
fronteiriças. Muitas das irregularidades que ainda podem
observar-se nos traçados medievais devem-se à presença de córregos
recobertos posteriormente, árvores cortadas ou a obstáculos naturais que
outrora serviam de mirões para delimitar a propriedade rural. (\ldots{}) Nos 
domínios das cidades, começamos a dar-nos conta, por fim,
de que as nossas descobertas obtidas depois de tanto trabalho na
arte de fazer planos, especialmente se se tem em conta o ponto de vista
higiênico, são ainda uma recapitulação, em termos de nossas próprias
necessidades sociais, dos lugares-comuns implícitos nos bons princípios
medievais.\footnote{Citado em Paulo Ferreira Santos, \textit{Formação das
  cidades no Brasil colonial}, 2008, p.~22. Grifo nosso.}
\end{quote}

Avenida Mem de Sá e Rua do Riachuelo são, melhor dizendo, o
entroncamento de duas cidades do Rio de Janeiro que coexistem em um
tensionamento resistente a qualquer tentativa de síntese dialética. Toda
sanha pela possível reconciliação costuma ser uma entre outras
estratégias de conservação de uma filosofia da história, hegemônica, do
progresso ou de suas metonímias eleitorais, economicistas e
culturalistas: a democracia, a modernização, a revitalização, a Arena
Maracanã. A ambição pela modernização da área central e a implementação
de inovações tipológicas não atingiram da noite para o dia as áreas
centrais da cidade, isto é, aquelas freguesias que são objeto de crítica
nas reminiscências dos viajantes. Seja pela valorização do solo que
adensava as freguesias\footnote{Freguesias eram as áreas de abrangência
  das paróquias e, como os sacramentos valiam enquanto registro civil,
  elas acabaram se tornando, além de divisões eclesiásticas,
  delimitações administrativas.} do Sacramento e Candelária, seja pela
função industrial e comercial, a rua e a construção domiciliar quase não
serão modificadas entre as décadas de 1850 a 1870. Não serão
modificadas, enfim, pela herança histórica do lugar, quer dizer, pela
resistência à temporalização da história redutível ao aspecto
cronológico do tempo. O tempo histórico do progresso, da modernização,
da \textit{revitalização}, é o tempo cumulativo segundo o modelo da
sucessão cronológica, é o tempo capitalizado da venda de crédito, é o
tempo da produção de mais-valia. Opõe-se a essa temporalidade linear e
vazia da história a ideia benjaminiana de tempo. O tempo histórico em
Benjamin é o tempo saturado no qual cada gesto remete a uma série de
gestos passados que nunca passaram completamente, continuando a habitar
gestos presentes e dando-lhes uma densidade propriamente histórica. No
estatuto da história em Benjamin, compreende-se o presente como uma
contração de múltiplas séries passadas. Tempo no qual coisas que
desaparecem não passam por completo, pois não é este o seu destino,
porque estamos numa série de repetições de sussurros de antepassados que
permite que ``pessoas absolutamente sem glória surjam do meio de tantos
mortos, gesticulem ainda, continuem manifestando sua (\ldots{})
invencível obstinação em divagar, compensando talvez o azar que lançara
sobre eles, apesar de sua modéstia e anonimato, o raio do
poder''.\footnote{Michel Foucault, ``Vida dos homens infames''
  {[}1977{]}, em \textit{Ditos e Escritos IV: Estratégia, poder-saber},
  trad. Vera L. A. Ribeiro, 2ª ed.~(Rio de Janeiro, Forense
  Universitária, 2006), p.~210.} A imagem benjaminiana da história como
``objeto de uma construção, cujo lugar não é formado pelo tempo
homogêneo e vazio, mas por aquele saturado pelo
tempo-de-agora'',\footnote{Walter Benjamin, Tese XIV, ``Sobre o conceito
  de história'' {[}1940{]}, em Michael Löwy, Walter Benjamin:
  \textit{Aviso de incêndio --- uma leitura das teses ``Sobre o conceito de
  história''}, trad. Jeanne Marie Gagnebin e Marcos L. Müller (São
  Paulo, Boitempo, 2005), p. 119.} essa imagem pode ser traduzida por
outra, por uma metáfora, na verdade --- metáfora que Freud criou em
\textit{O mal-estar na civilização} (1930) enquanto tentava ilustrar a
estrutura do sujeito psicanalítico. Conquanto se supere o erro de achar
que o conceito de \textit{esquecimento} significa ``destruição do traço
mnemônico'', Freud tende à premissa contrária, a de que na economia
psíquica (por conseguinte, no tempo histórico) nada que uma vez se
formou é a rigor esquecido. E em seguida toma um exemplo: a evolução da
Cidade Eterna, Roma.\footnote{``Seguramente, ainda muita coisa antiga se
  acha enterrada no solo da cidade ou sob as construções modernas. É
  assim que para nós se preserva o passado, em sítios históricos como
  Roma. Façamos agora a fantástica suposição de que Roma não seja uma
  morada humana, mas uma entidade psíquica com um passado igualmente
  longo e rico, na qual nada que veio a existir chegou a perecer, na
  qual, juntamente com a última fase de desenvolvimento, todas as
  anteriores continuam a viver. Isto significa que em Roma os palácios
  dos césares e o \textit{Septizonium} de Sétimo Severo ainda se ergueriam
  sobre o Palatino, que o Castelo de Sant'Angelo ainda mostraria em suas
  ameias as belas estátuas que o adornavam até a invasão dos godos etc.
  Mais ainda: que no lugar do palácio Caffarelli estaria novamente, sem
  que fosse preciso retirar essa construção, o templo de Júpiter
  Capitolino, e este não apenas em seu último aspecto, tal como o viam
  os romanos da época imperial, mas também naqueles mais antigos, quando
  ainda apresentava formas etruscas e era ornado de antefixas de
  terracota. Onde agora está o Coliseu poderíamos admirar também a
  desaparecida Domus Aurea, de Nero; na Piazza della Rotonda veríamos
  não só o atual Panteão, como nos foi deixado por Adriano, mas também a
  construção original de Agripa; e o mesmo solo suportaria a igreja de
  Maria Sopra Minerva e o velho templo sobre o qual ela está erguida.
  Nisso, bastaria talvez que o observador mudasse apenas a direção do
  olhar ou a posição, para obter uma ou outra dessas visões.'' Sigmund
  Freud, \textit{Obras completas, vol.~18 --- O mal-estar na civilização,
  novas conferências introdutórias e outros textos (1930-1936)}, trad.
  Paulo C. de Souza (São Paulo, Cia. das Letras, 2010), p.~15-6.} A
imagem é a seguinte: uma cidade na qual todos os estágios do seu
desenvolvimento estão atualizados num mesmo plano, criando um lugar
fisicamente absurdo e irrepresentável. É uma tessitura impossível --- por
ser a justaposição de um espaço sincronicamente preenchido por sucessões
históricas distintas --- da tentativa de representar espacialmente o
devir histórico.

\begin{quote}
Não há dia em que eu pise no velho cais da Praça \textsc{xv} sem lembrar que ali
vivem, consagrados na memória das pedras, os marujos que quebraram as
chibatas da marinha de guerra do Brasil na revolta de 1910. Na
materialidade bruta da Pedra do Sal ressoam batuques de primitivos
sambas e berram todos os bodes imolados aos deuses que chegaram da
África nos porões dos negreiros, acompanhando seu povo. (\ldots{})
Existem lugares de esquecimento, territórios do efêmero, e lugares de
memória, territórios da permanência. Estes últimos são espaços que,
sacralizados pelos homens em suas geografias de ritos, antecedem a sua
própria criação e parecem estar aí desde a véspera da primeira manhã do
mundo.\footnote{Luiz Antônio Simas, \textit{Pedrinhas miudinhas: ensaios
  sobre ruas, aldeias e terreiros} (Rio de Janeiro, Mórula Editorial,
  2013), p. 36.}
\end{quote}

O ``velho cais da Praça \textsc{xv}'' --- onde ainda visitamos o Paço e um morro
do Castelo que apenas conhecemos pelas curvas de nível dos mapas
antigos, mas cujo nome nada abala --- estava situado na antiga freguesia
de São José,\footnote{Em um \textit{Cartograma do Cholera-Morbus durante o
  ano de 1895 na cidade do Rio de Janeiro,} publicado em 1896 e
  organizado pelo instituto Sanitário Federal, encontramos a atual Praça
  \textsc{xv} situada na freguesia de São José. Não é incomum, antes da República
  --- e o próprio Noronha Santos nos deu prova em \textit{As freguesias do
  Rio antigo}, de 1965 ---, situar na freguesia da Candelária a Praça \textsc{xv}
  (Antiga Várzea de N. S. do Ó; Terreiro da Polé; Largo do Carmo;
  Terreiro do Paço; Largo do Paço; Largo do Palácio; Praça D. Pedro II,
  denominação dada em 1870, antes de se chamar, por decreto, Praça
  Quinze de Novembro).} por onde corria a Misericórdia, rua lamentável e
verdadeira na dor, que ainda em 1905 João do Rio debochava pelas
``hospedarias lôbregas, a miséria, a desgraça das casas velhas e a cair,
os corredores bafientos, é perpetuamente lamentável''.\footnote{João do
  Rio, \textit{A alma encantadora das ruas: crônicas} {[}1904-07{]} (São
  Paulo, Cia. das Letras, 2008), p.~35.} A ``materialidade bruta da
Pedra do Sal'', como escreve Luiz Antônio Simas, corresponde à área que
concentrou, e concentra, a região portuária, com antigos trapiches,
estaleiros, o morro da Conceição e a Matriz de Sta. Rita defronte à
Igreja --- que nomeia a freguesia de Sta. Rita --- sob a Rua Visconde de
Inhaúma, funcionou no \textsc{xviii} o primeiro cemitério de pretos novos (dos
escravizados que morriam imediatamente após o desembarque).

A mais populosa das freguesias urbanas, pelo menos entre as décadas de
1860 e 1880, é a freguesia de Santana, nas imediações do Campo de
Santana, do morro da Providência, do Cabeça de Porco e da primeira Casa
de Santo da Cidade, e que, em 1870, abrigou mais de 20\% da população de
cortiço do Rio de Janeiro. A freguesia da Candelária orbita em torno das
ruas do Ouvidor e da Direita (atual 1º de Março), e foi o efetivo centro
da cidade pelo menos até meados da década de 1880. Abrigava bancos,
escritórios, consulados, edifícios comerciais com sobrados habitados por
franceses e portugueses, e um Mercado de Peixe que se julgou
incompatível com a valorização econômica do solo central, e que
remanesce sob um nome de rua. A freguesia do Sacramento foi
provavelmente a área mais afetada pelas obras de melhoramento da virada
do século, não coincidentemente a área que melhor mesclou moradias de
classes nas situações sociais mais desiguais. Na segunda parte do \textsc{xix}, a
freguesia de Sacramento localizava grande parte dos teatros, a Academia
Imperial de Belas Artes e a Escola Nacional de Engenharia, mas também a
maior parte das indústrias e de casarios antigos distribuídos ao longo
de uma malha viária extremamente densa e complexa. Trata-se de um
retângulo que corresponde hoje à região entre o Largo da Carioca e a Av.
Presidente Vargas, e da Av. Rio Branco à região que compreende o Saara.
Outras duas freguesias completam a região urbana central em meados de
1870, ambas de criação recente, Santo Antônio e Espírito Santo,
estabelecidas respectivamente em 1854 e 1865.

Em 1852, a Câmara realizou o aterro do Saco do São Diogo (região de
mangue que se estendia da Praça Onze à baía), e com a resolução dos
atoleiros que cercavam Santa Teresa e Rio Comprido permitiu-se a
ocupação da Cidade Nova. ``A rapidez da ocupação dessa `Cidade Nova' foi
tão intensa que, a partir de 1865, criou-se a freguesia do Espírito
Santo, (\ldots{}) tendo sido desmembrada de terrenos pertencentes às
freguesias de Santo Antônio, Engenho Velho, Santana e São
Cristóvão''.\footnote{Mauricio de Abreu, \textit{Evolução urbana do Rio de
  Janeiro} (Rio de Janeiro, IPP, 2008), p.~41.} Assim nos apresenta
Roberto Moura as transformações que sofre a região e a densidade
histórica do Espírito Santo no curso do \textsc{xix}:

\begin{quote}
A praça Onze, cercada por casuarinas, e imortalizada como sede do
Carnaval popular e do samba no início do século, se constituía no único
respiradouro livre de toda a área do bairro. Já no século \textsc{xviii}, chamada
de Rossio Pequeno, era local aberto de uso comum junto aos mangues onde
a população jogava seu lixo, como o Rossio Grande, atual praça
Tiradentes, na época também área de serventia e gueto dos ciganos na
cidade. Com o desenvolvimento do bairro, a praça é urbanizada em 1846,
quando são plantadas as árvores e colocado em seu centro um chafariz
projetado por Grandjean de Montigny, arquiteto vindo com a Missão
Francesa trazida pelo conde da Barca em 1816. A partir da ocupação da
Cidade Nova pela gente pobre deslocada pelas obras {[}da prefeitura de
Pereira Passos{]} (\ldots{}) a praça se tornaria ponto de convergência
desses novos moradores, local onde se desenrolariam os encontros de
capoeiras, malandros, operários do meio popular carioca, músicos,
compositores e dançarinos, dos blocos e ranchos carnavalescos, da gente
do candomblé ou dos cultos islâmicos dos baianos, de portugueses,
italianos e espanhóis.\footnote{Roberto Moura, \textit{Tia Ciata e a
  pequena África no Rio de Janeiro} {[}1983{]} (Rio de Janeiro, Secr.
  Municipal de Cultura --- Dep. Geral de Doc. e Inf. Cultural --- Divisão
  de Editoração, 1995)\textit{,} p.~57.}
\end{quote}

Por fim, a freguesia Santo Antônio, que teve de antemão a jurisdição
sobre os bairros da Lapa, Catumbi, Estácio e Sta. Teresa, e recebeu na
primeira metade do \textsc{xix} suntuosas chácaras da camada média mais abastada.
Tanto Brás Cubas, deputado, quanto Bentinho, filho de político de
Itaguaí, ali vieram a residir. O Brás de Machado dispunha de chácara
herdada no Catumbi, possivelmente composta de senzala, capela e serviço
de abastecimento. Já em 1870, com um crescimento populacional
vertiginoso, os casarões de três janelas de frente, varandas e alcovas,
da parte menos suburbana de Matacavalos, se reciclariam na forma de
estalagens e pequenas hospedarias. Com a intensificação da imigração,
seus edifícios geminados, preocupados em aproveitar da melhor forma
possível a proximidade com centro, seriam destino de galegos e ibéricos
empregados na indústria e no comércio varejista. Em Espírito Santo, a
participação dos moradores de cortiços foi de quase 4 mil habitantes
para uma população de 17.427 em 1870.\footnote{Sonia Gomes Pereira,
  \textit{A Reforma Urbana de Pereira Passos}\ldots{}, 1998, p.~270.}

As freguesias de Sacramento e Candelária não nasceram da linha reta por
desleixo ou falta de asseio. Da fisionomia da Cidade Velha se chega à
imagem da rua como lugar de circulação adequado ao meio de transporte de
que se dispõe (o trajeto a pé, o cavalo em montaria ou atrelado à
liteira; diferentemente dos bondes, os veículos à roda não exigiam a
regularidade das ruas). A rua é também, em segundo lugar, aliada da
necessidade de defesa externa. O temor de invasão era fator importante
para uma malha viária orientada pela proximidade das casas geminadas. De
modo que as frontarias das casas e os casarios fundidos em um bloco
homogêneo funcionaram como autênticas muralhas contra invasões marítimas
no tempo em que a cidade começava a descida do morro para a várzea. ``O
quilombola e o corsário'', diz Lima Barreto, ``projetaram um pouco a
cidade''.\footnote{Lima Barreto, \textit{Vida e Morte de M. J. Gonzaga de
  Sá} {[}1919{]} (São Paulo, Ed. Brasiliense, 1956), p\textit{.} 66.}
Ademais, a identidade de cada rua em particular, se era sinuosa ou
regular, se há alargamentos com súbitos estreitamentos, obedecia às
vicissitudes sociais e ao sistema da vida. Ruas mais próximas às
catedrais e aos prédios públicos costumavam receber um volume de
alargamento distinto dos arredores. Guardadas as proporções, a cidade de
Ouro Preto está bem próxima desse tipo de organicidade: os caminhos, os
passeios, eram em formas variáveis, ``em certos ângulos importantes as
ruas alargavam-se, para dar maior espaço ao panorama urbano e lugar às
aglomerações ocasionais sem perturbar o trânsito''.\footnote{Paulo
  Ferreira Santos, \textit{Formação das cidades no Brasil colonial}, 2008,
  p.~21.} Se considerarmos a antiga topografia da cidade do Rio de
Janeiro, há ainda outro sentido para a fisionomia de suas ruas. Aquele
salpicar de morros e colinas, que ora afastam ora separam as partes
componentes da cidade, às vezes fazendo a comunicação com os arrabaldes
por meio de estreitos vales, outras vezes originando numerosos becos sem
saída, frequentemente em L. ``Ali, uma ponta de montanhas empurrou-as;
aqui, um alagadiço dividiu-as em duas azinhagas simétricas''.\footnote{Lima
  Barreto, \textit{Vida e Morte de M. J. Gonzaga de Sá} {[}1919{]}, p. 67.}
Quer dizer, as partes se reúnem por acidente, enquanto uma ou outra
freguesia vive segregada, porque a cidade, lógica segundo seus próprios
parâmetros, não rivalizou com o local onde se assentou. A cidade, se
``não é regular com a estreita geometria de um agrimensor, é,
entretanto, com as colinas que a distinguem e fazem-na ela
mesma''.\footnote{\textit{Ibidem,} p.~66.}

Seria preciso pensar a troco de que funcionaram os juízos negativos dos
viajantes a respeito da fisionomia das ruas da cidade. Seria preciso
pensar como e a partir de qual marco histórico essa relação entre ruas
acanhadas e o mau cheiro da cidade foi sendo reforçada e alavancada pelo
dispositivo médico-higienista. Percebe-se, no entanto, como estamos bem
distantes ainda de supor que a Academia Imperial de Medicina ou as
Faculdades de Medicina no Brasil lançariam as bases da percepção da rua
como efeito intransigente da velha conveniência do antigo colono
português com respeito à insalubridade. O projeto, a estrutura
programática, a remodelação da cidade sob a urgência do gerenciamento da
boa circulação, da purificação do ar e da anulação dos miasmas está
ligado certamente à sanha de formalização da ciência e consciência
médicas. Mas os jogos de poder que processam a verdade higienista, ou
melhor, o \textit{a priori} histórico que desbloqueia a sensibilidade para
a condição defeituosa da cidade, tudo isso já estaria mais ou menos
sendo preparado para que a prática higienista fizesse da cidade enferma
um \textit{objeto} predileto. Já estariam disponíveis na segunda metade do
\textsc{xix} as condições para que o dispositivo enquadrasse até ao detalhe o
sintoma que lhe é mais caro, sintoma esse que define a necessidade de se
fazer passar a cidade por uma cirurgia urbanística futura: a febre
amarela persiste e a mortalidade é recorde. No máximo, a importância do
higienista foi ter institucionalizado, circulado seu ramo de competência
em torno da questão da rua e do cortiço, e situado a posição do sujeito
higienista que fala de tal ou qual lugar (a posição do médico em uma
instituição criada para promover a Higiene Pública, a posição do
higienista em relação à população de cortiço, em relação à Câmara, à
Secretaria de Polícia etc.), situando enfim a identidade do morador de
cortiços (atrelada futuramente à imoralidade ou aos números do crime).

De fato existiu a cidade dos higienistas. A essa altura não faria
sentido pensar o contrário ou demonstrar como ela foi um erro. Não
consiste então em deduzir que a cidade anti-higiênica não existiu. Mas
não se quer com isso dizer que preexistiu, à ``ideologia do
higienista'', algo como uma imagem de cidade resguardada do poder social
das classes abastadas e valores correlatos. De fato existiu a cidade
anti-higiênica, mas muito mais como produto ou como efeito, não como
origem. Como será possível algo que não existia concentrar tamanha
consistência ontológica? Como se elaboram os sólidos contornos
discursivos e mecanismos de poder capazes de trazer à luz ou fazer
emergir sob a condição de \textit{coisa} algo cuja clarividência não foi
manifesta?

\begin{quote}
Casamentiss faiz malandris se pirulitá. Em pé sem cair, deitado sem
dormir, sentado sem cochilar e fazendo pose. Copo furadis é disculpa de
bebadis, arcu quam euismod magna. Admodum accumsan disputationi eu sit.
Vide electram sadipscing et per.
\end{quote}

É como se substituíssemos --- segundo os usos possíveis de Nietzsche e
Foucault --- uma metafísica da representação por uma história política da
verdade, cujos elementos histórico-filosóficos primordiais passassem a
ser o sentido e o valor, e deixassem de ser o conhecimento e a
ideologia, a verdade e o erro, o saber em recuo em relação ao poder. Em
Nietzsche, o sentido indica qual vontade ocasionalmente se apoderou de
uma coisa e chegou por fim a tomar a palavra. O valor diz respeito à
materialidade da força, diz respeito à natureza moral do impulso em
questão. Ele diz: por muito tempo se procurou a razão de algo, e a
história de algo na sua finalidade e a utilidade de algo em sua forma,
``o olho tendo sido feito para ver, e a mão para pegar''. ``Mas todos
esses fins'', segundo a \textit{Genealogia da moral},

\begin{quote}
todas as utilidades são apenas indícios de que uma vontade de poder se
assenhorou de algo menos poderoso e lhe imprimiu o sentido de uma
função; e toda a história de uma ``coisa'' {[}em nosso caso, a cidade
anti-higiênica, o cortiço como uma matriz de epidemias ou como circo de
imoralidades{]} (\ldots{}) pode desse modo ser uma ininterrupta cadeia
de signos de sempre novas interpretações e ajustes, cujas causas nem
precisam estar relacionadas de maneira meramente casual.\footnote{Friedrich
  Nietzsche, \textit{Genealogia da moral: uma polêmica} {[}1887{]}, trad.
  Paulo César de Souza (São Paulo, Cia. das Letras, 2009), p.~61 (2ª
  Dissertação, §12).}
\end{quote}

É desse modo que a emergência histórica de um objeto de conhecimento, de
um universal, de uma coisa é uma sucessão de ``processos de subjugamento
que nela ocorrem, mais ou menos profundos, mais ou menos
interdependentes, juntamente com as resistências que a cada vez
encontram, as metamorfoses com o fim de defesa e reação, e também o
resultado de ações contrárias bem-sucedidas''.\footnote{\textit{Ibidem,}
  p.~61 (2ª Dissertação, §12).} Isso nos permite indagar como esse
dispositivo fez com que certos conceitos ou objetos que não existiam se
tornassem algo, algo que não é uma ilusão, porque é um conjunto de
``práticas reais, que estabeleceu isso e, por isso, o marca
imperiosamente no real''.\footnote{Michel Foucault, \textit{Nascimento da
  biopolítica} {[}1978-79/2004{]}, trad. Eduardo Brandão (São Paulo,
  Martins Fontes, 2008), 27.}

O método não consistiria em isolar invariáveis como o cortiço, a
epidemia etc., entendidos como representações invariantes, para em
seguida fazer a sua história, deduzir deles práticas políticas concretas
ou conduzir uma representação ao seu referente determinante, de modo a
se restabelecer um núcleo de verdade historicamente silenciado que
carece de ser redescoberto pela História. Como em Foucault, não se trata
de mostrar

\begin{quote}
como esses objetos ficaram por muito tempo ocultos, antes de ser enfim
descobertos, não se trata de mostrar como todos esses objetos não são
mais que torpes ilusões ou produtos ideológicos a serem dissipados à luz
da razão que enfim atingiu seu zênite. Trata-se de mostrar por que
interferências toda uma série de práticas --- a partir do momento em que
são coordenadas a um regime de verdade ---, por que inferências essa
série de práticas pôde fazer que o que não existe (\ldots{}) se tornasse
porém uma coisa, uma coisa que no entanto continuava não existindo. Ou
seja, (\ldots{}) o que eu gostaria de mostrar é que foi certo regime de
verdade e, por conseguinte, não um erro que fez que uma coisa que não
existe possa ter se tornado uma coisa.\footnote{\textit{Ibidem}, p.~26-7.}
\end{quote}

Uma mudança estratégica do dispositivo médico-higienista a partir de
meados da década de 1860: já não surpreende a febre amarela, ela
infiltra o cotidiano dos jornais, dos periódicos especializados,
infiltra a literatura, os consultórios, o anfiteatro das faculdades. A
epidemia possui um calendário anual de conflagração, ela é um fenômeno
de veraneio, não há mês de dezembro sequer nas décadas de 1860-1870 em
que não ressurja o debate. Nas memórias \textit{Brasil: terra e gente},
Canstatt define, em 1871, um pouco do que será a experiência da epidemia
no contexto do nascimento da cidade anti-higiênica: ``compatriotas que
conheci durante minha permanência na cidade, e a quem as circunstâncias
não permitiam escolher à vontade seus domicílios, asseguraram-me que com
uma longa permanência no Rio arrisca-se dez anos de vida''.\footnote{Oscar
  Canstatt, \textit{Brasil: terra e gente, 1871}, 2002, p.~304.} Apaga-se
aos poucos a ligação da epidemia com causas acidentais e passageiras,
como condições atmosféricas, meteorológicas. Ela tampouco é redutível à
importação de um gérmen nocivo. A epidemia possui causas históricas,
imanentes e urbanas, sua ação se torna permanente, sua terapêutica é de
ordem normativa e indiretamente individualizante. Por sua vez, os modos
de investimento do dispositivo médico-higienista combinam-se ou se
alternam em um tensionamento de três objetos: (1) hábitos e costumes que
não aderem ou não são adequados às formas de exigência de convívio
social precisam ser policiados; (2) este meio naturalizado que é a
cidade para os seus citadinos, o meio urbano e as ruas nas quais
crescemos e trabalhamos na cotidianidade do seu presente absoluto, essa
cidade grosseira e suja que nos foi legada, e que a partir de um ponto
se tornará passível de um projeto de reurbanização (e já não é preciso
dizer que a reforma urbana do prefeito Haussmann em Paris tornou
possível ao turista brasileiro na Europa a ambição de agir
historicamente sobre a sua cidade); (3) a arquitetura do edifício
domiciliar e os usos do casario reciclado na forma de cortiços e
estalagens na área central, que darão abrigo ao negro e ao operário
imigrante.

Focos de infecção aumentam proporcionalmente à população. A cidade
esperara 22 anos (entre 1850 e 1872) para passar de 166.419 habitantes
para 270.773, e não necessitaria mais do que 18 anos para crescer 94,2\%
entre 1872 e 1890. Em 1869, morreriam em decorrência de febre amarela
271 doentes, número de pouca expressividade se considerarmos uma
mortalidade geral de 8.294 pessoas naquele mesmo ano.\footnote{Cf.
  Estatísticas oficiais sobre febre amarela no Rio de Janeiro, entre
  1850 e 1907, em Plácido Barbosa e Cassio Barbosa de Resende, \textit{Os
  serviços de saúde pública no Brasil, especialmente na cidade do Rio de
  Janeiro de 1808 a 1907 (esboço histórico e legislação) --- Primeiro
  Volume} (Rio de Janeiro, Imprensa Nacional, 1909).} Porém os estragos
feitos pela febre amarela na década seguinte demonstram seu caráter
epidêmico com tanta ou mais intensidade do que no terrível ano de 1850.
O ano de 1873 registrou oficialmente 14.804 óbitos em geral, a quarta
parte desses corresponde a pessoas que sucumbiram à febre amarela
(daqueles, inclusive, 1.676 morriam por complicações de ``outras
febres'' e 422 foram vítimas de erisipelas, disenterias ou
diarreias).\footnote{Cf. Cândido Barata Ribeiro, \textit{Quais as medidas
  sanitárias\ldots{}}, 1877, p.~57.} Em 1876 morreriam 5.476 infelizes
doentes de febre amarela, o que correspondia a mais de 40\% do total de
óbitos na cidade naquele ano.

A febre amarela tem sua razão de ser em causas fixas e permanentes, e
afirmar o contrário seria negligenciar a gravidade da sua ação
recorrente. A matéria-prima das epidemias reside nas condições naturais
e históricas da região habitada. Ao final dos 13 anos em que esteve à
frente da Junta de Central de Higiene Pública, o emérito higienista
Francisco Paula Candido procurou demonstrar como as epidemias de febre
amarela, peste e \textit{cholera morbus} são moléstias infecciosas
resultantes da ação recíproca de dois elementos: o gérmen e o
miasma.\footnote{ARQUIVO NACIONAL. MAÇO IS 4-24, Série Saúde ---
  Higiene e Saúde Pública --- Instituto Oswaldo Cruz, sem paginação.}
Aqui tem havido miasmas há muito e em circunstâncias acessórias das mais
diversas. Os miasmas são a \textit{matéria-prima das epidemias} e só com a
reciprocidade da ação de ambos resultará o fenômeno epidêmico. O gérmen
morbífico da febre amarela, ou de qualquer outra epidemia infecciosa,
não se desenvolverá para a produção de uma grande epidemia desde que na
presença da \textit{matéria-prima da epidemia}: o miasma. O \textit{gérmen}
é um mero \textit{excitador} que intervém nas condições sanitárias da
cidade. Ele apenas desdobra-se em compostos orgânicos capazes de
propagar a respectiva moléstia na condição de existirem em um país, ou
mesmo em um indivíduo, as emanações em estado receptível, quer dizer, os
miasmas. Explosões epidêmicas contra as quais se alvoroça o governo não
existiriam, portanto, caso fosse eliminada a preexistência de miasmas. O
Barão de Lavradio, sucessor como presidente da Junta, vale-se dessa
estatística mortuária entre os anos de 1869 e 1876 para concluir um de
seus relatórios nesses termos: ``Estas perturbações progressivas e
constantemente aumentadas não podem em minha opinião ser devidas a
causas acidentais e passageiras, como sejam as condições meteorológicas
e atmosféricas, (\ldots{}) e sim a causas permanentes, cuja ação se
exerce de um modo contínuo''.\footnote{Cândido Barata Ribeiro,
  \textit{Quais as medidas sanitárias\ldots{}}, 1877, p.~59.} O mesmo
Barão de Lavradio, no livro chamado \textit{Esboço Histórico das epidemias
que têm grassado na cidade do Rio de Janeiro desde 1830 a 1870,}
reavalia os tempos idos de 1850 e observa que, se não fossem as
calamidades da febre amarela de então,

\begin{quote}
não se teria talvez tomado tão cedo a importante medida da remoção dos
enterramentos nas igrejas, reclamada desde 1829 pela sociedade de
medicina, porque a superstição religiosa, ou antes o fanatismo, e a
quebra de mesquinhos interesses teriam ainda achado pretextos para
fazê-la adiar; não teríamos por certo uma repartição de saúde embora mal
organizada para satisfazer os fins de sua criação, mas que, apesar
disso, não deixa de ter prestado bons serviços, quer quanto à higiene
pública, quer quanto à polícia sanitária (\dots{}). \footnote{José P. Rego,
  \textit{Esboço Histórico das epidemias que têm grassado na cidade do Rio
  de Janeiro desde 1830 a 1870} (Rio de Janeiro, Typographia Nacional,
  1872), p.~203.}
\end{quote}

No capítulo anterior nos aproximamos da hipótese de que foi em função de
certa visibilidade do cadáver que o dispositivo médico-higienista, a
partir da segunda metade do \textsc{xix}, estica a narrativa sobre a epidemia de
febre amarela do ``controle sobre uma nova cultura fúnebre'' para aquilo
que será a sua maior ambição nas décadas seguintes: a desarticulação dos
princípios miasmáticos em seus núcleos potenciais de emissão. Procuramos
ali estudar a instância da reflexão sobre a epidemia \textit{na} prática
higienista e \textit{sobre} a prática higienista. Pouco nos auxiliaria
tomar a ``consciência de si'' da medicina higienista como ponto de
partida. Ao contrário, é preciso ter as práticas concretas como pontos
de partida, os diferentes acontecimentos, êxitos, as ações oportunas que
se pautam por esse suposto algo que é a experiência da epidemia, e a
partir daí ver qual história podemos fazer com essas coisas.\footnote{Essa
  não muito segura tarefa de investigar a experiência da epidemia
  (inclusive) na \textit{prática} do dispositivo higienista (já que tanto
  ou mais viável seria uma história das ideias médicas ou uma história
  da política interna das instituições de saúde) no mínimo nos reserva
  do risco de submeter um espaço-tempo às legislações e códigos que ele
  produz. Porque, se levamos ao pé da letra a legislação sobre
  inumações, por exemplo, o \textit{Código de Posturas da Câmara Municipal
  do Rio de Janeiro}, promulgado em 28 de janeiro de 1832, já ali se
  fazia observar a proibição de ``enterrarem-se corpos dentro das
  igrejas, ou nas sacristias, claustros dos conventos, ou quaisquer
  outros lugares nos recintos dos mesmos'', com multas de ``30\$ de
  condenação'', e ``oito dias de cadeia'' aos coveiros que fizerem as
  covas (§1º da Seção I: Saúde Pública --- Título I: Sobre cemitérios e
  enterros).} Seria possível fazer outra história que não a que
propusemos: uma história da medicina social no Brasil que não fosse além
do projeto inicialmente gestado, revelar quais foram os requisitos e as
funções visadas, e definir quais eram os ideais conduzidos pela
legislação. Seria possível, paralelamente, estabelecer o saldo entre o
previamente idealizado e o que se desdobrou efetivamente procurando os
níveis de correspondência entre discursivo e extradiscursivo.
Parece-nos, entretanto, que as coisas não seriam tão simples, como se
entre relações discursivas e a determinação das práticas políticas se
estabelecesse uma sorte de causalidade vertical instantânea.

Optamos então por um descentramento, uma passagem ao exterior em relação
à função da ciência, para tentar fazer um tipo de análise estratégica
assumindo um ponto de vista que saltasse para fora das relações entre a
instituição higienista, seu berço e sua história. Descentramento que
teve a ver com a sublimação do ``ponto de vista interno da função pelo
ponto de vista externo das estratégias e práticas''.\footnote{Michel
  Foucault, \textit{Segurança, território, população} {[}1977-78/2004{]},
  trad. Eduardo Brandão (São Paulo, Martins Fontes, 2008), p.~158.} Ora,
não é possível tudo dizer em qualquer tempo. Ou melhor: embora nem
sempre o que falamos se encadeie com o que vemos, há um regime histórico
das coisas ditas que é inseparável dos enunciados que ali ganham corpo
na medida em que os condiciona. Esse regime de enunciabilidade, ou
formação discursiva, é irredutível a estatutos e pretensões de verdade
científicas, por isso nosso campo de trabalho não é a ciência médica
enquanto tal. Interessa-nos como se distribuem enunciados possíveis em
seus espaços históricos de projeção e dispersão. Quando dizemos que nem
tudo pode ser dito em qualquer tempo é porque, em uma formação histórica
através da qual um enunciado se projeta, pouco importa a intenção que se
esconde por trás do que é dito, importa ver como esse enunciado e não
outro enunciado manifesta essa realidade (``não há possível nem virtual
no domínio dos enunciados; nele tudo é real, e nele toda realidade está
manifesta: importa apenas o que foi formulado, ali, em dado momento, e
com tais lacunas, tais brancos'').\footnote{Gilles Deleuze,
  \textit{Foucault} {[}1986{]}, trad. Claudia Sant'Anna Martins (São
  Paulo, Brasiliense, 2005), p.~14.} Por que esse recorte de realidade e
não outro? Aqui, entre memória e esquecimento, não há nenhuma
exterioridade ou duplicação metafísica, no sentido de que caberia
redirecionar a questão nos termos: como uma dada formação histórica
impõe um repertório de problematizações, ou horizonte de objetivações,
que não é fixo ou invariável, mas que sofre rupturas, descontinuidades,
quebras epistemológicas? A noção de esquecimento ganhará então a imagem
de um espaço branco, ou uma lacuna que não é outra lacuna senão um vazio
instaurado, de maneira que o esquecido não é o já ausente nem o registro
que se perdeu. O esquecido não é a versão carente de sentido da memória.
O esquecimento é capturado e age por vias estratégicas submetido à ordem
da memória, sendo necessário, portanto, integrar seu silêncio no
movimento que uma narrativa histórica realiza para se firmar. É o que
diz Foucault em sua \textit{História da loucura}: ``A história só é
possível sobre o fundo de uma ausência de história, no meio desse grande
espaço de murmúrios que o silêncio espreita, como sua vocação e sua
verdade''.\footnote{Michel Foucault, ``Folie et déraison (prefácio)''
  {[}1961{]}, em \textit{Ditos e Escritos I --- Problematizações do
  sujeito}, trad. Vera L. A. Ribeiro (Rio de Janeiro, Forense
  Universitária, 1999), p.~156.}

A partir das décadas de 1860 e 1870, parece-nos, os objetos do
dispositivo médico-higienista (as redes de razões morais, estéticas e
higiênicas que sustentam a reurbanização e a extinção dos cortiços)
apenas se articulam porque há uma abertura bastante singular do
horizonte ``fenomenológico'' sobre a cidade. Se, como dissemos, uma
formação discursiva, ou regime de enunciabilidade, é uma multiplicidade
de discursos (textos literários, atas da Academia Imperial de Medicina,
um despacho administrativo, um Tratado de Higiene etc.); se são
encadeados segundo regras de formação (sujeitos, métodos, objetos,
tecnologias de poder etc.) nem sempre gerando unanimidade, mas se
hierarquizando diferentemente, entrando em disputa ou soma de posições
etc., então esperamos da História, em princípio, a determinação da
relação entre os enunciados --- o que é dito --- e o campo de visibilidade
de cada época. O que permitiu que o traçado viário proveniente do
período colonial fosse considerado insalubre pelo discurso higienista?
Em outras palavras: como a espacialização das ruas da cidade segundo as
normas do antiestético e anti-higiênico se articulam de maneira
estratégica com as práticas e os enunciados do dispositivo
médico-higienista de controle dos cortiços e alinhamento da cidade? É em
última instância o jogo entre ver e dizer, visibilidades e
enunciabilidades --- desde que se mantenha presente a ideia de que um
dispositivo não é redutível às instituições de higiene pública ou
academias de medicina.

Novamente, como se produz a espacialização das ruas da cidade segundo as
normas do antiestético e anti-higiênico, e como essa espacialização se
articula estrategicamente com os enunciados do dispositivo
médico-higienista de controle dos cortiços e quadriculamento das ruas da
cidade? Como surge a tendência a naturalizar essa relação, já bastante
frequentada pelos viajantes europeus do \textsc{xix}, entre a sinuosidade das
ruas do centro da cidade, o mau cheiro e a insalubridade? Para situar
essa questão não será preciso retornar à época anterior à introdução da
febre amarela e a da cólera nos portos brasileiros, quando se mantinha
geralmente a crença de que a posição geográfica no país o defendia da
importação das moléstias pestilenciais próprias de outros climas. Esse
trabalho, que, pela sua autoridade, não foi obscurecido, já realizou
Sérgio Buarque de Hollanda em \textit{Visão do Paraíso} (1959). Navegantes
e aventureiros de outrora --- em épocas de descobrimentos --- cantavam o
encontro com essa paragem mais ou menos remota, onde nunca, em tempo
algum, morrera qualquer um de ``pestilência''. Onde os homens não
adoeciam, ou, se já doentes, logo os curava o bom temperamento da terra.
Advertiam os primeiros viajantes ibéricos sobre o bom clima, os ares de
uma terra edênica tão delgada, um céu tão benigno, uma água boa e até as
ditosas constelações que aqui prevaleciam. O Brasil não consentia haver
em toda sua costa o mal pernicioso da peste.

Ao invés de resgatar o declínio dessa percepção, percepção de uma terra
cujas atrozes pestilências desaparecem ao influxo de uma natureza sem
par, e compará-la com as memórias do \textsc{xix} a fim de marcar o que muda e o
que permanece, optamos por uma tarefa menor: identificar a produtividade
de um dispositivo de saber-poder que instaura imperiosamente no real um
regime de veridicção que permite deslocar as problematizações sobre a
epidemia (suas causas, funções, suas etapas de desenvolvimento) para
enunciados sobre a boa ou a má cidade, a boa e a má rua, a verdadeira
moradia e a moradia inadequada, a saúde e a epidemia. Tudo isso --- esses
modelos de inteligibilidade de o que desenvolve a epidemia e o que nos
previne contra ela, esses procedimentos discursivos altamente
especializados --- seguirá princípios de racionalização bastante ricos
nas receitas dos higienistas.

Sobretudo após as ocorrências de febre amarela do final da década de
1860 e início da década de 1870, a transformação da experiência da
epidemia é reforçada pela mudança que sofre a percepção da cidade, pelo
nascimento do nexo entre a forma das ruas e a insalubridade, a forma das
casas e a imoralidade e a febre amarela. Entre essas diferentes
práticas, que vão do policiamento e fechamento de cortiços aos projetos
para uma profunda cirurgia urbana, é possível estabelecer, nas palavras
de Foucault, uma coerência pensada, racionalizada,

\begin{quote}
uma coerência estabelecida por mecanismos inteligíveis que ligam essas
diferentes práticas e os efeitos dessas diferentes práticas uns aos
outros e vão, por conseguinte, permitir julgar todas essas práticas como
boas ou ruins, não em função de uma lei (\dots{}), mas em função de
proposições que serão, elas próprias, submetidas à demarcação do
verdadeiro e do falso.\footnote{Michel Foucault, \textit{Nascimento da
  biopolítica} {[}1978-79/2004{]}, 2008, p.~25-6.}
\end{quote}

Policiamento da circulação: o que é uma boa rua? O que é uma boa
circulação e o que é a má circulação? Há um repertório de
problematizações que ganha vida própria, e são os engenheiros da Escola
Politécnica, os higienistas e também a polícia e a municipalidade que
situarão o debate em torno de regiões da experiência como a cidade e o
cortiço. Do que se trata de organizar quando se fala em organizar a
circulação urbana? É de se arriscar dizer, como pensa Foucault, que
organizar a circulação tem a ver com gerir relações de força, como quem
escoa ou canaliza fluxos de poder, ou os retém por vezes. Seria então
``preciso fazer uma `história dos espaços' --- que seria ao mesmo tempo
uma `história dos poderes' --- que estudasse desde as grandes estratégias
geopolíticas até as pequenas táticas do \textit{habitat}''.\footnote{Michel
  Foucault, \textit{Microfísica do poder}, 1979, p.~212.} Uma história dos
espaços que investigasse sob quais formas a cidade e o cortiço foram
problematizados, e a partir de quais regiões da experiência tornaram-se
objetos de cuidado, elementos para reflexão, matéria para a
normatização. Um mapeamento histórico que permita questionar como estes
elementos foram inscritos no real, e o mais importante, como eles foram
subordinados a um regime que demarca o verdadeiro e o falso.

O objeto deste capítulo, concernente à experiência da epidemia, é pensar
o desenvolvimento de um dispositivo de saber-poder que baliza
efetivamente a realidade sanitária urbana segundo a artificialidade de
uma relação causal entre cidade colonial e a produção de miasmas, ou
cortiços e insalubridade. O que é uma boa rua e uma má rua? O que é uma
verdadeira moradia e uma moradia inadequada? As condições que entre nós
possibilitaram esse histórico de problematizações estão ligadas à
submissão legítima da rua colonial e do cortiço à demarcação do
verdadeiro e do falso conforme os critérios da verdade higienista. A
capacidade de introduzir na materialidade urbana nexos de causalidade
antes inexistentes, e subordinar esse cercado de realidade a um novo
regime que recorta o verdadeiro e o falso, a saúde e a epidemia; é dessa
valoração sobre a natureza das ruas e da habitação que uma nova
experiência da epidemia extrai um caráter, um ponto de referência e uma
razão de ser. ``Parto da decisão'', diz Foucault, ``ao mesmo tempo
teórica e metodológica, que consiste em dizer: suponham que os
universais não existem''.\footnote{Michel Foucault, \textit{Nascimento da
  biopolítica} {[}1978-79/2004{]}, 2008, p.~5.} Ou antes: suponhamos que
tais nexos de causalidade --- ruas estreitas, cortiços e febre amarela ---
inexistam (e tais nexos de fato parecerão absurdos às políticas públicas
de controle da febre amarela do início do \textsc{xx}). O que os faz emergir e
adquirir regularidade enunciativa?

As memórias de viajantes suíços, franceses, norte-americanos, alemães de
passagem pelo Rio de Janeiro ao longo do século \textsc{xix} propõem certos nexos
de causalidade que ganharão uma impressionante regularidade nos ciclos
de epidemias de febre amarela da década de 1870, talvez muito em função
da institucionalização do higienismo. Nossa inquietação é menos sobre se
esse discurso condiz com a realidade do que sobre a partir de quais
categorias um novo recorte de realidade é regularmente caracterizado, a
partir de quais categorias será possível balizar verdadeiro e falso,
saúde e epidemia. Tais narrativas produzem sem dúvida certa verdade ---
elas têm compromisso com a disputa por uma narrativa hegemônica como
alternativa a outras. O tensionamento existe, nunca deixou de havê-lo.

O fato de a cidade ter se transformado do ponto de vista dos
melhoramentos de infraestrutura não impediu a proliferação das
narrativas que sustentassem o nexo causal ao qual nos referíamos, e é
isso, por exemplo, que reflete o caráter político e estratégico da
verdade que se destaca. O que tais narrativas ensaiam projetar (de forma
intencional e, entretanto, não subjetiva) são nexos causais ou regras de
formação, que não dependem necessariamente da dinâmica dos campos de
visibilidade para alcançarem estatuto de ``real''. Um regime de luz e um
regime de linguagem preexistiram ao dispositivo médico-higienista, eles
coexistiam sem que tivessem a mesma origem e a mesma formação, sem que
houvesse nivelamento entre ambos, entre jogos de luzes e enunciados. Se
as visibilidades têm como referencial o mapeamento e a disposição das
luzes e das sombras, a enunciabilidade tem como referência não a verdade
ontológica das coisas, mas as condições para o desdobramento de toda uma
rede móvel e conflituosa de ideias que marca uma época.

\begin{quote}
Entre os dois não há isomorfismo, não há conformidade, embora haja
pressuposição recíproca e primado do enunciado. Mesmo \textit{A
Arqueologia do Saber}, que insiste no primado, dirá: nem causalidade de
um a outro, nem simbolização entre os dois, e se o enunciado tem um
objeto, é um objeto discursivo que lhe é próprio, que não é isomorfo ao
objeto visível.\footnote{Gilles Deleuze, \textit{Foucault} {[}1986{]},
  2005, p.~70.}
\end{quote}

A cidade era a cidade colonial de ruas estreitas, o que chamamos de
enunciados seguia o regimento do nexo entre mau cheiro e malha viária. O
que mais tarde vem se somar a isso, integrando-se nessa nova formação
histórica --- e, digamos, em uma nova ``época'' para história da
experiência da epidemia ---, são práticas e tecnologias de poder que
mobilizam e institucionalizam a produção de uma nova realidade.

Vejamos. Viajantes como Agassiz e Canstatt estiveram na cidade em 1865 e
1871, respectivamente. Julgaram defeituoso ou mal edificado o calçamento
das ruas do centro, atribuindo à municipalidade o mesmo grau de incúria
e desleixo que autores que visitaram a cidade nas décadas de 1840 e
1830. Consta, porém, como muitas ruas das freguesias centrais já em 1854
haviam substituído os pés-de-moleque pelo calçamento com
paralelepípedos,\footnote{Cf. Sonia Gomes Pereira, \textit{A Reforma
  Urbana de Pereira Passos}\ldots{}, 1998, p.~97.} contornando a
desigualdade de alguns terrenos, corrigindo os buracos que estagnavam
águas como focos de infecção e precavendo os transeuntes do incômodo das
antigas calçadas.

Não é exagero dizer que a partir das décadas de 1860-70 acontece um
impulso em direção a melhoramentos materiais da cidade, a administração
pública se revela disposta a remediar os efeitos da imprevidência
mantida durante as epidemias da década anterior. É durante esse período
que começam as obras do canal do mangue da Cidade Nova, servindo não só
de agente de comunicação entre a Praça XI e a Praia Formosa, mas
anulando o mau cheiro das emanações da região de mangue. Somado a isso,
um acontecimento de importância: inicia-se a grande obra dos esgotos e
despejos públicos após um certo número de tentativas malogradas. Alguns
autores dirão ter sido a Corte uma das primeiras capitais do mundo a
contratar uma companhia para instalar um moderno sistema domiciliar de
esgotos (``para uns foi a terceira capital, depois de Londres e Paris;
para outros, apenas Hamburgo e as maiores cidades da Inglaterra
precederam nessa iniciativa o Rio de Janeiro'').\footnote{Jaime Larry
  Benchimol, \textit{Pereira Passos: um Haussmann tropical}, 1992, p.~73.}
As obras começam a sair do papel em 1862, após não curto período de
adiamentos, justificativas e laranjas.

Na verdade, em 1853 o Estado Imperial abria concorrência para uma
empresa que empreendesse a construção de um sistema de esgotamento
sanitário. Duas propostas foram à disputa. A Companhia Hanquet propunha
esgotar a cidade através do aperfeiçoamento da antiga peregrinação dos
\textit{tigres}: instalação de tubulações nas residências de modo a
conduzir esgotos e águas servidas a barris lacrados e desinfetados. Os
dejetos, transportados em carros fechados para fora da cidade, seriam em
seguida incinerados. Contrapondo-se à proposta, o inglês John F. Russel
apresentou seu projeto à exploração do serviço: ``consistia na
construção de rede de esgotos em toda a área central da cidade e
transporte dos esgotos coletados para (\ldots{}) um tanque de
precipitação química. O material sólido seria vendido como adubo e o
efluente da estação encaminhado até o mar''.\footnote{Eduardo César
  Marques, ``Da Higiene à construção da cidade: o Estado e o saneamento
  no Rio de Janeiro'', \textit{História, Ciências, Saúde --- Manguinhos},
  Vol. II (2), Jul.-Out. 1995, p.~58.} Como parte da estratégia de
convencimento das autoridades, Russel e seu sócio Lima Júnior
experimentam a tecnologia para um sistema de \textit{main drainage}
(``considerado seguro, moderno e eficiente pelos engenheiros ingleses do
\textit{Civil Engineer Institute of London}'')\footnote{Verena Andreatta,
  \textit{Cidades quadradas, paraísos circulares: os planos urbanísticos
  do Rio de Janeiro no século \textsc{xix}} (Rio de Janeiro, Mauad X, 2006),
  p.~133.} na Casa de Detenção, empregando braços dos próprios
encarcerados. Em 1857 o governo decide conceder o privilégio à dupla de
sócios, com um prazo pré-fixado para início das obras. Passados 18 meses
nada havia sido feito.

\begin{quote}
Após sucessivos adiamentos e justificativas, o contrato foi transferido
para a empresa de capital inglês \textit{The Rio de Janeiro City
Improvements Company} em maio de 1863. O contrato elaborado pelo governo
imperial deixava aberta a possibilidade de utilização de capitais
estrangeiros nas empresas concessionárias. Ao que tudo indica, a empresa
de Russel era apenas ponta de lança dos interesses da \textit{City}: seus
estudos preliminares foram realizados na Europa, e o projeto definitivo
levava a assinatura de Edward Gotto, futuro presidente da \textit{City}.
Segundo Coelho, Lima Júnior e Russel receberam 89 mil libras esterlinas
em troca das transferências da concessão.\footnote{Eduardo César
  Marques, ``Da Higiene à construção da cidade'', 1995, p.~58.}
\end{quote}

Seguem finalmente as obras, e algumas ruas centrais ficam
temporariamente intransitáveis por causa das escavações abertas pelo
serviço da companhia. Em 1864, enquanto ainda não se concluem os
trabalhos, algumas firmas privadas exploram o serviço de recolhimento de
detritos nas residências. Uma das mais importantes do ramo foi a Nova
Empresa de Matérias Fecais, localizada na ``Rua da Alfândega, que cobra
dois mil e quinhentos réis pela assinatura de um mês. A localização da
cocheira junto à casa exige trabalho dobrado de limpeza e
higiene''.\footnote{Delso Renault, \textit{Rio de Janeiro: a vida da
  cidade refletida nos jornais} (Rio de Janeiro, Civilização Brasileira,
  1978), p.~261.} Em 1866, a \textit{City} inicia a entrega dos seus
serviços nos três distintos iniciais em que é dividida a concessão: São
Bento (Arsenal), Gamboa e Glória. Nesse mesmo período a rede de esgotos
é estendida às freguesias centrais da cidade e, em 1868, a empresa em
pleno funcionamento já cobrava taxas mensais pelos serviços prestados.
Os números da rápida ascensão do sistema de esgotamento são
impressionantes. ``Dados do Ministério da Agricultura para o ano de 1875
mostram que 14.827 casas estavam ligadas à rede de esgotos, o que
representava 46,5\% do total de 30.000 habitações existentes no
município''.\footnote{Jaime L. Benchimol, \textit{Pereira Passos: um
  Haussmann tropical}, 1992, p.~73.} Segundo Pereira Rego, com esses
melhoramentos cessam despejos nas praias em barris conduzidos em
carroças ou à cabeça pelos escravizados.\footnote{Cf. José P. Rego,
  \textit{Esboço Histórico das epidemias\ldots{}}, 1872, p. 208.}

Em 1871, a municipalidade distribui mijadouros e latrinas públicas pelo
centro das praças e sobre os passeios das freguesias centrais,
providência que se julgou indispensável em uma cidade como a do Rio de
Janeiro, ``prova do interesse que há pelo melhoramento do nosso estado
sanitário, pelo embelezamento desta cidade, e pela moralidade
pública''.\footnote{``Discurso pronunciado pelo Exmo. Sr.~Conselheiro
  Dr.~José Pereira Rego na sessão solene da Academia Imperial de
  Medicina em 30 de Junho de 1871''. Annaes Brasilienses de Medicina ---
  Tomo \textsc{xxiii}, n. 2, julho de 1871, p.~18.} Tudo nos dá a entender que a
cidade se tornaria mais asseada, com uma aparência geral mais agradável.
Valas esvaziadas, cessação dos despejos nas praias e no Campo de
Santana, esperava-se que o mau cheiro chegaria a um termo, mas não é
essa impressão que compartilham médicos e engenheiros na década de 1870.
Quanto ao Campo de Santana, descrito por estrangeiros como Pfeiffer e
Expilly como um lugar detestável, onde se podia encontrar lixo acumulado
e cadáveres de cães e equinos dos anos 1820 até 1870 (o Campo era o
lugar mais próximo para lavagem de roupa de que dispunha a população dos
bairros centrais mais antigos --- ornado com chafariz e bicas da água
abastecida pelo Rio Andaraí), ele sofreria nos anos 1870 trabalhos de
paisagismo que o transformariam em um suntuoso parque público. Os
despejos (assim como a lavagem de roupa no local) ficariam proibidos.

\begin{quote}
Carl von Koseritz, imigrante alemão radicado no Rio Grande do Sul, ao
visitar o Rio de Janeiro nos anos 1880, não conteve seu entusiasmo
diante do crescimento de uma cidade que ele visitara nos anos 1850.
(\ldots{}) Koseritz, tal como Expilly nos anos 1860, mencionou o mau
cheiro do Rio de Janeiro e sua corrupção atmosférica, principalmente na
área da Prainha, freguesia de Santa Rita. (\ldots{}) Porém os imensos
pântanos, próximos da parte mais antiga da cidade, já tinham sido
aterrados e o outrora sujo Campo do Santana foi transformado ``num dos
mais belos parques do mundo''.\footnote{Luís Carlos Soares, \textit{O
  ``Povo de Cam'' na Capital do Brasil}, 2007, p.~36.}
\end{quote}

O mau cheiro do Rio de Janeiro e a corrupção atmosférica são
esporadicamente associados à presença tóxica do \textit{tigre} no
exercício das ``atividades de esgotamento'' da sociedade, mas também à
repugnância diante da população em situação de rua. Seidler não precisou
de muito para levantar mais de um porquê para nossas ruas serem tão
sujas. É uma população comodista, como ele diria, comodistas em manejo
da coisa pública, comodistas no tratamento das fachadas das casas.
Indiscrição das fachadas, que praticamente transbordam a pouca largura
das ruas. Indiscrição nos usos da rua.

\begin{quote}
Com isso chego a um outro grande mal a que, com grande pesar dos órgãos
mais nobres, se está exposto em todas as ruas, praças públicas e
principalmente na praia, a todas as horas do dia e da noite. É que os
moradores do Rio são muito comodistas e por isso não gostam de
comodidades à distância adequada (\dots{}).\footnote{Carl Seidler, \textit{Dez
  anos no Brasil} {[}1835{]}, 2003, p.~62-3.}
\end{quote}

Obscenidade (das entranhas das ruas) como forma de caracterizar a
mendicância e romantizar a miséria. Existem outras formas mais ou menos
ardilosas --- bordejamos alguns desses aspectos no início do capítulo. O
reverendo Kidder em 1838 diferenciava os meros vadios que se faziam
passar por mendigos, dos indigentes verdadeiros. Esses últimos, com
exclusividade, ``necessitavam da caridade pública'' e deviam ``ter plena
liberdade de exercer a mendicância''.\footnote{Daniel Parish Kidder,
  \textit{Reminiscências de viagens e permanências no Brasil} {[}1845{]},
  2001, p.~90.} Vadios ou autênticos, a indiscreta população na prática
da mendicância, a esmolar na porta das missas, a compor o quadro
antiestético e anti-higiênico da rua colonial, era ``devida à brandura e
ao descaso mesmo, da polícia'',\footnote{\textit{Ibidem}, p.~90.} segundo
Kidder. Sim, as coisas tendem a mudar de figura, pelo menos no âmbito
institucional. Na segunda metade do \textsc{xix} a chefia de polícia empregaria
contra os vagabundos-mendigos (em sua maioria ex-cativos, ``gente de
cor''\footnote{Luís Carlos Soares, \textit{O ``Povo de Cam'' na Capital do
  Brasil}, 2007, 189.} cega e idosa, que apoiava em um bastão a
elefantíase) hábil expediente. O \textit{Diário do Rio de Janeiro}, na
década de 1850, estampa o cotidiano das prisões de vadios e pedintes.
São em sua maioria estrangeiros, alforriados ou escravizados de ganho,
não livres. Kidder recorda ocasião em que se oferece aos guardas uma
gratificação de 10\$000 por pedinte que se conseguisse prender e levar à
Casa de Correção. ``Dentro de poucos dias as autoridades recolheram nada
menos de cento e setenta e um vagabundos''.\footnote{Daniel Parish
  Kidder, \textit{Reminiscências de viagens e permanências no Brasil}
  {[}1845{]}\textit{,} p.~90.}

Desde 1852 existia um Asilo da Mendicidade, na praia de Sta. Luzia, mas
ele não pôde abrigar suficientemente o afluxo da população flutuante e
os comodistas de ``todas as horas do dia e da noite''. O número de
transeuntes na mendicância, inimigos do asseio das ruas, ``não excedia a
500'', segundo o Dr.~Moraes e Valle em meados do \textsc{xix}, número ``que era
extremamente insignificante numa cidade populosa como a capital do
Império brasileiro''.\footnote{Luís Carlos Soares, \textit{O ``Povo de
  Cam'' na Capital do Brasil}, 2007, p.~187.} Quando esse estado de
coisas chega ao auge da tolerância, o controle administrativo do
policiamento da cidade começa a ganhar um padrão e o êxito se faz
visível nas ruas, a partir da década de 1860, na vigilância dos
comportamentos inaceitáveis. O controle se exemplifica no número cada
vez maior de prisões efetuadas por violações às normas aceitáveis da
ordem pública. Em 1862, 7.290 detidos passaram pela cadeia da delegacia
central de polícia. ``As prisões de 1862 pelos motivos mais comuns ---
vadiagem e violação do toque de recolher --- quase igualam o total de
prisões registradas em 1850''.\footnote{Thomas H. Holloway,
  \textit{Polícia no Rio de Janeiro: repressão e resistência numa cidade
  do século \textsc{xix}} (Rio de Janeiro, Ed. Fundação Getúlio Vargas, 1997),
  p.~195.} Em 1865, a ``vadiagem'' e o ``vagar fora das horas'' integram
quase metade das prisões por ofensas à ordem pública, enquanto 15
pessoas eram presas por mendicidade --- embora Holloway, em seu
\textit{Polícia no Rio de Janeiro}, argumente ser pouco provável haver
esse número desprezível de mendigos nas ruas da cidade. As estatísticas
sobre a prisão refletem certa descriminalização da mendicância como
infração punível nesse período. Há novamente um aparente acréscimo de
tolerância com relação à população em situação de rua.

\begin{quote}
Eram vários os problemas causados pelos mendigos, e eles persistiam
apesar dos inúmeros esforços empreendidos ao longo dos anos pelas
autoridades policiais, ora paternalistas, ora repressivas. Os
mendigos eram repugnantes, inconvenientes, anti-higiênicos e causavam má
impressão aos visitantes da cidade; a vida nas ruas era prejudicial às
boas qualidades morais e aos bons hábitos de trabalho desejáveis. A
polícia ainda prendia algumas pessoas por mendicância, mas, contanto que
os pedintes não criassem problemas mais graves (\dots{}).\footnote{\textit{Ibidem},
  p.~196. Grifo nosso.}
\end{quote}

O sistema policial começa a adotar uma atitude mais permissiva em
relação à mendicância nesse curto espaço de tempo. É claro que seria
absurdo, ou no mínimo extremamente oneroso, querer fixar o que rege os
limites entre ``vadiagem'' e ``mendicância'' neste grande guarda-chuva
que são os delitos ``contra a ordem pública''. Pra todo efeito, a linha
é tênue, sobretudo antes da ampliação e reforma do sistema judicial em
1871. ``Acabar com a confusão anômala entre as autoridades policiais e
judicial foi a principal característica da reforma de 1871, que tratou
de ampliar os sistema judicial para que este assumisse as funções antes
desempenhadas por chefes de polícia''.\footnote{\textit{Ibidem,} 1997,
  p.~227.} Quer dizer, o prestígio para qualificar e julgar os delitos é
transferido aos membros da nova elite judicial: era agora frequente aos
funcionários da polícia a necessidade de um mandado escrito emitido por
juiz para qualquer prisão que não fosse em flagrante. Isso, no entanto,
não surtiu efeito no número crescente de detentos, segundo as
estatísticas de 1875. Nesse ano, foram presas 391 pessoas por vadiagem e
137 por mendicância,\footnote{Cf. Relatório do ministro da Justiça,
  1877. (Fonte: Relatórios anuais do chefe de polícia do Rio de Janeiro
  e do ministro da Justiça, Arquivo Nacional, vários anos)} ou seja, a
mendicância volta a responder por um número significativo de prisões no
período de surtos epidêmicos, embora a violação do toque de recolher
diminua drasticamente (desde 1825 o ``toque de Aragão'' era assinalado
pelas igrejas às 22h). Três anos depois seriam suspensos os sinos do
decreto de Aragão --- eles tocariam pela última em 19 de setembro de
1878. Pois bem, grande parte das prisões em 1875 (72\% delas)
correspondiam a violações da ordem pública, porém o número de 137 presos
por cometerem mendicância é bastante modesto: a estrutura policial não
daria conta do registro dos crimes, de modo que das 9.994 detenções
realizadas em 1875 apenas 987 foram discriminadas segundo a sua
qualidade. Ou seja, há talvez 10 vezes mais indivíduos em situação de
rua, em meados da década de 1870, do que revelam os relatórios do chefe
de polícia da cidade.

Até às vésperas da abolição, podia-se encontrar entre os indigentes do
Rio de Janeiro um número considerável de escravizados forçados à
mendicância. Ao fim de suas vidas, nem todo senhor concedia a liberdade
ou lançava no abandono seus escravizados antigos, idosos, inválidos e já
improdutivos. Em 1867, escreve Perdigão Malheiro:

\begin{quote}
Um fato de revoltante especulação é mandarem esmolar pelas ruas escravos
cegos, enfermos ou aleijados, em proveito exclusivo dos senhores!
Abusando estes da credulidade e da caridade pública! Chegando-se mesmo
ao abuso ainda mais censurável de comprá-los para tal fim! --- A Polícia
desta Corte tem tido ocasião de o saber.\footnote{Agostinho Marques
  Perdigão Malheiro, \textit{Escravidão no Brasil: ensaio
  histórico-jurídico-social --- Parte 3ª} (Rio de Janeiro, Typographia
  Nacional, 1867), p.~129, n.~422.}
\end{quote}

Muitos viam na doença e na desgraça dos seus escravizados fontes de
exploração da caridade pública. Havia ainda uma corja de inescrupulosos
negociantes que adquiriam velhos escravizados enfermos com o fim
exclusivo de lançá-los na mendicância. Ao fim do dia, esses escravizados
retornavam aos seus senhores, que lhes tiravam a quantia obtida na
atividade de pedinte. Há, para todo efeito, uma presença massiva de
pessoas em situação de rua a compor, nesse terceiro quartel do século
\textsc{xix}, o quadro da cidade, integrando o nexo entre a sinuosidade das ruas,
o mau cheiro e a insalubridade.

Inútil querer dar a última palavra, nas narrativas daqueles viajantes,
sobre seu compromisso ou não com a verdade, assim como não funciona
apelar para a objetividade histórica da cidade em seu invólucro de
verdade. No lugar de uma historiografia como passividade contemplativa
de uma unidade coerente de fenômenos perguntamo-nos pela \textit{"}vontade
da verdade historiográfica``. Como quem, na esteira de Nietzsche,
rechaçasse a tendência de dissolver um fenômeno histórico em fenômeno do
conhecimento --- isso caso optemos por entender conhecimento histórico
não como a tarefa de forjar uma''totalidade racional``, cuja
consistência e unidade implicam um grau de silenciamento, certa não
latência, um desconhecimento daquilo que não se deixa encerrar, isto é,
o caráter inconcluso e inventivo do presente (com seus paradoxos,
contingências, com sua pressa com o inominável). Interessa-nos chamar
atenção de uma narrativa histórica pela parte que lhe cabe enquanto
integrada em relações de poder, na justa medida em que presumimos não
haver relação de poder sem constituição correlata de um regime de
verdade. A tarefa está mais próxima de uma''topografia das condições de
possibilidade" para a elaboração de um discurso higienista que
preencheu, disputou narrativas, transmitiu ou procurou naturalizar nexos
causais mais ou menos comuns. O discurso do dispositivo higienista
ajudou a compor categorias por meio das quais a cidade pôde ser
\textit{experienciada}. A indisposição diante da cidade colonial tornou-se
sensibilidade recorrente para os engenheiros da década de 1870, como um
Luiz Rafael Vieira Souto. Em plena epidemia de febre amarela no ano de
1875, o catedrático da Escola Politécnica decide imprimir um volume com
suas colunas de jornal, críticas a uma série de estudos\footnote{Ver
  nota 280 deste livro.} (encomendados na época pelo Estado Imperial),
para um plano de melhoramentos urbanos pensados para os arrabaldes do
Engenho Velho, Andaraí e S. Cristóvão. ``Não é em tais lugares, mas no
centro da cidade, que a população se acha diariamente aglomerada'' ---
seguem as palavras de Vieira Souto ---, é no centro da cidade ``que as
ruas são estreitas, tortuosas, mal arejadas e sem escoamento pronto para
as águas das chuvas; é aí que as casas são apertadas além de todo
limite, sem luz, sem ventilação e outras condições indispensáveis à
saúde''.\footnote{Vieira Souto, \textit{Melhoramento da cidade do Rio de
  Janeiro: crítica dos trabalhos da respectiva comissão} (Rio de
  Janeiro, Lino C. Teixeira \& C., 1875), p.~10.} Aí se acumulam
edifícios sem arquitetura, ruas em desalinho, feiras de gêneros em
espaços impróprios que, enquanto não receberem o socorro do Estado ``hão
de sempre contribuir para o nosso mau estado sanitário''.\footnote{\textit{Ibidem},
  p.~10.}

Parece-nos, quando a experiência da febre amarela é mediatizada por
problematizações de elementos materiais que definem a realidade urbana;
quando, dentro de um programa de execução pautado, segundo opinião de
Vieira Souto, pelo embelezamento do espaço urbano (desenhado por
engenheiros e acadêmicos), ganha relevo a urgência de uma reforma que
elimine ``\textit{focos de infecção que concorrem poderosamente para as
mesmas moléstias}''\footnote{\textit{Ibidem,} p.~12.} (causados
sobremaneira pelas ``más condições higiênicas da cidade''); parece-nos
que temos aqui, com alguma clarividência, um outro regime de verdade que
constitui uma nova cena, que não apenas não depende da chancela dos
médicos, mas que conquista uma vida independente dos Tratados de
Higiene.

Caso não se entenda por conhecimento histórico a tarefa de forjar uma
``totalidade racional'' (cuja objetividade e unidade implicam a
intolerância com aquilo que não se deixa capturar na expectativa do
isomorfismo entre palavras e coisas), pode-se alojar essas pesquisas em
uma fresta, aberta por Foucault, entre as palavras e as coisas, e pensar
um eventual conceito de verdade que ``não se defina por uma conformidade
ou forma comum, nem por uma correspondência entre as duas
formas''.\footnote{Gilles Deleuze, \textit{Foucault} {[}1986{]}, 2005,
  p.~73.} Conceito de verdade que dê conta dessa forma de integração
estratégica entre as valorações negativas sobre a situação urbana da
Corte, em convívio com as positividades da implantação de alguns
elementos instraestruturais ocorridos na cidade no terceiro quartel do
\textsc{xix}. Seria possível suspeitar haver um jogo instável e difícil, em que
uma narrativa age como veículo ou instrumento de poder, ao mesmo tempo
em que o silêncio e a negligência servem para produzir algum efeito de
poder esperado.

A omissão dos viajantes a respeito do pioneiro esgotamento da cidade, e
a ênfase dedicada ao nexo entre epidemias e a fisionomia das ruas tanto
se repetem nos artigos, livros e relatórios oficiais de médicos e
engenheiros quanto se transformam em negação e crítica severa. Em 1871
não tivemos uma epidemia de febre amarela, mas em relatório oficial o
Dr.~Guilherme J. Teixeira presta esclarecimentos de boatos sobre uma
epidemia de tifo na Glória --- freguesia onde exercia o cargo de
Presidente da Comissão Sanitária. O higienista dita as razões da sua
preocupação com a condição sanitária (constituição médica da época) da
cidade.

\begin{quote}
Se me fosse, porém, concedido emitir juízo relativamente à causa
provável da constituição médica atual, a que me tenho referido, diria
que ela acha sua explicação no infeccionamento do ambiente, devido às
imprudentes e profundas escavações do solo, praticadas em épocas
impróprias pelas Companhias de Gás e de Esgoto, pois creio ser de
simples intuição que um solo como o nosso, abundantemente provido de
matérias orgânicas, só pode ser escavado desde abril até outubro
inclusive o contrário (como se procede atualmente) é arriscar a saúde
pública.\footnote{Arquivo Nacional. MAÇO IS 4-27, Série Saúde --- Higiene
  e Saúde Pública --- Instituto Oswaldo Cruz, sem paginação.}
\end{quote}

Documentos com avaliações semelhantes --- segundo os quais o sistema de
esgotos implementado cooperou na infecção do solo, comprometeu a
salubridade geral, foi incapaz de atenuar a febre amarela etc. --- são
encontrados aos montes nos arquivos das instâncias da Saúde Pública
submetidas à administração da Junta. Um sistema de esgotos é medida
preliminar, mas insuficiente. Quer dizer, mesmo que se faça remover uma,
duas ou três causas que debilitam a saúde pública --- diz Barata Ribeiro
---, como as emanações dos esgotos,

\begin{quote}
dos pântanos, dos monturos, ficarão ainda o ar limitado das habitações,
as evaporações úmidas da terra, que absorvem também produtos mefíticos
das decomposições orgânicas que sobre ela se operam, e a atmosfera
destes extensos vales a que se chama ruas para produzirem efeitos que,
embora diversos dos primeiros em suas manifestações, são-lhe análogos em
seus resultados finais.\footnote{Cândido Barata Ribeiro, \textit{Quais as
  medidas sanitárias\ldots{}}, 1877, p.~89-90}
\end{quote}

Este parágrafo de Barata Ribeiro faz coro com o tratado \textit{Higiene e
saneamento das cidades}, do higienista francês Jean-Baptiste
Fonssagrives, de 1873. É bastante instrutivo como tanto o professor de
Higiene da Faculdade de Medicina de Montpellier quanto Barata Ribeiro
definem \textit{rua} como a ``unidade higiênica da cidade''.\footnote{\textit{Ibidem},
  p.~91; Jean-Baptiste Fonssagrives, \textit{Hygiène et Assainissement des
  villes} (Paris, J.-B. Ballière \& Fils, 1874), p. 96.} Quer dizer,
mede-se o valor de uma cidade do ponto de vista da salubridade, por
aquilo que valem as ruas que constituem a cidade. Ele diz --- aplicando a
metáfora que ecoa na tese de Barata: ``As casas transformam, com efeito,
uma rua em um vale mais ou menos profundo, onde o fundo é representado
pela calçada, os regatos pelos regos de enxurrada e as colinas
adjacentes pelas casas''.\footnote{\textit{Ibidem}, p.~105. Tradução
  nossa.} Fonssagrives foi cirurgião militar e, assim como Boudin, Hallé
e Motard, foi um respeitado cultor da Geografia Médica no curso do \textsc{xix}.
Estudou em obra pioneira --- o \textit{Tratado de Higiene Naval}, de 1856
--- as lições de higiene náutica para militares e marinheiros de navios
mercantes. Teria sido importante em algum momento, para médicos
militares europeus engajados na progressão da empresa colonialista em
países da África e América no \textsc{xix}, inventar tecnologias que facilitassem
a conservação da saúde do homem do mar, a despeito do seu modo
transitório de vida. A política colonialista somada à mundialização do
comércio acirrou o problema de por que ``certas doenças estariam
circunscritas a determinadas regiões do globo, enquanto outras tinham
ali um impacto diferenciado e um padrão de endemicidade
distinto''.\footnote{Flavio C. Edler, \textit{A Medicina no Brasil
  Imperial}, 2011, p.~54.} Crescem em importância fatos relativos à
influência exercida pelos diversos climas a que o marinheiro pode estar
exposto; multiplicam-se estudos das causas e da natureza das doenças que
podem atacá-lo durante as viagens e nos portos em que desembarca.
Autoridades médicas francesas sabiam que, nas regiões tropicais e
coloniais, o verdadeiro inimigo do médico, do marinheiro, do soldado, da
administração colonial era a doença tropical. Processos patogênicos,
morbidades ou epidemias como a malária ou a febre amarela estavam entre
os ``flagelos mais temidos pelos europeus, sendo responsáveis pela
maioria das mortes nos trópicos. Durante muito tempo, os médicos
atribuíram aos trópicos um número significativo de doenças, por
considerá-los locais patogênicos por excelência''.\footnote{Rosa Helena
  de S.G. de Morais, ``A geografia médica e as expedições francesas para
  o Brasil: uma descrição da estação naval do Brasil e da Prata
  (1868-1870)'', \textit{História, Ciências, Saúde --- Manguinhos}, vol.14,
  n.1, Rio de Janeiro, jan.-mar. 2007, p.~52.} Há uma diferença
epistemológica aqui: interessante ver como as práticas mobilizadas por
essa Geografia Médica não se confundem com o tipo de determinismo da
antiga ``topografia médica'' --- com seu conceito operativo
(anteriormente debatido via Sydenham) de ``constituição epidêmica''.
Contra a posição que circunda certas ``regiões naturais'', regiões que
são unidades de elementos como solo e clima (determinantes exclusivas da
patologização do espaço), a Geografia Médica evoca tanto a causalidade
social da epidemia quanto o recurso a uma noção mais rigorosa de
``higiene''. Segundo Edler, ``a noção de higiene, contrapondo-se à ideia
de `região natural', serviria como um antídoto aos fatores morbígenos do
clima''.\footnote{Flavio C. Edler, \textit{A Medicina no Brasil Imperial},
  2011, p.~56.} Isso possibilita a Barata Ribeiro e Fonssagrives
teorizarem --- sob a ótica da corrida civilizatória higienista --- nas
bordas de uma verdadeira sociologia das cidades. Não à toa, por exemplo,
\textit{Higiene e saneamento das cidades},\footnote{Jean-Baptiste
  Fonssagrives, \textit{Hygiène et Assainissement des villes}, 1874.} de
1873, inicia-se com uma tediosa exposição de paralelos entre cidade e
campo, ou com a descrição das funções recíprocas exercidas por cidade e
campo, para daí extrair os diferentes fenômenos sociais, entre os quais
as epidemias, decorrentes ou da vida urbana ou da vida campesina. Como a
divisão entre cidade e campo possui para esses autores um aspecto
funcional de natureza econômica (a cidade é o palco da história, teatro
do monumental e da administração pública, e o campo é a moenda da nação
e o seu celeiro), há tanto a boa quanto a má proporção entre o elemento
urbano e o elemento rural dentro de um país. Pois bem, \textit{Higiene e
saneamento das cidades} discute e avalia o êxodo de campesinos, traça a
salubridade comparativa entre campo e cidade, traz quadros comparativos
entre taxas de mortalidade e natalidade na cidade e no campo, faz entre
os que nascem no campo ou na cidade a comparação entre sua compleição,
força, estatura, inteligência etc., filosofa sobre tendências sociais ao
suicídio, à criminalidade, à loucura. Tudo apenas para repetirmos, como
disse Barata, que por mais que se faça remover as emanações dos esgotos,
os pântanos etc., ficará ainda por refazer ``a atmosfera destes extensos
vales a que se chama ruas para produzirem efeitos que, embora diversos
dos primeiros em suas manifestações, são-lhe análogos em seus resultados
finais''.\footnote{Cândido Barata Ribeiro, \textit{Quais as medidas
  sanitárias\ldots{}}, 1877, p.~90.}

Mas afinal, uma rua, que é uma rua? Melhor: o que é uma boa e uma má
rua? Ou, o que faz uma rua boa, para um governo que se deixará animar
pelo sentimento de responsabilidade com a salubridade pública?
Fonssagrives enumera sete pontos para o estudo da rua: seu comprimento,
largura, profundidade, forma, inclinação, natureza do calçamento e os
acessórios da rua, como as marquises, valas etc.\footnote{Jean-Baptiste
  Fonssagrives, \textit{Hygiène et Assainissement des villes}, 1874,
  p.~97.} Alguns desses aspectos exigem da Higiene considerações de
maior importância que outros. Diz-se, por vezes, que a rua é um vale.
Outras vezes chamam-na órgão vivo ou sistema, não nos fugindo de vista a
predileção por algumas metáforas --- levadas adiante por urbanistas
modernos --- que carregam a tematização da cidade de um sentido próprio:
avenidas são como \textit{artérias}, as casas são como as \textit{células},
as praças arborizadas como os \textit{pulmões} da cidade, os monumentos
como um \textit{rosto} bonito de perfil, afinal, a cidade é como um
\textit{organismo} sóbrio, são, bem ventilado e bem afeiçoado.\footnote{Le
  Corbusier, \textit{Urbanismo} {[}1924{]}, trad. Maria E. Galvão, 2ª
  ed.~(São Paulo, Martins Fontes, 2000), p.~64, 70, 158.} Sem dúvida a
imagem da cidade como \textit{organismo} não era a única, mas bastante
comum. Em 1875, Vieira Souto sai em defesa da abertura de duas novas
avenidas cariocas, a \textit{Vinte e Oito de Setembro}, e uma
segunda\footnote{Projetada por Rebouças e, bem mais tarde, incorporada
  em parte por Passos com o nome de Avenida Beira-Mar.} que contornasse
a praia de Sta. Luzia e seguisse até a praia de Botafogo: ``essas duas
avenidas, \textit{verdadeiros pulmões facultados à nossa cidade}, seriam
ligadas entre si com uma (\ldots{}) que percorreria os terrenos vagos
resultantes do arrasamento dos morros do Castelo e Sto.
Antônio''.\footnote{Vieira Souto. \textit{Melhoramento da cidade do Rio de
  Janeiro}, 1875, p.~35.} Semelhantemente, em 1878, a Junta é consultada
pelo Governo para colaborar com os engenheiros da ``Comissão de
Melhoramentos da Cidade do Rio de Janeiro''.\footnote{1875 é o ano em
  que a cidade ganha seu primeiro plano urbanístico em acepção moderna.
  Trata-se do plano elaborado pela Comissão de Melhoramentos da Cidade
  do Rio de Janeiro, nomeada pelo Imperador em 27/05/1874, e formada por
  três jovens engenheiros, entre eles Francisco Pereira Passos. ``As
  propostas expostas nos relatórios dessa Comissão atendem a três
  problemas principais: o saneamento, a circulação e a valorização de
  novas áreas de expansão, melhorando suas condições higiênicas e
  `dotando\dots{} de mais beleza e harmonia' as suas construções.'' Maria
  Pace Chiavari, ``As transformações urbanas do século \textsc{xix}'', em Giovana
  R. Del Brenna (org.), \textit{Rio de Janeiro de Pereira Passos: uma
  cidade em questão II} (Rio de Janeiro, Index, 1985), p.~587. O
  Primeiro Relatório da Comissão, entregue em janeiro de 1875,
  compreende o vetor norte de expansão da cidade, ``do Campo da
  Aclamação até a raiz da Serra do Andaraí, compreendendo os bairros
  Cidade Nova, Andaraí, Engenho Velho, São Cristóvão e Rio Comprido; do
  Segundo Relatório, de 28/02/1876, referente à cidade velha e à zona
  sul, compreendendo os bairros da Glória, Catete, Botafogo e
  Laranjeiras (\dots{}).'' Sonia Gomes Pereira, \textit{A Reforma Urbana de
  Pereira Passos}\ldots{}, 1998, p. 130. O projeto urbanístico de 1875
  não viria a ser executado, mas sua importância esteve em suscitar pela
  primeira vez de forma oficial e declarada um estudo para a
  reurbanização da Corte. Três décadas mais tarde, na prefeitura de
  Pereira Passos, algumas diretrizes conceituais da Comissão comporão a
  estrutura programática do \textit{Plano de Melhoramentos da Cidade}
  (executado entre 1903 e 1906). Fisicamente, duas coisas em comum: um
  programa de obras para formação da frente marítima setentrional da
  cidade; depois, alargamento, abertura e retificação de ruas
  (combinados a técnicas de alinhamento dos quarteirões preparando a
  estrutura de expansão da cidade), que definiriam a ventilação e a boa
  circulação (de pessoas, bens e meios de transporte) com vistas ao
  favorecimento da salubridade pública. Portanto, uma diretriz teórica
  comum, se comparamos o projeto de 1875 e a cidade de Pereira Passos. A
  fundamentação da necessidade das obras segue aqui três eixos:
  saneamento, circulação (ventilação e policiamento) e o embelezamento
  da capital.} Pede-se que dê conta da Comissão discriminando as medidas
urgentes indispensáveis ao melhoramento do estado sanitário. O
Presidente da Junta responde com uma lista de tópicos, todos os pontos
sendo devidamente contemplados nos relatórios da Comissão: a fixação de
normas reguladoras da edificação de cortiços e estalagens no centro da
cidade; a proposta do aterro de pântanos remanescentes nos arrabaldes e
um esquema de drenagem das correntes das bacias do Canal do Mangue; um
programa de obras para formação de um cais geral na área setentrional da
cidade; e, afinal, o que por ora nos interessa, a fixação de uma
estrutura urbana usando a técnica de alinhamento dos quarteirões pela
abertura e alargamento de algumas ruas estreitas e insalubres. Urge
prolongar, na voz do Barão de Lavradio, ``algumas ruas sem inutilizar o
plano dos melhoramentos gerais estabelecidos pela Comissão dele
encarregado, a fim de abrir desde já \textit{grandes artérias à ventilação
e arejamento da Cidade}''.\footnote{BR RJAGCRJ 8.4.22 Fundo Câmara
  Municipal --- Série Higiene Pública, p.~25-30.} Diz-se, a rua é o
sistema respiratório das cidades, avalia-se a saúde de uma cidade pela
condição de suas ruas. Só que nem sempre as cidades foram corpos vivos
nem as artérias funcionaram como imagem para ruas e avenidas.

Há outra imagem de cidade, com ruas que não nascem do argumento de
favorecer a ``boa circulação''. Porque a ``boa'' circulação evocada na
fundamentação da necessidade de se alargar ruas insalubres não é apenas
circulação de bens regidos pelo interesse comercial. Boa circulação,
como percebemos, quer dizer em primeiro lugar favorecer a ventilação
para o arejamento de elementos insalubres. E talvez o mais importante:
boa circulação significa alargamento de grandes colmeias sociais e
policiamento da má circulação de pessoas. Quer dizer, segundo a régua
moral da cidade moderna, há bons pedestres e outros não tão bons.

Um traçado viário como um tabuleiro de xadrez (ou o modelo urbanístico
do acampamento militar) prioriza um tipo intolerante e agressivo de
luminosidade. Essa organização de cidade pensa circulação no sentido do
policiamento da boa e da má circulação.~Essa rua --- que por conveniência
chamamos ``moderna'' ---, figurada na imagem dos vasos de um sistema
circulatório, é o critério que caracteriza uma dada cidade como um
organismo malsão e ``antiestético'', e outra cidade qualquer como um
organismo sóbrio e salubre.

Dizíamos que há essa outra imagem de cidade --- essa cidade é a cidade
colonial dos portugueses, a cidade ``sem método'' de Sérgio Buarque,
cidade que ``não é produto mental''.\footnote{Sérgio Buarque de Holanda,
  \textit{Raízes do Brasil} {[}1936{]}, 1995, p. 110.} A cidade
portuguesa, bem entendido, não é um organismo. Nela, a irregularidade da
planta somou-se ao acidentado do terreno para valorizar aspectos e modos
de vida recusados pela medicina higienista. Há um higienista da segunda
parte do século \textsc{xix}, do qual falaremos adiante, chamado Adrien Proust.
Proust compila em uma imagem essa tecnologia urbana. Importa atentar
sobremaneira para a relação que se faz entre sinuosidade, insalubridade
e --- novamente --- Oriente. Foi extremamente comum, há pouco mais de 100
anos, a comparação entre a Cidade Velha, a cidade portuguesa, e o
``aspecto repugnante de certas cidades do Oriente''\footnote{\textit{Mensagem
  do Prefeito do Distrito Federal lida na sessão do Conselho Municipal
  de 1º de Setembro de 1903} (Rio de Janeiro, Typographia da Gazeta de
  Notícias, 1903), p.~7-8.} (isso é Pereira Passos em um discurso
famoso). Defendia-se que o aspecto da Corte Imperial não fosse muito
alheio a essa lógica. Diz Proust:

\begin{quote}
Na maior parte das cidades do Oriente, os corredores estreitos,
tortuosos, irregulares e sujos separam casas, que não dão acesso à rua
senão por uma pequena porta, e cujas janelas dão para um pátio interno.
Esse estado de coisas, ligado em parte ao clima, em parte aos costumes e
ao estado de civilização, não difere sensivelmente daquele que
apresentam na Idade Média as mais importantes cidades da Europa. As
casas irregularmente construídas muitas vezes dispostas de maneira a
comprometer, nos andares superiores, o pouco de ar e de luz (\ldots{}),
uma calçada não pavimentada, imundícies em toda parte. Esta é a
fisionomia da maior parte das cidades, mesmo as mais célebres de alguns
séculos atrás, e os vestígios deste estado de coisas nos rodeiam ainda
hoje.\footnote{Adrien Proust, \textit{Traité d'Hygiène} {[}1877{]}, 2ª
  ed.~(Paris, G. Masson Éditeur, 1881), p.~632. Tradução nossa.}
\end{quote}

À Cidade Velha contrapõe-se a Cidade-Organismo. A Cidade Velha carioca é
suja e perigosa, porque suas ruas são estreitas e irregulares, mas já
insistimos suficientemente no nascimento desse nexo de causalidade. A
fisionomia das ruas da Cidade Velha demonstra que não foi a rua que
materializou o caminho do pedestre, mas o contrário, foi a memória dos
pedestres que materializou a memória dos paralelepípedos. Ademais, na
Cidade Velha a rua é a extensão da casa. Na Cidade-Organismo a casa é
uma célula, a rua é uma artéria ventilada, a cidade é um organismo
sóbrio, são e bem afeiçoado.\footnote{Como nasce a rua ocidental (para
  não dizermos ``cidade moderna'')? Soa paradoxal, mas é João do Rio
  quem sugere que o nascimento da rua dos arrabaldes segue o mesmo
  roteiro dos projetos que rasgam quarteirões coloniais para a largada
  de uma nova corrida civilizatória. O que comanda o traçado da cidade
  ocidental é a abertura e o consequente policiamento da rua: ``A
  princípio capim, um braço a ligar duas artérias. Percorre-o sem pensar
  meia dúzia de criaturas. Um dia cercam à beira um lote de terreno.
  Surgem em seguida os alicerces de uma casa. Depois de outra e mais
  outra. Um combustor tremeluz indicando que ela já se não deita com as
  primeiras sombras. Três ou quatro habitantes proclamam a sua
  salubridade ou o seu sossego. Os vendedores ambulantes entram por ali
  como por terreno novo a conquistar. Aparece a primeira reclamação nos
  jornais contra a lama ou o capim. É o batismo. As notas policiais
  contam que os gatunos deram num dos seus quintais. É a estreia na
  celebridade, que exige o calçamento ou o prolongamento da linha de
  bondes''.~João do Rio, \textit{A alma encantadora das ruas}
  {[}1904-07{]}, 2008, p.~34.} É com esse espírito que os higienistas
europeus do \textsc{xix} definirão a ``boa rua''. A boa rua pertence à
Cidade-Organismo, é a própria cidade do Ocidente.

Fonssagrives argumenta que as casas transformam a rua em um vale, cujo
leito é o pavimento, as cheias são os riachos, e as colinas adjacentes
são as casas. A ideia é que, na medida em que os vales são lugares
notoriamente pantanosos e insalubres, também o são as ruas caso não se
estabeleça uma proporção conveniente entre a altura das casas e a
largura das ruas. ``É incontestável que ao Norte ruas espaçosas são
indispensáveis à salubridade de uma cidade, seu grau de abertura
favorece a evaporação, a ventilação, e permitem à luz --- onde o clima é
pouco generoso --- penetrar até o fundo dos corredores e das
casas''.\footnote{Adrien Proust, \textit{Traité d'Hygiène} {[}1877{]},
  1881, p.~633.} Embora condições diferentes pareçam ser sensivelmente
preferíveis nas cidades do Oriente, é provável que com o desenvolvimento
da civilização se tenda progressivamente a alargar o diâmetro das vias.
Há, não obstante, regras gerais aplicáveis a todas as geografias, leis
universais relativas não à largura definitiva das ruas, mas às
proporções. Elas prescrevem, em função da proporção entre altura das
casas e largura das ruas, uma ventilação conveniente, uma quantidade
suficiente de luz; e distribuem o calor, pela ``comparação entre os dias
de chuva, de seca, de calor e de frio com a natureza do vento que os
acompanha; a consideração dos sítios geográficos vizinhos; (\ldots{})
para determinar de uma maneira racional o mais conveniente a se fazer no
caso de uma cidade ou de uma habitação''.\footnote{Adolphe Motard,
  \textit{Traité d'Hygiène Générale} \textit{--- Tome Premier}, 1868, p.~583.
  Tradução nossa.} Há um copertencimento entre a qualidade da habitação
e a largura conveniente das ruas, e essas coisas irão se anunciar mais
fortemente no Brasil no contexto das epidemias da década de 1870.
Caminham efetivamente juntos o aperfeiçoamento da malha viária e a
substituição do casario antiquado e insalubre por unidades prediais
higiênicas. O plano dos engenheiros, da Comissão de Melhoramentos da
Cidade do Rio de Janeiro, para uma reforma da cidade (que depois de 30
anos ganharia outra forma, mas sairia do papel obedecendo à mesma
mentalidade), ficou pronto em 12 de janeiro de 1875. Diz o relator que
cumpre

\begin{quote}
designar a largura das calçadas e passeios laterais nas novas ruas e
praças, e a altura das arcadas ou pórticos contínuos no caso de haver
vantagem em cobrir os passeios com estas construções, indicar quais ruas
que devem ser desde já abertas ou alargadas e retificadas, e
aquelas cujo alargamento e retificação devem ser feitos à medida
que se reedificam os prédios existentes, a fim de que tais reedificações
fiquem subordinadas aos novos alinhamentos adotados; propor, finalmente,
todos os melhoramentos que possam interessar à salubridade pública
(\dots{}).\footnote{Cf. Jaime Larry Benchimol, \textit{Pereira Passos: um
  Haussmann tropical}, 1992, p.~140.}
\end{quote}

O alargamento de algumas ruas e a tarefa de rasgar alguns quarteirões
para aberturas de novas ruas, dispostos a permitir a fácil circulação de
ar e facilitar o escoamento da cidade, concorrem para o saneamento em
geral à medida que realizam simultaneamente o desmonte das habitações
estreitas e insalubres. Motard indica de que maneira a Inglaterra,
perseguindo em suas cidades a reforma, seguiu o exemplo que lhe deu a
administração parisiense, que, por trabalhos incessantes de salubridade,
diminuiu as taxas de mortalidade decorrentes das epidemias coléricas.
``Após o alargamento das ruas e a elevação de belas fachadas, o interior
das habitações deverá ser escrupulosamente observado; a reforma das
habitações insalubres é uma das mais difíceis e uma das mais importantes
reformas às quais a higiene pública deve se propor''.\footnote{Adolphe
  Motard, \textit{Traité d'Hygiène Générale} \textit{--- Tome Premier}, 1868,
  p.~574.} Quer dizer, a existência ou não do casario antiquado e dos
quarteirões anti-higiênicos estão submetidas aos novos alinhamentos
adotados nos melhoramentos que a cidade aspira. Ou seja, não se deve nem
se pode influir irrestritamente sobre a propriedade das casas. A aurora
do liberalismo reconhece na propriedade privada uma sorte de cláusula
pétrea irremovível. Oficialmente, a administração pública não irá
exercer poder irrestrito sobre os modos de habitar da população, ainda
que se saiba ser a higiene privada um dos elementos essenciais da
salubridade da cidade. Pode-se, entretanto, ainda que os conselhos da
medicina se choquem com vontades refracionárias, influir sobre a
propriedade da cidade.

Enquanto a ``propriedade corporal'' e a ``propriedade das casas''
dependem da possibilidade de as vontades se dobrarem diante de um
conselho, ``a propriedade da cidade é diferente: ela se impõe e se
executa pela autoridade (\dots{}). A propriedade é o eixo da higiene urbana,
tal como é da higiene pessoal, e é preciso persuadirmo-nos de que não há
uma única violação destas prescrições que resta impune''.\footnote{Jean-Baptiste
  Fonssagrives, \textit{Hygiène et Assainissement des villes}, 1874,
  p.~143. Tradução nossa.}

O comprimento das ruas é uma condição que seria de pouca importância
para a Higiene, fossem elas cortadas, de distância a distância, por
praças ou quadras, ou por transversais que fornecessem meios de
ventilação e circulação mais fáceis. O mesmo não ocorre com sua largura.
As casas, nas antigas cidades, bordejavam ruas geralmente estreitas.
``As ruas da Pompeia tinham às vezes nada mais que 4 metros de largura,
as mais largas não tinham mais de 7 metros, compreendendo as
calçadas''.\footnote{\textit{Ibidem}, p.~98.} Via de regra, a determinação
da largura que convém dar às ruas está subordinada a duas condições
essenciais:

\begin{quote}
1º o clima; 2º a altura média das casas. Eu dizia há pouco que os climas
extremos têm, deste ponto de vista, necessidades opostas, e que uma
largura de rua que seria plenamente suficiente para uma cidade do Sul,
inundada de luz, calor e poeira, e podendo sofrer mais a seca do que a
humidade, não ofereceria condições de insalubridade para uma cidade como
a Normandia ou a Bretanha. No Norte, é necessário que tudo esteja
disposto para sustentar a penúria do sol e facilitar a evaporação da
humidade; no Sul, abrigar-se na sombra é tanto uma necessidade de
bem-estar quanto de saúde. É necessário, portanto, no primeiro caso,
ruas mais espaçosas.\footnote{\textit{Ibidem}, 1874, p.~99-100.}
\end{quote}

Fonssagrives estima que as ruas da cidade do Norte devam ter um mínimo
de 12 metros de largura, e nas cidades do Sul uma média de 10 metros.
Uma rua de 10 metros já ofereceria em cidades de menor população,
segundo o higienista, as facilidades suficientes, por exemplo, para a
circulação dos carros. Barata Ribeiro, em 1877, é da mesma opinião:
``nos países quentes'', ele diz, ``a largura de 12 metros é suficiente;
além deste termo a ação do sol é muito enérgica, e as nuvens de pó
levantadas pelos ventos que obram sobre uma superfície muito larga,
tornam-se, além de incômodas, maléficas''.\footnote{Cândido Barata
  Ribeiro, \textit{Quais as medidas sanitárias\ldots{}}, 1877, p.~91-2.}
Os engenheiros do Plano urbanístico de 1875-1876 serão não menos
ousados. O segundo relatório da Comissão de Melhoramentos, com as
propostas centradas nas operações sobre a Cidade Velha, estipula um
limite inferior que fixa em 13,2m a largura das ruas, ``ainda que
desenhe alinhamentos com 10m de largura, para `evitar grandes despesas
em desproporção com a importância das mesmas ruas'\,''.\footnote{Verena
  Andreatta, \textit{Cidades quadradas, paraísos circulares}, 2006,
  p.~164-5.} Não obstante, há destaque para a proposta de abertura de
uma nova rua, em substituição à Sete de Setembro, que teria novos 18m de
largura, ligando a Praça da Constituição à Praça D. Pedro II. Como não
poderia deixar de ser, o Plano da Comissão não se ausentou do debate
sobre determinação de normas para a altura dos edifícios em função da
largura das ruas. Fixou edifícios de 12m, 16m e 20m para ruas que
tenham, respectivamente, menos de 7m, 7-10m e mais de 10m de largura. E
estabeleceu ``limites sobre recuos da edificação, e também tamanhos
máximos dos elementos salientes das fachadas e composição de altura nas
esquinas''.\footnote{\textit{Ibidem}, p.~162.}

Quanto à orientação de uma rua, decorrência da geometria dos quarteirões
e sua densidade populacional, Fonssagrives, como os demais higienistas
de seu tempo, justifica sua regulamentação não só em função da direção
dos ventos dominantes, mas da salubridade ou insalubridade dos lugares
que o vento percorre. J.-N. Hallé, considerado por alguns historiadores
o pai da Higiene Pública, e A. Motard desenvolveram, cada um à sua
maneira, o problema das mudanças na composição normal do ar atmosférico
nos contextos de aglomeração urbana.

A descoberta da identidade dos resultados da combustão e da respiração,
as mudanças que o ar experimenta nos pulmões e na superfície da pele, as
qualidades novas que o sangue recebe ao passar pelos vasos pulmonares,
apresentam sobre um novo ponto de vista as relações do homem com o ar
que ele respira e com a atmosfera que o rodeia. Se, então, o ar do
entorno experimenta mudanças, também experimentarão mudanças os nossos
órgãos e também receberão efeitos as funções pulmonares. Portanto, que
conheçamos ``suficientemente bem os efeitos do fluido atmosférico, em
todas as partes nas quais ele entra em relação com a matéria nutritiva;
no estômago e nos intestinos, com a massa alimentar, ou com o alimento
nutritivo''.\footnote{Étienne Tourtelle e Jean-Noël Hallé, \textit{Traité
  d'Hygiène} (Paris, chez M. Gautret, 1838), p.~82. Tradução nossa.}
Motard, escrevendo em 1868, 30 anos após Hallé, agrega também problemas
de outra ordem. Todos os princípios estranhos lançados na atmosfera ou
nos permitem observar a putrefação da sua composição natural --- dessa
forma envenenando o sangue ---, ou porventura ocasionam o desarranjo na
proporção entre oxigênio e o gás carbônico. Sabe-se, diz Motard, após as
experiências dos fisiologistas com a respiração animal, que um mesmo ar
respirado certo número de vezes altera-se ``até conter de 8\% a 9\% de
ácido carbônico, e que nestes casos ele não sofre mais alterações no
pulmão, ou seja, ele asfixia. Este efeito se produz de maneira
fulminante quando o homem está imerso numa atmosfera
irrespirável''.\footnote{Adolphe Motard, \textit{Traité d'Hygiène Générale
  --- Tome Premier}, 1868, p.~543.} As seguintes consequências
resultarão, portanto, da permanência prolongada da aglomeração de um
grande número de pessoas em um mesmo espaço mal ventilado: ``a elevação
da temperatura, a umidade estagnada, a produção de ácido carbônico, a
acumulação dos produtos das exalações pulmonar, cutânea ou mórbida, a
degeneração mais ou menos pútrida destes''.\footnote{\textit{Ibidem,}
  p.~557.} Tais emanações suscetíveis de viciar o ar agem sobre os
órgãos dos homens nos anfiteatros fechados, nas alcovas, nos cemitérios,
e, naturalmente, nas ruas estreitas e sem orientação regular. Daí
decorre a caução científica para a introdução dos imperativos
higienistas nos debates urbanísticos, já que afinal, segundo Hallé, ``é
principalmente à arte de construir os edifícios, de dispor o espaço
público, e promover uma livre circulação de ar, que se deve em parte a
categoria de `grande cidade'\,''.\footnote{Étienne Tourtelle e Jean-Noël
  Hallé, \textit{Traité d'Hygiène}, 1838, p.~36.} Para além de uma má
proporção do ar atmosférico ocasionada por sua falta de renovação,
existem os germes, os germes da vida que estão por toda parte: em cada
gota d'água, em cada balão de ar. ``Em todos os lugares, se me permitem
esta expressão que define meu pensamento, \textit{a vida se alimenta da
morte}. É de se admirar que a presença destes germes que preenchem a
atmosfera tenham há tanto tempo chamado a atenção dos
higienistas''.\footnote{Adolphe Motard, \textit{Traité d'Hygiène Générale
  --- Tome Premier}, 1868, p.~246.} A questão da existência, no ar
atmosférico, de numerosos germes capazes de dar origem às fermentações
alcoólica, acética, láctica, pútrida, e de desenvolver entre os vegetais
e os animais uma variedade de produções parasitárias, teve, portanto,
desde esse momento uma importância particular.

\begin{quote}
Os trabalhos de Pasteur lançaram sobre esta questão um grande
esclarecimento. Esse cientista conseguiu filtrar um volume considerável
de ar em uma rolha de algodão, substância facilmente solúvel em éter
etílico, a fim de isolar amostras de poeira mecanicamente retidas no
algodão. Pasteur encontrou através desta experiência um grande número de
corpúsculos arredondados, que (\dots{}) povoam prontamente os seres viventes
ou vegetais. Temos razão em considerar estes corpúsculos germes
organizados.\footnote{\textit{Ibidem}, p.~250.}
\end{quote}

Sendo assim, as influências prejudiciais da má circulação do ar são de
dois tipos: primeiramente, podem resultar da decomposição lenta de
produtos orgânicos que apodrecem nas ruas estreitas, habitações e
espaços fechados, possibilitando a existência e proliferação de um
grande número de ``corpúsculos arredondados'', tais como confirma
Pasteur. Germes organizados que se acumulam nas grandes aglomerações
urbanas, viciando o ar e cobrindo os solos. Tais como todas as bestas
que, onde é que existam, se alimentam da vida, as inumeráveis legiões de
parasitas e de animálculos se aproveitam para fazer do organismo humano
sua presa e desencadear epidemias. O segundo tipo, capaz de desvirtuar a
composição normal do ar atmosférico, é efeito exclusivo da não renovação
do ar em espaços confinados. Sejam esses espaços ruas, alcovas ou
anfiteatros, o excesso de ácido carbônico expelido pelos pulmões e os
odores pútridos que se desprendem da pele agem sobre os órgãos dos seres
viventes, gerando asfixia.

Logo, a orientação de uma rua, como pensa Fonssagrives, influi bastante
sobre sua salubridade e sobre o bem-estar das casas que margeiam. É
necessário apreciar o valor da orientação de uma rua, tomando em
consideração ``a proteção que ela fornece contra as emanações insalubres
que os ventos conduzem de passagem''.\footnote{Jean-Baptiste
  Fonssagrives, \textit{Hygiène et Assainissement des villes}, 1874,
  p.~107.} Interrogando-se pelas condições dos edifícios públicos, ruas
e casas, da Corte Imperial em 1877, Barata Ribeiro destaca que o ar
limitado, assim como o grande elemento de combustão que existe nessa
cidade populosa, como também o aparecimento anual de moléstias
epidêmicas graves, tudo isso decorre, entre outras causas, da má
construção da cidade. Tudo encontra explicação na limitação do acesso a
dois grandes elementos da saúde e da vida: o ar e a luz. Somos aqui
levados a reproduzir integralmente uma crônica de Barata Ribeiro, depois
de perambular sorrateiramente pela freguesia do Sacramento, tanto pela
representação babélica da vida das ruas tortuosas, longas e esguias,
quanto pela maneira como a espacialização das ruas se articula com as
manifestações tipológicas do antiestético e anti-higiênico.

\begin{quote}
Quem de certas horas da tarde por diante, e principalmente à noite,
transitar pelas ruas que na cidade se estendem do campo do Sant'Anna
para baixo, e a que chamaremos a parte comercial da cidade,
principalmente em algumas regiões deste território em que mais se
concentra a vida, não pode deixar de sentir com repugnância as
impressões desagradáveis da atmosfera que aí respira. Parece que as
superfícies das calçadas, dos lajedos e dos edifícios, e até desta praga
de quiosques que nos vieram roubar vida, deslocando do ar um volume
igual ao que ocupam, expira com toda a energia as impurezas enormes que
durante o dia derramou-lhe na atmosfera baixa, pesada e estagnada a
multidão de animais e de coisas que por aí se agitou, e procura sorver a
longos tragos o ar mais puro que lhe traga a noite. (\ldots{})
Entretanto, nem a estas horas aí se pode viver; as ruas tortuosas,
esguias e longas, marginadas por altos edifícios, embaraçam a ação das
brisas suaves que sopravam outrora em nossa cidade e as casas, com uma
lotação que não comporta a área estreita que abrangem, atiram pelas suas
aberturas lufadas de um bafo quente e pestífero, rico dos produtos
orgânicos de exalação pulmonar e cutânea, e dos de fermentação que nelas
se opera pelo acúmulo de substâncias de todo o gênero e só pobres, ou
antes paupérrimas de oxigênio e portanto de vida.\footnote{Cândido
  Barata Ribeiro, \textit{Quais as medidas sanitárias\ldots{}}, 1877,
  p.~88-89.}
\end{quote}

O Dr.~Barata Ribeiro, Lente substituto na seção de Ciências Médicas da
Faculdade de Medicina do Rio de Janeiro na década de 1870, foi junto com
Paula Candido e Pereira Rego um nome dos mais influentes higienistas
brasileiros do \textsc{xix}. 1892 é uma data importante na sua biografia,
requerendo breve menção. É um ano forte em termos simbólicos, se nos
propomos a medir como o dispositivo médico-higienista atua na esfera das
instituições com respaldo do executivo: ano em que ele, o higienista
Barata Ribeiro, se torna prefeito do Distrito Federal. Mas seria um erro
apresentar 1892 como a estaca zero a partir do qual as demolições de
cortiços se tornam política de Estado. Será apenas ocasião para que
conflitos antigos entre autoridades, pela decisão de desalojar, impedir
construção ou demolir cortiços, rendam um considerável material para
jornais de alcance, como a \textit{Gazeta de Notícias} e o \textit{Jornal do
Brazil}. (Evoco especialmente as marretadas que colocam abaixo a célebre
estalagem da ``Cabeça de Porco'', resultado de uma verdadeira operação
militar em 1893. Uma série de reportagens põe sob holofotes, na antiga
região de Sta. Rita, um arsenal de cavalaria policial, operários,
oficiais do exército, políticos e engenheiros municipais, higienistas e
um sem número de anônimos com seus pertences despejados nas calçadas.) A
política de repressão dos cortiços é, como ainda iremos assinalar, um
acontecimento com precedentes que datam de 20 anos antes desse episódio.
O cortiço se torna questão de saúde pública após as epidemias da década
de 1870, ou seja, o episódio do Cabeça de Porco é efeito tardio, de uma
tarefa política em curso.

O Dr.~Cândido Barata Ribeiro se torna Lente Substituto da Faculdade de
Medicina do Rio de Janeiro com uma tese apresentada em 1877, \textit{Quais
as medidas sanitárias que devem ser aconselhadas para impedir o
desenvolvimento e propagação da febre amarela na cidade do Rio de
Janeiro?} É um pouco o imaginário de como formatar outra sociedade,
sonho político da sociedade sem cortiços e, portanto, efetivamente
alvejada da desordem, do desviante, dos perigos da cidade portuguesa. A
ideia fixa de Barata Ribeiro --- ideia que o higienista hiberna durante
os anos anteriores, até a aportunidade surgida pela indicação de
Floriano Peixoto para ocupar o cargo de prefeito --- é a utopia da
sociedade higienizada, sociedade que depende da ``epidemização'' da
pobreza para se consolidar. A rigor, a tese de 1877 faz paráfrase das
ideias do \textit{Tratado de Higiene} de Achille Proust, contemporâneo de
Barata Ribeiro, membro da Academia de Medicina francesa e acadêmico na
Faculdade de Medicina de Paris. Com Proust, via Barata Ribeiro, a
vanguarda higienista no Brasil passa de uma modesta ciência da
profilaxia envolvida em ``princípios puramente preventivos e
profiláticos'',\footnote{Adrien Proust, \textit{Traité d'Hygiène}
  {[}1877{]}, 1881, p.~2.} e se projeta como ``ciência verdadeiramente
sociológica, cujos princípios aplicáveis ao indivíduo estudam-no também
nas suas múltiplas relações sociais''.\footnote{Cândido Barata Ribeiro,
  \textit{Quais as medidas sanitárias\ldots{}}, 1877, p.~54.} A Higiene
não apenas atua com vistas a prevenir contra a doença --- sua ambição
agora guia ao ``melhoramento da \textit{espécie humana}, ao
desenvolvimento do seu bem-estar físico e moral'',\footnote{Adrien
  Proust, \textit{Traité d'Hygiène} {[}1877{]}, 1881, p.~2.} ou como diz
Barata: a Higiene reclama para si ``como último \textit{desideratum} a
perfectibilidade humana''.\footnote{Cândido Barata Ribeiro, \textit{Quais
  as medidas sanitárias\ldots{}}, 1877, p.~54.}

Sim, o homem está cercado de perigos. Sua frágil existência é ameaçada
por mil flagelos destrutivos, mas não são os flagelos do meio natural
circundante que sozinhos mortificam o corpo. Há qualquer coisa de
funesto nos amontoamentos de homens, chamados cidades, que fazem da
sociedade o asilo do crime e da imoralidade. Impraticável devolver o
homem à dinâmica da sua primitiva condição. É sobre confiar à espécie
uma nova tábua de leis naturais no sentido de melhorar a humanidade do
homem por vir. A medicina tem na universalidade sociológica colaboração
suficiente para desovar uma espécie de humanidade higienizada, espécie
protegida de vícios arraigados que mortificaram tanto a ``saúde
individual'' quanto o ``bem-estar da espécie''.\footnote{Étienne
  Tourtelle e Jean-Noël Hallé, \textit{Traité d'Hygiène}, 1838, p.~VI.
  Tradução nossa.} Mas encaminhar o bem-estar da espécie não negligencia
haver algo de específico na saúde do corpo individual (há homens de
compleição mais apta à febre amarela, como os europeus do norte, não
aclimatados; há epidemias de cólera que sacrificam em maior número
africanos etc.).

Por ora, a Higiene leva a cabo o bem-viver ou bem-estar da espécie na
consideração do que jaz de universal, idêntico e generalizante na
humanidade: homens enquanto espécie. Espécie pensada como população,
como sociedade civil desde as suas variáveis existenciárias: suas
relações de produção e o homem como produto do trabalho, suas formas de
amar e a natalidade, suas formas de morrer e a mortalidade, suas formas
de circular e a rua, suas formas de habitar e o lar. Isto é, embora o
brasileiro ou o europeu se adapte às condições materiais e sociais
disponíveis para construir um teto, ainda que em cada sociedade a
habitação dependa distintamente das condições de cada clima em
particular, existem leis universais,

\begin{quote}
regras gerais aplicáveis a todas as construções e que as prescrevem para
estabelecer uma ventilação conveniente, observar certas regras relativas
às proporções, fornecer uma quantidade suficiente de luz, distribuir
igualmente o calor, e estes são os pontos que em um tratado de higiene
geral devem tomar a atenção do médico.\footnote{Adrien Proust,
  \textit{Traité d'Hygiène} {[}1877{]}, 1881, p.~550.}
\end{quote}

Reinventar a trama viária da cidade e combater a insalubridade das casas
fazem parte de um mesmo projeto. Mas a reforma da parte antiga da
cidade, a correção da estreiteza e sinuosidade das ruas, foi objeto de
um sem número de projetos que aguardarão até o entresséculos para se
materializarem. O combate aos cortiços é mais antigo. E é mais ardiloso:
envolveu o emprego de uma dinâmica de forças permanente sobre um
registro setorizado da população, exigiu uma tecnologia de poder mais
militarizada. Ambos solicitaram o debate sobre a salubridade pública,
ambos foram atravessados por uma nova forma de \textit{experienciar} a
epidemia, forma esta que não tinha a ver com instituir quarentenas,
sitiar uma cidade, lavar as ruas, contar os corpos, recolher nos
lazaretos, dividir a cidade em quarteirões onde se estabeleça em cada o
poder de um intendente etc. O que caracteriza essa nova forma de
\textit{experienciar} a epidemia?

Durante a primeira epidemia de febre, em 1850, tratou-se de frear a
degeneração da atmosfera por meio de normas provisórias de fiscalização,
policiamento e vigilância, para que o estado sanitário da cidade
recuasse à sua condição anterior, para que se suspendesse esse acaso
infeliz e passageiro que era uma epidemia. Algumas décadas depois,
durante as epidemias da década de 1870, a Junta escreve ao Ministro dos
Negócios reclamando providências complementares que atenuem a atual
ordem de coisas relativas ao estado higiênico da cidade. As principais
medidas emergenciais consistem na irrigação das ruas, na retirada dos
estrangeiros recentes do centro da cidade, no cuidado para que se evite
a abertura de esgotos durante o sol ardente etc. Todos esses
procedimentos poderão auxiliar contra os efeitos da epidemia,
``\textit{enfraquecendo os elementos de infecção já em grande escala
acumulados na atmosfera que nos circunda}''. Entretanto, ``\textit{a
adoção destas medidas não evitará por certo o rompimento de qualquer
epidemia grave, para cujo desenvolvimento há causas acumuladas desde
muito tempo}''.\footnote{Arquivo Nacional. MAÇO IS 4-27, Série Saúde
  --- Higiene e Saúde Pública --- Instituto Oswaldo Cruz, sem paginação.}
Há algum tempo Pereira Rego já fazia sentir esse raciocínio nos
relatórios anuais da Junta. É uma mudança na função e experiência da
epidemia, que desencadeará estratégias de poder que respondem
especialmente a um novo tipo de urgência.

\begin{enumerate}
\def\labelenumi{\arabic{enumi}.}
\setcounter{enumi}{3}
\tightlist
\item
  2.2 Cortiços: a epidemização da miséria na cidade fronteiriça
\end{enumerate}

Dispositivos de poder se distinguem entre si pela ação que visam
empregar ou produzir. Existem diagramas ou esquemas de poder que através
de seus métodos expressam tais ou quais fins, quer dizer, há diagramas
pensados para esse ou aquele alvo, para esse ou aquele recorte de
realidade. Em 1850 a tecnologia de vigilância e controle extensivos
colocaram em funcionamento o estado de quarentena: enquadramentos e
distribuição dos \textit{vivos} nos cômodos mais ventilados das
residências ou nas partes altas da cidade; localização e custódia do
\textit{amarelento} em lazaretos e enfermarias; individualização e exceção
do \textit{cadáver} e eliminação do seu caráter patológico.

Em 1876 o desenvolvimento da epidemia virá ligado a causas locais
acumuladas desde há muito. A febre amarela aparecerá ligada a motivos
permanentes cuja atividade é exercida de modo contínuo, o que exigirá
uma profunda cirurgia urbana, significando tanto a reurbanização da
Cidade Velha quanto o fechamento de habitações anti-higiênicas. Trata-se
então de duas coisas bastante destacadas dos cordões sanitários dos
verões anteriores.

Em primeiro lugar, é a cidade que aparece como um campo de possível
intervenção, é o público, são as ruas, as praças, os edifícios públicos.
Em vez de ``atingir os indivíduos como um conjunto de sujeitos de
direito capazes de ações voluntárias'', diz Foucault, ``em vez de
atingi-los como uma multiplicidade de organismos, de corpos capazes de
desempenhos'', vai-se procurar atingir ``uma multiplicidade de
indivíduos que são e que só existem profunda, essencial, biologicamente
ligados à materialidade dentro da qual existem''.\footnote{Michel
  Foucault, \textit{Segurança, território, população} {[}1977-78/2004{]},
  2008, p.~28\textit{.}} O que se procura atingir, o recorte de realidade
no qual intervirá o poder, não é a individualidade de vivos, doentes ou
mortos em sua isolada imediaticidade. Na medida em que essa realidade
funciona como um nó onde séries de acontecimentos são produzidos por
esses grupos e populações, o campo de intervenção do poder são as
circunstâncias ou a realidade urbana. Trata-se assim de gerenciar a vida
da população em uma certa horizontalidade, em um todo, otimizando a
circulação através de uma transformação urbanística e normalizando o
circuito dos acontecimentos urbanos em função do bem-viver geral ou do
bem-estar da vida coletiva.

Em segundo lugar, os cortiços. No cortiço as táticas serão acopladas: os
riscos de epidemia compõem estratégias que possibilitam práticas de
erradicação do anti-higiênico e do controle de uma população difusa e
indesejada. Embora boa parte do poder exercido se ampare no argumento da
promoção da saúde pública, virão acopladas às demandas
médico-higienistas demandas morais, motivos do controle da subversão da
ordem etc. Se, como dizíamos, os diagramas de poder que irão ativar a
reurbanização da cidade no entresséculos são pautados pela tarefa de
gerenciar, dirigir, salvar e cuidar pelo bem-viver de uma
multiplicidade, as relações de poder que investem cortiços desde a
década 1870 serão pautadas por outra coisa: pela divisão, pela
interdição, pelo imperativo de impedir, de isolar, de despejar,
desocupar, constituindo por isso um bloco, um tipo estratégico cuja
lógica interna flerta com a violência e cujo alvo é muito mais
setorizado.

É porque no cortiço, segundo Barata Ribeiro, ``acha-se de tudo: o
mendigo que atravessa as ruas como um monstro ambulante; a meretriz
impudica, que se compraz em degradar corpo e alma, os tipos de todos os
vícios e até\dots{} o representante do trabalho''.\footnote{Cândido Barata
  Ribeiro, \textit{Quais as medidas sanitárias\ldots{}}, 1877, p.~96.} O
cortiço é semelhante às fronteiras, não há clara distinção entre o
cidadão e o criminoso, não haverá distinção clara entre o vagabundo e o
``representante do trabalho''. Pode-se transitar de um polo ao outro
dentro da mais grotesca arbitrariedade policial. Um pouco mais tarde, em
1884, o chefe de polícia da Corte escreve à Câmara Municipal reclamando
medidas sobre o estado de ruína em que se encontravam ``os cubículos do
grande cortiço da Rua da Relação, o qual foi condenado a despejado há
mais de dois anos, (\ldots{}) a fim de que se sirvam mandar demolir os
ditos cubículos, que servem hoje de valhacouto a vagabundos e para a
prática de atos imorais''.\footnote{BR RJAGCRJ 41.3.35 --- Fundo Câmara
  Municipal --- Série Cortiços e Estalagens, sem paginação.} O cortiço é
a propriedade privada sujeita à violação. Em um primeiro momento,
prioriza-se a fragmentação dos sujeitos de direito: o poder policial que
interdita e reprime é o poder que reproduz o morador de cortiço como
indivíduo suspeito. Enquanto isso acontece, os enunciados que assimilam
o morador de cortiços entre as atribuições das instituições de saúde
pública são os mesmos que demarcam seu perigo para a salubridade urbana.
Na medida em que ambos --- poderes e enunciados --- estiverem
constituídos, caberá a nós situar a questão sobre onde esses dois
elementos se encontram e acham um horizonte comum para atuar em
conjunto. Trata-se em princípio de mostrar como a formação dos dois,
mesmo sendo diferentes, serviram de pontos de apoio um ao outro, e deram
sequência a uma espécie de fragmentação territorial. Cria-se no espaço
urbano um inédito desmembramento, novo jogo de luzes e sombras, arranjo
de visibilidades em larga medida sustentado pela cisão entre a cidade e
os cortiços. Arranjo de luzes que replica um arranjo político, garantido
por um projeto de sociedade, e com efeitos que farão eco sensível na
história posterior do Rio de Janeiro. (Aqui, pontualmente, tudo muito
semelhante às contradições que hoje se vive nos contornos do morro e do
asfalto.)

Vejamos como dentro deste jogo complexo e instável o discurso higienista
pôde ser ao mesmo tempo instrumento e efeito de poder, ou então
obstáculo, escora, ponto de partida para uma estratégia oposta.
Dispositivo médico-higienista, que é isso? A questão delimitará e
reintroduzirá a última etapa deste capítulo: o debate higienista segundo
as formas de problematização da habitação coletiva na segunda metade do
\textsc{xix}, no Centro Velho da cidade.

Pouco nos auxiliaria saber se foi o higienista quem intuiu o morador de
cortiços como uma ameaça à salubridade pública ou se foi o policial o
primeiro a projetar o sujeito como suspeito porque morador de cortiços.
O critério da originalidade ou de quem primeiro proferiu tal ou qual
enunciado é pouco pertinente quando nos orientamos por uma análise de
dispositivos concretos. Normalmente, a relação entre o que se diz e o
que se vê não está previsivelmente orientada a repetir padrões de
causalidade ou redirecionamento unilateral. Esse aspecto já se revelou
útil na leitura das memórias dos europeus de passagem pela Corte. O
dispositivo médico-higienista não é um sistema ou uma estrutura, mas uma
sorte de relação instável entre coisas ou elementos heterogêneos. Os
elementos atuam como linhas ou vetores, linhas de natureza distinta
(visibilidades/enunciabilidades/forças, ou somente linhas de
\textit{discursividades} e \textit{extra-discursividades}). As linhas formam
processos sempre desequilibrados, uma vez que o dispositivo não é
simplesmente local de adequação entre linhas de força e regimes
discursivos. ``O discurso veicula e produz poder; reforça-o mas também o
mina, expõe, debilita e permite barrá-lo. Da mesma forma, o silêncio e o
segredo dão guarida ao poder, fixam suas interdições''.\footnote{Michel
  Foucault, \textit{História da sexualidade I: a vontade de saber}
  {[}1976{]}, trad. Maria T. Albuquerque e J. A. Albuquerque (Rio de
  Janeiro, Graal, 1977), p.~96.} O dispositivo é uma maneira
estratégica, portanto multipontual e volúvel, de fazer funcionarem
relações de poder em uma função, e fazer funcionar uma função através
dessas relações de poder. Como o poder integra esta máquina de
enunciados e visibilidades que se chama ``dispositivo''? Temos enfim
aqui, rebatida, a encruzilhada que registra uma guinada decisiva na
proposta de Foucault, três eixos ou três fatias de realidade: as forças
em exercício, a funcionalidade dos enunciados, a objetividade do que nos
vêm ao encontro como visível ou não, são todos vetores ou tensores. ``De
modo que as três grandes instâncias que Foucault distingue
sucessivamente (\ldots{}) não possuem de modo algum contornos
definitivos, mas são cadeias de elementos variáveis relacionadas entre
si''.\footnote{Gilles Deleuze, ``¿Qué es un dispositivo?'' {[}1988{]},
  em \textit{Michel Foucault, filosofo} (Barcelona, Editorial Gedisa,
  1990), p.~155. Tradução nossa.} Só que, se essa relação (ou mesmo
cortes de relação) se dá entre redes ou cadeias de elementos variáveis,
o que referencia essas redes e possibilita integrá-las como vetores de
um só dispositivo?

Certamente não são os universais tomados como categorias gerais ou entes
da razão como a Ciência, o Estado, a Lei, a loucura, a doença. O
dispositivo médico-higienista não se reporta à medicina ou à higiene
como redomas do conhecimento sobre a saúde, nem são os enunciados destas
últimas o núcleo da nossa investigação. ``O dispositivo propriamente
dito é uma rede que se pode estabelecer entre estes
elementos'',\footnote{Michel Foucault, ``Le jeu de Michel Foucault''
  {[}1977{]}, em \textit{Dits et Écrits III --- 1980-1988} (Paris,
  Gallimard, 1994), p.~299. Tradução nossa.} trata-se de um conjunto
heterogêneo que agrega discursos, instituições, planos urbanísticos,
decisões regulamentares, leis, enunciados científicos, proposições
filosóficas, morais etc. Os dispositivos se reportam à historicidade das
formas de problematização, formas de problematização que são
inseparáveis das ``regras de existência para os objetos que aí se
encontram nomeados, designados ou descritos, para as relações que aí se
encontram afirmadas ou negadas''.\footnote{Michel Foucault, \textit{A
  arqueologia do saber} {[}1969{]}, trad. Luiz F. B. Neves (Rio de
  Janeiro, Forense Universitária, 2008), p.~102-3.} Em outras palavras:
uma forma de problematização possui um limiar de existência, um limiar
instaurado pelas descontinuidades históricas que nos separam do que não
podemos mais ver ou dizer, e que logo estão fora do nosso campo de
experiências possíveis, estão fora inclusive de um campo de
racionalidade científica que qualificou o que foi aceito como enunciado
científico e o que não foi. Se nos for permitido pensar formas de
problematização como a delimitação dos problemas que emergem em tal ou
qual temporalidade histórica, nelas então vemos desmembrar-se essa
identidade em que gostamos de nos olhar para conjurar as rupturas da
história, na medida em que estabelece ``que somos diferença, que nossa
razão é a diferença dos discursos, nossa história a diferença dos
tempos, nosso eu a diferença das máscaras''.\footnote{\textit{Ibidem},
  p.~148.} Formas de problematização delimitam um campo de experiências
possíveis, sendo justamente essa abertura de possibilidades históricas
que é aqui definida. Já o dispositivo é a sorte de formação que, em um
dado momento histórico, teve como papel responder a uma urgência,
responder a um repertório de problemas em demanda.

\begin{quote}
O dispositivo tem então uma função estratégica dominante. Que poderia
ser, por exemplo, a absorção de uma massa de população flutuante que uma
sociedade tornara incômoda a uma economia de tipo essencialmente
mercantilista: houve aí um imperativo estratégico, atuando como matriz
de um dispositivo, que foi se tornando pouco a pouco o dispositivo de
controle-assujeitamento da loucura, da doença mental, da
neurose.\footnote{Michel Foucault, ``Le jeu de Michel Foucault''
  {[}1977{]}, em \textit{Dits et Écrits III --- 1980-1988}, p.~299.}
\end{quote}

O dispositivo é a rede de vetores que dá acesso a formas de
problematização, ele é um conjunto heterogêneo e multipontual de
vetores, e ao mesmo tempo a lógica das relações de projeção, dispersão
ou contradição que se pode estabelecer entre as diferentes linhas ou
vetores. Assim se compreende como, sob essa análise, uma Secretaria de
Polícia pôde integrar essa rede de instituições, técnicas e saberes que
tiveram a destreza de dotar o cortiço da capacidade de abrigar a causa
eficiente das epidemias. Não somente a polícia, mas os engenheiros da
Inspetoria Geral de Obras Públicas, os fiscais de freguesias, a Câmara
Municipal e, naturalmente, a Junta. Isso não significa que os numerosos
aparelhos de poder estivessem desmembrados uns dos outros, impotentes na
atividade de constituírem um dispositivo de conjunto, a ponto de não
poderem esboçar em caráter implícito grandes estratégias anônimas,
objetivas e, no entanto, quase mudas. O extremo oposto também não tem
apoio empírico: seria ingênuo querer contar com um alinhamento orgânico
e coeso de instituições, técnicas de poder, decisões parlamentares,
resoluções administrativas etc. pondo em marcha engrenagens de um poder
verticalizado e irreversível. O dispositivo atua de fato em conjunto, há
infiltração e atravessamento de funções estratégicas concretas, o que
não significa reproduzir a imagem de uma unidade de poder cujo recurso
uniforme e exclusivo fosse fazer funcionar a interdição e materializar o
funcionamento da lei. Segundo esse modelo jurídico e pobre de
compreender o funcionamento do dispositivo, o poder teria como princípio
exclusivo ``a potência do `não', incapacitado de produzir, apto apenas a
colocar limites, seria essencialmente antienergia; esse seria o paradoxo
de sua eficácia: nada poder, a não ser levar aquele que sujeita a não
fazer senão o que lhe permite''.\footnote{Michel Foucault, \textit{A
  arqueologia do saber} {[}1969{]}, 2008, p.~83.}

Deparamo-nos em nossa investigação histórico-filosófica com a capacidade
de se fazerem mover certas funções através de relações de poder
reversíveis, relações que envolvem negociações, porosidades, pontos de
fuga, alternativas para novos rearranjos. Isto posto, ao invés de
identificarmos núcleos de poder, optamos por pensar fluxos de forças em
procedimentos estratégicos. Em vez de a fórmula do poder segundo a
imagem da guerra, de conflito cujo termo reestabelece o ordenamento
estático da paz, a genealogia pensa o poder em sua mobilidade,
reversibilidade e precisão microfísica das relações de força; em vez de
o poder visto pela chave de uma concepção jurídica ou essencialista, a
genealogia pensa o poder definido como ``relacional''. Parece-nos
razoável a imagem de ``campos de força'' em desequilíbrio, fluxos de
força que se irradiam em direções diagonais e difusas. É possível dispor
as formas de problematização da habitação coletiva e o campo de
experiência da epidemia na década de 1870 utilizando esses instrumentos
e ferramentas conceituais e metodológicas.

Do momento de sua criação à definição de suas atividades, a Junta
Central de Higiene Pública pôde contar com dois níveis de atribuições
que, inovadoras em termos de conjunto, possibilitaram estratégias de
poder mais ou menos felizes na maneira como responderam às suas
urgências. Aos estudos e relatórios dedicados aos problemas urbanos
ligados à salubridade geral, à indicação das providências à sua
consecução, agrega-se o exercício da polícia médica --- duas frentes de
atuação que nem sempre preencherão as expectativas do regulamento
oficial da Junta. Pois bem, a tarefa de responder enquanto instituição
de consultoria, sob demanda do governo, para prestar socorro teórico e
produzir conselhos favoráveis à saúde pública, lhe foi herdada da
Academia Imperial de Medicina (AIM). Embora, segundo Edler, o espectro
da AIM extrapolasse o limitado papel de instrumento consultivo da
política imperial em matéria saúde pública,\footnote{Além de
  proporcionar a organização profissional e a regulamentação do ensino
  médico no Brasil, a Academia firmou as regras de produção dos fatos e
  teorias médicas no contexto da experiência epistemológica da
  anatomoclínica europeia. ``Quando a Sociedade de Medicina foi criada,
  a Higiene e a Anatomoclínica passaram a dispor de uma trincheira
  estrategicamente orientada para enfraquecer a influencia dos antigos
  cirurgiões portugueses e daqueles formados nas escolas
  Médico-Cirúrgicas da Corte e da Bahia. Nesse período, marcado pela
  crença da dependência da patologia e da terapêutica médicas aos
  fatores climático-telúricos circunscritos ao meio ambiente, a rejeição
  da herança colonial da Fisicatura-mor e do legado de informações
  médicas, mais ou menos impressionistas, descritas pelos viajantes
  naturalistas, impôs-se como pré-condição a afirmação do novo
  \textit{ethos} profissional. Em torno da Academia de Medicina, uma elite
  médica empenhou-se, na produção de um conhecimento original sobre a
  patologia brasileira. Desde sua criação, até meados do século, ela
  conseguiria monopolizar duas importantes tarefas: ao mesmo tempo em
  que se impusera como instrumento da política imperial da saúde
  pública, tornara-se o principal árbitro das inovações
  médico-científicas, contribuindo tanto para sancionar novas
  tecnologias em diagnóstico e terapêutica, quanto novos conceitos e
  teorias estritamente voltados para o conhecimento da patologia
  brasileira.'' Flavio C. Edler, ``Doença e lugar no imaginário médico
  brasileiro'', Anuário IEHS, 21, 2006, p. 389.} a Academia monopolizou
a atualização da agenda higienista brasileira pelo tempo que duraram
suas ambições administrativas no Império. Esse quadro sofre uma
reviravolta súbita na irrupção das catástrofes epidêmicas que se
seguiram a 1850. Em curto espaço de tempo, a febre amarela se
transformaria ``na principal questão de saúde pública no Brasil. Tal
fato (\ldots{}) contribuiu para a criação da Junta Central de Higiene
Pública, que deslocaria a Academia do papel central que até então
representara para a saúde pública''.\footnote{Idem\textit{,} p.~391.} A
criação da Junta irá marcar uma descontinuidade profunda no que diz
respeito à apreciação das questões caras à higiene pública. Trata-se da
invenção de uma instituição que não é redutível a instrumento de estudo
das epidemias, moléstias, epizootias, e dos meios adequados para
prevenções e tratamentos. Aparecem submetidas, por exemplo, à
presidência da Junta, estabelecimentos como o Instituto Vacínico e o
Provedor da Saúde do Porto do Rio de Janeiro. O mais importante para
nós: ficarão a cargo da instituição tarefas de natureza policial, ou
seja, as que exigem a atribuição \textit{explícita} de um poder de
polícia. Isso quer dizer: autorização para fiscalizar, examinar,
inspecionar e executar a vigilância.

O primeiro regulamento da Junta, sancionado no contexto da primeira
epidemia de febre amarela, prevê o policiamento do exercício da medicina
e demais profissões conexas, ou seja, controle do charlatanismo,
fiscalização de médicos, cirurgiões, boticários, dentistas e parteiras
autorizados para o exercício da profissão. As atribuições de, como
autoridade policial, superintender armazéns, matadouros, boticas,
confeitarias, cadeias, colégios, oficinas, cemitérios --- ``e em geral
todos os lugares donde possa provir dano à saúde pública, ou pelas
substâncias que se fabricam ou pelos trabalhos que se operam''\footnote{Decreto
  nº 828, 29 de setembro de 1851 (Coleção das Leis do Império do Brasil
  --- 1851, p.~259, vol.~1, pt.~II.).} --- não livrou o regulamento
original de ser alvo de críticas amargas por parte do higienista que
primeiro presidiu a casa. Essas coisas deixam a par do que efetivamente
esteve ao alcance da Junta, ao longo de algumas décadas de atividade,
tendo em vista o amplo raio de atribuições policiais que lhe foram
incumbidas. Paula Candido assim se pronuncia a respeito em 1857:

\begin{quote}
Desgraçadamente os encargos de perscrutar as causas da insalubridade,
tão sabiamente impostos à \textit{Junta} pela lei de sua criação
(\ldots{}) foram ampliados nos regulamentos, de modo a abranger uma
infinidade de atribuições policiais e enleadas formalidades, às vezes
mesmo ridículas, que mal quadram a quem deve prestar não interrompida
atenção aos multiplicados agentes modificadores da saúde pública, que
exigiam ciência, dedicação e insano trabalho. (\ldots{}) Não bastando
essas dificuldades, surgiram outras: esse estudo difícil e apurado deveu
ser interrompido para cuidar-se das desastrosas epidemias de febre
amarela e de cólera morbus e não só cuidar, em face do perigo, de
medidas de momento, urgentes e prontas, como, antecipando a parte que
devia entrar em tempo oportuno no complexo das medidas, tratar
imediatamente das que a ocasião reclamava, \textit{das medidas preventivas
contra importações epidêmicas}, isto é, traçar o regime sanitário dos
portos do Império, porquanto urgia que este importante ramo de higiene
pública nos preservasse de futuras calamidades. Ora, além de todos estes
embaraços, ainda se distraía a atenção da Junta para ir inspecionar: se
uma cova ou sepultura só tinha, por desleixo do funcionário, 6 ½ palmos
no lugar de 7 de profundidade; se o terreno era bem impermeável (\dots{});
se o taverneiro vendia queijo ardido; se o farmacêutico, o médico e a
parteira matriculavam com todos os rigores da formalidade os seus
títulos; se o especulador analfabeto curava esclerose, tísica, chagas
etc. (\dots{}).\footnote{Plácido Barbosa e Cassio Barbosa de Resende,
  \textit{Os serviços de saúde pública no Brasil}, 1909, p.~68.}
\end{quote}

Um decreto de 1876, indiferente ao inchaço da maquinaria institucional,
e em resposta à segunda grande mortandade por febre amarela no \textsc{xix},
redefine as instruções relativas aos serviços sanitários a cargo da
Junta Central de Higiene Pública. O texto inclui no rol de atribuições
da Junta um elemento decisivo e novo, não porque dependia de
regulamentação legislativa para alcançar urgência, mas justamente porque
já qualificava a maneira como a epidemia vinha sendo
\textit{experienciada} nas bordas da lei. Este elemento decisivo, que era
o cortiço, apareceria integrado como fator nas formas de problematização
das epidemias de febre amarela na década de 1870. A letra da lei prevê
como parte dos trabalhos da Junta o ``estudo sobre as condições
higiênicas dos edifícios públicos e particulares que se construírem.
Saneamento dos cortiços ou estalagens e dos dormitórios
públicos''.\footnote{Decreto nº 6.406, de 13 de dezembro de 1876
  (Coleção de Leis do Império do Brasil --- 1876, p.~1243, vol.~2,
  pt.~II).} Em 1873, a Câmara Municipal já havia aprovado Posturas que
proibiam construções de ``habitações vulgarmente chamadas
\textit{cortiços}, entre as praças de D. Pedro II e Onze de Junho, e todo
o espaço da cidade entre as ruas do Riachuelo e do
Livramento''.\footnote{Rio de Janeiro. Câmara Municipal. Código de
  Posturas da Ilustríssima Câmara Municipal do Rio de Janeiro e Editais
  da mesma Cidade. Posturas de 5 de dezembro de 1873: ``Proíbe a
  construção de habitações chamadas `cortiços'; e marca a direção dos
  veículos de condução ou de transporte pela rua de Gonçalves Dias.''}
Uma vez que estaria vedada, quando não houvesse autorização da Câmara
Municipal, a construção de novos cortiços no perímetro urbano da Corte,
deduz-se sua existência massiva nas freguesias centrais (talvez com
exceção da região que compreende a Candelária). Em 1876, uma nova
Postura sobre o mesmo tema, dessa vez refinando o objeto de interdição:
não serão de fato permitidas novas construções dos vulgos ``cortiços,
quer sejam assim denominadas, quer sejam chamadas casinhas ou com nomes
equivalentes, no perímetro da cidade''.\footnote{Rio de Janeiro. Câmara
  Municipal. Código de Posturas da Ilustríssima Câmara Municipal do Rio
  de Janeiro e Editais da mesma Cidade. Postura de 1º de setembro de
  1876: ``Sobre construção de cortiços''.} Aparentemente as disposições
não vinham sendo observadas pelos proprietários corticeiros, que ou
burlavam os projetos submetidos à apreciação dos engenheiros (vez em
quando argumentaram se tratar não de cortiços, e sim ``casas de porta e
janela'', casinhas ou equivalentes), ou negociavam a ``vista grossa''
dos fiscais das freguesias, como diz Luís Edmundo,\footnote{Luís
  Edmundo, \textit{O Rio de Janeiro do meu tempo --- vol.~2} {[}1938{]}, 2ª
  ed.~(Rio de Janeiro, Conquista, 1957), p.~357, citado em Jaime Larry
  Benchimol, \textit{Pereira Passos: um Haussmann tropical}, 1992, p.~134.}
ou ainda --- como em uma postura que submete as ``construções no interior
dos terrenos'' à licença da Câmara --- levantavam ``tabiques e outras
divisões (\ldots{}) para formação de quartos ou cubículos''\footnote{Rio
  de Janeiro. Câmara Municipal. Código de Posturas da Ilustríssima
  Câmara Municipal do Rio de Janeiro e Editais da mesma Cidade. Postura
  de 5 de maio de 1886: ``Sobre construções''.} no interior das
estalagens ou casas de alugar quartos.

A pergunta que então se coloca: o que faz com que dada habitação
coletiva fosse ou não identificada como cortiço? São inúmeras as
controvérsias, movidas por policiais, autoridades sanitárias,
proprietários e fiscais, que giram em torno da definição da palavra
``cortiço''. A permissão ou negação da licença para a construção de uma
unidade habitacional, a decisão da Câmara pela interdição, demolição ou
melhoramentos de um cortiço passavam pelo rastreio de uma resposta
satisfatória e por uma disputa entre discursos sem expectativas de
reconciliação. Mas a urgência de definir os cortiços, presumimos, talvez
não derive apenas da tipologia das construções, e, portanto, do caráter
variável, disperso e múltiplo da identificação do cortiço enquanto
objeto regido por um princípio de individuação. A história de uma
``coisa'' (o cortiço como matriz de epidemias ou esconderijo do crime e
da imoralidade) pode desse modo funcionar como uma cadeia de poderes e
saberes, de interpretações e ajustes sempre novos, cujas causas não
precisam estar relacionadas de maneira casual.

Uma chave possível para resituar a controvérsia em torno do cortiço
talvez passasse em diagonal por sua caracterização de esconderijo do
indivíduo \textit{suspeito}, aquele que ameaça a ordem pública e a
moralidade dos costumes. Em segundo lugar, o prejuízo que o cortiço gera
à saúde pública estaria ligado à sua tendência de concentrar
aglomerações de anônimos nocivos ao estado sanitário da cidade, por
constituírem as aglomerações nos cortiços ``uma das maiores causas de
insalubridade desta cidade e o foco donde emanam os primeiros casos das
molésticas infectuosas de índole epidêmica que nos flagelam todos os
anos, segundo o demonstra a observação dos clínicos desta
Capital''.\footnote{BR RJAGCRJ 43.1.25 / Fundo Câmara Municipal / Série
  Cortiços e Estalagens / AGCRJ Códice 43.1.25 --- Estalagens e Cortiços
  --- Requerimentos e outros papéis relativos à existência e à
  fiscalização sanitária e de costumes dessas habitações coletivas ---
  1834-1880, p.~50-1.} É essa a posição dos higienistas da Junta,
aparentemente reforçada pela posição da clínica médica.

João Vicente Torres Homem, um dos mais destacados clínicos brasileiros
do \textsc{xix}, escreveu e atuou no campo da clínica particular, e também na
Santa Casa de Misericórdia, sem por isso deixar de frequentar o debate
sobre as práticas higienistas em sua produção acadêmica. Chegou a ocupar
a prestigiada cátedra de Clínica Médica da Faculdade de Medicina, e
nesse período publicou muitas das chamadas ``memórias'' --- via de regra
baseadas em notas de aula recolhida por alunos ---, que se tornariam
clássicos no ensino. Uma delas traz o nome \textit{Estudos Clínicos sobre
as Febres do Rio de Janeiro}.

Pois bem, apareceu em 1877 um tal livro, desse clínico que
diferententemente dos primeiros catedráticos da Faculdade --- que se
lançaram na carreira política ou no exercício de cargos de altos
públicos --- foi dos poucos membros da elite médica a optar estritamente
pela atividade acadêmica e clínica.\footnote{\textit{Cf}. Luiz Otávio
  Ferreira, ``João Vicente Torres Homem: descrição da carreira médica no
  século \textsc{xix}'', \textit{Physis --- Revista de Saúde Coletiva}, vol.~4,
  n.~1, 1994, p.~57-78.} O livro traz uma descrição comum e nada
destoante do que então a política ou a polícia médica definiam por
``cortiço''. O contexto é o seguinte: Torres Homem discorre sobre as
causas da endemicidade da febre amarela; as causas são as condições
topográficas e o pouco cuidado do governo com o que diz respeito à
higiene pública. Às condições que nos ``foram dadas pela natureza à bela
cidade de S. Sebastião, \textit{e que certamente não poderiam ser
removidas}, juntam-se outras que procedem da incúria com que são
tratadas entre nós as questões da salubridade pública pelas altas
personagens que nos governam''.\footnote{João Vicente Torres Homem,
  \textit{Estudo clínico sobre as febres do Rio de Janeiro} (Rio de
  Janeiro, Livraria clássica de Nicolao Alves, 1877), p.~207.} Quais
sejam: o estado imundo das ruas, praças e praias; a maneira
inconveniente por que funcionam os esgotos da \textit{City improvements};
e os cortiços, os inúmeros focos de infecção que representam os chamados
``cortiços'',

\begin{quote}
verdadeiros antros, onde a vida e a saúde da classe pobre são
sacrificadas à sórdida ambição dos proprietários, onde em um estreito
cubículo, sem ar nem luz, acumulam-se três, quatro e mais pessoas, que
ali dormem, comem, e tudo fazem, sorvendo lentamente em uma atmosfera
infecta o veneno que lhes mina o organismo, e envenenando-se
reciprocamente (\dots{}).\footnote{\textit{Ibidem,} p.~207.}
\end{quote}

A análise da nocividade dos cortiços integra a urgência, ou um conjunto
de urgências, muito mais rico, que não é outro senão a localização da
origem das moléstias em um fator patogênico bastante difuso e
massificado: as aglomerações urbanas. Torres Homem narra como se
distancia da medicina que chegou um dia a considerar a constituição
médico-epidêmica ``uma influência geral que dá lugar a maior número de
moléstias de certa ordem, e imprime a estas moléstias caráter comum
especial''.\footnote{\textit{Ibidem,} p.~60.} Foi preciso que se esperasse
despertar um espírito clínico para que os sistemáticos da escola
fisiológica pusessem de lado as extravagâncias dos estudos das
constituições médicas. E que notável reação se desse para que a medicina
prática, ele diz, se desprendesse dos exageros que filiavam as epidemias
às influências das estações calmosas e soubesse aceitar a complexidade
da dependência das moléstias em relação às influências sociais.

\begin{quote}
Aglomerações dos povos constituindo centros populosos podem exercer
influência na evolução patogênica das epidemias. (\ldots{}) A
aglomeração considerada inteiramente independente das influências de
costumes e da higiene pública só por si pode muito contribuir para o
desenvolvimento das moléstias pestilenciais. (\ldots{}) Considerada
porém a aglomeração conjuntamente com outros fatores, ainda mais
saliente se torna a sua preponderância.\footnote{João V. Torres Homem,
  \textit{Estudo clínico sobre as febres do Rio de Janeiro}, 1877, p.~72.}
\end{quote}

Temos alguma dimensão da presença da população de cortiço nas freguesias
centrais (Candelária, São José, Sta. Rita, Sacramento, Santana, Sto.
Antônio, Espírito Santo) pelas estatísticas fornecidas pelos estudos de
Antônio M. Pimentel, e pelos trabalhos mais recentes de Eulália Maria
Lobo e Lia de Aquino Carvalho.\footnote{Antonio M. de Azevedo Pimentel,
  \textit{Subsídios para o estudo de higiene do Rio de Janeiro} (Rio de
  Janeiro, Tipografia e Lit. De Carlos Gaspar da Silva, 1890); Eulália
  M. Lahmeyer Lobo, \textit{História do Rio de Janeiro: do capital
  comercial ao capital industrial e financeiro} (Rio de Janeiro, IBMEC,
  1978), 2 vols.; e Lia de Aquino Carvalho, \textit{Contribuição ao estudo
  das habitações populares: Rio de Janeiro, 1866-1906} (Rio de Janeiro,
  Secretaria Municipal de Cultura --- Dep. Geral de Doc. e Inf. Cultural
  --- Divisão de Editoração, 1995).} O censo de 1870 identificava uma
população nessas freguesias de 138.607, o número dos que residiam em
cortiços em 1868 era de 17.412 pessoas. Em relação à população total de
cortiços na área urbana, a participação da população de cortiço das
freguesias de São José, Santana e Santo Antônio correspondia nesse
período a 58\%. Entre 1870 e 1890, a proporção entre os moradores de
cortiços e a população de freguesias centrais sobe de 12,5\% para
17,6\%. Só que, apesar da alta densidade populacional, diz Torres Homem,
conquanto possa influir como fator patogênico, o descalabro em matéria
de higiene (as inumeráveis maneiras de se produzir, pela ação de um
miasma desprendido da decomposição da matéria orgânica, uma moléstia
infecciosa) é a garantia de desenvolvimento de uma epidemia.

Por esses motivos a obtenção da licença para a construção de cortiços
nessa época obedecia a um ritual jurídico bastante truncado. Exigia-se,
para tais edificações, o deferimento do projeto pela Câmara, acompanhado
de um laudo da Junta, que por sua vez apreciava tratar-se ou não de um
cortiço, ou simplesmente considerava ser habitável o edifício ou
negligente no tocante às disposições higiênicas. O procedimento contava
ainda com uma consulta ao Fiscal da respectiva freguesia e com a palavra
do engenheiro do Distrito, encarregado de verificar se o prospecto se
achava conforme as Posturas Municipais. Dependia da Câmara a última
palavra --- até o ano de 1892 o cargo de prefeito nem sequer existia ---,
mas o deferimento da Junta gozava de mais prestígios que outras
instâncias da burocracia. Muito embora a pressão exercida pelas
instâncias de engenheiros municipais que fiscalizavam as obras ---
pressão quase sempre animada pela queda de braços entre eles e os
médicos, o que poderia transformar o processo em uma verdadeira maratona
---, as decisões pelo licenciamento costumavam estar alinhadas com a
decisão do presidente da Junta. Em 11 de Outubro de 1870, o peticionário
Antonio Guimarães pedia licença para construir algumas ``casas de porta
e janela'' dentro do seu terreno na Rua do Livramento. A Câmara atende
imediatamente ao pedido alegando que o projeto obedecia às disposições
da Postura de 1º de Agosto de 1855, ``devendo-se também ouvir a Junta de
Higiene Pública''. Correrão dois meses até que o Barão de Lavradio
encaminhe a resposta:

\begin{quote}
cobrindo a petição em que Antonio Joaquim M. Guimarães, pede licença
para construir dentro do seu terreno à rua do Livramento, nº 113 algumas
casinhas (vulgo cortiços), tenho a honra de devolver a referida petição,
informando achar-se más condições de ser deferida.\footnote{BR RJAGCRJ
  41.3.35 --- Fundo Câmara Municipal --- Série Cortiços e Estalagens, sem
  paginação.}
\end{quote}

Ora, as decisões a favor do licenciamento não eram raras, sem que no fim
das contas se tivesse toda dimensão de como os critérios administrativos
se aplicavam. Em 1873, o Barão de Lavradio concede licença para
construção de cortiço de 54 quartos na freguesia do Sto. Antônio. No
códice de documentos quase ilegíveis, consta que seriam três casas, além
de um cortiço com 54 quartos repartidos ao longo de corredores em linhas
paralelas. A distribuição tinha o sentido de possibilitar o escoamento
das águas dos terrenos e a única condição imposta pela Junta para a
liberação das obras era que se calçassem com paralelepípedos as áreas
comuns.

\begin{quote}
A Junta Central de Higiene Pública, a quem foi presente o ofício de V.
Ex.ª de 11 de Setembro findo, remetendo para informar o requerimento de
Francisco da Silva Ayrosa, pedindo para construir cortiços no seu
terreno à Rua do Rezende nº 75, tem a honra de informar a V. Exª, que se
pode conceder a licença requerida, uma vez que sujeite-se ele às
resoluções da Illma Câmara a este respeito, e dê fácil escoamento às
águas, ponderando que lhe parece excessivo o número de quartos, que o
Peticionário quer construir no fundo do terreno, atenta a extensão da
área do mesmo.\footnote{\textit{Ibidem}, sem paginação.}
\end{quote}

Esses processos poderiam tramitar por meses, e uma licença
preliminarmente autorizada pela comissão de obras da Câmara, com alvará
já expedido pela justiça, corria o risco de sofrer uma reviravolta e
resultar no carimbo de indeferida. É o caso de João Julio da Silva, que
em abril de 1883 requer da Câmara licença para construir na Rua da
Guarda Velha nº 24 (atual Treze de Maio) ``casinhas de acordo com o
prospecto''.\footnote{\textit{Ibidem}, sem paginação.} A Câmara manda
consultar o Fiscal da freguesia de S. José. O Fiscal passa adiante a
matéria da petição, para as mãos do engenheiro do distrito, a quem
competirá o parecer que apura ``se o prospecto se acha conforme às
Posturas''. A vistoria é feita e o engenheiro assina a licença,
``obrigando-se o suplicante a construir as casinhas não só segundo o
prospecto como também segundo a planta junta, a fim de que com a
supressão das áreas não prejudique as condições higiênicas das casinhas
projetadas''. A Câmara acata a licença, dando a permitir o início das
obras. Dois meses mais tarde a Câmara volta atrás na decisão, ``à vista
do ofício do Presidente da Junta Central de Higiene Pública'', que
conclui se tratar de construção de cortiço projetado em área proibida. O
corticeiro recorre sem economia de eloquência. ``Porquanto não é nem
nunca poder-se-ia considerar em tempo algum Cortiço'', mas uma pequena
estalagem ``em sua maior parte ocupada por empregados da casa, e em
outra por artistas e outros empregados que saem de manhã e entram à
noite --- só homens''. O fato de ser ocupada por ``empregados da casa''
indica se tratar de uma quitanda ou equivalente, cujo proprietário é um
comerciante que explora cortiços construídos nos fundos do terreno,
submetendo os empregados ao acordo de serem fregueses diretos da venda.
Aqui, o cenário é mais ou menos semelhante àquele desenvolvido por
Aluísio Azevedo em \textit{Casa de Pensão} (1887). As imediações da
Misericórdia eram servidas da má reputação das antigas estalagens,
locandas e tascas, que vendiam comida a preço ínfimo e abrigavam
estudantes, carroceiros, marujos, soldados, e possivelmente o baixo
meretrício --- cuja prática nessa época não parecia restrito às áreas
pouco frequentadas pela elite. As meretrizes que conseguiam afirmar na
Justiça seu direito de locomoção, podendo residir onde desejassem,
enfrentavam ``a resistência constante da polícia que, a despeito das
decisões judiciais, pouco a pouco vai restringindo seu
espaço''.\footnote{Marcos Luiz Bretas, \textit{A guerra das ruas: povo e
  polícia na cidade do Rio de Janeiro} (Rio de Janeiro, Arquivo
  Nacional, 1997), p.~103.} De qualquer modo, o destaque --- no recurso
de João Julio da Silva --- em se tratar de uma casa circulada por
trabalhadores honestos ``que saem de manhã e entram à noite --- `só
homens'\,'', reforça a boa fama do negócio. O suplicante anexa ao
recurso a planta do projeto, e argumenta estar de acordo ``com as
dimensões e alturas de pontos por esta Ilustríssima Câmara exigidos''. O
recurso, que viria a ser indeferido em definitivo no dia 18 de dezembro
de 1883 (isto é, quase oito meses após a abertura do processo), termina
com uma queixa delicada ao desleixo profissional do presidente da Câmara
e uma denúncia ao poder arbitrário e inconsequente ostentado pela Junta:
o alvará só foi inutilizado, ele diz, ``por se lembrar o Sr.~Exmo.
Presidente de mandar à Junta de Higiene, e esta por sua vez mandou um
encarregado que por indisposição entendeu prejudicar o suplicante em sua
respectiva informação. E será muito para lamentar-se.''

O poder da Junta é de natureza ``relacional'', poder que não é anterior
ou externo às relações que o constituem. Ele não é objeto que possa ser
circunscrito em contornos ontológicos, mas antes uma materialidade cuja
manutenção exige o balanço das relações nas quais as trocas acontecem.
Isso permite intuir jogos ou campos de força onde pares ou os múltiplos
pontos de apoio não agem regidos somente por uma compreensão legalista
da lei. É também a lei objeto de negociação, são seus direcionamentos
reversíveis e oportunistas, sua matéria lugar de atravessamentos
capilares, equilibrando ou desequilibrando um arranjo de demandas
científicas, econômicas, morais, políticas e estéticas. O indeferimento
de uma licença para construção de um cortiço, por exemplo, não pode
prevenir contra a possibilidade de que a obra aconteça. Há focos de
resistência, linhas de fuga que se atualizam em relação aos interesses
da Câmara, à capacidade de fiscalização da Polícia ou à crença na
universalidade cientificista dos doutores. Há por vezes na papelada da
administração municipal um esclarecimento mais ou menos criterioso das
razões que orientaram a interdição de uma construção. O cumprimento ou o
desacato da interdição passa pelos limites da habilidade do higienista
de fazer valer a retórica do bem-viver e do bem-estar da saúde coletiva.
Ou ao menos dá a ver como se preenchem as lacunas da tolerância das
autoridades. Em 1882, por exemplo, um Sr.~Bastos e Alves, proprietário,
``pede para construir 14 casinhas em seu terreno à Rua dos Inválidos nº
46'', do que a Junta terá ``a honra de informar que a pretensão do
peticionário não deve ser deferida'' por três motivos:

\begin{quote}
1º porque as casinhas que o peticionário quer construir constituem
verdadeiros cortiços; 2º porque está provado que tais edificações dentro
da cidade são verdadeiros focos de insalubridade, que favorecem o
desenvolvimento e propagação de todas as epidemias; 3º porque contraria
a Postura de 5 de Dezembro de 1873.\footnote{BR RJAGCRJ 41.3.35 --- Fundo
  Câmara Municipal --- Série Cortiços e Estalagens, sem paginação.}
\end{quote}

Há um motivo de natureza legal, outro de natureza técnico-científica,
outro de natureza policial e estética. Interessa-nos justamente este, ou
seja, o que define, após realização da vistoria, que as casinhas
``constituem verdadeiros cortiços''. É relevante a informação de que o
indeferimento dessa obra em particular não foi acatado pelo
proprietário. No relatório do número de estalagens na freguesia de Sto.
Antônio, apresentado em 10 de Julho de 1884 pelo fiscal Carlos Pereira
Rego (felizmente ou não, o fiscal tem o sobrenome idêntico ao de José
Pereira Rego, presidente da Junta até 1881), consta estar localizada no
nº 46 da rua dos Inválidos uma estalagem com 35 quartos, arrendada pelo
Sr.~Bastos e Alves. José Pereira Rego, durante o período que tomou parte
nos trabalhos da Câmara como Vereador, apresentou em 1866 proposta de um
``Projeto para regulamentação da construção das estalagens e cortiços''.
A Câmara engavetaria, contentando-se em manter algumas das resoluções na
Postura de 1873 (a que interdita edificações de cortiços na Cidade
Velha). Mas o projeto é ilustrativo de uma coisa, sobre a qual havíamos
insistido enquanto pensávamos o processo de naturalização dessa espécie
de contiguidade entre mau cheiro e ruas estreitas e sinuosas. O
dispositivo médico-higienista não distingue anti-higiênico de
antiestético (é claro, o ``anti-higiênico'' em jogo não é a falta de
controle político-científico de condutas e circunstâncias que permitem
balancear a melhor saúde possível). Anti-higiênico aqui é metonímia de
insalubre, e insalubre é a base material e social capaz de prejudicar a
saúde de uma multiplicidade. Portanto, defeitos e irregularidades das
construções não concorrem somente para emperrar a salubridade de um meio
urbano, cujas circunstâncias entrelaçam a saúde de indivíduos em uma
cadeia de dependências mútuas. Quando o dispositivo chancela a
insalubridade de um cortiço qualquer, porque lhe parece defeituoso seu
estado da perspectiva da higiene pública, lança mão refletidamente da
sua irregular fisionomia, da ausência de proporções arquitetônicas, de
um senso estético mal delineado, mesquinho, feio. As denúncias da
Secretaria de Polícia não caracterizam os cortiços apenas como abrigos
de desordeiros ou focos de insalubridade, são casas ``construídas
extravagantemente'',\footnote{AGCRJ Códice 41.3.36 --- Fundo Câmara
  Municipal --- Série Cortiços e Estalagens, sem paginação.} ``em estado
de abandono absoluto'',\footnote{BR RJAGCRJ 41.3.35 --- Fundo Câmara
  Municipal --- Série Cortiços e Estalagens, sem paginação.} achando-se
``em estado de ruína os cubículos do grande cortiço da rua da
Relação''.\footnote{\textit{Ibidem}, sem paginação.} O projeto legislativo
de Pereira Rego clamava atenção para a urgência com que nossos cortiços
``ao mesmo tempo que contribuem para destruir todo embelezamento da
principal e talvez a primeira cidade da América Meridional, concorrem
igualmente para emperrar o seu estado higiênico''.\footnote{BR RJAGCRJ
  44.2.7 / Fundo Câmara Municipal / Série Habitações Coletivas AGCRJ.}
Tudo para dizer que o sonho político-científico de uma sociedade sem
cortiços veicula um projeto de embelezamento arquitetônico da capital.
Um artigo de 1872, publicado por um doutor nos \textit{Annaes Brasilienses
de Medicina}, censurando o que ali se define como ``cortiços lamacentos,
insalubres e imorais'', avalia que a ``arquitetura do Rio de Janeiro,
confiada simplesmente a analfabetos mestres de obra, maus pedreiros ou
péssimos carpinteiros, tem horror a libertar-se da mesquinhez em que
vive''.\footnote{``Concorrerá o modo por que são dirigidas entre nós a
  educação e instrução da mocidade, para o benéfico desenvolvimento
  físico e moral do homem?''. \textit{Annaes Brasilienses de Medicina} ---
  Tomo \textsc{xxiii}, n.~11, abril de 1872.} Algo parecido nos conta Pereira
Rego em discurso pronunciado no aniversário de 42 anos da Academia
Imperial de Medicina:

\begin{quote}
(\ldots{}) cumpre não deixar ao arbítrio de cada um construir casas como
lhe convier, torna-se indispensável adotar um plano em geral de
edificações, em que sejam previstas estas condições. Talvez os espíritos
minimamente escrupulosos enxerguem na adoção desta medida um atentado
aos direitos e à liberdade dos edificadores; mas, assim opinando, devem
concordar que igual atentado se dá em outros preceitos que a lei impõe
com relação às alturas, à forma das frentes e outras condições dos
edifícios, e que os preceitos estatuídos para garantir a salubridade e
asseio das habitações, devem ser prescritas por lei e não unicamente
recomendados.\footnote{``Discurso pronunciado pelo Exmo. Sr.~Conselheiro
  Dr.~José Pereira Rego na sessão solene da Academia Imperial de
  Medicina em 30 de Junho de 1871''. \textit{Annaes Brasilienses de
  Medicina} --- Tomo \textsc{xxiii}, n.~2, julho de 1871.}
\end{quote}

O ``Projeto para regulamentação da construção das estalagens e
cortiços'' é uma tosca, porém incisiva, tentativa de normatizar as
construções seja conforme padrões de higiene pública, seja conforme
imperativos de simetria, alinhamento e proporções arquitetônicas
validadas pelo gosto da moda. O autor do projeto se presta então ao
trabalho de estabelecer as proporções e condições adequadas para os
cortiços que, construídos, não afetarão o embelezamento e a salubridade
pública: eles deverão ser assoalhados e não asfaltados, cobertos de
telha, forrados ou não, porta e janela frontais, uma janela de fundos
disposta em vidraças, 15 palmos de largura sobre 20 de fundo (quando
tiver um só compartimento), 32 quanto tiver dois etc.

Um motivo de natureza legal, outro de natureza técnico-científica, e um
de natureza policial e estética. Incluiremos um quarto tensor, um pouco
mais tênue e discreto, mas não menos operante. Há um motivo de natureza
econômica, que é justamente o que melhor extrapola o poder de decisão da
Junta em sua política de ``segurança pública'' adotada a respeito dos
cortiços. Ele pode ser melhor visualizado nas polêmicas que envolvem não
a construção, mas sim o fechamento, a interdição e a demolição de
cortiços em plena atividade.

Normalmente, ações de despejo ou ordens para demolição de um edifício
tinham como origem uma denúncia dirigida à Câmara. Qualquer denúncia à
falta de salubridade de uma habitação coletiva poderia partir de um
particular, da imprensa, do fiscal da freguesia, do chefe de polícia, ou
da própria Junta. O fechamento de um cortiço poderia esbarrar em
trâmites judiciais de ambas as partes, tornando sua execução uma
possibilidade custosa e lenta. Chalhoub narra como o caso da Cabeça de
Porco enfrentou ações de despejo e tentativas de demolição de quase 10
anos antes, até ser finalmente desmontado em 1892, como resultado de um
cerco higienista que incluiu inúmeras instâncias do poder executivo e da
administração pública. Em 1884, membros da comissão vacínico-sanitária
apresentaram um laudo da situação da Cabeça de Porco, estabelecendo
lotação máxima para algumas casas, ordenando a demolição ou a limpeza de
outras. Foram recorrentes as desinfecções por vítimas da febre amarela
no local, com emprego de cloros e soluções corrosivas, incinerando
colchões, travesseiros e peças de roupa. Em 1886, chegou-se a ordenar
diligências com o objetivo de determinar a demolição para viabilizar a
abertura de uma nova rua. Dois anos depois, os proprietários recebiam
intimação da Junta ``para fazer despejar os inquilinos dos mesmos
prédios no prazo de 40 dias a pretexto de estarem insanáveis as ditas
casas''. Segundo Chalhoub,

\begin{quote}
os proprietários alegavam ainda que, logo em seguida à intimação dos
higienistas, o subdelegado da freguesia de Santana havia se dirigido à
estalagem para dizer aos moradores que os despejaria à força. Alguns
moradores teriam aproveitado o ensejo para interromper o pagamento do
aluguel. Dona Felicidade {[}Perpétua de Jesus, proprietária{]} e seus
parceiros diziam que o cortiço estava em ``perfeito estado de
salubridade, e que a localidade em que se acha nada sofreu por ocasião
das epidemias que têm assolado esta cidade''. Argumentava também que a
intimação fora ilegal, por ``manifesta incompetência da Autoridade'', e
pediam indenização por perdas e danos.\footnote{Sidney Chalhoub,
  \textit{Cidade febril}, 1996, p.~189-90.}
\end{quote}

Poder-se-ia pensar que todas as demandas estão submetidas à rede de
razões da medicina, mas não. A Higiene não é o boneco de dedo que
gesticula em nome de outras modalidades de poder menos ostensivas.
Algumas polêmicas que resultam em despejo ou em processo de higienização
de ruas inteiras ranquearam, sem preferência ou hierarquia, enunciados
higienistas, tecnologias morais e práticas policialescas. Apesar de a
Higiene funcionar, com frequência, como mote para o saber que investe
práticas policiais, como dizíamos, existe aí um acoplamento tático. Sem
dúvidas, não se pode desarticular o cortiço sem que antes ele se torne
um perigo, uma ameaça, assim como não se executam manobras políticas sem
a veiculação, a promoção, a produção e a adoção de certas técnicas de
saber. Algumas ruas receberam por anos investimentos dessa natureza,
denúncias de ordem diversa, recursos jurídicos, laudos médicos,
interesses em discórdia, razões filantrópicas, sustentados por um
projeto político de objetivação de indivíduos suspeitos e de corpos
anti-higiênicos. Tudo isso possibilitado \textit{aprioristicamente} pela
capacidade de se fazer realizar no cortiço uma matriz da epidemia ou o
esconderijo do crime e da imoralidade. Como o cortiço, esse
desmembramento do espaço, essa fronteira no interior da cidade, se
tornou problema? Não foram necessários direcionamentos unânimes nem a
unilateralidade de um poder tirano, mas antes a enunciabilidade dessas
urgências, a visibilidade desses espaços, a esquematização dessas
relações de força. Insistíamos, há casos de despejos ou demolições
atravessados ao mesmo tempo por embates de interesses econômicos,
morais, policialescos, científicos e jurídicos. Há situações em que
quase todos esses elementos parecem atuar em conjunto.

Em 1874, Henrique Teixeira de Carvalho, morador da freguesia de Sto.
Antônio, escreve recordando o presidente da Câmara Municipal que

\begin{quote}
desde a epidemia das febres passou por Decreto de não se fazer mais
cortiço algum, e como sábado e domingo se fizeram sete cortiços de
tábuas de forro na Chácara da Relação, e tendo-se prevenido a dois
guardas fiscais e ao Senhor fiscal com duas cartas, e não se fez caso
até agora presente, e já bastante a imundição (sic) na dita chácara, e
os Srs. Guardas fiscais fazem a Vista Grossa, e por isso pede a V. Exma.
se digne de mandar informar que julgo ainda estarem por
acabar.\footnote{BR RJAGCRJ 43.1.25 --- Fundo Câmara Municipal --- Série
  Cortiços e Estalagens, p.~35.}
\end{quote}

Aparentemente a situação da Relação se agravara quando, em 29 de abril
de 1878 alguns cortiços foram ali condenados pela Câmara e pela Comissão
Sanitária da freguesia. Em 7 de outubro do mesmo ano, o presidente da
Junta escreve à procuradoria do município alertando que tinham expirado
os prazos para demolição de tais cortiços em ``estado de ruína e
péssimas condições higiênicas de seus cubículos''.\footnote{BR RJAGCRJ
  41.3.35 --- Fundo Câmara Municipal --- Série Cortiços e Estalagens, sem
  paginação.} Espera-se, pois, diz a presidência da Junta, ``que a Ilma.
Câmara desta vez mandará executar a demolição ordenada garantindo deste
modo a vida e saúde dos pobres infelizes ali moradores que, por falta de
recursos, procuram semelhantes domicílios''. Os lapsos de filantropia do
Barão de Lavradio não evitaram a réplica incisiva do procurador. A Junta
de Higiene labora em engano, diz, ``quando supõe que a condenação em uma
vistoria administrativa autoriza a Ilma. Câmara a demolir qualquer
edificação''.\footnote{\textit{Ibidem.}} A rigor, condenar um edifício à
demolição ``\textit{só se pode fazer por sentença judicial passada em
julgado, como já se consegue a respeito de alguns dos ditos cortiços
para cuja demolição estou tratando de execução das respectivas
sentenças}''. Em 1883, já a instituição sob comando de outro presidente,
escreve a Junta à Câmara reclamando o ``fechamento de cortiço nº 1 da
Rua da Relação, pertencente a José Gonçalves, em vista de suas péssimas
condições higiênicas''.\footnote{\textit{Ibidem.}} Dessa vez o conflito de
interesses na Rua da Relação não colocará em xeque a lentidão do
aparelho jurídico frente à urgência da saúde pública. A Câmara pede
imediatamente ao fiscal da freguesia que confirme a vistoria que
condenara o cortiço, do que responde o fiscal que nenhuma ``providência
pode a fiscalização desta Freguesia apresentar em relação à reclamação
que a Junta de Higiene Pública faz (\dots{}), visto que \textit{nenhuma lei há
que autoriza a Ilma. Câmara mandar fechar cortiços por falta de
condições higiênicas}''.\footnote{\textit{Ibidem}.} Por ora, os cortiços
da Rua da Relação não seriam demolidos, mas sim desocupados, conforme
atestam alguns ofícios da Secretaria de Polícia da Corte do ano de 1884.
O chefe de polícia descreve existirem na Relação edifícios compostos por
``cortiços de madeira desocupados e em estado de abandono absoluto'' ou
``cubículos de um grande cortiço'' em ruínas, ``o qual foi condenado e
despejado há mais de dois anos''. Solicita então ``a demolição imediata
dos referidos'', por se prestarem ``de valhacouto a vagabundos e
criminosos e constituindo-se foco de insalubridades''. Quer dizer, as
idas e vindas que marcam as ações de despejo, as quedas de braço entre
autoridades, a resistência dos moradores e a espontânea reocupação de
cortiços na Rua da Relação, todos esses elementos parecem deixar de fora
a única personagem que explora o cortiço e dele extrai renda, aquele
para quem o cortiço se explica por razões plenamente econômicas. Esses
proprietários que exploravam diretamente os cortiços (que poderiam ser
casinhas construídas sob as fachadas de um comércio por um arrendatário,
que poderiam ser objeto de ganhos especulativos ou serem levantados com
tapumes dentro de antigas residências aristocráticas que perderam
qualquer utilidade para seu proprietário) metamorfoseavam modestas
moradias em fontes de acumulação para um pequeno capital mercantil. As
fontes levantadas por Jaime Benchimol identificam a origem ilustre dos
proprietários de cortiços, através de uma série de relatórios
encomendados aos fiscais pela presidência da Câmara em abril de 1876.

\begin{quote}
Na relação de 1876, constam entre os proprietários de cortiços muitas
personalidades e instituições ilustres do Império. Na freguesia da
Glória, por exemplo, figuram, às vezes com mais de um cortiço, o Banco
Predial, a Santa Casa de Misericórdia, os conselheiros José Feliciano de
Castilho e Simões da Silva, um procurador, um juiz de órfãos\dots{} até
mesmo o cônsul da Argentina, Manuel de Frias, era proprietário de um
cortiço na Rua da Carlota, com quatro quartos.\footnote{Jaime Larry
  Benchimol, \textit{Pereira Passos: um Haussmann tropical}, 1992, p.~135.}
\end{quote}

Quão complexa é a rede ou cadeia de elementos e variáveis que qualificam
o dispositivo higienista. Pouca coisa teria efetividade se projetássemos
instituições de saúde pública atuando isoladamente, dedicadas a fazer
valer a verdade dos seus enunciados de controle político-científico da
saúde. O dispositivo é o que permite que partes contrárias ou
dissonantes trabalhem em conjunto, na manutenção de um mesmo repertório
de problematizações possíveis, submetendo uma realidade à demarcação do
verdadeiro e falso, forjando formas de objetividade cuja consistência
ontológica se estabeleceu historicamente. Uma análise histórica de
dispositivos permite mapear inclusive, na capilaridade das razões que
regem as formas de controle dos cortiços, os próprios limites de poder
da Junta, ou a medida da impotência política frente às demandas teóricas
que a Higiene confia a si mesma.\footnote{Em 1864, achava-se
  estabelecida na Praia Formosa, nº 73, uma fábrica de cola e carvão
  animal pertencente à \textit{Blanc \& Cia}, da qual se queixaram os
  moradores de Santana pelo mau cheiro que dela constantemente se
  exalava. Também um Subdelegado da freguesia de Santana denunciou o
  atentado à salubridade naquele lugar, alertando à Junta ser
  conveniente a remoção de tal estabelecimento. Em ofício de 19 de março
  daquele ano, Pereira Rego escreve ao Conselheiro Dr. José Bonifácio
  d'Andrade e Silva, Ministro dos Negócios do Império. Ele diz:
  ``conquanto o regulamento da Junta a autoriza pelo artigo 49 a fazer
  remover tais estabelecimentos, quando nas condições deste, para fora
  dos povoados, entendeu todavia ela que nada deveria resolver sem
  submeter este negócio à consideração do Governo uma vez que o dono do
  estabelecimento apresente uma licença da Ilma. Câmara Municipal que
  lhe concedeu montar naquele lugar a sua fábrica''. Cria-se um impasse
  entre o direito do proprietário da fábrica --- garantido por uma
  licença concedida pela Câmara --- e os preceitos da higiene pública ---
  igualmente regidos por lei, e empregados pelos encarregados da Junta e
  seu presidente. Apesar das razões jurídicas que dão caução à licença
  concedida pela Câmara, há razões científicas que a Junta não pode
  dissimular, tendo em vista os prejuízos que a fábrica traz para os
  moradores de Santana. No entanto, e assim termina o ofício dirigido ao
  Ministro, ``não pode a Junta, do modo como se acham as coisas,
  promover a execução do seu regulamento \textit{sem provocar não só
  conflito entre as suas atribuições e as da Ilma. Câmara, mas ainda
  questões de outra ordem que a Junta não deseja motivar}.'' Arquivo Nacional. 
  MAÇO IS 4-25, Série Saúde --- Higiene e Saúde Pública ---
  Instituto Oswaldo Cruz, p.~14.}

O cortiço põe em risco a monumentalidade da cidade, seu estado
sanitário, a ordem pública. As demolições dos cortiços também põem em
risco o direito liberal à cidade, porque infringem o caráter
mercadológico da cidade, a valorização mercantil do solo urbano, o
direito burguês à reificação da moradia popular. Autores como Maurício
de Abreu e Jaime Benchimol escreveram sobre como as grandes intervenções
urbanísticas do entresséculos serão oportunas na habilidade de articular
forças destoantes sob o manto de um \textit{interesse público maior}, o
capital. Quando, 30 anos mais tarde, a exploração direta do cortiço se
tornasse insustentável, desfrutariam seus proprietários de ganhos
especulativos com os terrenos localizados nas áreas que seriam atingidas
pelas obras e que, portanto, tenderam a se valorizar. ``Caso o imóvel'',
nas imediações do centro ou zona sul, ``fosse desapropriado --- como na
época de Pereira Passos --- podia obter uma indenização vantajosa, sob a
forma de dinheiro ou então de títulos que lhe asseguravam a permanência
de sua renda''.\footnote{Jaime Larry Benchimol, \textit{Pereira Passos: um
  Haussmann tropical}, 1992, p.~135.} Mas por ora ainda nos situamos em
um campo de disputas abertas, de onde começam a emergir a identidade do
sujeito suspeito e do corpo anti-higiênico (assim como em outro tempo ---
da composição de outras tecnologias de poder e outros regimes de
veridicção --- se assistirá ao nascimento do indivíduo favelado).

\textit{O Cortiço} (1890) de Aluísio de Azevedo é um salpicado de
personagens que oscilam de uma habitação coletiva a outra, driblando a
rota das demolições e despejos da segunda metade do \textsc{xix}. Não há de fato
menção à instituição de higiene pública, mas sim um protagonista
suspeito de febre amarela, dois hóspedes varridos pela epidemia enquanto
``três outros italianos estiveram em risco de vida''.\footnote{Aluísio
  Azevedo, \textit{O cortiço} {[}1890{]} (São Paulo, Círculo do Livro,
  n/d), p.~145.} Aluísio narra histórias de trabalhadores pobres, alguns
miseráveis, amontoados em uma habitação coletiva. Pululam grupos de
lavadeiras e fainas, uma escravizada de ganho, a curandeira, a visita de
uma meretriz, famílias de imigrantes, pagodeiros e capoeiras,
trabalhadores de pedreira, o pequeno burguês, os hóspedes de um cortiço
rival, o Cabeça-de-Gato (cujo ``legítimo proprietário era um abastado
conselheiro, homem de gravata lavada, a quem não convinha, por decoro
social, aparecer em semelhante gênero de especulações'').\footnote{\textit{Ibidem},
  p.~145.} Poucos são os motivos e pormenores que permitiriam pensar a
experiência da epidemia no quadro do romance, e pouco importa na
verdade. Salvo engano, a ocasião em que, em uma roda de italianos, se
comia tagarelando e lançando ``à porta da casa uma esterqueira das casas
de melancia e laranja''. Pelo que os amaldiçoa o vendeiro e dono do
cortiço: ``Tomara que a febre amarela os lamba a todos!''.\footnote{\textit{Ibidem},
  p.~116.}

No centro da narrativa, um rega-bofe triunfal que servirá de ocasião
para o encontro dos futuros amantes, e que determina o curso ulterior da
trama. Acontece uma luta entre dois rivais, um negro brasileiro com sua
navalha, um cavouqueiro português e seu varapau minhoto. O salseiro
termina com um ventre rasgado, o capoeira fugido, a multidão em êxtase e
a chegada dos urbanos.

\begin{quote}
A polícia era o grande terror daquela gente, porque, sempre que
penetrava em qualquer estalagem, havia grande estropício; à capa de
evitar e punir o jogo e a bebedeira, os urbanos invadiam os quartos,
quebravam o que lá estava, punham tudo em polvorosa. Era uma questão de
ódio velho.\footnote{\textit{Ibidem}, p.~124.}
\end{quote}

Enquanto se precipitam apitos da polícia e as espadeiradas que abalam o
portão da estalagem, o cortiço empunhava sobras de lenha e varais de
ferro para defender ``a comuna, onde cada um tinha a zelar por alguém ou
alguma coisa querida''.\footnote{\textit{Ibidem}, p.~124.} As mulheres
rolam as tinas, arrastam móveis, restos de colchões e sacos de cal
formando uma barricada ``numa solidariedade briosa, como se ficassem
desonrados para sempre se a polícia entrasse ali pela primeira
vez''.\footnote{\textit{Ibidem,} p.~124.} O portão se rompe, entram
urbanos recebidos a pedradas e garrafadas. Já se desbaratava o inimigo
quando, sem que ninguém saiba de onde, irrompe um incêndio para lamber
em chamas as 100 casinhas da estalagem de Botafogo.

\begin{quote}
Fez-se logo medonha confusão. Cada qual pensou em salvar o que era seu.
E os policiais, aproveitando o terror dos adversários, avançaram com
ímpeto, levando na frente o que encontraram e penetrando enfim no
infernal reduto, a dar espadeiradas para a direita e para a esquerda,
como quem destroça uma boiada. A multidão atropelava-se, desembestando
num alarido. Uns fugiam à prisão; outros cuidavam em defender a casa.
Mas as praças, loucas em cólera, metiam dentro as portas e iam invadindo
e quebrando tudo, sequiosas de vingança.\footnote{\textit{Ibidem}, p.~125.}
\end{quote}

Talvez a força do livro venha da fácil sedução por uma saída
interpretativa, que é em parte a contaminação do plano alegórico pela
ordem do real com temperos de uma poética da animalidade tropical.
Tomado este caminho, rapidamente reforçamos uma sequência organizada por
Aluísio no romance, um mecanismo já outras vezes repisado no dispositivo
médico-higienista, como segue exemplo de um artigo dos \textit{Annaes
Brasilienses de Medicina} em 1872: ``O Rio de Janeiro, bem observado, é
semelhante a um vasto cortiço, onde se abrigam classes de imigrantes
imundos e ignorantes. Contra os erros desta classe, que (\ldots{})
também conta com alguns ricos e poderosos, que pode fazer a classe
inteligente e culta deste país?''\footnote{``Concorrerá o modo por que
  são dirigidas entre nós a educação e instrução da mocidade, para o
  benéfico desenvolvimento físico e moral do homem?'', \textit{Annaes
  Brasilienses de Medicina} --- Tomo \textsc{xxiii}, n.~11, abril de 1872, p.~18.}
O artifício é sobrepor o movimento da narrativa ao movimento social para
fazer girar a roda da representação do cortiço como ``ao mesmo tempo um
sistema de relações concretas entre personagens e uma figuração do
próprio Brasil''.\footnote{Antonio Candido, ``De cortiço a cortiço'',
  \textit{Novos Estudos CEBRAP}, n.~30, julho de 1991, p.~121.} O cortiço,
como o próprio Rio de Janeiro, pode tomar assim a forma de bando,
``aglomeração tumultuosa de machos e fêmeas'',\footnote{Aluísio Azevedo,
  \textit{O cortiço} {[}1890{]}\textit{,} p.~76.} ``coisa viva, uma geração,
que parecia brotar espontânea, (\ldots{}) daquele lameiro, e
multiplicar-se como larvas no esterco''.\footnote{\textit{Ibidem}, p.~26.}
Dia após dia socando-se gentes ``com aquela exuberância brutal de
vida''.\footnote{\textit{Ibidem,} p.~26} Um Rio de Janeiro em miniatura,
por ser uma mistura de raças, cores, sóis e raízes na conivência
democrática com a informalidade e a promiscuidade que imperam nessas
famílias. Inversamente, sendo o Rio de Janeiro um cortiço em maior
escala, acotovelam-se brancos, negros, mulatos, todos igualmente
selvagens no trato, sujeitos à exploração de um português em ascensão.
Tal como em Antonio Candido: ``em nenhum outro romance do Brasil tinha
aparecido semelhante coexistência de todos os nossos tipos raciais,
justificada na medida em que assim eram os cortiços e assim era o nosso
povo''.\footnote{Antonio Candido, ``De cortiço a cortiço'', 1991,
  p.~120.}

É natural, por conseguinte, que civilizar o cortiço seja um pouco
civilizar nosso povo, policiá-lo, dar a ele educação, porque tanto o
povo quanto o cortiço são um mingau de existências aglomeradas, mosto
fermentado de uma coisa ambígua, sujeito fronteiriço entre trabalho e
rua. O povo, com seus folclores e sua riquíssima diversidade, é o não
Eu, e, como o gado, é também o povo um anônimo. Ou seja, Firmo, o
capoeira da navalha, é um negro, antes mesmo de ser o Firmo músico em
horas vagas. Jerônimo é imigrante português antes de ser Jerônimo. No
limite, todos os moradores da ``Estalagem de São Romão'', antes de serem
diferença (ou, como ensina Foucault, antes de cada identidade
projetar-se como ``diferença das máscaras'')\footnote{Michel Foucault,
  \textit{A arqueologia do saber} {[}1969{]}, 2008, p. 149.} são
\textit{moradores de cortiço}.

Já na montagem do elenco do romance orienta-se a justificar o
desnorteamento da metralhadora punitiva dos urbanos, porque o cortiço ---
e estas não são mais expressões de Aluísio, muito embora aparentem ---
existe na sobreposição da ``miséria andrajosa e repugnante (\ldots{}) ao
lado do vício e do lodaçal impuro do aviltamento moral'', vizinha ``do
leito do trabalhador honesto, que respira à noite a atmosfera deletéria
deste esterquilíneo de fezes!''\footnote{Cândido Barata Ribeiro,
  \textit{Quais as medidas sanitárias\ldots{}}, 1877, p.~96.} A violência
policial logo aí toma lugar e se justifica, pode aí se exercitar sem que
se comprometa, seja para o leitor do romance, seja para a polícia, a
ambiguidade do sujeito fronteiriço, trabalhador e vagabundo, homem e não
homem. É na ambiguidade do sujeito fronteiriço, produção pioneira de um
dispositivo médico-higienista, que se aloja a tecnologia da barbárie na
repressão policial. O dispositivo fornece a brecha identitária de que
carece a violência difusa da polícia tal como ilustrada em \textit{O
cortiço}. No limite, Aluísio esculpe a face grotesca do poder policial,
uma face dentre tantas outras, ao mesmo tempo em que adere e dispõe as
condições de possibilidade para a seletividade da violência que a
polícia distribui sem semblantes de reverência. Bem entendido, é uma
violência difusa, na medida em que se aloja na ambiguidade identitária
do morador de cortiços, mas é também seletiva no plano do território,
porque assenta o cortiço, alvo desses investimentos, na fronteira entre
a cidade e seu limite, a cidade e a não cidade. O que se chamou
\textit{naturalismo} em \textit{O cortiço} retroalimenta ou pelo menos tem a
ver com essa hipótese.

O que não deixa de possibilitar excessos na leitura social que Aluísio
introduz do conflito entre cortiço e autoridade. Ele coloca na boca de
uma personagem anônima, interrogada pelo subdelegado, uma fala
intencionalmente anacrônica: ``Os rolos era sempre a polícia quem os
levantava com as suas fúrias! Não se metesse ela na vida de quem vivia
sossegado no seu canto, e não seria tanto barulho!''\footnote{Aluísio
  Azevedo, \textit{O cortiço} {[}1890{]}, p.~127.} O que sustenta a fala é
a perspectiva de que a polícia concentra suas energias em produzir medo.
Um afeto de medo justificando uma política de segurança pública, a
política de segurança pública sabendo reinventar o medo. Medo se
reinventando por outros meios, dentro de um mesmo ciclo, quantas vezes
for necessário, para marcar nos corpos as cicatrizes dessa memória. Eis
um debate que teria lugar de sobra nas circunstâncias que dão ensejo às
políticas de segurança no Rio de Janeiro do século seguinte. Mas talvez
não naquelas circunstâncias. O poder de polícia não se justificava por
ele mesmo: havia que dar suas razões. Não era um poder que se
autoalimentava e autorreproduzia, suas aberrações não se restringiam em
produzir afetos de medo nem se justificavam apenas na seguridade da
propriedade privada. De fato, o poder nos tempos áureos de práticas
higienistas é um poder que produz sujeitos, e não um poder tirano,
destrutivo e fora de si.

O poder de polícia fez parte de uma rede de saberes e poderes que
produziram nosso ``sujeito fronteiriço'', porque morador de cortiços,
porque amálgama de corpo anti-higiênico e criminoso em potencial. A rede
complexa, chamada dispositivo, integrou uma ampla e múltipla equipe de
urbanistas, médicos, policiais e políticos, tal como até aqui venho
tentando pensar. Mas ``nenhum ramo das ciências médicas abrange uma
série tão complexa de estudos como a higiene''.\footnote{``Discurso
  pronunciado pelo Exmo. Sr.~Conselheiro Dr.~José Pereira Rego na sessão
  solene da Academia Imperial de Medicina em 30 de Junho de 1871'',
  \textit{Annaes Brasilienses de Medicina} --- Tomo \textsc{xxiii}, n.~2, julho de
  1871.} E sem dúvidas os higienistas tiveram um papel mais ou menos
notável porque forneceram enunciados que transitaram em tantas mesas de
discussão, assim como souberam exportar valores científicos que aderiram
com facilidade às urgências daquela conjuntura social e política. A
identificação do sujeito suspeito com o anti-higiênico é veementemente
repetida na atmosfera acadêmica daquelas décadas. Basta folhear um
periódico importante, os \textit{Annaes Brasilienses de Medicina} --- que
reproduzia as atas das sessões da AIM, onde também escrevia sua elite
médica ---, para que se presuma que o pensamento e a prática de Barata
Ribeiro tiveram consentimento e cumplicidade dentro ou fora dos
circuitos acadêmicos.

A febre amarela de 1873 custou 3.659 vidas; somente no mês de janeiro,
foram 949 mortos. Em fins de janeiro o presidente da Academia --- também
presidente da Junta, Dr.~José Pereira Rego --- convoca sessão
extraordinária na AIM e expõe o quadro de uma epidemia que acredita ser
a ``mais grave e mortífera das que já no país tem havido, e que portanto
reclama altamente a atenção e os cuidados não só da Academia, senão
também de toda a corporação médica e os de todos que se interessam pelo
bem e salvação da humanidade''.\footnote{``Sessão geral extraordinária
  em 27 de janeiro de 1873 da Academia Imperial de Medicina'',
  \textit{Annaes Brasilienses de Medicina} --- Tomo \textsc{xxv}, n.~1, junho de
  1873, p.~4-6}

Os membros da Academia não se fazem muito solícitos em prestar socorros,
em retaliação ao papel decorativo que assumira essa corporação médica
frente ao ``ápice da pirâmide'' política (segundo expressão de um
titular, garfando o governo e suas instituições de saúde). A esse
respeito o presidente da Junta elenca oito medidas que vinham sendo
tomadas no calor da hora: irrigação das ruas, retiradas dos imigrantes e
navios para fora do circuito da cidade, nomeação de comissões médicas
para acudirem amarelentos etc. Entre as medidas avaliadas por Pereira
Rego, ou seja, listada ao lado de outras talvez tão importantes quanto,
figura a ``inspeção dos cortiços existentes, sua remoção ou diminuição
do número dos moradores''. O que vemos, entretanto, do início ao fim da
sessão aberta é uma troca de tiros girando em torno de um único tema:

\begin{quote}
Quantas e quantas vezes se tem ventilado entre nós questões de higiene;
quantas e quantas vezes as nossas vozes se tem erguido para aconselhar,
para pedir providências, para chamar a atenção dos que governam sobre
pontos de palpitante interesse público, e tudo, tudo isso em vão?
Profligamos essas casas imundas, insalubres, chamadas cortiços,
verdadeiros focos de infecção permanente e essas espadas percucientes
continuam como dantes a ameaçar os habitantes da capital!\footnote{\textit{Ibidem},
  p.~11-2.}
\end{quote}

O Dr.~Ataliba de Gomensoro sugere a disseminação dos moradores do bairro
infectado e que se restrinja o número de pessoas em cada cortiço. É caso
de ``entrar nesses imundos cortiços nessas tascas em que habitam dúzias
de indivíduos quando a cubagem do ar é apenas suficiente para seis ou
oito, e disseminá-los, espalhá-los, obrigá-los a respirar um ar
suficiente''. Medida enérgica, de ``execução um pouco difícil'', mas em
casos de gravíssimo perigo os poderes habilitados ``devem ser
absolutos''. O Dr.~Peçanha da Silva aplaude o projeto, mas indaga onde
iríamos depositar a pobreza; ao lado dele um terceiro doutor reforça a
fala garantindo que a causa da epidemia e do ``grande destroço que vem
fazendo, são principalmente os cortiços que existem na cidade em grande
número, e com aglomerações de moradores''. Ao fim, o presidente retoma a
sessão. Se nos acostumamos com o tom incisivo do presidente da Junta
cobrando efetividade da Câmara (exigindo seriedade ``para o assunto de
que se trata, um dos mais importantes da Higiene Pública desta Corte, e
solicitar com instância a sua severidade na concessão de licenças da
ordem desta pelos males incalculáveis que à saúde pública desta cidade
tem trazido a edificação de cortiços'')\footnote{BR RJAGCRJ 43.1.25 ---
  Fundo Câmara Municipal --- Série Cortiços e Estalagens, sem paginação;

  AGCRJ Códice 43.1.25 --- Estalagens e Cortiços --- Requerimentos e
  outros papéis relativos à existência e à fiscalização sanitária e de
  costumes dessas habitações coletivas --- 1834-1880, p.~59.} encontramos
agora a voz comedida e fios soltos de humanismo no trato com os pares
acadêmicos. Responde dizendo que a prática do despejo seria por certo de
suma utilidade, mas

\begin{quote}
para onde se mandariam mais de vinte mil pessoas que neles habitam? Quem
as sustentaria, quem as guardaria para não fugirem e voltarem aos focos
de infecção? Quem pagaria os salários reclamados, uma vez que terão
saído contra sua vontade de seus domicílios? Quantas habitações para
acomodar toda essa gente nas condições em que vivemos? Não se tratava de
duas ou três mil pessoas, mas de número superior a vinte mil, às quais
se devia dar aposento e comida (\dots{}). Já vê portanto a Academia, que a
Junta de Higiene não iria propor uma medida que ela sabia que não podia
ser executada.\footnote{``Sessão geral extraordinária em 27 de janeiro
  de 1873 da Academia Imperial de Medicina'', \textit{Annaes Brasilienses
  de Medicina} --- Tomo \textsc{xxv}, n.~1, junho de 1873, p.~29-30.}
\end{quote}

Não surpreende que frente à corporação médica a Junta barganhe sua
impotência política na condução dos temas da saúde pública, tendo em
vista seu fracasso em frear a grande mortalidade. Não surpreende também
que justifique o rigor pouco satisfatório na execução das demolições e
remoções com mostras de humanismo. Os livros de Pereira Rego que tratam
dos mesmos temas ritualizam a comoção com a pobreza da mesma maneira,
uma bandeira que definitivamente tem pouco espaço em seus documentos
oficiais na administração.

Para o Dr.~Luiz Corrêa de Azevedo, também da AIM, ``uma liberdade
exagerada no viver e fazer deste município''\footnote{``Os esgotos da
  cidade do Rio de Janeiro (City Improvements) pelo Sr.~Dr.~Luiz Corrêa
  de Azevedo''. \textit{Annaes Brasilienses de Medicina} --- Tomo \textsc{xxv},
  n.~10, março de 1874, p.~12.} é a raiz e o princípio das calamidades.
Esse excesso de liberdade na maneira de viver fez rangerem soluções de
gabinetes médicos pela organização de um plano de casas para pobres,
``construídas sob todos os preceitos higiênicos e colocadas fora do
coração da cidade''.\footnote{Cândido Barata Ribeiro, \textit{Quais as
  medidas sanitárias\ldots{}}, 1877, p.~97.} O plano de casas para
pobres (não se poderia esperá-lo surgir das mãos de particulares, mas
sob ordens municipais) deveria vir da aplicação de um regulamento
especial, controlado pela mais restrita vigilância, eliminando o risco
de se reproduzir a já conhecida inoperância da legislação ativa que
regulamentava construções privadas. Um simples projeto para
regulamentação da construção das estalagens e cortiços, como aquele
proposto por Pereira Rego em 1866, deixa de ser suficiente. De pouca
valia é se pensar em ``construção das casas, das ruas, das praças, dos
hotéis, das estalagens para imigrantes com inteira liberdade do
capricho, do mau gosto, e do egoísmo de cada um''.\footnote{``Concorrerá
  o modo por que são dirigidas entre nós a educação e instrução da
  mocidade, para o benéfico desenvolvimento físico e moral do homem?''
  (questão imposta pela mesma Academia em 21 de dezembro de 1871,
  desenvolvida e respondida pelo seu membro titular Luiz Corrêa de
  Azevedo), \textit{Annaes Brasilienses de Medicina} --- Tomo \textsc{xxiii}, n.~11,
  abril de 1872, p.~24.} O resultado seria novamente a proliferação de
lares anti-higiênicos e antiestéticos, porquanto se trata de uma
população que subsiste na ``falta absoluta de conhecimentos de higiene
do corpo, na educação do povo''.\footnote{``Os esgotos da cidade do Rio
  de Janeiro (City Improvements) pelo Sr.~Dr.~Luiz Corrêa de Azevedo'',
  \textit{Annaes Brasilienses de Medicina} --- Tomo \textsc{xxv}, n.~10, março de
  1874, p.~12.}

Uma liberdade exagerada no viver e fazer não será governada com leis que
estipulem proporções entre o alargamento das ruas e a forma das
fachadas. Presidiria ainda a mais obscura ignorância e desmazelo nos
repartimentos internos. Porque ``uma casa é para eles um objeto sem
grandes reclamações: basta-lhes paredes, telhados, cubículos, áreas, e
uns corredores, para considerarem-se servidos do lado da
propriedade''.\footnote{``Concorrerá o modo por que são dirigidas entre
  nós a educação e instrução da mocidade, para o benéfico
  desenvolvimento físico e moral do homem?'', \textit{Annaes Brasilienses
  de Medicina} --- Tomo \textsc{xxiii}, n.~11, abril de 1872, p.~24.} E, afinal, a
um proprietário ou construtor de habitações para o povo, ``que lhe
importa que um imundo cortiço, que lhe dá interesse, dê ao município
moléstias, miséria, crápula, o roubo e a imoralidade
revoltante?''\footnote{\textit{Ibidem,} p.~23.}

Todas essas considerações reclamam pronto remédio. À Junta caberá máxima
atenção seja à higiene, seja à moralidade. Nenhuma delas sobrepuja a
outra, na medida em que visto sob o plano da higiene ``o cortiço é um
foco de infecção; em frente da moral é um escândalo público''.\footnote{Cândido
  Barata Ribeiro, \textit{Quais as medidas sanitárias\ldots{}}, 1877,
  p.~96-7.} Compreende-se logo onde se situa o projeto que se prepara
para os cortiços. Lembremos: o que definha nos cortiços são ``pilhas de
corpos humanos''. Além de todas as funções orgânicas dos anônimos que o
povoam, no cortiço lava-se, cozinha-se, criam-se aves, condições o
bastante para que a cidade adoeça das ``emanações de centros aglomerados
de homens que economizam até na limpeza do corpo''.\footnote{``Concorrerá
  o modo por que são dirigidas entre nós a educação e instrução da
  mocidade, para o benéfico desenvolvimento físico e moral do homem?'',
  \textit{Annaes Brasilienses de Medicina} --- Tomo \textsc{xxiii}, n.~11, abril de
  1872, p.~17-9.} Para o cortiço, lodaçal da imoralidade, foco daquela
insalubridade produzida por aglomerações anônimas, para o cortiço ---
conclui Barata Ribeiro --- ``só vemos um conselho a dar a respeito: a
demolição de todos eles, de modo que não fique nenhum para atestar aos
vindouros e ao estrangeiro, onde existiam as nossas sentinas sociais, e
sua substituição por casas em boas condições higiênicas''.\footnote{Cândido
  Barata Ribeiro, \textit{Quais as medidas sanitárias\ldots{}}, 1877,
  p.~96-7.}

A irrupção do cortiço como campo de intervenção do dispositivo
higienista significa, dentre outras coisas, que a Junta não limita suas
atividades ao indivíduo sadio/amarelento. O corpo anti-higiênico como
objeto, diferente do corpo doente, é muito mais rico e difícil, ricopois
envolve séries de variáveis que extrapolam o nível da doença, difícil
porque, na cabeça de Barata Ribeiro, são ``pilhas de quartos e pilhas de
corpos humanos!''\footnote{\textit{Ibidem}, p.~96-7.} Atrela-se imperiosa
e essencialmente o indivíduo à fronteira onde vive, definindo-se então
em um ato a virtualidade do que se é, do que se quer ser, do que se é
suspeito de ser. Um território novo, espaço-fronteira onde se dispõe um
agregado anônimo de corpos. Aí se extrapola a materialidade do cortiço
em sua função de moradia para atingir outra ordem de realidade. Algo que
não existia ganhou um vocabulário, ganhou critérios e desbloqueou a
invenção de identidades.

Os moradores dos cortiços pertenciam a um estrato de definição difícil.
Dir-se-ia em princípio se tratar de uma população ativamente produtiva,
em sua maioria constituída por ``artistas'' (38\% desse contingente):
costureiras, alfaiates, cigarreteiras, barbeiros, sapateiros,
cavouqueiros etc. Desempregados participavam com 21\% do
total.\footnote{Cf. Manuel C. Teixeira, ``A habitação popular no século
  \textsc{xix} --- características morfológicas, a transmissão de modelos: as
  ilhas do Porto e os cortiços do Rio de Janeiro'', \textit{Análise
  Social}, vol.~\textsc{xxix} (127), 1994 (3º), p.~576.} Pereira Rego estipula
uma população de cortiço de 20 mil pessoas em 1873, integrantes de uma
``classe do povo menos favorecida da fortuna'',\footnote{AGCRJ Códice
  41.3.36 --- Fundo Câmara Municipal --- Série Cortiços e Estalagens.}
aquela justamente que não poderá arcar com moradia particular. Mas em um
ofício de 19 de março de 1860\footnote{AGCRJ Códice 6.1.37 --- Escravos /
  Assuntos: Casas alugadas ou sublocadas a escravos, muitos dos quais
  fugidos e malfeitores --- Ofício do Chefe de Polícia, 1860.} o chefe de
polícia denuncia existir na cidade ``um grande número de casas alugadas
diretamente a escravos, ou a pessoas livres, que parcialmente as
sublocam a escravos''.

Tornar-se-ia urgente, ao modo de fazer da polícia, reprimir semelhante
abuso, ``proibindo-se alugar, ou sublocar qualquer casa, ou parte dela a
escravos'', mesmo munidos de autorização dos senhores para esse fim.
Certamente viriam a se tratar de cortiços, conquanto o ofício
qualificava as casas como ``verdadeiras espeluncas, onde predominam o
vício, e a imoralidade debaixo de mil formas diferentes'', valhacoutos
de ``escravos fugidos e malfeitores''. Sabemos pelo historiador Luiz
Carlos Soares que, entre os cativos, somente escravizados de ganho
conseguiam alojamento em habitações coletivas, na medida em que podiam
arrecadar quantias suficientes para ``viverem por si'', isto é, arcar
com alimentação e aluguéis sem interferência direta dos senhores. Os que
``não tinham condições para conseguir um lugar para morar, simplesmente
dormiam ao relento, pelas ruas, praças e, os mais afortunados, em seus
locais de trabalho. Obviamente (\ldots{}) sujeitos à prisão pela
Polícia, pois era proibido que `andassem fora das horas'\,''.\footnote{Luís
  Carlos Soares, \textit{O ``Povo de Cam'' na Capital do Brasil}, 2007,
  p.~98.} De um modo ou de outro, o tempo dos cortiços no Rio de Janeiro
foi o tempo da intensificação das lutas dos escravizados pela libertação
e também foi contemporâneo da massiva iniciativa estatal para realizar a
imigração de europeus. É possível a partir daqui começar a extrair
algumas conclusões.

Em 5 de maio de 1869, a coluna de ``publicações a pedido'' do
\textit{Jornal do Commercio} traz uma breve carta dirigida ao chefe da
polícia: ``Pedimos a S. Ex. mandar dar busca em vários cortiços da
Corte, que se estão transformando em asilo de escravos fugidos''.
Segundo os trabalhos de Sidney Chalhoub em \textit{Visões de Liberdade},
tornou-se uma evidência, nas últimas décadas da escravidão na Corte,
``que os escravos vivendo `sobre si' contribuíam para a desconstrução de
significados sociais essenciais à continuidade da instituição da
escravidão''.\footnote{Sidney Chalhoub, \textit{Visões de liberdade},
  1990, p.~235.} Quer dizer: se o cativeiro é definido como uma relação
de sujeição e dependência pessoal, é razoável que se presuma naquele que
vive ``sobre si'', e consegue autorização para ter autonomia nos seus
modos de habitar, se portar, circular e vender trabalho, um foco de
resistência à infraestrutura de uma sociedade escravista. Ademais, é
fundamental que a autonomia para viver de forma desassistida, alheio ao
comando do teto de um senhor, significasse para o escravizado a
diminuição das chances dos açoites ou, no limite, uma chance de escapar
da possibilidade que porventura se apresentasse de ser objeto de
transações de venda. O acirramento do controle da realidade urbana
assume para nós uma contrapartida do dispositivo médico-higienista à não
aceitação dessa estratégia política microfísica --- que são os cortiços
--- de antecipação da ruptura com um presente escravista. A vida no
cortiço, em poucas palavras --- e aqui puxamos os fios que nos
interessarão no próximo capítulo ---, é um horizonte de experiência
clandestina para a projeção de uma existência futura não escravizável.

\begin{quote}
São vários os exemplos de escravos que moravam em cortiços, ou que
tinham suas amásias morando em cortiços; além disso, encontram-se
famílias de ex-escravos que conseguiam se reunir e passar a morar juntos
em habitações coletivas após a liberdade. Com frequência, era nestas
habitações que os escravos iam encontrar auxílio e solidariedade
diversas para realizar o sonho de comprar a alforria a seus senhores; e,
é claro, misturar-se à população variada de um cortiço podia ser um
ótimo esconderijo, caso houvesse a opção pela fuga.\footnote{Sidney
  Chalhoub, \textit{Cidade febril}, 1996, p.~28-9.}
\end{quote}

As tarefas higienistas de embelezamento do espaço urbano, de eliminação
dos focos de infecção e restabelecimento da moralidade passam pela
normatização dessa \textit{liberdade exagerada no viver} --- a vida de
cortiço ---, na medida em que são os cortiços um perigo aos serviços,
procedimentos e relações sociais necessárias para a manutenção da cidade
escravocrata. Como dizíamos no início, o antigo pavor diante da
virtualidade de um levante de cativos se redefinirá na habitação da
massa da população produtiva desassistida. Mais ainda: além de defender
a sociedade contra a extinção dos valores morais e estéticos que
naturalizaram a escravidão sob um racismo de Estado, é preciso defender
a sociedade contra os riscos virtuais de uma conflagração anárquica
contra séculos de cativeiro. As gerações de distanciamento social, o
sofrimento provocado pela diáspora, pela exacerbação do racismo ilegal e
pela consciência castigada com injustiças diárias bem poderiam eclodir,
amanhã, em convulsões anárquicas que conflagrariam toda uma sociedade.

Não é fortuito o nascimento do cortiço ser posterior à abertura de
possibilidades que se seguem ao fim do tráfico --- ainda que os debates
que culminam na lei de 28 de setembro de 1871 (que aboliu o direito dos
senhores escravizarem os filhos das cativas) tenham levado a uma redução
da população escravizada urbana no município. Essa redução também pode
ser atribuída à alta taxa de mortalidade em decorrência da febre
amarela, ou ao movimento de alforrias, mas sobretudo ao grande volume de
venda interna de escravizados para preenchimento de postos em fazendas
do sudeste cafeeiro. Mesmo assim, dos 228.743 habitantes das freguesias
urbanas e suburbanas da cidade, 37.567 eram africanos escravizados
(16,42\%) e 191.176 eram livres. De acordo com o recenseamento de 1872,
48,5\% dessa população livre eram imigrantes europeus, ou seja, 69.661
residentes na Corte ou eram imigrantes europeus, ou imigrantes africanos
libertos. Os estrangeiros eram em sua grande maioria portugueses, de
modo que, em fins do \textsc{xix}, a população do Rio de Janeiro já seria
superior à população do Porto. ``Imigrantes estrangeiros correspondiam a
30,6\% da população da cidade em 1872 e a 24,8\% em 1900. Aos
portugueses correspondia uma parte importante (\dots{}): no final do século
os portugueses eram cerca de 20\% da população total da
cidade''.\footnote{Manuel C. Teixeira, ``A habitação popular no século
  \textsc{xix}'', 1994, p.~570.} Apesar de ser a única forma de habitação
disponível para portugueses recém-imigrados e negros em suas práticas
silenciosas de sublevação da instituição escravocrata, os cortiços foram
instrumentos de \textit{segregação social} para os novos imigrantes, e
instrumentos de \textit{segregação social e racial} para os africanos
alforriados. Portanto, tendemos a pensar que a vida na habitação
coletiva estaria carregada de pesos e valores de gravidade inteiramente
distintos para os grupos em evidência.

Partimos de perguntas que talvez soassem estranhas à matriz de
preocupações com as quais nos ocupamos no primeiro capítulo: o que
permitiu que o traçado urbano herdado do período colonial fosse
considerado insalubre pelo discurso higienista tardio? Como nasce a
relação de contiguidade entre a valoração negativa das ruas estreitas e
sinuosas e o diagnóstico sobre a insalubridade, sobre o mau cheiro da
cidade? Propomos, em resposta, que não foram os higienistas que lançaram
as bases de percepção da rua como causa intransigente da uma conivência
com mau cheiro da cidade. Logo, narramos de que modo o \textit{a priori}
histórico que ativou a sensibilidade para a condição defeituosa da
cidade de S. Sebastião já estaria sendo preparado para que o discurso
higienista objetivasse a cidade como um campo potencial de intervenção.
No máximo, a importância preliminar do dispositivo médico-higienista foi
ter institucionalizado, circulado um ramo de competências em torno da
rua e do cortiço como formas de problematização quase que exclusivas no
contexto das epidemias de febre amarela da década de 1870. Sem dúvidas,
as práticas e os enunciados do dispositivo médico-higienista ajudaram a
compor categorias por meio das quais a cidade pôde ser
\textit{experienciada} de certa maneira. Mas talvez sua maior conquista
tenha sido de fato as atribuições para que se forjasse a identidade do
``morador de cortiços'' alinhado com a imoralidade, o antiestético e o
anti-higiênico. Os modos de investimento do dispositivo
médico-higienista combinaram-se e se alternaram em um tensionamento de
vários objetos, mas, parece-nos, seus grandes êxitos tiveram a ver com
as estratégias de investimento e as tentativas de normatização de
hábitos e costumes que não aderiam às exigências burguesas de um
convívio social policiado. E, claro, nada disso teria sido possível sem
que \textit{experiência da epidemia} de febre amarela aparecesse, a partir
de meados das décadas de 1860-1870, mediatizada por problematizações dos
elementos materiais que definem a realidade urbana. A produtividade
desse dispositivo de saber-poder teve a ver com a habilidade para
instaurar no real um regime de veridicção que permitiu deslocar as
problematizações sobre a epidemia de febre amarela para enunciados sobre
a boa ou a má cidade, a boa e a má rua, a verdadeira moradia e a moradia
inadequada, a saúde e a epidemia. Só assim começa a se apagar lentamente
a ligação da epidemia com causas acidentais e passageiras, como
condições atmosféricas, meteorológicas etc. A epidemia ganha causas
históricas, imanentes e urbanas, sua ação se torna presente e
permanente, sua terapêutica assume uma ordem normativa e
individualizante.

As coisas em um determinado momento se bifurcam. Combater a
insalubridade das habitações coletivas e reinventar a trama viária da
cidade são duas intenções que se cruzam, que convivem em um mesmo
projeto, que estabelecem um jogo de mutualidades, mas que praticam
estratégias de poder distintas e se estabelecem em momentos diferentes
da história da cidade.

O plano de erradicação dos cortiços envolveu o emprego de uma dinâmica
de forças permanente sobre um registro setorizado da população, isto é,
exigiu uma tecnologia de poder mais militarizada. As relações de poder
que investiram os cortiços desde a década de 1870 foram pautadas pela
divisão, pela interdição, pelo imperativo de isolar, de despejar,
desocupar, constituindo por isso um bloco, um tipo estratégico cuja
lógica flertou com a violência e cujo alvo foi um setor específico da
população da cidade (as referidas aglomerações anônimas e difusas: os
negros e os imigrantes recentes). Para que isso fosse um dia possível, o
dispositivo médico-higienista se concentrou em tornar o cortiço e seus
integrantes uma ameaça ao estado sanitário da cidade, à beleza ofuscante
do nosso litoral, à moralidade dos costumes, à segurança de uma
sociedade de estrutura escravocrata. Nasce daí esse sujeito fronteiriço:
o \textit{morador de cortiços}, que habita o limiar da cidade e da não
cidade, amálgama do corpo anti-higiênico e do criminoso em potencial, do
antiestético e do subversivo, do lar e da rua. Sujeito indesejável que
resistirá às investidas políticas para despejá-lo do coração da cidade.
De tal maneira que as habitações coletivas de outrora remanescem --- em
menor escala obviamente ---, mas sobretudo naquelas regiões centrais que
um dia compreenderam as freguesias de Sto. Antônio, Espírito Santo,
Santana e Sta. Rita.

Já a realidade urbanística da cidade, o público, a fisionomia das ruas,
as praças, como objetos de uma intervenção efetiva será um caso tardio.
O campo de intervenção do poder será aqui a realidade urbana na medida
em que funciona como um nó para circuitos cotidianamente repetidos pela
população. Tratar-se-á então de gerenciar a vida da população em um
todo, otimizando a circulação pela intervenção macrourbanística. Só que
a ``reforma'' da Cidade Velha, a correção da estreiteza e sinuosidade
das ruas, objetos de um sem número de projetos ao longo do \textsc{xix}, terão
que aguardar Pereira Passos e os valores nele incorporados.

Em que medida o aburguesamento da rua no entresséculos passou por uma
condenação dos hábitos coloniais e empreendeu não só a radical
remodelação do perímetro urbano do Rio de Janeiro, mas sobretudo uma
normatização dos nossos corpos segundo parâmetros de outro continente?
Ali se achou preferível dar ouvidos a assomos de vaidade, e encobrir a
miséria e o desamparo dos cortiços sob o clarão majestoso do
neoclássico. Sacrificam-se gordas somas em homenagem às afetações de uma
elite de lambedores da pereba terceiro-mundista. Fica aí traçado como
vocação seu século inteiro de bulevares e a sombra das confeitarias para
desfile da sobrecasaca. Pequena elite de insensatos, rebentos bastardos
de um capitalismo de segunda mão, organismos reduzidos a um aparelho
digestivo que vive de roer, que vive de morder, de golfar, de galhofar,
de patuscar, de acumular reservas de barriga. Incapazes de remover os
ídolos dos avós paternos, saquaremas, escravocratas, frouxos ditadores
de origami, tornaram-se liberais felpudos que não concebem a diferença
básica entre direitos e privilégios. São os nossos príncipes herdeiros
que sentem náusea de tudo que é preto, povo e chão de fábrica. Não é
porque lastimamos a injustiça na história, inquirindo a história e
finalmente condenando-a. Não é justiça que se acha em julgamento --- se
talvez fosse, todo passado seria digno de ser condenado, mas não. Porque
repudiamos essas gerações de répteis não deixamos de resultar das suas
aberrações, dos seus erros, dos seus assaltos de histeria, mesmo dos
seus crimes. A libertação de um ciclo não se faz apenas a golpes de
picareta mas com todo um ciclo novo, toda uma segunda natureza rasante.
Acerta o doutor da Academia Imperial de Medicina: o Rio de Janeiro é
semelhante a um vasto cortiço. E eu acho que é preciso sentir um pouco
as entranhas famintas das ruas para entender o que quero dizer e aonde
se quer chegar.\footnote{Cf. Sérgio Ortiz de Inhaúma, \textit{Dioilson}
  (Rio de Janeiro, CLAE, 2017).} Indústrias fechadas, fachadas
destruídas e pichadas, lixo transbordando em ruas entupidas. Penha,
Oswaldo Cruz, Marechal, Colégio, Madureira, Engenho de Dentro, Piedade,
é tudo mais ou menos uma desolação. A subcena da cidade não sobrevive da
caridade dos prefeitos eleitos, seu jeito miserável e sustentável de
sobreviver é o modo atrasado como ela existe. E ela estará morta no dia
em que respirar para fora da alternativa de existir anarquicamente. Sua
invenção incendiária é duvidar que um progresso exista e nos aguarde
consagrar. Eu espero estar morto no dia em que o subúrbio progredir, eu
tenho uma relação obscena com os subúrbios. Não somos um grupo de
marginais, como querem todos que (despejaram o grosso da população de
cortiços para fora da Cidade Velha e) hoje despacham frotas de
infantaria para conter nossos incêndios que não são para serem contidos.
Eu não romantizo a miséria, ela que em sua natureza incendiária insiste
no atraso que é a sua resistência. O subúrbio aprendeu a dar conta do
recado, do abandono. Tudo o que inventamos, que foi depois explorado por
algum desgraçado como mercadoria e descartado como sacolas plásticas,
nós não abandonamos. O que não nasceu para produzir valor não nasceu
para o abandono, esse foi o acordo. Benjamin sabia onde botava as mãos
quando repudiou algo em Marx, Marx quando havia dito que as revoluções
são a locomotiva da história mundial. Porque talvez as coisas se
apresentem mesmo de maneira completamente diferente. É possível que as
revoluções sejam o ato, pela humanidade que viaja neste comboio, de
puxar os freios de emergência. Se assim for, as revoluções sempre
ocorrem nos subúrbios, elas sempre estiveram ali --- o que falta é
permitir que os garotos fiquem vivos. Quem ali cresce, quem não tem
sobrenome e remanesce, não quer parque olímpico, legado de
infraestrutura, segurança pública, quer é os moleques --- só que vivos.
Os moleques vivos, esse é o acordo. Ninguém se importa com promessas de
progresso, com impostos e polícia, deixem os moleques só que vivos, o
subúrbio se ajeita, os moleques só que vivos, os subúrbios sempre se
ajeitam, é deixar os moleques --- só que vivos.

\begin{enumerate}
\def\labelenumi{\arabic{enumi}.}
\setcounter{enumi}{4}
\tightlist
\item
  3 OS ABUSOS DA LIBERDADE E AS VÉSPERAS DO PRESENTE
\end{enumerate}

Como neutralizar, através da polícia médica, a prática dos curandeiros,
tão arraigada no cotidiano de cura daquela sociedade? É com essa
pergunta que iniciamos. Não podemos apostar no fortalecimento de um
dispositivo médico-higienista se não pudermos seguir a trilha dos
espaços deixados vazios. De Pai Manoel à Cabocla do Castelo, percebe-se,
há uma disponibilidade de práticos da cura, alheios à medicina
acadêmica, cujo sucesso popular não é explicado pela carência de
recursos para recorrer a diplomados. Por fim, reportamo-nos à margem na
qual não se pode medir a higiene pública com a régua da ciência médica.
Constantemente esteve a imprensa a noticiar que tal ou qual autoridade
demoliu um cortiço ou desalojou seus moradores. O que torna o cortiço a
única propriedade violável? A epidemia? O crime?

\chapter{Xavier Bichat e Pai Manoel na linha de cura}

\begin{quote}
Eu fui criado no colo de um Preto Velho E suas pernas me embalavam ora
Na harmonia de um acalanto Ora no cavalgar de mar bravo Seu sorriso me
guardava sob os ombros Por vezes eu me agarrava em seu pescoço E olhava
seus dentes brancos me servindo de patuá.
\end{quote}

(``Estrada do Engenho'', Marcos Nascimento)

Um erveiro de origem \textit{Nàgô} escravizado no engenho em Muribeca,
freguesia de Recife, obteve autorização do presidente da província para
tratar vítimas da \textit{cholera morbus} no Hospital da Marinha. Ali
ocupando enfermaria com três leitos, acomodou pacientes. Era 1856, a
epidemia de cólera tinha exterminado 3.338 recifenses em dois ou três
meses. O presidente da Comissão de Higiene Pública da província, Joaquim
d'Aquino Fonseca, que recebeu a notícia do ``curandeiro'' como um
escárnio, renunciara ao cargo. Pai Manoel teria sido de origem
\textit{Nàgô}, presumimos --- da nação iorubafone de \textit{Kétu­} ---,
primeiro porque foi esse o último grupo de africanos a ser traficado em
Pernambuco, no início do \textsc{xix}, segundo porque manejava as
folhas.\footnote{É como o complexo cultural \textit{Jejê-Nàgô} chama as
  ervas, matos, condimentos, cascas e raízes, providas de virtudes
  transmitidas a iniciados apenas.} ``No contexto dos grupos
\textit{Jêje-Nàgô} esta vida vegetal assume relevância particular, uma vez
que o vegetal desempenha papel preponderante em todos os níveis da
existência do \textit{egbé}''.\footnote{José Flávio Pessoa de Barros,
  \textit{O segredo das folhas: sistema de classificação de vegetais no
  candomblé jéje´-nagô do Brasil} (Rio de Janeiro, Pallas / UERJ, 1993),
  p.~38.} Em combinações apropriadas, as folhas formam preparações
medicinais ou ingredientes indispensáveis à prática dos rituais, quer
dizer, a ação curativa está diretamente ligada ao conteúdo sagrado que
se lhe empresta o culto aos \textit{òrisàs}. A pimenta-da-costa, a
pimenta-malagueta, as cebolas-do-reino etc., maceradas em água doce ou
salgada (ou marafo), faziam a garrafada administrada por Pai Manoel.
Porque trabalhava na linha de cura nos faz pensar que o erveiro e
mateiro \textit{nàgô} pode ter exercido no Recife algo semelhante ao cargo
de \textit{Bàbálósányìn}, sacerdote do culto a \textit{Òsányìn}, orixá das
folhas, aquele que guarda o segredo das ervas.

Mas esse acontecimento apaixonante quase ensaguentou a contragolpe o
Recife de meados do \textsc{xix}. Na carta de demissão ao ministro dos Negócios
do Império, o presidente da Comissão acusa alguns que ocupam certa
posição na sociedade --- em cujo número entrou o Capitão Francisco de
Paula Gonçalves da Silva, o irmão do Dr.~Gervásio Gonçalves da Silva
(cujo sogro era senhor de Pai Manoel), além do chefe do Estado maior da
Guarda Nacional Sebastião Lopes Guimarães e outros --- de tomarem o
erveiro sob sua proteção. Conta em seguida que certo padre, lente no
Ginásio, pregava do púlpito da igreja de Sta. Cruz contra os médicos,
``dizendo que só os que morriam eram os pretos e os pardos e que, como o
preto do sogro do Dr.~Gonçalves da Silva os curava, os médicos queriam
matá-lo''.\footnote{Arquivo Nacional. MAÇO IS 4-23, Série Saúde ---
  Higiene e Saúde Pública --- Instituto Oswaldo Cruz, sem paginação.} Pai
Manoel continuou no Hospital até o dia em que o último dos seus faleceu,
e seu insucesso contra a cólera não foi exclusivo. ``Naqueles três meses
em que o cólera atingiu a cidade, `não havia remédio nem dieta com que
se contasse para evitar a morte dos acometidos pelo mal. Tudo era
experimentado infrutiferamente'.''\footnote{Rosilene Gomes Farias, ``Pai
  Manoel, o curandeiro africano, e a medicina no Pernambuco imperial''.
  \textit{História, Ciências, Saúde --- Manguinhos}, Rio de Janeiro, v. 19,
  supl., dez. 2012, p.~217.} O constrangimento dos médicos aumentou na
proporção do entusiasmo popular em torno de Pai Manoel, que indiferente
a deserções e conflitos na medicina autorizada, continuou Recife afora a
medicar e comercializar banhos, dietas e curas. Foi advertido de que não
podia mais exercer a medicina, ignorou a intimação e a Comissão pediu
sua prisão. Os adeptos do erveiro saem aos solavancos às esquinas e
farmácias e protestam contra os médicos.

\begin{quote}
A população insuflada, exaltava-se, e os pretos cativos se tornavam
insolentes, os desordeiros, à espera da ocasião favorável; formavam
grupos que percorressem as ruas, vociferavam contra os médicos e
boticários que se viam expostos a ditos insultuosos; jamais procuravam
dar força a exaltação popular, entretanto que fazia a autoridade
policial? Nada. Permitiam aos grupos que se expressavam para dar assalto
às boticas e fazia acompanhar o preto por ordenanças do Corpo de
Polícia, o que animava a população; e os membros da Comissão recebiam
avisos de pessoas fidedignas que se preparava uma sublevação em que os
médicos seriam as vítimas.\footnote{Arquivo Nacional. MAÇO IS 4-23,
  Série Saúde --- Higiene e Saúde Pública --- Instituto Oswaldo Cruz, sem
  paginação.}
\end{quote}

Pelo episódio de Pai Manoel, pode-se pressentir o que se passa em
algumas províncias em meados do \textsc{xix} e ainda hoje em uma boa medida: a
presença de práticas de cura alternativas à medicina
acadêmica.\footnote{Pai Rafael --- um querido amigo que por alguns anos
  esteve à frente de um antigo terreiro em São Cristóvão, na zona norte
  carioca --- foi por mim consultado sobre se por algum ``acaso'' o
  médico africano do \textsc{xix} poderia ter encantado nas canjiras de umbanda,
  ao que me respondeu: ``Pai Manoel é um espírito até comum. São muitos:
  das almas, do cruzeiro, de Angola\dots{} com certeza este se encantou e
  está por aqui!''} Na cultura das ruas cariocas, isso vai das
rezadeiras da Baixada Fluminense a erveiros e mateiros do Mercadão de
Madureira, há vovós e pretos velhos em terreiros de Umbanda subúrbio
adentro; ``parteiras de dom'' e a pajelança cabocla amazônica no Norte e
Nordeste, que tratam ``doenças naturais'' com ``benzeções'' e prescrevem
``remédios `da terra', isto é, ervas, raízes, folhas, óleos e outros
produtos da vasta farmacopeia popular''.\footnote{Raymundo Heraldo Maués
  e Gisela Macambira Villacorta, ``Pajelança e encantaria amazônica'',
  em Reginaldo Prandi (org.), \textit{Encantaria brasileira: o livro dos
  mestres, caboclos e encantados} (Rio de Janeiro, Pallas, 2004), p.~27.}
Contudo, existem como contrapartidas duas tendências do dispositivo
médico-higienista: primeiramente, a tendência à universalização das suas
práticas; e outra, à censura e enquadramento dos rituais de verdade
médica ligados a cultos do sagrado popular.

As duas tendências complementam-se, e não foram absolutamente efeitos do
êxito alcançado por higienistas no entresséculos. Elas são anteriores,
constitutivas, formam o pedágio para o privilégio de poder exercer a
medicina. Isso porque a disputa por prestígio, para o representante da
medicina acadêmica, dependeu ou foi precedido por técnicas de
policiamento que buscaram desqualificar outros saberes e artes de curar.
Típico dos procedimentos de verdade ritualizados pela medicina
acadêmica, uma epistemologia enraizada na universalidade do saber e do
conhecimento. Mas é preciso refazer a sucessão dos fios, porque a
palavra medicina provoca imagens que estão muito ligadas aos rígidos
contornos institucionais e intelectuais que conquistou recentemente. Uma
coisa é a clínica no século \textsc{xix}, a medicina das reações patológicas,
seus meios para corrigir o organismo; outra muito diferente são as
ambições urbanísticas de práticas higienistas, sua antropologia, seus
métodos para conter as epidemias, sua noção de saúde, a polícia médica.
Qual a relação entre as duas?

Perceber o mórbido é também a maneira de perceber o corpo. Assim,
poderíamos dizer que a clínica identifica o espaço da doença no próprio
espaço do organismo, enquanto a higiene persegue agentes mórbidos
externos, que são tanto a manifestação das doenças individualizadas no
corpo quanto a doença em seu estado universal, puro, sob a forma de
miasmas. Mas infelizmente a genealogia dessa relação não é tão simples
assim. Poderíamos deduzir que a clínica explica a doença de dentro para
fora do organismo, e a higiene de fora para dentro, mas não se tratam de
diferenças entre concepções da doença. Tampouco a diferença é a maneira
de se reconhecer a doença. Diríamos de antemão que, politicamente,
clínica e higiene não existiriam independentes uma da outra. A clínica
carece para sobreviver da polícia médica praticada pelo dispositivo
médico-higienista. A Higiene precisa da clínica anatomopatológica para
positivar-se em termos de verdade científica. Não é apenas uma
dependência teórica, é uma espécie de jogo ou política da verdade cujo
saldo positivo foi a formação de um dispositivo médico-higienista.

Profissionalmente prestigiada como um \textit{corpus} técnico criado para
a correção das disfunções do organismo, a clínica médica tal como a
conhecemos tem uma história muito recente. Essa história não é
independente da realidade que fez prosperar a Higiene --- essa estranha
``Medicina Política'' que mais de perto nos atrai ---, embora não
possamos falar em ``medicina do geral''. Não há ``ciência do geral'',
assim como não há princípios de organização racional da ``cultura
científica''. Foi preciso que desabilitássemos a filosofia platônica nas
pesquisas epistemológicas e nos recusássemos a ver em toda nova ideia
aplicada uma reminiscência, um simples retorno à ideia primitiva do
gênio de um precursor. ``Regionalizar'' é a precaução imposta à
pluralidade de um campo empírico recém-aberto como a Higiene.
Regionalizando, reconhecendo que há regiões particulares de experiências
científicas, foi então indispensável ``indagar em que condições esses
setores particulares recebem não apenas autonomia, mas ainda
autopolêmica, isto é, um valor de empresa sobre as experiências
novas''.\footnote{Gaston Bachelard, \textit{O racionalismo aplicado}
  {[}1949{]}, trad. Nathanael C. Caixeiro (Rio de Janeiro, Zahar, 1977),
  p.~143.} Regiões do saber não são, portanto, naturalmente dadas, elas
são constituídas, balizadas por um gênero de racionalismo.

O que podemos apontar é que, pelo menos até meados da década de 1870, o
currículo das faculdades de medicina nacionais carecia de um consenso
elementar sobre fundamentos teóricos dos saberes e práticas clínicas.
Por sua vez, a ``profissão médica'' tendia a ser uma realidade
fortemente estratificada, pendulando entre uma pequena elite médica
diplomada de faculdades e uma diversidade de práticos populares
(homeopatas, parteiras, curandeiros itinerantes), que aplicavam terapias
muitas vezes prescritas pela própria comunidade médica --- sangrias,
purgações, banhos etc. Enquanto isso, a medicina europeia, como domínio
científico, atravessava no mesmo período uma crise e um processo de
reestruturação de fundamentos, o que implicou progressivamente novos
procedimentos e padrões epistemológicos. Dessa transformação nasceu,
segundo Foucault, o ``grande corte na história da medicina ocidental'',
que data precisamente do ``momento em que a experiência clínica
tornou-se o olhar anatomoclínico''.\footnote{Michel Foucault, \textit{O
  Nascimento da Clínica} {[}1963{]}, 1977, p. 167-8.} A anatomia
patológica forneceu o maior contingente para o progresso da ciência do
diagnóstico, segundo Torres Homem: ``antes dos trabalhos de Bichat, o
tratamento das moléstias era unicamente baseado nos sintomas que
apresentavam''.\footnote{João Vicente Torres Homem, \textit{Elementos da
  Clínica Médica --- seguidos do anuário das mais notáveis observações
  colhidas nas enfermarias de clínica médica em 1869} (Rio de Janeiro,
  Nicoláo A. Alves, 1870), p.~41.} Foucault fala em ``experiência''
clínica, fala em ``olhar'' anatomoclínico, e essas duas palavras alertam
sobre a transformação dos critérios a partir dos quais olhar e doença
estarão ligados um ao outro. É uma relação que determina mudanças em
termos de diagnóstico e terapêutica antes ``unicamente baseados nos
sintomas''. Mas não apenas.

Entre o período pré-clínico e a anatomia patológica, transforma-se, no
limite, o próprio príncípio de decifração ou a forma fundamental de
espacialização do corpo como organismo. Vejamos a partir de um trecho de
uma aula de Bichat o que começa a fazer apelo através da experiência da
anatomoclínica.

\begin{quote}
Consideremos as doenças sob este aspecto, façamos a \textit{abstração} dos
sistemas que, conjuntamente àquele que é afetado, concorrem para a
formação de um órgão. Daí estabelecemos a consequência de que cada
sistema pode afetar-se isoladamente. A abertura de cadáveres bem o
prova, uma vez que nos mostra que quase todas as doenças locais têm sua
sede no tecido particular de um órgão afetado. (\ldots{}) Tomemos o
pulmão como exemplo. É um órgão composto pela pleura, pelo tecido
pulmonar e pela membrana interna. Na pleuresia, só a pleura está
inflamada, mas o tecido pulmonar e a membrana mucosa estão intactos. Na
peripneumonia é, do contrário, o pulmão que inflama, enquanto suas duas
membranas estão saudáveis. As tosses com catarro têm a ver com a
membrana mucosa, enquanto a parênquima e a membrana serosa permanecem em
estado natural. Este exemplo pode servir como critério de comparação
para todos os outros órgãos.\footnote{Xavier Bichat, \textit{Anatomie
  pathologique}, 1825, p.~12, tradução nossa.}
\end{quote}

Foi preciso deslocar a atenção para os sistemas constituídos pelos
``tecidos'' mais do que para os órgãos na sua fictícia individualidade.
Saem os órgãos, como unidades de medida que referenciam uma nosologia
tradicional, e em seu lugar entram os tecidos. Não porque sejam
caracterizações mais corretas, mas porque órgãos são generalizações de
um olhar que nasceu para localizar, medir, classificar. Os órgãos, antes
dados primeiros para um olhar local e circunscrito, retornam como
modificações da matéria de que são feitos os tecidos. O olhar integrador
da anatomoclínica os constitui agora como relativos a elementos
tissulares de que são feitos e aos sistemas aos quais pertencem. A
``abstração'' à qual Bichat se referia acima --- que, com efeito,
multiplicou as superfícies visíveis da experimentação médica ---
significou o seguinte: os tecidos, ``eles são os elementos dos órgãos,
mas os atravessam, aparentam e, acima deles, constituem vastos
`sistemas' em que o corpo humano encontra as formas concretas de sua
unidade''.\footnote{Michel Foucault, \textit{O Nascimento da Clínica}
  {[}1963{]}, 1977, p. 146}

Bichat descreve pontualmente como a anatomoclínica pôde distinguir no
pulmão as afecções que ocorrem no tecido pulmonar daquelas que atingem a
pleura ou que afetam somente a membrana serosa. Afecções que seriam
ininteligíveis em um contexto epistemológico que porventura não
atribuísse às membranas uma individualidade tissular distinguível dos
órgãos que elas revestem. A análise que permitiu identificar no coração
o pericárdio e distinguir no cérebro a aracnoide é a mesma que
identifica de que maneira a presença de ``tecidos da mesma textura
através do organismo permite ler, de doença em doença, semelhanças,
parentescos, todo um sistema de comunicações''\footnote{\textit{Ibidem},
  p.~148} que não é local, mas que se expande em linhas diagonais
capazes de desdobrar ou ramificar os espaços lesionados. ``Que importa
que a aracnóide, a pleura e o peritônio se situem em diferentes regiões
do corpo, se suas membranas têm uma conformidade geral de estrutura? Não
sofrem lesões análogas no estado de flegmasia?''\footnote{\textit{Ibidem},
  p.~150} Daí se estabelecer que cada sistema pode afetar-se
isoladamente, podendo inclusive irradiar em diagonais efeitos
patológicos a partir de um foco primitivo. A doença deixa de ser uma
espécie viva em um quadro de identidades e diferenças para assumir a
figura da lesão, da inflamação, dentro de uma cadeia de efeitos móveis e
progressivos no tempo.

Novamente: não é tanto a concepção de doença o mais importante na
leitura de Foucault, mas muito mais esse horizonte de visibilidade que
redistribui o espaço corporal em uma zona de tridimensionalidade que
expõe antigos sintomas visíveis à incerteza. ``O olhar clínico tem esta
paradoxal propriedade de \textit{ouvir uma linguagem} no momento em que
\textit{percebe um espetáculo}. Na clínica, o que se manifesta é
originariamente o que fala'',\footnote{\textit{Ibidem}, p.~122.} a
geografia do visível é, de ponta a ponta, a sintaxe verbal dos
acontecimentos. Mas em algum momento começou-se ``a falar sobre coisas
que têm \textit{lugar} num espaço diverso do das palavras''.\footnote{Michel
  Foucault, \textit{As palavras e as coisas} {[}1966{]}, 2007, p. 317.} E
a caracterização da região tissular abriu precedentes para que
acontecimentos patológicos oferecidos à percepção viessem a se situar em
uma terceira dimensão do organismo até então interditada. Um dia todo
sintoma --- a essência mórbida, o estado patológico ou indício do mesmo
--- foi potencialmente um signo --- o pulso, a temperatura corporal, o
estado da respiração ---,o signo era apenas um sintoma lido. Já ``para
uma percepção anatomoclínica, o sintoma pode perfeitamente permanecer
mudo e o núcleo significativo (\dots{}) revelar-se inexistente. Que sintoma
visível pode indicar com certeza a tísica pulmonar?''\footnote{Michel
  Foucault, \textit{O Nascimento da Clínica} {[}1963{]}, 1977, p. 182.}
Coisa semelhante faz Torres Homem quando explica que ``sem a anatomia
patológica era inteiramente impossível reconhecer-se um grande número de
afecções que invade frequentemente o organismo humano; as
degenerescências vicerais, as produções heteroplásticas, nunca seriam
estudadas''.\footnote{João V. Torres Homem, \textit{Elementos da Clínica
  Médica\ldots{}}, 1870, p. 42.}

Antes da anatomoclínica, o ser da doença se dava no espaço analisável da
representação, assemelhando-se a uma natureza estranha que aderia ao
corpo e se alojava em um órgão como uma presença mórbida ou parasitária.
A doença agora não apenas surge ligada à corrupção da vida, como também
tem a ver com o próprio ciclo temporal de um organismo cujos recuos ou
escapes antecipam no trajeto da existência sua condição de acontecimento
transitório, finito. O acento se desloca aos poucos da doença para o
doente, entre patologia e fisiologia os limiares ficam embaçados. Se
agora --- para evocar a famosa definição de Bichat que diz ser a vida a
totalidade das funções que resistem à morte --- o clínico aloja na vida a
precariedade que é a própria ameaça prematura da morte, logo, doença não
é só o infortúnio aproximado da morte, mas sua latência, sua iminência
rotineira. Isso é pertinente porque se o fenômeno patológico tende a ser
uma variação mórbida de um funcionamento ``normal'' do organismo,
torna-se improvável definir-se doença sem, de antemão, questionarmos o
estatuto do que seria ``saúde''. Isso mais tarde tornará possível um
Claude Bernard,\footnote{Georges Canguilhem, \textit{O normal e o
  patológico} {[}1943{]}, trad. Vera L. A. Ribeiro (Rio de Janeiro,
  Forense Universitária, 2012), p. 33-7.} na ideia de uma continuidade
entre normal e o patológico, já que ambos não seriam qualitativamente
opostos --- pelo contrário, manifestam-se no organismo segundo uma mesma
lógica. De modo que, no fim das contas, acaso existirá estado normal ou
saúde perfeita para que se possa falar em saúde-doença tal como falamos
em coisas objetivas como ``lama'', ``cabelo'', ``poeira''? Ou podemos
dizer que a saúde perfeita é um mero tipo ideal, um conjunto de
qualidades idealmente acumuladas em um conceito que nos permite elevá-lo
a um ``conceito relativo a um ponto de referência''? É com problemas
dessa ordem que \textit{O normal e o patológico} se debate, e Georges
Canguilhem encaminhará brilhantemente conclusões
histórico-epistemológicas que conservam toda importância. Para todo
efeito, é digno de nota esse percurso que o conceito de saúde irá traçar
historicamente, de um conjunto de preocupações filosóficas que
pertenciam à anatomoclínica para os mecanismos de normatização das
práticas higienistas. É talvez até mesmo desnecessário estabelecer esses
fenômenos em dois momentos sucessivos.

Um último apontamento, para recuperar o fio de um assunto há pouco
abandonado:Bichat relativiza a morte sob o prisma da infinitude clássica
e assenta a ciência médica sobre o solo epistêmico de uma analítica da
finitude. Há de fato algo de novo em Bichat que o transforma em um
pensador moderno muito fundamental.

\begin{quote}
No século \textsc{xviii}, a doença pertencia tanto à natureza quanto à
contranatureza, na medida em que tinha uma essência ordenada, mas que
era de sua essência comprometer a vida natural. A partir de Bichat, a
doença vai desempenhar o mesmo papel misto, mas agora entre a vida e a
morte. Entendamo-nos bem: conhecia-se, bem antes da anatomia patológica,
o caminho que vai da saúde à doença e desta à morte. (\ldots{}) O que já
conhecemos: a morte como ponto de vista absoluto sobre a vida e abertura
(\ldots{}) para sua verdade. Mas a morte é também aquilo contra que, em
seu exercício cotidiano, a vida vem se chocar; nela o ser vivo
naturalmente se dissolve: e a doença perde seu velho estatuto de
acidente para entrar na dimensão interior, constante e móvel da relação
da vida com a morte. Não é porque caiu doente que o homem morre; é
fundamentalmente porque pode morrer que o homem adoece.\footnote{Michel
  Foucault, \textit{O Nascimento da Clínica} {[}1963{]}, 1977, p. 177.}
\end{quote}

A doença é a dobra interior da vida, é a vida modificando-se para
adquirir a mineralidade que lhe é coextensiva. A morte não é então o
instante da extinção inadvertida, é potência coextensiva e sobressalente
na enfermidade. ``Ela é coextensiva à vida e ao mesmo tempo dissemina-se
na vida sob a forma de mortes parciais''.\footnote{Gilles Deleuze,
  \textit{Michel Foucault: as formações históricas} {[}1985-86{]}, trad.
  Claudio Medeiros e Mario Marino (São Paulo, N-1 edições / Ed.
  Politeia, 2018), aula 6.} Não paramos de morrer um minuto, morremos
mortes parcelares, paralelas, múltiplas, o tempo todo. Ela não é o átimo
em que a tesoura rompe o trabalho da fiandeira, é movimento em
progressão. Porque a morte está multiplicada na vida, porque não é o
recolhimento de algo, e sim disseminação, patologia e fisiologia
articulam-se necessariamente a partir de Bichat. Isso significa que o
desvio da vida é da natureza da vida mesma, porém de uma vida que conduz
à morte. O que permite a Foucault enfatizar a ``importância que adquire
o conceito de `degeneração' desde o aparecimento da anatomia patológica.
Noção antiga: Buffon a aplicava aos indivíduos ou séries de indivíduos
que se afastam de seu tipo específico'';\footnote{Michel Foucault,
  \textit{O Nascimento da Clínica} {[}1963{]}, 1977, p. 177-8.} os
higienistas também a utilizarão para designar os componentes morais,
sociais e urbanísticos que viabilizam epidemias e seus efeitos mórbidos.
Vejamos.

Em seu capítulo sobre a degeneração dos animais, no Tomo XIV da
\textit{Histoire Naturelle}, Buffon elenca quais influências realizam a
mudança, a alteração, a degeneração da natureza das espécies. E aqui se
precipita a problemática com a qual eu tentei me ocupar na escrita de
uma dissertação que se chama \textit{O devir do conceito de ``meio'' entre
os séculos \textsc{xvii} e \textsc{xix}, segundo a História das Ciências de Georges
Canguilhem}.\footnote{Claudio Medeiros, \textit{O devir do conceito de
  ``meio'' entre os séculos \textsc{xvii} e \textsc{xix}}, 2014.} Para o naturalista
Georges-Louis Leclerc, o Conde de Buffon (1707-1788), a má influência do
meio sobre o organismo desencadeia a ação desordenada das forças sobre
as moléculas orgânicas, traindo, nesse movimento, a manutenção da
espécie originária conforme a obediência ao seu princípio de
individuação orgânico. Pois bem, o papel do componente mecânico e
antropogeográfico dessa teoria do meio (\textit{milieu}) na feitura do
conceito buffoniano de ``degeneração'' situa aquela dissertação, de
certa forma, como um tipo de prólogo desta atual genealogia de um
dispositivo médico-higienista. Afinal, isso que se acostumou a chamar de
``teoria dos miasmas'' (usamos ``teoria'' em uma acepção pouco rigorosa)
não tem suas raízes teóricas na medicina, mas no modelo de
``antropologia'' que amparou uma \textit{História Natural}. O problema de
Buffon foi, bem entendido: como explicar as causas que concorreram para
produzir a variedade dos povos? Em outras palavras: como a ``tintura do
Céu''\footnote{Georges Louis Leclerc de Buffon, \textit{Histoire
  Naturelle, générale et particuliere --- Tome Troisième}, 1749,
  p.~315-16, tradução nossa.} agiu sobre a superfície alterando a cor da
pele (a pele consiste precisamente na variação primeira: ``a mais
marcante'')\footnote{\textit{Ibidem}, tradução nossa.} na definição das
diferentes raças que compõem o gênero humano? Ele dizia: a cor da pele,
que constitui o sintoma mais nítido da degeneração do homem, é
determinada por três causas. ``A primeira é a influência do clima, a
segunda que deve muito à primeira, é a alimentação, e a terceira, que
talvez deva tanto à primeira quanto à segunda, são os
costumes''.\footnote{\textit{Ibidem}, p.~447-8, tradução nossa.} Nada
prova melhor que ``o clima é a principal causa da variedade na espécie
humana do que a cor dos Hotentotes''.\footnote{\textit{Ibidem}, p.~519,
  tradução nossa.} Sim, Buffon escreve paralelamente à ``história do
indivíduo'' --- o que implica a geração do homem, sua formação, sua
condição nos diferentes anos de sua vida, seus sentidos, sua estrutura
corporal conhecida por meio das dissecações anatômicas etc. --- uma
``história da espécie humana'', cujos fatos são ``extraídos das
variedades encontradas em homens de diferentes climas''.\footnote{\textit{Ibidem},
  p.~371, tradução nossa.} Em duas palavras, seu legado ilustre foi esse
gesto que buscou fundamentar cientificamente, e em pleno \textsc{xviii}, a ideia
de que é dos países policiados situados na zona temperada de onde se
deve tomar a representação da ``verdadeira cor natural do homem, é daí
que se deve tomar o modelo ou a unidade a qual é preciso reportar todas
as outras nuances de cor e de beleza''.\footnote{\textit{Ibidem}, p.~528,
  tradução nossa.} Fechamos o parêntese de uma demanda que ainda não
havia sido preenchida em termos de história das ciências neste trabalho:
a Higiene e as práticas policiais que lhe são coextensivas são herdeiras
e inseparáveis da realidade do projeto colonialista europeu na América,
tal como das formas biopolíticas de racialização que conhecemos pela
experiência de vida. A invenção da raça em Buffon impulsionou a
contrapartida ao perigo encarnado na existência do negro assim que
escravizado. O problema colocado pela ordem colonial era o da raça
enquanto princípio de exercício de poder e mecanismo de adestramento das
condutas.\footnote{Cf. Claudio Medeiros, ``A filosofia política de
  Achille Mbembe: racismo e saída da democracia'', \textit{Ensaios
  Filosóficos}, vol.~\textsc{xviii}, dez.2018.} Racializar significou ora
naturalizar a servidão, ora produzir uma margem de liberdade que fosse
subsídio para que se incorporasse a universalidade dos valores da
civilização europeia. Não irei cometer o ridículo de defender que tudo
isso se mantinha em estado embrionário no pensamento de Bichat. Mas não
estamos sozinhos quando presumimos que o problema que se coloca no
coração do vitalismo fisiológico --- a distinção entre vivente e não
vivente --- e muito do seu poder residem na inovadora abordagem de Bichat
da questão sobre como um vivente se torna não vivente, como a vida se
transforma em morte. Elizabeth A. Williams, em seu \textit{The physical
and the moral: anthropology, physiology, and philosophical medicine in
France (1750-1850)}, escreve que aquela famosa concepção de Bichat ---
segundo a qual a \textit{vida} é o conjunto ou a totalidade das funções
que resistem à morte --- ``automaticamente privilegia o problema do meio
no qual a vida se sustenta, e encoraja Bichat a explorar problemas do
`\textit{milieu}', problemas que para os outros estavam fora do domínio da
fisiologia''.\footnote{Elizabeth Williams, \textit{The physical and the
  moral: anthropology, physiology, and philosophical medicine in France,
  1750-1850} (Cambridge, Cambridge University Press, 1994), p.~96.
  Tradução nossa.}

A autora descreve como no século \textsc{xix} os médicos de Montpellier defendiam
a expansão da medicina (que se reduzia a mera expressão terapêutica)
para uma forma filosófica de ciência médica que investigasse o amplo
alcance das circunstâncias morais e sociais na influência sobre estados
de saúde e doença.\footnote{\textit{Ibidem}, p.~25. Tradução nossa.}
Coisas que nos fazem aderir fortemente ao antipositivismo de Foucault,
segundo o qual é preciso ser muito míope para acreditar que a
anatomoclínica nasceu no ``jardim livre em que, por um consentimento
comum, médico e doente vêm a se encontrar, em que a observação se faz,
no mutismo das teorias, pela claridade única do olhar, em que, de mestre
a discípulo, a experiência se transmite abaixo das próprias
palavras''.\footnote{Michel Foucault, \textit{O Nascimento da Clínica}
  {[}1963{]}, 1977, p. 58.} Isso nos trouxe de volta às formas
institucionais imperiais que atualizaram certas relações de força com um
tipo difuso de ``charlatanismo''. Porque essas práticas são
indissociáveis da formação de uma ciência médica e de suas pretensões de
universalidade.

No Brasil, pela Lei de 13 de Outubro de 1832 e pelo artigo 25 do
Regulamento da Junta de 29 de Setembro de 1851, não se podia exercer a
medicina ou quaisquer dos seus ramos sem título conferido pelas nossas
Escolas ou sem a habilitação de diplomas estrangeiros. O conteúdo das
leis veio a ser reeditado e atualizado em outras leis e regulamentos ao
longo de toda a segunda metade do \textsc{xix}. Os ``práticos'' da cura ---
unificados sob o nome ``charlatões'' --- serão multados e acusados de
estelionato nos jornais, nos artigos de medicina e nas teses dos
ilustrados em geral. De fato, os interessados na cruzada
anticharlatanismo serão os médicos em disputa por mercado, as casas de
saúde e faculdades de medicina disputando poder e verdade. Só que
felizmente não é sempre que as coisas acontecem tendo a justiça no topo,
regando as instâncias inferiores, fazendo-as participar do seu protocolo
sob o princípio de um código fixo.

Teria havido, por exemplo, em relação à trajetória de Pai Manoel e de
outros práticos da cura na Corte Imperial, não apenas uma estreita
margem de tolerância, resguardada por complexos de solidariedade e
pertencimento tecidos pelos africanos em diáspora. Não que esses
complexos não contribuíssem, mas teria havido agenciamentos que atuavam
em frestas, agenciamentos ínfimos e cautelosos, sem dúvida, caminhando
em parapeitos, nem por isso menos numerosos e capilares. São na verdade
essas linhas de fuga as determinações primeiras, ao passo que as
estratégias de poder que vêm a contrapelo serão apenas algumas entre os
vários componentes de agenciamentos coletivos mais amplos. Há, por
exemplo, esse testemunho de Gilberto Freyre, a respeito da ampla adesão
daquela sociedade pernambucana aos serviços oferecidos por Pai Manoel.

\begin{quote}
Não eram só os doentes pretos e os pardos que corriam dos mucambos e das
senzalas para o parceiro: também brancos finos de sobrado. (\dots{})
Ofereciam-lhe ``carro para conducção rapida'', (\ldots{}) carro de
cavalo que naqueles dias era privilégio ou regalo só de brancos, de
fidalgos, de senhores. E em informação ao ministro do Império teve de
confessar o então presidente da província que, no meio da ``conflagração
epidemica'' que se estendia pelo país inteiro, surgia em Pernambuco o
preto Manoel ``com aura extraordinaria'', trazido do interior da
província ``como um signal de redempção''(\dots{}), todos a afirmarem que o
preto já operara ``curas instantaneas e que o povo o applaudia''. Isto
em contraste com ``a descrença a respeito dos medicos'', que as
devastações da cólera nas casas-grandes e principalmente nas senzalas e
da febre amarela nos sobrados --- principalmente nos sobrados das
capitais --- vinha acentuando na população.\footnote{Gilberto Freyre,
  \textit{Sobrados e mucambos} {[}1936{]} (São Paulo, Global Editora,
  2013), p.~401.}
\end{quote}

Quer dizer que já não podemos falar em agentes de captura, que
tardiamente comporão uma rede ou uma estratificação, melhor dizendo, não
podemos apostar no surgimento de um dispositivo médico-higienista --- que
é por definição urbano e policial, e que costura ou faz as relações de
força funcionarem através de \textit{estratificações} --- se não pudermos
seguir a trilha dos espaços deixados vazios. Como ``os dispositivos de
poder são constitutivos da verdade, se há uma verdade do poder, deve
haver aí, como contraestratégia, uma espécie de poder da verdade contra
os poderes''.\footnote{Gilles Deleuze, ``Désir et plaisir'', em
  ``Foucault aujourd'hui'', \textit{Magazine Littéraire}, n.~325, Paris,
  out. 1994, p.~20 (tradução Luiz B. L. Orlandi).} Ou seja, quer haja
poderes e contrapoderes, há também contraverdades, contranarrativas. E
como estratégias de poder e procedimentos de verdade atuam em conjunto,
a instabilidade das estratégias de poder ameaçadas dependem de
contraverdades que façam estremecer ou apenas resistam à naturalização
de um dado regime de verdade.

Sim, podemos denominar contraverdades certos procedimentos manejados
pelos ``práticos'' da cura. Como também podemos testar a noção de
``quilombos de saberes'', se por \textit{quilombos} não reconhecermos
colônias rurais ou comunidades primitivas à margem da dinâmica urbana.
Um quilombo era uma realidade reconhecida por autoridades locais, que
vigiavam essas comunidades ou reprimindo, ou negociando tolerâncias.
Quilombos como heterotopias, canais, redes de comércio de informação
entre forros, fugidos e escravizados. Redes garantidas pela resistência
ao desaparecimento, pela renovação das práticas de sobrevivência, e que
não deixaram de alarmar as autoridades um dia sequer. O jongo, as casas
de axé, a capoeira, a filosofia de \textit{Ifá}, a medicina de Pai Manoel
e dos seus são para nós ``quilombos de saberes''. E entre suas maiores
contribuições talvez esteja a capacidade de não ceder ao desencantamento
desse cômodo da realidade que são as cidades, que foram um dia colônias
periféricas de uma Europa mercantilizada. Ao longo do tempo em que
``descredibilizamos as possibilidades que vagueiam o
invisível'',\footnote{Luiz Antonio Simas, \textit{Fogo no mato: a ciência
  encantada das macumbas} (Rio de Janeiro, Mórula, 2018), p.~105.} os
quilombos de saberes preservaram, como patrimônios seus, certos modos de
racionalidade que nossa cultura não dimensiona, seja por falta de
repertório, seja mesmo pelo voluntarismo narcísico que sustentou o
etnocentrismo colonizador.

\begin{quote}
Concordamos com Pessoa de Castro, quanto ao fato de ter sido a
introdução contínua de escravos de uma mesma procedência étnica no meio
urbano, fator relevante para a viabilização de uma resistência maior.
(\ldots{}) Cabe ressaltar que, em escritos do início deste século, Nina
Rodrigues encarava o candomblé como um foco de resistência cultural e
como centro de fermentação para sublevações e rebelião social (\dots{}).
Albuquerque, analisando a formação social brasileira afirma: `neste
sentido as práticas rituais afro-brasileiras foram um aspecto particular
de luta social, de vez que a situação do escravo o impedia de ter
condiçoes de resistência legal aos níveis econômico e político'. Assim é
que as Casas de Culto podem ser encaradas como fator de coesão social,
homogeneizando as rivalidades procedentes do continente africano que
porventura existissem na população escrava.\footnote{José F. P. Barros,
  \textit{O segredo das folhas}, 1993 , p.~12-3.}
\end{quote}

Este trecho de \textit{O segredo das folhas}, de José Flávio Pessoa de
Barros, cita pontualmente uma posição defendida por Nina Rodrigues em
seu \textit{O animismo fetichista dos negros baianos} e que nos é
paradoxalmente muito cara: dos \textit{egbés}, das comunidades onde se
renova a adoração aos \textit{òrìsàs} e aos \textit{égúns} (nossos
ancestrais ilustres assim que encantados) veio certo estoque de vida
ingovernável, uma força propulsora periodicamente acessada em
oportunidades de sublevação e rebelião racial e social. E é o mesmo Nina
Rodrigues --- o médico bastante reconhecido pelo papel como ideólogo do
racismo --- que nos apresenta outro acontecimento desconcertante
envolvendo outra epidemia, dessa vez em Salvador, quase 40 anos depois:
em 1893, correram boatos de que uma \textit{cholera morbus}, que vinha se
manifestando na Europa, viria a Salvador\textit{.} Espalhou-se, da noite
para o dia nos terreiros de arrabaldes, a notícia de que um orixá teria
trazido ao sacerdote de uma casa de culto recado dizendo que a cidade
estaria a ponto de ser invadida pela peste.

\begin{quote}
Como único recurso eficaz para conjurar o perigo iminente indicava ele o
ato expiatório ou votivo de levar cada habitante uma vela de cera a
Santo Antonio da Barra, que, tendo a sua igreja situada na entrada do
porto, podia facilmente impedir a importação da epidemia. Para logo
levar uma vela a Santo Antonio da Barra tornou-se a preocupação
exclusiva de toda a população, e a romaria tomou proporções tais que em
breve quase não havia mais espaço na igreja para receber as velas
votivas.\footnote{Raimundo Nina Rodrigues, \textit{O animismo fetichista
  dos negros baianos} {[}1896-97{]} (Rio de Janeiro, Fundação Biblioteca
  Nacional/Ed. UFRJ, 2006), p.~116.}
\end{quote}

Nada excepcional o desprezo do antropólogo maranhense com as crenças
``fetichistas'' que arrastavam um público que ``seria incalculável se
não fosse mais simples dizer de um modo geral que é a população em
massa, à exceção de uma pequena minoria de espíritos superiores e
esclarecidos''.\footnote{\textit{Ibidem}, p.~116.} Porém, chama a atenção
a população \textit{em massa}, o prestígio e a influência de certas
práticas que mantiveram sob suspeita a \textit{razão universal} médica. E
não só: a expansão do horizonte no interior do qual o mundo, o corpo e
as práticas de cura estariam dados, o fato de desestabilizarem em
qualquer medida a soberania de um modo de saber médico comprometido com
essa \textit{experiência comum e necessária} que a medicina acadêmica
queria para si. Isso remete a uma característica fundamental do sistema
\textit{Nàgô}, no que diz respeito à experiência do ``corpo como rito'' e
da ``vida como figura de encantamento''. Tudo um pouco avesso a uma
época em que a anatomia patológica fundava Bichat e sua clínica. E
certamente bastante estranho ao cartesianismo mecanista em medicina que
mencionávamos no primeiro capítulo.

Elisabeth Williams explica que mesmo no ``século \textsc{xix} a concepção
mecanicista do corpo, que em última análise deixou resultados na
medicina experimental e em suas concomitantes instrumentalidades
tecnológicas, foi virtualmente triunfante em toda a Europa''.\footnote{Elizabeth
  Williams, \textit{The physical and the moral}, 1994, p.~23, tradução
  nossa.} Mas se recolhemos o corpo no organismo, o coração na bomba
ativada em termos de volume e fluxo, o organismo no encadeamento de
órgãos-engrenagem; se a estrutura mecanicista de racionalidade fornece a
imagem do organismo em analogia com os relógios de algibeira,
reconhecemos que o corpo está tão naturalmente predisposto ao cálculo
quanto as regras de um dispositivo mecânico. A aposta dessa medicina,
que torna a totalidade histórica do corpo uma colônia de um tipo de
catequeze epistemológica, precariza inevitavelmente formas outras de
problematização ética da vida para além do paradigma biopolítico. Ora,
enquanto o dispositivo médico-higienista quis realizar aspectos da sua
ontologia, a transgressão ao colonialismo nos termos de um sistema de
conhecimento terapêutico \textit{Nàgô} se expressou mais ou menos da
seguinte maneira: ``todo objeto, ser ou lugar consagrado só o é através
da aquisição de \textit{àse}. Compreende-se assim que o `terreiro', todos
os seus conteúdos materiais e seus iniciados, devem receber \textit{àse},
acumulá-lo, mantê-lo e desenvolvê-lo''.\footnote{Juana Elbein dos
  Santos, \textit{Os Nàgô e a morte}, traduzido pela Universidade Federal
  da Bahia (Petrópolis, Vozes, 1986), p.~40.} Entre os objetos
impregnados de \textit{àse} encontram-se raízes, folhas e todos aqueles
compostos vegetais que não surtirão seus efeitos medicinais a não ser
que tenham passado por uma consagração que revitaliza, neles mesmos, uma
quantidade suficiente de \textit{àse}. O \textit{àse} é uma sorte de poder
de realização, substância reagente e força catalizadora que permite que
as coisas recebam sua existência dinâmica e venham a ser o que são.
``Receber \textit{àse} significa incorporar os elementos simbólicos que
representam os princípios vitais e essenciais de tudo o que existe, numa
particular combinação que individualiza e permite uma significação
determinada''.\footnote{\textit{Ibidem}, p.~42.} E por que destacá-lo?

Porque há uma prática ritual que ativa e distribui o \textit{àse} --- nas
combinações feitas entre grandes variedades de elementos do reino
animal, vegetal e mineral --- e isso assume aqui o sentido preciso de
dizer que não há doutrina ou ciência possível, em uma tradição
afrobrasileira nascente, que não seja vivida através da experiência
ritual, ou seja: ``o conhecimento só tem significado quando incorporado
de modo ativo''.\footnote{\textit{Ibidem,} p.~45.} Em um contexto
apropriado e produzido por um iniciado delegado para tal função, a
palavra cantada, acompanhada ou não de tambores, evoca a presença do
\textit{àse} dos ancestrais --- dos seus antepassados ilustres, entre eles
os próprios \textit{òrìsàs} --- cujo poder acumulado no ``sangue'' das
folhas, animais e minerais transmitem a ação e mobilizam a atividade
litúrgica --- e terapêutica, no nosso caso. Logo, virtudes medicinais
fitoterápicas são como que preparadas ou desencadeadas em sua particular
eficácia pela palavra cantada.

Cantar, tocar o tambor sincopado ou chamar as folhas pelas denominações
corretas em \textit{yorùbá} permitem que a força contida em todas as
coisas seja pelo próprio rito deflagrada. ``Assim, as `cantigas de
folha' --- \textit{orin ewé} --- são uma forma especial de detonar o
\textit{àse} potencial das espécies vegetais''.\footnote{José F. P.
  Barros, \textit{O segredo das folhas}, 1993, p.~40.} Não por acaso, pelo
menos no Rio de Janeiro imperial, as praças da polícia se referiam às
casas de culto utilizando termos como \textit{batuques} ou \textit{zungús.}
No período republicano, o Código Penal de 1890 prevê punições ao
curandeirismo, mas durante o Império ainda não estava tipificado como
contravenção. Nossos práticos caíam nas malhas da lei acusados de
\textit{batuques}, badernas, vadiagem, \textit{zungús},\footnote{Espaços de
  convivência de forros e da escravatura de ganho urbana, funcionando
  tanto como habitação coletiva quanto como pequenas cantinas ou
  armazéns.} estelionato. É por estelionato, por exemplo, que um sujeito
chamado Laurentino Innocêncio dos Santos daria entrada na Casa de
Detenção da Corte em 1879.

Em Pendura Saia, Cosme Velho, tinha uma casa de \textit{zungú} o prático
curandeiro Laurentino Inocêncio dos Santos, e consta que tirava bons
proventos de~\\
sua medicina. O recinto onde o curandeiro funcionava, para alcançar para
uns fortuna e para outros saúde, e que a polícia profanou, dizem os
jornais seria apenas um quarto com dois vistosos e ricamente alfaiados
altares com~imagens de diversas invocações. A polícia deu com ele na
cadeia da 11ª estação policial, no início de março de 1890. Não seria a
primeira nem a última vez que cairia nas malhas da lei, conforme a
pesquisa feita por Eduardo Possidonio em seu \textit{Entre ngangas e
manipansos}. O historiador reencontra um ano depois outra denúncia no
\textit{Gazeta de Notícias}, um anônimo rogando ao ``Sr.~chefe de polícia
para dar providências para o fim de evitar que certo curandeiro, nas
Laranjeiras de nome Laurentino, cobre 12\$ por consulta e para tratar
300\$ ou 400\$ e ainda seja atrevido com as famílias que vão ali
seduzidas''.\footnote{Eduardo Possidonio, \textit{Entre ngangas e
  manipansos} (Salvador, Sagga, 2018), p.~115.} A tabela de valores
discriminando preços para consultas ou serviços da casa marca a disputa
entre o curandeiro e algum possível médico, naturalmente crítico dos
altos honorários que poderiam muito bem ser investidos com honestidade
em clínicas particulares. Denúncias dessa ordem foram comuns nos jornais
da Corte, assim como anúncios que ofereciam tratamento para o corpo e
para a ventura dos moradores dessa cidade. Por exemplo, em Sacramento,
nos fundos do sobrado nº 57 da Rua da Conceição, um sujeito que atendia
pelo nome de Felippe Miguel refinava raízes na companhia de santos
católicos, vasilhames com caramujos, preparativos para banhos contendo
ervas líquidas para tratar amantes amarrados, erisipela ou situações
menos graves. Atenta a denúncias por badernas e batuques, a polícia
manda confiscar os tambores e as tíbias dos sambas e sessões privadas, e
também os frascos com as receitas e a louça sarapintada do altar do
curandeiro de misteriosa indústria.\footnote{Cf. \textit{ibidem}, p.~58-9.}
A batida policial ocorreu em 1871 e quase sempre essas notícias
compareciam à crônica policial em um clima de punitivismo e prazer com o
exótico. Foi nesse mesmo ano de 1871 que Machado ambientou \textit{Esaú e
Jacó}. O romance abre com duas senhoras de Botafogo subindo o Castelo.

``Toda a gente falava então da cabocla do Castelo, era o assunto da
cidade; atribuíam-lhe um poder infinito, uma série de milagres, sortes,
achados, casamentos''.\footnote{Machado de Assis, \textit{Esaú e Jacó}
  {[}1904{]} (Porto Alegre, L\&PM, 2014), p.59.} Desceram o Castelo após
consulta com a cabocla que realizava vidências sobre as ``coisas
futuras'',\footnote{\textit{Ibidem}, p.~57.} quando Natividade ``tirou da
bolsa uma nota de dois mil-réis, nova em folha, e deitou-a à bacia'' de
um \textit{irmão das almas} (um sujeito que recolhia esmolas para as almas
negligenciadas, mortos desconsolados, sem missa ou sem cerimônia). As
irmãs Perpétua e Natividade são de família de boa posição, residiam em
Botafogo, ``tinham fé, mas tinham também vexame da opinião, como um
devoto que se benzesse às escondidas''.\footnote{\textit{Ibidem}, p.~54.}
Afeitas ao mistério e à encantaria dos cultos populares brasileiros, mas
de uma simpatia seletiva. As esmolas do lacaio servem para que nenhum
distraído desconfiasse que estivessem ao pé do Castelo por outro motivo
que não fosse assistir à missa na igreja de São José. Nem mesmo permitem
que o cocheiro as deixasse no princípio da ladeira, para que não
desconfiassem da consulta. ``Se as descobrissem, estavam perdidas,
embora muita gente boa lá fosse''.\footnote{\textit{Ibidem}, p.~59.}

O número de ``gente boa'' que lá se ia consultar caboclos nas suas
aflições, nas suas desgraças, dos que creem no poder do sagrado, dos que
zombam deles em público, mas ocultamente os requisitam e consultam, era
um número incalculável, como lembra mesmo Nina Rodrigues. Ao lado dessa
reserva cerimoniosa da ``gente boa'' subsistiu seja o mistério que deu
popularidade aos curandeiros na cidade, seja a eficiência para tratar
moléstias leves ou para cuidar das sérias.\footnote{Cf. Gabriela dos
  Reis Sampaio, \textit{A história do feiticeiro Juca Rosa: cultura e
  relações sociais no Rio de Janeiro imperial} --- Doutorado em História
  (Campinas, Unicamp, 2000), p.~241, 244.} De Pai Manoel à Cabocla do
Castelo, existia uma notável disponibilidade de práticos da cura cujo
sucesso não pode absolutamente ser explicado pela carência de recursos
para recorrer a clínicos diplomados. Um público composto pela elite
branca demonstra que erveiros e benzedeiros em geral tinham seus
serviços solicitados não porque não havia médicos e cirurgiões de
prontidão na capital. De acordo com o Almanak Laemmert, de 1850 ---
conforme Flavio Coelho Edler ---, havia nessa capital 235 profissionais
de medicina. Em 1881, entre os ``cidadãos ativos'' da capital (isto é,
entre os 5.928 eleitores por possuírem renda mínima de 400 mil-réis),
encontravam-se 398 médicos.\footnote{Cf. Flavio C. Edler, \textit{Ensino e
  profissão médica na corte de Pedro II} (Santo André, Univ.
  Federal do ABC, 2014), p.~74.}

Se aceitarmos a veracidade do índice, podemos pensar que médicos de
formação acadêmica precisaram travar áspera batalha não só com os
erveiros da terra, mas com a própria burguesia de sobrado, que lhes
furtava o prestígio ou se recusava, sem sentimentalismos, a trocar o
emprego do saber sigiloso das ervas pela oferta de medicina científica.
A propaganda negativa nos jornais e a polícia médica exercida pela
instituição de Higiene Pública são um dos motivos que nos ajudam a
compreender a sensível transformação desse cenário principalmente a
partir do final da década de 1880. Esse primeiro motivo não é estranho
ao conjunto de temas que tentamos desenvolver ao longo deste livro. O
segundo motivo, cuja contribuição impecável se fez presente no trabalho
de Flavio C. Edler --- \textit{Ensino e profissão médica na Corte de Pedro
II} ---, teve a ver com o pequeno alcance da capacidade da medicina
nacional, enquanto especialização técnico-científica, de equacionar e
resolver problemas técnicos de interesse social --- a rigor, o tema das
epidemias. Embora ambos os motivos confluam mais ou menos no mesmo
período,

\begin{quote}
os médicos deveriam alcançar um consenso básico em torno de dois pontos:
a validade dos fundamentos teóricos de seu saber/prática, isto é, de sua
especialidade; e a relevância pragmática das técnicas profiláticas e
terapêuticas. Ora, esse consenso parecia extremamente difícil naquela
conjuntura. Isto porque a organização da Medicina em bases nacionais
deu-se no momento da crise que se abateu nos fundamentos do saber
médico. Ou como expressava um médico baiano, num momento `em que o solo
médico treme e parece querer abrir-se sob nossos pés'.\footnote{\textit{Ibidem},
  p.~58-9.}
\end{quote}

Há um terceiro motivo, que nos ajuda a pensar o que dificultou a
projeção da clínica médica no Brasil em fins do \textsc{xix}, e que aponta para
desafios que envolvem a assimilação da atividade clínica do médico
diplomado pela população: como se obtém a abertura que dará poder de
intervenção ou acesso ao paciente sob a lógica do corpo como organismo?

Barbeiros, sangradores, parteiras, homeopatas, curandeiros, existiu toda
uma divisão social do trabalho conformada ao sistema de necessidades
medicinais daquela sociedade. Ora suscitavam do enfermo uma dietética,
ora certa terapia farmacológica, ora realizavam pequenas intervenções
cirúrgicas. Contudo, o que o uso das folhas suscita de notável dentro do
complexo cultural \textit{Jêje-Nagô} não é tanto a farmacopeia
\textit{yorùbá} e seu valor medicinal. Porque \textit{oògun}, as receitas de
uso medicinal, são apenas uma das várias aplicações às quais se destina
o segredo das folhas.

Gostaria de destacar dois pontos antes de voltarmos a desenvolver o
problema do corpo como \textit{organismo} na clínica.

O desbloqueio do poder fitoterápico das folhas dependia de encantações
transmitidas oralmente, evocadas por um sacerdote cuja atribuição
específica é o conhecimento e utilização das espécies vegetais. É como
explica Pierre Fatumbi Verger: seria até difícil traçar ``uma linha de
demarcação entre os assim chamados conhecimento científico e prática
`mágica'. Isso ocorre devido à importância dada (\ldots{}) à encantação,
\textit{ofò}, pronunciada no momento de preparação ou aplicação das
diversas receitas medicinais, \textit{oògun}''.\footnote{Pierre Fatumbi
  Verger, \textit{Ewé: o uso das plantas na sociedade iorubá} (São Paulo,
  Cia. das Letras, 1995), p.~23.} Fatumbi, em seu \textit{Ewé: o uso das
plantas na sociedade iorubá}, após coletar alguns milhares de receitas,
disponibiliza 447 delas em seu livro, as quais distribuiu nas
categorias: \textit{Oògun} (que inclui cicatrizantes, analgésicos,
sedativos, estimulantes etc.), \textit{Ibímo} (receitas relativas à
fertilidade, gravidez e nascimento), \textit{Orísà} (trabalhos relativos
ao culto aos orixás), \textit{Àwúre} (receitas de uso benéfico, como
obtenção de sorte, prosperidade, conquista amorosa), \textit{Àbilú}
(receitas de uso maléfico) e \textit{Idáàbòbò} (poções e antídotos contra
as maldições). Quer dizer que para um cirurgião africano como Pai
Manoel, ou o histórico Juca Rosa --- o Pai Quibombo\footnote{Talvez o
  mais reconhecido sacerdote e cirurgião afro-brasileiro da segunda
  metade do \textsc{xix}. Em meados da década de 1870, atendia na Rua do Núncio,
  perto da Rua Senhor dos Passos. Gabriela dos Reis Sampaio,
  historiadora, dedicou-lhe a tese \textit{A história do feiticeiro Juca
  Rosa}, 2000.} ---, ou inúmeros outros que atendiam em \textit{zungús} ou
fundos de estalagens, as propriedades das folhas delegaram ao negro a
tarefa de medicalizar-se com seus próprios recursos, mas não só isso.
Roger Bastide, no livro \textit{As religiões africanas no Brasil}, explica
como os ``mistérios para o preparo de filtros de amor'' permitiram às
escravizadas ``desforrarem-se do desprezo das patroas brancas'', ou
mesmo preparar venenos (\textit{àbilú}) ``que enfraqueciam o cérebro dos
senhores, fazendo-os cair em inanição e morrer lentamente (chamavam-se
estas plantas venenosas de `ervas para amansar os senhores')''. Ou
enfim, certas prescrições compostas por três, seis ou mais folhas
diferentes que faziam ``abortar as mulheres grávidas para não aumentar o
número de escravos''.\footnote{Roger Bastide, \textit{As religiões
  africanas no Brasil} (São Paulo, Livraria Pioneira Editora/EDUSP,
  1971), p.~79, citado em José F. P. Barros, \textit{O segredo das
  folhas}, 1993, p.~39.} Portanto, um primeiro aspecto notável: tendo o
colonialismo submetido o escravizado ao desabamento cognitivo, à
desordem das memórias, à quebra das origens, ao trauma da partida; a
herança daqueles que se reconstituíram a partir dos cacos e conseguiram
driblar a matança física e epistemológica são, no mínimo, estratégias
radicais de guerra e sobrevivência. Só que aqui o campo de batalha se
confunde com o campo de mandinga, como explicam Simas e Rufino. ``O
campo de batalha é o lugar das estratégias, já o de mandinga é onde se
praticam as frestas. A partir das sabedorias aqui reivindicadas não há
batalha sem mandinga e mandinga sem batalha''.\footnote{Luiz Antonio
  Simas, \textit{Fogo no mato}, 2018, p.~105.}

Uma segunda coisa --- que nos foi transmitida no templo \textit{Ifádàrá} de
Pedra de Guaratiba, pelo nigeriano \textit{Bàbáláwo} \textit{Thomaz}
\textit{Ifámoagum}, em uma conversa em fevereiro de 2019. Assim como os
demais elementos do culto \textit{nagô}, o sistema de orientação para o
uso das folhas não faz sentido como ciência quando desarticulado da
experiência ritual. A consulta com o sacerdote cirurgião é antes de tudo
uma consulta a \textit{Ifá} (o sistema divinatório \textit{yorùbá}). Durante
a consulta, \textit{Òrúnmilá}, o orixá do destino, envia uma resposta à
demanda apresentada. A resposta, assim como o remédio, embora seja
exclusiva daquele caso, não é única nem definitiva, mas circunstanciada.
Quer dizer que quando \textit{Ifá} envia \textit{oògún}, ele não envia um
dicionário doméstico de medicina. De qualquer modo, chega-se ao
\textit{oògún} quando todo o procedimento ritualístico é obedecido. Quer
dizer que --- como me explicou o \textit{Bàbáláwo} --- ``uma mesma
preparação de folhas pode fazer efeito nas mãos de um sacerdote que sabe
o que está fazendo, e pode não fazer efeito nas mãos de quem não sabe''.
Claro que a cantiga adequada não pode deixar de ser dita para que se
libere o \textit{àse} e o remédio funcione, mas há, da parte do sacerdote,
outros requisitos para que se produza a eficácia (como por exemplo a
forma como se coletam as espécies: as folhas devem ser colhidas pela
manhã bem cedo, é preciso chamar as folhas pelos seus nomes quando se
entra no mato, deve-se na ocasião mascar grãos de pimenta-da-costa para
reforçar o poder da fala etc.). Cabe, por outro lado, ao consulente
prestar a devida oferenda ou sacrifício em homenagem a \textit{Òrúnmilá}
pelo remédio obtido. Logo, o conhecimento só tem significado quando
incorporado de modo ativo, ou até mais do que isso: não são só os
segredos de \textit{Òsányìn} que se encontram indissociáveis da
experiência iniciática. Digamos que não só a sociedade \textit{egbè} ---
cuja presença não se limita ao espaço físico do terreiro ---, como o
corpo do iniciado e também ``todos os objetos rituais contidos no
`terreiro', dos que constituem os `assentos' até os que são utilizados
de uma maneira qualquer no decorrer da atividade ritual, devem ser
consagrados, isto é, ser portadores de \textit{àse}''.\footnote{Juana
  Elbein dos Santos, \textit{Os Nàgô e a morte}, 1986, p.~37.}

Quando do processo iniciático se ``faz cabeça'', passa-se a incorporar a
corporeidade do \textit{egbé} (da comunidade de terreiro) e nesse
movimento reafirmamos não o conteúdo sagrado do culto apenas, mas o
encantamento do todo. Tudo parte primordialmente do rito. É rito o
momento em que se é impregnado de \textit{àse}, e é em função de sua
conduta ritual que o corpo passa a receptor e impulsor de \textit{àse}.
Quando se ``faz cabeça'' constitui-se o ``corpo como rito'' --- tal como
é rito a vida no \textit{egbè}, o que permite o caráter de um destino
pessoal. Daí, o mais importante: destravamentos de cômodos de realidade
ativados na experiência ritualística de expansão do que está dado,
porque tanto as coisas orgânicas quanto as inorgânicas, tanto as
minerais quanto as humanas passam a ser ingredientes
encantados,\footnote{Cf. Vladimir José de Azevedo Falcão, \textit{Ewé, Ewé
  Ossain --- Um estudo sobre os Erveiros e Erveiras do Mercadão de
  Madureira} (Rio de Janeiro, Barroso Edições/Ilú Aye, 2002), p.~34.}
com um detalhe a ser dito: por detrás das múltiplas máscaras das coisas,
espreita a matéria primordial e sagrada à qual foram subtraídas as
coisas --- \textit{àse}. E como o \textit{àse} é territorialmente assentado
(o \textit{egbé} e o terreiro físico são locais que contraem ``por
metáfora espacial o solo mítico da origem e faz equivaler-se a uma parte
do território histórico da diáspora''),\footnote{Muniz Sodré,
  \textit{Pensar nagô} (Petrópolis, Vozes, 2017), p.92.} o indivíduo, em
vida, não está em face de sua própria finitude, recolhido no drama de
seus sucessos íntimos e fracassos. O indivíduo, o seu corpo como centro
de inscrição de \textit{òrìsás}, está em face da finitude da corporeidade
de seu \textit{àse} ancestral no \textit{egbé}. Ele está radicalmente em
face ``àquilo que se encontra, vindo de nossos ancestrais, quando
chegamos ao mundo''.\footnote{\textit{Ibidem}, p.~89.} Aí reside um papel
de formação ética poderosíssimo do sistema de pensamento Jêje-Nàgô, e
que é aparentemente ininteligível para esse outro sistema cultual,
ritualístico e universalizante --- e muito mais familiarizado em uma
cultura globalista ---, o qual chamamos capitalismo.\footnote{Que outros
  ritos mobilizamos, homens libertários de um mundo técnico, para além
  dos rituais liberais de compra e venda entre sujeitos de direito?
  Nenhum. Não estabelecemos nenhum outro tipo de relação ritualística
  com o corpo ou com a remessa imanente da nossa ancestralidade. A maior
  astúcia do capitalismo foi conseguir territorializar todo um horizonte
  de modo a cegar para a historicidade das suas determinações fáticas.
  Equiparado, por assim dizer, a um sistema religioso, seria uma
  experiência ritualística sem o mistério e que, na medida em que achata
  o que chamamos ``realidade'' ao nível do funcionamento de mercado,
  zela pela retração de tudo cuja abertura comprometeria seu caráter de
  fundado e constituído. A totalidade concreta capturada pela
  mundialização do capital obscureceu o perfil transitório de formas de
  objetividade de nossos fenômenos sociais e fez crer que suas
  determinações são categorias intemporais, comuns a todas as formas de
  vida social. Para dizer de outro modo, \textit{há no capitalismo a
  tendência a servir de medida para a própria desmesura da diferença
  ontológica.} Existirão portanto mais códigos --- e esses códigos
  entrariam na escala da produção de valor e do consumo --- do que fluxos
  à espera de serem codificados, mais olhos do que coisas a serem
  vistas, mais objetos do que sujeitos desejantes, menos tempo
  desperdiçável do que tempo disponível, ``civilização em excesso, meios
  de subsistência em excesso, indústria em excesso, comércio em
  excesso''. Karl Marx, \textit{Manifesto do partido comunista}
  {[}1848{]}, trad. Sueli Cassal (Porto Alegre, L\&PM, 2002), p.~34.
  Marx --- depois Benjamin e Agamben --- percebeu como o capitalismo é ao
  mesmo tempo aquele que testemunha a queda da transcendência de Deus e
  aquele que o substitui. Esse Deus não está morto, estando implicado no
  destino do homem sob a forma de outra natureza e possuindo uma
  magnífica potência de recuperação, uma reserva magnífica de força
  plástica, que se manifesta sempre que algum acontecimento tensiona
  seus limites. Sempre que algo fura bloqueios ele devora. Trata-se de
  um Deus ausente, mas implicado em seu próprio compromisso universal
  com um sentimento burguês de falta, culpa, dívida. Ele é o ``reflexo
  religioso do mundo real'' (Karl Marx, \textit{O Capital} --- Livro I
  {[}1867{]}, trad. Rubens Enderle, São Paulo, Boitempo, 2013, p.~154)
  que possibilita que um ``processo de produção domine os homens, e não
  os homens o processo de produção'' para ser então considerado pela
  ``consciência burguesa como uma necessidade natural'' (\textit{Ibidem,}
  p.~156). A pergunta então seria: foi o capitalismo totalizador a ponto
  de impedir outras práticas de ritualização do corpo e encantamento do
  nosso horizonte pós-colonial? A pergunta é retórica e não é grande
  coisa. Também não é necessário nos estendermos. Ou melhor: acaso
  batemos no teto do que nos foi transmitido pela nossa ancestralidade?}

De uma vez por todas, não são só as práticas dos cirurgiões
afrobrasileiros\footnote{Em uma aula aberta realizada pelo Departamento
  de Filosofia da UFRJ, em março de 2019, Simas nos sugeriu que o
  curandeirismo carioca, que na historiografia costuma aparecer
  associado à matriz \textit{Nagô}, mereceria um dia ser reavaliado
  segundo os procedimentos de cura ameríndios. Há muitos estudos sobre
  as ervas, mas ao mesmo tempo várias folhas não tinham exemplares no
  continente africano --- foram então transmitidas pelas tradições
  ameríndias, e demarcam a presença tupi na medicina popular. O
  professor cita a história emblemática de Zé Pilintra. Nos documentos
  de apreensão policial (a prática será alvo de processos judiciais até
  pelo menos 1949), teria sido comum a menção ao catimbó e à jurema. Os
  mestres da jurema são catimbozeiros. Ora, Zé Pilintra é o mestre do
  catimbó, foi iniciado na jurema por um indígena Cariri no nordeste
  brasileiro, antes de migrar para a atual Lapa e firmar residência no
  figurino da malandragem.} do \textsc{xix} que são indissociáveis de um poder de
realização possibilitado pelo encantamento da realidade. O mesmo é
``válido para a consagração de cada `assento' ou objeto ritual, para a
elaboração do \textit{àse} que será `plantado' em cada iniciado, para a
seleção das oferendas a serem sacrificadas em cada circunstância
ritual''.\footnote{Juana Elbein dos Santos, \textit{Os Nàgô e a morte},
  1986, p.~43.} Logo, esse traço fundamental para uma ``ciência
encantada das macumbas'', a experiência do ``corpo como rito'', a
existência como figura de encantamento, são coisas bastante estranhas às
concepções modernas de corpo como \textit{organismo} ou de saúde como
\textit{higiene}.

O corpo antes do rito não possui necessariamente pessoalidade, ele é
pré-pessoal, pré-histórico, enquanto ainda não se constituiu como
endereço para a ancestralidade. Talvez por isso que, enquanto não
pudermos pensar candomblés e umbandas como religiões \textit{históricas},
não deveríamos pensar o complexo cultural Jêje-Nàgô dentro dos moldes
das religiões. O corpo não nasce acabado, mas não é complicado
interpretá-lo como já, propriamente, lugar de interseção de \textit{àse.}
A rigor, entre o povo de candomblé, o verdadeiro sujeito é o \textit{àse}.

Tampouco em Nietzsche ou Foucault o corpo nasce pronto, mas lá é um
arranjo de investimentos de poder e saber que ajudam a definir para si
um destino --- e figuras de encantamento, em se tratando do povo de
\textit{àse}. Mapear como o corpo é espacializado em nossas sociedades
biopolíticas é, a rigor, estabelecer que ele não é anterior à história,
mas dela ele se fez efeito. Esse mapeamento é a atividade central para
uma filosofia que está a postos no ``ponto de articulação do corpo com a
história. Ela deve mostrar o corpo inteiramente marcado de história e a
história arruinando o corpo''.\footnote{Michel Foucault,
  \textit{Microfísica do poder}, 1979, p.~22.}

Falamos no primeiro capítulo sobre como as velhas estruturas
hospitalares não eram espaços de cura, mas casas de recolhimento e
assistência social para a população pobre. Em algumas décadas esse
domínio hospitalar será aquele em que o fator patológico aparecerá na
singularidade de um acontecimento que é da alçada da experiência
clínica, porque a partir da década de 1860 há uma novidade para a
população do Rio de Janeiro: surgem as ``Casas de Saúde''. Somente no
segundo semestre de 1862, serão criadas quatro delas, totalizando um
conjunto de nove. Ao contrário dos hospitais --- Sta. Casa de
Misericórdia, Beneficência Portuguesa e das Ordens de São Francisco de
Paula, Terceira da Penitência e Carmo ---, as Casas de Saúde
destinavam-se a uma clientela rica. Já não bastará ao médico a instrução
doutrinária para que se possa tratar de doentes; é preciso que aos
conhecimentos teóricos ele reúna outros que só lhe podem ser fornecidos
pela experimentação. Para Torres Homem --- lente de clínica médica na
Faculdade de Medicina e chefe da clínica interna nesse mesmo período ---,
as únicas fontes de tais conhecimentos são ``as enfermarias de um
hospital. Aí se apresentam as moléstias em grande número, com todas as
suas variedades; com muita facilidade pode ser verificada a exatidão das
doutrinas que os livros ensinam''.\footnote{João V. Torres Homem,
  \textit{Elementos da Clínica Médica\ldots{}}, 1870, p. 19.} O princípio
de que o saber-poder médico se forma no próprio leito do doente data
desse contexto e por uma razão que consideramos precisa. O corpo doente
ganha profundidade, é lugar de um teatro, o corpo doente manifesta na
sua própria convalescência e precariedade uma plataforma de
investimentos. Uma vez doente, acolhido em uma casa de saúde e
setorizado em um quadro clínico, individualizado no leito, ele ganha
espessura de objeto.

Do corpo somático e epidérmico, dessa materialidade corporal à qual o
patriarcado referia interdições morais ligadas ao toque e à visão,
começa a se oferecer um domínio novo de exercício de poder e
objetivação: um \textit{corpo} \textit{biológico}, um \textit{organismo}. E,
justamente, essa qualificação do corpo como domínio da experimentação e
realidade biológica passível de se objetivar é coisa estranha ao
cotidiano das práticas aculturadas naquela sociedade. Tudo isso é
correlativo ao que podemos chamar, com Foucault, de um procedimento de
\textit{exame}. O aparecimento da clínica no Brasil, como no mundo, pode
ser identificado --- embora certamente não se esgote aí --- ``na mudança
ínfima e decisiva que substitui a pergunta `o que é que você tem?'
(\ldots{}) por essa outra em que reconhecemos o jogo da clínica e o
princípio de todo seu discurso: `onde lhe dói?'\,''.\footnote{Michel
  Foucault, \textit{O Nascimento da Clínica} {[}1963{]}, 1977, p. \textsc{xviii}.}
Entra em discussão o \textit{corpo biológico} atravessado pelo direito ao
exame, submetido ao laboratório da cura, à obrigação de ser examinado,
auscultado, vacinado, tocado, percussionado. E creio que será fazendo a
história das relações entre o corpo e as tecnologias de poder que o
investem que podemos compreender algumas das razões das dificuldades que
existiam para neutralizar erveiros e mateiros na paisagem das práticas
de cura na cidade. Meu corpo, sobre o qual eu assegurava algum arbítrio
para dizer \textit{não}, corre agora o risco de ser, por esse outro modo
de gestão do corpo, biologizado por um poder laico e científico. O corpo
do qual o saber médico possuirá em breve a jurisdição (lembremos das
forças que desencadearão uma Revolta da Vacina) não foi apenas
elaboração da filantropia médica ou do sacerdócio do homem de ciência,
foi um domínio recortado e percorrido por um tipo de poder inovador para
a época.

Como se obtém a abertura que dará ao médico poder de intervenção ou
acesso ao paciente sob a lógica do corpo biológico, do corpo como
organismo? Há um romance de José de Alencar chamado \textit{Diva}, escrito
em 1864, que conta como um jovem médico é chamado por um amigo a prestar
socorro à irmã que adoecia. O médico, um tipo distinto, de ``natureza
crioula de sangue europeu'', veio recém-chegado de Paris, o ``estágio
quase obrigatório dos jovens médicos brasileiros''. A infeliz ardia em
febre e sentia pontadas sobre o coração. ``Todos os sintomas pareciam
indicar uma afecção pulmonar''. Emília recolhida na cama, em estado
letárgico, a família aflita, e uma tia que lhe prestava cuidados de
cabeceira.

\begin{quote}
--- Minha senhora, disse eu, é necessário auscultar-lhe o peito. ---
Então, Sr.~doutor, aproveite enquanto ela dorme. Se acordar, nada a fará
consentir. A senhora afastou a ponta da cobertura, deixando o seio da
menina envolto com as roupagens de linho. Mal encostei o ouvido ao seu
corpo, teve ela um forte sobressalto, e eu não pude erguer a cabeça tão
depressa, que não sentisse no meu rosto a doce pressão de seu colo
ofegante. O que passou depois foi rápido como o pensamento. Ouvi um
grito. Senti nos ombros choque tão brusco e violento, que me repeliu da
borda do leito. Sobre este, sentada, de busto erguido, hirta e
horrivelmente pálida, surgira Emília. Os olhos esbraseados cintilavam na
sombra: conchegando ao seio com uma das mãos crispadas as longas
coberturas, com a outra estendida sob as amplas dobras dessa espécie de
túnica, ela apontava para a porta. --- Atrevido!\dots{}\footnote{José de
  Alencar, \textit{Diva} {[}1864{]} (Rio de Janeiro, Ed. Letras e Artes,
  1964).}
\end{quote}

Existem os protocolos sobre como se deve conduzir um interrogatório,
existe o esquema do inquérito ideal. O princípio de que o saber médico
se forma no leito do doente é anterior ao \textsc{xix}, mas certas perguntas que
``obriguem uma mulher honesta a corar e perturbar-se''\footnote{João V.
  Torres Homem, \textit{Elementos da Clínica Médica\ldots{}}, 1870, p. 50.}
exigem a integração, na experiência, da visibilidade hospitalar como
maneira de obter, intensificar e extrair poder. Convém que o médico
dirija o olhar atento para todas as regiões. São comuns os erros a que
se expõe quando não se procede ao ``exame da parte afetada sem roupa
alguma''.\footnote{\textit{Ibidem}, p.~47} Só que ``infelizmente este
exame tão útil não pode ter lugar senão em muitos poucos casos. Nos
hospitais só pode ser feito nos homens, porque a decência o impede nas
mulheres; na clínica civil, a não ser em circunstâncias especiais, ele
nunca é praticável''.\footnote{\textit{Ibidem}, p.~47} A enfermaria é uma
máquina de fazer-ver espelhada em uma arquitetura. Dentro dessa
distribuição de luzes é possível assegurar um funcionamento contínuo e
automatizado para certo jogo desproporcional de fluxos de forças.

É absolutamente formidável como esse sistema de produção de
visibilidades induz objetos à especialidade de fornecerem mais-poder. E
aí o paciente é tanto foco sobre o qual se opera uma relação, quanto um
suporte ou catalizador que promove a si próprio como intensificador de
um mais-poder. Quer dizer que o mais-poder médico supõe uma máquina de
fazer-ver, técnicas ópticas que orientem a um regime de luzes. Mas esse
regime de luzes, que converterá poder em mais-poder, virá acoplado a um
sistema de documentação para produção de um mais-saber. E trata-se de um
mais-saber que fará do médico depositário de ``segredos importantes de
que depende muitas vezes o futuro de famílias inteiras; diante dele não
há mistérios, o véu que os encobre se rompe para mostrar-lhe a
verdade''.\footnote{\textit{Ibidem,} p.~50.} Assim, por exemplo, Torres
Homem e Pinel estarão de acordo sobre o fato de que não basta saber
``onde dói'', também não basta reservar a anatomia do corpo ao olhar
minucioso. A anatomoclínica não só atravessa a barreira do toque como
leva o organismo a comunicar o que a inteligência do paciente não sabe
dizer. No limite desse encontro entre o médico e o doente, se é levado a
confessar tanto aquilo que precisa ser resgatado de um trabalho
minucioso de rememoração quanto o que não se pensa senão sob as formas
do despudor e do segredo. Por exemplo,

\begin{quote}
quando se procura saber de uma moça solteira qual o estado da
menstruação, como se desenvolveu, quanto tempo dura, qual a quantidade e
natureza do sangue perdido, quais as perturbações e irregularidades que
aparecem; quando, interrogando uma mulher casada, se lhe pede
informações sobre as prenhezes precedentes, as circunstâncias
comemorativas de seus partos, a idade crítica; quando, junto a um moço
bem educado, se lhe pergunta pelos hábitos do onanismo; a um homem
sério, pelos excessos venéreos, pelos acidentes sifilíticos que podia
ter contraído etc., nestes casos o médico deve observar todas as
conveniências, o menor descuido em sua linguagem pode comprometê-lo e
prejudicar sua reputação. Regra geral, quando se trata de moléstias
venéreas, nunca se deve interrogar o marido na presença da mulher, ou
esta diante daquele: evita-se deste modo que a paz doméstica seja
perturbada.\footnote{\textit{Ibidem}, p.~50.}
\end{quote}

A historiografia reconhece em detalhes o higienismo quando fechado na
sua institucionalidade, mas não se trata disso, pois a higiene não é
causa dela própria, assim como a experiência da epidemia não é o que a
Junta diz sobre a epidemia. Mas se nos for permitido pensar que o
organismo está para a anatomoclínica tal como o corpo higiênico está
para as práticas higienistas, então ambos não passam de aspectos
parciais de uma objetivação do corpo que é mais ampla politicamente.
Essa objetivação biológica do corpo é resultado da explosão de
investimentos políticos alinhados com uma lógica de achatamento e
desencante do mundo. Chegará o dia em que o organismo e o corpo
higiênico serão não apenas aceitáveis, mas óbvios, estranhos à
historicidade dos seus fundamentos. Veremos a seguir como a rotina das
pequenas tensões que envolvem instituições de higiene e moradores de
cortiços foram formas de exterioridade de uma objetivação biológica do
corpo que esteve longe de se generalizar apenas pela difusão de uma
consciência moral, ou pela enunciação de o que é saúde ou o que ela
deixa de ser. O que é ``forma de exterioridade''?\footnote{Gilles
  Deleuze, \textit{Michel Foucault: as formações históricas}
  {[}1985-86{]}, 2018, aula 7.} Remete à ideia de que esses temas não
terminam na hospitalização do enfermo como a única janela que dá ao
clínico poder de acesso ao paciente-organismo. Historicamente os meios
de hospitalização foram ``etapas transitórias, são variáveis de uma
função de exterioridade. São, ao pé da letra, variáveis de uma função de
exterioridade e as funções de exterioridade podem prescindir das
variáveis''.\footnote{\textit{Ibidem,} aula 7.} O que importa não são as
instituições exatamente, são as funções de exterioridade diversas que se
servem, por exemplo, dos meios de quadriculamento do enfermo. São os
recursos para a proliferação das tecnologias de poder sobre o corpo no
tecido social, a qual gerou, por exemplo, um corpo higiênico, a casa
higiênica, a rua salubre, a cidade esteticamente inaceitável.

A clínica está para as formas de objetividade nos leitos dos pacientes
assim como a higiene está para as objetividades que circulam nos espaços
urbanos (os miasmas, a epidemização da miséria, a fisionomia das ruas
etc.). Mas ainda corremos o risco da simplificação. Porque será afinal
da tecnologia moral do corpo, desse tipo de relação de força que integra
o corpo, que terão origem o dispositivo médico-higienista e a
experiência da epidemia que o constitui em fins do \textsc{xix}. Sim, a proporção
entre as práticas clínicas e higienistas é simplificadora demais, não só
porque uma coisa depende filosoficamente da outra, mas porque são comuns
os casos em que acontecem atravessamentos, capturas mútuas, somas e
engavetamentos entre as duas práticas. Optamos por enfatizar o tema da
polícia médica, os instrumentos de fiscalização dos quais se serviam as
instituições de higiene para enquadrar certo gênero de charlatanismo: os
curandeiros. Eles não desapareceriam por completo. Migram para os
subúrbios, territórios em fuga.

O destino dos curandeiros nos últimos dias do Rio de Janeiro como Corte
do Império é testemunhado por Policarpo. Policarpo é um personagem
criado por Machado em um período em que escrevia para o \textit{Gazeta de
Notícias}. Transcrevemos na íntegra a crônica de Machado, que apareceu
em 14 de junho de 1889. Ela perderia parte da força se fosse resumida.

\begin{quote}
\textit{BONS DIAS!}

Ó doce, ó longa, ó inexprimível melancolia dos jornais velhos!
Conhece-se um homem diante dum deles. Pessoa que não sentir alguma coisa
ao ler folhas de meio século, bem pode crer que não terá nunca uma das
mais profundas emoções da vida − igual ou quase igual à que dá a vista
das ruínas de uma civilização. Não é a saudade piegas, mas a
recomposição do extinto, a revivescência do passado, à maneira de Ebers,
a alucinação erudita da vida e do movimento que parou.

Jornal antigo é melhor que cemitério, por esta razão que no cemitério
tudo está morto, enquanto que no jornal está vivo tudo. Os letreiros
sepulcrais, sobre monótonos, são definitivos: \textit{aqui jaz, aqui
descansam, orai por ele!} As letras impressas na gazeta antiga são
variadas, as notícias aparecem recentes; é a galera que sai, a peça que
se está apresentando, o baile de ontem, a romaria de amanhã, uma
explicação, um discurso, dois agradecimentos, muitos elogios; é a
própria vida em ação.

Curandeiros, por exemplo. Há agora uma verdadeira perseguição deles.
Imprensa, política, particulares, todos parecem haver jurado a
exterminação dessa classe interessante. O que lhes vale ainda um pouco é
não terem perdido o governo da multidão. Escondem-se; vão por noite
negra e vias escuras levar a droga ao enfermo, e, com ela, a consolação.
São pegados, é certo; mas por um curandeiro aniquilado, escapam quatro
ou cinco.

Vinde agora comigo.

Temos aqui o \textit{Jornal do Commercio} de 10 de setembro de 1841. Olhai
bem: 1841; lá vão quarenta e oito anos, perto de meio século. Lede com
pausa este anúncio de um remédio para os olhos: ``\dots{} eficaz remédio,
que já restituiu a vista a muitas pessoas que a tinham perdido, acha-se
em casa de seu autor, o Sr.~Antônio Gomes, Rua dos Barbonos, nº 76''.
Era assim, os curandeiros anunciavam livremente, não se iam esconder em
Niterói, como o célebre caboclo, ninguém os ia buscar nem prender;
punham na imprensa o nome da pessoa, o número da casa, o remédio e a
aplicação.

Às vezes, o curandeiro, em vez de chamar, era chamado, como se vê nestas
linhas da mesma data:

``Roga-se ao senhor que cura erisipelas, feridas, etc., de aparecer na
Rua Valongo nº 147''.

Era outro senhor que esquecera de anunciar o número da casa e da rua,
como o Antônio Gomes. Este Gomes fazia prodígios. Uma senhora conta ao
público a cura extraordinária realizada por ele em uma escrava, que
padecia de ferida incurável, ao menos para médicos do tempo. Chamado
Antônio Gomes, a escrava sarou. A senhora tinha por nome D. Luísa Teresa
Velasco. Também acho uma descoberta daquele benemérito para impigens,
coisa admirável.

Além desses, havia outros autores não menos diplomados, nem menos
anunciados. Uma loja de papel, situada no Rua do Ouvidor, esquina do
Largo São Francisco de Paula, vendia licor antifebril, que não só curava
a febre intermitente e a enxaqueca, como era famoso contra cólicas,
reumatismo e indigestões.

De envolta com os curandeiros e suas drogas, tínhamos uma infinidade de
remédios estrangeiros, sem contar as famosas \textit{pílulas vegetais
americanas}. Que direi de um \textit{óleo Jacoris Asseli}, eficaz para
reumatismo, não menos que o \textit{bálsamo homogêneo simpático}, sem nome
de autor nem indicações de moléstias, mas não menos poderoso e buscado?

Todas essas drogas curavam, assim as legítimas como as espúrias. Se já
não curam é porque todas as coisas deste mundo têm princípio, meio e
fim. Outras cessaram com os inventores. Tempo virá em que o quinino, tão
valente agora, envelheça e expire. Neste sentido é que se pode comparar
um jornal antigo ao cemitério, mas ao cemitério de Constantinopla, onde
a gente passeia, conversa e ri.

Plínio, falando da medicina em Roma, afirma que bastava alguém dizer-se
médico para ser imediatamente crido e aceito; e suas drogas eram logo
bebidas, ``tão doce é a esperança!'' conclui ele. O defunto Antônio
Gomes e os seus atuais colegas bem podiam ter vivido em Roma; seriam lá
como aqui (em 1841) verdadeiramente adorados. Bons curandeiros! Tudo
passa com os anos, tudo, a proteção romana e a tolerância carioca; tudo
passa com os anos\dots{} ó doce, ó longa, ó inexprimível melancolia dos
jornais velhos!

\textit{BOAS NOITES.}\footnote{Machado de Assis, \textit{Bons dias!},
  introdução e notas de John Gledson, 3ª ed.~(Campinas, Ed. da Unicamp,
  2008), p.~273-275.}
\end{quote}

No que dependesse das elites médicas, a tarefa de refundação de um país
escravocrata e a indução a uma sociedade formalmente moderna se daria
antes pelos efeitos da adoção de uma cartilha de valores cosmopolitas
como rota fuga do passado, em direção à terra prometida pelo
positivismo. Está de prova o projeto de marginalização social e racial
com suas faíscas de sucesso. Curiosamente o Brasil não se soube ainda um
território pós-colonial escravocrata e, nas poucas vezes em que se
reencontrou consigo nos complexos de saberes não europeus, não viu aí
estratégias de defesa contra o desaparecimento, muito menos
agenciamentos de homens que fizeram sua própria história, mas objetos de
estudo que completam o herbário das ciências humanas.

Este capítulo teve a ambição de recompensar a nostalgia do Policarpo de
Machado com algum fôlego de entusiasmo com o Brasil marginal, não
oficial. Por isso eu já não sei se abrimos jornais velhos com a
``alucinação erudita da vida e do movimento que parou''. Os curandeiros
foram a aposta metodológica que fizemos (há inúmeros outros quilombos de
saberes, que desde as vésperas do presente alimentam nossa potência de
agir) de que dá para pensar algo no limite das fronteiras entre cidades.

Ainda não batemos no teto do que nos foi transmitido, e não se é
totalmente engolido pela lógica do capital quando se opera nos espaços
vazios deixados. A Cidade Velha lega a experiência de que não há
dialética possível entre valores e espaços historicamente incompatíveis.
Ou melhor: o que herdamos, mesmo que pálido e com uma força propulsora
raramente acessada, veio de um mesmo estoque de vida profana e
ingovernável. Boa fatia do nosso colapso como sociedade eu acho que veio
da tentativa furiosa de querer castrar o ingovernável. Em troca do quê?
A segurança da vida de prédio, o bacharelismo e a atual californização
do fascismo se abraçam em desespero, nunca foram capazes de produzir
para o Brasil nada de importante. Nem abriram mão do purismo, do sacro,
do higiênico, do clássico. O que produzem é limitado a seus círculos de
suspeita e recessão. Quanto aos subúrbios para onde migrarão nossos
curandeiros: vai governar o quê aí? Sobre Pai Manoel: sequer há rastro
ou contorno dessa primeira natureza de civilizadores e fermentadores de
Brasis aos quais presto minhas oferendas? No céu da Boca do Mato, Serra
dos Pretos Forros, na voz de um velho, o verde musga o segredo das
folhas.

\section*{Um Estado em outro Estado: a liberdade exagerada no viver e fazer}

\begin{quote}
Gilberto Freyre condenava a Avenida Central, elogiando ruas estreitas
como a do Ouvidor, cheias de sombra e portanto mais adequadas ao calor
tropical. E fazia a apologia do morro da Favela como um exemplo de
``restos do Rio de Janeiro antes de Passos, pendurados por cima do Rio
novo''.
\end{quote}

(\textit{O mistério do samba,} Hermano Vianna)

Antônio Vianna (1883-1952), chamado ``cronista da cidade'' de Salvador,
foi folclorista na primeira metade do \textsc{xx} e escreveu algumas páginas
sobre a atividade privada de facultativos e agentes de higiene pública.
Algumas décadas depois que o médico da \textit{Diva} de Alencar foi
escorraçado pelo atrevimento de auscultar uma febril, Vianna celebrava
os preparativos para recepção de um doutor em casa de família abastada
ou, no mínimo, decente. É de um completo desvelo e obsessão. Por isso o
adotamos. Ao aviso de ``Chegou o Doutor!'', acontecia algo mais do que o
socorro de um agoniado.

A visita despertava inusitada atividade, tudo na casa convocado ``para o
fim último de ter tudo em ordem material e moralmente''.\footnote{Antônio
  Vianna, \textit{Quintal de nagô e outras crônicas} (Salvador, Publicação
  da Universidade Federal da Bahia, 1979), p.~29.} Ninguém era pego de
surpresa por mais insólita que fosse a doença. Peças adquiridas, mas de
uso compulsório desapareciam no fundo de baús. Preparavam-se às pressas
os materiais do ``arsenal higiênico''\footnote{\textit{Ibidem}, p.~29.}: a
roupa de cama novíssima, perfumada com uma preparação de sândalo e
alfazema, a toalha de pano de preço engomada, um jarro de louça para a
água das mãos e um sabonete de cheiro, cálice de fino vinho do Porto!
Pianos polidos apresentáveis, ``caderno e papel pautado para a receita,
a caneta e a pena, como vieram do armarinho. As pessoas mais velhas a
postos no quarto a visitar. Os serviçais a vigiarem as crianças para que
não fizessem bulha''.\footnote{\textit{Ibidem,} p.~30.} O silêncio
respeitoso, mas ofegante dos velhos em fileira, enquanto o esculápio
desce do cavalo, janelas em alvoroço. Depois um moleque que sai a correr
para buscar o tempero esquecido pela cozinheira. Constitui ``espetáculo
obrigatório a permanência do médico no local''.\footnote{\textit{Ibidem},
  p.~31.} Nem que fosse uma palestrinha pelo período de um café.

Vianna atribui o exagero de fatos ao costume provinciano do
soteropolitano. E diz que os ambulatórios de assistência social
alteraram muito o cenário de coisas. Mas as praxes higiênicas modernas
não apagarão do mapa a ``auréola divinatória''\footnote{\textit{Ibidem,}
  p.~31} do doutor nem da memória das casas o compromisso com os banhos
e cloros. Tudo isso são quadros muito bonitos de infância, mas é de se
perguntar como eram em dias comuns as alcovas burguesas. Ele não diz.
Mas podemos presumir porque no mesmo livro de crônicas Vianna desce dos
sobrados e vai ver a casa da ``pobreza derrancada''. E aí mora o
escândalo. Implacável com africanos, ele consagra a expressão que dá
nome ao livro: \textit{quintal de nagô}, que era

\begin{quote}
o comparativo mais ferino que se poderia, antigamente, fazer a uma casa
suja e desarrumada. Seriam, ainda, resquícios preconceituosos dos que
olhavam os pretos com a mentalidade da senzala? Nem sempre. A Bahia
provinciana vivia exposta às invasões de endemias (\dots{}). Conhecia-se a
crônica verbal das grandes epidemias, ninguém duvidava de que pudessem
repetir os terríveis quadros da cólera de 1855, das visitas periódicas
da varíola, do sarampo, da catapora, da tosse convulsa, da papeira, do
sangue novo e, por último, da peste bubônica, que veio assentar, de
chofre, arraial dos domínios clássicos da febre amarela.\footnote{\textit{Ibidem},
  p.~39.}
\end{quote}

A moradia nagô é anti-higiênica, não porque os mucambos repetem a
natureza das senzalas. Essa é questão superada na mitologia do 13 de
maio. A casa suja do africano tem uma raiz histórico-social, não se é
espontaneamente sujo e desarrumado quando não se tem capacidade de
escolha moral. Faltará apenas ao africano a liberdade da qual não gozam
os ignorantes. Para isso é questão de educá-lo na higiene, ensiná-lo a
ser livre. Como? É possível ser modesto sem deixar de ser decente, como
nas famílias emergentes que ``procuravam imitar as abastadas no trato
das coisas domésticas. Encarnavam bem o tipo da pobreza
cheirosa''.\footnote{\textit{Ibidem}, p.~40.}

Há algo novo na estrutura dessa sociedade que substituiu a oposição
entre senhores e escravos, algo que tem a ver com uma ``guerra surda
entre a profilaxia da elite e o fatalismo das classes
humildes''.\footnote{\textit{Ibidem}, p.~40.} As regras raciais do mundo
escravista deixaram de ser prerrogativas no imaginário dos sobrados
porque a República lhes ofertou a mais irrestrita anistia moral. A
República conciliou contrários --- essa é sua força compressora
constituinte ---, e agora a única coisa que nos diferencia é uma estranha
força de ordem econômica. Contra ela não há vítima nem algoz, porque sua
lei é amoral, pré-histórica, não humana. Mas podemos hierarquizar, como
em uma pirâmide, as formas econômicas elementares que reorganizaram as
classes na nova sociedade. Vianna então transforma em fato social a
qualidade do zelo com o assoalho doméstico, esboçando para seus leitores
uma sociologia da faxina.

Assim, no topo da pirâmide, ``os que habitam os altos traziam os
assoalhos lavados, diariamente, com abundante camada de areia. As
paredes e os móveis espanados, quando não protegidos estes por panos de
crochê ou sobrecapas de pano de linho''.\footnote{\textit{Ibidem}, p.~40.}
Logo abaixo dela virá a pobreza cheirosa: ``o chão esfregado a miúde,
mostrava o cimento claro, sem ranhuras, polvilhado de areia alva, para
ajudar a limpeza''.\footnote{\textit{Ibidem,} p.~40.} Na base da pirâmide
moravam os tipos de rua, os que comem em cuias e tigelas de barro.
Carentes de tudo, não têm luxos de varrer assoalhos, porque casa já não
havia, ``permaneciam à beira das calçadas, à sombra das árvores, ao pé
das escadas dos sobrados''.\footnote{\textit{Ibidem}, p.~41.}

Por último os africanos, mas esses constituem ``modelo à
parte'',\footnote{\textit{Ibidem}, p.~41.} não porque abaixo da linha da
indigência, mas porque para eles não teria havido chance de ingresso na
pirâmide social da faxina.

\begin{quote}
Ao penetrar-se-lhes a moradia, ficava-se na incerteza da definição
domicílio ou trapiche. Realizava-se ali a tácita função das duas
finalidades. Moravam ali dezenas de criaturas, acomodadas em bancos,
caixões, tábuas desmontáveis, mesas, tamborete e jiraus de varas e
palha. Ninguém deixaria de dormir sossegadamente. Os corredores
atravancados de volumes exerciam, na prática, as funções de quitandas,
botequins, ponto de reuniões para das decisões mais sérias entre os
parceiros, que compareciam aos conselhos deliberativos da grei.
Conselhos respeitosos, sussurrantes, de resoluções sigilosas a ouvidos
estranhos.\footnote{\textit{Ibidem}, p.~41.}
\end{quote}

Os anos que margeiam o 13 de maio voltarão a revirar antigos pavores
recalcados de que reuniões e conselhos ``sussurrantes, de resoluções
sigilosas a ouvidos estranhos'' agitassem a subcena das ruas. Aquele
domingo de 1888 despertaria opiniões tateantes, como este editorial que,
se não apaziguava ninguém, pelo menos rezava que a ``conduta dos
libertados deve ser tão digna como a ação dos libertadores. (\ldots{})
Não há vencedores nem vencidos, mas pessoas nobilitadas pela glória,
pela felicidade e pelo progresso da grande nação''.\footnote{\textit{O
  Fluminense}, domingo, 13 de maio de 1888, p.~1.} O mal-estar crescia
na medida das queixas dos escravocratas com essa gente isenta de um
sentimento de gratidão com a prova de bondade dos ex-senhores.
Pareceu-lhes não ter sido a abolição a realização de um tensionamento
originário em uma sociedade fraturada. Para eles, o 13 de maio fora
muito mais o evento inaugural de destravamentos de \textit{liberdades
excedidas} que instabilizam relações hierárquicas notabilizadas nas
antigas regras. Um S. Domingos, que pusesse à prova o dia do grande
desabamento da ordem das coisas, não foi possibilidade que se botou para
escanteio. Portanto, quando um Antônio Vianna admite haver uma maioria
que ignora princípios de higiene doméstica e seu ``senso do perigo
constante'',\footnote{Antônio Vianna, \textit{Quintal de nagô} , 1979,
  p.~40.} ele não se refere apenas ao fato de que a carnificina das
epidemias visita todas as camadas da pirâmide. Zungus, estalagens,
quintais de nagô não medem o senso de perigo que representam suas
próprias existências desassistidas. Como garantir uma sociedade na qual
o africano --- que desconhece o bom uso da liberdade --- deixou de estar
sob a mira do chicote? Dentro de algumas décadas a febre amarela não
será mais o que foi outrora e, quando esse dia chegar, a higiene
doméstica da grande ``grei'' africana passará de afronta à salubridade
pública, a um valor do qual compartilham seus sobreviventes.

Defender a sociedade contra a extinção dos mesmos valores sistêmicos que
naturalizaram a escravidão é defendê-la contra o perigo implicado na
existência da habitação da massa da população alforriada ou liberta. O
interessante é que o antigo poder senhorial, a partir de agora,
multiplica-se, ganha vozes e rostos familiares. Pelo menos no Rio de
Janeiro, ele ganha volume de ideologia, alcança figuras de subjetivação
que darão às práticas da polícia e dos agentes de higiene pública um
incremento em termos de capacidade de infiltração. Porque as denúncias
ganham autonomia em relação às instituições de fiscalização, elas vêm
agora do vizinho de porta, às vezes do próprio senhorio. Os preceitos da
higiene solidarizam corações daqueles que, em geral, zelarão pela ordem.
É bem nesse sentido que relemos a ambição de um Adolphe Motard, já
mencionada na apresentação deste livro: a finalidade da higiene geral
não é produzir saúde nos agoniados, mas fazer feliz o homem comum,
moralizar instintos, produzir no sujeito comum uma segunda natureza que
reserve às práticas higiênicas a determinação de uma Moral. Pois bem, a
Higiene se consagra uma Moral não no momento em que falseia o precário
conhecimento sobre as epidemias, mas no instante em que arvora, ou
absorve na economia dos seus enunciados, posições que apenas cegamente
divergem entre si. De maneira que tanto aqueles que querem o fim dos
cortiços pelo prejuízo à paisagem urbana, ou os que se justificam na
falta de desvelo moral, ou ainda quem denunciasse a falta de assistência
pública e o desprezo dos poderes, todos adotam os princípios da higiene
como critério. Mas a essa constelação de motivos virá se somar um
ingrediente estrangeiro, mas nem por isso menos familiar àquela
sociedade: a liberdade em excesso. Por essas e outras, na Niterói de
1889, uma habitação coletiva, ``a estalagem do Lopes, na rua de S. José,
cada dia se vai celebrizando''.\footnote{\textit{O Fluminense}, 26 de maio
  de 1889, p.~2.}

O inspetor de quarteirão ali tem ido discursar em apelo à moralidade.
Depois, caninamente escapole sob um coro de ``frases melífluas e
soantes, capazes de fazer arrolhar o mais duro ouvido''. Nesse projeto
de republiqueta ``aninha-se muita gente de condição duvidosa, parte dela
ninfas de 13 de Maio'' capazes de ``cenas repugnantes, provocações,
desordens e aulas completas do mais decente vocabulário''. São as
próprias famílias da vizinhança que infestam o jornal \textit{O
Fluminense} contra a frouxidão da secretaria de polícia. As denúncias
eram comuns, algumas falam que as ninfas, ``completamente embriagadas,
promovem desordens e com palavras e gestos que ofendem a moral pública,
privam as famílias de chegar às janelas de suas casas a qualquer hora do
dia ou da noite''.\footnote{\textit{O Fluminense}, 27 de abril de 1887,
  p.~2.} Sofria a moralidade pública do ``solfejo nauseabundo que se
ensaia a toda hora'', isso porque cheira mal a estalagem do Lopes aos
narizes daquele 1889. Assim cheiravam prédios semelhantes no Rio de
Janeiro fazia algum tempo. Fazia 10 anos desde que um relatório da
Comissão Sanitária do Sacramento escrevia sobre certas casas que exalam
``um cheiro especial que as distingue de qualquer uma outra''.\footnote{BR
  RJAGCRJ Códice 8.4.24 Fundo Câmara Municipal --- Série Higiene Pública,
  p.~37.} O ar é muito viciado, muito alterado pelo produto da exalação
pulmonar nas ``casas habitadas por pretos minas''\footnote{\textit{Ibidem},
  p.~37.} cujos vícios se igualam a epidemias em termos de contágio.

É um fenômeno novo. Existiu a velha cidade colonial, a topografia que
cooperava para a insalubridade, a herança dos portugueses que não
encararam suas construções sob o ponto de vista da higiene. A década de
1870 atribuiu prejuízos físicos e morais à aglomeração urbana, e até
então tínhamos rigorosamente uma medicina da profilaxia social. Sim, o
período de então requenta aquelas urgências, mas opera com um
ingrediente que não tinha sido mapeado suficientemente pelo regime
enunciativo próprio do nosso dispositivo.

Isso se desenha às portas do 13 de maio e ganha outra textura no período
pós-abolicionista. É que a atmosfera incendiária do risco de fim do
paternalismo senhoril se deu menos pelos prejuízos nos cafezais --- a
imigração era uma realidade consumada e promissora --- do que pelo
desmanche de antigas hierarquias (isso explicará o incrível silêncio
aberto entre 1888 e 1917, quando o primeiro Código Civil fica pronto).
Como escreve Wlamyra Albuquerque --- na sua pesquisa impecável sobre a
cidadania negra no \textsc{xix} ---, ``fazer transbordar para a sociedade
pós-abolicionista as regras sociais do mundo escravista foi o principal
empenho das elites. Entre as formas de salvar os ex-senhores do desatino
estava a de garantir-lhes a exclusividade da condição de
cidadão''.\footnote{Wlamyra Ribeiro de Albuquerque, \textit{O jogo da
  dissimulação: abolição e cidadania negra no Brasil} (São Paulo, Cia.
  das Letras, 2009), p.~123.}

Como diz o presidente da Comissão Sanitária do Sacramento, o Dr.
Peregrino Freitas, ``a higiene pública, apesar de ser um ramo das
ciências médicas, não está confinada nos limites daquele
domínio''.\footnote{BR RJAGCRJ Códice 8.4.24 Fundo Câmara Municipal ---
  Série Higiene Pública, p.~38.} Nesse caso está a parte a que nos
referimos: a parte em que não se pode medir a higiene pública com a
régua da ciência médica nem com a régua do ódio de classe.

``A freguesia da Gávea está enfestada de vadios'', dizia um jornal em
1889. Assina a denúncia não o inspetor de quarteirão, mas ``a moralidade
pública''.\footnote{\textit{Gazeta da tarde}, 30 de julho de 1889, p.~2.}
Foliões frequentadores de zungus e reuniões ilícitas desmoralizam as
famílias locais, que pediram providências. Trata-se de cortiços que
``pululam de libertos, que preferem a vadiação ao trabalho honesto e
retribuído. E não satisfeitos com isto tratam de transviar os que se
acham empregados e de atraí-los para os cortiços e para as suas
orgias.'' O subdelegado Vergueiro faz a locatária do cortiço assinar
termo de bem viver, porque bailes-orgias não têm licença da polícia, e
``por ter ainda no seu interrogatório declarado que não tinha ocupação
alguma, e que isso não era da conta da autoridade''. A suspeição com a
festa negra reanimava a promessa de que o dia do desabamento da ordem de
mármore e barbárie será indistinguível da paixão trágica dos grandes
carnavais das roças e ranchos. Os casos são incontáveis, como esse outro
cortiço na Rua Itapiru, onde ``moram alguns devotos de Baco que quase
todas as noites, até bastante tarde, gritam e berram''.\footnote{\textit{Gazeta
  de notícias}, 26 de maio de 1888.} Tudo se passava como se a comoção
popular que as pestes e revoltas despertam nascesse do mesmo tutano dos
festejos.

Certa profilaxia racial se organizaria para conter abusos de liberdade.
Não que eles inexistissem antes do 13 de maio. Mesmo porque a conquista
do fruto do próprio trabalho significou para o escravizado de ganho uma
relativa conquista do direito de mobilidade, e isto gerou reações. Dá
para dizer que africanos e descendentes conquistaram espontaneamente
ambos --- falávamos sobre isso em outro contexto. A possibilidade de
viverem por si não transformava relações de trabalho apenas, implicava
redução de eventuais sevícias senhoriais e possibilitou reocupação de
laços familiares um dia interditados. O 13 de maio torna extinta a
escravidão no Brasil --- é uma lei proibitória, não mais do que isso. Até
porque, como nos ensina Wlamyra Albuquerque, o 13 de maio, em linhas
gerais, libertou poucos negros em relação à população de cor. ``A
maioria já havia conquistado a alforria antes de 1888 através das
estratégias possíveis''.\footnote{``Osório Duque Estrada computou os
  seguintes números da população escrava no país: em 1873 (1.541.345);
  1883 (1.211.946) e 1887 (723.419). Já na Bahia entre 1864-87, a
  população escrava caiu de 300 mil para 76.838 pessoas. A mortalidade,
  as alforrias e o comércio interprovincial de escravos justificavam
  essa estatística. Em Salvador, o decréscimo era ainda mais expressivo.
  Em 1887, na capital da província, onde o número de alforrias era mais
  elevado, estavam matriculados exatos 3.172 escravos. Concordando com
  essa estimativa, João José Reis contabiliza que `entre 1872 e o último
  ano da escravidão, a parte escrava da população soteropolitana teria
  declinado de perto de doze por cento, para algo em torno de dois e
  meio por cento'\,''. Wlamyra R. Albuquerque, \textit{O jogo da
  dissimulação}, , 2009, p.~96.}

A escravidão do homem pelo homem é extinta sob a condição de outras
formas impessoais de produzir não liberdades. Foi comum pensar que
qualquer índice de liberdade dada ao escravizado era um índice de
violência subtraído do senhor --- logo, estaríamos a falar de margens de
possibilidades, por assim dizer, pré-legais. Mas com a crise do modelo
escravista, liberdade se torna sinônimo de libertação, liberdade se
torna abolição da condição no interior da qual o Eu se constituiu como
objeto do outro. Um conceito de liberdade radicalmente associado à
experiência moderna porque apenas dotado de sentido concreto na oposição
à realidade da escravatura e da servidão.\footnote{Cf. Achille Mbembe,
  \textit{Sair de grande noite: Ensaio sobre a África descolonizada}
  {[}2010{]}, trad. Narrativa Traçada (Luanda, Edições Mulemba, 2014),
  p.~53.} Ganhávamos assim um conceito negativo de liberdade: liberdade
se tornava um bem que se conquistava, um direito que se adquiria, mas
cujo conteúdo é a negação de algo, a extinção do cativeiro. Mas não
apenas. Liberdade apenas alcança valor quando a escravidão é absorvida e
interditada no plano legal, e aí ela se realizava. Nos tempos do
imperador, a escravidão era a violência na sua forma pura, os séculos de
exploração do negro pelo branco foram ritualizados não pelo Direito, mas
pelos transbordamentos do colonialismo de outrora. Já em 1880, José do
Patrocínio escrevia: ``Hoje ninguém mais pode impedir que haja entre o
senhor e o escravo uma suspeição, que se há de aumentar dia a dia. O
senhor pelo temor da abolição, o escravo pela convicção de que a sua
posição não tem base nem na lei, nem na natureza''.\footnote{José do
  Patrocínio, \textit{Gazeta de Notícias}, 6 de setembro de 1880.} A
indústria infernal do escravismo proporcionou uma sociedade obcecada por
esse gênero de parasitismo. O senhor de escravos, o ``hábito do
absolutismo e da tirania para os escravos reduziu os fazendeiros e os
senhores de engenho à impossibilidade de tratar com homens
livres'',\footnote{André Rebouças, \textit{Conferederação abolicionista}.
  \textit{Abolição imediata e sem indenização}, Panfleto n.~1 (Rio de
  Janeiro, Typ. Central, 1883), p.~20.} como bem dizia Rebouças. Chicote
empunhado, julgavam-se dispensados de raciocinar; o escravocrata não
admite a réplica, não assimila a contradição. Então aguardou-se da Lei
que saneasse a degradação moral dessas relações, a mesma Lei que nem se
deu ao luxo de regular as relações de cativeiro. As surras, infâmias,
espoliação, tortura acompanhada de morte, o ter a vida em risco foi a
mediação para a conquista da liberdade. Mas trata-se, como dizíamos, de
uma liberdade que não viria como motivação ou finalidade que
ultrapassasse a própria conservação da vida, a reprodução da existência.
Daí que o negro brasileiro, uma vez liberto, jamais se reconciliaria com
esse mundo objetivo que ele próprio criou, mas esse mundo objetivo por
ele criado se tornaria estranho, autônomo em relação a ele, e por isso o
aliena, subjuga-o. O trabalho de suas mãos significou a servidão, a
miséria e a não humanização de sua natureza. A autocriação por meio do
trabalho não foi peça decisiva nas experiências de sua
emancipação.\footnote{Paul Gilroy, \textit{O Atlântico negro: modernidade
  e dupla consciência} {[}1993{]}, trad. Cid K. Moreira (São Paulo, Ed.
  34, 2001), p.~100.} Porque o gesto descolonizador, as práticas de
descontinuar antigas subjetividades servis, como veremos no final,
viriam de outras matrizes de autocriação poética. Viriam de lutas
políticas transversais, diríamos, pouco afeitas à governamentalidade
biopolítica moderna. No 13 de maio, poderíamos pensar que trocávamos uma
liberdade ilimitada por uma liberdade juridicamente instituída, mas
melhor seria dizer apenas que variamos entre a conivência com figuras da
animalidade útil e a criminalização de uma prática que nunca deixou de
contar com o mutismo da estrutura jurídica. A escravidão se tornaria um
dia crime, e nada mais, essa foi a oferta civilizatória com a qual a
modernidade recompensou a diáspora africana. ``A escravidão está
extinta'', eles disseram, ``então não façam muito barulho, deixem a
Justiça zelar por vocês, porque vocês estão libertos e a abolição é a
realização da liberdade que todos queríamos. Agora que estamos quites,
não há nada para além da lei sancionada, a lei garante que somos
igualmente livres, então não olhem para trás, o Brasil é o futuro
prestes a se realizar''. Isso não significa que o Direito seja uma farsa
--- ainda que o seja ---, mas o 13 de maio não foi uma demanda da Justiça,
e sim do controle. O que agora teremos pela frente são margens legais
onde a tecnologia policial mediará, sistematicamente, relações então
desfeitas entre senhores e escravizados. E daí a necessidade maior de
aprimorar instrumentos de controle urbano. Isso passa pela eficiência do
dispositivo médico-higienista diante da urgência de individualizar os
indóceis da higiene doméstica. O crescimento da população de cortiço e a
demanda pelo controle urbano eficiente implicarão a modulação da
capacidade de discernir melhor o suspeito de morar em cortiço. A
aparição de uma profundidade identitária, até então ausente na
documentação das instituições que se ocupavam de cortiços, é um
aprimoramento de processos de racialização: os nagôs e os pretos minas e
as suas imprecauções higiênicas; as ninfas do 13 maio e o seu vozerio
libertino; os libertos e portanto vadios, rivais do trabalho.

O ``morador de cortiço'' ganhará a substancialidade racial que compensa
sua mobilidade de habitação. ``Sabemos porém, e sabem-no todos, que
dezenas e dezenas dessas habitações têm sido fechadas pela polícia e que
cada uma delas dava abrigo a centenas de pessoas. O que é feito de toda
essa gente? Onde se oculta?''\footnote{\textit{Gazeta da Tarde}, 20 de
  fevereiro de 1884, p.~1.} Provavelmente nos cortiços que escapam à
fúria das comissões. Aí a população formiga. O pretexto da higiene
condenou um cortiço que tinha 100 habitantes, um seu cortiço vizinho
passa a 300.

\begin{quote}
Expulsa, precipitadamente, do seu domicílio, cada uma dessas famílias
procurou o domicílio de uma família conhecida, e, num cubículo onde até
hoje moravam 6 ou 8 pessoas, moram, agora, por força das circunstâncias
--- 18 ou 20. Eis como o povo resolveu a questão. Expulso daqui
refugiou-se acolá. Estava disperso pela cidade, e agora acumulou-se em
determinados lugares. Os cubículos que escaparam à interdição e que em
geral têm um número de habitantes muito superior à sua lotação,
duplicaram e triplicaram o número de seus tristes hóspedes.\footnote{\textit{Ibidem}.}
\end{quote}

Um candeeiro de querosene, trouxas de saias de chita, caixas de chapéus
com trapos, tudo arremessado ao meio da rua, os guardas, os curiosos.
Retornam aos poucos os inquilinos, a mobília inutilizada na rua. Avançam
contra as portas dos cômodos para catar o que ficou no caminho. Depois
os carroceiros, as carroças abastecidas e a mudança para algum cortiço
de uma rua vizinha ou algum abrigo de parente. Normalmente --- pelo
decreto que desde 1882 regulamentava o serviço de saúde
pública\footnote{Cf. Decreto nº 8.387, 19 de janeiro de 1882.} ---,
estalagens e dormitórios públicos podiam ser multados por superlotação.
Mas só podiam ser interditados pelo período de um a três meses, em casos
de reincidência. Constatado dano à saúde pública, a Junta remetia à
Câmara pedindo que fossem feitos os melhoramentos acusados pelo
engenheiro do Ministério dos Negócios do Interior. Na prática, muitas
casas não eram desalojadas durante as obras de reparo, então permaneciam
em atividade enquanto durassem os empenhos e padrinhos. Autoridades
sanitárias pressionaram por uma legislação menos tolerante. O presidente
da Comissão Sanitária de Sta. Rita pedia em 1880 que se procedesse
sumariamente ao fechamento, para só depois permitir que fossem feitas as
reformas necessárias --- ``até se obter o completo saneamento dessas
numerosíssimas habitações das classes pobres do Rio de Janeiro que
decerto não achariam abrigo em parte alguma''.\footnote{BR RJAGCRJ
  Códice 8.4.24 Fundo Câmara Municipal --- Série Higiene Pública, p.~18.}
O doutor --- cuja opinião merece o mesmo respeito que a sua memória ---
dizia que se não fossem tomadas medidas severas, outros dirão com
justiça ``que somos indiferentes à perda de tantos'' que ``representam
pela maior parte''\footnote{BR RJAGCRJ Códice 8.4.24 Fundo Câmara
  Municipal --- Série Higiene Pública, p.~18.} braços vigorosos e úteis.

Embora o arrasamento das estalagens fosse medida extrema, não era por
isso menos comum. Aquelas casas cujos proprietários privilegiados
sustentavam alguns favores sobreviviam por mais tempo (há casos de
vereadores que vendiam concessões para construção de
cortiços).\footnote{Cf. \textit{Gazeta da Tarde}, 21 de novembro de 1885,
  p.~2.} José do Patrocínio, editor do \textit{Gazeta da Tarde}, chegou a
insinuar que Conde d'Eu era proprietário de um verdadeiro cortiço
erguido a fidalgas alturas. Apesar de tudo encontramos indeferidos, em
sua grande parte, os pedidos de licença ou concessão para construção no
período posterior a 1880, fossem eles requeridos por figurões, militares
do alto escalão ou especuladores médios.

Em 1891, o Diretor da Casa da Moeda escreveu à Junta denunciando o abuso
que representava a existência de uma enorme estalagem na Rua General
Caldwell nº 89, e que ocupava todo o espaço compreendido entre os fundos
do Quartel do Regimento Policial e a Casa da Moeda. Era uma das mais
vastas da cidade. Estendia-se em largura por detrás dos prédios 87, 89,
91, 93, 95, 97, 101 e 103 da referida rua. Continha 114 cômodos ou
casinhas, com portas em torno do terreno e em duas alas unidas pelo
fundo no meio da área da estalagem. Dizia que a estalagem se encontrava
``na pior situação sanitária pela sua má disposição e aglomeração de
moradores, e morrendo aí quase diariamente diversas pessoas de febre
amarela, chegando, segundo sou informado, essa mortandade até
17''.\footnote{BR RJAGCRJ Códice 43.1.27 Fundo Câmara Municipal --- Série
  Cortiços e Estalagens, p.~6.} Ou seja, não era treinado só na
especialidade de imprimir cédulas o Diretor. Era também fluente na
cartilha de uso comum de uma experiência da epidemia gerada no colo do
higienismo. Denunciar a má disposição sanitária e a aglomeração dos
moradores é encomendar tanto um ``arrasamento de todos esses antros de
imundícias'' quanto a percepção de que ali há ``um escândalo moral,
sanitário e econômico''.\footnote{\textit{Ibidem}, p.~6.} O vocabulário da
higiene doméstica popularizou-se, nas notas dos jornais, nas soluções
práticas do cotidiano, nos laudos de polícia. Atribuir a falta de asseio
ou a má condição higiênica a uma rua, a uma casa, a um indivíduo era
lançar sobre ela o escárnio e a suspeita.

Pois bem, o Inspetor de Higiene recebe com entusiasmo a denúncia e
despacha para o Ministério exigindo a demolição, por desapropriação,
desse cortiço que vem proporcionando ``elementos para o aparecimento de
casos repetidos de febre amarela''.\footnote{\textit{Ibidem}, p.~8.} O
engenheiro encarregado prepara uma visita à estalagem e classifica,
minuciosamente, tudo o quanto é capaz de ver sua sensibilidade, entre o
que ``as mais censuráveis infrações das leis higiênicas de
construção''.\footnote{\textit{Ibidem}, p.~11.} O relatório retorna à mesa
do Inspetor, que por fim despacha com uma segunda nota pedindo a remoção
do ``perigoso foco de febres de mau caráter que, em épocas anormais,
assumem o caráter epidêmico''.\footnote{\textit{Ibidem}, p.~15.} A
resposta vem de cima em tom seco e frio. ``Entendo que o Ministério do
Interior deve limitar-se a promover a execução das disposições do
regulamento sanitário.''

Quer dizer, a providência solicitada pela Inspetoria Geral de Higiene ---
a demolição, por desapropriação --- não podia ter lugar, porque a isso se
opõe o disposto no art. 83 do regulamento sanitário. A desapropriação só
poderia ter lugar por utilidade pública ou municipal, e não era o caso.
Portanto a estalagem permaneceria fechada, mas por pouco tempo.
Encontramos no obituário dos jornais italianos residentes naquela
estalagem, e que faleceram entre 1891 e 1892, por razões as mais
diversas, nenhuma delas a febre amarela. Poucos anos depois, 1895, o
proprietário do edifício despacharia o seguinte requerimento.

\begin{quote}
\textit{Cidadão Prefeito da Capital Federal}
O Coronel João Leopoldo Modesto Leal, proprietário da estalagem da rua
General Caldwell nº 89, pede licença para reconstruir a referida
estalagem, obedecendo ao plano que apresenta por cópia, confeccionado
segundo as prescrições da lei de 15 de setembro de 1892.
\end{quote}

Seguia em anexo uma planta com uma Vila Operária de 40 casas geminadas
de porta e janela, organizadas ao longo de dois corredores. Cada unidade
continha dois cômodos frontais de 2,4m × 4,4m e uma cozinha nos fundos
de 2m × 4m. São proporções estranhas, se compararmos com os projetos
para vilas operárias das primeiras décadas do século \textsc{xx}. Mas se
lembrarmos que havia bem pouco tempo as casas não eram equipadas com
água encanada e esgoto, e que para as lavadeiras, carregadores de água
etc. a distinção entre trabalho doméstico e trabalho de rua nunca esteve
bem delineada, saberemos que a distinção entre casa e rua não são coisas
sobre as quais podemos fazer algum juízo acertado. A importação da ideia
de lar como ambiente de intimidade e conforto não é coisa com a qual se
tinha qualquer compromisso imediato.

O pedido do Coronel João é indeferido, tanto pela Secretaria de Obras
quanto pelos higienistas, por se tratar de um autêntico \textit{cortiço}.
Parece-nos que a estalagem havia sido interditada e fechada em
definitivo, o que possivelmente motivara a refundação completa do
edifício. Entre o fechamento da estalagem e o ano em que foi indeferida
a obra de refundação das casinhas, entra em vigor um protocolo legal
muito menos truncado. O pedido de maior severidade dos médicos já tinha
sido escutado.

A lei nº 85 de 20 de setembro de 1892 transferirá para a municipalidade
os serviços de higiene e a polícia sanitária, regulando a demolição de
edifícios que representam perigo e embaraço para a população. Três meses
depois que a lei é sancionada, assumirá Barata Ribeiro, o próprio. Já
antes de ocupar a cadeira de prefeito, na condição de membro do Conselho
da Intendência Municipal, despachara uma circular encomendando dos
fiscais das freguesias urbanas um mapeamento das estalagens de cada
jurisdição que precisavam ser fechadas ou demolidas. Era questão de
tempo. O temperamento enérgico do prefeito atualizou a lei na realidade.
Pereira Passos não criou nada que não pertencesse já à ordem do
admissível.

Cortiços diziam respeito também a imigrantes portugueses que trabalhavam
em pedreiras, campesinos europeus improvisados na manufatura ou na
indústria das ocupações informais que a capital oferecia. Alojar o
imigrante proletário tornava-se dever cívico e função da caridade
pública. Lembremos que \textit{O Cortiço} de Aluísio Azevedo tratava do
aculturamento de um aldeão português, sua corrupção pela festa, a
reforma dos apetites, o abrasileiramento pela influência do meio. O
vigor europeu foi considerado fator social da maior importância; daí a
comoção para alojar o proletário de modo mais humanitário, em habitações
mais decentes e higiênicas, ao invés de largá-lo nas aventuras dos
quiosques e encruzilhadas. Da imigração dependia a correção da nossa
deformidade civilizatória, e o Rio de Janeiro como rota da imigração
dependia de um porto que rivalizasse com os portos salubres do mundo.
``Mãos à obra, o governo deu toda a autonomia à Junta, a polícia está
disposta a auxiliá-la, a salubridade da cidade está pois confiada agora
e de fato às comissões sanitárias e à Junta no que diz respeito aos
cortiços''.\footnote{\textit{Gazeta da Tarde}, 21 de dezembro de 1883.}
Não é que o cortiço não se deixasse domar pela verdade da ciência médica
--- o que de fato acontecia ---, mas chegávamos a um estado de coisas em
que, da perspectiva dos seus moradores, estavam desfeitos os limiares
entre a municipalidade, a higiene pública e a polícia. A habitação
coletiva, seus moradores e o proprietário conseguiram erigir ``um Estado
em outro Estado''.\footnote{BR RJAGCRJ Códice 43.1.25 Fundo Câmara
  Municipal --- Série Cortiços e Estalagens, p.~79.} Em face às leis
higiênicas e de salubridade pública os cortiços exerciam a mais completa
rebeldia; sem que os arrendatários pagassem licenças nem impostos
Municipais ou do Tesouro em relação ao número dos cubículos, exerciam um
radical liberalismo; porque não estavam sujeitos às leis policiais,
negavam entrada às autoridades, quando não garantiam esconderijo para
seus suspeitos em seus labirintos.

Um engenheiro das Obras Municipais incomodava-se com um cortiço que dava
fundos ``para o mar, sendo destarte acessível a visitantes noturnos que
vão ali pernoitar para escaparem à ação da polícia''.\footnote{BR
  RJAGCRJ Códice 43.1.26 Fundo Câmara Municipal --- Série Cortiços e
  Estalagens, p.~18.} Um delegado de polícia que denunciava uma
estalagem abandonada ``contendo muitos quartos em ruínas, que podem
servir de abrigo, durante a noite, a vagabundos e gatunos''.\footnote{\textit{Ibidem},
  p.~97.} O argumento de que os cortiços na cidade do Rio de Janeiro são
``focos de infecção terríveis e concorrem para o desenvolvimento da
febre amarela''\footnote{\textit{Gazeta da Tarde}, 21 de dezembro de 1883.}
já se fazia indissociável, em plena década de 1880, do argumento segundo
o qual ``são os habitantes dos cortiços que mais contribuem para as
estatísticas criminais''.\footnote{\textit{O Fluminense}, 7 de novembro de
  1880, p.~2}

Rasuremos daqui em diante o princípio do Dr.~Peregrino, presidente da
Comissão Sanitária do Sacramento. Não é tanto a Higiene que, apesar de
ser um ramo das ciências médicas, não está confinada nos limites deste
domínio. Digamos que é o poder de polícia que, apesar de ser a
especialidade das praças e da cavalaria, não está confinado nos limites
da delegacia. Concluímos que, quando em nossa sociedade a experiência da
epidemia passou enfim a ser colonizada pela prática higienista, o
``modelo técnico-médico da cura'' conduziu o ``esquema
político-moral''\footnote{Michel Foucault, \textit{Vigiar e Punir}
  {[}1975{]}, trad. Raquel Ramalhete (Petrópolis, Vozes, 2009), p.~234.}
de quadriculamento para fora de si mesmo. E foi neste movimento de
``exteriorização'' que a polícia tomou a palavra para dizer aquilo que
Johann Moritz Rugendas, em 1825, escrevia sobre a vida dos escravizados
no Rio de Janeiro: vocês gozam em geral de muita liberdade, pois têm o
dia inteiro disponível para tratar de seus negócios, bastando-se
recolherem-se à noite; seus senhores só se preocupam com vocês na medida
em que se faz necessária a cobrança hebdomadária. Quanto mais o senhor
está distante de vocês em lugar e categoria, mais liberdade usufruem,
menos são inspecionadas e controladas suas ações, mais pálida fica a
comparação entre o africano e o ser humano. Vocês estão tendo muitas
alegrias e eu preciso explorar as paixões tristes de vocês, eu preciso
delas para exercer meu poder.\footnote{``Uma boa parte da população
  escrava do Rio de Janeiro está no serviço doméstico dos ricos e da
  alta sociedade: são artigos de luxo, que têm muito mais a ver com a
  vaidade dos senhores que com as necessidades reais da casa. A maioria
  desses escravos veste uns \textit{librés} de gênero muito antigo, e
  esses \textit{librés}, combinados às cestas que carregam nas cabeças,
  tornam-nos verdadeiras caricaturas. Eles têm pouco ou nenhum trabalho;
  a alimentação deles é muito boa; e, no geral, são tão inúteis quanto
  os servos dos grandes senhores da Europa --- por isso imitam-lhes os
  vícios com grande facilidade. Boa parte dos escravos das grandes
  cidades está sujeita a pagar aos senhores todas as semanas, ou mesmo
  diariamente, uma determinada soma, que procuram obter exercendo uma
  profissão: carpinteiros, sapateiros, alfaiates, marinheiros, porteiros
  etc. Conseguem, desta maneira, ganhar facilmente uma quantia maior do
  que o senhor exige; e conquanto, em seus negócios, economizem,
  conseguem resgatar sua liberdade em um espaço de nove a dez anos. Se
  isso, porém, não chega a acontecer com a frequência com que
  esperaríamos, é porque os negros possuem disposições de se deixarem
  arrastar às despesas as mais extravagantes --- sobretudo em termos de
  roupas, tecidos e fitas de cores berrantes. Dispensam nesse tipo de
  coisa quase tudo o que ganham. Eles gozam em geral de muita liberdade,
  e sua existência é bastante suportável, porque eles têm o dia inteiro
  disponível para tratar de seus negócios, bastando que se recolham à
  noite. Os senhores com eles não se preocupam, senão o necessário para
  que lhes assegurem o pagamento semanal. De manhã, antes de partirem, e
  à noite, quando voltam, ganham farinha de mandioca e feijão; o almoço
  eles têm que providenciar por conta própria. Há mulheres escravas que
  ganham seu sustento da mesma forma, tornam-se enfermeiras, lavadeiras,
  vendedoras de flores e frutas.'' Em RUGENDAS, Maurice. \textit{Voyage
  pittoresque dans le Brésil --- 4º Division: Moeurs et Usages des
  Négres}. Paris: Engelmann, 1827, p.~17-18. Tradução nossa.}

Acerta em cheio aquele doutor da Academia Imperial de Medicina: o Rio de
Janeiro é semelhante a um vasto cortiço. O cortiço é uma das
instituições que, por espontaneidade nossa, felizmente nos rege.
Assentou os seus arraiais por toda a cidade, no meio do mais distraído
indiferentismo dos poderes, nos centros mais populosos, ao lado dos
palácios, por toda a parte, enfim, onde havia um lote de terra
disponível. Então,

\begin{quote}
umas célebres comissões médicas, às quais o povo tem chamado
\textit{comissões cínico-sanitárias} entendeu que se a pobreza ali corria
sérios riscos de vida e que no meio da rua, ao sol e à chuva, dormindo
pelas praças ou acampando à bela estrela, poderia estar em melhores
condições. Alguns moços louros tomaram a si o papel de anjos
exterminadores, e chamando o gladio policial em seu auxílio, já que a
tradicional espada de fogo da cena bíblica se tinha apagado e extinto,
há muito tempo, expulsou desses paraísos terrestres os seus naturais
hóspedes.\footnote{\textit{Gazeta da Tarde}, 16 de fevereiro de 1884,
  p.~1.}
\end{quote}

Como nos tempos de sedição, em que os amotinadores ganham as ruas
gritando ``Fecha! Fecha!'', as comissões sanitárias glosaram esse mesmo
mote e centenas de habitações foram interditadas, e seus hóspedes
despejados, empurrados para territórios em fuga, alargaram as
fronteiras. ``Quando o querem fazer desaparecer, ele resiste, como se
tivesse força igual a que determina a consolidação incessante da crosta
da terra''.\footnote{\textit{Gazeta da tarde}, 31 de janeiro de 1885,
  p.~1.} E de feito, essas ruínas merecem estar destinadas a dar-nos
finalmente terra firme para substituir o pântano policial, que até agora
ameaça entre nós o encantamento da vida na cidade e a hospitalidade que
nos ensinaram nossos mais velhos.

Afirma Lima Barreto que ``o quilombola e o corsário projetaram um pouco
a cidade''.\footnote{Lima Barreto, \textit{Vida e Morte de M. J. Gonzaga
  de Sá} {[}1919{]}, 1956, p\textit{.} 66.} O corsário projetou as
fortalezas, o quilombola projetou a ritualização dos espaços. E pode-se
pensar uma vida quilombola de cortiço, como pequenos abolicionismos da
vida cotidiana. Atuando nas frestas e vielas, cobertos com
\textit{mariwo}, nos jogos de avanço e recuo com relação a acordos
superestruturais. Lá estiveram os corticeiros daquela cidade quase
desacontecida, como posseiros urbanos, porque o que fizeram foi ocupar
terras devolutas e sem função social, dando uma natureza pré-legal à
luta política. Cortiços foram quilombos do território urbano, atuando na
dinâmica de estratégias contra poderes não estatizáveis. Explicamos:
sabe-se como essa Corte foi pouco amiga de espaços neutros, quadrados,
monocromáticos. A linha reta, sisuda e gelada, perdia para as
imprevisíveis dilatações e curvas dos nossos morros, ruas, rios e
quarteirões. Entre todos os lugares de passagem como bondes e estações,
entre lugares abertos ou de parada transitória, como cafés, praias,
portos, havia espaços outros, absolutamente outros: contraespaços,
espaços hackeados ou heterotopias. São as pequenas utopias realizadas,
só que ligadas a recortes singulares de tempo, possuidoras de ``um
sistema de abertura e de fechamento que as isola em relação ao espaço
circundante''.\footnote{Michel Foucault, \textit{O corpo utópico --- As
  heterotopias} {[}1966{]}, trad. Salma T. Muchail (São Paulo, n-1
  edições, 2013), p.~26.} Trata-se de um isolamento, mas que contém um
dentro-e-fora não assinalável. Melhor dizendo, são assentamentos
provisórios, nomadismos, territórios plásticos, porque não operam
segundo as coordenadas da fronteira ou da contradição, mas na lógica da
sobreposição, da dobra, do acavalamento. Por isso, qualificados pela
polícia como labirínticos, falanstérios da perdição, núcleos da
desobediência à jurisdição dos territórios. Suas portas e janelas,
embora escancaradas, são dadas apenas ao olhar iniciado, ao não
estrangeiro. Quilombos foram heterotopias, canais, redes de comércio de
informação entre forros e fugidos. E os cortiços, nossos quilombos
urbanos, reeditam sua funcionalidade hospitaleira, agora para os negros
livres em processo de descolonização ética. Então poderíamos testar de
fato a noção de quilombo para caracterizar a dinâmica da vida de cortiço
e identificar aí a constituição de um abolicionismo microfísico, um
abolicionismo quilombola capaz de fazer explodir aquilo que o 13 de maio
jamais pôde deixar de negligenciar: descontinuar antigas subjetividades
servis, descongestionar novas subjetivações, mobilizar a
transversalidade nas lutas. Um ``Estado em outro Estado'', desde que se
compreenda seu caráter heterotópico, desde que se compreenda que nesse
gesto se excede o próprio Estado para tornar instáveis os autoritarismos
e ameaçar a legitimidade do poder policial. Até porque operar na lógica
das heterotopias significa não estar articulado conforme a contradição
entre territórios excludentes ou opostos uns aos outros. Isso não
significa fazermos a celebração de um mundo boêmio e sem raízes ---
justamente o contrário. Significa na verdade essa ideia de Beatriz
Nascimento, que consiste em interpretar quilombos como ``sistemas
sociais alternativos, ou no dizer de Ciro Flamarion: brechas no sistema
escravista''.\footnote{Beatriz Nascimento, ``O conceito de quilombo e a
  resistência negra'' {[}1985{]}, em Alex Ratz, \textit{Eu sou atlântica:
  Sobre a trajetória de vida de Beatriz Nascimento} (São Paulo,
  Instituto Kuanza / Imprensa Oficial, 2006), p.~121.} Ou seja --- e isso
é o mais importante aqui ---, quilombo foi o destino da fuga apenas
porque as brechas, as frinchas, são o destino do quilombo. A ``fuga
implica numa reação ao colonialismo''\footnote{\textit{Ibidem}, p.~122.} e
o quilombo, tendo sido o destino precursor da fuga, também é a
instituição que procede como frinchas no sistema, traduzindo
territorialmente as instabilidades da manutenção do colonialismo. Como
isto se dá? Se a arquitetura colonial construiu fortalezas, o modernismo
funcionalizou os espaços e o capitalismo conformou o solo urbano à
acumulação de renda, o quilombo inventou a ritualização dos espaços.

Existe um ditado iorubá que diz: ``\textit{Ibiti enià kò si, kò si
imalè}'': ``Ali onde não há ser humano, não há deuses''. Quer dizer,
existem religiões (e até instituições políticas) que querem dar conta de
tudo. O candomblé, que se sabe não onipotente, não responde tudo. E sabe
que não responde tudo, e esta é uma virtude sua. Esse pode ser um
primeiro sentido para o ditado, a saber: \textit{aí onde não há orixá, não
habita o mistério, nem perspectivas, nem encruzilhadas.} A modernidade
no Ocidente, que por tanto tempo se orgulhou de pensar a finitude do
homem, não pensou a finitude do próprio Ocidente. Talvez seja por isso
que quando não for possível pensar candomblés e umbandas como religiões
históricas talvez não seja relevante pensá-los dentro dos moldes de
religiões. Existem várias umbandas, e o candomblé é resultado de
elaboração de diversas culturas africanas, produto de várias afiliações.
Existem também vários candomblés (Angola, Congo, Efan etc.), quer dizer,
as divindades africanas são territoriais, estão ligadas à terra, por
isso a mobilidade geográfica da religião gerada pela diáspora quase não
teria feito sentido se os orixás não acompanhassem seus filhos à
América. Aí está outro sentido para o ditado ``ali onde não há ser
humano, não há deuses'': \textit{o orixá, a divindade, existe na
territorialidade, não na transcendência.} Assim como o \textit{egbé} é
local que contrai por metáfora espacial o solo mítico da origem e faz
equivaler-se a uma parte do território histórico da diáspora, também o
quilombo. Aí o indivíduo em vida não está em face de sua própria
finitude. O indivíduo, centro de inscrição de \textit{òrìsás}, está em
face da finitude da presença espectral dos ancestrais escravizados. Ele
está radicalmente em face daquilo que se encontra --- vindo de nossos
ancestrais --- quando chegamos ao mundo, em face dessa presença espectral
ritualisticamente territorializada. Mais do que isso. Como me explicou
recentemente Katiúscia Ribeiro, cultuamos um ancestral porque antes de
ele ser encantado ele existiu aqui. Mas nossa ancestralidade --- ela me
disse --- não parte do acorrentamento, da violência física, ela parte de
um momento em que não existiam correntes e não existia quem
acorrentasse. Aí residiu certa abertura para uma necessária
\textit{segunda abolição}, que não consistiu simplesmente em abolir o
outro ou abolir as formas de servidão perpetradas pelo outro. Também não
foi a transição de uma consciência danificada para uma consciência
radicada na autodeterminação dos apetites, ou abandonada na apatridade
liberal de um existir em sentido próprio. O que foi, e ainda é, essa
segunda abolição? Seguindo Achille Mbembe, diríamos que a segunda
abolição consistiria em ``se autoabolir libertando-se da parte servil
constitutiva de si''.\footnote{Achille Mbembe, \textit{Sair de grande
  noite} {[}2010{]}, 2014, p.~54.} Mas podemos também rasurar essa
fórmula, atrelando-a à ritualização dos espaços como prática operadora
das subjetivações.

Foi necessária uma segunda abolição, dissemos, muito mais complexa e
muito menos instantânea que a primeira, porque foi expansão do movimento
da descolonização, ao mesmo tempo em que foi recusa ao conceito de
descolonização entendido como transferência do poder da metrópole para
as elites crioulas. Segundo Nabuco, em 1883,

\begin{quote}
a emancipação dos escravos e dos ingênuos, posso repeti-lo porque esta é
a ideia fundamental deste livro, é o começo apenas da nossa obra. Quando
não houver mais escravos, a escravidão poderá ser combatida por todos os
que hoje nos achamos separados em dois campos, só porque há um interesse
material de permeio. Somente depois de libertados os escravos e os
\textit{senhores} do jugo que os inutiliza, igualmente, para a vida livre,
poderemos empreender esse programa sério de reformas (\dots{}).\footnote{Joaquim
  Nabuco, \textit{O abolicionismo} {[}1883{]} (Rio de Janeiro, Nova
  Fronteira, 2000), p.~170.}
\end{quote}

Seria no Parlamento e não em quilombos, zungus e cortiços urbanos onde
se haveria de ganhar a causa da liberdade. Nabuco se apressa então em
traduzir a comoção popular, a simpatia inerte e envergonhada pelas
vítimas da escravidão, em ``liberdade, não afiançada por palavras, mas
lavrada em lei''.\footnote{\textit{Ibidem}, p.~28.} Em 1883, a causa dos
escravizados parecia moralmente ganha --- assim ele queria. Para ele, o
abolicionismo seria oportunidade de transformar a consciência moral da
dívida, o senso de justiça que devolve ao outro a humanidade subtraída
em previdência política fática. Como se pela consolidação da lei a
autoridade não só liberasse o escravizado do cativeiro, mas publicasse
que ``todos nós, brasileiros, somos responsáveis pela escravidão, e não
há como lavarmos as mãos do sangue dos escravos''.\footnote{\textit{Ibidem},
  p.~168.} O que ele esperava do 13 de maio era que, através da lei e
graças a ela, o Brasil se reconhecesse como país racista e escravocrata.
Aquilo que era transmitido ao mundo inteiro, ``a reputação que temos em
toda a América do Sul, de \textit{país de escravos}'',\footnote{\textit{Ibidem},
  p.~163.} essa identidade nacional, que ``ao mundo civilizado'' era
óbvia, entre nós corria o risco do esquecimento.\footnote{``Entretanto,
  não é menos certo que de alguma forma se pode dizer: `A vossa causa,
  isto é a dos escravos, que fizestes vossa, está moralmente ganha.'
  Sim, está, mas perante a opinião pública, dispersa, apática,
  intangível, e não perante o parlamento e o governo, órgãos concretos
  de opinião; perante a religião, não perante a Igreja (\dots{}); perante a
  ciência, não perante os corpos científicos, os professores, os homens
  que representam a ciência; perante a justiça e o direito, não perante
  a lei que é a sua expressão, nem perante os magistrados,
  administradores da lei; (\ldots{}) perante os partidos, não perante os
  ministros, os deputados, os senadores, os presidentes de província
  (\dots{}).'' \textit{Ibidem}, p.~28.} Ora, houve política de reparação da
barbárie ética e social, que é a obra impecável do escravismo? Houve
autocrítica? Sequer o país escravocrata se olhou no espelho? Para
Nabuco, a História não dá saltos, era preciso escrever de forma a
apaziguar, esclarecer sobre o fato de que adiar a emancipação seria
instigar ``sintomas crescentes de dissolução social''.\footnote{\textit{Ibidem,}
  p.~167.} Era urgente convencer que a escravidão ``em vez de ser uma
causa de progresso e expansão, impede o crescimento natural do
país''.\footnote{\textit{Ibidem}, p.~164.} Desde que votada a lei de 28 de
setembro, o governo brasileiro tratava de fazer acreditar o mundo que
escravidão não mais havia. Antes disso, a primeira iniciativa pela
consolidação de uma lei pela causa dos escravizados foi promovida não a
favor da abolição, mas contra o tráfico. Sobre a lei de 1850, ``dizia-se
que a escravatura, uma vez extinto o viveiro inesgotável da África, iria
sendo progressivamente diminuída pela morte''.\footnote{\textit{Ibidem},
  p.~1-2.} E não bastou tentar fazer a escravidão desaparecer confiando
a tarefa à mortalidade progressiva de africanos e à baixa expectativa de
vida\dots{} Para todos os efeitos, podemos destacar Nabuco, artificialmente,
como ápice de uma vertente do movimento abolicionista que perseguiu a
eliminação de dispositivos institucionais coniventes com a escravidão.
Qual é então a natureza da controvérsia? Hoje, confortavelmente
distantes dessa série de acontecimentos históricos, pode parecer
controverso considerar um sistema econômico latifundiário-escravagista
como os joelhos do ordenamento jurídico-político do país e,
simultaneamente, recorrer a ele para que reconsidere recursos. Se aquilo
que naturalizou a relação senhor-escravizado foi justamente a
complacência moral, associada à complacência oficializada pela
marginalidade da escravidão em relação à lei, como exigir do sistema
legal garantias? Até porque, oficialmente, para os escravocratas que
tiveram a ousadia de exigir indenizações talvez sequer escravidão teria
havido!

Absurdo duvidar do efeito avassalador das campanhas abolicionistas no
parlamento e na imprensa. A propaganda abolicionista, em sua fase mais
bonita e heroica, conjugou não só políticos e militares republicanos,
mas robustas lideranças negras (e nem por isso unificadas em termos de
projeto), como Luiz Gama, José do Patrocínio e André Rebouças. Mas o que
mais nos atrai, voltando ao ponto, são as lutas transversais em oposição
às lutas centralizadas conduzidas pelo abolicionismo parlamentar. Como
operavam as brechas quilombolas garantidas pela resistência ao
desaparecimento, pela renovação das práticas de sobrevivência? Afinal,
os tipos de movimento por moradia popular, no centro da cidade, e a
velocidade da compreensão da sua urgência aparentemente não se
expressaram dentro das mesmas coordenadas dos balanços de conjuntura de
sistemas como partidos ou instituições sociais.

\begin{quote}
O pensamento político há muito não tem mais como escapar de um grave
dilema contemporâneo: como determinada ação de defesa da vida processada
fora dos parâmetros do diálogo entre singulares ou do conflito regulado
por leis, sem programa político ou organização prévia, insistentemente
continua a ser considerada qualquer coisa menos ação política?\footnote{Edson
  Teles, ``Direitos Humanos, ação política e as subjetivações
  oceânicas'', \textit{Philósophos}, v. 23, n.~1, Goiânia, jan./jun. 2018,
  p. 235-264.}
\end{quote}

Pode-se pensar as vidas quilombolas de cortiço como pequenos
abolicionismos da vida cotidiana, e enxergar em nossos posseiros urbanos
a proliferação de lutas que não foram interpretáveis dentro das
coordenadas políticas fornecidas pelo conflito entre republicanos e
monarquistas. Lutas marginais, sem porta-vozes gregários, sem grandes
gestos heroicos, lutas descongestionadas dos sistemas habituais de
referência do poder imperial. E por que dizemos ``posseiros''? Porque o
direito à circulação e à moradia no centro da Cidade Velha não foi
alguma coisa que se procurou resgatar pela via da Lei e do Direito. O
gesto descolonizador dos quilombolas urbanos foi radicalmente a ocupação
ritualizada do território, porque foi nessa ritualização da rua que se
deu nossa aventura civilizatória. No Brasil, há um ``enfeudamento'' da
terra que é equivalente à racialização da servidão de negros --- como nos
ensinou André Rebouças ---, isso porque os ``exploradores da raça
africana são simultaneamente grandes monopolizadores de
terra''.\footnote{André Rebouças, , 1883, p.~18.} A abolição jurídica
era metade da tarefa, porque o modelo de ``exploração do homem pelo
homem''\footnote{\textit{Ibid,} p.~6.} é indissociável do ``monopólio
territorial, o enfeudamento da terra, o landlordismo''.\footnote{\textit{Ibid,}
  p.~40.} O Brasil oficial se enclausurou excessivamente no sistema das
minas e da fazenda, tivéssemos perpetuado seus valores e ambições
estaríamos sufocados em um deserto de paixões tristes e corpos sem viço.
Se nos fosse possível enxergar dentro da terra, veríamos que nestas
cidades que também crescem para baixo, todo bem-estar e progresso foram
invenções do suor e dos cadáveres negros, árabes, ameríndios etc. As
riquezas do Brasil oficial são nossas também. Só que extraoficialmente,
no intervalo dos ciclos econômicos, recriou-se, pela ritualização dos
espaços, alguma coisa sem vocação para o capital e para o colonialismo:
a rua.

\section*{Conclusão}\label{conclusuxe3o}

A subcena das ruas não circulou com hora marcada nem permissão dos
guardas. Nossos bairros em construção foram erguidos como mutirões de
domingo --- os baldes subiram de mão em mão, como diz o poeta Marcos
Nascimento. As obras não estão paradas. A rua e o edifício do Rio de
Janeiro do século \textsc{xix} propuseram uma inquietação filosófica, que foi
sendo destravada da constelação de arquivos históricos e de algum estudo
em filosofia contemporânea. Este texto é sobre o conceito de liberdade,
o problema da liberdade. Ou menos a liberdade e mais seu abuso, o
excesso de liberdade, o problema da \textit{liberdade excedida}. Enquanto
a ideia de liberdade mobilizou estruturas jurídicas e a racionalidade
``liberal'' daquela sociedade por quase 40 anos --- pensamos entre a lei
Euzébio de Queiroz e o 13 de maio ---, o problema da liberdade em
excesso, da liberdade excedida, prossegue por uma linha de aprimoramento
sem termo. É uma veia aberta. Não que uma tenha substituído outra, mas
certamente ambas são como mitos de origem do Brasil oficial.
Paradoxalmente poder-se-ia dizer que o problema da liberdade excedida
ganha relevo na crise da noção de liberdade. Porque o quadriculamento da
liberdade excedida começa a cobrir, com excelência, aquilo que a
privação da liberdade jurídica e a carência de cidadania não conseguiram
evitar: a vida no que tem de ingovernabilidade inventiva, a cidade no
que tem de encruzilhada, os corpos nos seus odores, a festa como fator
civilizador, o amor em seus devires, o tempo não produtivo. Produzir uma
liberdade assistida, privatista, fornecer uma liberdade para o sujeito
fronteiriço, era o papel da parafernália jurídica na qual se baseou a
República. Higienizar o corpo, tornar salubres tipos de rua, aburguesar
o negro, desarmar o imigrante suspeito de imoralidade, essas foram as
funções do poder de polícia. A polícia cumpria os sonhos de
monumentalização da cidade enquanto parecia prescindir do conteúdo
jurídico que versava sobre a cidadania de sujeitos de direito. Ela,
melhor do que ninguém, realizava o sonho da burguesia de sobrado, e esse
sonho consistia em fechar o tampão da cidade colonial. Acontece que a
polícia realizou-se não na medida em que fazia valer o direito --- nem na
medida em que o suspendia integralmente ---, mas ``no ponto em que o
direito se inverte e passa para fora de si mesmo, e em que o
contradireito se torna o conteúdo efetivo e institucionalizado das
formas jurídicas''.\footnote{Michel Foucault, \textit{Vigiar e Punir}
  {[}1975{]}, 2009, p.~211.}

Já que a noção de liberdade excedida esteve em íntima conexão com a
tecnologia policial, por que não estudamos a polícia como instituição?
Por que, ao invés disso, estudar como funciona o dispositivo
médico-higienista? Primeiro porque não é a polícia que nos interessou,
mas as práticas de poder correspondentes que, com sua colaboração,
ganham a complexidade de uma tecnologia. Inscrever o fenômeno da polícia
no drama histórico da sua institucionalização seria perder de foco algo
fundamental, algo que não poderia ser explicado à luz da legislação, da
justiça, dos regulamentos e burocracias. Mas onde identificamos as
oportunidades em que uma Junta Central de Higiene Pública excede a si
mesma, lá onde a fronteira entre regimes de verdade e relações de poder
se insinua, existe aí uma margem em que poder e saber são reciprocamente
pressupostos. E então o poder, que não podia ser conhecido senão
indiretamente, será redescoberto em relações de saber. Nos regimes de
verdade, nos jogos entre horizontes de enunciados e visibilidades, é-nos
dado um ``saber do poder''. Não que o poder só exista nas relações de
imanência com a verdade --- mesmo porque há diferença de natureza que os
tornam irredutíveis um ao outro ---, mas foi esse um caminho que tomamos.
Significa então que o poder será uma ``figura de tecnologia política que
se pode e se deve destacar de qualquer uso específico''.\footnote{Michel
  Foucault, \textit{Vigiar e Punir} {[}1975{]}, 2009, p.~194.} Digamos que
Foucault seja nominalista quando estuda o poder, pois no poder não há
fundamento ou matéria própria para o conhecimento: o poder é a física de
impor uma ação qualquer sobre uma multiplicidade humana
qualquer.\footnote{Cf. Gilles Deleuze, \textit{El poder: curso sobre
  Foucault II} {[}1986{]} (Buenos Aires, Cactus, 2014), p.~74-77.} É o
afetar e o ser afetado, a maneira pela qual afetos são distribuídos, o
acontecimento naquilo que ele tem de relacional. Concretamente, do poder
se espera que produza efeitos tais quais \textit{induzir},
\textit{suscitar}, \textit{extrair}, \textit{pastorear}, \textit{impor uma
tarefa}, \textit{otimizá-la} etc., isso é uma coisa. Deve-se então
prescindir da ambição de tentar caracterizá-lo segundo um ``uso'' ou uma
``função'' específica (\textit{curar}, \textit{corrigir}, \textit{educar},
\textit{moralizar}), porque usos e funções são temas que nascem do saber.
Assim como os ``objetos'' (o \textit{doente}, o \textit{negro}, o
\textit{suspeito}, o \textit{criminoso}) nos quais a verdade escava uma
interioridade, também pertencem ao saber. Se o saber é um composto de
visibilidades e enunciabilidades, e o poder --- em sentido abstrato --- é
cego e mudo, o poder de polícia será melhor reconhecido nas relações de
captura com as práticas higienistas --- com a Medicina Política ---, nas
funções concretas aí desempenhadas, na emergência das objetividades.

Uma segunda coisa: o excesso de liberdade é um problema de polícia, não
do Direito, mas ele não se tornaria urgência social, não se tornaria
fenômeno de cultura se não levasse em conta ou não fosse incorporado
pelas implicações científicas forjadas por certa experiência da
epidemia.

Quando em nossa sociedade a experiência da epidemia passou enfim a ser
colonizada pela prática higienista, o ``modelo técnico-médico da cura''
conduziu o ``esquema político-moral''\footnote{Michel Foucault,
  \textit{Vigiar e Punir} {[}1975{]}, p.~2009, p.~234.} de quadriculamento
para fora de si mesmo. E foi nesse movimento de ``exteriorização'' que a
polícia cumpriu seu destino como fenômeno cultural. Porque, pensando
bem, o que propôs a análise foucaultiana de natureza e dimensões
microfísicas senão a advertência de o quão pouco prática é a perspectiva
historiográfica que avalia as coisas nas vésperas dos grandes
cataclismos ou sob o clarão dos dias de festa? Há de se reconhecer
descontinuidades históricas nas coisas de todos os dias, nas pequenas
invencionices práticas: a pressa para se enterrar um morto, o
desaparecimento dos curandeiros dos classificados, as claraboias nas
alcovas, o alargamento de ruas familiares, o aspecto antiestético de uma
estalagem, a invenção do corpo anti-higiênico. Não porque esses
fragmentos, coisas miúdas e sem importância, sejam a miniaturização de
um sentido mais universal e totalizante. Mas porque é aí que o poder
ignora princípios de individuação e preserva notavelmente um
desequilíbrio arcaico, sua condição de não estratificado e,
consequentemente, seus pontos inumeráveis de enfrentamento e focos de
instabilidade.

Ocupamo-nos da dinâmica das estratégias de lutas e contra poderes não
estatizáveis, sob a perspectiva de que uma ``estratégia só poderá ser
segunda em relação às linhas de fuga, às suas conjugações, às suas
orientações, às suas convergências ou divergências''.\footnote{Gilles
  Deleuze, ``Désir et plaisir'', 1994, p.~20.} E ficaríamos felizes, ao
final de tudo, se apenas esta última indicação de Deleuze fizesse
sentido, mesmo que, sozinha, reverberasse fora do texto. Mas se nos
pedissem que falássemos as mesmas coisas de um jeito mais leve e direto,
escreveríamos sobre aqueles ancestrais cuja existência foi de um
perigoso excesso de liberdade.\footnote{Falamos em ancestrais
  históricos, e não em vias de acesso a um berço identitário do qual,
  oportunamente, careceríamos, seja por autossatisfação ou porque as
  manhãs são sempre belas e salubres. Disse-me uma vez um poeta do
  subúrbio que Inhaúma é mais antiga que África.} Pai Manoel foi uma
dessas existências encantadas. E, sim, é urgente que um dia se faça a
história inacabada desses encantados, apenas para que nós, perseguidos
pelo passado da barbárie, fôssemos igualmente perseguidos pelo passado
de transbordamentos. Para que uma vez mais a liberdade excedida que nos
é destinada seja ritualizada. Porque no nascimento das nossas ruas houve
primado das linhas de fuga sobre a infraestrutura das fronteiras. Não
batemos no teto do que nos foi transmitido do passado em termos de
práticas de liberdade.

Um exemplo apenas, um falso desvio, e terminamos.

Teria sido natural à população da África Centro-Ocidental ter
improvisado, aqui no Rio de Janeiro, uma experiência de sagrado mais
elástica que as fronteiras entre religião-profano do catolicismo. Se o
complexo religioso da África Centro-Ocidental atravessa um Atlântico e
uma floresta tropical para assimilar, de forma criativa, o indígena
velho, o índio encantado, qual a nacionalidade do Caboclo, por exemplo,
na umbanda? O Caboclo, na canjira dos encantados, qual o seu protocolo?
Aí a noção de sincretismo e nada é a mesma coisa. Ora ela impõe unidade
de contrários, ora mantém um eixo subalterno e suficientemente pouco
coeso para se deixar afetar sem produzir, a contragolpe, efeito
similar.\footnote{Cf. Luiz Antonio Simas, \textit{Fogo no mato}, 2018,
  p.~69.}

Não era excepcional para o povo deslocado do Congo e de outros lugares
da África Central cultuarem os ancestrais dos habitantes mais antigos de
sua nova terra --- neste caso, os índios brasileiros ---, tidos como
transformadores em espíritos locais da água e da terra. Eduardo
Possidonio explica um princípio básico dessas identidades beduínas de
respeito aos primeiros habitantes do local.\footnote{Cf. Eduardo
  Possidonio, \textit{Entre ngangas e manipansos}, 2018, p. 187.}
Acreditava-se que sem a benção dos espíritos nativos a comunidade nova
iria murchar e morrer. Logo, quem por último chegasse procurava quanto
antes unir-se em casamento com antigos locais. Assim estariam
apaziguando ancestrais que não eram os seus próprios, mas dos quais
dependeriam nesta nova terra: o caboclo indígena velho, vestido na
samambaia, cindido com uma cobra coral.

Fronteiras itinerantes, que fazem rasuras no espaço para oferecerem
modos de constituição de identidades nos dois lados da linha. É fenômeno
de mão dupla, ou melhor, encruzilhada, como dizem Simas e Rufino, vindo
de negros e brancos, tendo influência ameríndia, e podendo ser
compreendido como prática de resistência. Ou como fenômeno de fé: a
incorporação de deuses e crenças do outro é vista como acréscimo de
força vital.\footnote{Cf. Luiz Antonio Simas, \textit{Fogo no mato}, 2018,
  p.~69.}

Mbembe chama de assimilação criativa esse princípio de negociação,
pirataria e entrelaçamento.\footnote{Cf. Achille Mbembe, \textit{Crítica
  da Razão Negra} {[}2013{]}, trad. Sebastião Nascimento (São Paulo, n-1
  edições, 2018), p.~175.} É outra coisa, diferente do monoteísmo
cristão anterior ao comércio com África. O cristianismo traz em sua
matriz filosófica já um projeto de universalização que prepara o
colonialismo. Como? Primeiro, é um Deus que faz fronteira com o
paganismo e com o profano. ``Quem não está comigo é contra mim''. Mas o
ciúme divino e a fronteira entre santos e pecadores são
contrabalanceados através de outra figura da violência: a possibilidade
de conversão como sinal da misericórdia e~piedade~de Deus --- quer dizer,
a redenção. E a redenção sob o preço do abandono da antiga existência
corrompida. A conversão exige o apagamento do passado, abolição da
diferença e adesão a uma humanidade universal. Essa universalidade
encarnada pelo conceito de humanidade é a mesma que comandou a violência
da empresa colonialista.

A origem dos candomblés de caboclo, por sua vez, estaria no ritual de
antigos negros de origem banta ``que na África distante cultuavam os
inquices --- divindades africanas presas à terra, cuja mobilidade
geográfica não faz sentido --- e que no Brasil viram-se forçados a
encontrar um outro antepassado para substituir o inquice que não os
acompanhou à nova terra''.\footnote{Reginaldo Prandi; Armando Vallado;
  André Ricardo de SOUZA, ``Candomblé de Caboclo em São Paulo'', em
  Reginaldo Prandi (org.)\textit{, Encantaria brasileira: o livro dos
  mestres, caboclos e encantados} (Rio de Janeiro, Pallas, 2004),
  p.~121.} Os caboclos são espíritos de antigos índios que povoavam este
território, antigos caboclos eleitos por bantos como os verdadeiros
ancestrais em terras nativas. Cultuamos, portanto, caboclos por terem
sido os primeiros donos da terra em que vivemos, os executados pela
espada mas não pela Bíblia.

Foram os donos e, portanto, são agora guias, flutuando no ar e na terra.
Mas alguns caboclos, como os nossos curandeiros do \textsc{xix}, são originários
de lugares imaginários, como o Atlântico. Condenados a andar beduínos em
terras de outro, é como se lá estivessem no imenso mar comum a nossos
antepassados escovando o tempo, flutuando sem fronteira. Não se
afogaram, encantaram-se, fazem a cama de noivo no colo de Iemanjá. E
``me passava um pensamento: nós, os da costa, éramos habitantes não de
um continente, mas de um oceano''.\footnote{Mia Couto, \textit{Terra
  sonâmbula} {[}1992{]} (São Paulo, Companhia de Bolso, 2015), p.~24.}
Partilhamos, eu e a liberdade excedida dos encantados, a mesma pátria,
que é o Atlântico.

\chapter{Documentação consultada}\label{documentauxe7uxe3o-consultada}

\section*{Fontes}\label{fontes}

\begin{enumerate}
\def\labelenumi{\arabic{enumi}.}
\item
  APIAN, P. \textit{La Cosmographie}. Paris: par Vivant Gaultherot, 1551.
\item
  BARBOSA; RESENDE. \textit{Os serviços de saúde pública no Brasil,
  especialmente na cidade do Rio de Janeiro de 1808 a 1907 (esboço
  histórico e legislação) --- Primeiro Volume.} Rio de Janeiro: Imprensa
  Nacional,
\item
\item
  BICHAT, Xavier. \textit{Anatomie pathologique}, dernier cours de Xavier
  Bichat: d'après un ms. autographe de P.-A. Béclard avec une Notice sur
  la vie et les travaux de Bichat / par F.-G. Boisseau. Paris: chez
  J.-B. Baillière, Libraire, 1825.
\item
  BOUDIN, J. \textit{Traité de géographie et de statistique médicales et
  des maladies endémiques.} Paris: J.-B. Baillière et Fils, 1857.
\item
  BUFFON, \textit{Histoire Naturelle, générale et particuliere, avec la
  description du Cabinet du Roy, Tome Quatorzième}. A Paris, de
  l'Imprimerie Royale, 1749.
\item
  CHERNOVIZ, P. L. N. \textit{Diccionario de Medicina Popular} ---
  \textit{Volume Terceiro.} 2ª ed.~Rio de Janeiro: Eduardo \& Henrique
  Laemmert, 1851.
\item
  CHERNOVIZ, P. L. N. \textit{Diccionario de Medicina Popular --- Volume
  Primeiro A-F.} Paris: A. Roger \& F. Chernoviz, 1890.
\item
  DEBRET, Jean-Baptiste. \textit{Viagem pitoresca e histórica ao Brasil}.
  Tomo I, volume II. Rio de Janeiro: Martins, 1949.
\item
  DURAND-FARDEL. «~Des maladies contagieuses et infectieuses. A propos
  d'un mémoire de M. Audouard.~» \textit{Revue médicale française et
  étrangère, journal des progrès de la médicine hippocratique}, t. II.
  Paris: 1850.
\item
  EWBANK, Thomas. \textit{Life in Brazil, or a Journal of a visit to the
  land of the cocoa and the palm.} New York: Harper \& Brothers
  publishers, 1856.
\item
  FONSSAGRIVES, J.-B. \textit{Hygiène et Assainissement des villes}.
  Paris: J.-B. Ballière \& Fils, 1874.
\item
  FRACASTORO, J. \textit{La contagion, les maladies contagieuses et leur
  traitement}. Paris: Société d'éditions scientifiques, 1893.
\item
  GARDNER, G. \textit{Travels in the interior of Brazil.} London: Reeve
  Brothers,
\item
\item
  HALLÉ, J.-N. \textit{Traité d'Hygiène}. Paris: chez M. Gautret, 1838.
\item
  LA CLOTURE, L. \textit{Observations sur les maladies epidémiques.}
  Paris: De l'imprimerie de Vincent, 1770.
\item
  MOTARD, A. \textit{Traité d'Hygiène Générale --- Tome Premier.} Paris: J.
  B. Baillière et Fils, 1868a.
\item
  \_\_\_\_\_\_. \textit{Traité d'Hygiène Générale --- Tome Second.} Paris:
  J. B. Baillière et Fils, 1868b.
\item
  PERDIGÃO MALHEIRO, A. M. \textit{Escravidão no Brasil: ensaio
  histórico-jurídico-social --- Parte 3ª}. Rio de Janeiro: Typographia
  Nacional, 1867.
\item
  PIMENTEL, Antonio M. de Azevedo. \textit{Subsídios para o estudo de
  higiene do Rio de Janeiro}. Rio de Janeiro: Tipografia e Lit. De
  Carlos Gaspar da Silva, 1890.
\item
  PROUST, A. \textit{Traité d'Hygiène}. 2ed. Paris: G. Masson Éditeur,
  1881.
\item
  PTOLOMEU. \textit{Tetrabiblos.} Trad. inglesa J. M. Ashmand. London: W.
  Foulsham \& CO.
\item
  REBOUÇAS, A. \textit{Conferederação abolicionista}. \textit{Abolição
  imediata e sem indenização}. Panfleto n.~1. Rio de Janeiro: Typ.
  Central, 1883.
\item
  REBOUÇAS, Manuel Maurício. \textit{Dissertation sur les inhumations en
  géneral (leurs resultats fâcheux lorsqu'on les pratique dans les
  églises et dans l'enceinte des villes, et des moyens d'y rémedier par
  des cimetières extra-muro).} Thèse présentée et soutenue à la Faculté
  de Médicine de Paris. Paris: l'imprimerie de Didot le Jeune, 1831.
\item
  REINHIPO, R. M. \textit{Trattado Unico das bexigas e sarampo, oferecido
  a D. João de Sousa.} Lisboa: na oficina de João Galrao, 1683.
\item
  RIBEIRO, Cândido Barata. \textit{Quais as medidas sanitárias que devem
  ser aconselhadas para impedir o desenvolvimento e propagação da febre
  amarela na cidade do Rio de Janeiro?} (Tese apresentada à Faculdade de
  Medicina do Rio de Janeiro como primeira prova de concurso de Lente
  Substituto a um lugar vago, na seção de Ciências Médicas). Rio de
  Janeiro: Typographia do Direito. 1877.
\item
  RIBEYROLLES, Charles. \textit{Brazil Pittoresco --- Tomo II}. Rio de
  Janeiro: Typographia Nacional, 1859.
\item
  ROSA, Joaquem Ferreira da. \textit{Tratado único da constituição
  pestilencial de Pernambuco}. Lisboa: Oficina de Miguel Manescal, 1694.
\item
  REGO, J. P. \textit{Esboço Histórico das epidemias que têm grassado na
  cidade do Rio de Janeiro desde 1830 a 1870}. Rio de Janeiro:
  Typographia Nacional, 1872.
\item
  REGO, J. P. \textit{Historia e Descripção da Febre Amarella Epidemica
  que grassou no Rio de Janeiro em 1850}. Rio de Janeiro: Typographia de
  F. de Paula Brito, 1851.
\item
  SIGAUD, J. F. X. \textit{Du Climat et des Maladies du Brésil}. Paris:
  Fortin, Masson et Cie, Libraires, 1844.
\item
  SOUTO, Vieira. \textit{Melhoramento da cidade do Rio de Janeiro: crítica
  dos trabalhos da respectiva comissão.} Rio de Janeiro: Lino C.
  Teixeira \& C., 1875.
\item
  SYDENHAM, T. \textit{Médicine Pratique}. Paris: chez Théophile Barrois
  le jeune, 1784.
\item
  SYNDENHAM, T. \textit{The Works --- Vol. 1.} Translated from the latin
  edition by R. G. Latham. London: printed for the Sydenham Society,
  1848.
\item
  TORRES HOMEM, J. V. \textit{Elementos da Clínica Médica --- seguidos do
  anuário das mais notáveis observações colhidas nas enfermarias de
  clínica médica em 1869}. Rio de Janeiro: Nicoláo A. Alves, 1870.
\item
  \_\_\_\_\_\_. \textit{Estudo clínico sobre as febres do Rio de Janeiro}.
  Rio de Janeiro: Livraria clássica de Nicolao Alves, 1877.
\item
  TOURTELLE, \textit{Traité d'Hygiène}. Paris: chez M. Gautret, 1838.
\item
  VALENTIN, L. \textit{Traité de la fiévre jaune d'Amérique}. Paris:
  Méquignon Libraire, 1803.
\item
  VIANNA, A. \textit{Quintal de nagô e outras crônicas}. Salvador:
  Publicação da Universidade Federal da Bahia, 1979.
\end{enumerate}

\section*{Documentos oficiais}\label{documentos-oficiais}

\subsection*{Arquivo geral da cidade do Rio de
Janeiro}\label{arquivo-geral-da-cidade-do-rio-de-janeiro}

\begin{enumerate}
\def\labelenumi{\arabic{enumi}.}
\item
  AGCRJ Códice 43.1.25 --- Estalagens e Cortiços (Requerimentos e outros
  papéis relativos a existência e à fiscalização sanitária e de costumes
  dessas habitações coletivas --- 1834 a 1880)
\item
  AGCRJ Códice 41.3.36 Fundo Câmara Municipal --- Série Cortiços e
  Estalagens (Ofícios da Secretaria da Polícia e do Ministério do
  Império sobre as medidas a adotar com referência aos cortiços - 1860)
\item
  AGCRJ Códice 43.3.26 --- Fundo Câmara Municipal --- Série epidemias
  (Febre Amarela --- Medidas Higiênicas --- Portaria do Ministro do
  Império Visconde de Monte Alegre, etc. --- 1850)
\item
  AGCRJ Códice 44.2.7 --- Habitações coletivas, estalagens ou
  ``cortiços'' --- Vários papéis sobre medidas higiênicas reclamadas
  pelas autoridades, projetos de posturas e outros, concernentes ao
  assunto --- 1855, 1864 a 1866 e 1868
\item
  AGCRJ Códice 6.1.37 --- Escravos --- Assuntos: Casas alugadas ou
  sublocadas a escravos, muitos dos quais fugidos e malfeitores ---
  Ofício do Chefe de Polícia (1860)
\item
  BR RJAGCRJ 8.3.7 Fundo Câmara Municipal --- Série Higiene Pública
  (Higiene e Saúde Pública / Avisos / 1850-1854)
\item
  BR RJAGCRJ Códice 8.4.6 Fundo Câmara Municipal --- Série Higiene
  Pública
\item
  BR RJAGCRJ 8.4.20 Fundo Câmara Municipal --- Série Higiene Pública
  (Salubridade do Rio de Janeiro / Vários papéis, reclamações, projetos,
  pareceres etc. / 1830 a 1888)
\item
  BR RJAGCRJ 8.4.22 Fundo Câmara Municipal --- Série Higiene Pública
  (Salubridade / Editais, medidas higiênicas; instruções; projetos de
  posturas; desinfecções; limpeza; melhoramentos na cidade para
  preservá-la de epidemias; cortiços etc.)
\item
  BR RJAGCRJ Códice 8.4.24 Fundo Câmara Municipal --- Série Higiene
  Pública
\item
  BR RJAGCRJ 41.3.35 --- Fundo Câmara Municipal --- Série Cortiços e
  Estalagens
\item
  BR RJAGCRJ Códice 41.3.37 Fundo Câmara Municipal --- Série Cortiços e
  Estalagens
\item
  BR RJAGCRJ Códice 43.1.27 Fundo Câmara Municipal --- Série Cortiços e
  Estalagens
\item
  BR RJAGCRJ 43.1.25 --- Fundo Câmara Municipal --- Série Cortiços e
  Estalagens
\item
  BR RJAGCRJ Códice 43.1.26 Fundo Câmara Municipal --- Série Cortiços e
  Estalagens
\item
  BR RJAGCRJ 43.3.27 Fundo Câmara Municipal - Série Epidemias
\item
  BR RJAGCRJ 44.2.7 Fundo Câmara Municipal - Série Habitações Coletivas
\item
  AGCRJ Códice 46.4.47 Fundo Câmara Municipal --- Série Casas para
  Operários e Classes Pobres
\item
  AGCRJ Códice 48.4.59 Fundo Câmara Municipal --- Série Prostituição
\item
  AGCRJ Códice 48.4.61 Fundo Câmara Municipal --- Série Prostituição
\item
  \textit{Posturas e editas da Câmara Municipal do Rio de Janeiro:
  1832-1890,} Códices 6-1-18, 6-1-28, 18-1-72 e 18-2-2 a 18-2-12
\end{enumerate}

\subsection*{Arquivo Nacional}\label{arquivo-nacional}

\begin{enumerate}
\def\labelenumi{\arabic{enumi}.}
\item
  MAÇO IS 4-23 --- Série Saúde --- Higiene e Saúde Pública --- Instituto
  Oswaldo Cruz (1850-1859)
\item
  MAÇO IS 4-24 --- Série Saúde --- Higiene e Saúde Pública --- Instituto
  Oswaldo Cruz (1850-1859)
\item
  MAÇO IS 4-25 --- Série Saúde --- Higiene e Saúde Pública --- Instituto
  Oswaldo Cruz (1860-1869)
\item
  MAÇO IS 4-26 --- Série Saúde --- Higiene e Saúde Pública --- Instituto
  Oswaldo Cruz (1860-1869)
\item
  MAÇO IS 4-27 --- Série Saúde --- Higiene e Saúde Pública --- Instituto
  Oswaldo Cruz (1870-1879)
\item
  MAÇO IS 4-28 --- Série Saúde --- Higiene e Saúde Pública --- Instituto
  Oswaldo Cruz (1870-1879)
\end{enumerate}

\subsection*{Legislação}\label{legislauxe7uxe3o}

\begin{enumerate}
\def\labelenumi{\arabic{enumi}.}
\tightlist
\item
  Coleção de Leis do Império --- 1850, 1851, 1876
  (http://www2.camara.leg.br/atividade-legislativa/legislacao)
\end{enumerate}

\subsection*{Imprensa e revistas
científicas}\label{imprensa-e-revistas-cientuxedficas}

\begin{enumerate}
\def\labelenumi{\arabic{enumi}.}
\item
  Annaes Brasilienses de Medicina (1853-1874)
\item
  Diario do Rio de Janeiro (1850-1853 / 1876)
\item
  Cidade do Rio de Janeiro (1880-1889)
\item
  Gazeta da Tarde (1880-1889)
\item
  Gazeta de Notícias (1880-1892)
\item
  Jornal do Brasil (1880-1892)
\item
  Jornal do Commercio (1850)
\item
  O Fluminense (1880-1889)
\item
  Revista do Instituto Polytechnico Brasileiro. (1872-1873)
\end{enumerate}

\chapter[Posfácio]{Posfácio \subtitulo{\textit{Dialética da quarentena de todos os dias}}}

\begin{flushright}
\textsc{claudio medeiros e victor galdino}
\end{flushright}

\epigraph{A modernidade costumava medir tudo à luz do atraso arcaico do qual 
nos pretendia arrancar; daqui em diante, tudo se mede à luz de seu desmoronamento próximo.}{\textit{Aos nossos amigos}, \textsc{comitê invisível}}

Terminado o carnaval, o ano já engatando no meio de março, veio a
notícia da primeira morte. Da noite para o dia, nossos amigos tinham se
isolado e lidavam com isso das mais variadas maneiras. Naquela mesma
semana, trocávamos áudios sobre quarentena, apareceram ideias, nossos
estudos se cruzaram de forma inesperada. É uma experiência curiosa
escrever junto. Somos educados no fetichismo autoral da universidade,
pequeno iluminismo cotidiano da escrita solitária e individual, papel
que cumprimos não sem boa dose de adoecimento. Pode parecer óbvio, mas a
escrita conjunta era também uma demanda por algo mais saudável,
tentativa de nos limparmos um pouco desses efeitos estranhos que o
mandarinato acadêmico nos deixou, como diz nosso amigo Jonnefer Barbosa.
É o tipo de oportunidade que, se planejada, provavelmente não teria
funcionado. E, aliás, as coisas saírem diferentes do planejado é uma
coisa bonita em um mundo que se caracteriza atualmente pela proximidade
do desmoronamento.~

Era preciso, primeiro, resistir à tentação de \textit{adequar} ou querer
dizer pela teoria aquilo que desejamos ---- esse curioso jogo em que o
filósofo sempre parece vencer. Mas uma primeira leva de textos de
renomados filósofos ia aparecendo a toque de caixa, aproveitando uma
oportunidade que era a de todos nós que, de uma forma ou de outra,
assumimos o exercício crítico como vocação profissional. Mas havia algo
de desconfortável, um sentimento de que não estávamos realmente aptos a
jogar aquele jogo ---- quem nos conferiu a potência da futurologia ou da
visão da totalidade? A insistência na afirmação da adequação entre
teorias pré-pandêmicas e realidade parecia apenas um exercício de ``eu
avisei'' que nos saturava. Na impotência melancólica que segue a morte
dos sonhos do progressismo iluminado, soa apenas como outra maneira de
verbalizar a velha imagem da história como algo que certificará os que
estão do lado certo. \textit{Mas nós somos aqueles que a História sempre
abandonou}.

Nosso texto ganhava um rosto, não seria sobre o que um vírus poderia
trazer de utópico ou distópico, mas sobre o conflito entre isolamento
compulsório e normalidade da forma de vida metropolitana, o desejo de
retomar a normalidade por vias analgésicas, a importância da disputa
pelas ``imagens do retorno''. O futuro não nos interessava a não ser
como algo que ``só se constrói na dinâmica de um presente''.\footnote{Jacques
  Rancière, ``Uma boa oportunidade?'', 26.mai.2020.}
Os vírus simplesmente não faziam parte de nossos engajamentos cotidianos
pois estávamos precisamente fugindo deles. É difícil medir o sucesso
dessas tentativas de distanciamento. Talvez nosso texto tenha sido lido
como mais do mesmo. Talvez estivéssemos contaminados demais. O que nunca
negaríamos é que somos um tanto ridículos, como absolutamente
\textit{todos os} filósofos. É a partir desse universal que nos
autorizamos a falar e integrar a eternamente estranha comunidade
filosófica. Então chegou o texto de Paul B. Preciado, e foi como um
alento: ``De todas as teorias da conspiração que havia lido, a que mais
me seduziu foi uma que dizia que o vírus havia sido criado em
laboratório para que todos os \textit{losers} do planeta pudessem
recuperar seus e suas ex ---- mas sem qualquer obrigação de retorno
verdadeiro. Estufado de lirismo e de angústia acumulados durante uma
semana de doença, de medos e de dúvidas, a carta à minha ex não era
apenas uma declaração de amor tão poética quanto desesperada, mas era,
sobretudo, um documento vergonhoso para quem a assina {[}\dots{}{]}. O novo
estado de coisas, com sua imobilidade escultural, oferecia um novo grau
de \textit{what the fuck}, mesmo naquilo que apresentava de
ridículo''.\footnote{Paul B. Preciado, ``La conjuration des losers'',
  27.mar.2020,
  \textless{}https://www.liberation.fr/debats/2020/03/27/la-conjuration-des-losers\_1783349{\textgreater{}.}}

Não escrevemos cartas de amor, mas outros documentos de nosso estado de
espírito. Queríamos dizer algo, mas não queríamos estar certos sobre
algo longe demais do nosso alcance. ``Todos os gênios são sanguessugas,
por assim dizer. Eles se alimentam da mesma fonte ---- o sangue da
vida''.\footnote{Henry Miller, \textit{Sexus {[}1949{]}} (Nova York,
  Groove Press, 1965), p.~19.} Nós que somos ninguém participamos dessa
\textit{conspiração encantada de} losers. Nem por isso deixamos de
finalizar projetos, inventar outros, e manter nossa saúde através de
encontros com nossos amigos via aplicativos. O ensaio, que levaria o
título \textit{Quarentene-se}, acabou tornando-se uma trilogia publicada
no site \textit{Outras Palavras} entre os dias 24 de março e 9 de abril.
Como lá, este posfácio não é nenhum exercício de futurologia. Também não
nos interessa reconhecer padrões sociológicos do passado no
comportamento da nossa sociedade diante da pandemia. Talvez tenhamos
mesmo escrito cartas de amor. Existe filosofia que não seja o
endereçamento de um amor? Não por acaso os filósofos nunca venceram pelo
argumento ---- felizes os que desistem de se perturbar com isso. Nós que
aprendemos a levantar barricadas por amor ainda queremos reunir blocos
de desejos e recusas nas praças e ruas do fim do mundo. Quem vai
projetar um futuro pós-pandêmico se mal conseguimos lidar com a
dialética caótica da quarentena?

\textit{Quarentene-se} tinha sido mobilizado tanto por \textit{Mármore e
Barbárie}, escrito ao longo de alguns anos, quanto por um acontecimento
incompreensível, uma vez que potencialmente capaz de interromper a
repetição da normalidade. Este acontecimento, a quarentena, fez com que
\textit{Mármore e Barbárie} conjurasse alguns gestos do passado em meio a
tantos fantasmas --- e até permitiu que continuassem a manifestar uma
``invencível obstinação em divagar''.\footnote{Michel Foucault, ``Vida
  dos homens infames'' {[}1977{]}, em \textit{Ditos e Escritos IV}, 2006,
  p.~210.} Achamos que um ensaio filosófico é o exercício de recorte
daquilo que, em acontecimentos, desestabiliza o presente. Ele não trata,
portanto, de acontecimentos \textit{em si}, como se tentasse descrever o
real através de um misterioso gesto científico que não é imaginado como
propriamente científico. Ele trata de singularidades radicais, da
inatualidade do imprevisível como abertura para desorientação das formas
de vida. Daí o caráter experimental do pensamento ao qual nos dedicamos,
daí esses exercícios conjuntos de saúde e pensamento. O que nos inquieta
é a possibilidade da crise, o momento crítico, a experiência de
suspensão das experiências possíveis diante de um presente desacontecido
no regime de quarentena. ``Desacontecimento'' não denota, em um golpe, a
reviravolta que outrora se esperou das jornadas revolucionárias. Como
escreveu Rancière no final de maio, não é óbvio que o confinamento nos
dirija a uma mudança de ``paradigma civilizacional'' em que o agente
revolucionário é a pandemia, quer dizer, o não humano.~

Em geral, o que testemunhamos como o desacontecimento do presente foi
sendo sugado para dentro da reorganização do cotidiano produtivo de
sempre. É o que chamaremos adiante ``dialética da quarentena''. De
nossas residências, assistimos à maneira pela qual esse movimento de
negação, o regime de quarentena, realizou-se na reafirmação da
normalidade: \textit{podemos parar precisamente porque podemos continuar},
ainda que de outra maneira. O vazio traumático deixado pela quarentena
teve de ser preenchido por um cotidiano governado pelo imperativo da
produtividade, ainda que, inicialmente, isso tenha ocorrido de forma
caótica, com a angústia do esvaziamento e a urgência pela positividade
nos empurrando violentamente de volta à circulação normal do mundo. Os
movimentos que caracterizam a normalidade não partem de um porto seguro.
Como não faria sentido falar de futuro enquanto enclausurados numa sala
de espelhos, o que esperamos de uma prática filosófica é apenas algum
repertório de conceitos que sirvam para visualizar problemas e passar em
revista aquilo que somos. Que pelo menos essas novas formas de
visualização possam dispor de um plano de imanência não usual, ou seja,
que coincida não com a garantia deste presente, mas com \textit{as
possibilidades de sua destituição}. Isso não seria possível sem um
exercício constante de desaprendizagem. Por amor à filosofia, não
negociamos seu poder de deslocamento psíquico, de estranhamento do
autoevidente. \textit{Deixemos a segurança para os funcionários da
delegacia de Platão}.

A epidemia na sua \textit{causalidade social} inexistiu ao longo de um
grande período na história da medicina. O contágio, a transmissão à
distância de pessoa a pessoa, é fenômeno recente. A experiência que
permite que cada pessoa seja potencialmente suporte de uma vida diminuta
infecciosa é uma experiência de meados do \textsc{xix}, quando se consolida o
movimento de \textit{descida} da doença, do \textit{corpo do mundo} e seus
cataclismos meteorológicos, para o interior do corpo-organismo. É nesse
interior que o ponto de partida para o contágio foi localizado, e isso
demanda uma dobra da atenção social, um olhar curvado para baixo e para
si. Os planetas deixaram de falar. Desaparecem os motivos para olhar em
sua direção buscando alguma forma de verdade ---- eles se tornaram
``realidades totalmente reduzidas à linguagem'',\footnote{Jacques Lacan,
  \textit{O Eu na teoria de Freud e na técnica da psicanálise}
  \textit{(1954-1955),} trad. Marie Christine Laznik (Rio de Janeiro,
  Jorge Zahar, 1985), p.~302.} aglomerados de propriedades calculáveis
sem um \textit{quem} que responda a nossas preces, angústias e esperanças.

Mesmo com Pasteur e o advento da microbiologia, algumas imagens
extra-científicas permanecerão nos imaginários sociais. A doença como
experiência atmosférica se insinua, em alguma medida, nas práticas de
higiene individual alheias às possibilidades efetivas de contágio pelo
contato com o \textit{outro} infectado. Práticas policiais e tendências
científicas que, no século \textsc{xx}, produziram em nós um corpo higiênico e
delimitaram \textit{socialmente} o conceito de contágio, não eliminaram de
vez a experiência atmosférica da epidemia. Pode-se higienizar o corpo no
lugar de purificar o ar de uma cidade, pode-se higienizá-lo contra um
inimigo etéreo cuja ameaça é de caráter estratosférico. Hoje, tudo se
passa como se o pânico individual, diante do fato de que nossos projetos
existenciais descarrilharam da normalidade, precisasse ser compensado
pela imagem da perda momentânea do equilíbrio da natureza ---- de todos
os lados, sofremos pressão para executar procedimentos de purificação
que eventualmente farão retornar a vida que \textit{deveríamos ter.} ``Se
cada um fizer a sua parte'' o retorno será automaticamente garantido por
hábitos de assepsia das residências. Todo cuidado é pouco quando o que
está em jogo é a plena indistinção entre vida biológica e econômica. Já
sem planetas transbordantes de sentido e sem integrar um cosmos
organizado, a população quarentenada precisa lidar com a suspensão
indefinida da normalidade recorrendo a paliativos que preencham um
\textit{vazio}. Porém, em tempos distantes, o retorno à normalidade após a
epidemia envolvia a manutenção do \textit{mundo}, ainda que com número de
habitantes drasticamente reduzido. Satisfeitos com as medidas para
aplacar sua ira, os deuses retornavam a seus afazeres, o cosmos se
reencontrava consigo.

A imagem da cidade colonial retrógrada e anti-higiênica, os primeiros
impulsos de contenção da aglomeração nauseabunda e, mais tarde, o
nascimento do indivíduo favelado foram filhos de uma corrida sanitária
de outra ordem, ativada com pioneirismo nos regimes de controle
implantados na segunda metade do \textsc{xix}. No Brasil, a quarentena foi
outrora o cordão sanitário emergencial que durou enquanto perduraram
circunstâncias epidêmicas. Enquanto único cordão sanitário permanente, a
utopia realizada da quarentena absoluta encontrou na \textit{metrópole e
na arquitetura moderna} seus melhores modos de expressão. O efetivo
processo de despatologização ou desinfecção da cidade, em marcha a
partir dos primeiros estados de emergência, dependeria, no pós-abolição,
de um projeto de profilaxia racial. Foi nesse movimento que projetos de
aburguesamento da rua, na passagem do \textsc{xix} para o \textsc{xx}, passaram por uma
condenação dos hábitos coloniais e empreenderam não apenas a radical
remodelação do perímetro urbano do Rio de Janeiro, mas uma normatização
dos nossos corpos segundo o parâmetro do domesticável. Se contarmos com
o fato de que o Rio de Janeiro oitocentista foi laboratório de epidemias
avassaladoras, como o retorno vinha sendo organizado ao longo desse
segundo momento?

O retorno, ao longo da sucessão das epidemias do \textsc{xix}, foi o retorno da
quarentena expandindo raios de ação: em princípio, o quadriculamento nos
cemitérios, depois a repartilha dos transeuntes indesejáveis através da
reinvenção estrutural da cidade, através de reformas de ``melhoramento''
urbano, capitalização do solo, gentrificação, racismo, cosmofobia e
extermínio. Na construção do Rio de Janeiro higienizado, a quarentena
reproduziu o retorno como desdobramento que não era outro senão a
intensificação de hierarquias, a tentativa de desencantamento das formas
de vida e o policiamento colonial da circulação de pretas e pretos.

A atual pandemia do Covid-19 nos coloca diante de uma situação sem
precedentes. Nossa cultura filosófica talvez nos levasse a crer que, em
meio ao colapso da política sanitária, os governos perdessem o controle
sobre as pessoas de modo que, como antes, a resposta seria uma
radicalização da quarentena no sentido de lidar com o problema da
circulação de pessoas. Obedecendo a essas expectativas, nossas formas
habituais de visualizar acontecimentos em termos \textit{biopolíticos}
destituiriam os próprios acontecimentos da singularidade de que são
dotados no contexto pós-colonial. Nossos neoliberais têm como imagem de
mundo um deserto organizado por tabelas, gráficos e números esvaziados
de qualquer subjetividade. A imagem do retorno como \textit{vacina
econômica} é vendida como algo que vale mais que qualquer medida de
isolamento social ---- o governo não precisa nos policiar no sentido
disciplinar, mas no sentido de nos fazer continuar circulando como se
isso bastasse para desfazer a ira do mercado, essa abstração cadavérica
para a qual seremos entregues ao sacrifício. Um pensamento mágico e
desencantado ao mesmo tempo. Esse retorno seria, então, negação da
quarentena como \textit{novidade} e \textit{abertura} e reafirmação brutal
da quarentena que já nos é familiar sob o nome cínico de
``normalidade''. É o que chamamos, por falta de outro nome melhor, de
``dialética da quarentena''. Assim como os escravocratas, contando com a
complacência da lei diante de uma escravidão extralegal, exigiram
indenizações no pós-abolição porque talvez sequer escravidão tivesse
havido, talvez os paramilitares apostem que não valerá a disputa pelo
retorno do \textit{mesmo} já que, em sua fantasia negacionista, sequer
existe pandemia.

Não à toa, até o momento, este estado de emergência não inovou em termos
de efeitos sociais: o cenário atual apenas nos ajuda a requalificar
conceitualmente uma condição que já é a nossa, a de sobreviventes em uma
\textit{democracia} \textit{de milícias:}

\begin{quote}
\ldots{} devemos nos proteger contra essa máquina de poder cuja ``política
sanitária''' é ela própria uma ameaça sanitária, que coloca em perigo
real nossa sobrevivência. E para os povos indígenas é ainda pior. O
Estado se esforça há muito tempo para separá-los de suas terras e de
seus corpos. E agora não faz nada para protegê-los da epidemia; ao
contrário, incentiva aqueles que são uma ameaça direta para eles, como
os garimpeiros. Então, a contradição pode ser superada? Talvez só por
fora desse governo assassino. Hoje, no Brasil, face à negligência do
Estado, alguns coletivos se organizam para se encarregar das tarefas
sanitárias, do cumprimento das regras de confinamento, etc. Em vez de
esperarem ser protegidos, protegem-se a si mesmos.\footnote{Eduardo
  Viveiros de Castro, ``O que está acontecendo no Brasil é um
  genocídio'', 19.mai.2020, trad. Francisco Freitas,
  \textless{}https://www.n-1edicoes.org/textos/104\textgreater{}}
\end{quote}

O bolsonarismo prescinde de guarda-costas institucionais. Todo cenário
institucional é para ele insuportavelmente postiço porque, a rigor,
enquanto fenômeno de cultura, ele não precisa de nenhuma coreografia
jurídica: ele funciona tanto melhor através da capilaridade dos grupos
paramilitares do que nas margens do jogo jurídico-institucional. É como
se, hoje, sujeitos à quarentena, fosse-nos possível ter, por uma faísca
de instante, uma visão retrospectiva de uma coisa muito cotidiana porém
nem sempre óbvia em sua radicalidade: na democracia paramilitar, o novo
coronavírus torna-se a versão química de instrumentos bélicos
\textit{high-tech,} como nossos próprios drones israelenses ou caveirões
voadores. Ambos são usados para conter um ``excesso de presença''.
Paralelamente, o obscurantismo na liberação das estatísticas mortuárias,
a opção pela baixa testagem e a subnotificação de casos suspeitos
tiveram a capacidade de colocar as fazendas de cadáveres à disposição da
polarização ideológica.~

A construção ideológica da pandemia (reduzida a moralismos e
performances identitárias de esquerda/direita) está na mesma esteira da
moralização de debates delicadíssimos como a execução penal nos
presídios, o varejo de entorpecentes, a invasão de reservas indígenas, a
renda básica universal etc. Essas coisas não nasceram da cabeça dos
mafiosos da situação, porém o bolsonarismo inova na capacidade de
governar em função de e através de crises desenvolvidas explicitamente
como programa. Que crise podemos dizer que tenha causado desconforto ou
interrompido o cotidiano de nossas instituições? O atual governo existe
\textit{positivamente} na crise, ele funciona tanto melhor quanto mais a
crise na saúde pública escava dezenas de milhares de covas rasas. Ele
funciona tanto melhor quanto mais as pessoas correm, de um lado para o
outro, desesperadas. O genocídio viral é uma possibilidade com a qual a
democracia das milícias conta como oportunidade de aceleração da
higienização racial que, como se tentou descrever neste livro, é o
processo que de fato possui raízes na vida social do Império e no
massacre das múltiplas formas de aquilombamento, abolicionismos e
fugitividades.~

Alguém poderia contestar que nossa qualificação de \textit{crise} ou
\textit{momento crítico}, no que diz respeito à pandemia, apenas faz
sentido se levamos muito a sério a existência do Estado como régua para
a normalidade. E com razão. A reprodução de um vasto repertório
teórico-acadêmico que desvia ou adia a perspectiva do boicote à
necessidade de Estado é mato. O fascismo encarnado na presidência não
dilacerou o laço social, senão sob a economia das expectativas da classe
média. A \textit{liberdade excedida} da diáspora mestiça nos subúrbios,
articulada em cenas deste livro, assim como os mutirões e redes de apoio
nas quebradas, que agora ressurgem para a contenção dos efeitos da
Covid-19, não são creditadas ao protagonismo do ativismo ou dos fiadores
do Estado. São reativação de táticas de sobrevivência e reservas de
energia memorizadas de uma biblioteca ancestral. Pertencem a um antigo
repertório de lutas sem porta-vozes gregários, sem gestos heróicos,
longe dos sistemas de referência do Império. Assim também o é a medicina
de Pai Manoel das Matas, os Caboclos voltados para a linha de cura, a
capacidade integradora das habitações populares no século \textsc{xix} no
aquilombamento simbólico de laços familiares interrompidos pelo
tráfico.~

O Estado deixa de ser uma instância de assistência social apenas para
quem espera que ele cumpra alguma missão salvacionista ---- para quem
adere a essa expectativa, é um fato que existe uma crise hoje. A recusa
à alternativa de visualizar as caracterizações sobre crise e normalidade
em termos de \textit{valor} compromete o progressista, não apenas na sua
missão de ser profeta de uma ciência política das crises. Também
escancara de que forma o sujeito, ao comprometer-se com a \textit{ciência
política das crises}, compromete-se com este mundo e a ele se oferece em
sacrifício. E aí o destino do sujeito fiador do regime do \textit{mesmo}
será não o de intérprete das crônicas da realeza, mas garantidor do
retorno à normalidade que o espera.~

Quantas tragédias não foram geradas na sucessão dos tronos? Não fugimos
à regra. Hoje, como em \textit{MacBeth}, \textit{Rei Lear} ou \textit{Júlio
César}, não é a sucessão do trono a maior urgência, mas o desabamento
deste mundo. É um problema de ordem \textit{cosmológica}. À morte do rei e
à iminência do fim do Estado seguirá a missão de salvar o mundo, o
\textit{meu} mundo. Há uma interdição freudiana, algo como uma interdição
ao incesto que lança o olhar para longe do fim do Estado ---- mas não do
fim da crise ---- sob o risco de desbloqueio de visões escatológicas
extremamente dolorosas.

\begin{quote}
\textit{MacDuff} --- A destruição concluiu sua obra-prima. Arrombou o
sacrilégio assassínio o templo ungido do Senhor, e a vida roubou do
próprio altar\ldots{} Ide até o quarto e a vista destruí ante outra
Górgona. Quero ficar calado. Ide vós mesmos, para depois falardes.
Despertai! Despertai! Traição e morte! Malcolm, Banquo, Donalbain,
depressa, sacudi esse sono de penugem, simulacro da morte, e vinde a
própria morte encarar. De pé! A imagem vede do grande julgamento.
Malcolm! Banquo! Vinde como das tumbas, como espíritos, para ver este
horror\dots{}~

\textit{MacBeth} --- Se eu tivesse morrido uma hora, apenas, antes de isto
se dar, teria tido uma vida abençoada. Doravante nada mais há de sério
no universo. Tudo é farandolagem; a honra e a glória já não existem\dots{} A
fonte, a origem, o princípio, secou de vosso sangue, a própria origem já
parou de todo.\footnote{William Shakespeare, \textit{MacBeth
  {[}1606/1623{]},} trad. Carlos Alberto Nunes (São Paulo,
  Melhoramentos, 1957), p.~213-14.}
\end{quote}

A grande sacada do sacerdote estadista foi inverter o jogo: fazer crer
que viver sem e mesmo \textit{contra} o Estado e suas perspectivas
teóricas --- revolucionárias ou hobbesianas --- é alguma forma de
mistério, envolto numa aura infernal que nem sempre afastará aqueles
cuja \textit{sanha de verdade} não conhece limites. Ao mesmo tempo, viver
\textit{com} o Estado nos aparece como desdobramento autoevidente do que
somos. E quando isso é colocado em questão, surgem viaturas pra todo
lado exigindo todo tipo de elaboração complexa de uma alternativa, para
que nos ocupemos da revolução enquanto atendemos aos critérios da
normalidade acadêmica. Surge daí uma situação inversa àquela que o
liberalismo nos vende: não é que cada um tem o poder de ``fazer sua
parte'' e mudar o mundo para melhor; é que cada pessoa é responsável
pela circulação da normalidade e não pode dela se sacudir sozinha. É
como se a imagem-trauma do fim do Estado só pudesse aparecer no discurso
progressista como uma peça do museu do futuro. Quem sabe boa parte do
pânico da classe média diante do descarrilhamento do mundo não seja só
reflexo de uma frustração com um Pai-Estado desorientado e nada
envergonhado da sua incapacidade para conter o vírus.

\begin{flushright}
Ilha do Governador, 15 de julho de 2020.
\end{flushright}


\part{Paratexto}\label{paratexto}

\chapter*{}

\section{O autor}\label{o-autor}

Claudio Vinicius Felix Medeiros é professor de Filosofia Geral na
Universidade Federal Fluminense, com atuação nos campos de Ética,
Filosofia Política, Filosofia da História e Filosofia Contemporânea,
tendo escrito artigos sobre os temas da liberdade, do racismo, da
descolonização e da ``biopolítica'' --- conceito extraído da obra do
filósofo francês Michel Foucault, no qual Medeiros apoia grande parte de
seu pensamento. Além disso, Medeiros dedica também parte de sua pesquisa
ao pensamento dos abolicionistas afro-brasileiros do século \textsc{xix} (como
André Rebouças, Luiz Gama e José do Patrocínio).

\textit{História da experiência das epidemias no Brasil} é um livro
derivado de sua tese de doutorado, e se destaca especialmente pela
riqueza de descrições e multiplicidade de fontes. Ora, Medeiros é também
escritor e poeta (autor de \textit{Mármore e Barbárie} e \textit{Zumbimalê
Pivete}), e isto dá a seu texto um caráter fronteiriço, entre literatura
e filosofia, narrativa histórica e experiência do pensamento. Para isso
mobiliza fontes literárias modernas e contemporâneas e rigor de pesquisa
arquivística.

\section{A obra}\label{a-obra}

Claudio Vinicius Felix Medeiros é professor de Filosofia Geral na
Universidade Federal Fluminense, com atuação nos campos de Ética,
Filosofia Política, Filosofia da História e Filosofia Contemporânea,
tendo escrito artigos sobre os temas da liberdade, do racismo, da
descolonização e da ``biopolítica'' --- conceito extraído da obra do
filósofo francês Michel Foucault, no qual Medeiros apoia grande parte de
seu pensamento. Além disso, Medeiros dedica parte de sua pesquisa ao
pensamento dos abolicionistas afro-brasileiros do século \textsc{xix} (como André
Rebouças, Luiz Gama e José do Patrocínio).

\textit{História da experiência das epidemias no Brasil} é um livro
derivado de sua tese de doutorado, e se destaca especialmente pela
riqueza de descrições e multiplicidade de fontes. Ora, Medeiros é também
escritor e poeta (autor de \textit{Mármore e Barbárie} e \textit{Zumbimalê
Pivete}), e isto dá a seu texto um caráter fronteiriço, entre literatura
e filosofia, narrativa histórica e experiência do pensamento. Para isso
mobiliza fontes literárias modernas e contemporâneas e rigor de pesquisa
arquivística.

Esta obra pode ser caracterizada como uma tentativa de extrair uma
legibilidade das epidemias e das práticas a elas associadas no Brasil ---
desde as mais antigas, de que se tem notícia ainda no período colonial,
na Bahia do século \textsc{xv}, até a mais recente pandemia do coronavírus e da
COVID-19. Isto é reforçado no posfácio, assinado por Medeiros e Victor
Galdino, dedicado exclusivamente a esta última. Realizado a partir de
uma leitura do período imperial tardio: as quatro últimas décadas do
século \textsc{xix}, às portas da Primeira República (e da Revolta da Vacina de
1904). Também desde o que acontecia na cidade imperial do Rio de
Janeiro, com seus cortiços e outras formas de coabitação que foram
historicamente marcadas, pelos discursos médico e policial, para
desaparecer, junto com o advento dos pobres e dos descendentes de
africanos como ``classes perigosas'' para as elites herdeiras da
colonização. Graças a esse recorte, conseguimos ver esta ampla narrativa
histórica das epidemias no Brasil por meio de uma narrativa cuidadosa e
rigorosa de um espaço-tempo preciso.

Por outro lado, a \textit{História da experiência das epidemias no Brasil}
também pode ser lida como uma história das relações de força que
constituem aquilo que chamamos de ``corpo'' em nosso país. Assim, é
também uma arqueologia, no sentido que o filósofo Michel Foucault dava a
esse conceito, a esse método filosófico-historiográfico, dos
\textit{regimes de verdade} que determinam a ``objetividade'' dos corpos e
a ``subjetividade'' que lhes é subjacente. Dessa forma, a leitura desta
obra possibilita também ao seu leitor uma reflexão mais profunda sobre a
relação entre os modos de se dizer um corpo e os modos de se compreender
o eu.

Nesse sentido, as duas possibilidades de leitura da obra de Medeiros
coincidem em uma reflexão sobre a reestruturação do corpo da cidade
(isto é, do governo e da política pública para o espaço urbano) a partir
de uma caracterização do corpo sob o signo da higiene --- uma categoria
que surge no discurso médico não apenas no âmbito da saúde biológica,
como também da moral. A passagem do século \textsc{xix} para o \textsc{xx} é também,
assim, um momento de consolidação dos discursos marginalizantes que irão
associar às ``classes perigosas'' o foco tanto das epidemias quanto dos
males sociais.

\subsection*{Corpo como microcosmo e corpo como
organismo}

Segundo Medeiros, ``um corpo é uma encruzilhada histórica de relações de
força.'' Isso significa que em sua obra a ideia de corpo, e a própria
materialidade social (a forma das cidades, por exemplo), é determinada
por relações de forças consolidadas historicamente. Quando se debruça
sobre a experiência das epidemias no Brasil, o autor propõe
conjuntamente a reconstituição dessas relações que determinam a
constituição dos corpos.

É interessante, nesse sentido, acompanhar a transformação da compreensão
médica pré-moderna do corpo como ``microcosmo'' para a do corpo da
ciência positiva como ``organismo''. Nem sempre os corpos foram
compreendidos como uma organização mecânica de órgãos. E, igualmente,
nem sempre a cura, a medicina, foi exercida segundo as mesmas bases.

Nos séculos \textsc{xvii} e \textsc{xviii}, por exemplo, a medicina imperial se fazia com
o domínio do Latim, da \textit{Física} de Aristóteles, do
\textit{Tetrabiblos} de Ptolomeu e dos trabalhos de Avicena, parte do
currículo na Universidade de Coimbra, junto à leitura e comentário dos
textos de Hipócrates e Galeno. Essa continuidade de uma medicina
hipocrática se dava segundo um entendimento do corpo como uma
imagem-semelhança do Universo. Hipócrates postulava que o corpo humano e
sua saúde eram determinados pelo clima, pela atmosfera, pelos astros.
Com a sua teoria das quatro substâncias do corpo, explicava, por
exemplo, a melancolia por um excesso de ``bile negra'', com a passagem
de Saturno sobre o céu terrestre. Daí o entendimento milenar do
melancólico como saturnino, como alguém sob o signo de Saturno.

Essa medicina hipocrática, exercida no início do Brasil Império, não
era, portanto, totalmente inimiga de outras práticas de cura animistas.
Na modernidade europeia, o corpo passa a ser entendido como mecanismo
desencantado (como nas descrições que Hobbes faz do coração como mola e
dos nervos como cordas, ou nas de Descartes, que compara o coração a um
relógio). Esse mecanismo desencantado, que passaremos a entender mais
como organismo que como microcosmo (ou seja, mais como tecnologia do que
como receptáculo da alma ou do espírito) irá compor o dispositivo
sanitarista que tentará banir as práticas de cura e cuidado não apenas
hipocráticas, como também dos erveiros e benzedeiros indígenas e
afrodescendentes. Aos poucos, o surgimento da ideia de higiene como
categoria moral serviria de amparo para a classificação racial e
positivista dos saberes afrodescendentes e ameríndios enquanto saberes
perigosos, pertencentes a classes tidas como anti-higiênicas, e que
deveriam, por extensão, serem expulsas (ou escondidas nas margens) do
espaço urbano.

\subsection*{Higiene e reformulação do espaço geográfico da
cidade}\label{higiene-e-reformulauxe7uxe3o-do-espauxe7o-geogruxe1fico-da-cidade}

Medeiros nos mostra em seu livro como a primeira epidemia de febre
amarela na Corte do Rio de Janeiro, entre 1849 e 1850, desembocou, junto
à importação dos discursos e saberes da ciência médica positiva, em uma
mudança na forma como o espaço público era pensado e organizado. Essa
epidemia causou, em uma população de 266.000 habitantes, 90.658
infectados e fez falecer mais de 4 mil vidas. A partir desse momento, os
higienistas não seriam apenas convocados em casos excepcionais, mas o
seu discurso passaria a integrar as decisões macropolíticas no que se
refere às questões de urbanismo e moradia.

O dispositivo médico-higienista, como o classifica Medeiros, passará,
aos poucos e ao longo das décadas, de um regime combativo a um regime
preventivo-combativo. Proliferará, num primeiro momento, técnicas de
quarentena, de emergência. Em seguida, o discurso médico passará a
reorganizar a repartição urbana dos espaços de isolamento e
quadriculamento da morte.

Por fim, integrará a completa reformulação, com o ponto de culminância
sendo a passagem para o século seguinte, desembocando nas reformas
urbanas de Pereira Passos no Rio de Janeiro, nas demolições e remoções
em massa de cortiços e habitações coletivas, e marginalização da
população afrodescendente no espaço público.

Com tudo isso, a pergunta fundamental que Medeiros tenta responder é:
como a condenação do estado sanatório da cidade resulta, nos discursos
de poder, na ideia de que as habitações coletivas são imorais e
antiestéticas? Ou seja, como o conceito de saúde se torna um dispositivo
do racismo no final e no período posterior à escravidão? Como tudo isso
acompanhou, ao mesmo tempo, a remodelação radical das cidades, com a
condenação dos hábitos ``coloniais'', e sobretudo a normatização
higienista dos corpos?

A resposta a essas perguntas ajuda a compreender a \textit{experiência}
das epidemias na história do Brasil que a obra de Medeiros explora. Essa
experiência compreende ao mesmo tempo a história da continuidade do
racismo na Primeira República e nos dias de hoje, bem como aquilo que
Medeiros classificará como regime de visibilidade das cidades, condução
contra o qual as populações em permanente estado de marginalização
social resistiram e resistem, com seus saberes-fazeres atacados pelas
instituições modernas, importadas para o Brasil no período analisado
pelo autor.

Por fim, é também por isso que Medeiros traz uma outra história a
contrapelo: a história dos fantasmas e espíritos, das ruas, das vielas,
dos cortiços, dos quilombos, das práticas dos corpos improdutivos e dos
curandeiros, que trazem uma outra forma de organizar o espaço, viver a
cidade e a rua, e de cuidar dos corpos.


\section{O gênero}\label{o-guxeanero}

Podemos, até o momento, compreender que o livro de Claudio Medeiros
transita entre os campos da filosofia e da história, que estão
devidamente demarcados no enredo de seu texto, conforme viemos tratando.
Porém, além disso, os campos da sociologia e da antropologia se
demonstram os ``panos de fundo'' que geram a base da pesquisa literária
e da narrativa textual que o autor desenvolve ao longo do livro para
engendrar debates reflexivos e concepções críticas acerca da história do
Brasil, especialmente no que tange à transição do período final da
monarquia à introdução da república no país.

Dentre as ciências sociológicas, a sociologia política é o subcampo que
mais se apresenta no presente livro. Responsável por analisar e
dissertar sobre os meios e modos de governança dos sistemas de
organização social e econômica dos Estados nas diferentes regiões do
mundo, em \textit{História da experiência das epidemias no Brasil} a
sociologia política apresenta-se no contexto brasileiro da Corte
Imperial, estabelecida na cidade do Rio de Janeiro.

Já com relação às ciências antropológicas, o subcampo da antropologia
cultural também se destaca no texto do autor. Reconhecida por almejar
compreender como se constitui a cultura dos diversos grupos humanos por
meio do estudo de seus costumes, religiões, línguas, etc., aqui a
antropologia cultural volta-se à pesquisa das comunidades africanas e
afrobrasileiras resistentes na capital do Brasil durante o seu período
monárquico.

Com estes temas em diálogo, o livro apresenta uma narrativa histórica
fundada nas memórias sociais presentes em arquivos e também nos
acontecimentos políticos relatados em livros e jornais da época. A
característica dissertativa do texto de Claudio Medeiros desvela uma
história desconhecida, mas importante, pois reconhece sua influência
significativa no processo de constituição da república brasileira ---
sistema de organização social e econômica que vigora até os dias atuais,
tendo passado por diversas mudanças e atualizações ao longo da história.

Compreende-se por narrativa o objeto de discussão que, entre filósofos e
historiadores, a partir da segunda metade do século \textsc{xx}, veio determinar
que o pensamento historiográfico, realizado com o estudo dos fenômenos
humanos e naturais do passado, mantém necessariamente uma lógica de
narração. Não há, entre os debates recentes, qualquer argumento que
dissimule o caráter narrativo do pensamento histórico.

É justamente a narrativa histórica que oferece as diferentes formas como
os educandos e educandas, juntos aos seus professores e professoras,
geram sentidos e até mesmo significados contemporâneos para determinadas
versões do passado. É através da narrativa que obtemos as possibilidades
de esclarecer ideias abstratas e fazer emergirem hipóteses sobre os
modos de vida e sobre os acontecimentos político-sociais de diferentes
organizações de uma sociedade do passado. Portanto, é a narrativa que
deflagra a história de um povo, por exemplo, mas também a história do
seu pensamento.

Sem a narrativa história de Claudio Medeiros, não seria fácil alcançar o
pensamento crítico sobre os modos como o Estado monárquico do período
lidou com os sujeitos doentes pela febre amarela ou pela varíola. Nem
mesmo seria compreensível reconhecer os movimentos de autocuidado e
resistência à marginalização que africanos e afrodescendentes
brasileiros vieram erguer frente às medidas sanitárias de contenção
destas epidemias. E, por fim, não poderíamos refletir sobre os casos
históricos sem reconhecer os meios e os modos do pensamento entre 1850 e
1889, ou seja, sem encontrar e discernir o contexto que forma a razão
daquele momento histórico.
