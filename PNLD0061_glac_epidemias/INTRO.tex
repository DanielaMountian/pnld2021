\chapter{Apresentação}

\begin{flushright}
\textsc{alexandre rosa}
\end{flushright}

\noindent{}A presente antologia reúne crônicas e contos de Lima Barreto. A grande
maioria desses textos foi publicada em jornais e revistas da época, que
posteriormente foram agrupados em livros e coletâneas. A seleção que a
leitora e o leitor têm em mãos foi dividida em duas partes --- Crônicas e
Contos.

A primeira parte, dedicada às crônicas, foi pensada de modo a reunir os
textos em pequenos conjuntos temáticos. Tais conjuntos vêm precedidos
por subtítulos. A ideia é apresentar o escritor de uma forma mais leve,
com textos curtos e linguagem fluída. Ao mesmo tempo, convidando a
leitora e o leitor a refletir sobre alguns dos principais temas que Lima
Barreto desenvolveu ao longo de sua vida, tanto na atividade
jornalística quanto em sua prosa de ficção, além de irem criando
intimidade com o estilo do autor.

Alguns temas que organizam essa primeira parte da Antologia são:
\textit{Autobiografia}; \textit{Cotidiano e vida nos subúrbios};
\textit{Futebol}; \textit{Nossa Política e nossos políticos};
\textit{Violência contra as mulheres} e \textit{Humor}.

A segunda parte da Antologia reúne o que de mais significativo Lima
Barreto escreveu sob o gênero conto. São deste escritor alguns dos
principais textos escritos, no Brasil, dentro desta modalidade de
narrativa curta: ``O homem que sabia javanês'', ``A nova Califórnia'' e
``Como o `homem' chegou'' podem ser considerados como grandes
realizações do conto brasileiro.

Nessa parte do livro, leitoras e leitores entrarão em contato com a
produção ficcional propriamente dita de Lima Barreto; algo para o qual
já vinham sendo preparados com a leitura das crônicas e, principalmente,
com o conjunto de textos selecionados na categoria \emph{Humor}.

Os contos não precisam ser lidos necessariamente na ordem em que se
encontram. Cada um concentra em si um universo particular daquilo que se
convencionou chamar de problemas e dilemas da condição humana; e podemos
dizer \emph{da condição humana em geral e brasileira em particular.}

O livro gerou um número grande de notas de rodapé, principalmente para a
contextualização de nomes de pessoas, localização geográfica,
informações sobre obras literárias e outras obras artísticas, termos
antigos ou específicos que não são mais utilizados hoje em dia, etc.
Mesmo assim, as e os estudantes encontrarão algumas palavras que vão
exigir uma consulta ao dicionário --- físico ou virtual. Essa é também
uma maneira de se habituar à leitura de textos mais antigos e
contemporâneos, para uma melhor compreensão da totalidade daquilo que
vai ser lido.
