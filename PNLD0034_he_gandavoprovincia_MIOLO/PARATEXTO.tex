\part{\textsc{paratexto}}

\chapter{Gandavo e a retórica histórica}
\hedramarkboth{Gandavo e a retórica histórica}{Clara C.~Santos e Ricardo M.~Valle}

\begin{flushright}
\textsc{clara c.~santos\\ricardo m.~valle}
\end{flushright}

\section{Sobre o autor}

\noindent{}Falando com rigor, nada efetivamente se sabe a respeito de Gandavo,
além de que possivelmente tenha escrito o livro (mais uma ortografia e
alguns outros manuscritos), e de que o tenha feito assinando com este
nome, Pero de Magalhães de Gandavo. Estamos certos, aliás, de que
sabemos até menos do que isso. A designação ``de Gandavo'' 
aparentemente não foi herdada como sobrenome, mas incluído pela pessoa 
do autor e pode ser que para constituir tradição familiar, com vistas a alguma 
fidalguia. Daí se poderia supor também que fosse um imigrado em Portugal 
e que a si passasse a designar pelo patronímico, em verdade ilustríssimo naquele tempo, 
porque sabemos que Gandavo também era Carlos \textsc{v}, nascido em Guantes, 
Flandres, cidade distinguida justamente por esse evento de enorme 
significação para o catolicismo europeu.

Conhecendo os trâmites político"-institucionais a que estava submetido na
hierarquia e valendo"-se dos verossímeis da invenção no gênero
histórico, Barbosa Machado inventou em meados do século 	\textsc{xviii} elementos
da vida do autor. E a recepção moderna tantas vezes apenas acreditou
como efetividade, deixando de lado as possibilidades de pensar, não o
fato particular, que pouco importa, mas as implicações institucionais
supostas a esse quase nada de que se tem notícia, um nome e um
livro.\footnote{ Ver as questões e categorias discutidas por João Adolfo
Hansen em ``Um nome por fazer'', acerca de Gregório de Matos. In: \textit{A sátira e o Engenho}. 
São Paulo/Campinas: Ateliê/Unicamp, 2004, e em “Autor”. In: \textit{José Luís Jobim}. Palavras da
crítica. Rio de Janeiro: Imago, 1992.} Contudo, desde a publicação
francesa, que é do início do século \textsc{xix}, tem"-se reiterado a mesma
informação biográfica, posteriormente acrescentada, ainda que sempre de
poucas notícias.

É verdade que se poderia inventar o verossímil mentiroso de um Gandavo
anônimo, de família portuguesa com ascendência marrana, cristãos"-novos
fugidos da Península Ibérica no tempo de Dona Isabel e Dom Fernando, e
das consequências inclusive jurídicas da ação política dos piedosos e
violentos \textit{reyes católicos de España}, pais de Carlos \textsc{v}. A família
marrana, em processo de limpeza de sangue, serve o Império Católico em
Flandres no tempo do imperador, numa sabidamente lenta ascensão
estamental, por acumulação de dignidades em ofícios letrados. Em 1568,
com a revolução de Orange nos Países Baixos --- com a revisão dos pactos,
as reformas institucionais e as alterações no direito e costume ---, a
família portuguesa de origem judaica, e letrada em âmbito católico, sai
dali para o Brasil, procurando postos subalternos para homens com
algumas letras. Na província portuguesa, Pero de Magalhães (ou como
quer que se tenha chamado) ocupa alguma função anônima na província
portuguesa desta costa do Brasil, onde enriquece e forja documentos
para abreviar a carreira na volta à Europa. Em Lisboa, muito rico e
provavelmente ignorante, ou ao menos cheio de maus acentos no uso da
língua, torna"-se um adulador para conseguir nomeação de escrivão, ou
cronista na Torre do Tombo, seguindo talvez tradição paterna. Paga um
poeta ilustre, um varão de armas sem dinheiro, um impressor e um
historiador para forjar uma obra que lhe conferisse a autoridade de
historiador que lhe ajudaria a receber o cargo tornando"-se mais próximo
de um título de fidalguia. Descoberta a fraude, por irregularidades com
a licença do Paço, desenrola"-se o que é presumível. E com isso, a
memória de seu nome e de seu livro praticamente desaparecem, até que no
tempo da Academia Real de História o livro não fosse reconhecido como
irregular pelas instâncias de chancelaria, passando a ser mencionado,
mas pouco; até que, na Biblioteca  Lusitana, Diogo Barbosa Machado
cumprisse o lugar de bibliógrafo redigindo uma linha de sua vida, aqui
entendida como a espécie do gênero histórico cuja \textit{auctoritas} é
principalmente Plutarco. Com melhor engenho, seria possível inventar
outros particulares verossímeis para a vida de Gandavo, não sendo outra
coisa porém do que a aplicação dos procedimentos da arte de Diogo
Barbosa Machado, cuja notícia a crítica histórica raras vezes deixou de
considerar como o ``pouco que se sabe''.

Pero de Magalhães de Gandavo  tornou"-se um nome tão obscuro quanto o seu
livro. Desde a \textit{Bibliotheca portuguesa} de Barbosa Machado, que é de
meados do século \textsc{xviii}, diz"-se que fora natural de Braga, que o pai era
flamengo, que foi moço"-de"-câmara de Dom Sebastião, que trabalhou na
Torre do Tombo como copista, que permaneceu alguns anos no Brasil cuja
história escreveu, e que, após a publicação do livro, é nomeado
provedor da fazenda da cidade de Salvador, na Bahia, cargo que, diz"-se,
não exerceu. Versado pelo menos nas artes do \textit{trivium} e autor de uma
ortografia portuguesa, teria aberto uma escola na província, na região
entre o Douro e o Minho, onde também casara. A maior parte das ações e
funções institucionais que se lhe atribuíram, contudo, foram atos ou
ofícios que presumivelmente um homem de letras tinha dignidade para
exercer. São provavelmente verossímeis narrativos do gênero histórico,
inventados por tradições biblio e historiográficas de escrita de \textit{vit\ae}
de poetas; ou são notícias derivadas desses verossímeis em vertentes
historiográficas do século \textsc{xix}. O mistério que ronda o desaparecimento
de seu livro, aplica"-se a seu estranho sobrenome, que, sem ascendência
nem descendência certas, não se sabe hoje sequer a sílaba sobre a qual recai o acento.



\section{Sobre a obra}

\noindent{}Impressa em Lisboa, em 1576, na oficina de Antonio Gonçalves, a edição
da \textit{História da província Santa Cruz a que vulgarmente chamamos Brasil
feita por Pero de Magalhães de Gandavo}\footnote{ Nesta edição
seguimos o exemplar pertencente à Biblioteca Nacional de Lisboa da
\textit{História da província Santa Cruz a que vulgarmente chamamos Brasil
feita por Pero de Magalhães de Gandavo}. Lisboa: Officina de Antonio
Gouvea, 1576. Infelizmente a Biblioteca Nacional brasileira não tem
programa de digitalização fotográfica do acervo de obras raras,
dificultando as possibilidades de cotejo, para resolver problemas como
as aparentes irregularidades na página de Aprovação.} é um tratado da terra, isto
é, um laudo documental dos domínios do soberano, com tudo o que neles
houvesse, oferecido neste caso a um vassalo do Rei de Portugal como
louvor dos domínios do mesmo Rei. Pensado até aí, o documento parece
perfeitamente conformado no interior das instituições e regulamentos
institucionais a que então um impresso tinha de submeter"-se. Contudo, o
livro parece ter saído de circulação e o nome do autor praticamente
desaparece por quase dois séculos, principalmente em âmbito português.
A obscuridade do livro nos séculos seguintes à sua publicação é tanto
mais estranha se se tem em vista que, por intermédio de uma elegia e um
soneto de Camões, o livro é dedicado a um varão de armas em carreira
promissora nas Índias portuguesas, tendo sido impresso pela mesma
oficina impressora d'\textit{Os Lusíadas} (1572), que obtivera
alvará de Privilégio real para sair, apenas quatro anos antes.

%\subsection{o poder constituído das mesas censórias}

Na segunda metade do século \textsc{xvi}, as aprovações do Ordinário, da
Inquisição e do Paço atestavam a verdade da instrução que o livro
continha, segundo as Leis católicas e do Império português, que em
grande medida são extensão umas das outras. Sejam livros úteis, como os
de Lei e os Sagrados, sejam os de recreação, como os de história ou
poesia,\footnote{ Empregamos aqui a classificação que se lê nas páginas
iniciais do \textit{Corte na aldeia} (1619), de Rodrigues Lobo, que inclui como
livros de recreação, de um lado, os de histórias fingidas, compreendendo
as cavalarias de \textit{res ficta}, isto é, de matérias fingidas, análogos neste
sentido de muitos gêneros da poesia, e de outro lado os de história
verdadeira, como podem ser pensadas as crônicas, décadas, tratados da
terra, etc. (Cf.~\textit{Obras políticas morais e métricas do insigne Portugues
Francisco Rodrigues Lobo. Natural da Cidade de Leyria. Nesta última
impressão novamente correcta, e postas por ordem. Offerecidas à
Magestade sempre augusta do Sereníssimo Rey de Portugal. D. João \textsc{v},
nosso senhor}. Lisboa Oriental: Na Officina Ferreyriana. 1723;
pp.~3--4.)} somente a concórdia entre as mesas conferia a completa
autorização para a impressão e circulação do livro, cruzando"-o com a
tradição das autoridades e axiomas de doutrina ou instrução verdadeira;
entendendo por tradição aqui o sentido com que o Concílio de Trento
distiguia boas e más tradições discursivas, tendo em vista que 

\begin{hedraquote}
esta verdade, e disciplina se contém em livros escritos, e sem escritos,
nas Tradições, que recebidas pelos Apóstolos da boca de Cristo, ou
ditadas pelo Espírito Santo, dos mesmos Apóstolos como de mão em mão
chegaram até nós; seguindo o exemplo dos Padres Ortodoxos [\ldots{}]; como
ditadas pela boca de Cristo, ou pelo Espírito Santo, e por uma
contínua sucessão, conservadas na Igreja Católica [\ldots{}].\footnote{ Sessão 
\textsc{iv}, do Concílio de Trento. Citamos pela edição bilíngue do
século 	\textsc{xviii}: \textit{O sacrosanto, e ecumenico Concilio de Trento Em Latim, e
Portuguez}. Trad. João Baptista Reycend. Lisboa: Na officina Patriarcal
de Luiz Ameno, 1781, Tomo \textsc{i}, p.~55.}
\end{hedraquote}

Como ficou dito, os destinatários das duas versões manuscritas (a
primeira delas apenas hipotética) são as duas mais poderosas
representações políticas daquelas décadas. Um é o jovem rei, que tem
representação primordialmente jurídico"-militar --- e, no caso particular
de Dom Sebastião, sobretudo militar e piedosa, segundo a invenção da
fama do nome na instituição histórica, cujo fim, entre as funções
bibliotecárias do reino desde antes de Fernão Lopes, era conferir
legitimidade à autoridade das potestades instituídas no presente.
Independentemente da feição particular da representação tipológica que
assume, o rei devia ser reconhecido como autoridade que reúne a
soberania sobre a totalidade das potestades, ofícios, artes e demais
serviços do reino, uma vez que seja monarca. Já o destinatário do
\textit{Tratado da terra do Brasil} que efetivamente nós podemos ler é o tio do
mesmo monarca, o velho Inquisidor"-mor, Cardeal dominicano, Dom
Henriques, que tinha autoridade direta sobre as três instâncias de
censura: do Ordinário, da Inquisição e do Paço. A mesa do Ordinário dá
vista e aprovação, ou a nega, normalmente pela autoridade de um doutor
em leis ou teologia escolhido entre os bispos da diocese para a função
\textit{ad hoc} de vedor ou qualificador. Como junta ou conselho episcopal do
reino, os membros do Ordinário estão necessariamente sob a autoridade
do Cardeal"-Infante, ainda mais quando o príncipe"-sacerdote é o próprio
Inquisidor do Paço. Além disso, sabemos que mesmo o Conselho Geral da
Inquisição havia sido instalado em Portugal sob a ação do mesmo
Cardeal"-Infante, feito Inquisidor"-mor por seu irmão, Dom João \textsc{iii},
décadas antes. 

Segundo o Concílio de Trento, os livros de matérias sagradas e, por
necessidade, toda matéria sagrada contida em livros de qualquer
espécie, deveriam ser conferidos, sob pena, com o ``verdadeiro sentido, 
e interpretação das Escrituras'', isto é, ``o unânime consenso
dos padres''. Assim, decreta que 

\begin{hedraquote}
a ninguém é lícito imprimir, nem mandar imprimir Livros alguns de
matérias sagradas sem nome do Autor, nem vendê"-los daqui em diante;
nem também tê"-los em seu poder, sem serem primeiro examinados, e
aprovados pelo Ordinário, sob pena de excomunhão [\ldots{}]. E se forem
Regulares (os Autores) além deste exame, e aprovação, estarão
obrigados a impetrar também Licença dos seus Superiores, sendo por
eles examinados os livros, na forma das suas Constituições.\footnote{ Idem, ibidem. pp.~61--63.}
\end{hedraquote}

Além de valer para qualquer documento manuscrito, as penas se estendiam
aos impressores, vendedores, possuidores e leitores da obra que
ilegalmente fosse posta em circulação. A censura do Ordinário em
Portugal, como na Espanha, era já exercida antes do Concílio --- e como
mostra a própria folha de ``Aprovação'' desta
\emph{História da província Santa Cruz} ---, a licença do Ordinário
subordinava"-se à petição do Conselho Geral da Inquisição. E assim como
os clérigos regulares deveriam ter ainda a permissão dos superiores da
Ordem a que pertencessem, os súditos deveriam ter a aprovação de seu
rei, o que em Portugal se fazia pela censura do Paço.

No livro de Gandavo em particular, a Licença do Conselho Geral do Santo
Ofício da Inquisição acredita o testemunho do vedor de livros do
Ordinário; tudo feito no interior de rígidas praxes, instituídas e
acostumadas por décadas. A aprovação do Santo Ofício apenas ratifica,
aparentemente sem exame, a informação de que o livro não continha coisa
que se pudesse considerar inimiga da santa fé e do bom costume
cristãos, não colocando em suspeição nem os mistérios nem os usos da
fé, sempre segundo o costume apostólico romano e terminantemente
revogadas pelo Concílio quaisquer tradições de opinião contrária ou
fundadas na própria prudência e livre inteligência das coisas sagradas.
Segundo o Decreto aposto à mesma Sessão do Concílio:
\begin{hedraquote}
para refrear engenhos petulantes, determina: que ninguém confiado na sua
prudência, em matérias de Fé, e costumes, e edificação da Doutrina
Cristã, torça a sagrada Escritura para os seus conceitos particulares,
contra aquele sentido que abraçou, e abraça a Santa Madre Igreja, a
quem pertence julgar o verdadeiro sentido e interpretação das
Escrituras; nem se atreva a interpretar a mesma Escritura contra o
unânime consenso dos Padres; ainda que estas interpretações nunca hajam
de se dar a luz. Os que a isto contravierem, sejam pelos Ordinários
declarados, e castigados com as penas estabelecidas em
direito.\footnote{ Idem, ibidem. pp.~59--61.}
\end{hedraquote}

O livro de Gandavo aparentemente não se choca com o Decreto, mas não
deixa de ser notável que, no exemplar que transcrevemos, pertencente à
Biblioteca Nacional de Lisboa, que o publica digitalmente, a caixa de
texto da licença do Paço apresenta recuo maior de ambos os lados e o
alinhamento horizontal é ligeiramente pendente à esquerda, o que pode
indicar que a inserção foi feita posteriormente. A hipótese talvez
fosse excessiva, não houvesse notícia pela Biblioteca John Carter Brown
de que há exemplares com e sem a terceira caixa de texto.\footnote{ A
informação é de Rubens Borba de Moraes, que compara os exemplares da
Biblioteca \textsc{jcb} com o exemplar da Biblioteca Nacional do Rio de Janeiro.
Não pudemos, contudo, auferir a autenticidade da informação (Cf.
\textit{Bibliografia brasileira do período colonial}. São Paulo: Instituto de
Estudos Brasileiros, \textsc{usp}, 1969). A pesquisadora Valéria Gauz, que
trabalhou tanto na Biblioteca Nacional quanto na John Carter Brown,
menciona esse dado no texto ``Materialidade de livros -- \textsc{ii}'', disponível no site da \textsc{info}\textit{home}. 
O acervo da Biblioteca Nacional brasileira, contudo, não é
disponibilizado digitalmente.} Tudo isso, que é nada, na mera hipótese
sobre o particular, apenas permite insinuar problemas na tese que
explica diretamente o desaparecimento do livro por violação de segredo
de Estado, ainda que a hipótese continue válida. É mesmo muito
plausível que o desaparecimento do livro tenha sido causado por razão
de Estado, mesmo porque o acaso raras vezes é tão violento quanto as
instituições normativas. Esses particulares conhecíveis não teriam
maior interesse, se o manuscrito conservado não fosse dedicado
justamente à altíssima autoridade censória do reino, ao mesmo tempo que
sua versão impressa, recomendada por Camões, parece ter sido destruída
por força estatal, do contrário não teria sido elidido na quase
totalidade de seus exemplares. À aparente irregularidade da página de
``Aprovação'', se se confirmar que há volumes
sem a terceira licença, que é a do Paço, podem"-se fazer suposições
inúmeras que talvez tenham \mbox{interesse se} forem pensadas a partir dos
mecanismos institucionais e discursivos que constituem historicamente
essas relações de poder.

%\section{aos empobrecidos do reino}

Ainda na Dedicatória do \textit{Tratado da terra}, Pero de Magalhães, de
Gandavo, afeta modéstia, como tem de ser, para produzir retoricamente
benevolência no destinatário; para isso, elogia a utilidade da matéria
do seu tratado, com o que se quer demonstrar também a utilidade do
serviço do vassalo fiel em ofícios de letras:

\begin{hedraquote}
achei que não se podia dum fraco homem esperar maior serviço (ainda que
tal não pareça) que lançar mão desta informação da terra do Brasil
%Jorge: empreendeu
(cousa que até agora não empreendeu pessoa alguma) para que nestes Reinos
se divulgue sua fertilidade e provoque a muitas pessoas pobres que se
vão viver a esta província, que nisso consiste a felicidade e aumento dela.
\end{hedraquote}

Duas vezes repetida nesta carta dedicatória ao Cardeal"-Infante, bem como
no ``Prólogo ao Leitor'' da \textit{História da província Santa Cruz}, 
a recomendação da costa do Brasil para os
portugueses que estejam em pobreza na pátria é também empregada no
exórdio do livro impresso, sempre como argumento da utilidade do
sumário da terra, bem como dos serviços de quem o escreve, exercendo,
pois, dupla função, ao mesmo tempo retórica e política. Deveríamos
talvez supor que, sendo no mínimo gente de letras os seus leitores, a
condição de ``pobres e desamparados'' não designa o que as sinuosas 
classificações econômicas reconheceriam como
\textsc{c, d ou e}, nos sistemas de regulação por poder aquisitivo. Num corpo
político constituído por estados, ou ordens civis, os empobrecidos eram
mais provavelmente gente de baixa fidalguia, militares e letrados em
geral, cristãos"-velhos de ofício livre, possivelmente até
cristãos"-novos em processo de limpeza de sangue e lenta ascensão na
fidelidade institucional. A plebe mais baixa até estava implicada nesta
solução que o encarecimento da matéria do livro oferecia para a pobreza
dos pobres, mas estava suposta apenas como criadagem e companhia de
gente mediana, ou clientela de gente semiarruinada, ou ainda oficiais
mecânicos, mercenários, negociantes, exilados ou bandidos fugidos,
todos os pobres que por suposto também viriam, mas aos quais o livro de
Gandavo dificilmente se destinava, a não ser por exceção.

Tendo em vista os mecanismos da imensa hierarquia do império português,
o autor supõe demonstrar a utilidade de seu discurso colocando"-o a
serviço do necessário encorajamento às carreiras militares, jurídicas,
fiscais etc., na costa do Brasil, ainda designada província Santa
Cruz. Dirige"-se, portanto, à gente de carreira, isto é, gente
minimamente remediada para as instituições civis, educadas para o
preenchimento de funções direta ou indiretamente administrativas na
povoação e defesa dos novos domínios do reino, conforme o caso, mas
acometida por toda sorte de infortúnio que não fosse causado por crime.
Em outro uso, Antonio de Guevara, grande autoridade letrada do Império
Habsburgo no século \textsc{xvi}, avisava dos perigos da corte se dirigindo a
homens dignos de alguma distinção mas também empobrecidos por má
fortuna, recomendando os benefícios da recolha à província.\footnote{ Antonio 
de Guevara. \textit{Menosprecio de corte y alabanza de aldea} (1539).
Edición y notas de M. Martínez de Burgos. Madrid: Espasa"-Calpe, 1942.}
As diferenças entre os argumentos do elogio da aldeia, de Guevara, e da
recomendação feita nos exórdios dos breves tratados históricos da
terra, de Gandavo, são proporcionais aos decoros específicos, próprios
ao assunto e finalidade de cada livro. Em ambos os casos (e muitos
outros poderiam ser referidos), recomenda"-se o afastamento da corte
como um caminho difícil mas seguro para o exercício das boas virtudes
cristãs ou dos úteis serviços da fidelidade monárquica, com promessas
de prêmios nesta e na outra vida. Preenchendo o lugar retórico da
captação da benevolência --- no caso, pela defesa  e encarecimento da
matéria tratada ---, a recomendação de Gandavo, ainda que tenha
significação política, não precisaria ser lida com mais graves
consequências, ao menos não para que se pensem hipóteses sobre a origem
da pobreza do Brasil. A constituição política que está suposta nesta
exortação às carreiras do Novo Mundo ``dá sentido'', e talvez maior interesse, 
à descrição dos mantimentos da terra, bem como da barbárie dos nativos. 
Mas o sentido só ``é dado'' a nós que lemos o texto hoje, com outros hábitos de leitura. 
No seu tempo, o sentido já estava dado, salvo dissensos, que sempre existiram 
apesar dos sistemas de controle dos atos discursivos. 

``Primeiramente tratarei da planta e raiz de que os
moradores fazem seus mantimentos que lá comem em lugar de
pão.'' Assim começa o passo em que tratará da mandioca, de
que se extrai a farinha cuja técnica de uso Gandavo relata no
``Capítulo \textsc{v} --- Das plantas, mantimentos, e frutas que há
nesta província''. A mandioca é o análogo do trigo,
aproximado da raiz por sua finalidade, segundo modos de classificação
reconhecidos pelo português. Conforme a redação do \textit{Tratado} manuscrito:
\begin{hedraquote}
Nestas partes do Brasil não semeiam trigo nem se dá outro mantimento algum
deste Reino; o que lá se come em lugar de pão é farinha de pão: esta
se faz da raiz duma pranta que se chama mandioca, a qual é como inhame.
\end{hedraquote}

Como se dispõe a falar ``principalmente daquelas [plantas,
frutas e ervas], de cuja virtude e fruto participam os
Portugueses'' [Capítulo \textsc{v}], sempre segundo preceituam os
antigos retores do gênero histórico, a esta preeminência está suposto
que de trigo se faz pão, que é base da constituição alimentar do corpo
físico do vassalo cristão e base material para a consagração do Corpo
espiritual da comunidade de Cristo na eucaristia. Não por acaso a
mandioca, seu análogo, é o primeiro mantimento referido tanto na
\textit{História da província}, quanto no manuscrito do \textit{Tratado} da terra. Por
essa razão, também nas crônicas, tratados e histórias desta mesma
terra, escritos em português posteriormente, a mesma planta e seus usos
terão similar descrição e preeminência, sendo muitas vezes reiterada
sua analogia com o trigo, mesmo que o texto de Gandavo não tenha se
tornado fonte para a maior parte delas, devido a seu desaparecimento. O
caso da mandioca evidencia, portanto, que as descrições da fauna,
flora, costumes etc., eram redigidas segundo procedimentos
convencionais, orientados retórica, política e teologicamente.

O discurso de aconselhamento aos empobrecidos da pátria demonstra a
utilidade da matéria inventada, e por isso mesmo ocupa lugar
preeminente na escolha das matérias particulares de que tratará --- como
a abundância de mantimento, de água, de terra, de caça etc., e a
promessa de muitas pedras e metais ---, e que hoje se leem como elenco de
curiosidades especiosas da terra, por uma espécie de contaminação
turística de nossos tempos tristes, como outrora havia sido argumento
de ``porque me ufano'' de ser brasileiro, nas leituras românticas e
modernistas do texto, muitas vezes associadas institucionalmente a
jubileus comemorativos, como os cem anos da Independência, e aos
eventos bibliográficos, ou já editoriais, a eles ligados, direta ou
indiretamente. O aconselhamento aos empobrecidos não se contradiz pela
horrenda descrição dos índios e seus perigos, que se leem nas
terríveis, e exemplares, cenas de antropofagia que Gandavo produz na
última terça parte do livro. A exortação à carreira no ultramar não
seria de modo verossímil promissora em demonstração de virtude e heroísmo,
não houvesse obstáculos que a fé e a obediência do súdito deveriam
vencer para efetivamente melhorar de vida, isto é, adquirir dignidade
no interior da efetividade da instituição armada do Estado, e suas
demandas.


\section{Sobre o gênero}

O texto de Gandavo, em um primeiro momento, pode ser classificado no gênero do relato histórico que, em linhas fundamentais, define"-se como a narrativa em que um sujeito, inscrito em um determinado tempo histórico, debruça"-se sobre fatos, descrições e interpretações desse momento histórico no qual vive. Para o historiador francês Paul Veyne, o relato histórico é parcial e subjetivo, pois não consegue apreender a globalidade dos acontecimentos, apenas
aquilo que está ao alcance do narrador e, mesmo isso, não de uma forma pura, mas filtrado pela sua subjetividade e pelos objetivos de seu relato.
Para Marc Bloch, em uma linha parecida, o relato é apenas um ``vestígio'' da história, um pequeno pedaço do factual que, pela pena de um narrador, pôde"-se cristalizar no tempo e ser transmitido a gerações posteriores, sendo apenas uma das infinitas possibilidades de apreensão e compreensão de determinados fenômenos:

\begin{quote}
Quer se trate das ossadas
emparedadas nas muralhas da Síria, de uma palavra cuja forma ou emprego revele um
costume, de um relato escrito pela testemunha de uma cena antiga [ou recente], o que
entendemos efetivamente por documentos senão um ``vestígio'' quer dizer, a marca,
perceptível aos sentidos, deixada por um fenômeno em si mesmo impossível de captar?\footnote{\textsc{bloch}, Marc. \textit{Apologia da história}. Rio de Janeiro: Zahar, 2002, p.\,73.}
\end{quote}

Além de ``relato de viajantes'',  desde o século \textsc{xix} essa \textit{História da província Santa Cruz} foi lida como
, ``literatura de informação'' ou como \textit{Nossa
primeira história}, título que recebeu na edição de 1921--1922, entendido
como testemunho de impressões antigas dos portugueses nas terras
d'além"-mar. Contudo, esta simples história, ou tratado
descritivo, da costa do Brasil teve uma circulação muito restrita no
seu século, fazendo parecer que o livro foi recolhido após sua
impressão, não se sabe precisamente por quê. Segundo uma linha de
interpretação da recepção da \textit{História da província Santa Cruz},
sustenta"-se que o livro teria sido recolhido por revelar segredos de
Estado sobre a província portuguesa, como a posição de rios e cidades
da costa do Brasil, segundo o que os historiadores chamaram de
``política do segredo'' de Dom Manuel \textsc{i}.\footnote{ Cf.~Sheila 
Moura Hue e Ronaldo Menegaz.
``Introdução''. In:  \textit{Primeira História do
Brasil. História da província Santa Cruz a que vulgarmente chamamos
Brasil}. Rio de Janeiro: Jorge Zahar, 2004, p.~14.} O certo é que
permaneceu praticamente ignorado até a primeira metade do século \textsc{xix},
quando foi reconsiderado na edição e tradução de M. Henri Ternaux, em
Paris, em 1837; e no século seguinte foi ainda vertido para o inglês
por John B. Stetson Jr., em 1969. No início do século \textsc{xviii}, com o
fomento à divulgação das navegações e feitos portugueses pelas
Academias de História no reinado de Dom João \textsc{v}, o opúsculo de Gandavo
consta dos documentos então exumados pela Real Academia, o que se prova
com a ocorrência secundária no Bluteau e logo na Biblioteca de Diogo
Barbosa Machado, em meados do século \textsc{xviii}, fazendo a essa regra raras
exceções, sendo mantido ainda em ampla obscuridade até a edição
francesa na primeira metade do século \textsc{xix}.

Diferente de um testemunho empírico, o livro é composto como gênero
histórico, retoricamente regrado, em que o historiador, apoiado pelo
aconselhamento ético da Igreja Católica, tem por últimos fins exaltar,
pelo discurso, ações virtuosas de pessoas de caráter elevado e eventos
providenciais; levar adiante a fama do monarca e dos homens de armas a
quem é dedicado; e legitimar sua autoridade e senhorio, com direito de
propriedade, sobre terras, rios, espécies animais e vegetais, pedras,
metais etc., que se podem usar e dispor como próprios, conforme aos
grandes axiomas e aos anátemas que regulavam estas disposições na forma
das leis civis e eclesiásticas. Estes fins deveriam atingir"-se pela
aplicação de procedimentos e lugares retóricos, entre os quais a
amplificação da beleza, utilidade, fertilidade, abundância etc., como
louvor dos novos domínios da Cristandade portuguesa. Neste sentido,
relata, ou historia, as particularidades da terra e de sua conquista
como louvor do feito português, visando à perpetuação da empresa
marítima lusitana a partir de uma narrativa que segue principalmente os
modelos preceptivos de Menandro, retor, e Plínio, o Velho, entre outras
autoridades de escrita histórica.


