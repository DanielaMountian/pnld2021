


\textbf{Feitiço de amor e outros contos} reúne seis das narrativas macabras ou fantásticas de 
\textit{Phantasus} (1812--16), coletânea concebida em três partes, 
cada uma com sete textos. Os textos mais antigos e a ideia de integrá"-los 
numa composição intermediada por diálogos à maneira da série de contos 
\textit{Decameron} remontam à última década do século \textsc{xviii}, 
ou seja, do início do período romântico. A complementação, 
através de intervenções e de novos textos se deu por volta 
de 1810. Num poema apresentado ainda no começo do livro, Fântaso, 
deus grego dos sonhos com seres inanimados, é o guia que inicia o 
poeta pelas manifestações assustadoras da natureza no universo: 
o medo, a tolice, o gracejo, o amor. O leitor tem em mãos a primeira 
edição brasileira dedicada à ficção de Ludwig Tieck.   
        
\textbf{Karin Volobuef} é professora de Literatura na Faculdade de Ciências e
Letras da Unesp (Araraquara). Graduou"-se em Letras pela Unicamp, obteve
o mestrado na \textsc{usp} em 1991 com dissertação sobre E.T.A.~Hoffmann e doutorou"-se
na mesma universidade em 1996 com a tese sobre a ficção romântica alemã 
e brasileira \textit{Frestas e arestas: A prosa de ficção do Romantismo 
na Alemanha e no Brasil} (Unesp, 1999). Traduziu, entre outros, 
Fouqué, Arnim, Hoffmann, Tieck e Novalis, e publicou estudos sobre Kafka, 
Schiller, Chamisso, Canetti e Hofmannsthal. Para a Coleção de Bolso Hedra, 
reviu sua tradução de \textit{O pequeno Zacarias chamado Cinábrio}, de E.T.A.~Hoffmann. 

\textbf{Maria Aparecida Barbosa} é doutora em Literatura pela Universidade
Federal de Santa Catarina (\textsc{ufsc}) com tese sobre romantismo alemão e tradução 
literária. Como docente da mesma instituição, desenvolve pesquisas sobre Rainer 
Maria Rilke, E.T.A.~Hoffmann, Peter Weiss, Alfred Kubin, Kurt Schwitters. 
Traduziu, entre outros trabalhos, \textit{Cartas natalinas à mãe}, de Rilke, (Globo, 2007) 
e textos filosóficos de August W.~von Schlegel, W.~von Humboldt e Ernst Bloch.


