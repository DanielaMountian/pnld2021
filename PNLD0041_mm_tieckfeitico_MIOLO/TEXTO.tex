\chapter{Apresentação}


Primeira edição brasileira dedicada à ficção de Ludwig Tieck, escritor chave do Romantismo alemão, \textit{Feitiço de amor e outros contos} reúne seis das narrativas macabras ou fantásticas de  \textit{Phantasus}, coletânea publicada entre 1812 e 1816, que conquistou o sucesso junto ao público.
Concebida inicialmente com uma divisão em três partes, 
cada uma com sete textos, os textos mais antigos de \textit{Phantasus} e a ideia de integrá"-los numa composição intermediada por diálogos à maneira da série de contos 
\textit{Decameron} remontam à última década do século \textsc{xviii}, 
ou seja, do início do período romântico. A complementação, 
através de intervenções e de novos textos se deu por volta 
de 1810. 

Num poema apresentado ainda no começo do livro, Fântaso, 
deus grego dos sonhos com seres inanimados, é o guia que inicia o 
poeta pelas manifestações assustadoras da natureza no universo: 
o medo, a tolice, o gracejo, o amor. 
E foram através desses temas, representativos da concepção estética e literária de \textit{Phantasus}, que as organizadoras se orientaram para fazer essa seleção de seis narrativas curtas de Tieck.

Além de terem sido concebidas para publicação em uma antologia, essas narrativas relacionam"-se entre si ao trafegar pelo mesmo universo, marcado pela valorização da cultura oral camponesa, pelo macabro (também advindo da tradição popular medieval) e pela fantasia que marcou parte do movimento romântico.
Configura"-se assim não só um recorte preciso que apresenta o melhor da ficção de Tieck, mas também uma introdução ao romantismo alemão e alguns traços estéticos que orientaram essa escola literária.

\part[feitiço de amor e outros contos]{\textsc{feitiço de amor\break e outros contos} }


\chapter{O Loiro Eckbert}


\textsc{Em uma região} da Hercínia morava um cavaleiro que todos chamavam apenas
de Loiro Eckbert. Ele contava cerca de quarenta anos, mal alcançava
estatura mediana, e seus cabelos louros claros caíam curtos e lisos bem
rente ao semblante pálido e descarnado. Levava uma vida pacata e
reservada, e jamais se envolvia nas contendas de seus vizinhos; além
disso, só muito raramente era visto fora dos muros de circunvalação de
seu pequeno castelo. Sua esposa apreciava igualmente a solidão, e ambos
pareciam amar"-se do fundo de seus corações, sendo usual queixarem"-se
apenas do fato de que o céu se recusava a abençoar seu casamento com filhos.

 Só raramente Eckbert recebia visitas de hóspedes, e quando isso
acontecia, eles pouco alteravam o modo de vida habitual; a temperança
residia ali e a parcimônia em pessoa parecia ordenar tudo. Nessas
ocasiões, Eckbert ficava jovial e de bom humor, apenas quando ficava
sozinho é que se percebia nele certo ar taciturno, uma melancolia
silenciosa e retraída.

 Ninguém vinha ao burgo tão amiúde como Philipp Walther, um homem a quem
Eckbert se havia associado por encontrar nele uma forma de pensar muito
semelhante à sua própria. Ele residia em verdade na Francônia, mas com
frequência permanecia mais da metade do ano nas cercanias do burgo de
Eckbert coletando ervas e seixos e ocupando"-se em colocá"-los em ordem,
vivia de uma pequena fortuna e por isso não dependia de ninguém.
Eckbert muitas vezes acompanhava"-o em seus passeios solitários, 
e ano após ano os dois se uniam por uma amizade mais estreita.

 Há momentos em que uma pessoa é tomada de angústia se tiver de manter um
segredo que até então vinha ocultando de seu amigo com grande desvelo;
nessa hora, a alma sente um impulso irresistível de compartilhar tudo,
de descerrar frente ao amigo inclusive as coisas mais íntimas, a fim de
tornar essa amizade tanto mais sólida. Nessas ocasiões as almas se
revelam uma à outra em sua fragilidade, e de vez em quando também pode
suceder"-se de uma retroceder assustada diante da amizade da outra.

Já era outono quando em uma noite nebulosa Eckbert se achava sentado com
seu amigo e sua esposa Bertha junto ao fogo de uma lareira. As chamas
lançavam um vivo clarão através do aposento e brincavam no teto; a
noite espreitava lúgubre pelas janelas adentro, e as árvores no lado de
fora estremeciam com a fria umidade. Walther queixou"-se do longo
caminho de retorno que teria de percorrer, e Eckbert sugeriu"-lhe que
pernoitasse ali, passando parte da noite com uma conversa descontraída
e depois indo dormir até o amanhecer em um dos aposentos da casa.
Walther aceitou a proposta, e então foram trazidos o vinho e a ceia, o
fogo foi realimentado e a conversa entre os amigos tornou"-se 
cada vez mais alegre e espontânea.

Depois de os pratos terem sido retirados e os servos se afastado,
Eckbert tomou a mão de Walther e disse:

-- Meu amigo, vós deveríeis aproveitar a ocasião e ouvir de minha
esposa a história de sua infância, que é bastante incomum.

-- Com prazer -- disse Walther.

E sentaram"-se novamente junto à lareira.

Era então justamente meia"-noite, a lua espreitava de tempos em tempos
por entre as nuvens que passavam esvoaçantes.

-- Espero que vós não haveis de me considerar importuna -- começou
Bertha. -- Meu esposo diz que tendes uma maneira de pensar tão nobre que
seria errado ocultar algo de vós. Peço"-vos, porém, que por mais
inusitada que minha narrativa possa parecer não a tomeis por um conto
de fadas.

Nasci em uma aldeia, meu pai era um pobre pastor. As condições de meus
pais não eram das melhores, muitas vezes eles não sabiam de onde
poderiam tirar o pão. Mas o que eu lastimava bem mais era que meu pai e
minha mãe amiúde se altercavam por causa de sua pobreza e então um
fazia amargas censuras ao outro. Fora isso, constantemente diziam que
eu era uma criança tola e estúpida, incapaz de realizar até as tarefas
mais insignificantes, e, de fato, eu era por demais inepta e
desajeitada, sempre deixava cair as coisas, não aprendia nem a costurar
nem a fiar, não conseguia ajudar em nenhum serviço doméstico, somente a
penúria de meus pais era algo que eu compreendia muito bem. Com frequência
ficava então sentada em um canto com a cabeça cheia de fantasias sobre
como haveria de ajudá"-los se de um momento para outro me tornasse rica,
e como haveria de cumulá"-los de ouro e prata e me \mbox{deliciar} com seu
assombro; aí via espíritos elevando"-se pelos ares e me indicando
tesouros enterrados ou me dando pequenos seixos que se transformavam em
pedras preciosas, enfim, ocupava"-me das mais mirabolantes fantasias e,
quando depois disso tinha que me levantar para ajudar em algo ou
carregar alguma coisa, mostrava"-me ainda bem mais desajeitada porque
minha cabeça estava zonza com todos aqueles sonhos quiméricos.

Meu pai estava sempre muito zangado comigo por eu ser assim um fardo
totalmente inútil para eles; por isso, tratava"-me muitas vezes de modo
bastante cruel, e era raro receber dele uma palavra gentil. Assim
alcancei os oito anos de idade e nessa época foram
tomadas medidas sérias para que eu fizesse ou aprendesse alguma coisa.
Meu pai considerava que tudo não passava de capricho ou indolência de
minha parte a fim de passar meus dias em ociosidade; em suma: 
começou a me perseguir com veementes ameaças, e quando também elas não
trouxeram nenhum fruto, surrou"-me da maneira mais atroz dizendo que
essa punição seria repetida todos os dias já que eu não passava de uma
criatura inútil.

Durante toda aquela noite chorei amargamente, sentia"-me abandonada ao
extremo e com tamanha pena de mim mesma que desejava morrer. Temia o
alvorecer do dia, estava totalmente desnorteada e sem saber o que
fazer; desejava possuir todas as habilidades imagináveis, e não
conseguia entender por que era menos capaz do que as outras crianças
que conhecia. Estava à beira do desespero.

Quando despontou o dia, levantei"-me e, quase sem que o soubesse, abri a
porta de nossa pequena cabana. Encontrei"-me no campo aberto, pouco
depois estava em uma floresta em que ainda mal chegava a luz do dia.
Fui correndo sem parar e nunca olhava para trás, não sentia qualquer
cansaço, pois continuava acreditando que meu pai ainda poderia me
alcançar e, irritado pela minha fuga, tratar"-me"-ia com crueldade redobrada.

Quando alcancei o fim da floresta o sol já estava bastante alto; percebi
nesse momento que havia à minha frente algo escuro e encoberto por uma
densa névoa. Ora tive que escalar colinas, ora seguir por um caminho
que serpenteava por entre rochedos, e presumi então que devia estar
na serra circunvizinha, e comecei a sentir"-me apavorada naquela
solidão. Pois lá na planície nunca vira nenhuma montanha, e quando
ouvira alguém mencionando serras, a própria palavra já soara
assustadora aos meus ouvidos infantis. Não tive coragem de retornar,
foi meu medo justamente o que me impeliu adiante; muitas vezes olhava
sobressaltada para trás quando o vento passava sobre minha cabeça e se
infiltrava pelas árvores ou quando uma machadada longínqua ressoava
através da manhã silenciosa. Por fim, ao deparar"-me com carvoeiros e
mineiros e ouvir uma pronúncia estranha, por pouco não desmaiei de horror.

Perdoai minha prolixidade; sempre que falo dessa história,
involuntariamente torno"-me loquaz, e Eckbert, a única pessoa a quem a
narrei, sempre prestou tamanha atenção que me deixou mal"-acostumada.

Atravessei diversas aldeias e pedi esmolas, pois agora sentia fome e
sede; conseguia arranjar"-me razoavelmente com as respostas quando
alguém perguntava algo. Já avançara assim por uns quatro dias, quando
fui dar em uma pequena vereda que foi me levando cada vez mais para
longe da estrada principal. Os rochedos à minha volta começaram nesse
ponto a apresentar uma forma diferente, bem mais estranha. Eram
penhascos empilhados uns sobre os outros, que davam a impressão de que
o primeiro sopro de vento os faria despencar para todos os lados.
Fiquei em dúvida se deveria prosseguir. Durante as noites sempre havia
dormido na floresta, pois estávamos justamente na estação mais amena do
ano, ou então em cabanas de \mbox{pastores} \mbox{isoladas}; mas ali não encontrava
nenhuma moradia humana nem podia ter a expectativa de deparar"-me com
uma nesse descampado; os rochedos foram tornando"-se cada vez mais
tenebrosos, obrigando"-me diversas vezes a passar bem próximo a abismos
vertiginosos, e, por fim, até mesmo a trilha sob os meus pés
desapareceu. Fiquei absolutamente desconsolada, chorei e gritei, e o
eco de minha voz respondeu nos vales rochosos de uma maneira
aterrorizante. Então caiu a noite e escolhi um canto coberto de musgo
para nele repousar. Não pude dormir; durante a noite ouvi os ruídos
mais estranhos, que ora tomava por animais selvagens, ora pelo vento
gemendo entre as rochas, ora por pássaros inusitados. Rezei e adormeci
só muito tarde, pouco antes de amanhecer.

Acordei com a luz do dia batendo em meu rosto. À minha frente havia um
rochedo íngreme; escalei"-o na esperança de poder descobrir lá de cima
uma saída desse descampado e eventualmente divisar casas ou pessoas.
Mas quando alcancei o cimo, tudo ao meu redor, tão longe quanto a vista
alcançava, era igual ao lugar em que me encontrava, tudo estava
submerso em uma neblina perfumada, o dia estava cinzento e lúgubre, e
meus olhos não conseguiam distinguir nenhuma árvore, nenhum prado,
nenhum arbusto sequer, exceto umas poucas ramagens dispersas que haviam
crescido, solitárias e tristonhas, de fendas estreitas nas
rochas. Não é possível descrever a saudade que eu sentia de avistar ao
menos um único ser humano, ainda que ele fosse dos mais estranhos e me
inspirasse temor. A fome mortificava"-me enquanto isso, sentei"-me e
decidi"-me a morrer. Algum tempo depois, porém, a vontade de viver saiu
vitoriosa, reuni minhas forças e caminhei o dia inteiro sob lágrimas,
sob suspiros intermitentes. Por fim, já mal tinha consciência de mim,
estava com sono e esgotada, já mal tinha o desejo de viver e, ainda
assim, receava a morte.

Perto do anoitecer a região à minha volta pareceu tornar"-se um pouco
mais aprazível, minhas ideias e minha vontade reavivaram"-se, o desejo
de viver despertou em todas as minhas veias. Julguei então ouvir ao
longe o zunir de um moinho, acelerei meus passos e quão bem, quão leve
me senti quando realmente acabei por alcançar os limites do deserto de
rochedos, e mais uma vez estendiam"-se à minha frente bosques e prados
com longínquas e suaves montanhas. Era como se tivesse saído do inferno
e entrado no paraíso, a \mbox{solidão} e meu estado de desamparo nesse momento
já não pareciam mais assustadoras.

Em lugar de chegar ao esperado moinho fui dar a uma cachoeira, o que por
certo reduziu bastante minha alegria; estava colhendo com a mão um gole
de água do regato quando de súbito tive a impressão de ouvir a alguma
distância o som abafado de alguém tossindo. Nunca fora tão
agradavelmente surpreendida como nesse momento, caminhei naquela
direção e, na orla da floresta, divisei uma anciã que parecia estar
descansando. Estava trajada quase totalmente de preto, uma mantilha
negra cobria sua cabeça e boa parte de sua face, na mão segurava uma
bengala.

Aproximei"-me dela e pedi sua ajuda, a anciã convidou"-me a sentar ao seu
lado e deu"-me pão e um pouco de vinho. Enquanto eu comia, entoou com
voz esganiçada uma canção religiosa. Quando terminou, disse"-me para
acompanhá"-la.

A oferta me alegrou muitíssimo, não obstante a voz e o aspecto da anciã
me parecerem bizarros. Ela andava com bastante agilidade apoiada em sua
bengala, e fazia caretas a cada passada, e isso no início me fazia rir.
Os rochedos desabitados foram ficando cada vez mais para trás,
atravessamos uma suave campina e depois um bosque bastante extenso.
Quando chegamos ao fim dele o sol estava justamente se pondo, e jamais
me esquecerei da imagem e da sensação desse entardecer. Tudo se fundia
em delicados tons rubros e dourados, as árvores erguiam suas copas no
arrebol, e pelos campos derramava"-se um clarão encantador; as matas e
as folhas das árvores estavam imóveis, o céu límpido \mbox{parecia} um paraíso
de portas abertas, e o murmúrio das fontes e o ocasional zunir das
árvores atravessavam aquela risonha calmaria com um tom de jubilosa
melancolia. Minha alma juvenil alcançou então, pela primeira vez, uma
ideia do que era o mundo e suas particularidades. Esqueci"-me de mim e
de minha guia, meu espírito e meus olhos apenas voavam entusiasmados
por entre as nuvens douradas.

Subimos então a colina recoberta de bétulas, do alto via"-se um pequeno
vale repleto de bétulas, lá embaixo no meio das árvores havia uma
casinha. Um alegre latido soou em nossa direção e um ágil
cãozinho pulou na anciã abanando a cauda; depois ele veio ter comigo,
olhou"-me de todos os lados e em seguida retornou para junto da anciã
com trejeitos amáveis.

Quando descíamos pelo morro ouvi um cântico singular que parecia vir da
cabana, como se fosse de um pássaro; o canto era assim:

\begin{verse}
Doce solidão\\
Do bosque, que alegria\\
Dia após dia\\
E pelos tempos que virão\\
Oh, como me delicia\\
Doce solidão.
\end{verse}

 Estas poucas palavras eram incessantemente repetidas; esse canto, se
tivesse que descrevê"-lo, era quase como o som distante de uma charamela
e uma trompa de caça soando juntas.

 Minha curiosidade estava aguçada ao extremo; sem esperar pelo convite
da anciã entrei com ela na cabana. O crepúsculo já caíra, tudo estava
bem arrumado, havia algumas canecas em um armário na parede, vasos
misteriosos sobre uma mesa, junto à janela estava pendurado um pássaro
em uma pequena e reluzente gaiola, e era ele de fato quem entoava
aquelas palavras. A anciã arfava e tossia, parecia que não conseguia
mais se restabelecer, ora afagava o cãozinho, ora falava com o pássaro,
que apenas lhe respondia com sua canção habitual; na verdade, ela agia
como se eu nem estivesse presente. Enquanto fiquei assim a observá"-la,
diversas vezes senti um frio na espinha, pois seu rosto estava em um
movimento constante e distorcido, ao mesmo tempo em que a cabeça
balançava como se fosse de velhice de modo que se tornava impossível
discernir realmente as feições dela.

 Quando havia se restabelecido, ela acendeu uma luz, pôs uma mesa
diminuta e serviu a ceia. Então virou"-se para mim e disse"-me para
sentar em uma das cadeiras de vime trançado. Dessa forma fiquei sentada
bem a sua frente e a luz estava entre nós. Juntou suas mãos ossudas e
rezou em voz alta continuando a fazer caretas, de modo que eu quase
teria rido novamente; mas tomei o cuidado de controlar"-me para que ela
não se zangasse comigo.

 Depois da ceia, rezou outra vez, e em seguida ofereceu"-me um leito em
uma câmara muito pequena; ela dormiu na sala. Não permaneci desperta
por muito tempo, estava meio atordoada, mas durante a noite despertei
algumas vezes e então ouvia a anciã tossindo e falando com o cão
enquanto o pássaro, que parecia estar sonhando, cantava somente
palavras isoladas de sua canção. Esses sons, em conjunto com as bétulas
que murmuravam defronte a janela e o canto distante de um rouxinol, 
formavam uma combinação tão fantástica que eu ficava com a
impressão, não de ter despertado, mas de estar apenas caindo em um
outro sonho ainda mais estranho.

 De manhã a anciã me acordou e pouco depois impeliu"-me para o trabalho,
minha tarefa era fiar, e desta vez aprendi a fazê"-lo sem dificuldade,
além do mais também tinha que cuidar do cão e do pássaro. Rapidamente
acostumei"-me à lida doméstica, e todos os objetos ao redor se tornaram
conhecidos; tive então a impressão de que tudo era como deveria ser, já
não pensava que a anciã tinha algo de bizarro, que a localização da
casa era extravagante, e que havia algo de extraordinário no pássaro.
Mas sua beleza nunca deixou de chamar minha atenção, pois suas penas
reluziam em todas as cores possíveis, o mais formoso azul claro
alternava"-se em seu pescoço e corpo com o vermelho mais vivo, e quando
cantava enfatuava"-se de orgulho fazendo com que suas penas parecessem
ainda mais soberbas.

 Muitas vezes a anciã ausentava"-se e retornava apenas ao anoitecer,
então eu ia ao seu encontro com o cão e ela me chamava de minha menina
e filha. Com o tempo fui me afeiçoando bastante a ela, pois que nos
acostumamos a tudo, especialmente quando crianças. À noite ela
\mbox{ensinou"-me} a ler, logo assimilei a lição, e depois disso a leitura na
minha solidão tornou"-se uma fonte infinita de prazer, já que a anciã
possuía alguns livros antigos escritos à mão que continham histórias
mirabolantes.

 Até hoje a lembrança de como vivi naquela época continua parecendo"-me
estranha: sem receber a visita de nenhuma criatura humana, adaptada
somente a esse círculo familiar tão diminuto, pois o cão e o pássaro
davam"-me a mesma impressão que normalmente só pessoas há muito
conhecidas nos causam. Nunca mais pude recordar o curioso nome do cão,
embora o tivesse chamado tantas vezes naquele tempo.

 Já vivia assim com a anciã há quatro anos e devia estar com uns doze
anos, quando finalmente ela depositou maior confiança em mim e me
revelou um segredo: todos os dias o pássaro punha um ovo no qual se
achava uma pérola ou uma pedra preciosa. Já havia muito, eu percebera
que ela mexia às escondidas na gaiola, mas nunca me preocupara com
isso. Por ora ela incumbiu"-me da tarefa de recolher esses ovos durante
a sua ausência e guardá"-los cuidadosamente nos vasos misteriosos.
Daí por diante ela deixava alimentos para mim e passou a ausentar"-se
por períodos mais longos, semanas, meses; minha pequena roca chiava, o
cão latia, o pássaro mágico cantava enquanto a região na
circunvizinhança se mantinha tão serena que não me recordo de ter
havido durante todo esse tempo qualquer vendaval, qualquer tempestade.
Nunca ninguém perdeu o caminho e foi dar ali, nenhum animal selvagem
aproximava"-se de nossa morada, eu estava satisfeita e cantava, e meu
trabalho fazia os dias se sucederem\ldots{} O ser humano talvez fosse
bastante feliz se lhe fosse possível manter até o fim uma vida tão tranquila.

 A partir das poucas coisas que lia, ia formando uma ideia bastante
fabulosa do mundo e das pessoas; tudo se assemelhava a mim e a meus
companheiros: quando eram mencionadas pessoas alegres, eu não conseguia
imaginá"-las de outro modo a não ser como o pequeno lulu, damas
faustosas sempre tinham a aparência do pássaro, todas as mulheres
idosas, a da minha bizarra anciã. Também li um pouco sobre o amor, e
então fabricava na minha imaginação histórias fantasiosas envolvendo a
mim mesma. Imaginava o cavaleiro mais belo do mundo, dotava"-o de todas
as qualidades, embora realmente não soubesse, após todos esses
esforços, qual era a aparência dele; mesmo assim, sentia uma grande
pena de mim mesma quando ele não correspondia ao meu amor e nesses
momentos elaborava em pensamento, ou por vezes também em voz alta,
longos e tocantes discursos a fim de conquistá"-lo. Vós estais
sorrindo! Deveras, nós todos agora já passamos por esse tempo de juventude.

 Nessa época preferia mesmo ficar só, pois então era eu própria quem
mandava na casa. O cão amava"-me muito e fazia tudo o que eu queria; o
pássaro respondia a todas as minhas perguntas com seu cântico; minha
pequena roca  girava sempre com vivacidade, e assim, no fundo, nunca fui
tomada pelo desejo de mudanças. Quando a anciã retornava de suas
longas jornadas, elogiava minha dedicação, e dizia que, desde a minha
chegada, a casa estava mais bem cuidada, ela ficava contente com
meu crescimento e minha aparência sadia, enfim, tratava"-me como a uma filha.

``Tu és valorosa, minha menina!'', disse"-me ela certa vez com um
som estridente, ``se continuares assim, sempre haverás de passar
bem; por outro lado, sair do bom caminho nunca traz bons frutos, o
castigo é infalível e nunca é tarde demais para ele''. Quando ela assim
falou, não lhe dei muita atenção, pois era muito vivaz em minha maneira
de ser; mas à noite lembrei"-me de suas palavras e não consegui
compreender o que ela quisera dizer com aquilo. Refleti com cuidado
sobre cada palavra, decerto eu havia lido sobre riquezas e, por fim,
veio"-me a ideia de que suas pérolas e pedras preciosas provavelmente
fossem valiosas. Dentro em breve essa ideia acabaria adquirindo
contornos ainda mais definidos. Mas o que ela queria dizer com o bom
caminho? Ainda não conseguia entender perfeitamente o sentido de suas
palavras.

 Completei quatorze anos, e é uma desventura para o ser humano o fato de
alcançar a razão e, em troca, infalivelmente perder a inocência de sua
alma. Eis que eu compreendi claramente que, se assim o quisesse,
poderia apoderar"-me do pássaro e das joias quando a anciã estivesse
longe e partir com eles em busca do mundo sobre o qual havia lido. Aí
talvez até pudesse encontrar o formosíssimo cavaleiro de quem ainda não
me esquecera.

 No princípio essa era uma ideia como qualquer outra, mas enquanto
estava sentada junto à roda de fiar, esse pensamento sempre ficava
retornando contra a minha vontade, e acabei deixando"-me levar por ele
de tal modo que já me via magnificamente adornada e cercada de
cavaleiros e príncipes. Nas ocasiões em que me deixava levar assim,
tornava"-me bastante tristonha quando novamente levantava os olhos e
percebia estar na pequena cabana. Aliás, desde que fizesse minhas
tarefas, a anciã não me dava maior atenção.

 Certo dia minha senhoria partiu novamente, dizendo"-me que dessa vez
haveria de ficar longe por mais tempo do que de costume, ela exortou"-me
a cuidar muito bem de tudo e a não me entregar ao tédio. Despedi"-me
dela com certa aflição, pois tinha a sensação de que não tornaria a
vê"-la. Segui"-a com os olhos por um longo tempo, embora eu mesma não
soubesse por que estava tão assustada; era quase como se meu intento já
estivesse decidido sem que eu tivesse plena consciência disso.

 Nunca cuidei do cão e do pássaro com tamanha solicitude; meu afeto por
eles era maior do que antes. A anciã já estava ausente havia alguns
dias quando acordei com o firme propósito de abandonar a cabana com o
pássaro e de sair em busca do assim chamado mundo. Meu coração estava
apertado e cheio de angústia, desejei novamente continuar ali, e não
obstante essa ideia também me era repugnante; uma estranha batalha
travou"-se em minha alma, como se houvesse em mim dois espíritos
rebeldes em combate. Ora a plácida solidão parecia"-me tão encantadora,
ora se entusiasmava outra vez com a ideia de um mundo novo com toda a
sua maravilhosa diversidade.

 Não sabia que decisão tomar, o cão não parava de pular carinhosamente
em mim, os raios do sol derramaram"-se com alegria pelos campos, as
verdes bétulas reluziam: tive a sensação de ter algo muito urgente a
fazer, por conseguinte segurei o cãozinho, amarrei"-o dentro da sala e
tomei sob o braço a gaiola com o pássaro. O cão vergou"-se e choramingou
por causa desse tratamento inusitado, lançou"-me um olhar suplicante,
mas eu tinha receio de levá"-lo comigo. Em seguida tomei um dos vasos
repletos de pedras preciosas e coloquei"-o entre as minhas coisas, deixando os
demais onde estavam.

 O pássaro revirou a cabeça de um modo bizarro quando passei com ele
pela porta; o cão esforçou"-se muito em acompanhar"-me, mas teve de
ficar para trás.

 Evitando o caminho que levava aos rochedos agrestes, parti na direção
oposta. O cão latia e choramingava sem parar, e eu fiquei profundamente
comovida; o pássaro dispôs"-se algumas vezes a cantar, mas, como estava
sendo carregado, isso devia ser"-lhe incômodo.

 Enquanto prosseguia caminhando, os latidos soavam cada vez mais
fracos e, por fim, findaram de vez. Chorei e estive prestes a tomar o
caminho de volta, mas o anseio de ver algo novo impeliu"-me adiante.

 Já passara montanhas e alguns arvoredos quando caiu a noite e fui
forçada a procurar albergue em uma aldeia. Eu estava muito desajeitada
quando entrei na taverna, deram"-me um aposento e um leito, dormi
bastante tranquilamente apesar de sonhar com a anciã, que me ameaçava.

 Minha viagem transcorreu de forma bastante uniforme, mas quanto mais
avançava mais ia ficando atemorizada com a imagem da anciã e do
cãozinho; eu ficava pensando que, sem meu auxílio, ele provavelmente
morreria de fome; quando atravessava alguma floresta, tinha a impressão
de que a anciã de repente apareceria à minha frente. Dessa forma, era
sob \mbox{lágrimas} e suspiros que continuava meu caminho; em todas as
ocasiões em que parava para descansar e depositava a gaiola no chão, o
pássaro entoava sua canção fantástica e com isso fazia"-me recordar de
forma muito nítida as belas paragens que eu abandonara. Como a
natureza humana tende ao esquecimento, acreditava então que minha
viagem anterior durante a infância não tivesse sido tão tristonha como
a atual; desejei estar novamente naquela situação de outrora.

 Eu tinha vendido algumas pedras preciosas e, depois de uma jornada de
vários dias, cheguei a uma aldeia. Já na chegada tive uma sensação
estranha, assustei"-me e não sabia por quê; mas logo entendi os meus
sentimentos, pois era a mesma aldeia em que eu havia nascido. Como fiquei
admirada! Minha alegria, motivada por mil lembranças curiosas, foi
tamanha que as lágrimas correram pelas faces! Muitas coisas estavam
diferentes, haviam surgido casas novas, outras, que naquela época
tinham acabado de ser erigidas, agora estavam em estado decadente,
também avistei construções que sofreram incêndios; tudo era bem mais
diminuto e apertado do que eu esperava. Senti uma alegria infinita pela
expectativa de rever meus pais depois de tantos anos; encontrei a
casinha, a soleira tão familiar, a maçaneta ainda era exatamente como
outrora, foi como se tivesse sido apenas ontem que a fechei; meu
coração bateu com violência, abri com um gesto brusco\ldots{} Mas na sala
havia semblantes totalmente estranhos que me encaravam. Indaguei pelo
pastor Martin e disseram"-me que já havia morrido há três anos com sua
esposa. Rapidamente recuei e, em prantos, abandonei a aldeia.

 Eu havia imaginado que seria tão bonito surpreender meus pais com minha
riqueza inesperada; aquilo com que na infância eu apenas tinha podido
sonhar havia"-se tornado realidade devido a um acaso dos mais
extraordinários\ldots{} Mas agora tudo foi em vão, eu não podia dar essa
alegria a eles, e aquilo pelo que eu mais ansiara a vida toda estava
perdido para sempre.

 Em uma cidade agradável aluguei uma casinha com jardim e tomei os
serviços de uma criada que veio morar comigo. O mundo não era tão
maravilhoso como eu havia suposto, mas comecei a pensar um pouco menos na
anciã e em minha antiga moradia e, de modo geral, vivia bastante
satisfeita.

 O pássaro já não cantava fazia bastante tempo; por isso, não foi
pequeno o meu susto quando certa noite recomeçou e, dessa vez, com uma
canção modificada. Ele cantou:

\begin{verse}
Doce solidão\\
Do bosque, longe de minha visão.\\
Remorso principia --\\
Nos dias que serão!\\
Oh, única alegria,\\
Doce solidão.
\end{verse}

 Durante toda aquela noite não pude dormir, tudo me voltou à memória e,
mais do que nunca, senti que causara uma injúria. No dia seguinte a
visão do pássaro era"-me por demais odiosa, ele ficava olhando para mim,
e sua presença causava"-me temor. Passou a entoar sua canção
ininterruptamente e com voz mais alta e sonora do que antes fora seu
costume. Quanto mais o observava, maior era o meu pavor; por fim, abri
a gaiola, enfiei minha mão nela e peguei seu pescoço, apertei os dedos
com força, ele lançou"-me um olhar suplicante, soltei"-o, mas já estava
morto\ldots{} Enterrei"-o no jardim.

 A partir de então comecei a ficar inquieta por causa de minha criada,
pensei no que eu mesma fizera e imaginava que também ela algum dia
poderia me roubar ou até mesmo me assassinar. 

Já há algum tempo conhecia um jovem cavaleiro que me agradava
sobremaneira, dei"-lhe minha mão e, com isso, senhor Walther, minha
história chegou ao fim.

 -- Vós devíeis tê"-la visto naquela época -- interrompeu Eckbert com
sofreguidão. -- Sua juventude, sua formosura e que encanto incompreensível
lhe fora conferido através de sua educação solitária. Ela deu"-me a
impressão de um milagre e eu lhe dediquei um amor além de todas as
medidas. Eu não tinha posses, mas o amor dela permitiu"-me chegar a esse
bem"-estar; viemos residir aqui e, até hoje, nem por um momento nos
arrependemos de nossa união.

 -- Mas de tanto eu falar -- recomeçou Bertha -- a noite já vai bem
adiantada. Vamos nos recolher para dormir!

 Levantou"-se e foi ao seu aposento. Walther desejou"-lhe boa noite com um
beijo na mão, e disse:

-- Nobre senhora, agradeço"-vos, posso imaginar"-vos muito bem com o
estranho pássaro e cuidando do pequeno \textit{Strohmian}.

 Também Walther recolheu"-se, somente Eckbert continuou na sala, andando
inquieto de um lado para outro. 

-- O ser humano é realmente um tolo! -- desatou ele a falar. --
Primeiro, dou ensejo para que minha mulher narre sua história, e agora
arrependo"-me desse gesto de confiança! Não irá ele trair minha
amizade? Não irá contar a outros o que ouviu? Não poderá, já que assim
é a natureza humana, criar uma desditosa cobiça pelas nossas pedras
preciosas e por isso imaginar planos e se dissimular?

 Ocorreu"-lhe que Walther não se despedira dele tão cordialmente como
seria natural após uma confidência daquelas. Uma vez que a alma foi
tomada de desconfiança, acaba também encontrando em cada detalhe uma
confirmação. Por outro lado, havia momentos em que Eckbert se
repreendia por nutrir uma suspeita tão vil contra seu bom amigo e,
mesmo assim, não conseguia evitar senti"-la novamente. Durante a
noite inteira debateu"-se com esses pensamentos e dormiu bem pouco.

 Bertha estava doente e não pôde comparecer para o café da manhã;
Walther parecia não se preocupar muito com isso e inclusive despediu"-se
do cavaleiro com bastante indiferença. Eckbert não conseguia entender
seu comportamento; foi ver sua esposa, que ardia em febre, e disse"-lhe
que ela devia estar extenuada por causa da narrativa da noite.

 Desde aquela noite, as visitas de Walther ao burgo de seu amigo
tornaram"-se raras, e, quando acontecia de vir, logo partia outra vez
após algumas palavras insignificantes. Esse comportamento mortificava
Eckbert ao extremo, muito embora não demonstrasse nada para Bertha e
Walther, mas ambos deviam estar percebendo nele sua agitação interior.

 A doença de Bertha tornava"-se cada vez mais preocupante; o médico
meneava a cabeça; o rosado das faces dela
desaparecera e seus olhos iam ficando cada vez mais febris. Certa
manhã, mandou que chamassem seu esposo para junto de seu leito, as
servas tiveram que se retirar.

 -- Amado esposo -- começou --, preciso revelar"-te algo que quase me custou
o juízo e arruinou minha saúde, ainda que possa parecer em si um
detalhe insignificante. Tu deves lembrar"-te que sempre que narrava
minha história eu não conseguia recordar, a despeito de todo esforço
que fizesse, o nome do cãozinho com o qual convivi por tanto tempo.
Naquela noite, quando Walther se despedia de mim, ele disse de repente:
``Posso imaginar"-vos muito bem cuidando do pequeno \textit{Strohmian}''.
Será coincidência? Terá adivinhado o nome, ou terá feito a menção com
algum propósito? E, nesse caso, que ligação haverá entre esse homem e
meu destino? Por vezes digo a mim mesma que essa coincidência não passa
de simples fruto de minha imaginação, mas isso é real, absolutamente
real. Um pavor colossal apossou"-se de mim no momento em que uma pessoa
estranha auxiliou"-me dessa forma com minhas recordações. O que dizes,
Eckbert?

 Eckbert contemplou sua esposa doente com profundo pesar; permaneceu em
silêncio, pensativo, em seguida disse"-lhe algumas palavras de consolo e
deixou"-a. Em um aposento afastado, ia de um lado a outro com uma
agitação indescritível. Há muitos anos Walther vinha sendo o único a
frequentar sua casa, e não obstante era a única pessoa no mundo cuja
existência o oprimia e atormentava. Tinha a impressão de que haveria de
se sentir aliviado e feliz se essa única criatura pudesse ser afastada
de seu caminho. Tomou sua besta a fim de distrair"-se e caçar.

 Era um dia de inverno, sombrio e tempestuoso, e vasta camada de neve
cobria as montanhas e vergava os ramos das árvores até o chão. Vagueou
sem um destino certo, o suor cobria"-lhe a testa, não encontrava nenhum
animal selvagem e isso aumentava seu azedume. De súbito viu algo
movendo"-se a distância, era Walther coletando musgo das árvores; sem
saber o que fazia, apontou a arma, Walther volveu"-se, fez um gesto mudo
de ameaça, mas nesse instante o dardo partiu e Walther tombou.

 Eckbert sentiu"-se aliviado e tranquilo, contudo, um calafrio incitou"-o
a retornar a seu burgo; tinha um longo caminho pela frente, pois
percorrera uma grande distância a esmo pelas florestas adentro. Quando
chegou, Bertha já havia falecido; antes de morrer ela ainda falara
muito sobre Walther e a anciã.

 Eckbert viveu então por longo período em profunda solidão; noutros
tempos já costumava ser um pouco tristonho, pois a estranha história de
sua esposa o inquietava, sempre temera que algum incidente \mbox{funesto}
pudesse ocorrer; mas agora seu estado era de total desmoronamento
interior. O assassinato de seu amigo pairava"-lhe sem trégua diante dos
olhos, ele vivia censurando"-se interiormente.

 Em busca de distração, às vezes dirigia"-se até a cidade grande mais
próxima onde comparecia a festas e reuniões sociais. Ansiava por algum
amigo que preenchesse o vazio em sua alma, mas bastava recordar"-se de
Walther e a palavra amigo o deixava em sobressalto; convencera"-se de
que inevitavelmente haveria de sofrer desventuras com quem quer que
fosse seu amigo. Vivera por tanto tempo com Bertha em doce serenidade,
a amizade de Walther por tantos anos trouxera"-lhe contentamento, e
agora ambos tinham sido ceifados de modo tão brusco que em alguns
momentos sua vida mais lhe parecia um fabuloso conto de fadas do que
uma existência real.

 Um jovem cavaleiro, \textit{Hugo von Wolfsberg}, procurou a companhia
do calado e taciturno Eckbert e parecia sentir uma inclinação sincera
por ele. Eckbert sentiu"-se maravilhosamente surpreso, correspondeu à
amizade do cavaleiro tanto mais rapidamente quanto menos havia contado
com ela. Os dois passaram a ficar juntos com frequência, o desconhecido
realizava toda sorte de obséquios para Eckbert, um já quase não saía
mais a cavalo sem o outro, em todas as reuniões sociais eles se
encontravam, enfim, os dois pareciam inseparáveis.

 A alegria de Eckbert costumava durar apenas curtos momentos, pois
ele tinha a nítida sensação de que a afeição de Hugo se devia tão
somente a um engano: ele não o conhecia, não sabia sua história, e mais
uma vez ele foi tomado por aquele mesmo anseio de revelar"-se por
completo a fim de poder certificar"-se do quanto o outro era seu amigo.
Dali a pouco, porém, seu intento era tolhido por escrúpulos e pelo
temor de ser rejeitado. Havia momentos em que estava tão convencido de
sua infâmia que acreditava que nenhuma pessoa poderia estimá"-lo caso o
conhecesse um pouco melhor. Entretanto, não pôde refrear"-se; durante um
solitário passeio a cavalo revelou a seu amigo toda sua história,
perguntando"-lhe em seguida se poderia sentir amizade por um assassino.
Hugo ficou comovido e procurou consolá"-lo; Eckbert acompanhou"-o até a
cidade com o coração aliviado.

 Mas ele parecia estar amaldiçoado a ver nascer a suspeita sempre no
momento da confidência, pois, mal haviam penetrado no salão, contemplou
seu amigo quando iluminado pelas muitas velas, e sua expressão não lhe
agradou. Acreditou perceber um sorriso pérfido, notou que só falava
pouco com ele, que conversava bastante com os demais ao passo que a ele
parecia ignorar. Encontrava"-se ali na reunião um cavaleiro idoso que
sempre se mostrara um adversário de Eckbert e sempre indagara de modo
estranho sobre sua riqueza e sua esposa; a este juntou"-se Hugo e ambos
ficaram algum tempo conversando furtivamente e olhando para Eckbert.
Este agora via sua suspeita confirmada, considerava"-se traído, e uma
cólera terrível apossou"-se dele. Enquanto ainda mantinha os olhos fixos
naquela direção, de repente avistou o semblante de Walther, todos os
seus traços, toda sua figura, para ele tão familiar; continuava ainda
olhando para lá e ficou convencido de que não era ninguém senão
\textit{Walther} quem conversava com o ancião. Seu horror foi
indescritível; descontrolado, precipitou"-se para fora, ainda nessa
noite abandonou a cidade e retornou a seu burgo depois de errar o
caminho várias vezes.

 Qual um fantasma errante perambulou de aposento a aposento, seus
pensamentos estavam em completo torvelinho, pensamentos terríveis eram
sucedidos por outros ainda mais terríveis, e seus olhos foram
totalmente abandonados pelo sono. Muitas vezes pensou que havia
enlouquecido e que criava tudo aquilo em sua imaginação; em seguida os
traços de Walther voltavam à sua memória e tudo lhe parecia cada vez
mais enigmático. Decidiu sair em viagem a fim de colocar seus
pensamentos outra vez em ordem; a ideia de ter um amigo, o desejo de
companhia ele agora tinha abandonado para sempre.

 Partiu sem estabelecer uma rota definida, aliás, mal contemplava as
paisagens que se estendiam à sua frente. Quando já trotava com seu
cavalo há alguns dias, viu"-se de repente perdido em um labirinto de
rochas que em parte alguma permitiam descobrir uma saída. Finalmente
encontrou um velho camponês que lhe indicou um caminho que passava por
uma cachoeira; quis dar"-lhe algumas moedas em agradecimento, mas o
camponês as recusou.

-- Que importa? -- disse Eckbert consigo mesmo. -- Eu poderia acabar
imaginando outra vez que ele é Walther!

Nisso volveu os olhos novamente para trás e era Walther! Eckbert
esporeou seu corcel e correram tão rápido quanto este conseguia,
atravessando campinas e bosques até que o animal desabasse embaixo
dele. Sem se incomodar com isso, passou então a seguir sua viagem \mbox{a pé.}

 Subiu absorto por uma colina; pareceu"-lhe distinguir nas proximidades
um latido alegre ao qual se misturava o sussurro de bétulas, e ouviu
cantarem uma canção em tom singular:

\begin{verse}
Doce solidão\\
Do bosque, de novo que alegria.\\
Sempre estou são,\\
Aqui não mora ambição.\\
Outra vez me delicia,\\
Doce solidão.
\end{verse}


 Isto deu um golpe fatal na mente, no juízo de Eckbert; ele não
conseguia encontrar a chave do enigma: estaria sonhando agora ou teria
ele sonhado outrora com uma mulher chamada Bertha; as coisas mais
fantásticas mesclavam"-se às mais banais, o mundo ao seu redor estava
enfeitiçado, e ele não era capaz de qualquer pensamento, qualquer
recordação.

 Uma anciã de costas vergadas caminhava devagar, subindo a colina com
uma bengala e tossindo.

-- Estás trazendo meu pássaro para mim? Minhas pérolas? Meu cão? --
gritou ela dirigindo"-se a Eckbert. -- Vejas, a injúria causa seu próprio
castigo: ninguém senão eu era o teu amigo Walther, teu Hugo.

 -- Deus do céu! -- disse Eckbert de mansinho para si mesmo. -- Em que
tenebrosa solidão passei então minha vida!

 -- E Bertha era tua irmã.

 Eckbert caiu ao chão.

 -- Por que ela me abandonou desse modo pérfido? Caso contrário
tudo teria terminado bem e direito, seu tempo de provação já havia
terminado. Ela era a filha de um cavaleiro que a entregou a um pastor
para que a criasse, a filha de teu pai.

 -- Por que sempre pressenti essa terrível ideia? -- exclamou
Eckbert.

 -- Porque em tua infância mais tenra certa vez o ouvistes falando
sobre isso: por causa da esposa ele não podia criar essa filha junto a
si, pois era de outra mulher.

 Eckbert jazia enlouquecido no chão e sua vida se esvaia; em tons surdos
e emaranhados ouvia a anciã falando, o cão latindo e o pássaro repetindo sua canção.
\medskip

\hfill\textit{Tradução de Karin Volobuef}



\chapter{A Montanha das Runas}

\textsc{Um jovem caçador} sentava"-se pensativo no seio mais profundo entre picos
montanhosos, perto de uma revoada de pássaros, e na solidão ouvia"-se o
murmúrio dos riachos e da mata. Ele refletia sobre seu destino, sobre
como era bem jovem e como abandonara pai, mãe e terra natal e todos os
amigos do vilarejo em busca de novas paragens, a fim de se afastar do
círculo de acontecimentos habituais e sempre recorrentes. Erguia os
olhos, como se estivesse ligeiramente surpreendido por se encontrar
agora ali no vale ocupado dessa maneira. Grandes nuvens atravessavam o céu
e se perdiam atrás das montanhas, passarinhos gorjeavam no meio da
floresta e um eco lhes respondia ao longe. Desceu lentamente a encosta
da montanha e sentou"-se à margem de um regato que sussurrava espumando
sobre seixos salientes. Ficou escutando a melodia ritmada da água e o
marulhar parecia narrar"-lhe mil prodígios através de palavras
enigmáticas. O jovem não pôde deixar de sentir no íntimo uma grande
tristeza por não compreender a mensagem cifrada das águas. Seu olhar
vagou pelos arredores e ele teve consciência da própria felicidade e
bem"-aventurança. Assim, reuniu coragem e pôs"-se a entoar em voz alta
uma cantiga de caçador.

\begin{verse}

Por entre rochas, airoso\\
Vai à caça o jovem moço.\\
No meio da mata verde\\
Sua caça não se encerra,\\
Nem no escuro um tiro erra!

Late ruidosa a matilha\\
E se embrenha pela trilha.\\
Alto as cornetas retumbam\\
E corações valentes inundam.\\
Oh, que bela estação de caça!

Pendem as ramas frondosas\\
Às outonais brisas buliçosas\\
Da escarpada natureza.\\
Renas, cervos com destreza\\
Acertará bem no alvo!

Camponês renuncia à luta,\\
Marinheiro esquece a labuta.\\
Ninguém vislumbra a essa hora\\
Os olhos radiantes da Aurora\\
Que espalha o orvalho na relva

Senão quem a caça conhece\\
E à deusa Diana agradece.\\
Se a harmonia a alma invade\\
Um canto em enleio evade!\\
Ah, quão feliz é o caçador!
\end{verse}

Enquanto ele cantava, o sol tinha descido ao horizonte e longas sombras
tombavam perpendiculares pelo estreito vale. O frescor do crepúsculo se
espraiava rente ao chão e agora somente os cimos das árvores, bem como
os arredondados picos das montanhas refletiam dourados a luz da tarde.

Cada vez mais melancólico, Christian não queria voltar"-se para o bando
de pássaros, tampouco desejava permanecer ali: sentiu sua alma cheia de
solidão e ansiou pela companhia de pessoas. Então ele almejou os velhos
livros que sempre vira na casa dos pais e nunca quisera ler, embora o
pai o incentivasse; as reminiscências da infância lhe acorreram à
lembrança, os jogos com as crianças do vilarejo, os rostos conhecidos,
a escola que lhe fora tão opressiva, e ele teve vontade de retornar ao
lugar que deixara voluntariamente a fim de tentar a sorte em terras
distantes, nas montanhas, em meio a estranhos, numa ocupação diferente.
Mesmo quando a escuridão se impôs, o murmúrio do riacho tornou"-se mais
ruidoso e as aves noturnas começaram a circular em todos os sentidos em
voos aventurosos, ele se mantinha sentado, aborrecido e absorto em suas
reflexões. Com gosto ele teria dado vazão às lágrimas, não sabia o que
empreender ou por onde começar.

Sem pensar, puxou do solo uma raiz saliente e, de súbito, estremeceu
ao ouvir de dentro da terra um surdo gemido que se prolongou
em sons de choro se perdendo tristemente ao longe. Aquele pranto
penetrou até o âmago de seu coração; o comoveu como se tivesse tocado
sem querer a ferida na qual o sofrimento acompanha a agonia da
natureza. Ele se levantou com um salto e quis fugir, pois com certeza
já escutara certa feita a respeito das estranhas mandrágoras,
que ao serem arrancadas produzem lamentos plangentes capazes de levar o
homem à loucura com sua dor.

Quando fez menção de ir embora, viu atrás de si um forasteiro que o
olhava amigavelmente e lhe perguntou para onde estava indo. Christian
almejara companhia, no entanto, mais uma vez estremeceu constatando a
presença do companheiro cordial. O homem repetiu a pergunta:

-- Aonde você vai com essa pressa toda?

O jovem caçador procurou reaver seu espírito e explicou como a solidão
de súbito lhe parecera terrível e ele quisera se pôr a salvo, a tarde
estava tão escura, as sombras verdes das árvores tão lúgubres, o
regato só rumorejava lamentos e as nuvens do céu atraíam sua saudade
para além das montanhas.

O forasteiro disse:

-- Você é ainda muito jovem, não pode ainda suportar a severidade da
solidão; vou acompanhá"-lo porque num raio de milha não há casa ou
vilarejo, pelo caminho conversaremos e contaremos casos, dessa maneira
você espantará os maus pensamentos. Em uma hora a lua surgirá por
detrás dos montes, seu clarão sem dúvida iluminará também sua alma.

Os dois se puseram a caminho e o rapaz teve logo a impressão de que o
estranho era um velho camarada.

-- Como veio parar nessas montanhas? -- perguntou o homem. -- Ouvindo
o seu sotaque, percebe"-se que não é nativo.

Ao que o jovem respondeu:

 -- Ah, essa conversa poderia render muito, não vale a pena estender
esse assunto. Fui atirado para fora do círculo de meus pais e
familiares, deixei de ser senhor de mim mesmo, assim como o pássaro
preso na arapuca em vão se alvoroça, minha alma se debatia num enredado
de fantasias e desejos inexplicáveis.

Nós morávamos bem longe daqui, numa planície donde ao redor não se vê
uma montanha, sequer uma colina. Árvores isoladas ornavam a extensão
verdejante, mas de perto, campos, plantações férteis de cereais e
jardins se estendiam a distância a perder de vista, e um largo rio
refulgia como um espírito poderoso margeando campos e lavouras.

Meu pai era jardineiro do castelo e tinha a intenção de me educar para o
ofício; ele amava as plantas e flores acima de tudo e era capaz de
dedicar dias a fio aos seus cuidados e cautelas. Chegava a ponto de
afirmar que quase se comunicava com as plantas, pois, dizia, se
instruía pelo seu crescimento e prosperidade, bem como pela diversidade
de formas e cores das folhagens.

Eu, por minha vez, sentia aversão pelo trabalho de jardinagem, e tanto
mais quando meu pai procurava me persuadir ou obrigar"-me através de
ameaças. Queria me tornar pescador e cheguei a tentar, porém tampouco a
vida na água não me convinha, então fui enviado a um comerciante na cidade,
a quem também abandonei para retornar a casa paterna. Certo dia, porém,
escutei meu pai falando da região montanhosa por onde viajara em sua
juventude: das minas subterrâneas e seus mineiros, dos caçadores e seu
ofício, e subitamente despertou em mim um desejo claro, o sentimento de
ter, enfim, vislumbrado a vida para a qual fora predestinado.

Dia e noite eu sonhava e passava cismando sobre elevadas montanhas,
precipícios e florestas de \mbox{pinheiros}; minha imaginação criava enormes
rochedos, em pensamento eu ouvia o tumulto da caçada, as cornetas, o
berreiro dos cães e da caça. Tudo isso correspondia aos meus sonhos e
assim eu me via impedido de ter um instante de sossego ou de paz. A
planície, o castelo, o pequeno e limitado jardim de papai com seus
canteiros de flores bem alinhados, a morada medíocre e o céu que se
estendia triste pela vastidão sem abraçar sublimes cumes de morros,
tudo isso me oprimia e me angustiava. Eu estava convencido de que todos
os meus próximos viviam na mais deplorável ignorância e compartilhariam
comigo opiniões e impressões se sua alma por um segundo que fosse
tivesse consciência dessa condição miserável. Portanto, continuei a
vagar de um lado para o outro, até que numa bela manhã, tomei a
resolução de deixar para sempre a casa de meus pais.

Num livro eu encontrara informações a respeito do grande maciço mais
próximo com ilustrações de paisagens e, por conseguinte, planejei meu
percurso. Era o começo da primavera, eu me sentia sereno e jovial.
Apressei"-me em sair o quanto antes da planície e numa tarde avistei
diante de mim, lá longe no horizonte, os contornos sombrios da
cordilheira. No albergue, mal pude dormir, tamanha era minha ansiedade
por adentrar a região que eu considerava minha pátria: ao nascer do sol
já estava de pé e novamente em marcha. Ao meio"-dia eu me encontrava
rodeado pelas montanhas bem"-amadas e, meio inebriado, seguia o caminho,
aqui e acolá me detendo por uns instantes, a fim de olhar para trás,
encantado com a visão de tantos detalhes que me eram desconhecidos e ao
mesmo tempo de algum modo familiares.

Logo depois eu perdi completamente de vista a planície que ficara atrás
de mim. Amplas florestas vinham ao meu encontro, carvalhos e faias, do
alto de serras escarpadas, saudando"-me com o farfalhar grave de suas
folhas em movimento. Meu caminho me conduzia ao longo de precipícios
vertiginosos, as montanhas azuladas se perfilavam ao horizonte, imensas
e majestosas. Um mundo novo se descortinava ante meus olhos, eu não
sentia um pingo de fadiga. Dessa maneira, ao cabo de alguns dias
percorrendo grande parte daquele maciço, cheguei à casa de um velho
caçador que aos meus instantes rogos consentiu em admitir"-me e
instruir"-me na arte do ofício de caçador. Há três meses estou ao seu
serviço.

Tomei posse da região onde deveria me estabelecer como de um reino:
aprendi a conhecer cada rochedo, cada abismo da montanha. Eu estava
eufórico de alegria no exercício do ofício: nas manhãs frias quando nós
partíamos para a floresta, abatíamos árvores no bosque, eu praticava a
mira e a carabina e adestrava os nossos fiéis companheiros, os cães, às
suas habilidades. Agora estou instalado há oito dias nessas alturas,
perto da revoada de pássaros, no local mais solitário da montanha e à
tardinha abateu"-me uma tristeza no coração como nunca sentira antes;
tive a sensação de estar perdido e infeliz e desde então não consigo me
recuperar desse funesto estado de ânimo.

O forasteiro escutara tudo atentamente. Agora eles atingiam uma clareira
e saudava"-os amigavelmente a luz da lua que se mostrava crescente no
alto dos picos, formas indefiníveis, massas espessas e separadas que a
claridade pálida reunia em combinações enigmáticas. O maciço fendido
espraiava"-se ante ambos, ao fundo um pico mais elevado, sobre o qual
ruínas antigas descompostas pelo tempo assumiam um aspecto sinistro à
alva luminosidade.

-- Nossos caminhos se bifurcam neste ponto -- disse o homem. -- Eu sigo
por ali abaixo, ao lado do poço é minha morada; os metais são meus
vizinhos, as águas da montanha me contam prodígios maravilhosos, no
entanto você não pode me acompanhar. Mas dê uma olhada na Montanha
das Runas com suas muralhas escarpadas, como os vetustos rochedos nos
contemplam impassíveis e fascinantes! Você nunca esteve lá nos
píncaros?

-- Nunca! -- respondeu o jovem Christian. -- Certa vez, ouvi um velho
caçador contar coisas esquisitas sobre essa montanha, mas fui sonso o
bastante para esquecer o que ele falou. Lembro, porém, que naquela
noite fiquei muito amedrontado. Eu gostaria de escalar aqueles picos um
dia, pois as luzes devem ser mais brilhantes, a relva mais verdejante,
o mundo em torno mais curioso. Calculo que se possa até encontrar uma
ou outra maravilha de tempos passados.

-- É bem provável -- concordou o outro. -- Quem sabe procurar o que a
profunda intuição do peito almeja, esse encontrará naquelas alturas
amigos e preciosidades de épocas remotas, enfim, tudo aquilo a que
aspira ardentemente.

Com essas palavras, o velho desceu rápido sua trilha sem sequer se despedir 
de seu companheiro de viagem, e logo desapareceu entre as
ramagens dos arbustos. Pouco depois, extinguiu"-se também o tropel de
seus passos.

O jovem caçador não se admirou nem um pouco, apenas apressou as passadas para o
lado da Montanha das Runas, tudo o atraía naquela direção, as estrelas
como que lhe indicavam a trilha a seguir, a lua com um facho de
luminosidade mostrava as ruínas, nuvens resplandecentes migravam ao alto e do
fundo dos vales as águas e as florestas murmurantes lhe insuflavam
coragem. Era como se seus pés estivessem alados, seu peito palpitava,
ele sentia uma alegria tão intensa em seu íntimo, a ponto de
transformar"-se em angústia. Alcançou regiões onde jamais estivera
antes, os rochedos tornavam"-se mais escarpados, o verde se esmaeceu, as
paredes nuas o chamavam com vozes irritadiças, e um vento que talhava e
plangia o fustigava de frente. Prosseguia apressado e sem repouso, e
após meia"-noite chegou a uma estreita vereda que continuava bem rente a
um precipício.

Christian ignorou a profunda garganta escancarada que ameaçava
engoli"-lo, pois se sentia estimulado por fantasias confusas e desejos
incompreensíveis. Agora o arriscado caminho o conduzia ao longo de uma
elevada muralha que se erguia à altura das nuvens; o atalho se
estreitava cada vez mais e o jovem precisava se segurar em pedras
salientes a fim de não cair no imenso buraco. Finalmente, tornou"-se
impossível seguir adiante, a senda acabava sob uma janela, ele teve de
parar sem saber agora se retornava ou permanecia imóvel.

De súbito, ele viu uma luz que parecia se mover por trás da velha
murada. Fixou o olhar naquele brilho e percebeu distintamente um antigo
e amplo salão cintilante aos reflexos produzidos por uma extraordinária
tábua decorada com pedras e cristais diversos; esses reflexos se mexiam
e se confundiam de maneira enigmática conforme se movimentava a luz
portada por uma grande figura feminina que, imersa em reflexões,
deambulava indo e vindo pelo salão.

Ela aparentemente não pertencia à espécie mortal devido aos membros
fortes e ao semblante severo, todavia Christian estava encantado e
certo de nunca ter visto ou imaginado formosura semelhante. O jovem
tremeu e secretamente desejou que a mulher se aproximasse da janela e
percebesse sua presença. Enfim ela estacou, pousou a luz sobre uma mesa
de cristal, ergueu o olhar e cantou com voz penetrante:

\begin{verse}
Onde os anciões se atêm\\
Por que é que eles não vêm?\\
Vejo cristais pendentes\\
Das colunas de diamantes\\
Fluem lágrimas como fontes\\
De diáfana transparência.\\
Ondulações em fluência,\\
D'águas claras no interior\\
Nascendo no bojo em fulgor:\\
Que a alma aviva\\
E o coração cativa.

Almas aladas, venham!\\
Ao salão dourado, venham!\\
Ergam das trevas soturnas\\
Refulgentes as cabeças!\\
De seus corações e espíritos\\
Sedentos de nostalgia\\
Jorrem luminosas lágrimas\\
Cálidas, plenas de alegria!
\end{verse}

Pela diversidade de suas linhas, a tábua conformava um objeto bizarro e
místico; de vez em quando o jovem era dolorosamente ofuscado pelos
reflexos, mas em seguida as luzes verdes e azuis se mesclavam e
abrandavam novamente as sensações. Ele permanecia tragando com olhos
fixos todas as figurações que se apresentavam e ao mesmo tempo
profundamente ensimesmado na contemplação. Em seu coração abrira"-se uma
caverna de imagens e harmonias plenas de ânsias voluptuosas, alegres
tons alados e melodias tristes perpassavam sua alma infinitamente
comovida. Christian percebia crescer dentro de si um mundo de dor, a
esperança de miraculoso fortalecimento de fé e de altaneira consolação,
largas torrentes de água escorriam céleres transportando em seu curso o
peso da angústia.

Christian não reconhecia mais a si mesmo e assustou"-se assim que a bela
mulher abriu a janela e ofereceu"-lhe a mágica placa de pedras, dizendo
simplesmente: ``leve em minha memória''. Ao tocar o presente, ele sentiu
a diáfana figura entranhando em seu ser, a iluminação, a resplendente
beleza e o estranho salão tinham sumido. Algo como um véu de sombras
obscureceu seu espírito, ele tentou se apegar aos sentimentos
precedentes, aquele enlevo e o amor infindo, mirou a preciosa placa
onde se refletia a lua esmaecida e pálida.

Ainda segurava firmemente a placa em suas mãos crispadas,
quando rompeu a aurora, e então, exausto, exangue e meio adormecido,
ele desceu precipitado a íngreme montanha. Sobre o rapaz sonolento, 
o sol incidiu em pleno rosto, e despertando aos poucos ele se encontrou 
no ponto mais elevado de uma agradável colina.

Christian errou os olhos pelas imediações e reconheceu ao longe atrás de
si, quase invisíveis no horizonte, as ruínas da Montanha das Runas:
procurou a placa e não a encontrou. Admirado e atordoado, quis
concentrar"-se e repor o fio de suas lembranças, no entanto, sua memória
estava como que repleta de uma névoa espessa na qual se agitavam
caóticas e indistintas figuras amorfas. A vida inteira se abria às suas
costas como uma longínqua distância; o mais insólito e o mais banal se
misturavam intimamente e era impossível discerni"-los. Após uma
discussão consigo mesmo, acabou por crer que um sonho ou uma repentina
loucura o acometera na noite anterior. Entretanto, continuava a
incógnita sobre o fato de estar perdido naquela região desconhecida e
remota. Mal desperto ainda, desceu a colina e avistou uma trilha aberta
que o conduzia da elevada colina a uma ampla planície.

Tudo lhe era estranho, a princípio ele supusera que se aproximava de sua
terra natal, porém a região era extremamente diferente e, enfim,
concluiu que \mbox{deveria} estar além da fronteira meridional da cordilheira,
à qual ele ascendera na primavera pelo lado norte. Por volta do
meio"-dia, o rapaz se achou no meio de um vilarejo onde as cabanas
enviavam em direção ao céu uma fumaça tranquila, crianças brincavam na
relva verdejante de um adro solenemente decorado e dentro da capela
ressoava o som do órgão e as vozes dos fiéis.

Tudo em torno agora o enchia de suave emoção, de indescritível
melancolia, e ele não foi capaz de reprimir o choro. Os jardins
estreitos, as modestas cabanas com suas chaminés fumegantes, os campos
de grãos recentemente colhidos o lembravam das necessidades do pobre
gênero humano, da dependência da terra generosa, em cuja bondade é
mister confiar; ao mesmo tempo, o canto e a música do órgão envolveram
seu coração com um sentimento de piedade que ele até então não
experimentara. O incidente e os desejos noturnos se lhe afiguraram
nesse momento vis e pecaminosos, ansiou ardentemente aproximar"-se das
pessoas, abraçá"-las como irmãos num gesto puro e humilde, afastando"-se
dos sentimentos e intenções ímpios.

Atraente e sedutora lhe parecia a planície com o estreito regato que em
meandros diversos se insinuava por entre campos e jardins; com temor
ele lembrou a estada nos confins da montanha entre pedras solitárias,
almejou poder residir naquele pacato vilarejo e imbuído desse anelo
adentrou a igreja abarrotada.

O cântico terminara e o padre introduzira o sermão sobre as benesses
divinas na colheita: como sua bondade propiciava alimento suficiente
aos seres vivos, como os cereais proviam a conservação do gênero
\mbox{humano}, como o amor de Deus se manifesta sem cessar no milagre do pão e
o cristão piedoso pode, imbuído do fervor, celebrar perpetuamente a
ceia sagrada. Os fiéis estavam recolhidos em oração, os olhos do
caçador pousavam fixos no orador religioso, e atinaram bem ao lado do
púlpito para uma moça jovem que mais que toda a assembleia se devotava
à meditação.

Ela era esbelta e loira, em seus olhos azuis fulgurava a suavidade mais
cativante, e no semblante floresciam translúcidos os mais delicados
tons da tez. O jovem forasteiro nunca vivenciara semelhante experiência
em seu coração, nunca se sentira pleno de amor e, desse modo, inebriado
e entregue aos sentimentos mais doces e reconfortantes. Christian se
inclinou chorando quando o padre finalmente deu a bênção; ouvindo as
palavras sagradas foi invadido por uma potência invisível, e a visão
noturna reprimida nas profundezas longínquas como um fantasma.

Christian deixou a capela, deteve"-se sob uma frondosa tília e deu graças
a Deus por intermédio de uma oração fervorosa, pelo fato de ter se
livrado, embora sem merecer, das malhas do espírito maligno.

O vilarejo celebrava naquele dia a festa da colheita, e todos os
habitantes do lugar mostravam"-se animados e bem"-humorados; as crianças
se alegravam com as danças e os bolos, na praça central circundada por
árvores jovens, os moços ultimavam os preparativos para a festa
outonal, os menestréis se encontravam a postos e afinavam os
instrumentos musicais.

Christian saiu ainda a vagar pelos campos, a fim de centrar seu espírito
e seguir o rumo dos próprios \mbox{pensamentos}, depois retornou ao vilarejo
para \mbox{comemorar} a festa em júbilo. A loira Elizabeth também estava
presente com seus pais, e o recém"-chegado se imiscuiu na folia jovial.
Enquanto a moça dançava, o jovem entabulou uma conversa com o pai, que
era arrendatário, um dos mais ricos moradores do lugar. A juventude e
os modos do rapaz, ao que tudo indicava, o vinham agradando e, assim, em
pouco tempo, eles entraram num acordo, e Christian foi admitido a
serviço do homem, como jardineiro. O rapaz aceitou a incumbência, pois
esperava que os conhecimentos e as atividades que tanto desprezara
talvez lhe pudessem ser úteis nas atuais circunstâncias.

Uma nova existência se descortinava para Christian. Ele alojou"-se na
propriedade do patrão, e passou a ser considerado membro da família;
com sua condição, mudaram também seus trajes. Ele era tão bom, amigável
e disposto para o batente, que logo angariou a simpatia de todos da
casa, principalmente da filha.

Sempre quando ele a via encaminhando"-se à missa aos domingos, corria a
lhe ofertar um pronto ramalhete de flores, pelo qual ela agradecia com
amabilidade tímida; ele sentia a falta de Elizabeth nos dias em que não
a via e então, à tarde, narrava"-lhe contos e casos engraçados. Eles
nutriam cada vez mais um forte apego mútuo e os pais, que vinham
acompanhando o relacionamento, aparentemente não se opunham, pois
Christian era o rapaz mais diligente e bem apessoado do povoado; eles
próprios tinham sido tomados de amor e de confiança por ele desde
o primeiro instante. Um ano mais tarde Christian e Elizabeth estavam
casados. Mais uma vez chegara o outono, as andorinhas e os pássaros
canoros retornavam à região, o jardim exalava sua belíssima florada, o
casamento foi comemorado com toda a alegria, o noivo e a noiva estavam
embriagados de felicidade.

Tarde da noite, a sós em seu quarto, a jovem esposa confessou ao marido:

-- Não, você não corresponde à imagem que outrora em sonhos me
encantou, e a quem eu nunca esqueci, contudo sou feliz em sua companhia
e bem"-aventurada em seus braços.

Que alegria para a família, quando ao cabo de um ano se viu aumentada
com o nascimento da pequena filhinha que recebeu o nome Leonora. Por
certo Christian se tornava às vezes bem sério observando a criança, mas
sempre readquiria em seguida sua habitual jovialidade. Ele não pensava
mais sobre sua existência anterior, pois se sentia aclimatado e
satisfeito. Depois de alguns meses, entretanto, a imagem dos pais lhe
assomava com frequência à lembrança, ele pensava no orgulho de ambos,
sobretudo do pai, pelo seu plácido destino, sua condição de jardineiro
e cidadão. Inquietou"-se por ter abandonado completamente pai e mãe
durante tanto tempo, o próprio bebê o fazia lembrar"-se da alegria que
os filhos representam aos pais, e em decorrência, resolveu enfim pôr"-se
a caminho a fim de visitar a terra natal.

Ressentido, ele deixou sua esposa; todos lhe desejaram boa sorte, e na
bela estação ele iniciou a pé a peregrinação. Em poucas horas, percebeu
como a separação dos entes queridos lhe era dolorosa, pela primeira vez
na vida sentia a dor da separação. A paisagem estranha lhe parecia
quase agressiva, teve a impressão de estar perdido em meio à solidão
hostil. Então lhe sobrevinha a consciência de que a juventude passara,
ele encontrara uma pátria à qual pertencia e onde seu coração deitara
raízes; esteve a ponto de lamentar a negligência dos anos precedentes.
Encontrava"-se prostrado de tristeza quando entrou num albergue de
vilarejo com a intenção de aí pernoitar. Ele não compreendia porque se
afastara de sua amável esposa e dos novos pais que conquistara; de
manhã, aborrecido e taciturno, pôs"-se novamente a caminho, a fim de
prosseguir viagem.

Sua angústia crescia à medida que se aproximava da montanha, as
longínquas ruínas já se delineavam visíveis e se destacavam cada vez
mais distintas, vários picos e elevações sobressaíam"-se dentre a névoa
azulada. Seu passo era hesitante, com frequência ele se detinha e se
admirava do medo e dos arrepios que o acometiam mais intensos a cada
passada.

-- Eu a conheço bem, loucura! -- exclamou Christian. -- Mas quero resistir
viril à sua fascinação perigosa. Elizabeth não é uma quimera, eu sei
que ela agora está pensando em mim, esperando meu retorno e contando
amorosamente as horas de minha ausência.

Não estou enxergando florestas com cabelos negros à minha frente? O
regato não está me espiando com olhos cintilantes? Os grandes maciços
montanhosos não estão caminhando ao meu encontro?

No momento em que dizia essas palavras ele fez menção de se jogar ao
solo para repousar sob uma árvore, e foi quando viu assentado à sombra
dela um \mbox{homem} velho que examinava uma flor com meticulosa atenção. Ora
a erguia contra a luz do sol, ora a protegia com a mão, contava suas
pétalas envidando visíveis esforços para imprimir exatamente seu
aspecto na memória.

Ao se aproximar, Christian achou a figura familiar e logo não teve mais
dúvidas de que o homem com a flor era seu pai. O moço precipitou"-se nos
braços do velho com a expressão da alegria mais ardente; o pai estava
contente, mas não surpreso por revê"-lo tão subitamente:

-- Você está vindo ao meu encontro, filho? -- perguntou o velho. -- Eu
sabia que nos encontraríamos em breve, contudo não podia imaginar que
teria essa alegria hoje mesmo!

-- Mas como o senhor previu minha chegada, pai?

-- Graças a essa flor -- respondeu o jardineiro. -- Desde que me entendo
por gente, venho sonhando poder vê"-la um dia. Nunca consegui meu
intento, porque é uma espécie muito rara e cresce somente em regiões
montanhosas. Eu me pus a procurá"-lo, depois da morte de sua mãe, pois a
solidão em casa me sufocava e angustiava deveras. Não sabia que rumo
tomar, finalmente, decidi cruzar através da montanha, por mais
triste que se sucedesse a viagem. Andando a caminho, procurei a flor, porém
não a achava em parte alguma. Eis que de repente me deparei com ela
aqui no ponto onde a formosa planície começa a se estender. Por isso,
eu deduzi então que o veria e, quem diria, a flor preciosa me predisse
o porvir.

Eles se abraçaram novamente, e Christian chorou a perda da mãe, mas o
pai tocou sua mão e propôs:

-- Partamos, a fim de perder logo de vista as sombras das montanhas.
Meu coração se oprime na presença das formações selvagens e escarpadas,
com as terríveis crateras e os regatos gementes. Retomemos a direção da
doce e familiar planície.

Os dois retomaram de volta o itinerário que Christian cumprira até ali,
e o rapaz mostrava"-se bastante satisfeito. Contou ao pai de sua alegria
recente, da criança e da nova terra. Sua própria história o inebriava e
enquanto contava tudo ao pai se convencia de que sua felicidade era
\mbox{completa}. Assim, entre assuntos tristes e histórias pitorescas chegaram
ao povoado. Todos exultaram ao ver a viagem finalizando tão cedo,
sobretudo Elizabeth. O velho pai veio morar juntamente com eles,
investiu seu modesto patrimônio no negócio; eles constituíam um círculo
familiar dos mais harmônicos e unidos. O campo prosperava, o gado
reproduzia, a casa de Christian converteu"-se em alguns anos numa das
mais consideráveis da localidade; além disso, ele logo se viu pai de
várias crianças.

Cinco anos se passaram, quando um viajante estrangeiro se deteve no
vilarejo e se estabeleceu na casa de Christian, pois era a de melhor
aspecto no lugar. Era um homem amável e loquaz, que fazia longas
digressões sobre suas viagens, brincava com as crianças, lhes oferecia
presentes e conquistou a simpatia de todos. Sentiu"-se bem na região, o
que o levou a decidir por permanecer uns dias naquelas paragens, mas os
dias se transformaram em semanas e, no final das contas, \mbox{meses}. Ninguém
se surpreendeu com aquela estada prolongada, pois todos tinham se
habituado a considerá"-lo membro da família. Christian apenas se quedava
uma vez ou outra pensativo, pois era como se outrora o viajante lhe
tivesse sido familiar, entretanto não conseguia se recordar de nenhuma
situação na qual poderia tê"-lo conhecido. Três meses mais tarde, enfim,
o forasteiro se despediu, dizendo:

-- Queridos amigos, um destino fascinante e incríveis pressentimentos
me atraem às regiões montanhosas próximas daqui, um encanto ao qual não
posso resistir. Eu os deixarei agora e não sei se retornarei ao
povoado. Trago uma elevada soma em dinheiro comigo, que ficará mais
segura em suas mãos, por isso lhes peço que o guardem. Se eu não voltar
dentro de um ano, então devem conservá"-lo para si, tomem"-no como
presente de agradecimento pela demonstração de amizade.

Em seguida o estrangeiro partiu e Christian guardou o dinheiro. Ele o
trancou com cautela, e de vez em quando, tomado de escrúpulo doentio,
ia conferir se não estava faltando um pouco, e fazia muito caso
daquilo.

-- Esse dinheiro poderia assegurar nossa felicidade, disse certa
ocasião ao pai. Se o forasteiro não retornar, nós e as crianças
estaremos todos providos no futuro.

Mas o velho aconselhou:

-- Deixe de pensar no ouro, filho. A felicidade não consiste nisso,
graças a Deus nada nos faltou até hoje. Abandone definitivamente as
ideias dessa natureza.

Durante a noite, Christian estava sempre se levantando a fim de acordar
os homens para o trabalho, bem como de aviar uma providência ou outra.
O pai se preocupava com aquele zelo excessivo que poderia
prejudicar"-lhe a juventude e a saúde, por isso, levantou"-se numa noite,
pensando em exortar o filho a limitar a dedicação. Para seu grande
espanto, porém, encontrou o jovem sentado à mesa num recanto mal
iluminado muito aplicado a contar as moedas de ouro.

Condoído, o velho disse:

-- Meu filho, será que você perdeu o juízo de vez? Será que o vil
metal só veio nos trazer desgraça? Tome tento, meu filho, caso
contrário o inimigo maligno vai lhe sugar o sangue e a vida.

-- É, eu mesmo não me entendo mais, não tenho sossego nem de dia, nem
à noite. Veja como o dinheiro fica me espiando de soslaio com esse
fulgor avermelhado se imiscuindo ao fundo do meu coração! Escute como
tine o sangue do ouro! Eu o percebo me chamando quando durmo, ouço
música ou o vento sopra, quando os passantes sussurram na rua. Quando o
sol brilha, tudo que vejo são dourados olhos ofuscantes, querendo
segredar"-me ao ouvido palavras de amor; e na penumbra da noite, da
mesma maneira, tenho de me levantar para satisfazer seu desejo de amor.
Então sinto como ele internamente geme e suspira de prazer ao toque de
meus dedos. Ele se torna mais rubro e maravilhoso de alegria: repare
você mesmo na pujança do encantamento.

Tremendo e banhado em lágrimas, o velho pai tomou o filho nos braços,
rezou, e disse em seguida:

-- Christel, você precisa se voltar mais às palavras de Deus, ir à
igreja com assiduidade e fervor, senão vai cair em perdição e se consumir na
mais decadente miséria!

As moedas foram novamente guardadas, e o velho se tranquilizou. Já
transcorrera mais de um ano desde a partida do estrangeiro, e ninguém
tivera notícia de seu paradeiro. O velho então cedeu aos rogos do filho
e o dinheiro foi investido em terrenos e outras operações. No vilarejo
a riqueza do jovem arrendatário provocou rumores e Christian parecia
extraordinariamente contente e realizado, de maneira que o pai estimou
vê"-lo bem"-sucedido: todo o medo da alma do pai se dissipou. Qual não
foi, portanto, seu assombro, quando numa tarde Elizabeth o chamou a um
canto e entre lágrimas confidenciou que não compreendia mais o
comportamento do marido: o homem falava coisas confusas num sono
atormentado, vagando pelo quarto sem se dar conta disso e contando
fabulações prodigiosas que provocavam nela calafrios de tanto horror. O
mais assustador na história toda, contudo, era a \mbox{jovialidade} que
Christian demonstrava o dia inteiro, rindo cínico e insolente com um
olhar ausente e alheio.

O pai se assustou e a esposa aflita continuou:

-- Sem cessar ele fala do forasteiro e insiste em afirmar que o
conhecera anteriormente, pois o sujeito na verdade é uma mulher de
beleza esplendente. Como se isso não bastasse, ele não quer sair ao
campo ou trabalhar no jardim, porque diz que ouve um dolente gemido
subterrâneo tão logo arranca uma raiz. Treme apavorado à visão de
plantas ou de ervas, como se fossem fantasmas.

-- Bondoso Deus! -- gritou o velho pai. -- Será que a cobiça maldita
dominou implacável o coração do meu filho, perdendo"-o para sempre? Terá
o coração deixado de ser humano para ser metal? Quando alguém não ama
mais uma flor é sinal de que perdeu o amor e a fé em Deus!

No dia seguinte, o pai fez uma caminhada com o filho e lhe repetiu as
aflições de Elizabeth. Aconselhou"-o à piedade, à devoção e à meditação
religiosa. Christian respondeu:

-- Eu o faria com muito gosto, pai, muitas vezes experimento um
profundo bem"-estar e tudo fica bem. Por longo tempo, anos a fio,
consigo esquecer a verdadeira feição de minha natureza e levo com
serenidade uma vida estrangeira. Mas então, de repente, qual uma nova
face da lua, surge em mim o astro regente que sou eu mesmo e reprime a
existência estrangeira. Eu poderia ser bem feliz, porém, certa vez,
sucedeu numa noite estranha, da minha mão gravar no fundo de minha alma
um sinal misterioso: com frequência a figura feiticeira dorme e
repousa, creio ter"-se esvaído completamente, todavia, como um veneno,
ela de súbito mana vibrante, pulsa e se difunde. Com isso, a figura
envolve meu pensamento e sentimento, transforma meu caráter, melhor
dizendo, traga tudo em torno. Do mesmo modo como o louco é tomado de
espanto ante a visão da água, e o veneno contido em seu corpo redobra o
violento efeito, algo semelhante ocorre comigo à visão de figuras
angulosas, linhas, raios, formas que estimulam e dão luz à feição
íntima encerrada em meu seio. Meu espírito e meu corpo padecem a
angústia. Tão logo a alma acolhe a angústia engendrada pelas imagens do
exterior, ela a enfrenta imediatamente em meio a um embate tormentoso
com a realidade, tentando recompor"-se e readquirir a paz.

-- É uma constelação infeliz -- comentou o velho pai -- essa que o
arranca de nosso meio. Você nasceu para a vida pacata, um temperamento
que se inclinava à calma e ao cultivo das plantas. A impaciência,
entretanto, o levou longe ao convívio com os rochedos escarpados, as
falésias cindidas; as formas angulosas perturbaram seu coração,
insidiando"-o à cobiça devastadora pelo metal. Antes você tivesse sempre
se preservado e se guardado da influência espetacular das montanhas,
era a maneira como eu tencionava educá"-lo, mas não era para ser assim.
A humildade e a paz de seu cândido espírito subsumiram ante a revolta,
a aspereza e a presunção.

-- Nada disso! -- disse Christian. -- Lembro"-me perfeitamente como foi uma
planta que primeiro me revelou o infortúnio de toda a terra, somente
desde então compreendo os suspiros e gemidos provindos de todas as
partes e perceptíveis pela natureza inteira, por menos que apure os
ouvidos: em plantas, ervas, em flores e árvores pulsa e palpita
dolorosamente uma única e grande ferida, elas são o cadáver de
magníficas pedras de outrora; oferecem a nossos olhos a mais
assustadora decomposição. Hoje eu entendo bem a mensagem que aquela
raiz me enviava através do lamento proveniente das profundezas; em sua
dor ela perdeu a consciência de si e me revelou tudo. Por isso toda a
flora vicejante se irrita contra mim e quer sacrificar"-me; quer apagar
a figura bem"-amada de meu coração e, a cada primavera tenta me cativar
com fisionomia cadavérica. Ilícito e pérfido o modo como ela se
apoderou de sua alma, meu pobre ancião, as plantas tomaram conta de seu
coração. Pergunte às plantas, você se surpreenderá ouvindo"-as falar.

O pai contemplou tristemente o filho e não encontrou argumentos para
contradizê"-lo. Ambos retornaram para a casa em silêncio, e o velho teve
também de se horrorizar ante a jovialidade de Christian, pois era a
expressão de uma natureza estranha, como se fora outra criatura
desajeitada e mal enjambrada atuando e agindo de dentro do moço com
autonomia.

Novamente se celebrava a festa da colheita, os fiéis se encaminhavam à
capela, Elizabeth e as crianças se vestiam, a fim de assistir à missa.
Christian também fazia os preparativos para acompanhá"-la, mas à porta
da igreja deu meia"-volta e mergulhado em seus pensamentos, saiu do
vilarejo. Assentou"-se no cimo da colina e contemplou as redondezas como
da primeira vez. Viu as chaminés fumegantes, ouviu o cântico e o som do
órgão, vindos de dentro da capela; crianças com roupas domingueiras
brincavam e corriam pela relva viçosa:

-- Como dissipei minha vida com um sonho! -- disse consigo mesmo. 
-- Já se passaram anos desde que desci pela primeira vez a serra ao encontro das
crianças; as que naquela ocasião faziam algazarra no adro, hoje estão
lá dentro compenetradas. Eu entrei juntamente com elas no edifício,
hoje, porém, Elizabeth não é mais uma moça na flor da idade, perdeu o
viço da juventude, não posso mais como antes buscar ardoroso o brilho
de seu olhar; negligenciei por descuido um bem elevado e eterno por um
perecível e efêmero.

Invadido pela nostalgia, Christian se dirigiu à floresta próxima e se
refugiou sob suas sombras mais espessas. Um silêncio lúgubre o
circundava, nenhuma brisa movimentava a folhagem. Nesse ínterim, ele
viu a distância um homem avançar em sua direção, e em seguida
reconheceu o forasteiro. Um frêmito perpassou"-lhe rapidamente, logo lhe
veio à mente que o outro poderia estar vindo reclamar a restituição do
dinheiro. Quando a pessoa foi chegando mais perto, Christian entendeu a
que ponto se equivocara, pois a silhueta que julgara perceber se
desfigurou em si mesma. Era uma mulher velha de extrema feiura que se
aproximava, vestida em trapos imundos, um lenço esfarrapado prendia
ralas cãs; ela mancava e se apoiava numa muleta. Com voz cavernosa,
interpelou o jovem, perguntando"-lhe nome e condição. Ele respondeu
detalhadamente, e por sua vez também quis saber:

-- Mas quem é você?

-- Me chamam de mulher da floresta, qualquer criancinha conhece minha
história. Você nunca ouviu falar de mim?

A essas últimas palavras ela deu meia"-volta e Christian acreditou
reconhecer entre as árvores o véu dourado, o andar altaneiro, o corpo e
a postura altivos. Quis alcançá"-la, mas seus olhos a perderam de vista
no lusco"-fusco.

Nesse instante, um objeto fulgente caído à relva atraiu seu olhar.
Abaixou"-se para apanhá"-lo e reviu a tábua encantada cravejada de pedras
preciosas multicores em estranhas formações, a mesma que ele perdera há
tantos anos. As figurações e as luzes coloridas \mbox{surtiram} efeito
poderoso sobre todos os seus sentidos. Ele apertou o objeto com força,
convencendo"-se de que o tinha de novo em mãos e se precipitou ligeiro
de volta ao vilarejo. Deparou"-se com o pai e exclamou:

-- Veja, pai, eis aquilo de que sempre lhe falava e julgava não ter
visto, senão em sonho! Eu o possuo agora, ele é real e verdadeiro.

O velho pai considerou o objeto longamente e, por fim, disse:

-- Meu filho, eu sinto um arrepio de terror no fundo do coração
reparando semelhante alinhamento de pedras. E tenho um pressentimento
sobre o desígnio da conjunção de tais luzes. Repare o brilho frio das
pedras, o olhar cruel que lançam faiscante, sedento de sangue como dos
rubros olhos do tigre. Atire para longe de si essa escrita que o faz
frio e cruel, talvez ainda petrifique seu coração:

\begin{verse}
As flores brotam viçosas\\
Após um sono florescem,\\
Tal qual criança em repouso\\
Com brando sorriso, amanhecem.

Alvor as cores matiza\\
Fulgura em íris a tez,\\
E as surpreende o astro\\
Com um beijo, volúpia e avidez. 

Sob beijos desfalecem\\
Penas e amores padecem\\
Independente das cores,\\
Entristecem, morrem de amores.

Suprema alegria à flor\\
Amofinar"-se de dor\\
Entregar"-se exangue à morte\\
Fenecer em doce sorte. 

Então perfumes exalam\\
D'almas castas em folia\\
E os zéfiros rodopiam\\
Vento em mistério e magia.

Humano, o amor audaz,\\
É sutil nos artifícios!\\ 
Inspira langor atroz\\
D'alma dourada delírios:\\
Desejo, saudade, paixões.
\end{verse}

-- Incomensuráveis e maravilhosos tesouros -- respondeu o filho -- devem
ainda existir no ventre da terra. Quem poderia achá"-los, recolhê"-los e
possuí"-los! Quem poderia abraçar a terra junto ao peito como uma esposa
querida aspirando que, cheia de angústia e paixão, de bom grado lhe
ofereça o que guarda de mais precioso em seu interior! A mulher da
floresta me chamou, partirei à sua procura. Perto daqui há um filão de
mina arruinado, cavado por um mineiro há séculos: talvez eu a encontre lá!

 Saiu às pressas. Inutilmente o velho pai se esforçou para retê"-lo,
pouco depois, todavia, perdeu o moço de vista. Após algumas horas, com
muita dificuldade, o velho chegou ao veio abandonado: viu as pegadas
dos pés impregnadas na areia ao vão da entrada, e voltou em prantos
para casa, convencido de que, dominado pela demência, o filho por ali
adentrara e afundara nos poços abismais de água acumulada.

Desde então o velho pai afligiu"-se sem cessar em torrentes de lágrimas.
O vilarejo inteiro pranteou o jovem arrendatário, Elizabeth ficou
inconsolável, as crianças lastimaram alto. Meio ano mais tarde o velho
pai faleceu, os próprios pais dela o seguiram pouco tempo depois, e ela
precisou tomar sozinha as rédeas do vultoso negócio. Graças ao volume
de trabalho pôde evadir"-se do padecimento; a educação dos filhos e a
supervisão dos domínios não lhe davam tempo para remoer"-se em dor e
aflição. Ao fim de dois anos, ela decidiu finalmente casar"-se e deu a
mão a um homem jovem e disposto, que a amara desde a juventude.

Não demorou muito e tudo passou a transformar"-se naquela casa. O gado
amofinou, valetes e servos os atraiçoaram, cabanas de provisões foram
destruídas pelo fogo, pessoas da cidade que lhes deviam dinheiro
fugiram sem pagar as dívidas. Em decorrência dos desacertos, o mestre
julgou apropriado vender alguns acres e campos, mas uma avalanche de
bancarrotas e ruínas os colocou em novos apuros. A única explicação
possível era que o dinheiro adquirido por meios prodigiosos buscava
fluir em fuga alucinada vazando por todas as sendas.

Nesse entretempo o número de crianças aumentava, Elizabeth e o marido no
desespero negligenciaram e se tornaram morosos. Ele procurou se
distrair e começou a beber habitual e forte vinho que o deixava
mal"-humorado e irritado, de maneira que Elizabeth chorou lágrimas
quentes e abundantes.

À medida que a sorte os abandonava, os amigos no vilarejo se esquivavam
de sua companhia, e sucedeu então que alguns anos mais tarde se viram
completamente isolados, sobrevivendo dia após dia em condições de
extrema penúria. Não lhes sobraram mais que minguadas cabras e uma
vaca, a qual Elizabeth vez ou outra guardava até mesmo junto das
crianças.

Certa feita, ocupada na pradaria com um trabalho e amamentando
um bebê, com Leonora a seu lado, ela divisou ao longe uma figura
esquisita subindo em sua direção. Era um homem vestido com um gibão em
frangalhos, descalço, o rosto moreno crestado pelo sol, que a barba
estropiada contribuía para desfigurar ainda mais; não trazia a cabeça
coberta, mas trançara ramos verdes por entre as madeixas, o que tornava
mais estranho e incompreensível seu aspecto selvagem. Sobre as costas
ele carregava num saco amarrado uma pesada carga, caminhava apoiando"-se 
numa jovem pinha. Ao chegar mais perto depositou o fardo no chão e
tomou fôlego arfando.

Ele desejou bom dia à mulher que estava assustada ante aquela visão
inusitada; a menina se aninhou junto da mãe. Depois de um instante de
repouso ele disse:

-- Estou retornando agora de uma árdua peregrinação pelas montanhas
mais selvagens da terra, mas em compensação trouxe comigo finalmente os
tesouros preciosos que só a imaginação pode conceber ou o coração 
desejar. Veja aqui, admire!

Com isso ele desamarrou o saco e despejou seu conteúdo: era um monte de
pedregulhos, entre os quais contavam grandes pedaços de quartzo, bem
como escórias.

-- Essas joias valiosas -- prosseguiu ele -- ainda nem foram
cinzeladas e polidas, por isso lhes falta no momento o olho e o olhar;
o fogo exterior com seu fulgor esconde"-se incrustado no cerne, mas basta
 martelar a fim de inspirar medo e nenhuma
dissimulação mais lhes servirá, e nisso é possível ver como são 
espíritos inocentes.

A essas palavras ele pegou uma pedra dura e a bateu com violência contra
outra, de modo a provocar faíscas.

-- Vocês viram a cintilação? -- perguntou eufórico. -- Essas pedras são
puro fogo e luz, iluminam as trevas com seus sorrisos, no entanto nunca
o fazem voluntariamente.

Depois disso, tornou a guardar tudo com cuidado no saco, que amarrou
fortemente com um cordão.

-- Eu a conheço bem, você é Elizabeth! -- disse ele tristemente.

A mulher se surpreendeu.

-- Como você sabe meu nome? -- perguntou trêmula de presságios.

-- Ah, Deus do céu! -- respondeu o pobre homem. \mbox{-- Eu} sou Christian, que
numa ocasião chegou aqui como caçador, você não está me reconhecendo?

Tomada de pavor e profunda piedade, ela não sabia o que falar.

Christian lançou"-se ao seu pescoço e a beijou.

-- Ah, Santo Deus! Meu marido está vindo! -- sobressaltou"-se.

-- Fique sossegada -- tranquilizou"-a --, eu estou praticamente morto; lá
na floresta me aguarda a bela, a exuberante, enfeitada com um véu
dourado. Ah! Aí está minha filha preferida, Leonora! Venha cá, minha
cara, coração, dê"-me um beijo, um único para que eu sinta ainda uma vez
sua boca sobre meus lábios, em seguida irei embora.

Leonora chorava; apertava"-se contra o corpo da mãe que entre soluços e
lágrimas a aproximou do viajante. Ele a puxou para si, cingiu"-a nos
braços e a estreitou junto ao peito. Depois partiu em silêncio, e ao
longe no meio da mata elas o viram conversar com a pavorosa mulher da
floresta.

-- O que há com vocês? -- perguntou o marido de Elizabeth ao encontrar
as duas debulhadas em lágrimas.

Ninguém quis responder.

Desde então, porém, o infeliz nunca mais foi visto.
\medskip

\hfill\textit{Tradução de Maria Aparecida Barbosa}

\chapter{Os elfos}


\textsc{-- Onde estará Marie,} nossa filha? -- perguntou o pai.

-- Ela está lá fora no gramado -- respondeu a mãe -- 
brincando com o filho de nosso vizinho.

-- Espero que não corram de lá e se percam -- disse o pai com
ansiedade --, eles são tão estouvados.

A mãe foi dar uma olhada nos pequenos e levar"-lhes o lanche da tarde.

-- Como está quente! -- disse o menino, enquanto a menininha
se servia avidamente das cerejas vermelhas.

-- Tenham cuidado, crianças -- disse a mãe --, e não se
afastem muito da casa e nem entrem na floresta. 
Eu e o pai vamos para a lavoura.

O jovem Andres respondeu:

-- Não vos preocupeis, pois temos medo da floresta e ficaremos aqui
perto da casa onde há pessoas ao redor.

A mãe entrou de volta e pouco depois retornou em companhia do pai. Eles
fecharam a casa e dirigiram"-se ao campo para acompanhar o trabalho dos
lavradores, e ao prado para ver como ia a colheita de feno. Sua casa
ficava sobre um pequeno morro verde, cercada por uma delgada paliçada
que circundava o pomar e o jardim florido. A aldeia estendia"-se um
pouco mais abaixo logo nas proximidades e, além dela, ficava o castelo
do Conde. Martin tinha arrendado do fidalgo a grande propriedade, e
vivia tranquilamente com a esposa e sua única filha. A cada ano
conseguia guardar algum dinheiro, o que o levava a ter planos de
tornar"-se um homem rico, já que trabalhava bastante, o solo era
produtivo e o conde não lhe cobrava em demasia.

Enquanto caminhava com a esposa pelos seus campos, alegremente lançou um
olhar ao redor e disse:

-- Como esta região é diferente, Brigitte, daquela onde vivíamos
antes. Aqui tudo é tão verde, a aldeia inteira está repleta de árvores
frutíferas, o chão é coberto de ervas vistosas e flores, todas as casas
são alegres e asseadas, os habitantes prósperos, e até tenho a
impressão de que as florestas aqui são mais garridas e o céu mais azul,
e, até onde a vista alcança, tudo o que se vê enche os olhos e o
coração de prazer e jovialidade, tamanha é a generosidade da natureza.

-- E basta atravessarmos o rio -- disse Brigitte -- e já
parece que estamos em outras terras, pois tudo lá é tão triste e árido.
Todos os viajantes também confirmam que nossa aldeia é, de longe, a
mais bonita em toda a redondeza.

-- Exceto aquele barranco coberto de abetos -- respondeu o
homem. -- Veja só como está tudo escuro e desolado nesse recanto
afastado e como ele destoa da risonha paisagem em volta: atrás dos
pinheiros escuros, uma cabana fuliginosa, as cocheiras arruinadas e o
riacho correndo melancolicamente.

-- É verdade -- dizia a mulher enquanto ambos se detinham e observavam 
o panorama. -- Basta alguém aproximar"-se daquele lugar e já é tomado
de tristeza e apreensão sem nem saber por quê. Quem são as pessoas que
vivem ali? E por que se mantêm apartadas de toda a comunidade como se
tivessem a consciência pesada?

-- Pobre populacho -- respondeu o jovem arrendatário. -- 
A julgar pelas aparências, devem ser ciganos que realizam roubos e
trapaças em outras cercanias e talvez aqui tenham seu esconderijo.
Surpreende"-me que o conde os tolere.

-- Também é possível -- afirmou a mulher com suavidade
-- que seja apenas gente pobre, com vergonha de sua pobreza, pois
nunca ouvi falar de qualquer mal que tivessem praticado. O que incomoda
é não comparecerem à igreja e ninguém saber de fato do que vivem, pois
não cultivam nenhuma lavoura e é impossível que tirem seu sustento de
sua pequena horta, totalmente abandonada.

-- Só Deus sabe -- continuou Martin enquanto voltavam a
caminhar -- a que tipo de negócio se dedicam, pois nenhuma pessoa
os visita, o lugar onde vivem parece enfeitiçado e banido,
e até os moleques mais curiosos não têm coragem de ir lá.

E assim foram conversando enquanto se dirigiam à lavoura. 

Aquela área sombria da qual falavam ficava em um lugar um pouco mais
retirado da aldeia. Em uma escarpa rodeada de abetos havia uma cabana e
várias construções quase em ruínas, sendo raro aparecer fumaça
subindo de lá, e mais raro ainda avistar alguém. De tempos em tempos,
alguém mais indiscreto ousava aproximar"-se um pouco mais, tendo
avistado no banco defronte à cabana algumas mulheres repulsivas em
trajes maltrapilhos, que carregavam no colo crianças \mbox{igualmente} feias e
sujas. Cachorros pretos rondavam o lugar; nas horas após o escurecer,
um homem mal"-encarado que ninguém conhecia, atravessava a ponte do
riacho e sumia cabana adentro; e mais tarde viam"-se, em torno de uma
fogueira, diferentes vultos que se moviam como sombras na escuridão.
Aquele declive, os pinheiros e a cabana cheia de avarias sem dúvida
contrastavam fortemente com a alegre paisagem verde, com as casas
brancas da aldeia e com o magnífico castelo novo, criando um efeito dos
mais estranhos.

As duas crianças tinham comido as frutas e agora começaram a apostar
corrida para ver quem chegava primeiro, brincadeira em que a ágil e
pequena Marie sempre conseguia compensar a dianteira de Andres, que era
mais lento.

-- Assim não é possível! -- gritou ele por fim. \mbox{-- Vamos}
tentar uma corrida para mais longe; então veremos quem ganha!

-- Como quiseres -- disse a menina --, mas não podemos
correr até as águas.

-- Não -- respondeu Andres --, mas naquela colina, a um
quarto de hora daqui, está uma grande pereira. Vou correr por aqui,
contornando pela esquerda o barranco de abetos; tu podes ir pelo campo
à direita, e assim apenas nos reencontramos quando já estivermos lá em
cima. Desse modo veremos quem é o melhor.

-- Certo, -- disse Marie, já começando a correr -- assim
também não nos estorvamos indo pelo mesmo caminho, e o pai sempre diz
que a distância até a colina é igual, não importando se circundamos a
morada dos ciganos por esta orla ou por aquela.

Andres já havia saído em disparada e Marie, que se dirigira para a
direita, não conseguia mais avistá"-lo.

``Como ele é tolo'', disse consigo, ``bastaria eu criar coragem e 
atravessar a ponte, passar pela cabana e sair do lado de lá, e assim 
eu certamente chegaria muito antes dele''.

E no mesmo instante ela já estava em frente ao riacho e ao declive de
abetos.

-- Será que devo ir? Não, é horrível demais -- disse ela.

Do lado de lá havia um cãozinho branco que começou a latir com todas as
suas forças. Por causa do susto que levara, o animal lhe pareceu um
monstro, e ela saltou para trás.

-- Ai -- ela disse --, agora o moleque com certeza já
ganhou uma grande dianteira, e isso porque estou parada aqui sem 
me decidir.

O cachorrinho continuava latindo. Ao observá"-lo com mais atenção, ele
não mais lhe pareceu horrível mas, ao contrário, um bichinho muito
mimoso: usava coleira vermelha, com um sino brilhante, e
sempre que erguia a cabeça ou se movia ao latir, o sino soava de modo
encantador.

-- Preciso arriscar! -- exclamou a pequena Marie. -- Vou
correr o mais ligeiro que puder, e então, rápido, rápido, já sairei do
outro lado. Eles certamente não irão me engolir inteira!

Assim dizendo, a vivaz e corajosa criança saltou sobre a ponte, passou
correndo pelo cachorrinho que ficou quieto e abanou a cauda, e logo
tinha chegado ao fundo. Ali os abetos escuros ocultavam a visão da casa
de seus pais e da paisagem circundante.

Quão surpresa ela ficou. Estava em meio a um jardim com flores coloridas
e alegres. Tulipas, rosas e lírios brilhavam com as cores mais
esplêndidas; borboletas azuis e vermelho"-douradas balançavam"-se nas
flores; em gaiolas de arame lustroso pendiam das treliças pássaros
multicores que entoavam canções adoráveis; crianças em túnicas
brancas e curtas, os cabelos amarelos em cachos e olhos claros,
saltavam de um lado a outro; algumas brincavam com pequenos cordeiros,
outras alimentavam os pássaros ou colhiam flores e com elas se
presenteavam mutuamente, ainda outras comiam cerejas, uvas e damascos
avermelhados. Nenhuma cabana estava à vista, havendo, ao contrário, no
centro da área uma casa bela e espaçosa com uma porta de bronze e
cercada de altivas estátuas. A surpresa de Marie era tão grande que a
deixou desnorteada. Mas como não era acanhada, logo foi até onde estava
a primeira criança, ofereceu"-lhe a mão e desejou"-lhe dia bom. 

-- Então finalmente vieste nos visitar? -- disse a
esplendorosa criança. -- Eu te vi correndo e saltando lá fora e o
medo que tiveste de nosso cachorrinho.

-- Vós absolutamente não pareceis ciganos e salafrários -- 
afirmou Marie --, ao contrário do que Andres sempre dizia. 
Ele não passa de um ignorante que fica falando pelos cotovelos.

-- Fica conosco -- disse a fabulosa pequena -- e decerto
irás te divertir muito.

-- Mas apostei uma corrida com Andres.

-- Irás retornar para junto dele cedo o suficiente. Toma e prova isto!

Marie comeu, e as frutas lhe pareceram tão doces como nunca havia
experimentado, e Andres, a corrida e a proibição de seus pais foram
totalmente esquecidos.

Uma mulher de elevada estatura, usando um vestido fulgurante,
aproximou"-se delas, e perguntou sobre a criança estrangeira.

-- Formosa dama, -- disse Marie -- cheguei aqui por
acaso, e cá estão pedindo"-me para ficar um pouco.

-- Tu sabes, Zerina -- disse a bela --, que isso pode ser
concedido apenas por um tempo muito breve. E ainda assim deverias
ter"-me perguntado primeiro.

-- Eu pensei -- disse a criança reluzente -- que, como já lhe fora
permitido atravessar a ponte, eu poderia fazê"-lo. Além disso, também a
vimos muitas vezes correndo pelo campo e tu mesma te agradaste de sua
vivacidade. E ela decerto terá de nos deixar em breve.

-- Não, quero permanecer aqui -- afirmou a estrangeira --, pois aqui é
tão lindo, e aqui estão as melhores brincadeiras e também morangos e
cerejas. Lá fora as coisas não são tão primorosas.

A mulher vestida de dourado afastou"-se com um sorriso, e então muitas
das crianças saltaram aos risos em volta da entusiasmada Marie,
gracejaram e convidaram"-na a dançar, outras traziam cordeiros ou
brinquedos maravilhosos, outras cantavam enquanto tocavam instrumentos
musicais. Ela preferiu, no entanto, ficar junto à companheira que
primeiro tinha vindo ao seu encontro, pois era a mais gentil e afetuosa
de todas. A pequena Marie repetiu muitas vezes:

-- Quero sempre permanecer convosco e que sejais minhas irmãs.

O que levava todas as crianças a rir e a abraçá"-la.

-- Agora vamos brincar de um jogo muito bonito -- Zerina disse.

Ela foi rapidamente até o palácio e retornou trazendo uma caixinha
dourada contendo um pólen cintilante. Com os pequenos dedos apanhou um
pouco do pó e espalhou alguns grãos sobre o chão verde. Imediatamente
lá estavam relvas tão espessas que ondulavam rumorejantes; poucos
momentos depois emergiram da terra roseiras deslumbrantes, que
cresceram velozes e se encheram de botões e espalharam um doce aroma
por todo o lugar. Também Marie pegou grãos de pó e, depois de
espalhá"-los, viu \mbox{surgirem} lírios brancos e cravos nas mais variadas
cores. A um sinal de Zerina, as flores desapareceram novamente e outras
apareceram em seu lugar.

-- Agora -- falou Zerina -- prepara"-te para algo maior.

Ela colocou duas sementes de pinheiro no chão, cobrindo"-as de terra com
os pés. Dois arbustos verdes estavam já diante delas.

-- Segura"-te em mim com força -- ela disse, e Marie cingiu os braços
em torno do seu delgado corpo.

Então sentiu"-se erguida para cima, pois as árvores cresceram debaixo
delas com enorme velocidade. Os altos pinheiros moviam"-se e as duas
crianças seguravam"-se e trocavam beijos e abraços enquanto flutuavam
para lá e para cá nas nuvens vermelhas do entardecer. Os demais
pequenos escalavam com grande \mbox{agilidade} os troncos das árvores para
cima e para abaixo, empurrando"-se e brincando às gargalhadas quando se
encontravam. Se na agitação uma das crianças despencasse lá de cima,
ela voava pelos ares, abaixando lentamente e assentando"-se sobre a
terra em segurança. Por fim, Marie sentiu medo; então a outra menina
emitiu alguns sons, e as árvores afundaram novamente no chão com a
mesma rapidez com que haviam subido até as nuvens.

Atravessaram a porta de bronze do palácio. Em um salão redondo estavam
sentadas por toda parte muitas mulheres formosas, umas mais jovens e
outras mais velhas, que saboreavam as mais doces frutas ao som
encantador de uma música invisível. Na abóbada do teto estavam pintadas
palmas, flores e folhagens pelas quais galgavam figuras infantis,
balançando"-se e assumindo posições graciosas. Conforme os sons da
música, as imagens iam mudando e tingindo"-se com as mais picantes 
cores: ora o verde e o azul ardiam em faíscas flamejantes, ora a luz
dessas cores empalidecia"-se retornando ao aspecto anterior; a púrpura
chamejava e o ouro ateava"-se em fogo; então parecia que as crianças
nuas nas guirlandas de flores realmente estavam vivas e inspiravam e
soltavam o ar pelos lábios de rubi, e que de tempos em tempos ficava à
mostra o brilho dos dentinhos alvos e os olhos azuis celestes piscavam.

Degraus de bronze conduziam do salão até uma vasta câmara subterrânea.
Ali havia grandes montes de ouro e prata, em meio aos quais refulgiam
pedras preciosas de todas as cores. Junto às paredes ficavam
recipientes fabulosos que pareciam repletos de tesouros. O ouro estava
trabalhado em múltiplas formas e tremeluzia em amistoso carmim. 
Muitos anõezinhos ocupavam"-se separando as peças e
colocando"-as nos recipientes; outros, corcundas e de pernas tortas, com
longos narizes rubros, carregavam pesados sacos, dobrando"-se sob sua
carga, tal como moleiros ao peso do cereal, e despejavam arquejantes os
grãos de ouro no chão. Então saltavam desajeitados a torto e a direito,
lançando"-se atrás das esferas douradas que rolavam para todos os lados,
prestes a se esconderem; e não era raro que o zelo dos anões fizesse
uns derrubarem os outros, o que os levava a cair desengonçados no chão. 
Quando Marie riu de seus métodos e de sua feiura, fizeram caras rabugentas 
e lançaram olhares mal"-humorados. Ao fundo, estava sentado um
homenzinho velho e encurvado, a quem Zerina cumprimentou com grande
deferência, e que apenas agradeceu com um grave aceno de cabeça.
Segurava um cetro na mão e usava uma coroa sobre a cabeça, e todos os
demais anões pareciam reconhecê"-lo como seu senhor e obedecer aos
sinais de sua mão.

-- O que houve desta vez? -- perguntou com aspecto ranzinza
quando as crianças chegaram um pouco mais perto. 

Marie estava com medo e permaneceu calada, mas sua companheira respondeu
que elas apenas tinham vindo para dar uma olhada nas câmaras.

-- A infantilidade de sempre! -- disse o ancião. -- Esse ócio nunca
termina?

Em seguida, voltou"-se novamente a suas atividades, mandando pesar o
ouro e selecionar as peças, \mbox{atribuindo} incumbências a uns anões e
repreendendo asperamente outros.

-- Quem é esse senhor? -- perguntou Marie.

-- Nosso Príncipe dos Metais -- respondeu a menina enquanto saíam dali.

Pareciam estar novamente ao ar livre, pois se encontravam junto a um
grande lago, contudo, não havia sol e não podiam ver o céu acima de
suas cabeças. Um pequeno barco as recebeu, e Zerina remou com muito
afinco. A viagem foi rápida. Quando chegaram ao centro da lagoa, Marie
viu que milhares de córregos, regatos e canais saíam do lago
espalhando"-se em todas as direções.

-- Estas águas à direita -- disse a brilhante menina -- correm debaixo
de vosso jardim; por isso tudo ali floresce com tanto viço. Por aqui se
chega ao grande rio caudaloso.

Subitamente, emergindo de todos os canais e do lago, inúmeras crianças
vinham nadando. Muitas traziam grinaldas de lírios aquáticos e juncos,
outras carregavam enfeites de coral vermelho, e ainda outras sopravam
em conchas retorcidas. Um rumor confuso soava alegremente nas margens
escuras. Entre os pequenos nadavam formosas mulheres e muitas vezes
algumas crianças saltavam em direção a uma ou a outra, dependurando"-se
aos beijos a seus pescoços e nucas. Todos cumprimentavam a forasteira.
Após atravessarem aquela balbúrdia, elas foram se afastando e
enveredaram por um pequeno riacho, que ia se tornando mais e mais
estreito. Finalmente, o bote parou. Acenaram em despedida e Zerina
bateu no rochedo. Como se fosse uma porta, ele se entreabriu, e uma
figura feminina muito rosada ajudou"-as a desembarcar. Zerina perguntou:

-- Estão todos bem animados e vigorosos?

-- Estão em plena atividade -- respondeu aquela -- e tão bem"-dispostos
como é de esperar, em especial porque o calor está realmente muito agradável. 

Elas subiram por uma escada em espiral, e de repente Marie viu"-se em um
salão exuberante, tão iluminado que, ao entrar nele, seus
olhos ficaram ofuscados pelo clarão da luz. Alfombras de cor escarlate
cobriam as paredes de brasas purpúreas e, quando seus olhos se
habituaram, vislumbrou com enorme espanto que na tapeçaria havia imagens
dançando jubilosas e movendo"-se para cima e para baixo, as quais tinham
formas tão belas e eram tão graciosas que não poderia haver nada mais
formoso: o corpo delas parecia de cristal avermelhado, dando a impressão de
deixar entrever o sangue correndo e fluindo por ele. Aqueles seres riam
para a criança estrangeira e cumprimentavam"-na com diferentes
inflexões, mas quando Marie quis aproximar"-se, Zerina de repente a
puxou com força e exclamou:

-- Tu te queimarás, Marie, pois é tudo fogo!

Marie sentiu o calor.

-- Por que -- perguntou ela -- essas adoráveis criaturas não saem de lá
e se juntam a nós para brincar?

-- Assim como tu vives no ar -- respondeu aquela \mbox{--,} elas precisam
sempre permanecer no fogo e aqui fora acabariam perecendo. Olha como
estão bem ali, como riem e soltam gritos de prazer. Aquelas lá embaixo
são as que espalham rios de fogo por todos os recantos abaixo da
terra, fazendo crescerem as flores, os frutos e o vinho. Esses córregos
tintos acompanham as águas dos riachos, e assim as criaturas ígneas
sempre têm bastante a fazer e ficam satisfeitas. Mas para ti está muito
quente aqui, voltemos ao jardim.

Por ali a paisagem tinha"-se transformado. O luar derramava"-se sobre
todas as flores, os pássaros estavam silenciosos e as crianças dormiam
em grupos variados nos caramanchões verdes. Marie e sua amiga, porém,
não sentiam cansaço e preferiram perambular pela noite quente  de verão
até o amanhecer, conversando sobre as mais variadas coisas. 

Quando nasceu o dia, saciaram"-se com frutas e leite, e Marie disse:

-- Proponho agora algo diferente: que saiamos até os pinheiros e
vejamos como está lá agora.

-- Com prazer -- disse Zerina. -- Então poderás conhecer nossos
guardas, que certamente irão agradar"-te, e que ficam postados sobre as
muralhas entre as árvores.

Foram caminhando pelos jardins floridos, pelos garbosos arvoredos repletos
de rouxinóis, e depois sobre colinas com videiras. Finalmente, após
acompanharem por longo tempo as curvas de um límpido riacho, chegaram
aos abetos e ao declive que demarcava os limites da área.

-- Como é possível -- perguntou Marie -- que tenhamos um percurso tão
longo aqui dentro se o círculo lá fora é tão pequeno?

-- Não sei o motivo -- respondeu a amiga --, mas é assim que as coisas são.

Subiram até onde estavam os abetos sombrios, um vento frio que vinha de
fora do barranco soprou em sua direção, e parecia haver uma névoa
cobrindo a paisagem em torno por uma longa extensão. No alto achavam"-se
postados vultos de aparência extravagante, cujas faces cobertas de pó
farinhento faziam lembrar das horripilantes cabeças de corujas brancas.
Estavam envoltos com casacos peludos de lã e seguravam acima das
cabeças guarda"-sóis revestidos de peles estranhas. Com asas de morcego,
que contrastavam de modo curioso com os roclós, eles farfalhavam e
esvoaçavam sem parar.

-- Estou com vontade de rir e contudo sinto horror -- disse Marie.

-- Estes são os nossos bons e dedicados guardas -- explicou sua
pequena companheira. -- Eles ficam aqui a postos, fazendo soprar um golpe
de ar frio que inspira medo e inexplicável aflição em quem quer que tente
chegar perto de nós. Estão no momento cobertos dessa maneira, porque
está chovendo e fazendo frio lá fora, o que não conseguem suportar. Lá
embaixo nunca chega neve e vento, nem o ar frio, de modo que vivemos em
eterna primavera e verão; mas se aqui em cima os guardas não fossem
constantemente substituídos, não conseguiriam resistir.

-- Mas, afinal, quem sois? -- perguntou Marie enquanto desciam de volta 
ao local de onde emanava o aroma das flores. -- Ou será possível que vós não
tendes um nome, pelo qual possam ser conhecidos? 

-- Somos chamados de elfos -- disse a gentil criança. -- E no mundo contam"-se 
histórias sobre nós, eu própria já as ouvi.

Sobre a planície elevou"-se uma grande balbúrdia.

-- A ave formosa acaba de chegar! -- clamavam as crianças, e todos
acorriam ao salão.

Já de longe viam como na soleira se apinhavam jovens e idosos, todos
exultantes, e de dentro soava uma música festiva. Entraram e viram a
grande circunferência repleta de silhuetas das mais variadas, e todos
acompanhavam com o olhar um grande pássaro de resplandecente plumagem
que lentamente voava em círculos rente à abóbada. A música ecoava
ainda mais alegre do que o habitual, as cores e luzes alteravam"-se
com maior rapidez. Por fim, a música silenciou, e o pássaro foi pousar
sobre uma suntuosa coroa que flutuava debaixo de uma janela que de cima
iluminava a abóbada. A plumagem da ave era púrpura e verde, sendo
trespassada por luzentes listras douradas; sobre sua cabeça havia um
diadema formado de diminutas penas tão radiantes que brilhavam como
pedras preciosas. O bico era vermelho e as pernas azuis cintilantes.
Conforme se movia, todas as cores piscavam embaralhando"-se, fascinando
quem o olhava. Seu tamanho era o de uma águia. Mas logo ele abriu o
bico reluzente e uma melodia extremamente doce e cheia de emoção brotou
de seu peito, em tons mais belos do que os de um rouxinol inebriado de
amor. Mais poderoso tornou"-se seu canto, derramando"-se para todos os
lados como raios de luz, fazendo com que todos, inclusive as próprias
crianças em idade mais tenra, chorassem de alegria e encanto. Quando
terminou, todos se curvaram diante dele. Voou novamente em círculos
pelo salão, arrojou"-se pela porta afora e ruflou contra a claridade do
céu, onde logo se tornou um brilhante ponto vermelho. Por fim, sumiu.

-- Por que todos vós estais tão contentes? -- perguntou Marie e
inclinou"-se em direção da bela criança que hoje lhe parecia ser menor
do que ontem.

-- O rei está chegando! -- disse a pequena. -- Muitos de nós ainda
não o viram; e para onde quer que vá, ele leva alegria e bem"-aventurança. Há
tempos tínhamos a esperança de que viesse. Ansiávamos com fervor ainda
maior do que o vosso, quando, após um longo inverno, aguardais a primavera;
e agora ele nos anunciou sua chegada por meio daquele belo mensageiro.
Este pássaro magnífico e inteligente, que serve ao rei como emissário,
é chamado de fênix. Ele vive longe, na Arábia, sobre uma árvore que só
existe uma única vez no mundo, assim como também não há uma segunda
fênix. Quando sente que está envelhecido, reúne bálsamo e incenso e
forma com eles um ninho, incendeia"-o e nele queima a si mesmo. Ele vai
entoando um canto até morrer, e das perfumadas cinzas ergue"-se a fênix
rejuvenescida e com renovada beleza. É raro ele voar onde possa
ser visto, e quando isso acontece, de séculos em séculos, elas
anotam o sucedido em livros de memórias, e ficam no aguardo de algum
evento fabuloso. Mas agora, minha amiga, é chegada a hora de partires,
pois não te é permitido avistar o rei.

Por entre a multidão vinha se aproximando a formosa mulher vestida de
dourado. Com um aceno, \mbox{chamou} Marie para junto de si e levou"-a até um
alpendre isolado.

-- Tens de nos deixar, criança amada -- disse ela. \mbox{-- O rei} deseja
estabelecer sua corte aqui por vinte anos, talvez mais. De agora em
diante, abundância e prosperidade irão espalhar"-se sobre uma vasta
paisagem ao redor, mas em especial aqui nas imediações. Todas as fontes
e riachos tornar"-se"-ão mais generosos, mais fecundos os campos e jardins, 
o vinho mais nobre, os prados mais espessos e a floresta mais
verdejante e fresca. O sopro da brisa será mais suave, nenhum granizo
causará danos, não haverá ameaça de inundação. Toma este anel e
lembra"-te de nós. Mas atenção: jamais conta a ninguém sobre nossa
existência, caso contrário, seremos obrigados a fugir daqui. E então
todos à volta, assim como tu, seriam privados da fertilidade e
bem"-aventurança que propiciamos. Beija mais uma vez a tua companheira
de brinquedos e diga adeus.

Elas saíram do alpendre. Zerina chorou, Marie curvou"-se para abraçá"-la,
e elas se separaram. Imediatamente Marie já se encontrou sobre a ponte
estreita, um ar gélido soprava atrás dela vindo dos abetos, o cãozinho
latiu carinhosamente fazendo soar seu sininho. Ela ainda olhou para
trás, mas logo correu para o espaço aberto, pois a escuridão dos
pinheiros, a fuligem da cabana em ruínas e as sombras do crepúsculo
infundiram"-lhe temor e insegurança. 

``Como meus pais devem ter se preocupado comigo nesta noite!'', 
disse a si mesma quando alcançou a lavoura. ``E eu sequer posso
contar"-lhes onde estive e o que avistei; e, a bem da verdade, 
jamais acreditariam em mim.''

Dois homens passaram por ela e a cumprimentaram, e ela ouviu 
como um deles dizia:

-- Que formosa mocinha! De onde ela será?

Com passos mais apressados foi aproximando"-se da casa dos pais, mas as
árvores, ontem carregadas de frutos, hoje estavam estéreis e sem
folhas, a casa havia sido pintada de maneira diferente, e um novo
celeiro fora construído ao lado dela. Marie estava tomada de assombro,
acreditando que era tudo um sonho. Confusa, abriu a porta da casa;
junto à mesa estava sentado seu pai, entre uma mulher desconhecida e um
rapaz estranho.

-- Meu Deus, pai! -- exclamou ela. -- Onde está a mãe?

-- Mãe? -- disse a mulher tomada de pressentimento, e ergueu"-se de
forma brusca. -- Ai! Será que tu és\ldots{} Sim, sim, com certeza és Marie,
nossa única e amada Marie, que se perdera e que julgávamos morta!

Ela imediatamente a havia reconhecido por um pequeno sinal marrom sob o
queixo, pelos olhos e pelo seu porte. Todos a abraçaram, todos estavam
comovidos e felizes, e os pais derramaram lágrimas. Marie ficou
surpreendida por quase alcançar a mesma estatura do pai, não podia
entender como a mãe pudera mudar e envelhecer tanto, e perguntou o nome
do rapaz.

-- Ora, este é o filho de nosso vizinho, Andres -- disse Martin. --
Como afinal retornaste de forma tão inesperada após sete longos anos?
Onde estiveste? Por que nunca mandaste notícias?

-- Sete anos? -- perguntou Marie, sem conseguir ordenar suas ideias e
memória -- Sete anos inteiros?

-- Sim, isso mesmo! -- disse Andres rindo e
apertou"-lhe a mão calorosamente. -- Eu ganhei a aposta, querida Marie: há
sete anos alcancei primeiro a pereira e já retornei para cá, enquanto
tu, que és a mais lenta, só chegaste hoje!

Continuaram fazendo"-lhe perguntas e insistindo para que contasse o que
houve, mas ela, sabendo da proibição, não pôde responder. Aos poucos,
eles próprios foram moldando e colocando em sua boca a história de que
ela se perdera, subira em uma carroça que passava, fora levada a um
lugar desconhecido, não pudera informar às pessoas de lá onde seus pais
moravam, pouco depois fora conduzida a uma cidade muito distante em que
gente bondosa a criou e amou; e, tendo essas pessoas agora morrido, ela
finalmente voltou a pensar em sua terra natal, tinha aproveitado uma
oportunidade de viajar e assim conseguido retornar.

-- Não importa o que aconteceu! -- exclamou a mãe. -- Importa apenas que
a tenhamos de volta, minha filhinha. Minha única filha, que é tudo para mim!

Andres ficou até a ceia, e Marie ainda estava desnorteada. A casa
parecia"-lhe pequena e escura, suas roupas eram asseadas e simples, mas
pareciam"-lhe totalmente estranhas. Ela olhou para o anel de ouro em seu
dedo, que faiscava de maneira fabulosa e trazia engastada uma pedra
rubra como brasa. Quanto o pai lhe perguntou, respondeu que o anel
também fora um presente de seus benfeitores.

Ansiava pelo sono e logo foi recolher"-se. Na manhã seguinte, sentiu"-se
mais serena, suas ideias estavam mais bem ordenadas, e agora podia
conversar e dar respostas ao povo da aldeia, pois todos vieram
cumprimentá"-la. Andres já estava de volta logo que os primeiros
chegaram, e mostrava"-se muito atencioso, feliz e prestativo. A menina
em flor, com seus quinze anos, tinha"-lhe causado uma profunda
impressão, e ele passara a noite em claro. O conde chamou Marie ao
castelo, onde ela mais uma vez repetiu sua história, que agora já lhe
era corrente. O velho senhor e sua cortês esposa admiraram sua boa
educação, pois ela era modesta sem ser acanhada, respondia a todas as
perguntas com polidez e sabia expressar"-se muito bem. Ela perdera o
receio diante de pessoas e ambientes requintados, pois quando comparava
estas salas e figuras com as maravilhas e a sobranceira beleza que
presenciara na residência secreta dos elfos, o esplendor terreno
afigurava"-se"-lhe opaco e as pessoas quase pequenas. Os cavalheiros
jovens ficaram especialmente encantados por sua beleza. 

Era fevereiro. As árvores cobriram"-se de folhagem mais cedo do que nunca, 
o rouxinol apareceu com uma precisão inusitada, a primavera
esparramou"-se pelas terras com um vigor como até os mais idosos ainda
não tinham visto. Por toda parte surgiam pequenos arroios que regavam
os campos e pradarias; as colinas pareciam estar ficando mais elevadas,
as videiras cresciam mais fortes, as árvores frutíferas floresceram
como jamais o tinham feito, e uma bênção intumescida de aromas pendia
sobre a paisagem em uma pesada nuvem \mbox{florida}. Tudo germinava além do
esperado, nenhum dia era penoso, nenhuma tempestade danificava as
frutas, o vinho avolumava"-se enrubescido em enormes cachos de uvas, e
os habitantes da aldeia manifestavam uns aos outros sua estupefação e
sentiam"-se como envoltos por um doce sonho. O ano seguinte foi igual,
mas todos já estavam mais habituados aos prodígios. No outono Marie
aquiesceu aos rogos de Andres e de seus pais: tornou"-se sua noiva e no
inverno com ele se casou.

Muitas vezes ela se recordava com profunda saudade de sua visita atrás
dos pinheiros e então ficava séria e em silêncio. Ainda que tudo à sua
volta fosse tão lindo, ela conhecera uma beleza ainda maior, de modo
que um luto manso envolveu seu ser numa suave melancolia. Era"-lhe
doloroso ouvir seu pai ou o esposo falando dos ciganos e mandriões que
moravam no sombrio barranco. Frequentemente tinha ímpetos de defender
aqueles que sabia serem os benfeitores de toda a redondeza, em especial
frente a Andres, que parecia lançar suas reprimendas com exaltada
sofreguidão, mas refreava as palavras, trancando"-as em seu peito. Assim
passou"-se um ano, e o seguinte foi alegrado pela chegada de uma
filhinha, que chamou de Elfriede,\footnote{ O nome Elfriede 
reúne os termos \textit{Elf} (elfo) e
\textit{Frieden} (paz; amizade), podendo significar ``amiga dos elfos'' ou
``aquela que mantém a paz com os elfos''.} como recordação dos elfos.

 O jovem casal vivia com Martin e Brigitte na mesma casa, que era
suficientemente espaçosa, e ajudavam os pais a conduzir os trabalhos,
que se haviam multiplicado. A pequena Elfriede em breve demonstrou ser
dotada de habilidades e propensões incomuns, pois aprendeu a andar
muito cedo, e já sabia falar quando ainda não tinha um ano de idade.
Depois de alguns anos era tão inteligente e ponderada, e de uma beleza
tão prodigiosa, que todas as pessoas a contemplavam maravilhadas, e a mãe
não conseguia conter a impressão de que ela se parecia com aquelas
crianças esplendorosas do barranco de abetos. Elfriede não gostava da
companhia de outras crianças, evitando, e inclusive temendo, suas
brincadeiras ruidosas, e preferindo ficar sozinha. Ela costumava
retirar"-se para um canto do jardim, onde lia ou diligentemente fazia
pequenos trabalhos de costura. Também era usual que ficasse sentada
profundamente absorta em si mesma, ou andasse energicamente de um lado
a outro falando sozinha. Seus pais não se incomodavam com isso, pois
ela era saudável e se desenvolvia muito bem, embora às vezes ficassem
preocupados com as respostas e sensatas observações que fazia.

-- Crianças assim tão inteligentes -- repetia a avó Brigitte -- não
vivem muito. Elas são boas demais para este mundo. Fora isso, a beleza
desta criança vai além da natureza, e ela não conseguirá encontrar seu
lugar nesta Terra.

A menina tinha a peculiaridade de só deixar"-se servir com muita má
vontade, dando preferência a fazer tudo ela mesma. Quase sempre era a
primeira a acordar de manhã, e então lavava"-se cuidadosamente e
vestia"-se sozinha. Do mesmo modo era meticulosa à noite, insistindo em
guardar por si mesma seus \mbox{vestidos} e demais roupas, e não admitindo que
absolutamente ninguém, nem mesmo sua mãe, arrumasse suas coisas. A mãe
aceitava tais teimosias porque não imaginava que tivessem qualquer
significado, mas quão surpresa ficou, quando em um dia de festa, ao se
prepararem para uma visita ao castelo, trocou sua roupa à força, não
obstante os gritos e lágrimas da menina, e se deparou com uma moeda de
ouro de formato incomum suspensa por um cordão em torno de seu pescoço.
De imediato reconheceu a moeda, idêntica àquelas que vira em
grande quantidade na câmara subterrânea. A criança estava muito
assustada, acabando por confessar que a encontrou no jardim, que
gostara muito dela e por isso a guardara com tamanho cuidado. A seguir,
pediu com tanta insistência e emoção, que Marie voltou a pendurá"-la no
lugar de antes. Esta, mergulhada em pensamentos e sem dizer palavra,
subiu então com ela para o castelo.

Na lateral da casa dos arrendatários ficavam alguns abrigos destinados
ao armazenamento de frutas e das ferramentas usadas na lavoura, e atrás
deles estendia"-se um gramado com um velho pavilhão que fora abandonado
após a instalação das novas construções, por ficar distante demais do
pomar. Aquele lugar solitário era o preferido de Elfriede, e a ninguém
ocorria a ideia de importuná"-la ali, de modo que os pais frequentemente
só a viam em uma das metades do dia. Certa tarde, a mãe estava nos
abrigos para arrumá"-los e recuperar um objeto perdido, quando percebeu
que um raio de luz estava entrando por uma fenda na parede. Pensou em
olhar pela fresta para observar sua filha, e aconteceu de haver ali uma
pedra solta que podia ser empurrada para o lado, permitindo"-lhe ver
exatamente o interior do pavilhão. Elfriede estava sentada sobre um
banquinho, e ao seu lado estava a já conhecida Zerina, e as duas
crianças brincavam e se divertiam em adorável harmonia. A elfa abraçou
a linda criança e lhe disse com tristeza:

-- Ah, querida menina, tal como agora brinco contigo eu já brinquei
com tua mãe, quando ela era pequena e nos visitou, mas os seres humanos
crescem muito rápido e logo ficam grandes e racionais. Isso realmente é
desolador: quisera que permanecesses uma criança por tanto tempo quanto eu!

-- Com muito prazer eu o faria -- disse Elfriede --, mas todos dizem
que sou precoce e tenho disposição para em breve já ser adulta e
abandonar as brincadeiras. Então eu também não te veria mais, Zerina!
Ah, é o mesmo que sucede às flores: como é esplêndida a macieira quando
seus inchados botões avermelhados estão se abrindo em flor! Nessa época
a árvore parece tão imensa e soberba, e todos que a veem têm a
expectativa de algo grandioso. Mas então vem o sol, a flor abre"-se com
muita delicadeza, e logo já aparece por baixo o caroço malvado que
depois afasta e arranca as coloridas e elegantes vestes; amedrontado,
não consegue impedir seu crescimento, e no outono transforma"-se em
fruta. Certamente uma maçã também é bela e apetitosa, mas não é
comparável à flor da primavera. O mesmo dá"-se com as pessoas. Não me
agrada a ideia de me tornar uma moça grande. Ah, se pudesse visitar"-vos
uma única vez!

-- Desde que o rei mora conosco -- disse Zerina --, isso é totalmente
impossível. Mas, minha querida, eu venho ver"-te com grande frequência,
e ninguém me vê, ninguém sabe de nada, nem aqui nem lá. Sem ser vista,
atravesso os ares, ou passo voando na forma de passarinho\ldots{} Oh, nós
ainda estaremos juntas por bastante tempo, enquanto fores pequena. O
que posso fazer para agradar"-te?

-- Deves amar"-me com muito afeto -- disse Elfriede --, com tanto afeto
quanto eu trago em meu coração. Mas façamos outra vez uma rosa.

Zerina tirou a conhecida caixinha do seio, jogou ao chão dois grãos e,
de repente, lá estava diante delas um arbusto verde com duas rosas
escarlates, as quais pendiam uma em direção da outra e pareciam
beijar"-se. Rindo, as crianças arrancaram as rosas, o que fez o arbusto
desaparecer novamente.

-- Quisera -- exclamou Elfriede -- que essa rubra criança, esse milagre
da terra não morresse tão rápido.

-- Dá"-me a rosa! -- disse a pequena elfa. 

Por três vezes Zerina exalou sobre o botão, e beijou"-o três vezes.

-- Agora -- disse, devolvendo a flor -- ela permanecerá fresca e aberta
em flor até o inverno.

-- Irei guardá"-la como se fosse uma imagem tua -- disse Elfriede --,
mantendo"-a bem protegida em meu quartinho, e vou beijá"-la todas as
manhãs e noites, como se fosses tu.

-- O sol já está se pondo -- disse aquela --, devo retornar para casa agora.

Então elas se abraçaram mais uma vez, e Zerina desapareceu.

À noite, Marie tomou sua filha nos braços com um sentimento de dor e
reverência. Passou a dar à adorável menina ainda maior liberdade do que
antes, e muitas vezes apaziguava seu esposo quando ele procurava pela
criança, o que vinha fazendo agora com maior frequência, pois não
gostava da reclusão da menina e temia que esse hábito a tornasse
simplória ou até estúpida. Sorrateira, a mãe foi mais vezes até a
fresta na parede, e quase sempre presenciava a pequena elfa brilhante
sentada ao lado de sua filha, ocupada em brincadeiras, ou em conversas sérias.

-- Tu gostarias de poder voar? -- Zerina perguntou uma vez a sua amiga.

-- Com que alegria! -- exclamou Elfriede.

Imediatamente a fada enlaçou a mortal, e flutuou com ela elevando"-se
acima do chão até o telhado do pavilhão. Em sua preocupação, a
assustada mãe esqueceu sua cautela e inclinou"-se com a cabeça para fora
para acompanhá"-las com o olhar. Do alto, Zerina ergueu o dedo em sinal
de ameaça, mas sorriu. Desceu novamente com a criança, abraçou"-a, e
desapareceu. Mais tarde voltou a repetir"-se que Marie fosse vista pela
criança prodigiosa, a qual sempre balançava com a cabeça ou ameaçava,
mas com um gesto afável.

Em diversas ocasiões também já havia sucedido de Marie dizer ao marido
durante uma discussão:

-- Estas sendo injusto com as pessoas pobres na cabana!

Quando então Andres insistia em saber por que ela tinha uma opinião
diferente de todas as pessoas na aldeia e inclusive contrária à do
próprio Conde, ela emudecia e ficava encabulada. Certo dia, após o
almoço, Andres mostrou"-se especialmente severo, afirmando que os vadios
deveriam ser declarados nocivos ao condado e expulsos de lá. Ela, em
sua indignação, exclamou:

-- Cala"-te, pois eles são teus benfeitores e de todos nós!

-- Benfeitores? -- perguntou Andres com surpresa. -- Esses andarilhos?

Em sua fúria, cedeu à insistência dele e, fazendo"-o prometer absoluta
discrição, contou"-lhe a história de sua infância. Como a cada palavra
ele ia se tornando cada vez mais incrédulo e zombeteiramente balançava
com a cabeça, ela o tomou pela mão e conduziu"-o até o recinto de onde
ele pôde avistar, muito surpreso, como a elfa brincava com sua filha no
pavilhão e a acariciava. Ele estava totalmente atônito. Uma exclamação
de assombro escapou"-lhe. Zerina levantou os olhos, de imediato
empalideceu e começou a tremer fortemente. Fez o gesto ameaçador, não
de forma amável, mas com expressão zangada, e disse então para Elfriede:

-- Não tens culpa, coração amado, mas eles nunca ganham em
sabedoria, não importa o quanto se considerem prudentes.

Ela abraçou a menina muito apressadamente, e, sob a forma de um corvo que
gritava roufenho, voou por cima da horta até os pinheiros.
 
Ao entardecer a menina estava muito calada e, chorando, beijava a rosa.
Marie estava amedrontada e Andres falava pouco. Anoiteceu. De um
momento para outro, as árvores começaram a murmurejar, pássaros voaram
ao redor com gritos assustados, um trovão fez tremer a terra, e
lamentos fizeram"-se ouvir pelos ares. Marie e Andres não tiveram
coragem de se levantar; mantiveram"-se embrulhados nas cobertas e
esperaram com angústia e temor pelo raiar do dia. De manhã tudo estava
mais calmo, e fazia silêncio quando o sol veio por detrás da floresta
jorrando sua luz.

Andres vestiu"-se, e Marie notou que a pedra do anel em seu dedo tinha
esmaecido. Ao abrirem a porta, os raios do sol lançaram"-se ao seu
encontro, mas a paisagem ao redor estava quase irreconhecível. O
frescor da floresta tinha desaparecido, as colinas tinham abaixado, os
riachos fluíam lentos e com pouca água, o céu parecia cinzento, e
quando voltaram o olhar para os abetos, eles não estavam mais escuros
ou mais tristes do que as demais árvores; e as cabanas atrás deles não
tinham nada de repulsivo. Vários moradores da aldeia vieram e contaram
sobre a estranha noite, e que haviam visitado o terreno onde viviam os
ciganos, os quais provavelmente partiram porque as cabanas
encontravam"-se vazias, e seu interior era como costumam ser as casas de
gente pobre, inclusive com coisas deixadas para trás. Elfriede disse
secretamente à sua mãe:

-- Durante a noite, quando não conseguia dormir e rezava de medo
durante o alvoroço, minha porta abriu"-se de repente e minha companheira
de brinquedos entrou para dizer"-me adeus. Trazia uma bolsa de viagem,
chapéu sobre a cabeça e um grande cajado de peregrino na mão. Ela
estava muito aborrecida contigo por seres a causa dos grandes e
dolorosos castigos a que está sendo submetida, apesar de sempre ter te
amado muito. E ela disse que todos saem daqui a contragosto.

Marie proibiu"-a de falar sobre isso. Nesse momento o barqueiro chegou do
rio e começou a contar histórias das mais extraordinárias. Ao cair da
noite um homem estrangeiro de grande estatura tinha vindo procurá"-lo;
queria alugar sua balsa até o amanhecer, porém com a condição de que
ele ficasse recolhido na casa e fosse dormir ou, ao menos, não
colocasse o pé para fora da porta.

-- Tive medo -- acrescentou o velho --, mas o estranho negócio não me
deu sossego. De mansinho dirigi"-me à janela e furtivamente olhei para o
rio. Nuvens imensas percorriam inquietas o céu e as distantes florestas
sussurravam aflitas. Era como se minha cabana tremesse, lamentos e
soluços rondassem a casa. De súbito avistei uma fulgurante luz branca
que se alargava e alargava, como se muitos milhares de estrelas caídas
se movessem para fora do barranco de abetos. Faiscando, atravessaram o
prado e se esparramaram em direção ao rio. Ouvi um barulho de pés, um
tinido, um sussurrar e murmurar que vinha se aproximando mais e mais, e
seguia rumo a minha balsa. Nela embarcaram todos: formas grandes e
pequenas que luziam; homens e mulheres ao que parecia, e crianças, e o
grande forasteiro conduziu"-os todos à outra margem. No rio, ao lado da
embarcação, nadavam muitos milhares de figuras claras, pelo ar
esvoaçavam luzes e névoas brancas, e todos reclamavam e se queixavam
por terem de viajar para tão, tão longe, e abandonar as queridas terras
a que estavam acostumados. Em meio a tudo isso soava o bater dos remos
e o marulho da água, e depois, subitamente fez"-se novamente silêncio.
Muitas vezes a balsa atracou para ser outra vez carregada. Levaram
consigo também muitos recipientes pesados, que pequenos e horripilantes
homenzinhos carregavam ou empurravam; se eram demônios, se eram
duendes, não sei. Então chegou um séquito magnífico rodeado por ondas
fulgurantes. Parecia ser um ancião sobre um pequeno corcel branco, em
torno do qual todos se aglomeravam. Mas apenas pude ver a cabeça do
cavalo, pois o animal estava oculto de alto a baixo por cobertas
suntuosas e resplandecentes. Sobre a cabeça o velho tinha uma coroa e,
enquanto ele passava, era como se o sol estivesse despontando ali e a
aurora lançasse seus raios avermelhados em minha direção. Assim foi
durante a noite inteira. Por fim, adormeci em meio àquela confusão,
parcialmente imerso em alegria, parcialmente em calafrios. Pela manhã
tudo estava quieto, mas o rio até parece ter escoado para longe, e
agora não me será fácil controlar a embarcação.

Naquele mesmo ano as plantas começaram a minguar, depois as florestas
pereceram, as fontes secaram, e aquela mesma região que antes era a
alegria de todo viajante que a atravessava, no outono estava desolada,
despida e seca. E, em meio ao mar de areia, só aqui e ali ainda era
possível encontrar algum pequeno recesso onde crescia relva de um verde
empalidecido. Todas as árvores frutíferas feneceram, os \mbox{vinhedos}
\mbox{degeneraram}, e a paisagem oferecia uma imagem tão triste que, no ano
seguinte, o conde e sua família deixaram o castelo e este foi se
deteriorando até virar ruína.

Dia e noite Elfriede observava com grande saudade sua rosa e pensava em
sua companheira de brinquedos. Da mesma forma como a flor foi perdendo
o viço e acabou murchando, também ela deixou pender a cabecinha e já
antes da primavera tinha definhado. Marie muitas vezes ia ao lugar
diante da cabana e soluçava pela felicidade perdida. Ela foi se
consumindo tal como a filha, e seguiu"-a em poucos anos. O velho Martin
mudou"-se com seu genro para a região em que vivera antigamente.
\medskip

\hfill\textit{Tradução de Karin Volobuef}


\chapter{Feitiço de amor}


\textsc{Emil estava sentado} à mesa profundamente imerso em pensamentos, e
aguardava seu amigo Roderich. A vela ardia a sua frente, a noite de
inverno estava fria, e hoje ele ansiava pela presença de seu
companheiro de viagem, embora em geral o preferisse longe de si. Esta
noite era diferente, pois pretendia contar"-lhe um segredo e pedir
conselho. Em todos os assuntos e situações da vida o insociável Emil
encontrava dificuldades tamanhas e obstáculos tão intransponíveis que
parecia uma ironia do destino que tivesse encontrado Roderich, que em
tudo era o oposto do amigo. Inconstante, frívolo, sujeito às impressões
do momento e instantaneamente entusiasmado por qualquer coisa, Roderich
lançava"-se em todos os empreendimentos, tinha soluções para tudo e
nenhuma tarefa parecia"-lhe difícil demais, nenhuma adversidade
conseguia afugentá"-lo; mas, iniciadas as atividades, perdia o fôlego e
o interesse tão impulsivamente quanto antes havia se deslumbrado por
elas. A partir daí, os estorvos em seu caminho não lhe serviam como
estímulo para aumentar seus esforços, mas apenas para despertar"-lhe
menosprezo por aquilo que iniciara com tanto ânimo. Desse modo,
Roderich abandonava sem motivos e esquecia todos os planos de forma tão
instantânea e imprudente quanto os iniciara. Por isso, não se passava
um dia sequer sem que os amigos se enredassem em \mbox{discordâncias} tão
fortes que sua amizade parecia irremediavelmente ameaçada de esmorecer.
contudo, \mbox{talvez} aquilo que parecia separá"-los fosse justamente o que 
com maior força os unia. Eles nutriam um profundo afeto um pelo
outro, mas  sentiam ambos enorme satisfação em poder queixar"-se
reciprocamente, amparados em sólidas justificativas. 

Emil era um rapaz rico com temperamento sensível e melancólico que, após
a morte de seus pais, havia entrado na posse da fortuna da família. Ele
iniciara uma viagem para melhorar sua formação, mas já fazia alguns
meses que se encontrava em uma cidade de grande porte; estava lá para
entregar"-se aos folguedos do carnaval, mas quase nunca comparecia a
eles; nessa cidade pretendera ainda se reunir com parentes para
discutir assuntos relativos a sua fortuna, mas nem sequer ainda os havia
visitado. No caminho para lá havia se deparado com o inconstante e
volúvel Roderich, o qual vivia em discórdia com seus tutores. Para
livrar"-se por completo deles e de suas incômodas advertências, Roderich
avidamente aproveitara a oportunidade que seu novo amigo oferecia de
levá"-lo como acompanhante em sua viagem. Ao longo do caminho eles já
haviam pretendido várias vezes separar"-se novamente; entretanto, a cada
altercação ambos percebiam com força redobrada o quanto eram
indispensáveis um ao outro. Mal desciam da carruagem em alguma cidade e
Roderich já havia visto todos os monumentos do lugar, já se esquecendo
deles no dia seguinte. Emil, ao contrário, levava uma semana inteira
debruçado sobre livros a fim de não perder nenhum dos lugares que, mais
tarde, deixava justamente de visitar por pura indolência. Logo após a
sua chegada, Roderich já tinha feito inúmeros amigos e visitado todos
os lugares públicos. Não raro conduzia seus novos amigos ao aposento
onde Emil ficara solitário, abandonando"-os ali aos cuidados dele tão
logo começava a considerá"-los enfadonhos. Com igual frequência
constrangia o modesto Emil ao elogiar sem medida suas habilidades e
conhecimentos perante homens eruditos e sábios, dando"-lhes a entender
que muito poderiam aprender dele sobre línguas, Antiguidade e artes,
ainda que ele próprio nunca aproveitasse a chance de ouvir seu amigo
sobre esses assuntos quando a conversa rumava nessa direção. Nas poucas
ocasiões em que Emil estava disposto a participar de alguma atividade,
era quase certo que seu inconstante amigo havia se resfriado em algum
baile ou passeio de trenó noturnos e se encontrava acamado, de modo que
Emil vivia na mais completa solidão embora convivesse com a mais vivaz,
irrequieta e comunicativa das pessoas.

Hoje Emil ficara acordado para aguardá"-lo, pois Roderich tinha lhe feito
a solene promessa de vir passar o serão em sua companhia para poder
ouvir o motivo de seu melancólico amigo achar"-se há semanas em estado
de inquietação e alarme. Enquanto esperava, Emil escrevia os seguintes versos.

\begin{verse}
Como é graciosa a primavera,\\
quando os rouxinóis cantam.\\
Como ressoam as árvores\\
e tremem as flores de prazer. 

Como é formoso o luar\\
quando alçam voo as brisas\\
entremeadas de fragrância\\
pelos silenciosos arvoredos.

Como são maravilhosas as roseiras,\\
quando enfeitam de amor os campos\\
e lançam olhares de encanto\\
ao luzir das estrelas.

Mas nada é como minha bela,\\
quando à luz trêmula,\\
está sozinha no quarto pequenino,\\
e eu cá à espreita fico.

Quando faz as tranças\\
com sua delicada mãozinha,\\
seu corpo une"-se à nívea veste,\\
e flores vão aos cachos castanhos.

Quando toma o alaúde,\\
sons transbordam,\\
saltam das cordas\\
e se oferecem risonhos.

Quando a voz brejeira\\
faz acordar melodias,\\
os sons vêm buscar guarida\\
nos recantos de meu coração.

Oh, quão malévolos eles são!\\
Ali se enroscam e sussurram:\\
``Só saímos quando te despedaçarmos\\
e aprenderes o que é amar''.
\end{verse}


Emil ergueu"-se com impaciência. Havia escurecido e Roderich não chegava
-- a quem Emil desejava contar sobre seu amor por uma desconhecida
que morava em frente e o forçava a ficar dias a fio em casa e noites
sem conta em claro. De repente, passos soaram escada acima, a porta
abriu"-se sem que ninguém batesse, e dois mascarados com vestes
coloridas e caras distorcidas entraram: um deles estava fantasiado de
Turco com indumentária em seda rubra e azul; o outro, fantasiado de
espanhol, coberto com trajes amarelo"-claros e avermelhados, e muitas
penas balançantes no chapéu. Quando Emil estava prestes a perder a
paciência, Roderich removeu a máscara, deixando à mostra seu familiar
sorriso e disse:

-- Ai, meu caro, que expressão rabugenta! Por acaso essa é tua
alegria de carnaval? Eu e nosso prezado jovem oficial estamos aqui para
buscar"-te, pois hoje é dia de baile no grande Salão de Máscaras. Como
sei que juraste solenemente trajar outras vestes que não as costumeiras
roupas pretas, quero que nos acompanhes. Venhas assim mesmo como estás,
pois já é bastante tarde.

Emil estava enfurecido e retrucou:

-- Pareces, como é teu hábito, ter esquecido por completo o que
acordamos. Lastimo muito (voltando"-se para o desconhecido) que não
possa acompanhar"-vos. Meu amigo foi muito precipitado se assumiu tal
compromisso em meu nome. Não posso de modo algum ausentar"-me daqui, já
que tenho assunto de extrema importância para conversar com ele.

O desconhecido, que era modesto e havia compreendido a disposição de
Emil, retirou"-se. Roderich, no entanto, apenas voltou a colocar a
máscara com gesto de grande indiferença, postou"-se à frente do espelho
e disse:

-- É bem verdade que a máscara confere uma aparência horrorosa, não
é mesmo? No fundo, ela não passa de uma invenção desagradável e de mau
gosto.

-- Isso não é nenhuma questão em aberto -- respondeu Emil
cheio de indignação. -- Fazer"-se passar por uma caricatura e anuviar o
espírito contam entre as diversões que procuras com maior predileção.

-- Já que não aprecias dançar -- respondeu o outro --, e tomas a dança
por uma invenção perniciosa, consideras que ninguém mais deve poder
entregar"-se à alegria. Como é lamentável quando alguém é limitado a
suas excentricidades.

-- Certamente, respondeu o amigo enfurecido, e eu tenho oportunidades o
bastante de observar isto em ti. Eu havia confiado que, após nossa
última conversa, havias de me fazer companhia hoje à noite, mas\ldots{}

-- Mas é carnaval! -- continuou o outro. -- E todos os meus conhecidos e
também algumas damas aguardam por mim no grande baile de hoje. Leves em
consideração, meu caro, que é uma verdadeira doença em teu íntimo que o
faz ser tão injusto com essas diversões e rechaçá"-las com tamanha veemência.

Emil disse:

-- Qual de nós dois merece ser chamado de doente é algo que não
quero examinar em detalhes. Teu incompreensível comportamento tolo, tua
sofreguidão por entretenimentos, tua caça por diversões que tornam
vazio teu coração não me parecem indicar uma alma \mbox{saudável}. Além disso,
em certas coisas tu bem poderias ceder à minha fragilidade, já que é
esse o nome que lhe dás, pois não há nada nesse mundo que me cause
maior aborrecimento do que um baile com sua música horrorosa. Já foi
dito por alguém que, aos olhos de uma pessoa surda que presenciasse uma
dança sem ouvir a música, um baile haveria de parecer um grupo de
loucos. Eu, ao contrário, acho insana essa própria música horripilante,
esse redemoinho de algumas poucas notas tocadas com desagradável
rapidez formando aquelas malditas melodias que aderem a nosso cérebro,
ou melhor, que se agregam a nosso sangue e das quais só conseguimos nos
ver livres após muito tempo -- isso é o que eu chamo de verdadeira
loucura e desvario. O único caso em que a dança poderia ser"-me de algum
modo suportável seria sem música.

-- Agora vejas, que paradoxo! -- respondeu o mascarado. -- Tu chegas ao
ponto de considerar as coisas mais naturais, inocentes e alegres do
mundo como algo monstruoso e até hediondo.

-- Não posso mudar meus sentimentos -- respondeu o sisudo --, desde a
infância esses sons sempre me deixaram infeliz e muito frequentemente
chegavam a me lançar no desespero. No mundo dos sons encontram"-se para
mim as fantasmagorias, espectros e fúrias, e é assim que eles tremulam
ao redor de minha cabeça e me encaram com as caretas mais horrendas.

-- Debilidade nervosa -- retrucou o outro --, semelhante a tua exagerada
aversão a aranhas e uns tantos outros animais inocentes.
 
-- Chama"-os de inocentes -- respondeu Emil com enfado --, porque
eles não te causam repugnância. Mas para alguém que, como eu,
experimenta a sensação de repulsa e aversão ao vê"-los, sendo tomado de
um horror inexprimível que inunda a alma e percorre todo o corpo, esses
monstros execráveis, como sapos e aranhas, ou ainda a mais asquerosa de
todas as criaturas, o morcego, não são indiferentes e insignificantes,
mas seres cuja existência se opõe à dele de modo absolutamente hostil.
De fato somos tentados a sorrir dos incrédulos, cuja imaginação não
consegue fazer as pazes com os fantasmas e lancinantes espectros, assim
como os demais filhos da noite, que vemos em períodos de doença ou com
que nos deparamos nas imagens de Dante, pois a realidade mais prosaica
em nosso derredor nos confronta com modelos pavorosos e disformes
dessas figuras. Acaso seríamos capazes de amar o belo se não nos
sentíssemos horrorizados perante tais fealdades?

-- Por que horrorizados? -- perguntou Roderich. -- Por que o grande
império das águas e dos mares haveria de nos apresentar essa abominação
a que tua fantasia se acostumou, em vez de justamente deixar à vista
figuras estranhas, interessantes e burlescas, de forma que toda essa
extensão causasse justamente a impressão de um cômico salão de baile?
Mas tua excentricidade ainda vai além, pois com o mesmo exagero com que
amas a rosa a ponto de idolatrá"-la, odeias as demais flores. Que mal te
causou, afinal, o bom lírio vermelho, ou ainda os outros rebentos do
verão? Do mesmo modo, desagradam"-te diversas cores, diversos odores e
muitas ideias. E não fazes nada para te endurecer contra tais estados
de espírito, ao contrário, te entregas delicadamente a eles. E ao
final, uma coleção de tais bizarrices irá ocupar o lugar que deveria
estar em posse de teu Eu.

Emil estava enfurecido até a última fibra de seu coração e não
respondeu. Tinha já desistido de confiar"-se a seu leviano amigo, que
aliás não demonstrava ter qualquer desejo de conhecer o segredo que o
melancólico companheiro havia anunciado com tamanha gravidade. Roderich
estava sentado na poltrona, brincando com indiferença com sua máscara.
De súbito exclamou:

-- Sejas bondoso, Emil, e me empresta teu casaco grande.

-- Para quê? -- perguntou aquele.

-- Estou ouvindo música lá adiante na igreja -- Roderich respondeu --, e
já tenho perdido esta ocasião em todas as últimas noites. Hoje parece
ser justamente o que preciso. Com teu casaco posso encobrir esta roupa
e também ocultar a máscara e o turbante, e tão logo a apresentação
tiver acabado, seguir diretamente para o baile.

Resmungando, Emil foi buscar o casaco no guarda"-roupa, entregou"-o para o
amigo, que já se erguera da poltrona, e se forçou a um sorriso irônico.

-- Cá está o punhal turco que comprei ontem -- disse Roderich
enquanto vestia o casaco --, guarda"-o. Não é bom carregar consigo tais
coisas sérias como se fossem brinquedos. Nunca se sabe se elas não
seriam mal utilizadas quando uma desavença ou outro tipo de discórdia
oferecesse a oportunidade. Amanhã nos veremos novamente. Passa bem e
procura distrair"-te.

Sem esperar por resposta, já descia apressado pela escada.

Quando Emil se viu sozinho, buscou esquecer sua fúria e avistar o
comportamento de seu amigo pelo lado ridículo. Observou o punhal
lustroso e formosamente trabalhado, e disse:

-- Como deverá sentir"-se o ser humano que empurra um ferro assim
aguçado para dentro do peito do oponente, ou até mesmo com ele fira 
um ente amado?

Guardou a arma, em seguida abriu cuidadosamente os batentes de sua
janela e olhou para o outro lado da ruela estreita. Mas não havia
nenhuma luz; na casa em frente reinava completa escuridão. A preciosa
figura que ali vivia e por essas horas habitualmente se desincumbia de
tarefas domésticas parecia ausente. ``Talvez até mesmo estivesse no
baile'', pensou Emil, ``embora isso não harmonizasse com seu modo de vida
recluso''. Repentinamente, porém, uma luz surgiu, e a menina que
costumava acompanhar sua desconhecida amada, a qual tanto de dia quanto
à noite muito se ocupava com a criança, atravessou o aposento
carregando uma vela e fechou as venezianas. Uma fresta permaneceu
iluminada, grande o suficiente para deixar à mostra, do ponto de vista
de Emil, uma parte do diminuto quarto. Era aí que ele permanecia em pé,
feliz, frequentemente até depois da meia"-noite, como enfeitiçado,
observando cada movimento da mão, cada expressão da face de sua amada.
Tinha prazer quando ela ensinava a criança a ler, ou lhe dava lições de
coser e tricotar. Havia procurado se informar e contaram"-lhe que a
menina era uma pobre órfã que a formosa jovem tinha compassivamente
acolhido para educá"-la. Os amigos de Emil não compreendiam por que ele
morava nesta ruela estreita em uma casa sem conforto, por que era tão
raro que ele procurasse a companhia de outras pessoas, e com o que ele
se ocupava. Sem ocupação e vivendo na solidão ele era feliz 
estando insatisfeito apenas consigo mesmo e com seu caráter insociável
que o impedia de ousar aproximar"-se e conhecer mais de perto aquela
bela criatura, não obstante ela o ter cumprimentado e agradecido de
modo tão gentil quando se viram algumas vezes durante o dia. Emil não
sabia que ela, por sua vez, também espreitava a janela dele sentindo"-se
igualmente aturdida, e não suspeitava que tipo de desejos fermentavam 
no coração do moça, nem a que esforços, a que sacrifícios
ela se considerava disposta tão somente para entrar na posse do amor dele. 

Após andar algumas vezes de um lado a outro, e após tanto a vela quanto
a criança desaparecerem novamente, de repente ele tomou a decisão de ir
ao baile. Era uma decisão totalmente contrária a sua inclinação e a sua
\mbox{natureza}, mas ocorreu"-lhe que talvez sua desconhecida pudesse ter feito
uma exceção em seu modo de vida reservado, dispondo"-se a uma vez também
desfrutar o mundo e suas recreações. As ruelas estavam fortemente
iluminadas, a neve crepitava sob seus pés, carruagens cruzavam à sua
frente, e mascarados trajando as mais variadas fantasias assobiavam e
cantarolavam enquanto passavam por ele. De muitas casas ressoava em sua
direção a odiosa música de dança, e ele não conseguia obrigar"-se a ir
pelo caminho mais curto que levava até o salão, rumo ao qual as pessoas
afluíam e se acotovelavam de todos os lados. Ele seguiu contornando a
velha igreja, contemplou a torre que se erguia alta e severa para o céu
noturno, e desfrutava o silêncio e solidão que reinava nesse lugar
remoto. Junto ao vão de uma grande porta de igreja, cujos múltiplos
entalhes ele sempre apreciara com muito prazer, pois o faziam pensar na
arte antiga e em tempos longínquos, ele tomou lugar para se entregar à
contemplação por uns poucos momentos. Não fazia muito que estava ali
quando uma figura atraiu sua atenção, a qual impacientemente andava de
um lado a outro, parecendo aguardar a chegada de alguém. Ao clarão de
uma lanterna que brilhava próximo a uma imagem de Nossa Senhora,
distinguiu com exatidão a face, assim como a invulgar roupagem. Era uma
anciã cuja aparência horrenda saltava aos olhos por estar usando um
corpete escarlate bordado em ouro, que contrastava grotescamente com a
fealdade da velha. A saia que usava era escura, o gorro sobre sua
cabeça também cintilava com ouro. No começo Emil acreditou estar diante
de uma fantasia de mau gosto, de um mascarado que houvesse errado o
caminho e se perdido por essas bandas, no entanto, sob a forte
iluminação logo se convenceu de que o velho rosto escurecido e cheio de
rugas era real e não uma imitação. Não demorou muito e surgiram dois
homens embrulhados em casacos, que pareciam acercar"-se do lugar com
passos cautelosos, muitas vezes lançando olhares de esguelha para os
lados para ver se ninguém os seguia. A velha  foi em sua direção.

-- Trazeis as velas? -- perguntou apressadamente e com uma voz áspera.

-- Cá estão -- afirmou um deles --, o preço vos é conhecido, acertai de
vez as coisas.

A velha parecia dar dinheiro, que o homem conferiu debaixo de seu
casaco.

-- Quero crer -- recomeçou a velha -- que elas foram vertidas
absolutamente de acordo com as regras e com o máximo de habilidade para
que a eficácia não seja perdida.

-- Não vos preocupeis -- disse aquele e afastou"-se rapidamente.

O outro, que ficara atrás, era um rapaz jovem; ele tomou a velha pela
mão e perguntou:

-- Será possível, Alexia, que tais ritos e fórmulas, estas
antiquadas e estranhas lendas, em que nunca pude acreditar, conseguem
aprisionar a livre vontade do ser humano, podendo inflamar o amor e o ódio?

-- Assim é -- retrucou a rubra mulher --, mas os ingredientes são
vários. Não bastam estas velas, vazadas à meia"-noite na lua nova e com
sangue humano, não bastam as fórmulas de bruxaria nem invocações. Elas
sozinhas não são suficientes, sendo  imprescindíveis ainda muitos
outros suplementos, que o conhecedor da arte bem sabe.

-- Então confio em ti -- disse o estranho. 

-- Amanhã, após a meia"-noite, estou a vosso serviço -- respondeu a
velha --, tenho convicção de que não havereis de ser o primeiro a ficar
insatisfeito com meus préstimos. Hoje, como ouvistes, estou
comprometida com outro cliente, cujos sentidos e juízo certamente nossa
arte deverá influenciar com grande eficácia.

As últimas palavras ela pronunciou com uma meia risada, e ambos se
separaram enveredando por direções distintas. Emil saiu trêmulo do
escuro nicho e elevou os olhos para a imagem da Virgem com a criança:

-- Diante de vossos olhos, oh Veneranda -- exclamou à meia"-voz --, esses
abomináveis têm a ousadia de realizar encontros para negociar suas
execráveis insídias. No entanto, assim como envolveis vossa criança em
amor, assim também um amor invisível segura a todos nós em braços
perceptíveis, e nosso pobre coração bate tanto nos momentos de alegria
quanto de receio ao encontro de um maior, que nunca irá nos abandonar.

Nuvens deslizavam sobre o topo da torre e o telhado escarpado da igreja,
as estrelas eternas olhavam fulgurantes e com afável circunspeção para
baixo, e Emil afastou"-se resoluto desses calafrios noturnos e voltou
seus pensamentos para a beleza da desconhecida. Enveredou novamente
pelas ruelas cheias de pessoas e guiou seus passos para o salão de
baile inundado de luzes, do qual lhe vinham ao encontro vozes, barulho
de veículos, e, em intervalos isolados, a música ruidosa.

No salão, logo perdeu"-se em meio ao tumulto da aglomeração, os
dançarinos pulavam ao seu redor, mascarados passavam para lá e para cá
ao seu lado, timbales e trompetes atordoavam seus ouvidos, e ele tinha
a impressão de que a própria vida humana não passava de mero sonho.
Atravessou as fileiras, e somente seu olhar mantinha"-se vigilante,
procurando aqueles olhos amados e aquela \mbox{cabeça} formosa com seus cachos
castanhos, por cuja visão hoje ansiava com mais força do que
usualmente. Em seu íntimo também censurava aquele ser adorado por
imergir e perder"-se nesse convulso mar de desordem e insensatez.

-- Não -- disse para si mesmo --, nenhum coração que ama quererá
entreabrir"-se a tais bramidos desoladores, nos quais a saudade e as
lágrimas são motivo de zombaria e de que os risos retumbantes de
trompetes selvagens escarnecem. O sussurro das árvores, o murmurinho
das fontes, o som do alaúde e de nobre canto que emana em profusão do
seio comovido são os sons no qual o amor vive. O alarido daqui, ao
contrário, é como os trovões e a sanha do desespero que ressoam
ensandecidos no inferno.

Não encontrou o que procurava, pois foi"-lhe impossível conformar"-se com
a ideia de que o amado rosto talvez tivesse se ocultado sob uma
detestável máscara. Já havia atravessado o salão três vezes de um lado
a outro e em vão observado com atenção todas as damas sentadas e não
mascaradas, quando o espanhol aproximou"-se dele e disse:

-- Que bom que tenhais vindo afinal. Com certeza procurais vosso
amigo, não?

Emil tinha se esquecido completamente dele; porém, disse envergonhado:

-- De fato, estou estranhando o fato de não encontrá"-lo aqui;
afinal, sua fantasia é especialmente visível.

-- Acaso sabeis o que aquele ser espantoso está fazendo? -- retrucou
o jovem oficial. -- Ele nem dançou nem permaneceu por muito tempo no
salão, pois pouco depois encontrou um amigo, Anderson, que acabara de
chegar do campo. A conversação deles logo enveredou para a literatura,
e como ele ainda não conhecia um poema recentemente divulgado, Roderich
não descansou enquanto não lhe abriram um dos aposentos mais
recônditos. Lá encontra"-se ele sentado à luz de uma vela solitária,
lendo para seu companheiro a obra inteira.

-- Isto é típico dele -- disse Emil --, já que é sempre levado por
estados de humor. Eu experimentei todos os meios, inclusive não me
abstive de discussões amigáveis para demovê"-lo do hábito de sempre
viver \textit{ex tempore} e fazer sua vida inteira desenrolar"-se como
um improviso. Mas essas loucuras lhe são tão caras ao coração que ele
prefere apartar"-se do mais querido amigo do que se separar delas. Esta
obra, que ele ama tanto a ponto de sempre levá"-la consigo, ele
recentemente quis ler para mim, e eu até havia expresso com ênfase meu
desejo de ouvi"-lo. Mal, porém, havíamos transposto o início e eu estava
rendido às belezas do poema, quando ele subitamente ergueu"-se de um
salto. Retornando envolto no avental de cozinha, mandou acender o fogo
conforme instruções muito específicas, disse querer assar bifes para
mim, embora eu não sentisse o menor desejo por eles, e afirmou ser ele
a pessoa que melhor sabe prepará"-los em toda a Europa, ainda que na
maioria das vezes ele os estrague.

O espanhol riu.

-- Ele nunca se apaixonou? -- perguntou.

-- À maneira dele -- respondeu Emil muito seriamente --, ou seja, como
se quisesse escarnecer de si e do amor: por muitas ao mesmo tempo, e,
de acordo com suas palavras, de modo desesperado, mas esquecendo"-se de
todas elas em oito dias.

Separaram"-se no tumulto, e Emil foi até o remoto aposento no qual ele de
longe já ouvia o amigo declamando em altos brados.

-- Ah, cá também estás! -- exclamou ele em sua direção. -- Isto vem a
calhar; acabei justamente o trecho no qual fomos interrompidos
recentemente. Senta"-te e poderás acompanhar minha leitura.

-- Não estou agora com disposição para isso -- disse Emil --, além
disso, considero esta hora e este lugar pouco apropriados a uma tal
recreação.

-- Por quê? -- Roderich retrucou. -- Tudo deve adequar"-se a nossa
vontade. Todos os momentos são adequados para que nos ocupemos de algo
tão nobre. Ou preferes dançar? Hoje faltam dançarinos, de modo que
bastariam algumas horas de piruetas e um par de pernas cansadas para
conquistares a gratidão e profunda simpatia de muitas senhoras.

-- Passes bem! -- exclamou aquele já da porta. -- Vou para casa.

-- Ainda uma palavra! -- chamou Roderich atrás dele. -- Amanhã, logo ao
amanhecer, partirei na companhia desse senhor em viagem para o campo,
onde passarei alguns dias. Mas baterei à sua porta para despedir"-me. Se
ainda estiveres dormindo, como é provável, não te preocupa em acordar,
pois em três dias já estarei novamente de volta. 

 E voltando"-se ao novo amigo:

 -- É o mais inusitado de todos os seres humanos! Tão pesado,
mal"-humorado e austero que não consegue saborear nenhuma alegria,
ou melhor, não há para ele qualquer alegria. Tudo deve ser nobre,
grande, elevado, seu coração deve tomar parte em tudo, ainda que se
trate de um teatro de marionetes. E quando não se realizam tais
pretensões, que são realmente bastante absurdas, então seu estado de
espírito se torna trágico e ele passa a considerar o mundo inteiro rude
e bárbaro. Lá fora, com certeza, ele exige que dentre os fantasiados
haja um Pantalone e um Pulcinella, com o coração a arder de saudade e
pleno de anseios sublimes, e que o Arlequim filosofe melancolicamente
sobre a futilidade do mundo, e, caso estas expectativas não se
confirmem, então certamente as lágrimas lhe subirão aos olhos, e, cheio
de mágoa e menosprezo, dará as costas a todo o espetáculo colorido.

-- Então ele é melancólico? -- o ouvinte perguntou.

-- Não exatamente -- respondeu Roderich. -- Ele apenas foi mimado em
demasia tanto por pais muito ternos, como por si mesmo. Ele tinha se
acostumado a abandonar o coração ao livre fluxo da maré, deixando"-o ao
sabor dos altos e baixos. Então, se agora faltar essa comoção, ele logo
grita ``Milagre!'' e deseja prometer prêmios para animar os físicos a
explicarem exaustivamente esse evento da natureza. Ele é a melhor
pessoa que vive sobre a face da Terra, mas todos os meus esforços para
levá"-lo a abandonar essa dissonância foram totalmente inúteis, e caso
eu não queira receber ingratidão como pagamento por minhas boas
intenções, devo deixá"-lo agir como lhe aprouver.

-- Talvez ele devesse procurar a ajuda de um médico -- observou o outro.

-- Faz parte de suas excentricidades -- respondeu Roderich --
menosprezar por completo a medicina, pois ele acredita que toda doença
tem a sua individualidade de acordo com cada ser humano, e portanto não
pode ser curada seguindo"-se observações feitas no passado ou mesmo nas
assim chamadas teorias. Seria mais provável que ele recorresse a velhas
curandeiras e tratamentos com amuletos e talismãs. Sob outro ponto de
vista, ele igualmente menospreza toda forma de precaução e tudo o que
chamamos de ordem e moderação. Desde a infância, seu ideal foi um homem
nobre, e seu propósito mais elevado, o de tornar"-se alguém assim, o que
significa principalmente uma pessoa que desdenha tudo, a começar pelo
dinheiro. E para não despertar qualquer suspeita de ser econômico, de
gastar de má vontade, ou de algum modo ter consideração por dinheiro,
ele o joga fora da maneira mais tola possível; apesar de sua renda
abundante, está sempre pobre e em dificuldades; e serve de bobo para
qualquer um que não seja totalmente nobre no sentido em que ele se
propôs a sê"-lo. Ser seu amigo é a maior tarefa de todas as tarefas,
pois ele é tão irritável que apenas basta tossir, não comer de maneira
nobre o bastante, ou até mesmo palitar os dentes para ofendê"-lo
mortalmente.

-- Ele nunca esteve apaixonado? -- o amigo do campo perguntou.

-- Quem ele haveria de amar? -- perguntou Roderich em resposta. -- Ele
menospreza todas as filhas da terra, e no momento em que percebesse que
seu ideal gosta de enfeitar"-se, ou até mesmo dança, e seu coração se
partiria. E mais terrível ainda seria se ela tivesse o infortúnio de
contrair um resfriado.

Emil enquanto isso se encontrava em meio ao tumulto. De súbito, porém,
foi assaltado por aquele temor, aquele pânico que tão frequentemente já
havia acometido seu coração ao encontrar"-se em uma multidão assim
inflamada. Fugiu para fora do salão, correu pelas ruelas desoladas, e
somente em seu aposento solitário tomou fôlego e recobrou novamente a
tranquilidade. A luz noturna já fora acesa e ele deu ordens para que o
criado se recolhesse. Em frente tudo estava silencioso e às escuras;
ele sentou"-se para extravasar em um poema suas sensações do baile.


\begin{verse}
No coração tudo calmaria,\\
agrilhoados estavam os desatinos.\\
Maldoso desejo falou alto\\
e o louco libertou.\\
Lá se ouvem timbales;\\
em risadas irrompe\\
o trompete ensurdecedor.\\
Flautas intrometem"-se\\
e pífanos saltitam\\
com gritos aguçados\\
junto aos violinos furiosos.\\
Tumultuada balbúrdia\\
e feroz estrondo\\
abatem com selvageria o inocente silêncio. 
 
Para onde rodopia a ciranda?\\
O que procura a multidão \\
que fervilha no sinuoso tropel?\\
As luzes tremulam,\\
a folia nos aproxima,\\
o estúpido coração está jubiloso,\\
que soem mais alto os címbalos,\\
que retumbem mais desvairados os apitos!\\
Trazei torpor ao sofrimento\\
Que ele se transforme em pilhéria! 

Tu me acenas, face adorável?\\
A boca sorri, os olhos cintilam;\\
venha aos meus braços,\\
até o giro nos separar outra vez.\\
A beleza findará um dia, bem o sei,\\
Os lábios emudecerão,\\
A morte a tomará nos braços.\\
Por que me acenas, crânio amável?\\
Virás hoje, talvez amanhã?\\
Eu ondeio na ciranda e passo por ti.

Nessa vertigem de prazeres\\
e nesses risos de hoje\\
talvez brote o veneno.\\
Oh, tempo esplêndido!\\
Acena"-me a bela\\
e torna"-se minha noiva,\\
enquanto a outra espreita\\ 
com atrevida petulância.\\
O que há de ser?
 
Cambaleamos todos\\
pelo salão dos nossos anos,\\
sem amor, sem vida, sem existência,\\
somente o sonho, somente a sepultura.\\
As flores e o trevo encobrem lá embaixo\\ 
horrores bem maiores,\\
dores mais lancinantes.\\
Que soem mais alto os címbalos e \qb{timbales}.\\
Que bradem mais forte as trompas!\\
Pulemos, giremos em roda sem descanso.\\
O amor não nos brindou\\
com vida ou com coração.\\
Com regozijo dançamos para o abismo \qb{funesto}! 
\end{verse}


Ele havia concluído e encontrava"-se diante da janela. Então, lá do outro
lado, ela apareceu, tão bela como nunca a vira antes. O cabelo castanho
desamarrado permitia que cachos rebeldes balançassem e ondeassem à
vontade sobre a alva nuca. Cobria"-se com uma roupa muito leve e tarde
da noite parecia ainda querer executar alguma tarefa doméstica antes de
recolher"-se, pois colocou velas em dois cantos do aposento, endireitou
o tapete sobre a mesa, e retirou"-se novamente. Emil ainda estava imerso
em doces sonhos, recordando na imaginação a visão de sua amada, quando,
para seu horror, a medonha velha escarlate atravessou o aposento
enquanto o ouro em sua cabeça e seu colo reluzia de forma hedionda à
luz das velas. Pouco depois, tinha desaparecido. Poderia acreditar em
seus olhos? Não teria sido alguma ilusão provocada pela noite e
fantasmagoricamente encenada pela sua própria fantasia?

Não. Ela reapareceu ainda mais medonha do que antes: agora longos
cabelos negros e grisalhos volteavam, selvagens e desordenados, sobre o
peito e as costas. A formosa jovem acompanhava"-a pálida, desfigurada,
os formosos seios descobertos, sua figura assemelhando"-se inteiramente
a uma estátua de mármore. Elas seguravam entre si a adorável menina,
que chorava e se aconchegava suplicante à bela, que não lhe dirigia o
olhar. A criancinha erguia suas pequeninas mãos em rogos, acariciava o
pescoço e as faces da pálida bela. Esta, no entanto, mantinha"-a
fortemente presa pelos cabelos, segurando com a outra mão uma bacia
prateada. Murmurando algumas palavras, a velha vibrou uma faca e cortou
o alvo pescoço da menina. Então, avultou"-se atrás delas algo que ambas
pareciam não enxergar, pois com certeza haveriam de ficar tão aterradas
como Emil. Um repelente e escamoso pescoço de dragão serpenteou para
fora das trevas, alongando"-se sempre mais e mais, e curvou"-se sobre a
criança que pendia inerte nos braços da velha. A língua negra sorveu
então do sangue rubro que jorrava em profusão, e um faiscante olho
verde, alcançando a fresta do outro lado da rua, cruzou"-se com o olhar
de Emil, e atingiu seu cérebro e seu coração. No mesmo instante Emil
caiu desfalecido.

Roderich encontrou"-o inanimado depois de algumas horas.

\asterisc

Em uma alegre manhã de verão, um grupo de amigos estava sentado em um
caramanchão verde junto a um saboroso desjejum. Em meio a risos e
gracejos, todos levantavam seus copos para brindar à saúde e felicidade
do jovem casal de noivos. O noivo e a noiva não estavam presentes, pois
a bela ainda estava ocupada em enfeitar"-se, e o jovem noivo vagueava ao
léu, ponderando sobre sua sorte, enquanto passeava solitário por uma
retirada aleia de árvores.

-- É uma pena -- disse Anderson -- que não possamos ter música. Todas
as damas presentes estão insatisfeitas; nunca desejaram tanto dançar
como justamente hoje, quando isso lhes é impossível. Mas a música
desagrada tanto a ele\ldots{}

-- Posso revelar"-vos -- disse um jovem oficial -- que a despeito disso teremos
um baile e que será um baile absolutamente desvairado e barulhento.
Tudo está organizado; os músicos já chegaram em segredo e se alojaram
sem que ninguém percebesse. Roderich cuidou de todos esses preparativos
e disse que hoje, mais do que nunca, não devemos ceder a todas as suas
excentricidades.

-- É bem verdade que ele já está muito mais humano e afável que
antes -- exclamou outro rapaz --, e por isso creio que ele não verá tais
alterações como algo desagradável. O próprio matrimônio, aliás, foi
algo tão súbito que surgiu contra todas as nossas expectativas.

-- Sua vida inteira -- continuou Anderson -- é tão inusitada quanto seu
caráter. Vós todos sabeis que ele chegou em nossa cidade no último
outono por ocasião de uma viagem que pretendia realizar, que se alojou
aqui durante o inverno, que quase só permanecia enclausurado em seus
aposentos como um melancólico, e que desdenhou nosso teatro e todos os
demais entretenimentos. Quase tinha rompido sua amizade com Roderich,
seu amigo mais íntimo, porque este tentara fazer com que se divertisse
e não cedera ao seu mau"-humor. Afinal, sua exagerada irritabilidade e
má vontade parecem ter resultado de uma doença que se aninhava em seu corpo,
pois como não deveis ignorar, quatro meses atrás ele foi acometido
por uma febre nervosa tão severa que nós já o dávamos por perdido.
Depois que seus delírios se acalmaram e ele novamente voltou a si,
tinha ficado quase por completo sem memória, restando"-lhe apenas a
lembrança de seus anos de infância e adolescência, de modo que não
conseguia recordar de nada da viagem ou do que ocorrera antes da
doença. Foi"-lhe necessário travar conhecimento outra vez com todos os
seus amigos, inclusive Roderich. Só pouco a pouco o seu íntimo foi"-se
alumiando, e o passado e tudo o que ele havia vivido foi retornando a
sua memória, porém sempre apenas em poucos rasgos. Seu tio havia"-o
trazido para sua casa a fim de tê"-lo junto a si e poder melhor cuidar
dele, pois estava como se fosse uma criança e deixava que fizessem tudo
com ele. Na primeira ocasião em que o levaram a passeio na primavera e
ele visitou o parque, viu uma moça sentada em um recanto apartado,
profundamente mergulhada em pensamentos. Ela levantou os olhos, seu
olhar encontrou"-se com o dele, e, como que arrebatado por um
incompreensível deslumbramento, ele mandou que parassem a carruagem,
desceu, sentou"-se a seu lado, tomou suas mãos, e verteu uma torrente de
lágrimas. Quem o acompanhava ficou novamente preocupado com seu juízo,
mas ele logo acalmou"-se, mostrando"-se alegre e falante. Apresentou"-se
aos pais da moça, e já na primeira visita pediu"-a em casamento, com o
que ela aquiesceu, já que seus pais não haviam recusado seu
consentimento. Ele estava feliz e revigorado; a cada dia estava mais
saudável e contente. Foi assim que, oito dias atrás, ele chegou aqui,
em minha propriedade no campo, para visitar"-me. Tudo lhe agradou
sobremaneira e em tal grau que não descansou enquanto eu não a vendi
para ele. Se quisesse, poderia muito bem ter"-me aproveitado e usado sua
paixão a meu favor e em sua desvantagem, pois quando ele deseja algo, é
arrebatado e intempestivo, fazendo de tudo para seu desejo realizar"-se
no mesmo instante. Imediatamente colocou em andamento os preparativos,
mandando vir móveis a fim de já viver aqui os meses de verão. E é por
isso que todos nós estamos aqui hoje, reunidos em minha antiga residência.

A casa era ampla e situada em uma das áreas mais bonitas. Uma das duas
laterais dava para um rio e aprazíveis colinas, havendo múltiplas
vegetações e árvores em toda a volta. Diretamente em frente abria"-se um
jardim com flores perfumadas. Ali, laranjeiras e limoeiros estavam
acomodados em um amplo abrigo, do qual pequenas portas conduziam a
despensas, porões e armazéns de alimentos. Do outro lado estendia"-se um
verde prado que terminava em um parque, para o qual não havia outra
entrada. Nesse ponto as duas longas alas da casa formavam um espaçoso
terreno, e, assentados sobre fileiras de colunas sobrepostas, três
andares de corredores abertos e largos interligavam todos os aposentos
e salas do edifício. Com isso, este lado da \mbox{residência} ganhava um
aspecto fascinante, até mesmo encantado, uma vez que nesses alpendres
espaçosos aparecia toda sorte de figuras nas mais variadas atividades.
\mbox{Caminhando} entre as colunas e saindo de cada aposento surgiam sempre
novas pessoas, reaparecendo depois em cima ou embaixo para em seguida
desaparecer em uma outra porta. Também era ali que os convidados se
reuniam para o chá e para o jogo, e com isso o conjunto, visto por
baixo, ganhava os ares de um teatro, diante do qual todos podiam parar
com prazer, na expectativa de um inaudito e divertido espetáculo.

O grupo de jovens convivas pretendia justamente levantar"-se, quando a
noiva, coberta de enfeites, atravessou o jardim e caminhou em sua
direção. Seu vestido era de veludo violeta, um colar faiscante
balançava no pescoço esplendoroso, preciosas rendas deixavam entrever
os seios amplos e alvos, o cabelo castanho coloria"-se de modo
encantador com as murtas e outras flores da grinalda. Cumprimentou a
todos com muita afabilidade, e os rapazes ficaram surpresos com sua
grande beleza. Depois de colher flores no jardim, ela dirigiu"-se para o
interior da casa a fim de acompanhar o preparo do banquete. As mesas
haviam sido dispostas no amplo corredor do térreo, onde elas
fulguravam com suas toalhas níveas e cristais; uma abundância de flores
nas mais variadas cores transbordava cintilante de vasos delicados,
fragrantes grinaldas verdes e coloridas enredavam"-se nas colunas. Era
adorável a imagem da noiva, que encantadoramente ia e vinha em meio ao
brilho das flores e por entre as mesas e colunas enquanto
supervisionava tudo. Depois ela desapareceu, reaparecendo no andar de
cima para abrir a porta de seu quarto.

-- Ela é a moça mais charmosa e bonita que já vi -- exclamou
Anderson. -- Nosso amigo tem sorte!

-- Até mesmo sua palidez -- comentou o oficial -- aumenta sua beleza:
as faíscas de seus olhos castanhos são ressaltadas pelas faces pálidas
e o cabelo escuro. E o maravilhoso, quase ardente rubor dos
lábios dá a seu semblante uma aparência que realmente enfeitiça.

-- O halo de recôndita melancolia que a circunda -- disse Anderson --
tem o efeito de uma moldura majestática.

O noivo juntou"-se a eles e perguntou por Roderich. Fazia tempos que não
o viam e ninguém sabia onde ele se achava. Todos foram procurá"-lo.

-- Ele está lá embaixo, no salão -- informou por fim um jovem a quem
se dirigiram --, em meio aos criados e cocheiros, para quem está
apresentando truques com cartas, com absoluto sucesso.

Ao chegarem, interrompendo as ressonantes demonstrações de assombro da
criadagem, Roderich não se deixou perturbar e continuou livremente a
produzir seus truques de magia. Quando terminou, seguiu com os demais
até o jardim e disse:

-- Faço isso apenas para fortalecer a fé dessas pessoas, pois estas
artes causam um abalo duradouro no livre"-pensamento típico desses
cocheiros e ajudam a convertê"-las.

-- Vejo -- disse o noivo -- que meu amigo, dentre seus muitos talentos,
também não deixa de cultivar o de charlatão.

-- Vivemos em uma época fabulosa -- respondeu --, hoje em dia não
se deve menosprezar nada, pois nunca se sabe que serventia terá.

Quando os dois amigos ficaram a sós, Emil caminhou novamente para a
sombreada aleia de árvores e disse:

-- Por que estou tão tristonho neste dia que deveria ser o mais
feliz de minha vida? Talvez não me acredites, mas asseguro"-te que não
combina comigo estar em meio a uma quantidade tão grande de pessoas,
dar atenção a todos, não negligenciar nenhum dos parentes dela nem os meus,
demonstrar reverência pelos pais, cumprimentar as damas, recepcionar os
recém"-chegados e prover que sejam adequadamente acomodados os serviçais
e cavalos.

-- Isto tudo se resolve por si mesmo -- exclamou Roderich --, vê, tua
casa parece feita sob medida para esse tipo de ocasião; teu mordomo,
que tem as mãos cheias de afazeres e não de lazeres, é perfeito para
colocar tudo em seu devido lugar, para salvar os convidados de toda e
qualquer confusão, e para atendê"-los com dignidade. Deixa tudo por
conta dele e de tua formosa noiva.

-- Hoje cedo, antes mesmo do amanhecer -- Emil disse -- vaguei pelo
arvoredo. Estava imbuído de espírito solene, sentia fortemente em meu
íntimo o quanto minha vida agora se tornaria definida e séria, como
este amor me deu lar e ocupação. Passei lá adiante pelo caramanchão e
ouvi vozes: era minha amada em uma conversa privada. ``Não sucedeu
tudo'', perguntou uma voz desconhecida, ``conforme eu havia dito? Não
está tudo exatamente da maneira como eu sabia que aconteceria?
Realizastes vosso desejo; ficai, portanto, satisfeita''. Não quis ir até
lá; mais tarde aproximei"-me do caramanchão, porém ambas já haviam ido
embora. Mas não consigo deixar de me perguntar vez após vez: o
significam aquelas palavras?

Roderich disse:

-- Talvez ela já te amasse há muito tempo, sem que o soubesses;
então serias ainda mais feliz.

Um rouxinol retardatário iniciava agora seu canto e parecia entoar
desejos de harmonia e regozijo aos apaixonados. Emil ficou mais
pensativo.

-- Vem comigo para te alegrares -- disse Roderich -- até a aldeia lá
embaixo. Verás um segundo casal em núpcias, pois não penses que és o
único a celebrar matrimônio hoje. Um jovem servo estava por demais
solitário e entediado, e por isso envolveu"-se com uma megera velha e
torpe. O tolo agora pensa que está obrigado a fazer dela sua esposa.
Nesse momento os dois já devem estar arrumados. Não percamos esse
quadro, pois com certeza é interessante.

O tristonho deixou"-se arrastar por seu alegre e falante amigo. Em pouco
chegaram à cabana, onde o cortejo estava justamente saindo para a
igreja. O servo trajava sua costumeira túnica de linho, distinguindo"-se
apenas por suas calças de couro, as quais havia pintado com as cores
mais claras possíveis. Sua expressão era simplória e ele parecia
encabulado. A noiva era queimada pelo sol; só alguns poucos resquícios
de juventude ainda eram visíveis nela. Estava vestida em roupas grossas
e pobres, porém limpas, e algumas tiras de seda vermelhas e azuis, já
algo descoradas, tremulavam de seu justilho. Mas o que realmente a
desfigurava era que haviam endurecido seu cabelo com gordura e farinha,
puxado"-o para longe da testa e prendido"-o com agulhas no alto da
cabeça. No topo disso estava a grinalda. Ela sorria e parecia alegre,
mas era tímida e estúpida. Os velhos pais seguiam atrás; o pai também
era apenas servo na quinta; e a cabana, os móveis e também a roupa,
tudo traía a mais completa pobreza. Um músico sujo e estrábico
acompanhava o séquito, fazendo caretas e tocando um violino feito de
pedaços de papelão e madeira grudados com visco, e que, em vez das
cordas, tinha três barbantes alinhados. O cortejo parou quando o jovem
senhor se juntou às pessoas. Alguns serviçais maliciosos, jovens
criados e empregadas faziam comentários jocosos e riam, zombando do
casal de noivos, especialmente as camareiras, que se consideravam mais
bonitas e estavam infinitamente mais bem vestidas. Um calafrio apossou"-se
de Emil, que se voltou para Roderich, mas este já tinha escapulido. Um
criado impertinente com cabelos à maneira de Titus,\footnote{ Penteado inspirado
no imperador romano Tito (século \textsc{i}): os cabelos são usados curtos
e cacheados.} empregado de um estranho, foi abrindo caminho para
aproximar"-se de Emil e, pretendendo parecer engraçado, exclamou:

-- E agora, meu digno senhor, o que dizeis do lustroso par de
noivos? Os dois ainda não sabem de onde irão tirar o pão de amanhã, mas
hoje à tarde irão oferecer um baile; o virtuoso ali já está encomendado.

-- Eles não têm pão? -- perguntou Emil. -- Existe algo semelhante?

-- Toda a miséria deles é conhecida das pessoas -- o outro continuou
tagarelando --, mas o sujeito diz que é afeiçoado a essa criatura, ainda
que ela não tenha trazido nada! Oh, sim, deveras, o amor é onipotente!
O bando de maltrapilhos nem tem camas, hoje precisarão dormir em cima
de palha; tiveram que pedir como esmola a cerveja com que pretendem se
embebedar.

Todos ao redor soltaram altas gargalhadas, e os dois infelizes, de quem
escarneciam, baixaram os olhos. Furioso, Emil afastou de si o palrador.

-- Tomai! -- gritou e enfiou cem ducados, que tinha recebido pela
manhã, na mão do estarrecido noivo.

Os anciãos e o par de noivos choraram alto, desajeitadamente lançaram"-se
de joelhos e beijaram suas mãos e vestes. Ele quis desvencilhar"-se.

-- Mantende com isso as privações longe de vós por quanto tempo
puderdes! -- exclamou atordoado.

-- Oh, até o final da vida, meu digno senhor, seremos felizes! --
exclamaram todos.

Ele não soube como tinha conseguido escapar. Estava sozinho e
apressava"-se pela floresta adentro com passos cambaleantes. Buscou o
lugar mais solitário e fechado, e assentou"-se sobre um outeiro coberto
de grama, não segurando mais o fluxo de lágrimas que agora rompia
livremente.

-- A vida me repugna! -- soluçava com profunda comoção. -- Não posso
estar contente e ser feliz; eu não quero isto! Recebe"-me logo, terra
amiga, esconde"-me em teus braços gélidos contra os animais selvagens,
que se chamam seres humanos! Oh, Deus do céu, como é possível que eu
repouse em frouxéis e me vista com sedas, que a uva me doe seu sangue
mais precioso, e o mundo todo insista em me ofertar dádivas de honra e
amor? Aquele necessitado é melhor e mais nobre que eu, a penúria é
sua ama de leite; desprezo e escárnio venenoso são seu brinde.
Pecaminosa parece"-me cada iguaria que saboreio, cada gole tomado em
copo polido, meu descanso em camas macias, o uso de ouro e joias, pois
que muitas milhares de vezes o mundo acossa milhares de infelizes, que
para matar a fome se contentam com o pão seco jogado fora, e que não
sabem o que é alívio. Oh, eu compreendo a vós santos piedosos, vós
rejeitados, vós escarnecidos, que doastes aos pobres tudo, até as
roupas do corpo, que amarrastes um saco em torno da pelve e, saindo
pelo mundo igualmente como mendigos, quisestes suportar as injúrias e
pontapés com que a arrogância bruta e o esbanjamento estróina expulsam
a miséria de suas mesas; vós próprios vos tornastes desditosos apenas
para afastar de si esse pecado da abundância.

Todas as imagens do mundo flutuavam como uma névoa ante seus olhos!
Decidiu"-se a ver todos os excluídos como seus irmãos e apartar"-se dos
felizes. Já há muito ele era aguardado no salão para a cerimônia de
casamento; a noiva estava apreensiva, os pais procuravam"-no no jardim e
parques. Por fim ele retornou, aliviado das lágrimas e do peso que
sentira, e a celebração solene foi realizada.

Todos transferiram"-se do salão térreo para o alpendre aberto para tomar
lugar às mesas. A noiva e o noivo abriam o caminho e os demais seguiam
atrás em cortejo. Roderich ofereceu o braço para uma jovem senhorita
que era vivaz e falante.

-- Por que será que as noivas sempre choram e durante o casamento
parecem tão taciturnas? -- perguntava ela enquanto subiam à galeria.

-- Porque neste momento, mais do que nunca, elas percebem a
importância e o mistério da vida -- Roderich respondeu.

-- A nossa noiva, porém, continuou aquela, ultrapassa em seriedade
a todas que já vi. Aliás, ela está sempre soturna e nunca é possível
vê"-la rindo com verdadeira alegria.

-- Isto mostra o quanto é nobre seu coração -- respondeu Roderich
indisposto, contrariamente a seus hábitos. -- Talvez a senhorita não saiba
que alguns anos atrás a noiva havia abrigado uma órfã, uma menina
absolutamente encantadora, a fim de educá"-la. A essa criança dedicava
todo o seu tempo, e o amor da tenra criatura era sua mais doce paga.
Esta menina tinha completado sete anos de idade quando se perdeu na
cidade durante um passeio, e apesar de todos os esforços empregados,
ainda não foi encontrada. Este acidente abalou a nobre donzela de tal
forma que, deste então, ela sofre de uma melancolia recôndita, e nada
consegue acalmar a saudade que sente por sua pequena companheira.

-- De fato, muito interessante! -- exclamou a jovem. -- Isto pode ter
desdobramentos românticos no futuro e dar a oportunidade para um
delicioso poema.
 
Todos se acomodaram à mesa; a noiva e o noivo tomaram lugar no centro,
tendo à frente a risonha paisagem. A conversa ia animada, muitos
brindes foram feitos, reinava uma absoluta atmosfera de júbilo, os pais
da noiva estavam bastante felizes, e somente o noivo estava quieto e
absorto em si mesmo, degustando muito pouco e não participando das
conversações. Ele assustou"-se quando sons de música foram derramados do
andar de cima; porém tranquilizou"-se de novo ao perceber que não
passavam de timbres suaves de trompa, que agradavelmente sussurravam sobre
os arbustos e se estiravam pelo parque, indo perder"-se na montanha ao
longe. Roderich tinha alocado os músicos na galeria mais alta, acima
dos comensais, e Emil ficou satisfeito com este arranjo. Perto do final
da refeição, mandou vir seu mordomo e, dirigindo"-se à noiva, disse: 

-- Querida amiga, permite que também os pobres tomem parte de nossa
abundância.

A seguir ordenou que várias garrafas de vinho, assados e outros pratos,
em fartas porções, fossem enviados ao par de noivos desafortunado a fim
de que este dia também fosse um dia de alegria do qual posteriormente
viessem a se lembrar com prazer.

-- Vê, amigo -- exclamou Roderich --, como tudo está tão bem ligado no
mundo! Meus circunlóquios e palavreados inúteis, que tão frequentemente
me reprovastes, ao final motivaram essa boa ação.

Muitos quiseram elogiar o anfitrião pela sua misericórdia e bom coração,
e a senhorita falou de bela atitude e magnanimidade. 

-- Fiquemos calados! -- disse Emil encolerizado. \mbox{-- Não} é nenhuma boa
ação, aliás, sequer é uma ação, não é nada! Se andorinhas e pintarroxos
se alimentam das migalhas que esta abundância deixou cair e as levam
para seus filhotes nos ninhos, então não haveria eu de me lembrar de um
irmão pobre que necessita de mim? Se eu pudesse seguir meu coração,
então vós decerto iríeis rir e escarnecer de mim da mesma maneira como
de outros, que rumaram para o deserto a fim de nada mais ouvir do mundo
e sua magnanimidade.

Todos estavam em silêncio, e Roderich reconheceu nos olhos em brasa do
amigo uma indignação das mais veementes. Querendo fazer com que ele
esquecesse sua irritação, rapidamente tentou guiar a conversa para
outros assuntos. Emil, no entanto, tinha se tornado inquieto e
distraído; principalmente seus olhos iam com frequência até a galeria
superior, onde os criados que habitavam o andar de cima se ocupavam de
vários afazeres. 

-- Quem é a repugnante velha que lá está tão atarefada e tantas
vezes passa com seu casaco cinza? -- perguntou finalmente.

-- Ela está a meu serviço -- disse a noiva --, e tem sob sua
responsabilidade as camareiras e criadas mais jovens.

-- Como podes tolerar tamanha feiura perto de ti? -- Emil perguntou.

-- Deixa"-a em paz! -- respondeu a jovem. -- Também os feios precisam
viver, e como ela é boa e honesta, pode nos ser de grande utilidade.
 
Todos se levantaram e cercaram o jovem cônjuge, mais uma vez
desejaram"-lhe boa sorte, e instaram com pedidos de permissão para o
baile. A noiva abraçou"-o muito afetuosamente e disse:

-- Meu amado, com certeza não haverás de negar meu primeiro pedido,
pois todos aguardamos isso com ansiedade. Não danço há um longo tempo e
tu nunca me viste dançando. Não estás curioso para ver como me saio
nesses movimentos?

-- Nunca te vi tão jovial -- disse Emil. -- Não quero ser um obstáculo à
tua alegria. Que apenas ninguém demande de mim que eu faça papel de
ridículo com saltos desajeitados.

-- Se fores um mau dançarino -- ela disse rindo --, podes estar seguro
de que todos irão deixar"-te em paz.

A seguir a noiva retirou"-se para trocar de roupa e colocar seu vestido
de baile.

-- Ela não sabe -- disse Emil a Roderich, com o qual ia caminhando -- 
que posso atravessar de um outro aposento para o dela usando uma porta
oculta. Vou surpreendê"-la enquanto se veste.

Quando Emil havia saído e muitas das damas também se afastaram para
providenciar as alterações de vestuário necessárias para a dança,
Roderich juntou"-se aos jovens e conduziu"-os a seu aposento.

-- Já está entardecendo -- disse ali. -- Logo estará escuro. Que agora
cada um coloque sua fantasia para que esta noitada seja bem divertida e
estouvada. Tudo o que puderdes inventar: não ficai constrangidos,
quanto mais extravagante, melhor! Quanto mais horrendas as caretas que
produzirdes, tanto maior será meu \mbox{elogio}. Não deverá haver nenhuma
corcova asquerosa demais, nenhuma barriga monstruosa demais, nenhum
traje absurdo demais para não ser exibido hoje. Um casamento é um
evento tão fabuloso: uma situação totalmente nova e inaudita é amarrada
no pescoço dos casados de modo tão repentino quanto um conto de fadas.
Nada nesta festa é suficientemente disparatado e insensato, pois só
assim poderá representar aos noivos tal mudança súbita e levá"-los a
flutuar para seu novo estado como em um sonho fantástico. Façamos então
desta noite algo totalmente tresloucado, e não aceitai nenhum argumento
daqueles que queiram se comportar com bom senso.

-- Não te preocupes -- disse Anderson. -- Trouxemos da cidade uma grande
mala repleta de máscaras e muitas peças de vestuário coloridas e
bizarras. Tu mesmo irás te surpreender.

-- Vede o que comprei de meu alfaiate! -- disse Roderich. -- É um
tesouro precioso que ele estava prestes a retalhar para fazer
toalhinhas! Ele havia adquirido este vestido de uma velha comadre que
certamente o utilizava para visitar Lúcifer no Blocksberg\footnote{
``Blocksberg'' é o nome da montanha mais alta da região de Harz, no norte
da Alemanha. Segundo antigas lendas, ali era o local de reunião de
bruxas e bruxos para realização de orgias e culto ao diabo. Goethe
ambientou a cena ``Noite de Valpúrgis'' de \textit{Fausto \textsc{i}} (publicado em
1808) no Blocksberg.} em roupas de gala. Vede este justilho
escarlate, com os galões e franjas dourados; este gorro que rebrilha a
ouro, o qual certamente há de me assentar de modo mui venerável; com
eles usarei uma saia de seda verde com enfeites amarelo"-açafrão e esta
máscara horrorosa. Assim, irei como anciã conduzir todo o grupo de
caricaturas para o quarto de dormir. Apressai"-vos para ficardes logo
prontos! Então iremos buscar solenemente a jovem esposa.

As trompas ainda tocavam, os convivas vagavam pelo jardim ou
encontravam"-se sentados defronte a casa. O sol havia se posto atrás de
nuvens sombrias, e a região estava imersa em desolado crepúsculo,
quando repentinamente um raio resplandecente irrompeu mais uma vez
debaixo da coberta de nuvens, e os arredores, mas principalmente o
edifício com seus corredores, colunas e circunvoluções de flores, ficou
como se estivesse borrifado com sangue escarlate. Foi então que os pais
da noiva e os demais espectadores viram o cortejo mais espalhafatoso
possível subir ao corredor de cima: Roderich ia adiante como a velha
rubra, e atrás vinham corcundas, caraças com amplos ventres, perucas
monumentais, Tartaglias, Pulcinellas e fantasmagóricos
Pierrôs, figuras femininas em amplas saias de aros e
penteados altíssimos, e uma variedade de figuras repelentes, todas como
se tivessem saído de um pesadelo. Eles avançavam fazendo brincadeiras,
girando e cambaleando, tropeçando e empertigando"-se pelo corredor até
desaparecerem em uma das portas. Apenas alguns dos espectadores tinham
chegado a rir, tamanha havia sido a surpresa diante daquela visão. De
repente soou um grito agudo que vinha dos aposentos internos, e para o
pôr do sol sangrento precipitou"-se a noiva pálida, que usava um vestido
branco curto no qual esvoaçavam pâmpanos de flores; os lindos seios
estavam totalmente descobertos, a plenitude dos \mbox{cachos} flutuava pelos
ares. Como ensandecida, os olhos revolvendo"-se, a face desfigurada, ela
correu pela galeria, e, cegada pelo medo, não encontrou nenhuma porta
ou escada. Um momento depois, Emil atirava"-se em seu encalço empunhando
bem alto o lustroso punhal turco. Ela chegou ao final do corredor, não
podia avançar mais, ele a alcançou. Os amigos mascarados e a velha
cinzenta tinham"-se lançado atrás dele. Mas ele já tinha perfurado
furiosamente seu peito e cortado o alvo pescoço; o sangue dela brotava
ao clarão do crepúsculo. A velha tinha se engalfinhado com ele para
detê"-lo; em luta, ele atirou"-se com ela por cima da balaustrada, e
ambos tombaram dilacerados aos pés dos parentes, que haviam
presenciado, emudecidos de horror, a cena sangrenta. No andar de cima e
no alpendre, ou das galerias e escadas acorriam, estavam de pé e
atropelavam"-se os abomináveis mascarados em variados grupos,
semelhantes a demônios infernais.

Roderich tomou o moribundo nos braços. Ele o havia encontrado brincando
com o punhal no aposento de sua esposa. Ela estava quase arrumada
quando ele entrou. À vista do ignóbil vestido rubro, a memória de Emil		\EP[-1]
se recobrou, o pesadelo daquela noite descortinou"-se aos seus sentidos;
rangendo os dentes, saltara sobre a trêmula noiva que se pôs em fuga.
Pretendia castigar o assassinato e a diabólica feitiçaria que
ela realizara. Antes de morrer, a velha confirmou o ato hediondo. De um
momento para outro, a casa inteira estava agora tomada de tristeza, luto e horror.
\medskip

\hfill\textit{Tradução de Karin Volobuef}

\chapter{O cálice}

\textsc{Da grande catedral} ressoavam as badaladas matinais. Na praça ampla,
homens e mulheres passeavam em diversas direções, carruagens
transitavam e padres se encaminhavam às suas igrejas. Ferdinand estava
de pé na larga escadaria, seguindo com os olhos o movimento, e
observando as pessoas que subiam para participar da celebração. 

Os raios de sol se refletiam das pedras brancas, todos buscavam sombra
se protegendo contra o calor que dali se irradiava; somente ele se
mantinha meditando longamente encostado a uma coluna sob o calor
escaldante sem ao menos senti"-lo, pois se perdia em recordações de
antigas reminiscências. Cismava sobre sua vida e se entusiasmava ante o
sentimento que a impregnara fazendo com que se desvanecessem todos os seus
demais anseios. 

No ano passado, à mesma hora, ele estivera naquele lugar, a fim de ver
mulheres e moças entrando à missa: com indiferença e um sorriso
insolente, vira desfilar as figuras variegadas, algum olhar amável
tinha se deparado com seu ar zombeteiro e alguma face virginal
enrubescido; atento ele seguira com o olhar os pezinhos delicados, como
eles subiam os degraus, e as compridas saias suspensas deixavam flagrar
por um instante os belos tornozelos. 

Nisso cruzara a praça uma jovem de porte esbelto e nobre, caminhando de
olhos modestamente abaixados. Com passadas descontraídas e ligeiras,
subira os degraus com elegância graciosa. O vestido de seda envolvia o
corpo lindo e as pregas pareciam dançar à música dos membros em
movimento. Num certo momento, ela quisera dar o último passo e, por
acaso, tinha erguido os olhos e um brilho de puro azul encontrara o
olhar de Ferdinand que foi perpassado como que por um raio. Ela pisara
em falso, ele se lançara rápido para sustê"-la, mas não teve como
impedi"-la de, por um instante, cair ajoelhada aos seus pés numa pose
encantadora. Ferdinand a erguera, ela não chegara a olhá"-lo
diretamente, nem respondera à pergunta se havia se machucado, mas
tinha ficado bem vermelha. Depois ele a seguira igreja adentro,
mantendo fixamente diante dos olhos a imagem da moça ajoelhada com o
seio palpitante. No dia seguinte ele havia retornado à igreja, cujo
portal doravante lhe afigurava sagrado. 

Antes, ele tivera a intenção de deixar o vilarejo, os amigos o esperavam
impacientemente na terra natal; mas desde então esse era o seu
lugar; seu coração se convertera. Ele a reviu com frequência, ela não o
evitou; contudo, eram segundos furtivos e roubados, pois a moça vivia
sob cerrada vigilância de sua família rica e de um noivo ciumento, bem
considerado na região. Os dois trocaram confissões de amor, mas não
sabiam como agir em tais circunstâncias: ele era um forasteiro e não
poderia oferecer à amada a grande fortuna a que ela fazia jus. 
 
Ele deplorava dolorosamente a própria pobreza, mas quando refletia sobre
a vida que vinha levando, se convencia de que era um felizardo, porque
sua existência estava santificada e seu coração batia sempre ao ritmo
enlevado do sentimento de amor. A natureza agora era sua amiga, todas
as belezas se manifestavam aos seus sentidos, a contemplação e a
piedade lhe eram familiares; ele entrava pelo portal à misteriosa
escuridão da nave com uma disposição de espírito totalmente diferente
daquela dos dias inconsequentes de outrora. Afastou"-se dos círculos de
amigos e não vivia senão pelo seu amor. Se por acaso passava pela rua
onde ela morava e a via à janela, se alegrava pelo resto do dia;
constantemente se falavam à luz crepuscular do entardecer, pois o
jardim da moça se limitava com o de um amigo dele que, todavia,
ignorava o segredo. Assim um ano se passou.

Todas essas cenas de sua nova vida lhe vinham agora à lembrança, ali à
porta da igreja. Ele levantou os olhos e eis que a nobre figura chegava
através da praça com leves passadas; a ele se assemelhava a um raio de
sol surgindo em meio à confusa multidão. Um canto doce elevou"-se de seu
coração, e quando ela se aproximou, ele tornou a entrar e estendeu"-lhe
a água benta. Os dedos claros estremeceram ao tocar os de Ferdinand; a
moça inclinou"-se com uma reverência suave. Ele a seguiu e ajoelhou"-se
nas proximidades. Todo o seu coração se fundia de amor e melancolia,
como se das feridas abertas por esse sentimento seu ser transbordasse
em torrentes de orações fervorosas. Todas as palavras do sacerdote
trespassavam seu peito como um frêmito; as notas da música lhe
inspiravam a mais ardente devoção; seus lábios tremeram quando a bela
moça comprimiu apaixonadamente o crucifixo do terço contra a boca
vermelha. Como fora possível que ele um dia desconhecesse a fé e o
amor? Nesse momento o padre ergueu a hóstia, a sineta ressoou, ela
curvou"-se humildemente e fez o sinal da cruz. Foi como se um raio
intensificasse em Ferdinand energias e sentimentos mais profundos, o
altar lhe pareceu animar"-se, a luz multicor dos vitrais adquiriu o
esplendor do paraíso. Lágrimas abundantes jorravam de seus olhos e
amenizavam o ardor da devoção.

A celebração chegou ao fim. Ferdinand mais uma vez apresentou a água
benta àquela que amava, ambos trocaram breves palavras e ela se
afastou. Ele ficou para trás, para não atrair as atenções; perseguiu"-a
com o olhar até que a barra do vestido sumisse na esquina extrema do
adro. Sentiu a angústia do viajante que, perdido em espessa floresta,
vê se extinguir a última luz crepuscular. Despertou de seus devaneios,
quando uma mão ossuda e seca tocou"-lhe as espáduas, ao mesmo tempo em
que alguém o chamava pelo nome. 

Ferdinand virou"-se e reconheceu o tímido Albert, que vivia à margem do
convívio com as pessoas, e cuja casa solitária não se abria senão para
ele.

-- Você ainda se lembra de nosso encontro? -- perguntou a voz rouca.

-- Claro que sim! -- respondeu o moço. -- E o senhor cumprirá hoje a promessa? 

Ao que o amigo disse que prontamente, se ele o acompanhasse. 

Os dois foram andando juntos pela cidade afora até uma rua deserta e
adentraram a um sobrado amplo. 

-- Hoje você precisa vir comigo à casa dos fundos, onde há um
aposento isolado e não seremos incomodados. 

Passaram por muitos cômodos, subiram lances de escadas, se enveredaram
por corredores. Ferdinand, que anteriormente estivera ali algumas vezes
visitando o amigo, se admirava agora ainda mais com o número de salas e
o formato incomum daquele enorme sobrado. Se surpreendia, sobretudo,
pelo fato de o velho, um solteirão sem família, habitar ali somente com
um criado, sem jamais ter se interessado em alugar a alguém uma parte
do espaço desnecessário. 

Enfim, Albert estacou ante uma porta e disse:

 -- Bem, cá estamos!

Entraram em um quarto de pé"-direito elevado, revestido de damasco rubro
emoldurado por ripas douradas; os móveis do ambiente estavam estofados
com o mesmo tecido e através das pesadas cortinas de seda vermelha, que
se mantinham fechadas, filtrava"-se uma luz púrpura. 

-- Me espere um instante! -- pediu o velho, indo a uma sala contígua.


Ferdinand se deteve nesse ínterim observando alguns livros espalhados,
nos quais viu caracteres estranhos, indecifráveis círculos e traços,
bem como desenhos singulares. Pelo pouco que pôde ler lhe pareceram
tratados de alquimia. Corria a fama de que o velho se dedicava à
fabricação de ouro. Em cima da mesa estava um alaúde com incrustações
incomuns de pérolas e de madeiras multicores, que representavam figuras
refulgentes de pássaros e flores. A estrela no centro fora esculpida
num grande pedaço de pérola, magnífico trabalho artístico, no qual
numerosas figuras em filigrana dispostas em círculos lembravam rosáceas
de igrejas góticas.

-- Ah, você encontrou meu velho alaúde! -- disse Albert ao voltar. --
Ele tem mais de duzentos anos, eu o trouxe como lembrança de minha
viagem à Espanha. Mas deixe"-o num canto e sente"-se aqui!

Ambos se sentaram à mesa, forrada da mesma maneira com tapeçaria
vermelha; o velho depositou sobre o tampo um objeto embrulhado. 

-- Por pura compaixão por sua juventude -- disse ele --, eu recentemente
lhe prometi predizer se no futuro você poderá ser feliz ou não, e é
chegado o momento de cumprir a promessa, embora você talvez pense que
se trate de embuste ou brincadeira. Não tenha medo, o que farei não
constitui perigo nenhum. Eu não lançarei conjurações aterradoras, nem
trarei aparições apavorantes. Minha tentativa pode fracassar em dois
casos: se seu amor não for sincero como me fez crer, meus esforços
serão estéreis e nada verei; se você estorvar o oráculo com perguntas
tolas, gestos bruscos, se levantando abrupto da cadeira, por exemplo,
arruinará a imagem. Portanto, você tem de me prometer que se comportará
com serenidade. 

Ferdinand deu sua palavra e o velho descobriu o objeto embrulhado que
trouxera. Era um cálice de ouro cinzelado com muita habilidade. Em
torno do pé largo, uma guirlanda de mirtilos se entremeava juntamente
com frutos e folhagens num relevo primoroso de ouro fosco e brilhante.
Um aro semelhante, ainda mais rico, configurava figuras miúdas
representando formas de minúsculos animais selvagens que fugiam ou
brincavam com crianças, e perfaziam o ornamento à meia altura do copo.
A borda já vinha abaulada, se recurvando ligeiramente para trás em
direção aos lábios; no interior refulgia o ouro vermelho. O velho
colocou o cálice entre ambos sobre a mesa, e fez um sinal, pedindo a
Ferdinand que se aproximasse.

 -- Você sente alguma coisa, quando seus olhos se perdem nesse
brilho? -- perguntou.

-- Sinto o fulgor se refletindo em minha alma; é como um beijo em
meu peito apaixonado -- respondeu o moço.

-- Muito bem! -- disse o velho. -- Agora concentre o olhar, mantenha"-o
fixo no ouro brilhante e pense tão intensamente quanto possível na
mulher amada. 

Os dois permaneceram em silêncio, absorvidos na contemplação do cálice
maravilhoso. Logo depois, o velho passou ainda silenciosamente a
esfregar com o dedo a \mbox{superfície} brilhante do cálice traçando círculos
regulares, a princípio lentos, em seguida mais rápidos, finalmente com
uma velocidade impressionante. Ele interrompeu o gesto e recomeçou no
outro sentido. O velho vinha procedendo aquele rito há alguns
instantes, quando Ferdinand acreditou estar ouvindo música, mas
proveniente do exterior, de uma rua distante, em breve os sons se
fizeram mais próximos, vibraram cada vez mais nítidos e distintos pelo
ar, e enfim não restava dúvida de que reverberavam do interior do cálice.

A música foi se tornando mais sonora, adquirindo uma potência tão
penetrante que o coração do moço palpitava e as lágrimas lhe assomavam
aos olhos. Diligente, a mão do velho continuava a friccionar em
diferentes direções a borda do cálice e era como se seus dedos
extraíssem faíscas cintilantes do ouro fosforescente e provocasse
estalidos. Após uns momentos, os pontinhos luminosos se multiplicaram e
se alinharam como sobre um fio, seguindo todos os movimentos da mão;
eles chispavam multicores e se apertavam cada vez mais densamente uns
aos outros, até que se mesclaram e conformaram linhas cheias. 

Agora era como se à claridade lusco"-fusco e avermelhada do ocaso, o velho
estendesse uma rede mágica sobre o ouro brilhante, pois à vontade ele
manejava os raios e com eles tecia agilmente uma tela na abertura do
cálice; os raios lhe obedeciam e se quedavam hirtos semelhantes a um
véu, interpenetrando"-se em seu próprio movimento oscilante. Assim que
os fios se encontraram bem enredados, o velho descreveu novamente os
círculos sobre a beirada, a música foi diminuindo cada vez mais sutil
até se tornar inaudível, enquanto a renda resplendente tremia
apreensiva. Sob as oscilações mais fortes, se rompeu, e os raios
tombaram gotejantes em chuva no cálice, e da precipitação elevou"-se um
vapor avermelhado girando em si mesmo formando círculos no ar e
pairando como espuma sobre a abertura do copo. Um ponto mais luminoso
atravessou célere pelos círculos vaporosos. E eis que surgiu a visão:
como um olho aberto de súbito na emanação, algo como cachos louros
dourados se enroscava e \mbox{enrolava} em anéis aéreos, suave rubor 
invadia intermitente a sombra hesitante.

Ferdinand reconheceu o semblante sorridente de sua bem"-amada, os olhos
azuis, as faces meigas, a boca rosada encantadora. A cabeça oscilou um
instante pendendo, depois se elevou altiva e ereta sobre o pescoço alvo
e volveu"-se em direção ao moço embevecido. O velho prosseguia
descrevendo círculos em torno do copo, e logo surgiram os ombros de uma
alvura radiante, e assim, à medida que a criatura ia se desvencilhando
de seu leito dourado e se espreguiçava fascinante se movendo no espaço,
tornou"-se perceptível o contorno delicado e firme dos seios com seus
finos botões de rosa florescendo em doces tons encarnados. 

A imagem da amada se inclinou em vagas em sua direção prestes a tocá"-lo
com os lábios ardentes, e Ferdinand acreditou perceber"-lhe o hálito.
Incapaz de se dominar em êxtase, se levantou para imprimir"-lhe um beijo
na boca, e imaginou no delírio tocar"-lhe os admiráveis braços e cingir
a figura nua arrancando"-a da prisão dourada. De imediato, um tremor
violento sacudiu a sutil imagem, a cabeça e o corpo se cindiram em mil
linhas confusas; uma rosa vermelha jazia ao pé do cálice espelhando
ainda o encantador sorriso. Ferdinand a pegou com um gesto ardente,
apertou"-a contra os lábios, mas sob a pressão da ânsia ardente, ela
feneceu e se esvaiu no ar. 

-- Você não cumpriu a palavra! -- exclamou o velho aborrecido. --
Lembre"-se de que a culpa é sua mesmo!

Embrulhou novamente o cálice, descerrou as cortinas e abriu a janela. A
clara luz do dia entrou e Ferdinand com o coração cheio de tristeza
saiu balbuciando mil perdões ao velho casmurro.  

 Ele se apressou pelas ruas da cidade. Pouco depois dos portões,
sentou"-se sob as árvores do bosque. A jovem lhe informara na igreja
que, ao entardecer, sairia de viagem num carro, em companhia de alguns
parentes. Embriagado de amor, Ferdinand vagava por ali, ora se sentava,
ora retomava a caminhada; sempre mantendo ante seus olhos a lembrança
da imagem adorada, aquela que vira surgir do ouro incandescente, e
desejara então vê"-la se adiantando para junto dele em todo o resplendor
de sua formosura. Por desdita, porém, bem ali a figura primorosa se
desfez em pedaços, e ele se enfureceu por ter destruído, com o amor
impaciente e a confusão dos sentimentos, a imagem e talvez a própria
felicidade. 

Quando, após o meio"-dia as pessoas no passeio se tornaram mais
numerosas, ele se refugiou da multidão no meio da floresta; mas ficou à
espreita e não perdeu de vista a trilha larga, observando atentamente
cada carro que cruzava o portão.

A tarde se aproximava, o sol crepuscular espalhava tênue luz
avermelhada. De repente veio saindo pelo portão uma carruagem ricamente
guarnecida em metais dourados que refulgiam ao clarão vespertino. Ele
correu à via larga. Inquietos, os olhos dela estavam à sua procura.
Graciosa e sorridente, ela pendeu o elegante busto sobre a portinhola;
ele acolheu o cumprimento e o aceno de amor; por um instante se
encontrou bem perto da carruagem, a amada o envolveu com um olhar
apaixonado, mas como se impelida pelo \mbox{movimento} do carro, recuou ao
aceno e a rosa que lhe enfeitava o seio caiu aos pés de Ferdinand. Ele
a tomou e a beijou; foi como se um presságio lhe anunciasse que não
voltaria a rever a mulher que adorava; que agora a felicidade se
rompera para sempre. 

\asterisc

Passos apressados subiam e desciam escadarias, a casa inteira estava em
polvorosa, todo mundo gritava e fazia barulho se preparando para a
festa do dia seguinte. A mãe da família era mais ativa e alegre que os
outros; a noiva deixava tudo acontecer e, na reclusão do quarto,
sonhava com o que o destino lhe reservara. Todos
esperavam ainda o filho, o capitão e a esposa, bem como as duas filhas
mais velhas com seus maridos; Leopold, o filho caçula, punha lenha na
fogueira com peculiar traquinice, aumentava a confusão e virava tudo de
pernas para o ar sob o pretexto de prestar ajuda. Ágata, sua irmã
solteira, procurava demovê"-lo daquele auxílio, que simplesmente
deixasse os outros sossegados. Mas a mãe interveio:

-- Deixe"-o fazer suas peraltices em paz, pois hoje um pouco mais ou
menos de desordem não fará diferença. Como eu estou com a cabeça
bastante cheia de preocupações e afazeres, peço a vocês todos que não
me importunem senão com coisas importantes que valha a pena eu saber:
se alguém quebrar a porcelana, se faltar uma colher de prata, ou os
criados arruinarem um ou outro copo, não venham a mim contar essas
ninharias. Tão logo acabe essa agitação das bodas, então mandaremos a
conta.

-- Isso mesmo, mamãe! -- gritou Leopold. -- Desígnios dignos de uma
rainha! Mesmo se uns serviçais quebrarem o pescoço, o cozinheiro se
embebedar e botar fogo na chaminé, se o \textit{sommelier} na afobação
desperdiçar ou liberar pródigo vinho: não a colocaremos a par dessas
infantilidades. A não ser que um terremoto derrube a casa, querida,
pois isso será impossível dissimular.

-- Afinal, quando será que ele tomará juízo? -- indagou a mãe. -- O que
hão de pensar suas irmãs, quando o reencontrarem insensato como da
última vez que o viram, dois anos atrás?

-- Serão obrigadas a fazer jus ao meu caráter! -- respondeu o
adolescente de pronto. -- Minhas irmãs terão de admitir: não sou cordato
como os maridos delas, que em poucos anos se transformaram da água para
o vinho, não necessariamente para melhor.

O noivo veio ter com eles e lhes perguntou onde estava a noiva. Mandaram
a camareira ao quarto chamá"-la.  

-- Cara mamãe, Leopold lhe apresentou meu pedido? -- quis saber o
noivo. 

-- Não que eu me lembre -- disse a mãe. -- Com a confusão que reina
nesta casa, ninguém tem memória razoável. 

A noiva apareceu e os jovens se cumprimentaram com alegria. 

-- O pedido sobre o qual eu comentava -- prosseguiu o noivo com a
sogra -- é que você não leve a mal se eu trouxer mais um hóspede à sua
casa já completamente lotada durante esses dias. 

-- Meu filho, você mesmo sabe que, embora a casa seja ampla, fica
difícil encontrar um quarto disponível.

-- Todavia, eu tomei certas providências, e mandei que preparassem
o quarto grande do anexo -- interveio Leopold.

-- Mas aquele cômodo não é conveniente, há anos vem sendo usado
como depósito de bagunça.

-- As instalações são excelentes, cara mamãe! -- respondeu Leopold.
-- Ademais, o amigo a quem destinei o quarto não se preocupa com ninharias,
não é afetado como nós. Por ser solteiro e viver sozinho a acomodação é
sob medida para ele. Como se não bastasse o empenho que tivemos de
persuadi"-lo ao convívio aqui com seus semelhantes. 

 -- Você não está se referindo ao seu sinistro alquimista e
necromante! -- gritou Ágata.

-- O próprio! -- respondeu o noivo. -- Se você quer chamá"-lo desse modo.

-- Nesse caso, não permita, querida mamãe -- continuou a irmã. -- O que
afinal quer um homem desses em nosso meio? Certa vez o encontrei
andando pela rua com Leopold e tive a impressão de que não é flor que
se cheire. Nunca frequenta a igreja, não ama Deus nem os homens.
Introduzir essa espécie de descrente a uma ocasião festiva atrai maus
augúrios. Sabe"-se lá o que ele pode fazer!

-- Como você é injusta! -- enfureceu"-se Leopold. -- Condena o pobre
homem sem ao menos conhecê"-lo. Porque não lhe agrada o nariz dele, que
não é jovem nem atraente, a seu ver torna"-se logo um necromante infame!

E o noivo reforçou o pedido junto à sogra:

-- Conceda, cara mamãe, que acomodemos nosso velho amigo; permita
que ele compartilhe da alegria geral. Suponho, Ágata, minha mana
querida, que ele vivenciou infortúnios que o tornaram cético e
misantropo; ele evita a sociedade humana e somente abre exceção a
Leopold e a mim. Devo"-lhe muita gratidão, ele foi o amigo que conduziu
meu espírito num momento crucial da vida. Posso até dizer, talvez tenha
sido unicamente graças a ele que me tornei digno do amor de Julie. 

-- A mim -- replicou Leopold -- ele coloca todos os seus livros à
disposição, melhor dizendo, antigos manuscritos, e também dinheiro,
fiado meramente em minhas palavras. As inclinações do nosso amigo são
bastante cristãs, irmãzinha, e quem sabe, se vier a conhecê"-lo melhor,
talvez você se torne indulgente com seus modos um tanto taciturnos e
passe a amá"-lo, por mais horrível que ele lhe pareça hoje em dia.

 -- Sendo assim, podem convidá"-lo! -- disse a mãe. \mbox{-- Leopold} me fez
escutar bastante a respeito desse amigo, deixou"-me curiosa para
conhecê"-lo pessoalmente. Mas vocês têm de assumir a responsabilidade
por lhe atribuirmos uma acomodação tão modesta, pois não temos outra melhor.

Os convidados chegaram neste ponto da conversa. Eram os membros da
família; as filhas casadas, bem como o oficial, juntamente com os
filhos. A velha senhora alegrou"-se ao rever os netinhos. Reinavam os
abraços de reencontros, as palavras amáveis. Feitas as saudações aos
recém"-chegados, o noivo e Leopold saíram a fim de buscar o velho soturno.

A maior parte do ano o sujeito vivia no campo, a uma milha distante da
cidade, mas mantinha perto dos portões um modesto alojamento circundado
por jardins. Foi ali que, por acaso, os dois jovens o haviam conhecido.
Agora eles se juntaram num café, onde haviam combinado um encontro.
Como anoitecia, resolveram retornar imediatamente à casa da família. 

A mãe da noiva acolheu o forasteiro calorosamente, as moças se
mantiveram um pouco mais reservadas; Ágata, sobretudo, estava muito
intimidada e evitava com cautela mirá"-lo diretamente nos olhos. Logo
após as primeiras conversas convencionais, os olhos do velho se fixaram
sobre a noiva, que chegara posteriormente à sala. Ele parecia encantado
e todos notaram como procurava enxugar furtivamente uma lágrima. O
noivo ficou feliz vendo"-o à vontade, e um pouco mais tarde, quando
ficaram a sós, perto da janela, o moço tomou"-lhe a mão e perguntou: 

-- O que você diz de minha querida Julie, caro amigo? Não é um anjo?

-- Oh, meu caro! -- respondeu o velho com voz emocionada --, jamais vi
tanta beleza e graça. Mais que isso, talvez fosse melhor dizer que ela
é bela, encantadora, divina, tanto, que tenho a impressão de tê"-la
conhecido em outros tempos: embora desconhecida, ela é uma imagem
familiar à minha imaginação; uma imagem que sempre esteve presente em
meu coração. 

-- Eu compreendo -- disse o jovem noivo. -- \mbox{O que} é verdadeiramente belo, 
grandioso e sublime pode \mbox{despertar} em nós assombro e surpresa. Não é,
entretanto, um assombro estranho, incomum ante o inusitado, mas é como
algo que torna nosso âmago íntimo transparente, que traz à tona
reminiscências remotas e dá vida às sensações mais caras. 

Durante o jantar, o estrangeiro participou pouco das conversas. Ele não
tirava os olhos da noiva, o que acabou por deixá"-la constrangida e
ansiosa. O oficial contou uma história de suas campanhas em batalhas, o
comerciante rico falou de negócios e da crise, e o proprietário de
terras, das melhorias que introduzira em seus domínios. 

Após a refeição, o noivo se despediu e se recolheu pela última vez aos
seus aposentos solitários. No futuro ele deveria habitar com a jovem
esposa um apartamento preparado naquela casa. A sociedade se dispersou
e Leopold conduziu o hóspede ao quarto do anexo. 

-- Nos desculpe a modéstia do quarto, minha mãe gostaria de
instalá"-lo com mais conforto, mas o senhor viu como nossa família é
bastante numerosa, e outros parentes devem chegar amanhã. Pelo menos, o
senhor não poderá nos escapar, porque não encontrará o caminho da saída
neste edifício cheio de labirintos.

Os dois atravessaram ainda outros corredores confusos; finalmente
Leopold se afastou, desejando"-lhe boa noite. O valete de quarto
depositou num canto duas velas de cera e ofereceu seus préstimos ao
estrangeiro, mas recusando todo e qualquer serviço, o outro ficou só.
Andando para lá e para cá dentro do quarto, ele pensava: 
 
``Como pode me acontecer uma coisa dessas? Como aquela imagem
brota agora novamente com tanto vigor de dentro do meu coração? 
Esqueci todo o passado e acreditei vê"-la em pessoa! Mais uma vez fui
jovem e ouvi o tom de sua voz como antigamente. Pareceu"-me ter 
acordado de um sonho ruim, mas não! Agora é que despertei: a
maravilhosa ilusão não passou de um doce sonho.''

Agitado demais para adormecer, o velho contemplava uns desenhos
inscritos nas paredes e examinava o próprio quarto. 

-- Hoje tudo está me parecendo familiar! -- gritou inquieto. -- Por um
momento eu imaginei que esta casa e este quarto não me eram estranhos. 

O velho procurava colocar em ordem a avalanche de lembranças e
pensamentos e ergueu alguns livros pesados empilhados num canto.
Enquanto os folheava, meneou a cabeça. Uma caixa de alaúde se apoiava à
parede; ele a abriu e retirou de dentro um estranho instrumento
bastante antigo e estragado, ao qual faltavam algumas cordas.

-- Não! Eu não estou enganado! -- exclamou atônito. -- Esse alaúde é de
fato singular, é o alaúde espanhol de meu amigo Albert, morto há tantos
anos! Lá estão os livros de magia e este é o quarto, onde ele tentou
evocar para mim o oráculo maravilhoso. As cortinas vermelhas
desbotaram, as franjas douradas empalideceram, porém todos os instantes
passados nesse ambiente estão impregnados de uma intensidade poderosa!
Por essa razão me assombrei quando caminhava com Leopold por corredores
confusos e intermináveis! \mbox{Oh, céus}! Foi aqui, sobre esta mesa, que a
imagem surgiu e cresceu do cálice, como se adquirisse viço e vida com
os tons rubro"-dourados! A mesma imagem me sorria hoje à noite, quase me
levando à loucura na sala, na qual tantas vezes estive entabulando
conversas íntimas com Albert!

Ele se trocou, mas dormiu pouco. Na manhã seguinte, levantou"-se bem cedo
e examinou atentamente o quarto. Abriu a janela, reviu então o mesmo
jardim e uma edificação diante de si, como naquela ocasião, com a
diferença de que nesse entretempo haviam sido construídas novas casas. 

 -- Quarenta anos decorreram desde então! -- suspirou. -- E cada dia
daquela intensa época conteve mais vida que toda a existência posterior! 

Vieram convidá"-lo a se reunir ao grupo. A manhã transcorreu entre
preparativos e conversas animadas, enfim a noiva fez a entrada triunfal
trajando o vestido do casamento. Desde o instante em que a viu, o velho
se agitou sobremaneira, e sua emoção não escapou a nenhum dos
presentes. Todos se dirigiram à igreja, e as bodas \mbox{foram} celebradas.
Quando se encontravam de volta a casa, Leopold perguntou à mãe:

-- Mamãe, o que a senhora está achando de nosso amigo, o velho ermitão?

Ao que ela respondeu:

-- Pela descrição que vocês haviam feito, eu o havia imaginado bem
mais assustador. Mas ele é um homem ameno e bondoso, o seu jeito
inspira muita confiança.

-- Confiança? -- exasperou"-se Ágata. -- Aqueles olhos ardentes em fogo,
as milhares de rugas, a boca fina e ressequida, o riso sinistro e
sarcástico. Não, Deus me proteja de semelhante amigo!  Quando querem
se imiscuir entre os humanos, os diabos escolhem uma criatura assim!

-- Eles preferem sem dúvida uma criatura mais jovem e charmosa --
respondeu a mãe. -- Aliás, não concordo nada com sua descrição. Esse
senhor tem um temperamento violento, um jeito reprimido, é verdade.
Talvez tenha vivido muitos infortúnios, como disse Leopold, o que lhe
tirou a espontaneidade peculiar às pessoas bem"-aventuradas, e o fez
tornar"-se desconfiado.

A conversa foi interrompida pelo restante do grupo que chegava naquele
momento. Puseram"-se todos à mesa, o estrangeiro tomou lugar entre Ágata
e o rico negociante. Assim que começaram a fazer os brindes, Leopold gritou:

 -- Esperem, meus caros amigos! Precisamos de nosso cálice de
festas, que será passado de mão em mão.

Ele fez menção de se levantar, mas a mãe acenou para que ficasse
sentado:

-- Você não conseguirá encontrá"-lo, pois troquei toda a prataria de
lugar! -- e saiu rapidamente ela mesma atrás do cálice.

O negociante comentou:

-- Como nossa velha mãe está animada e cheia de vida! Apesar da
idade e da corpulência, ela se movimenta com agilidade e suas feições
estão sempre alegres, iluminadas. Hoje, então, ela está especialmente
radiante, pois a formosura da filha a rejuvenesce ainda mais!

O velho concordou com as palavras do interlocutor. 

A mãe retornou com o cálice na mão. Encheram"-no de vinho, e o objeto
passou a circular a partir da ponta da mesa de mão em mão, cada um dos
convivas brindando ao seu desejo mais sincero e caro. A esposa bebia à
saúde do esposo, o noivo ao amor da bela Julie e assim por diante. A
mãe hesitou quando chegou sua vez. 

Um pouco insolente e grosseiro, o oficial disse:

-- Sabemos que a senhora considera todo homem infiel e indigno do
mérito de ser amado pela mulher! Nesse caso me pergunto, qual seria o
sonho sincero que acalenta?

A velha senhora o encarou, uma gravidade austera difundiu"-se pelo seu
rosto tranquilo:

-- Como meu filho me conhece bem e censura com rigor meus
sentimentos, que me seja permitido omitir meus pensamentos íntimos os
quais são julgados pelo meu caro, com seu amor sincero.  

 Ela passou o cálice ao vizinho sem beber, e a disposição do grupo ficou
abalada durante alguns minutos. 

 -- Diz"-se -- sussurrou baixinho o negociante, curvando"-se para o lado
do estrangeiro -- que ela não amou seu marido, mas a outro homem que lhe
foi infiel. Diz"-se também que foi a moça mais bonita da cidade. 

Quando o cálice dourado chegou às mãos de Ferdinand, ele o contemplou
estarrecido, pois era justamente o mesmo cálice do qual Albert lhe
evocara a maravilhosa imagem. Demorou seu olhar no ouro e nas ondas do
vinho; sua mão começou a tremer. Ele não se surpreenderia se do copo
mágico e brilhante tivesse florescido novamente a figura de outrora, e
com ela sua juventude distante. 

-- Não! -- disse ele algum tempo depois à meia"-voz. -- É vinho, o que
está luzindo assim incandescente. 

-- Claro! O que mais poderia ser? -- indagou rindo o negociante. -- Beba sossegado!

Um calafrio de susto perpassou o velho, com veemência ele proferiu o
nome ``Franziska'', e levou o cálice aos lábios ardentes.

A velha dama lançou"-lhe um olhar inquisitivo de curiosidade. 

-- De onde vem essa belíssima taça? -- falou Ferdinand muito
envergonhado do lapso.

-- Há muitos anos -- respondeu Leopold --, bem antes do meu nascimento,
meu pai comprou o cálice, juntamente com esta casa e todo o mobiliário,
de um velho solitário e reservado, que a vizinhança tachava de feiticeiro.

Ferdinand não queria dizer que conhecera bem o antigo proprietário, pois
sua existência toda agora se transformara e se confundira num sonho
estranho, e ele não podia conceber a ideia de permitir que outras
pessoas percebessem a perturbação que lhe acometia a alma.

 Após retirarem a mesa, ele ficou a sós com a anfitriã, porque os jovens
tinham saído a fim de aviar preparativos para o baile. 

-- Sente"-se aqui perto de mim -- convidou a mulher --, queremos
descansar, pois já passamos da idade de ir a bailes e, se não for
indiscreto perguntar, diga"-me se já tinha visto nosso cálice em algum
lugar, ou o que foi que o comoveu tão profundamente?

-- Oh, caríssima senhora! -- respondeu o velho --, me desculpe a
brusquidão e a emoção, mas desde que entrei nesta casa tenho a
impressão de não responder mais por mim, pois a cada instante esqueço
que meus cabelos são brancos e meus entes diletos morreram. Sua filha
encantadora, que hoje celebra o dia mais feliz de sua vida, é tão
semelhante a uma jovem que conheci e adorei em minha juventude que
chego a pensar num milagre! Não! Sua filha não se parece com minha
bem"-amada, semelhança é pouco: é ela própria! E esta casa também me foi
familiar, há muitos anos eu já tinha visto aqui a taça dourada em
circunstâncias bem singulares.

E, então, o velho lhe contou toda a aventura. 

-- Na tarde daquele dia -- assim terminou ele sua narrativa --, lá fora
no bosque eu vi minha bem"-amada pela última vez enquanto ela
atravessava de carruagem para uma viagem à casa de campo. Uma rosa
desprendeu"-se do seu seio e eu a conservei. Mas ela mesma eu perdi,
pois me foi infiel e casou logo em seguida.

-- Deus do céu! -- gritou a velha dama se levantando bastante
emocionada. -- Você não é Ferdinand!

-- Sou eu próprio, Ferdinand! -- respondeu o velho.

-- Eu sou Franziska! -- respondeu a mulher.

Os dois fizeram um gesto para se abraçar, mas retrocederam rapidamente.
Miraram"-se mutuamente com muita atenção, ambos procuravam encontrar sob
as ruínas da idade, os traços que haviam conhecido e amado. E tal qual
na sombra noturna da tempestade a fuga das nuvens negras deixa em
instantes fugidios entrever o brilho das estrelas para em seguida
dissimulá"-lo mais uma vez, do mesmo modo, por um momento lhes pareceu
ver mutuamente o fulgor de outrora em feições, olhos e lábios. Era como
se sua juventude chorasse e sorrisse ao mesmo tempo bem ao longe. Ele
se inclinou e beijou"-lhe a mão, enquanto duas lágrimas pesadas rolavam
abaixo, depois eles se abraçaram ternamente. 

-- Sua mulher morreu? -- perguntou Franziska.

-- Eu nunca me casei -- soluçou Ferdinand. 

-- Santo Deus! Então eu fui a infiel! -- disse ela torcendo as mãos.
 -- Não, não fui infiel. Assim que retornei do campo, onde estive por dois
meses, soube por todas as pessoas, inclusive pelos seus amigos, não
somente pelos meus, que você partira há tempos para desposar alguém em
sua terra natal. Mostraram"-me cartas dignas de confiança, insistiram
nisso, abusaram de meu desespero e de minha raiva. Paulatinamente me
deixei persuadir e acabei por concordar em ceder minha mão a um homem
de nobres qualidades. Meu coração e meus pensamentos, eu consagrei a
você para sempre.

 -- Eu, contudo, não viajara -- disse Ferdinand. -- Algum tempo mais
tarde tive notícia de seu casamento. Quiseram nos separar, minha cara,
e lograram fazê"-lo. Você é uma mãe bem"-aventurada, eu sou um homem que
vive no passado. A todos os seus filhos, eu quero amar como se fossem
meus. Como é estranho, desde então nós jamais nos vimos!
 
-- Eu raramente saía -- disse ela --, e como meu marido logo alterou o
sobrenome por causa de uma herança, você não podia suspeitar que nós
estivéssemos morando na mesma cidade.

-- Eu evitava o convívio -- disse Ferdinand --, vivia em completa
solidão. Leopold foi praticamente o único que conseguiu me persuadir
novamente ao convívio com as pessoas. Oh, minha cara, a maneira como nós
nos extraviamos e novamente nos unimos tem tudo de uma história de horror!

Os jovens encontraram os dois velhos desfeitos em lágrimas e
profundamente comovidos. Ambos se calaram a respeito do que lhes
acontecera no passado, não quiseram profanar um segredo tão sagrado. A
partir de então, o velho foi o amigo da casa. Só a morte separou as
duas criaturas que tinham se reencontrado de maneira tão
extraordinária, para pouco tempo depois novamente reuni"-las.
\medskip

\hfill\textit{Tradução de Maria Aparecida Barbosa}

\chapter[Eckart Fiel e Tannenhäuser]{Eckart Fiel e Tannenhäuser\subtitulo{em duas partes}}

\section*{Primeira Parte}


\begin{verse}
Sobre o plaino arenoso\\
O nobre duque orgulhoso\\
Padeceu hostil vergonha\\
Contra o país da Borgonha.

``O inimigo é vencedor!''\\
Diz, ``exauriu meu vigor.\\
Amigos me arrebatou,\\
Valetes me abateu.

Não tenho como lutar\\
Nem armas mais manejar.\\
Onde estará meu escol,\\
Eckart, bravo e fiel?

Trazia a parentalha\\
A toda árdua batalha\\
Mas hoje, infelizmente,\\
Da peleja está ausente.

A força inimiga cresce\\  
Pressinto que já me rende\\
Desertar da luta não quero\\
Gloriosa morte prefiro!''

Queixa"-se do avatar,\\
Tentou suicídio no azar,\\
Contudo, um  elã o anima\\
Eckart, bravo, se aproxima. 

Couraçado Eckart, fulmíneo,\\ 
Arroja"-se, luta exímio.\\
A prole segue o modelo\\
Paterno e viril apelo.

Borgonha percebe o sinal\\
Se imbui de alento vital!\\
Ora o inimigo do prélio\\
Foge em grande estrambelho.

O bravo Eckart se lança\\
Em meio à turba acossa,\\
Contudo banhado em sangue\\
O filho recolhe exangue.

Nem bem a falange cedia\\
Solene o duque anuncia:\\
``Foi bem"-sucedida a vitória\\
Embora d'horrível memória!

Valente geriu o guerreiro\\
Salvando vidas e o reino!\\
Mas o filho morto jaz\\
À vida ninguém o traz!''

Turba"-se prestes em pranto,\\
O herói se curva com espanto,\\
Recolhe a preciosa carga\\
Nos braços o filho abarca!
 
``Oh, Heinz, tu morres jovem\\
Se nem te tornaste homem\ldots{}\\
Mas supero acerba dor\\
Não guardo nenhum rancor. 

Pois nós, duque, o salvamos\\
Do ultraje o livramos,\\
Portanto, para seu brilho\\
Rendo"-lhe a vida do filho!''

O Borgonha emocionado\\
Sente o olhar marejado,\\
A grandeza do amor\\
A seu ver tinha valor.

Comovido ele chora,\\
``Herói amigo, o venero'',\\
Entre abraços de euforia,\\
Diz em páthos e alegria:

``Na guerra, fiel guerreiro\\
Sólido estandarte e brasão,\\
Na glória, oh braço direito,\\
O amarei como irmão.

Toda a nação no porvir\\
O tratará como a um rei,\\
Se posso recompensar,\\
Tesouros eu lhe darei.''

Ao propalar"-se a promessa,\\
Foi júbilo sem revel\\
O povo rendeu"-lhe às pressas\\
A alcunha de Eckart fiel. 
\end{verse}

 A voz de um velho camponês que entoava a canção ressoava através dos
campos, e o fiel Eckart, cheio de dor, estava sentado à beira da
encosta, imerso em lágrimas. O filho caçula que se encontrava junto
dele perguntou: 

-- Por que o senhor está chorando alto, meu pai Eckart? Se
o senhor é grande e forte, maior e mais forte que todos os outros
homens, quem lhe inspiraria medo?

 Enquanto isso, o grupo de caça do duque retornava. O Borgonha cavalgava
um garanhão magnífico, ricamente adornado, o ouro e os adereços do
príncipe soberano brilhavam e refulgiam ao sol crepuscular, de modo que
o jovem Conrad não podia admirar com nitidez, tampouco apreciar
convenientemente o esplêndido cortejo.

 O fiel Eckart se ergueu e olhou a cena com olhar sombrio, e o pequeno
Conrad, após perder o cortejo de vista, começou a cantarolar:

\begin{verse}
Se você quer conduzir\\
Portando espada e brasão\\
Em majestoso garanhão,\\
Então, precisa possuir\\
Brio, tutano, ímpeto viril,\\
Coragem varonil\\
Prá' respeito incutir!
\end{verse}

 O velho tomou o filho em seus braços e o estreitou junto ao peito,
lançando um olhar emocionado com seus olhos grandes e límpidos. 

-- Você escutou a canção daquele bondoso camponês? -- perguntou"-lhe então.

 -- Como não ouviria? -- respondeu o jovem. -- Ele cantou em alto e bom tom, e
como o fiel Eckart é você, ouvi com prazer.

 -- Esse mesmo duque é agora meu inimigo -- disse o velho pai. -- Ele guarda
cativo meu segundo filho, e deve tê"-lo enforcado, se eu for crer no que
dizem as pessoas do reino. 

 -- Pegue sua espada e não tolere isso -- disse o filho. -- Todos eles tremem
em sua presença e o apoiarão, pois você é o grande herói do reino.

 -- Não, meu filho -- disse o pai --, pois nesse caso seria eu a declará"-lo
inimigo. Eu não posso ser infiel a meu soberano, não, não tenho o
direito de romper a paz que jurei e depositei em suas mãos. 

 -- Mas o que ele quer de nós? -- indagou Conrad impaciente.

 Eckart se sentou novamente e explicou:

 -- Meu filho, essa é uma longa história, e mesmo se eu lhe contasse tudo
em detalhes, você mal compreenderia. O poderoso guarda sempre o maior
inimigo em seu próprio coração, assim ele o teme dia e noite: Borgonha
pensa agora que confiou demais em mim, e com isso fez criar uma
serpente em seu seio. Sou aclamado no país como valente, dizem aos
quatro ventos que ele me deve o reino e a vida, chamam"-me Eckart, o
fiel. Por isso os oprimidos e carentes se dirigem a mim, pedindo
auxílio; o que ele não pode tolerar; e todo aquele que deseja estar bem
aos olhos do príncipe nutre \mbox{similar} despeito: enfim, ele acabou por me
excluir de seu coração.

 Depois disso, o herói contou com palavras simples que o duque o banira
de sua presença e os dois haviam se tornado praticamente estranhos um
ao outro; tudo devido às suspeitas de que ele, Eckart, intentava
tirar"-lhe o ducado. Muito angustiado, continuou a contar como o
príncipe reteve seu filho prisioneiro e, por considerá"-lo traidor,
desejava matá"-lo.

 Conrad disse ao pai:

-- Então deixe"-me ir até o duque, velho pai, eu falarei com ele e o
farei ver as razões, reacendendo mais uma vez a amizade. Se de fato
enforcou meu irmão, deve ser um homem maligno e você precisa puni"-lo;
mas isso não é provável, porque ele não pode ser vil a ponto de
esquecer os grandes serviços que você lhe prestou.

-- Você não conhece o velho ditado? ``Se o poderoso sua ajuda almeja,
valor como amigo você enseja. Nem bem, porém, o interesse se esmaece,
a amizade também empalidece.'' Sim, dissipei minha vida inteira inutilmente: 
por que me elevar para em seguida mais do alto me precipitar? A amizade 
de um soberano é como um veneno mortal a ser empregado exclusivamente 
contra os inimigos, quem o possui pode até se matar por imprudência.

-- Vou ter com o duque -- exclamou Conrad. -- Farei com que se lembre de
coração de seus feitos, seus sofrimentos, e tudo voltará a ser como
outrora.
 
-- Você se esqueceu de que somos prescritos traidores; busquemos, portanto,
asilo num país estrangeiro onde possamos encontrar melhor fortuna.

-- Em sua idade -- respondeu Conrad -- você ainda pretende dar as costas à
terra natal? Não, deixe"-nos antes tentar de tudo. Eu me inclinarei aos
pés do Borgonha, me reconciliarei e o apaziguarei; pois o que o duque
poderia fazer contra mim, mesmo se nutre ódio e temor por você?

 -- Muito contrariado eu permito que você vá -- disse Eckart --, minha alma
não pressente nada de bom, contudo eu gostaria muito de
reconciliar"-me com ele, pois é um velho amigo, e gostaria também de
salvar seu irmão que, junto dele, padece na prisão.

  O sol lançava seus últimos raios suaves sobre a terra verdejante e
Eckart assentou"-se pensativo, apoiando"-se num tronco de árvore. Ele
olhou Conrad longamente e em seguida disse:

 -- Se você quer partir, meu filho, parta incontinenti antes que a noite
escureça de todo. As janelas do castelo ducal resplendem já iluminadas,
percebo lá longe os tons festivos de trombeta, talvez a esposa do
príncipe delfim tenha chegado e ele esteja mais amigável para nos receber.

 Contrariado, pois não depositava confiança alguma na sina, Eckart
deixou o filho seguir o caminho. O jovem Conrad, porém, estava tanto
mais animado porque tinha o leve pressentimento de que seria capaz de
converter a inclinação do príncipe que ainda há pouco tempo atrás
brincara com ele.
 
 -- Você tem certeza que retornará, meu filho querido? -- resmungou o velho
pai. -- Se eu o perco, não restará qualquer descendente de minha estirpe.

O rapaz o consolou e o cobriu de carinho. Finalmente, se separaram.

Conrad bateu à porta do castelo e foi admitido ao interior, o velho
Eckart permaneceu sozinho na noite.

-- Esse também eu perdi! -- queixava"-se na solidão. -- Não voltarei a ver seu rosto!

Enquanto ele assim lamentava, viu avançar hesitante, apoiado num cajado,
um ancião que queria descer do rochedo e a cada passo dava a impressão
de que cairia no abismo. Percebendo a fragilidade do ancião, Eckart
estendeu"-lhe a mão para ajudá"-lo a descer sem percalços.

-- Aonde o senhor está indo? -- indagou Eckart.

O homem sentou"-se e pôs"-se a chorar tão amargamente que as límpidas
lágrimas escorriam"-lhe pelas faces. Eckart quis consolá"-lo com palavras
suaves e razoáveis, mas o ancião, pleno de aflição, parecia nem se
importar com seu discurso bem"-intencionado, antes se abandonava ainda
mais languidamente à dor.

-- Que sofrimento o exauriu tanto? -- quis saber Eckart finalmente. -- O que o
deixou dessa maneira, absolutamente abatido?

-- Ah, meus filhos! -- gemeu o ancião.

Então Eckart pensou em Conrad, Heinz e Dietrich e ficou, ele próprio, desolado:	

-- Ah, melhor seria se estivessem mortos!
 
O herói teve um calafrio de susto ante aquelas estranhas palavras e
pediu ao ancião que lhe decifrasse o enigma. O outro explicou:

-- Nós vivemos realmente uma época singular que por certo em breve trará o
fim do mundo, pois os sinais espantosos dessa ameaça são visíveis. Toda
a desgraça está agora rompendo as antigas cadeias e vagando livre e
solta entre nós. A fé em Deus é como uma fonte se esgotando e se
desvirtuando, não encontra mais um leito ao qual se recolha; as forças
do mal se erguem atrevidas de seus recônditos e celebram seu triunfo.
Ah, meu caro senhor, nós nos tornamos velhos, mas não o bastante para
tais histórias prodigiosas! O senhor sem dúvida viu o cometa, a
maravilhosa luz celeste que brilhou lá no alto uma claridade profética.
O mundo inteiro prevê a desgraça e ninguém pensa em começar a
aprimorar"-se a si mesmo e evitar desse modo o castigo. Como se não
bastasse, os prodígios emergem do seio da terra, irrompem
misteriosamente do fundo à superfície, assim como a luz vinda do alto
brilha terrível entre nós. O senhor jamais ouviu falar da montanha que
as pessoas denominam Montanha de Vênus?

-- Jamais! -- respondeu Eckart. -- Desde que me entendo por gente.

-- Isso me surpreende -- refletiu o ancião --, pois a coisa é atualmente tão
conhecida quanto verídica. Dentro dessa montanha se refugiaram os
diabos que se salvaram dos centros mais remotos da Terra, quando a
ascensão da Santa Cristandade reverteu a idolatria dos cultos pagãos.
Aqui, diz"-se, acima de todos a Dama de Vênus reunia a corte em torno de
si, com todos os seus cortesãos infernais de voluptuosidade terrena e
desejos permissíveis, e assim a própria montanha transformou"-se numa
maldição após tempos imemoriais.

-- Mas onde se encontra essa montanha? -- perguntou Eckart.

-- Esse é o mistério -- respondeu o ancião --, ninguém pode dizê"-lo, senão
aquele que tenha se devotado a Satã. Nenhuma alma inocente tem a
ideia de querer procurar essa montanha. Um menestrel bem peculiar vem
subitamente das profundezas, enviado pelo inferno como legítimo
representante; ele percorre o mundo e toca com a flauta, fazendo os
tons soarem ecoando pelos campos longínquos. Ora, aquele que ouve essa
melodia torna"-se cativo dela com uma potência manifesta, mas
inexplicável, e é impelido sem cessar para longe, longe, em direção ao
deserto, ele não vê o caminho por onde segue, anda, anda sem se cansar,
as forças crescem concomitantes à pressa, nenhum poder o detém e assim o
curso furioso o conduz ao interior da montanha. Jamais, para toda a
eternidade, ele encontrará o caminho de volta.

Eis que o poder é rendido novamente ao inferno, e eis que de direções
diversas os peregrinos infelizes se põem em marcha rumo ao lugar onde
nenhuma salvação os aguarda.

Há tempos meus dois filhos não me proporcionavam mais alegria alguma,
eram malcriados e sem modos; desdenhavam os pais e a religião.
Enfim, essa música arrebatou"-os e seduziu"-os, eles foram embora para bem
longe, e o mundo se lhes afigurou tacanho: eles buscaram espaço no
inferno.

-- Nessas circunstâncias, o que o senhor pretende empreender para
achá"-los? -- perguntou Eckart.

-- Munido de meu cajado, eu me pus a caminho -- respondeu o ancião --, a fim
de percorrer o mundo e reencontrá"-los, ou padecer de cansaço e desgosto.

Tendo dito essas palavras, ergueu"-se com dificuldade e precipitou"-se a
caminhar tão rápido quanto lhe permitiam as pernas, como se qualquer
atraso pudesse impedi"-lo de alcançar o que lhe era mais caro no mundo.
Eckart acompanhou com o olhar o esforço inglório do ancião e, no fundo, o
considerava louco.

Caíra a noite, o dia clareava e Conrad não retornava. Então Eckart
passou a vagar pela montanha, volvendo olhares ansiosos ao castelo, mas
nenhum sinal do moço. Uma tropa barulhenta saiu lá de dentro, nisso ele
não buscava mais se dissimular, porém montou o cavalo que pastava livre
pela relva e cavalgou juntando"-se ao grupo de cavaleiros jovens e
bem"-humorados através da pradaria. Quando se reuniram, eles o
reconheceram, mas nenhum ousou ameaçá"-lo ou ofendê"-lo com alguma
palavra ríspida, ao contrário, silenciaram com respeito, o contemplaram
admirados e prosseguiram o caminho.

Eckart gritou para um dos escudeiros e indagou:

-- Onde está meu filho Conrad?

-- Oh, não me pergunte isso, pois a resposta só provocaria luto e
sofrimento!

-- E Dietrich? -- insistiu ele.

-- Não pronuncie mais esses nomes -- disse o velho escudeiro --, pois eles
estão mortos, a ira do soberano inflamou"-se contra eles, ele pensou em
puni"-lo infligindo a morte a seus filhos.

Um furor ardente se elevou da alma conturbada de Eckart; tomado de dor e
raiva, não conseguiu se dominar. Com violência, ele esporeou o cavalo e
transpôs o portão do burgo. Todos retrocediam à sua passagem com
reverente respeito, e assim ele cavalgou até a frente do palácio.
Desmontou do animal e com passos oscilantes subiu a suntuosa escadaria.

``Estou de fato neste lugar'', perguntava"-se de si para si, ``na residência
do homem que outrora foi meu amigo?''

Ele tentava raciocinar com clareza, diante de seus olhos, todavia,
moviam"-se imagens cada vez mais agitadas e, foi em semelhante estado
que atingiu finalmente o aposento do duque.

O príncipe de Borgonha não esperava absolutamente aquela presença e se
assustou sobressaltado quando viu Eckart a um passo de si.

-- Você é o duque de Borgonha? -- perguntou logo encarando"-o.

O duque confirmou que sim. 

-- E mandou matar meu filho Dietrich? 

O duque disse sim. 

-- E também meu filhinho caçula, Conrad? -- inquiriu Eckart, imerso
em dor. -- E como se não bastasse, você o mandou matar de modo bárbaro? 

Ao que o duque novamente respondeu que sim.

Nesse ponto, Eckart encontrava"-se subjugado e falou entre lágrimas:

 -- Ah, não me responda com frieza, Borgonha, porque não consigo suportar
suas palavras. Diga somente que lamenta, se arrepende e desejaria
imensamente voltar atrás, eu procurarei me consolar. Ouvindo"-o assim,
com todo o ímpeto de meu coração eu o detesto!

O duque, porém, disse:

-- Suma de minha vista, infiel traidor, pois você é o pior inimigo que eu
poderia desejar na face da Terra!

E Eckart respondeu:

-- Um dia você me chamou, sim, de amigo! Essa ideia hoje pode lhe parecer
bem remota, eu imagino. Nunca agi contra sua pessoa, sempre o honrei e
amei como meu soberano. Deus me livre de levar a mão à espada, como
muito bem poderia fazê"-lo, a fim de consumar uma vingança. Não. Por mim
mesmo, prefiro desaparecer de sua vista e morrer na solidão!

Após proferir essas palavras, ele ia se retirando, mas o Borgonha se
irritara no fundo de sua alma e ao seu sinal surgiram guarda"-costas com
lanças, cercando Eckart por todos os lados, e se viram na contingência
de expulsá"-lo dos aposentos com as armas pontiagudas.

\begin{verse}
Saltou sobre a montaria\\
Eckart, nobre guerreiro,\\
Afirmando, o mundo inteiro\\
Então nada lhe valia.

``Vi cedo a prole apartar"-se,\\
Nenhures diviso alívio,\\
Ninguém presta"-me auxílio,\\
O príncipe ameaça matar"-me.''
 
Veloz cavalgou à floresta,\\
N'alma profunda tristeza,\\
Clama o tormento e a fereza\\
Companhia alguma requesta:

``Com a raça humana indisposto,\\
Suspiro, anseio coragem,\\
A carvalho ou faia selvagem\\
Confidenciarei meu desgosto.

Filho que me agradasse\\
Comigo nenhum ficou,\\
Pois o destino os levou.\\
É sombrio meu desenlace!''

Afligia"-se Eckart raivando\\
Perdeu o sentido, infrene,\\
E célere galope o impele\\
Enquanto o dia raiando.

O cavalo, guia e amigo,\\
Dava pinotes, suava,\\
Eckart nem se importava\\
Do mundo inteiro inimigo.

Desvestiu da cabeça o elmo\\
E do alto lançou"-se ao solo\\
A si mesmo causando dolo.\\
Finar"-se, um intenso anelo!
\end{verse}


 Ninguém nas redondezas sabia onde Eckart se metera, porque ele
se embrenhara pelas florestas selvagens adentro e não se deixou
entrever por quem quer que fosse. \mbox{O duque} se acalmara e nesse ínterim
se arrependia de tê"-lo deixado partir, em vez de encarcerá"-lo. Por
isso certa manhã ele pôs"-se a caminho acompanhado de grande cortejo de
caçadores e pessoas da corte, a fim de percorrer as florestas à procura
de Eckart, pois acreditava que somente a morte do outro lhe propiciaria
segurança. Todos se dispunham à busca com afã e não arrefeciam o zelo,
embora o sol tivesse se posto sem que tivessem encontrado o menor
vestígio de Eckart.

 Uma tempestade irrompeu, nuvens pesadas e sombrias pairavam sobre a
floresta. O trovão ribombava e os raios incidiam sobre os elevados
carvalhos. Todos foram tomados de um medo indizível e se dispersaram em
meio aos arbustos ou campos baixos. O garanhão do príncipe de Borgonha
desembestou mata adentro e o escudeiro não conseguiu segui"-lo; o
esplêndido animal foi abatido e, uníssono às trovoadas, os gritos do
Borgonha em vão apelavam o socorro dos vassalos, pois não havia ninguém
que pudesse escutá"-lo.

 Como uma besta selvagem, Eckart vagara a esmo, tendo perdido a
consciência de si mesmo e de sua desgraça. Ele se perdeu e saciou a
fome com ervas e raízes. Irreconhecível o herói teria parecido a todos
os seus amigos, tão radicalmente os dias de desespero o haviam
desfigurado. Logo que a tempestade desabou, ele despertou de seu
embotamento, retomou consciência dos próprios sofrimentos e do
infortúnio. Ao lembrar"-se, proferiu lancinantes berros de dor pelos
filhos, arrancando tufos de cabelos brancos e clamou através do bramido
torrencial:
 
 -- Aonde, aonde foram parar vocês, pedaços de meu coração? Como pude ser
privado de toda minha força a ponto de não poder desforrar"-lhes o
assassínio? Por que retive meu braço, em vez de causar a morte de quem
inferiu a meu coração golpe assim fatal? Ah, eu, infame, mereço que o
tirano escarneça de mim, pois meu braço impotente e meu coração
traiçoeiro não resistiram ao assassino! Agora, agora ele deveria se
encontrar à minha frente! Debalde anseio neste instante pela vingança,
perdi a oportunidade da desforra!

 Assim veio se aproximando a noite, e Eckart errava sem rumo em seu
lamento. Nisso ouviu ao longe uma voz clamando por socorro. Dirigiu
seus passos àquela direção e acabou encontrando no meio da escuridão um
homem apoiado a um tronco de árvore, que, gemendo, lhe implorou auxílio
no sentido de orientar"-se no ermo. Eckart estremeceu ao escutar aquela
voz que lhe soava familiar, de repente seu espírito clareou e percebeu
que o caçador desgarrado era em pessoa o príncipe de Borgonha. Eckart
levantou a mão para sacar a espada e abater com um golpe o 
assassino de seus filhos. O furor insuflou"-lhe forças renovadas, ele
sentiu o firme propósito de exterminar o inimigo, mas subitamente
deteve o gesto recordando a promessa e a palavra dada. Pegou a mão do
inimigo e o conduziu à direção em que supunha situar"-se a trilha.

\begin{verse}
Na floresta escura e íngreme\\
Oscila, pende cansado\\
Mas Eckart  herói muito íntegro,\\
Ergue"-o e o leva carregado.

``Perdoe"-me, essa dor ferina\\
Que lhe inflijo, homem honrado.''\\
O outro responde ``é a sina,\\
No mundo de Deus, o fado.''

``Eu lhe asseguro benesses'',\\
Altivo príncipe augura,\\
``Caso ao lar vivo regresse'',\\
Enquanto às espáduas pendura.

O herói tem lágrimas quentes\\
Pelo rosto abatido a fluir.\\
Diz ``meu senhor, entrementes,\\
Louros não hei de exigir!''

``A dor do corpo estilhaça!''\\
Geme sentido da sorte.\\
Lhe intriga a ignota trilha:\\
``Você por acaso é a morte?''

``Não! Morte, não me chamo.''\\
Contrito o herói murmura:\\
``Deus, o poder soberano,\\
É luz, ampara e fulgura!''

``Ah, pura clarividência!'',\\
Diz em delírio imerso,\\
``Oh, pecados na consciência,\\
Perante Deus e o universo.

A culpa da morte vil\\
De filhos, três, do meu imo\\
Que vaga por brenha hostil\\
E deplora o mal do ímpio.''\\
 
Humildade bem devota,\\
Súdito e amigo ameno,\\
Eckart, nobre patriota,\\
Temperamento sereno! 

\textit{Mea culpa} o desalento!\\
Se nessas bandas o encontro\\
Débil, nem mesmo o enfrento,\\
Receio o fatal confronto.

Recebo nítido oráculo:\\
Pra' morte de descendência\\
Reza presságio vernáculo\\
Um desígnio de vingança!''

O herói diz: ``a carga é dura,\\
Esse fardo a transportar.\\
Relevo opresso, alma impura\\
De crimes plena a pesar.

A malsinada visão,\\
É fatal, embora fugaz.\\
Creia"-me, vai suceder\\
Em clima de muita paz!''
\end{verse}

 Nessa conversa eles prosseguiam o caminho, quando lhes veio ao encontro
uma figura humana; era Wolfram, o escudeiro do duque, que há muito
vinha procurando seu senhor. A noite tenebrosa ainda se mostrava sobre
eles, e nenhuma estrela despontava por entre as densas nuvens. O
príncipe se sentia muito debilitado e desejava alcançar um abrigo, no
qual pudesse dormir até o amanhecer: ao mesmo tempo, estremecia ante a
possibilidade de encontrar Eckart, cuja lembrança o assustava como um anjo 
vingativo. Na verdade, o príncipe não acreditava que sobreviveria à
noite, tinha calafrios cada vez que um sopro da ventania agitava as
grandes árvores, quando a tempestade subindo dos abismos da montanha
passava fustigando suas cabeças.

  -- Suba, Wolfram -- disse o duque em sua angústia --, suba ao alto desse
pinheiro, e veja se avista nas redondezas uma luzinha, uma casa ou uma
cabana aonde possamos nos dirigir.

 Com risco de vida, o escudeiro escalou a árvore elevada que a
tempestade agitava de um lado ao outro e cujos ramos se curvavam então
quase até o chão, de modo que o rapaz se balançava na grimpa como um
esquilo. Finalmente ele atingiu o cume e gritou:

 -- Lá embaixo no vale eu vejo brilhar uma luz, é para lá que devemos nos
encaminhar!

Imediatamente ele desceu e mostrou aos dois companheiros a direção;
depois de algum tempo todos os três perceberam o ditoso clarão, o que
inclusive levou o príncipe a se restabelecer um pouco. Eckart permanecia
mudo e concentrado, não dizia uma palavra e seguia o fio dos próprios
pensamentos. Quando chegaram à cabana, bateram à porta e uma senhora
velha e bondosa lhes atendeu: eles entraram e o forte Eckart fez descer
o duque de seus ombros. Esse último, no mesmo instante, caiu de joelhos
e com uma prece ardente agradeceu a Deus por tê"-lo salvado. Eckart
sentou"-se num canto escuro e lá encontrou dormindo o mesmo velho que
lhe confidenciara tempos atrás a desgraça de seus filhos, em busca
dos quais ele saíra vagando pelo mundo.

Ao terminar sua oração, o Borgonha disse:

-- Uma maravilha se operou em meu espírito nesta noite. A bondade de
Deus, bem como sua onipotência, jamais haviam tocado tão intimamente
meu coração empedernido. Em breve morrerei, algo mo diz, e meu maior
desejo é que Deus possa perdoar meus inúmeros e graves pecados. A vocês
dois, que me conduziram a esse lugar, quero antes do meu fim compensar
tanto quanto puder. A você, meu escudeiro, presenteio com ambos os
castelos localizados aqui nas montanhas próximas. No futuro, porém,
você deve denominá"-las Tannenhäuser em memória desta noite assustadora.

Virando"-se, o príncipe prosseguiu:

-- E quem é você, homem, você que se acomodou aí no canto? Venha,
aproxime"-se um pouco a fim de que eu possa recompensá"-lo também pelo
seu esforço e pela sua generosidade.

\begin{verse}
Um passo à frente, hesita,\\
Detém"-se à luz, irradiante\\
Estaca. O príncipe fita,\\
Silencioso, o triste semblante.

Nisso Borgonha surpreso\\
O herói encara a olhar\\
Se cala de susto preso,\\
O vassalo vem apoiar.

Fraqueja e com ar doloso\\
Ao solo cai indefenso:\\
``Oh, Deus!'', brada impetuoso,\\
``De fato é quem eu penso?

Deus, onde me esconderei?\\
De quem sem ira ou rancor\\
Pela dor que lhe causei\ldots{} \\
Salvou"-me, cabal protetor?''

Lastima o príncipe e chora.\\
Contido e silente, ora.\\
De súbito, o amigo cinge,\\
Só a emoção o impinge.

Eckart almeja armistício:\\
Doravante amor como auspício!\\
``Ao reino da majestade,\\
exemplo de fraternal lealdade.''
\end{verse}

 Assim se passou a noite. Na manhã seguinte vieram outros servos que
encontraram o soberano doente. Depositaram"-no sobre o lombo de um burro
e o levaram de volta ao castelo. Eckart não podia sair de seu lado, com
frequência ele lhe tomava a mão e a apertava contra o peito, olhando o
amigo com gesto de súplica. Eckart o abraçava nesses momentos e o
consolava com palavras afetuosas que o tranquilizavam. O príncipe
reuniu em torno de si todos os conselheiros e lhes disse que instituía
Eckart fiel como tutor de seus filhos, porque ele dera demonstração de
ser um homem de grande nobreza. Em seguida, o duque morreu.

 Desde então Eckart assumiu o governo com todo zelo e todos no reino
tinham de admirar a fortaleza viril de seu caráter. O tempo foi
passando e numa certa ocasião, espalhou"-se a história prodigiosa de um
menestrel que viera da Montanha de Vênus e percorria o país de ponta a
ponta seduzindo com sua melodia as pessoas, que sumiam depois sem deixar
vestígios. Muitos punham fé no boato, outros não, e Eckart relembrou o
ancião desafortunado.

  -- Eu o adotei como filhos -- confessou aos jovens príncipes órfãos
quando se encontravam juntos uma vez sobre a montanha defronte ao
castelo. -- Sua felicidade é agora como se fosse de minha descendência,
desejo que sua alegria me prolongue a vida após minha morte.

 Eles se instalaram na encosta de onde podiam vislumbrar uma ampla visão
do belo país e Eckart afugentou a lembrança dos filhos, pois
pareceu"-lhe que eles estavam vindo ao seu encontro das montanhas lá
adiante, enquanto ele percebia vindo de longe tons de uma música
encantadora.

\begin{verse}
Oh, não parecem ser sonhos\\
Dos mais recônditos sonos\\
Reverberando profundos\\
Como canções de defuntos?

Aos jovens homens seduz\\
E aos sons de magia conduz.\\
Se esparze aos ares, percute.\\
Desperta na sábia juventude\\
Um espírito rebelde de mudança\\
Que a regiões ignotas a lança.

``Vamos às montanhas, as fontes atraem!\\
Vamos aos campos, os bosques convidam!\\
Misteriosas vozes que anseiam\\
Ao paraíso terreno enleiam.''

O bardo magnífico e cativante\\
Dos filhos de Borgonha se aproxima\\
A música envolve, deslumbra\\
Supera o solar brilho que ilumina\\
Toda a florada inebriada enrubesce\\
O crepúsculo vermelho fosforesce\\
Entre as folhagens, ventos do sul,\\
Tudo tornado em ouro, o que dantes era \qb{azul.}

Tal qual névoa difundindo"-se em langor,\\
Éter que a Terra ao espírito conforma\\
Silencia"-se de súbito todo rumor,\\
O universo, uno em flor floresce.\\
Rochas esplendem, mais exuberantes,\\
As águas fluem, bem mais radiantes,\\
Tudo erra e encerra, repleto em tons,\\
Anelo unânime de terrenos dons.\\
A alma humana chamas desprende\\
Em doce delírio, bem refulgente.

Eckart se comove,\\
Profundo ele suspira,\\
A música o envolve,\\
Confuso se admira.

Lhe afigura novo o mundo,\\
Acende em cálidas cores\\
Não sabe o que o infunda,\\
Se sente entre delícias.
 
``Não evocam esses tons violentos,\\
Pergunta"-se Eckart, o intrépido,\\
A memória da esposa, rebentos,\\
Amor e encanto pretérito?''

Súbito um estranho horror\\
Acomete o herói num instante.\\
Bastou a consciência e o pavor\\
E sentiu"-se homem novamente.

Então percebe a inquietude\\
Das criaturas aos seus cuidados\\
À mercê de inimigo rude,\\
Indomáveis, ferozes soldados.

Prá' longe, reféns vão em levas,\\
Nem mais o tutor reconhecem\\
Agitam"-se em furor como vagas\\
De mar brônzeo e selvagem.

Indeciso sobre a atitude,\\
Apelam"-lhe senso e brio\\
Mas sente hesitar a virtude;\\
O herói estranha a si próprio.\\
Relembra a hora da morte,\\
Decisiva, definitiva,\\
De novo testemunha e suporta\\
Do príncipe a despedida.

Fortifica e clareia o espírito,\\
Mantém firme seu brasão\\
Enquanto tem lá no íntimo\\
A pujança do bardo no coração.
 
A espada ele quis brandir,\\
Decapitar oponente implacável,\\
Mas ouvindo o apito silvar,\\
Sentiu"-se fremir, vulnerável.

Emergem, vindos dos montes,\\
Bizarras imagens de anões,\\
Desordenados, medonhos,\\
Que fluem em vagalhões.

Os filhos feitos reféns\\
Rebelam"-se em ira e celeuma.\\
Os esforços de Eckart são vãos,\\
Se perdem em meio ao escarcéu.

Arrebata Eckart a corrente,\\
A pujança da horda o conduz\\
Resigna"-se, seguindo em frente,\\
Nada  à bravura o induz.

Aos trancos, atingem a colina\\
Donde sons de música ressoam\\
E, imediatamente, lá em cima\\
Embargam, silenciam, recuam.

A rocha ao meio se cinde,\\
E roja de dentro um tropel.\\
Se veem figuras surgindo,\\
Ao clarão misterioso do céu.

O herói desembainha a espada,\\
Clamando ``mantenho a palavra!''\\
Bravo, luta denodado.\\
Vinga! Da turba se livra.
 
As crianças recaptura\\
De volta à fortaleza as envia.\\
A batalha porém segue dura,\\
Fôlego extremo exigia. 

Porque mal caem à terra\\
Os anões reerguem, resistem,\\
Retornam dispostos à guerra,\\
Afoitos, com ímpeto, se batem.

Do alto Eckart vislumbra\\
As crianças bem longe, no vale,\\
Diz ora ``tomara eu sucumba\\
Na batalha, na sanha do embate!''

A espada faiscante brandindo,\\
Aos raios incidentes do sol,\\
Roja"-se em fúria anões ferindo\\
Que jaziam estirados em rol.

Os fidalgos atingem a fortaleza\\
Seguros, além do horizonte,\\
É quando o ferem com crueza\\
E da vida se despede no monte.

O herói expira aliviado\\
Lutando, um leão sob o arnês,\\
Inda na morte aliado,\\
Do príncipe borgonhês.

Após morto o genitor,\\
Assumiu o primogênito\\
Que grato tecia louvor:\\
``Nosso reino hegemônico
 
A Eckart damos graças,\\
Por tão nobre sacrifício.\\
Devo a vida à pertinácia\\
ao bravo que me salvou!''

Em Borgonha, por terra e mar,\\
Criou"-se a lenda e o fascínio:\\
O homem que queira ousar\\
Ao Monte Vênus, supíneo,

Pode enxergar lá no cimo\\
Guiando a posteridade\\
A alcançar prumo e rumo\\
Eckart fantasma na eternidade.

Embora inefável alma,\\
Tutela e zela leal,\\
Por isso os nobres fidalgos\\
Exaltam o herói tão fiel!
\end{verse}


\section*{Segunda Parte}

Tinham se passado mais de quatro séculos desde a morte de Eckart fiel,
quando na corte um nobre \mbox{Tannenhäuser} gozava na qualidade de
conselheiro real de grande consideração. O filho desse cavaleiro
superava em beleza todos os outros fidalgos do país, por isso era
bem"-amado e estimado por toda a gente. De súbito, contudo, o jovem
desapareceu após ter passado por algumas experiências prodigiosas, e
ninguém tinha informação sobre seu paradeiro. Desde o tempo de Eckart
fiel corria naquele país a lenda da Montanha de Vênus, à qual se dizia
que ele teria ido e por conseguinte estaria perdido para sempre.

Um de seus amigos, Friedrich von Wolfsburg, era entre todos aquele que
com mais pesar lamentara a história do sumiço de Tannenhäuser. Os dois
haviam crescido juntos e a amizade recíproca parecia a todos ter se
convertido numa necessidade vital. O velho pai de Tannenhäuser morrera,
ao cabo de alguns anos Friedrich se casou; já um círculo de crianças
felizes o cerca e ele ainda não tivera notícia alguma do amigo da
juventude, de modo que teve de considerá"-lo morto.

Uma tarde ele se encontrava ao portão de seu castelo, quando ao longe
avistou um peregrino se aproximando. O forasteiro vestia"-se com
trajes singulares, e tanto a aparência como o porte se afiguravam
bizarros ao cavaleiro. Assim que o homem chegou mais perto, Friedrich
julgou reconhecer o companheiro perdido e, finalmente, teve a certeza
de que o estrangeiro não podia ser outro senão o velho amigo
Tannenhäuser. Surpreendeu"-se, um calafrio estranho percorreu"-lhe ao
perceber claramente como os traços fisionômicos do outro haviam se
modificado completamente.

Os dois amigos se abraçaram e se assustaram mutuamente, pois ambos se
olhavam com surpresa como duas pessoas estranhas. Sucederam"-se às
questões muitas respostas confusas; Friedrich estremecia sem cessar
ante o olhar selvagem de seu amigo, no qual brilhava um fogo
incompreensível. Depois que Tannenhäuser descansou alguns dias,
Friedrich conseguiu apurar que o amigo se encontrava numa peregrinação a Roma.

Os dois amigos renovaram logo suas conversas de outrora e se recontaram
histórias da juventude. Todavia, Tannenhäuser dissimulava 
cuidadosamente as \mbox{informações} sobre onde estivera nos últimos tempos.
Friedrich, porém, insistiu quanto a isso, assim que recobraram a antiga
familiaridade. O outro tentou durante longo tempo se subtrair ao
questionamento, porém, no final acabou cedendo aos rogos implacáveis:

-- Bem, seja feita sua vontade, vou colocá"-lo a par de tudo. Não vá
me censurar depois, se a história o encher de aflição e horror.

Eles saíram e iniciaram um passeio através da relva verdejante, onde se
sentaram. O Tannenhäuser escondeu a cabeça dentro da relva verde e, com
soluços violentos, virou"-se para o outro lado sustendo a mão direita
que o amigo lhe estendera ternamente. O atormentado peregrino
reergueu"-se e iniciou sua narrativa da seguinte maneira:

-- Creia em mim, caro amigo, que a alguns de nós é atribuído no
nascimento um espírito maligno que durante toda a vida o atormenta sem
um minuto de repouso, enquanto não consegue atingir seus fins sombrios.
Foi o que aconteceu comigo, e minha existência inteira não passa de um
parto permanente e só despertarei no inferno. É por essa razão que eu
já cumpri tantos sacrifícios e ainda há muitos outros me aguardando no
decurso de minha peregrinação, assim talvez eu consiga obter o perdão
do Santo Padre em Roma: diante dele quero depositar o pesado fardo de
meus pecados, ou sucumbirei no tormento e morrerei desesperado.

Friedrich tentou consolá"-lo, mas Tannenhäuser não parecia dar ouvidos 
às sinceras atenções do amigo, após uma curta pausa prosseguiu com as
seguintes palavras:

-- Existe um conto antigo e maravilhoso, segundo o qual séculos atrás
vivia um cavaleiro chamado Eckart fiel. Naquela época, teria saído do
interior de uma montanha estranha um menestrel cuja música misteriosa
revelava no coração de quem a ouvisse uma nostalgia tão profunda,
desejos tão ardentes que todos seriam irresistivelmente cativados e
arrastados em turbilhão pela melodia, perdendo"-se enfim naquela
montanha. O inferno abrira em par suas portas às pobres criaturas
humanas e as seduzira com sons maviosos. Como criança eu estava sempre
escutando essa lenda sem me impressionar particularmente, mas depois de
algum tempo toda a natureza, qualquer som ou flor me levava a pensar na
lenda da música comovente.

Não posso exprimir"-lhe a melancolia e a indizível saudade que me
acometiam de repente, me retinham paralisado, e tentavam me
conduzir sempre que eu contemplava o movimento das nuvens e percebia
o magnífico e límpido azul mostrando"-se em seus interstícios, nem as
lembranças que os campos e florestas vinham despertar no fundo de meu
coração. Eu era sem cessar tomado pela graça e a plenitude da natureza
esplêndida, estendia meus braços como asas, a fim de unir"-me em
comunhão no seio da natureza pelas montanhas e vales, a fim de palpitar
com toda intensidade pelas ramagens e relva e aspirar ao meu ser a
abundância de sua bênção. Durante o dia, o espetáculo espontâneo da 
natureza me encantava, ao passo que meus obscuros sonhos noturnos me
angustiavam e me apresentavam terríveis imagens ao espírito, como se
quisessem me barrar a trilha para a vida venturosa.

Um desses pesadelos, sobretudo, deixou no bojo de minha alma uma
impressão inefável, embora me fosse impossível em seguida rememorar
nitidamente as imagens em minha imaginação. Parecia"-me que havia nas
ruas enorme rebuliço, eu ouvia confusamente as conversas indistintas,
adentrei pela noite escura à casa dos meus pais e apenas meu
pai estava presente e doente. Quando o dia amanheceu eu me atirei ao
pescoço de meus pais, abracei"-os ardentemente e cingi"-os carinhosamente
junto ao peito, como se um poder hostil fosse capaz de nos separar.
``Será que eu o perderei?'', perguntei a meu bem"-amado pai. Oh! Como serei
infeliz e só sem sua companhia neste mundo! Eles me consolaram, mas não
lograram afastar de minha lembrança essa imagem sombria.

Eu crescia me mantendo sempre distante dos outros meninos de minha
idade. Vagava amiúde sozinho através das campinas, e
assim me aconteceu certa manhã de perder o caminho e errar a esmo numa
cerrada floresta, chamando por socorro. Depois de ter por longo tempo
procurado em vão me orientar, encontrei"-me de súbito ante um cercado de
ferro que circundava um jardim. Através da grade eu vi à minha frente
belas aleias sombreadas, árvores frutíferas e flores, destacando"-se
trepadeiras de rosas que refulgiam ao brilho do sol. Um inexplicável
desejo de aproximar"-me daquelas rosas tomou conta de mim, e sem poder
reprimir o impulso, esgueirei"-me por entre as barras de ferro e me
introduzi no jardim. Tão logo entrei, lancei"-me ao solo, envolvi com
meus braços as roseiras e cobri de beijos os lábios vermelhos de
suas flores, vertendo torrentes de lágrimas. Após haver perdido um
tempo nesse êxtase, vi duas meninas que foram chegando por entre as
árvores, uma mais velha, outra de minha idade. Despertei"-me do
aturdimento para abandonar"-me a um encantamento maior. Meus olhos
incidiram sobre a mais jovem e no mesmo instante foi como se eu tivesse
me livrado de todos os meus sofrimentos desconhecidos. Acolheram"-me em sua
casa, os pais das duas crianças perguntaram meu nome e enviaram
notícias ao meu pai, que no final da tarde veio em pessoa buscar"-me.

A partir desse dia, o rumo incerto de minha vida enveredou"-se por uma
direção segura, meus pensamentos voltavam"-se incessantemente ao castelo e
à jovem amiga, pois, ao que tudo indicava, constituía isso a fonte que
saciaria os meus anseios. Esqueci as alegrias costumeiras, desdenhei
companheiros de folguedos e constantemente visitava o jardim, o
castelo e as meninas. Bem cedo tornei"-me uma criança da casa, ninguém
se admirava mais da minha presença e a cada dia que passava aumentava
minha afeição por Emma. Dessa maneira transcorriam as horas de minha
infância e uma ternura sublime apoderara"-se de meu coração, sem que eu
próprio tivesse consciência disso. Meu destino parecia agora ter"-se
cumprido, eu não tinha outro anelo senão aquele convívio e, ao findar 
do dia, esperava viver um amanhã semelhante.

Por essa ocasião, a família de Emma travou conhecimento com um jovem
cavaleiro que era também amigo de meus pais e que não tardou, assim
como eu, a ligar"-se a Emma. Eu o considerei desse momento em diante um
inimigo mortal. Mas eu não \mbox{conseguiria} \mbox{descrever} o sentimento que me
acometeu quando acreditei constatar que Emma preferia sua companhia à
minha. Naquele instante, foi como se a música que até então me
acompanhara se extinguisse em meu íntimo. Morte e ódio eram meus únicos
pensamentos, eu remoía planos mirabolantes enquanto ela percutia sons
do alaúde e cantava as árias que me eram familiares. Não procurei
dissimular mais minha irritação e com relação a meus pais, que me
censuravam, eu me mostrava rude e desobediente.

Pus"-me, então, a vagar sem destino no meio de florestas e sobre
rochedos, enfurecido comigo mesmo: decidi pela morte de meu
antagonista. Alguns meses mais tarde o jovem cavaleiro pediu aos pais a
mão da minha bem"-amada e ela lhe foi prometida. Tudo que me atraíra e
encantara maravilhosamente no conjunto da natureza havia se amalgamado
na imagem de Emma; eu não conhecia, nem queria conhecer outra
felicidade além dela, estava inclusive determinado arbitrariamente a
fazer coincidir num único dia sua perda e meu perecimento. Meus pais se
afligiam acompanhando meu embrutecimento; minha mãe adoecera, porém
isso não me sensibilizava: não me importei o mínimo que fosse com sua
condição e a via raramente.

O dia das bodas de meu rival se aproximava e com essa proximidade
aumentava minha angústia, que me empurrava com ímpeto às florestas e
por sobre os rochedos. Eu esconjurava Emma com as mais funestas
maldições. Naquela época, eu não tinha um só amigo, ninguém se ocupava
de mim, pois todos me consideravam um caso perdido.

Chegou a noite abominável da véspera do dia do casamento. Eu me perdera nas
pirambeiras dos rochedos íngremes e escutava lá embaixo a torrente
rugindo, às vezes me assustava com minhas próprias atitudes. Quando
enfim amanheceu, eu vi meu inimigo descendo as montanhas, o abordei com
insultos, ele se defendeu, sacamos nossas espadas e logo ele tombou sob
meu golpe furioso e letal.

Eu fugi precipitadamente sem olhar para trás, mas seus companheiros
transportaram o cadáver. Noites a fio eu perambulava ao redor da morada
que abrigava minha Emma e poucos dias depois distingui no convento da
vizinhança os dobres de finados e o canto fúnebre das religiosas. 
Informei"-me: disseram"-me que devido à dor pela morte do noivo a senhorita
Emma em seguida falecera. Não sabia o que fazer, hesitei se de fato
estava vivendo aquilo, se tudo era real. Tomei rapidamente o caminho da
casa de meus pais, e na noite seguinte cheguei tarde demais à cidade
onde moravam. Vi uma tremenda inquietação, as ruas estavam repletas de
cavalos e carros de equipamentos, lanceiros se misturavam em desordem e
davam confusamente a entender que era chegada a hora em que o imperador
partiria em campanha contra seus inimigos.

Ao chegar à casa paterna, vi lumiar uma única luz; um pesadume opresso
apertava"-me o coração. Quando bati à porta, meu pai em pessoa veio ao
meu encontro com passadas silenciosas. Imediatamente ocorreu"-me a
reminiscência do velho sonho dos anos da infância e emocionei"-me
fortemente sentindo como era o pesadelo de outrora agora tornado
realidade. Atônito, \mbox{perguntei}: ``Pai, por que o senhor está acordado a
uma hora dessas?'' Ele foi me conduzindo casa adentro, e respondeu: ``Eu
preciso velar, pois agora sua mãe também está morta''.

Suas palavras trespassaram minha alma como raios. Ele sentou"-se com
cautela, tomei lugar ao seu lado, o cadáver jazia estirado sobre o
leito, estranhamente escondido por um reposteiro. Meu coração batia em
rompantes. ``Eu me manterei em vigília'', disse o velho, ``porque minha
esposa permanece sentada perto de mim''. Fui perdendo os sentidos, fixei
o olhar num canto do aposento e após uns instantes algo se moveu como
um vapor, pairava e ondulava, a imagem familiar de minha mãe se
constituiu visível, me olhando com uma expressão de gravidade. Fiz
menção de fugir, mas não fui capaz de me mover, a figura materna me fez
um sinal e meu pai me reteve preso em seus braços, sussurrando baixinho
ao meu ouvido: ``Morreu de desgosto por sua causa''. Eu o abracei com
todo meu ardor filial, derramei lágrimas ferventes em seu peito.

Ele me beijou e eu estremeci ao contato daqueles lábios gelados como os
de um defunto que me tocavam. ``O que há, pai?'', indaguei apavorado. Ele
teve um estremecimento convulso e não respondeu. Num minuto eu senti
que ele esfriava, procurei ouvir os batimentos do coração, não
palpitava mais; em minha loucura triste eu continuava estreitamente
abraçado a um cadáver.

Como um clarão semelhante ao primeiro albor da aurora, eu vi através do
aposento escuro o fantasma de meu pai sentado junto à imagem de minha
mãe, e \mbox{ambos} me contemplaram com compaixão, como eu continuava a manter
num abraço o cadáver da pessoa querida. A partir daí meus sentidos se
turvaram; os criados me acharam pela manhã na câmara mortuária
desprovido de razão e de forças.

Tannenhäuser chegara até esse ponto de sua narrativa, enquanto o
amigo Friedrich escutava bastante admirado. Mas de súbito ele
interrompeu a história com uma \mbox{expressão} confrangida de dor. Friedrich
parecia um pouco constrangido e não dizia nada. Os dois amigos
caminharam de volta ao castelo, mas permaneceram ainda a sós numa sala.

Depois de uma pausa silenciosa, Tannenhäuser retomou o fio da narrativa:

-- Essas lembranças tristes sempre me tocam profundamente, eu não consigo
entender como pude sobreviver a isso tudo. Doravante a terra e a
existência se afiguraram para mim mortas e desoladas, eu me arrastava
desatinado e desgostoso dia após dia. Foi quando me juntei à companhia
de jovens tresloucados, e na bebida e nos prazeres busquei apaziguar o
espírito maligno que me habitava. A agitação ardorosa e peculiar voltou
à tona e eu próprio não era capaz de compreender meus ávidos anseios.
Um libertino chamado Rudolf tornou"-se meu amigo íntimo, mas sem cessar
ele ria de minhas queixas e de minha nostalgia. Cerca de um ano se
passou, minha angústia me levava às raias do desespero; eu me sentia
impelido a partir, a ir cada vez mais longe, a regiões ignotas; quis
precipitar"-me do alto das montanhas às pradarias verdejantes e
brilhantes, ao bramido gelado da torrente, a fim de \mbox{estancar} a sede
ardente e insaciável de minh'alma. Aspirava em meu ânimo o desejo da
morte, e novamente pairavam ante mim, e me enleavam a segui"-los como
dourados estratos crepusculares, a esperança e o anseio de viver. No
entretempo, ocorreu"-me que o inferno me cobiçava e me enviava esses
sinais de dores e alegrias para perder"-me; que um espírito pérfido
orientava as energias de minh'alma rumo às profundezas sombrias,
atraindo"-me ao abismo.

Resolvi abandonar"-me a ele a fim de livrar"-me das torturas que
se alternavam aos acessos de euforia. Na noite mais tenebrosa, eu subi uma
montanha íngreme e gritei a todos os pulmões o inimigo de Deus e dos
homens; e o fiz com tanta convicção, que tive o pressentimento de que
ele deveria me obedecer. Ele ouviu meu apelo, de repente se encontrava
ao meu lado e não me provocou nenhum horror. Durante a conversa com
ele, despertou em mim mais uma vez a crença na montanha misteriosa, e
ele me ensinou uma canção que me faria achar espontaneamente a trilha
certa que me conduziria até lá. Depois sumiu e pela primeira vez em
minha existência eu me vi só comigo mesmo, pois nesse instante eu
compreendia o conjunto de meus pensamentos dispersantes que ansiavam
divagar a partir do meu centro em busca de novas paisagens.

Pus"-me em marcha, e a canção que eu entoava em voz alta me levava
através de prodigiosas solidões, eu esquecera tudo o mais em mim e 
a minha volta. Ela me transportava como sobre imensas asas do desejo ao cerne
de minh'alma. Eu queria escapar à sombra ameaçadora que mesmo na
claridade nos espreita, às \mbox{dissonâncias} que em plena harmonia
musical se imiscuem e nos perturbam.

Foi assim que numa noite, quando o clarão pálido do luar brilhava ainda
difuso por trás de nuvens escuras, eu me deparei com a montanha.
Prossegui cantarolando a canção, e uma figura gigantesca assomou ante
mim e, com o cajado, fez"-me um gesto para retroceder. Fui chegando perto.

 -- Sou Eckart fiel! -- apresentou"-se a figura sobrenatural. -- A bondade
divina postou"-me aqui a fim de zelar e refrear o temor maligno do
homem.

Passei ao largo.

Meu caminho assemelhava"-se a uma galeria de mina subterrânea. A passagem
era tão estreita que eu precisava me esgueirar entre as encostas.
Percebia sussurros das correntes aquáticas ocultas e itinerantes, ouvia
os espíritos da terra configurando os minerais como o ouro e a prata,
para atiçar a cobiça humana; encontrei ali, escondidos naquele ermo, os
sons e as notas das quais se compõem a música terrena. Quanto mais me
aprofundava, mais sentia um véu se abrindo ante meus olhos.

Repousei um tempo e vi algumas silhuetas humanas avançarem vacilantes,
entre elas estava meu amigo Rudolf. Não conseguia entender como podiam
cruzar comigo, pois a senda era demasiadamente estreita, mas elas
passavam através das rochas sem me perceber.

Logo eu ouvi uma música; porém completamente diferente daquela que até
aí se alçara a meus ouvidos. Os espíritos em mim se empenhavam a ir ao
encontro dos tons. Atingi uma área livre e de todas as partes brilhavam
cores maravilhosamente luminosas. Eis ali tudo o que eu sempre
almejara. Bem junto ao peito eu sentia a presença do esplendor buscado
e finalmente encontrado; jogos de êxtase me impregnavam com toda a
força. Veio ao meu encontro o bulício estonteante dos pândegos deuses
pagãos; a Dama de Vênus à frente, e todos me saudavam. Eles foram
exilados ali pelo poder do Todo"-Poderoso e seu culto desapareceu da
face da Terra; dali então, em completo segredo, continuavam irradiando
sua influência.

Todos os deleites terrenos eu desfrutei e gozei em toda a plenitude;
insaciável era meu afã e infinito o prazer. As célebres belezas do
mundo antigo estavam presentes, o que o pensamento desejava eu possuía,
uma embriaguez seguia à outra, cada dia me parecia cintilar em cores
vibrantes. Jorros de vinhos requintados abrandavam minha sede ardente e
as figuras mais graciosas flutuavam pelo ar, moças nuas me cercavam
sedutoras, fragrâncias encantadoras rescendiam em torno de minha
cabeça, como do mais profundo âmago da natureza soava a música cujas
ondas sutis acalmavam a loucura ávida da volúpia; uma sensação de
horror vagando misteriosamente sobre os prados floridos aumentava o
inebriante langor. Quantos anos se esvaíram nesse enlevo eu não saberia
dizer, pois nesse estado não havia tempo nem diferenças: nas
florescências ardia o fulgor das mulheres e dos prazeres; o corpo das
mulheres exalava o viço das flores. As cores expressavam uma outra
linguagem, os sons articulavam palavras novas; o mundo sensível inteiro
estava entrelaçado num único buquê \mbox{formoso} e os espíritos festejavam
num triunfo excessivo e incessante.

Mas aconteceu, sem que eu pudesse conceber a razão, que em meio a todo
esse esplendor impuro, eu fui dominado por uma nostalgia da paz, uma
saudade da Terra antiga e inocente com suas alegrias castas e singelas,
e isso com o mesmo ímpeto que outrora me arrojara até ali. Atraía"-me
agora a possibilidade de viver a vida que os homens levam em total
inconsciência, com alegrias e tristezas se alternando; eu estava
saciado do delírio e dos devaneios, e de bom grado voltaria à minha terra
natal. Uma incompreensível graça da providência permitiu"-me o retorno,
de repente eu me encontrava mais uma vez no mundo, e penso agora
extravasar meu coração repleto de pecados diante do trono de nosso
Santo Padre em Roma, a fim de que ele me perdoe e eu reconquiste meu
lugar entre os homens comuns.

Tannenhäuser se calou e Friedrich o considerou detidamente com um
olhar perscrutador. Em seguida, ele tomou a mão do amigo e lhe disse:

-- Não consigo superar minha estupefação, tampouco entendo sua história.
Só me resta atribuir tudo isso a um fruto de sua fértil imaginação.
Porque Emma vive, Tannenhäuser, ela é minha esposa e você e eu nunca
nos batemos, nem nos odiamos, conforme você descreveu; você sumiu da
região antes de nosso casamento, sem nunca ter me feito qualquer
confidência sobre o amor que nutria por Emma.

Depois disso ele puxou o confuso amigo pela mão e o conduziu à sala
contígua, ao encontro da esposa, que retornara há pouco ao castelo após
uma visita de alguns dias à irmã. Mudo e ensimesmado, 
Tannenhäuser contemplou a aparência e a fisionomia da mulher, e batendo
a mão na testa, exclamou:

-- Por Deus! Essa é a aventura mais insólita que já me sucedeu!

Friedrich fez um relato coerente de tudo que lhe acontecera desde que
Tannenhäuser desaparecera, procurou dar a entender que uma loucura
estranha simplesmente o vinha perturbando há alguns anos.

-- Sei muito bem o que é isso! -- replicou Tannenhäuser. -- Agora vejo
fantasmas, sou louco; o inferno é que insidia essas tramoias, a fim de
evitar minha peregrinação a Roma e a remissão de meus pecados!

Emma procurou em vão dissuadi"-lo da ideia, recordando casos da infância
comum, mas Tannenhäuser não se deixou convencer. Tomou imediatamente
o caminho de Roma para obter enfim a absolvição divina.

Friedrich e Ema falaram ainda muitas vezes sobre o estranho peregrino. 
Alguns meses tinham transcorrido quando Tannenhäuser,
pálido e descaído, vestindo trajes de peregrino e descalço, entrou certo
dia nos aposentos de Friedrich, no momento em que ele ainda dormia.
Inclinou"-se e depositou"-lhe um beijo sobre os lábios, e rapidamente
pronunciou as seguintes palavras:

-- O Santo Papa não quer, nem pode jamais me conceder o perdão, preciso
retornar à minha antiga morada.

Em seguida ele se afastou apressadamente.

Quando Friedrich despertou, o infeliz já tinha desaparecido. Levantou"-se
e se dirigiu ao aposento da \mbox{esposa}, e as criadas o receberam aos
prantos: \mbox{Tannenhäuser} penetrara ali ao raiar do dia e pronunciara uma
sentença:

-- Que ela não estorve meu caminho!

Encontraram Emma assassinada.

Nem bem recobrara o espírito após o abalo, Friedrich foi acometido de um
pavor: pleno de agitação, correu para o ar livre. Tentaram retê"-lo, mas
ele contou que o beijo do peregrino sobre seus lábios queimava e
continuaria ardendo enquanto não o encontrasse novamente. Desse modo,
ele travou uma cruzada insana e inconcebível à procura da montanha
misteriosa e de Tannenhäuser, e desde então ninguém mais o viu. As
pessoas diziam que quem recebe um beijo de alguém da montanha não pode
resistir à fascinação, à violência do encantamento que o lança, ele
também, às profundezas subterrâneas.
\medskip

\hfill\textit{Tradução de Maria Aparecida Barbosa}

