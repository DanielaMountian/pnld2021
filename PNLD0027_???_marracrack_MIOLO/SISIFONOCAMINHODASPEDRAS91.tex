\chapterspecial{Sísifo no caminho das pedras}{}{}
 

\versal{MM} tem trinta anos. Trabalha com informática e vendas. É~conservador,
rígido, exigente quanto a seus princípios e com tendência moralista.
Fica mais à vontade na área do pensamento do que do sentimento. Tem
discurso claro, por vezes algo obsessivo, e girando em torno do trabalho
e obrigações familiares.

Vejo agora \versal{MM} à minha frente. Ele tem uma tristeza vaga no olhar parado.
Seu corpo está meio largado. Tenho a impressão de que ele seja só
cabeça, e que esta gira solta sem muita firmeza. Do pescoço para baixo o
restante parece pender quase sem movimento.

Os braços são finos, flácidos. As pernas são longas e não parecem dar
boa sustentação ao corpo. Há nele todo uma ausência de formas robustas,
que \versal{MM} atribui à droga maldita --- o crack.

\asterisc{}

\versal{MM} discursa um longo tempo sobre sua condição de dependente químico.
Repete falas retóricas a respeito de recuperação. Vem com o mesmo
blá"-blá"-blá de querer ser um ``novo homem''. Repudia, como outros o
fazem, a droga genérica e tece teorias batidas a respeito da droga mais
maldita --- o crack, novamente, é claro.

Confessa fumar três pedras ao dia, sistematicamente. Mas é responsável
pelos seus atos e paga com seu suado trabalho cada uma das pedras. Nunca
fica a dever nada na biqueira. Jamais recorre a expedientes escusos do
tipo furtar.

\asterisc{}

\versal{MM} desvela sua história.

Na infância foi garoto ``de família''. Era delicado, talvez bonito.
Tinha uns cabelos loiros compridos, chamava a atenção das pessoas.

O pai morreu quando ele estava com cinco anos e a mãe ficou quase ao
Deus dará vivendo com uma pensão de um salário mínimo.

A família, que morava em Ermelino Matarazzo, foi morar no Jardim das
Oliveiras, que é um subdistrito do Itaim Paulista. E~se o Itaim
Paulista é terra um pouco sem lei, o Jardim das Oliveiras de martírio
tem a alusão do nome e mais uma realidade brutal de miséria de vários
naipes.

É lugar notório pela violência, pelo crime, pelos garotos delinquentes,
pelo narcotráfico. É onde muitos meninos têm a perspectiva de serem
bandidinhos nas ruas para se imporem uns aos outros.

Quem sai do Jardim das Oliveiras e chega no bairro próximo do Itaim
Paulista tem um choque de civilização maior do que alguém que, vindo do
Itaim Paulista, chegue ao centro de São Paulo.

\asterisc{}

Aos treze anos \versal{MM} descobriu que era apenas o menino bonito cobiçado
pelas meninas, e o ``boyzinho'' odiado e invejado pelos garotos da
pesada.

Certo dia um incidente levou \versal{MM} a brigar com um garoto bandido que o
ameaçou com uma arma de brinquedo. Depois o ameaçou com uma arma de
verdade. Depois disparou um tiro contra ele que felizmente não pegou.

\versal{MM} correu apavorado para casa buscando ajuda da mãe. Mais tarde mudou de
estratégia. Fez amizade com um menino da turma da pesada que o protegeu.
Mas o menino exigiu de \versal{MM} adesão à turma, pois caso contrário todos o
``zoariam'' e, além dele virar saco de pancada, correria até o risco de
ser morto.

Este é um costume ou uma ``lei'' do Jardim das Oliveiras. Garoto sem
formação nos códigos locais cresce perigosamente em terreno árido onde
quem pode mais chora menos. A~melhor política juvenil é mostrar que se é
macho juntando"-se aos da pesada. Inclusive para sair por aí roubando.

Foi precisamente o que aconteceu.

\versal{MM} achou que se entrasse na turma da pesada teria o respeito. De fato,
teve.

Começou a fazer pequenos furtos em supermercados. Depois a roubar
carros. E~tudo ia dando certo. A~grana chegava na mão e era muita para
garotos pobres hipnotizados com tais facilidades.

A realidade, porém, provou ser diferente da ilusão inicial. \versal{MM} apanhou
da polícia, apanhou de malandros rivais, sofreu humilhações terríveis e
teve três passagens na antiga Febem. Provou, enfim, de vários martírios
desde a época do Jardim das Oliveiras quando viu sua carreira de menino
de família se desmoronar.

Quando completou dezoito anos \versal{MM} teve o que ele chamou de um ``estalo'',
aquilo que eu poderia também chamar, e como não, de um pequeno momento
de ``iluminação''.

Mas não foi um momento místico e nem vocacionado religiosamente. Apenas
um \emph{insight} oportuno vindo de uma psicologia intuitiva que a vida
propicia e da qual muita gente infelizmente não sabe tirar proveito.

\versal{MM} percebeu onde deveria colocar seus pés. Percebeu que não era sua
``praia'' andar em gangues atrás de aventuras escusas.

Acordou de um longo pesadelo, e não demorou para que ele e a mãe saíssem
do Jardim das Oliveiras e fossem morar num bairro melhor da Zona Leste
de São Paulo. Onde \versal{MM} resolveu encarar o mercado de trabalho e onde se
ajeitou no ramo de informática e vendas. Lá também conheceu sua esposa
de quem tem um filho hoje com quase dez anos.

\asterisc{}

\versal{MM} havia feito sua iniciação com várias drogas psicoativas no tempo da
sua turma da pesada no Jardim das Oliveiras. Bebeu, fumou maconha,
cheirou lança e experimentou farinha branca. Mas foi com vinte e cinco
anos, tardiamente, e já vivendo como trabalhador na área de informática,
que conheceu o crack.

Depois de alguns meses de uso tornou"-se dependente, embora a dependência
fosse marcada não pela busca crescente e compulsiva, mas sim pela
regularidade de duas, e depois sempre três pedras diárias. Fumando em
casa, sozinho, e cumprindo um monótono ritual.

\versal{MM} não viveu fase alguma de ``noia''. Não perambulou nas ruas. Não
torrou pertences na biqueira. Não foi estigmatizado como vagabundo. E~embora se sentisse meio largado e com pouca motivação, permaneceu
recluso no ambiente doméstico segurando"-se na corda bamba do emprego,
financiando seu ``vício'' com o trabalho e temendo ser despedido.

\versal{MM} sempre foi um rapaz cauteloso. Porém testemunhou alguns horrores do
que chama de ``vida errada''. Certa vez viu um amigo ser assassinado por
causa de dívida de droga, e trancou"-se durante quatro meses em casa num
estado de depressão.

\asterisc{}

\versal{MM} faz questão de rever sua trajetória de vida, até quando virou pai de
família e trabalhador, e desde as primeiras fases no Jardim das
Oliveiras, na época de sua adesão à gangue.

Ele me faz questão de dizer que, quando moleque, conheceu um ``paraíso''
da contravenção, para onde foi levado pelo medo de ser agredido, e para
onde também foi levado pela sedução de transgredir, de conhecer a
adrenalina do crime, o dinheiro fácil e o ganho de confiança da turma.

\asterisc{}

\versal{MM} dá uma longa pausa reflexiva e eu procuro ajudá"-lo a chegar a uma
possível e primeira conclusão de sua história de vida.

Neste momento eu me torno um pouco ousado com ele, e até mesmo um pouco
cínico. Proponho que ele faça uma espécie de inventário de sua vida.

Pois sendo assim, por que não dar início a uma conversa terapêutica
evocando o que teria ocorrido ao longo da via ``pedagógica'' popular
desde o tempo do Jardim das Oliveiras?! Seguindo o itinerário da porrada
e do \emph{bullying}. Onde alguns ``bem sucedidos'' entram no sistema e outros
são massacrados. Onde raros permanecem neutros.

Este é um caminho de acesso existencial a \versal{MM}, e se até aí, caro leitor,
não há muita novidade debaixo do sol, eu fico me perguntando melhor a
respeito da iniciação dele na droga e no crime.

Pois está presente, no caso deste rapaz, uma história de iniciação em
atividades suspeitas em alguém que, no fundo, tem um caráter um tanto
rígido, não tem queda para a contravenção, mas teve vulnerabilidade
suficiente para se deixar seduzir por ela. E~na fase tardia do crack, \versal{MM}
tornou"-se um adicto do tipo cauteloso, metódico. Que foge ao
estereótipo, que é até mesmo um adicto ``politicamente correto'' e tem
sido usuário anônimo, ou mesmo invisível.

Pergunto novamente a \versal{MM}, como sempre pergunto aos outros, a respeito do
prazer do crack. Ele me diz o que muitos dizem. É~o mesmo jogo de
palavras, de gírias, para expressar sensações impactantes e efêmeras.

Depois de alguma retórica banal para tentar me traduzir as peripécias do
que seria o prazer, \versal{MM} acrescenta uma mistura de tristeza e nostalgia na
confissão de um passado que não retorna, mas que lá no fundo persiste na
esperança de que retorne.

No entanto, existe aqui um detalhe significativo. \versal{MM}, ao mesmo tempo em
que segue evocando sua trajetória, parece querer dar vida àquele garoto
loiro e delicado~que vivia tranquilo antes do episódio violentíssimo de
\emph{bullying}, ao mesmo tempo em que ele parece querer sepultar o
outro garoto \versal{MM} delinquente.

Cautelosamente, eu digo que esse menino loiro e delicado perdido no
Jardim das Oliveiras habita dentro de \versal{MM} igualmente como habita também
no seu íntimo o menino contraventor. Que \versal{MM} portanto não se iluda
vendo"-se hoje apenas em seu pífio moralismo ``pequeno-empresarial'' do
tipo Zona Leste distante, com seu discurso social de pai exemplar e de
cidadão trabalhador.

\versal{MM} então admite certas dúvidas a respeito de quem seja ele
verdadeiramente ao longo das fases de sua vida. Cai em si pensativo e se
questiona e me questiona sobre o sentido daquela rotina insuportável e
monótona de fumar três pedras diárias, três pedras que se esgotam, cada
uma, em alguns minutos. E~ainda mais porque essas pedras são tão iguais,
tão impositivas, tão compulsivas. Tão monótonas. Tão repetitivas.

Eu digo a ele que as três pedras diárias de crack estão ali presentes, e
desde muito tempo, a lhe cutucar suas ilusões perdidas e renovadas a
cada dia. Mas é inútil demonizar as pedras enquanto, ao mesmo tempo, \versal{MM}
se organiza para financiar sua dependência de maneira ``correta'',
julgando"-se um trabalhador honesto e tentando se contrapor à imagem do
contraventor que havia sido.

Eis que, de repente, perpassa para mim que, lá do fundo de \versal{MM}, existe
uma malandragem dissimulada e uma arrogância sutil. E é quando eu
acredito ter um ``estalo'' para poder compreender melhor o caso \versal{MM}.
Lembro"-me do mito grego de Sísifo, que faço questão de transmitir a \versal{MM}
com didática insistência.

Conto a ele uma breve história ancestral.

Sísifo foi muito pretensioso. Arrogante. Desrespeitoso aos deuses.
Ousado, esperto e calculista. Planejou um golpe contra o deus dos
mortos, contra Hades, que significa o ``invisível''. Mas Sísifo
enganou"-se em sua malandragem, e seu castigo terminou no impasse
horrendo do ir e vir morro acima e morro abaixo sempre empurrando uma
pedra e a mesma pedra.

Enfim, a sós…

Tenho impressão de que \versal{MM} acabou de levar um susto. Este mito é sua cara
e seu jeito.

É como se a malandragem de Sísifo e a de \versal{MM} quisessem, ambas, tão
distantemente e cada uma a seu modo, desnudar o invisível. Como se as
monótonas pedras de \versal{MM} fossem um antiamuleto ou fossem bilhetes de uma
loteria maldita desde os tempos mais antigos. Desde os tempos bíblicos.
Desde os tempos da Babilônia, desde o início dos tempos da aventura
humana no planeta Terra. Bilhetes que ele, ao mesmo tempo, soubesse que
venham a ser, no fundo do seu ``ser'', uma grande farsa. Ou um
passaporte expirado para a ``terra do nunca''.

\versal{MM} está parado olhando para o infinito. Eu transito junto com ele no
domínio virtual de uma ilusão e de uma repetição! Ah, e eu descubro que
acabo provocando um pouco este \versal{MM}. Digo a ele que o seu uso diário de
crack é uma loteria maldita e é uma loteria de Sísifo.

Ele me devolve um sorriso vago, sem graça.

Eu acrescento que há ainda um porém: se a vida é um retorno a um ponto
de partida e se a vida pode ser um eterno retorno, nem todos os retornos
são iguais. Pode haver surpresas. De repente, numa encruzilhada, o
caminho, até de um Sísifo, pode~ser outro…

\begin{center}\asterisc{}\end{center}
%\begingroup\small

\emph{De início a entrevista correu como se nada de mais acontecesse. Mas
é justamente esse o x da questão. Note bem, leitor, esta frase que
repito: ``como se nada de mais acontecesse''.}

\emph{Vejamos: \versal{MM} era um usuário e um dependente de crack e foi usuário
de outras drogas. Seguiu uma trajetória de vida na periferia do Itaim
Paulista não tão diferente da trajetória de outros meninos que se
envolveram temporariamente no crime. Mesmo porque roubar e andar em
gangues pode não ser a ``praia'' de muitos deles durante muito tempo, e
pertencer à contravenção requer certa ``vocação''.}

\emph{No entanto, há um diferencial em \versal{MM} que eu captei depois, e eu
assim o fiz porque trouxe a meu favor, mais outra vez, a mitologia
grega.}

\emph{A~chave que lá encontrei foi o mito de Sísifo, e por isso exponho
agora uma questão interessante: tenho identificado, entre os dependentes
graves de crack, padrões diferentes ou uma tipologia que permite certa
categorização. Isso tem a ver com a questão complexa do prazer e da sua
repetitiva busca com relação à espera infinita da ``brisa''.}

\emph{Explico melhor: entre dependentes químicos em geral há, no mínimo,
duas categorias básicas: os ``buscadores de novidades'' que são
inquietos, morbidamente românticos e hiperativos, e os que, de um outro
lado, não buscam novidades, são apáticos.}

\emph{Os do segundo grupo buscam, ao contrário, a inação, a rotina e a
ausência de movimento. \versal{MM} teria tudo para ser um inquieto ``buscador de
novidades''. Curiosamente, ele não é. Revelou"-se um rapaz engessado numa
inércia depois de ter passado por aventuras no crime em decorrência de
um nefasto episódio de} bullying.

\emph{Quando \versal{MM} entrou na onda do crack, sua fissura não ficou
direcionada para fazer ``rolês'' andando loucão pelas ``quebradas'', mas
foi direcionada pela monótona insistência de se recolher em casa na
rotina infernal das pedras.}

\emph{Eis aí uma faceta um tanto obsessiva e seguindo na busca de uma
tola racionalidade adicta. Sem falar que \versal{MM} é um adicto atípico e até
``invisível'' socialmente. Pode"-se considerá"-lo até mesmo um adicto
``politicamente correto'' em razão da sua ``ética''.}

\emph{Foi por isso mesmo que o mito de Sísifo me ajudou
providencialmente a compreender \versal{MM}. Ainda mais porque, em contraposição
ao caso \versal{MM}, existem outros adictos próximos de um Ícaro (o que voou alto
demais), especialmente os jovens ``buscadores de novidade''.}

\emph{Esses últimos são mais ostensivos, a exemplo de alguns ``noias''
zumbis circulando por aí. Já os ``sísifos'', e possivelmente também os
``narcisos'', carregam todos eles sua pedra ou carregam sua imagem
refletida monotonamente. Seu suspeito e efêmero prazer estiola"-se no
conformismo e na rotina que eles detestam, mas eles estão posicionados
frente a ambos --- pedra e imagem --- e ligados pela compulsão.}

\emph{Este é o \versal{MM} talvez mais verdadeiro que acabei descobrindo por
detrás de uma fachada de mesmice e caretice. Este é o \versal{MM} que se revelou
paradoxalmente rico no meio da inação.}

\emph{A ``brisa'' e a busca da ``brisa'' variam muito para cada usuário,
embora a droga química seja a mesma. É~justamente por isso que
dependência de drogas é assunto complexo, é terreno móvel. É~justamente
por isso que não hesito em afirmar: o que decide o entendimento de
um caso são as especificidades individuais e não as generalidades.}

\emph{Quanto ao diagnóstico de \versal{MM}, eu considero a possibilidade de uma
distimia (depressão leve), mas também consigo identificar em \versal{MM} um
extremado tipo psicológico pensamento (na acepção junguiana) e uma certa
rigidez neurótica.}

\emph{Há nele, é claro, muitos sofrimentos e ``traumas'' do passado,
como o episódio repulsivo de} bullying. \emph{Todavia, antes de se pensar em
transtornos definidos ou até em doenças, é melhor considerar que os
problemas de \versal{MM}, além de serem ditados pela monotonia das três pedras
diárias de crack, decorrem da vida pequena como sendo, por vezes, cruel
e opressiva, cuja rotina de economia no mercado dos produtos ilícitos
não dá chances para saídas criativas.}

\emph{Pelo contrário, creio que se possa dizer que um insosso princípio
microeconômico de realidade alarga porcamente as possibilidades de
consumo na vida de um dependente.}

\emph{Se \versal{MM} encontrou uma saída numa insossa repetição consumista, ele o
fez na droga pesada, embora já numa fase em que tivesse virado um
trabalhador supostamente normal.}

\emph{Pois é, caro leitor, os vários caminhos da adicção são complicados
quando imersos no anonimato e quando fogem aos estereótipos empobrecidos
midiaticamente. Ainda mais quando imersos na invisibilidade depressiva
de um Sísifo de periferia.}
%\endgroup