\chapterspecial{Nunca mais}{}{}
 

Dezenove anos, mas parece bem menos. Tem aparência de menino novo.

Apresenta"-se que nem um robô. Meio duro, como se fosse um soldado
oriental obedecendo ordens.

A princípio não me olha nos olhos. Mantém o olhar para a frente sentido
infinito.

O tipo físico dele, e como se veste, é o comum: bermuda, camiseta preta
um pouco rota, boné preto, sandálias havaianas.

Sua expressão baça e vaga parece não dizer nada. A~voz é para dentro,
baixa, economizando palavras, como se tivesse preguiça de falar.

Não se mostra agressivo nem é mal educado. Está sumamente indiferente.

À primeira vista seria confundido com quem sofresse de profunda
depressão.

Ele responde secamente às minhas perguntas.

Seu discurso, que não está fragmentado, tem uma coerência básica em meio
a períodos longos de incômodo silêncio.

Mas não verifico cisão psicótica, embora o quadro do momento seja
depressivo e as palavras dele fluam lentas com pobreza de expressão e
exiguidade de respostas.

Ele é um dependente gravíssimo de crack. Que está acomodado a viver
somente em função do uso e da expectativa da pedra.

Mas agora percebo que ele está querendo dizer alguma coisa mais
profunda.

De repente, entrevejo, na postura contemplativa e monótona do rapaz, um
flerte antecipado e insistente com algo que já está na ponta de sua
língua.

Que está quase expresso no seu movimento labial. Que eu já intuo e o
rapaz logo me adianta antes que eu fale ou pergunte qualquer coisa.
Simplesmente ele me diz, como se referisse a um outro estado de
consciência:

— É a brisa da morte.

Ah, respondo prontamente, e pouco à vontade: então é isso que você quer
dizer, a ``brisa da morte''?!

Ele apenas ecoa:

— É a brisa da morte.

Aqui, dentro de uma saleta, sinto"-me convidado a continuar num ``jogo''
de provocações e insisto em dizer que considero essa busca pela ``brisa
da morte'' interessante.

Afinal de contas, se todos nós vamos morrer um dia, a todos nós
indistintamente interessa especular a respeito do que seria essa ``brisa
da morte''.

No entanto, reflito que haveria aí um porém: não se trataria da morte
vulgar e sim da morte com vigor e mistério, ou daquele domínio
desconhecido de cujas fronteiras nunca se volta, como diz Hamlet
(\emph{ah, from whose bounds no traveler returns…}).

Eis tudo o que me vem à mente e o que rápido acabo de pensar com meus
botões enquanto faço uma ponte parodiando Hamlet, enquanto continuo a
falar com meus botões.

\asterisc{}

A seguir faço uma pausa, e especulo que o caso do rapaz confirma a
suspeita de que no curso profundo do prazer negativo com a droga, e na
adicção mais severa, há um enamorar"-se pela morte, há um flerte por uma
``adrenalina'' radical no limite máximo dessa radicalidade.

Este rapaz incorpora o sentido terminal do ato de entregar"-se com
volúpia passiva e enlouquecida a uma substância psicoativa que, na
ideação do dependente grave, é passaporte para outro domínio que ele
julga ainda poder alcançar.

Como se, ao drogar"-se profundamente, ele realizasse aquelas viagens
feitas por Hamlet na imaginação e em seu suposto fingimento de
insanidade.

\asterisc{}

Mas quem eu vejo à minha frente, de novo e insistentemente, é um quase
menino.

Quem eu vejo à minha frente é uma criatura muito pobre. Que nunca teve
contato com pai ou mãe. Que mora com uma avó --- outra criatura simplória
que se desespera com a inação do neto, sabendo que ele usa drogas
pesadas e fica grande parte do dia num canto, ensimesmado só pensando em
usar, ou fazendo o diabo e se articulando para usar.

Ele ainda é um rapaz socialmente marginalizado. Sua educação formal é
nula. Sua consciência de cidadania é ridícula. Seu futuro é o futuro de
uma ilusão.

Se o denominador comum de sua adicção é a busca de uma profunda
embriaguez, esse denominador comum é também um vale-tudo consumista que
generaliza o que entra pela boca ou pelo nariz.

Como dizem os psicanalistas, trata"-se de uma paixão oral. Eu diria que,
nessa paixão voraz e primitiva, tudo o que cai na rede é peixe, sem
limite ao que possa ser devorado compulsivamente.

\asterisc{}

Faço ao rapaz perguntas triviais. Quero saber do seu dia a dia pois
acredito que o dia a dia para a maioria das pessoas deste planeta
chamado Terra seja rotineiro ou então medíocre.

Fazendo tais perguntas eu me reporto a trivialidades compatíveis com sua
idade jovem, com suas possíveis futilidades consumistas.

Mesmo porque, apesar de ele ter uma cara de ``mano'', apesar de um jeito
típico de periferia \versal{ZL} de Sampa, ele parece ter ainda a libido de um
jovem de dezenove anos.

Mas ele nega qualquer desejo ou atividade sexual. Nega até a punheta que
todos os meninos do planeta batem.

Ele insinua secamente o primado da pedra sobre o sexo.

A pedra fala mais alto. As meninas, por mais ``gostosas'', não o
interessam. E~eu não consigo indagar sobre uma possível
homossexualidade, que talvez não seja o caso porque todo ele é uma
negação absoluta da sexualidade. Nem homo nem hétero!

Aos poucos eu o percebo melhor através de seus sucessivos vazios. Ele
não é somente uma negação da sexualidade. Ele é uma negação da
possibilidade do amor e da vida de relação. Pois admite que nunca gostou
de ninguém. Tampouco da avó que o criara e que insiste no seu
tratamento.

\asterisc{}

De repente baixa na sala mais um silêncio opressivo. Horrendo.

Eu olho para o rapaz como se eu fosse o personagem de Edgar Allan Poe
olhando para aquele corvo de seu famoso poema.

Súbito vejo à minha frente uma imagem humana quase imóvel, de olhos
mortiços e em movimento sutil ou mínimo.

Sinto"-me congelado. Porque essa figura olha para mim. Ela me encara.

Minha sensibilidade aguçada antecipa aquela frase dele. Aquela mesma
frase que eu pressinto.

Pois ela virá agora como um recado para que depois nada mais seja
conversado neste encontro.

Sim, aquela mesma frase acaba escorrendo dos lábios do rapaz, e mesmo
num tom baixo, acaba fazendo um barulho preciso e contundente como o de
uma tampa de madeira ao se fechar.

Ele a anuncia mais uma vez, e agora --- parece"-me --- com sutil e
mórbido deboche:

Ele já a desenha no vácuo do olhar vazio. Feito um eco.

— É a brisa da morte.

Eu encerro a entrevista e saio da sala como se mal tivesse despertado de
um pesadelo e com a sensação de que alguma alma não se levantará ---
nunca mais.

\begin{center}\asterisc{}\end{center}
%\begingroup\small

\emph{O~título inicial teria sido ``A brisa da morte'', precisamente aquela
pancada forte na alma que ouvi lá do fundo dos abismos e das sombras. O}~link \emph{com} O Corvo\emph{, de Edgar Allan Poe, veio depois quando da feitura do
conto"-crônica, e a imagem do corvo sinistro e misterioso me pareceu a
melhor síntese literária do que a adicção profunda alcança em termos da
radicalidade de uma embriaguez.}

\emph{Digo embriaguez mais do que química e intoxicante, no mínimo
simbólica e além do simples efeito de qualquer droga. Porque droga, para
o adicto grave, é apenas meio, droga é apenas passaporte para outros
domínios imaginados loucamente. Não é um fim.}

\emph{Mas o que resulta da relativa insanidade do caso deste rapaz é um
entregar"-se à máxima compulsão, um entregar"-se àquilo que pode levar à
depressão severa, ou talvez a um transtorno de personalidade, à miséria
social, e a muito mais do que o consumismo ilícito e banal.}

\emph{Eu me lembro que o rapaz era, a bem dizer, um quase menino. O~atendimento dele começou difícil, emperrado, em meio a um silêncio
opressivo. Tanto assim que eu achei que esse atendimento, a princípio,
morreria na rotina.}

\emph{Verifiquei, todavia, que o atendimento não era nada rotina. Nem
ele era típico dentre os usuários pesados de crack, se bem que se
aproximasse um pouco do estereótipo do ``noia'' meio zumbi com seu olhar
de hipnotizado.}

\emph{Ele até me pareceu hostil, mas não era de verdade, e sua presença
despertou possivelmente algum preconceito meu. Entretanto, de repente eu
descobri sua inocência algo insana no fato dele não ser nada agressivo.
Ele era na verdade um sonhador romântico doente, um tipo possuído, como
outros mundo afora, da ânsia ingênua pela ``brisa'' infinita.}

\emph{A~expressão dele ``brisa da morte'', se foi macabra, foi síntese
existencial. Foi expressão que depois, conforme minha imaginação, me
levou a uma lembrança de um horror gótico degradado e ao título ``Nunca
Mais''.}

\emph{De resto, não confirmo aqui os diagnósticos desse indivíduo com
precisão. Acredito que seja o caso de se falar cautelosamente em
síndrome depressiva, pois o rapaz não tinha mais energia, não tinha mais
libido.}

\emph{Dele se poderia falar que encarnava um instinto de morte, algo
``tanático''. No entanto, um simples rótulo de depressão não diz muito
porque o rapaz, na sua radicalidade mórbida, era coerente quando
investido de uma energia das sombras.}

\emph{Na verdade, ele se colocava além da depressão como transtorno ou
doença, além da depressão tal como é descrita pela psiquiatria.}

\emph{Havia nele um estado de rebaixamento radical do humor, quase uma
abulia (ausência de vontade), ou uma anedonia (ausência de prazer), mas
junto a uma sofisticação mórbida e até melancólica ao extremo, revelada
naquele enamorar"-se da morte levado a sério como fantasia de um futuro
ausente fincado na presenticidade da compulsão pela pedra.}

\emph{Eis uma situação que exige mais do que apenas diagnósticos com \versal{CID}
tal ou \versal{DSM} tal. Pois se o caso deste rapaz ficasse apenas limitado a uma
tentativa de diagnosticar e medicar eu teria perdido a chance de ter
alguma empatia, eu teria dispensado a ``viagem'' breve em uma morbidez
radical alheia.}

\emph{Portanto, melhor ``viajar'' junto a ele, melhor esticar um difícil
e tormentoso diálogo e ainda especular, como faz Hamlet, sobre os
limites da vida e depois recorrer à literatura, do que recorrer apenas à
medicina ou à psiquiatria!}
%\endgroup