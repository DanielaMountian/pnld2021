\chapterspecial{Sem choro nem vela}{}{}
 

Atendi, pela segunda vez, um homem dependente de crack que não parecia
ter este ``vício''. Ele tinha uma aparência normal. Não era ``detonado''
como dizem por aí ficarem essas pessoas. Tinha até um jeito conservador,
sério, e nele havia a estranheza de um olhar inquieto e uma insistência
em abordar assuntos delicados.

Quando o vi pela primeira vez eu o julguei alcoolista devido a esse
jeito conservador. A~dependência de crack foi novidade. E~bem menos
novidade foi sua rigidez, sua obsessão e postura excessivamente
moralista em querer ser algoz de si mesmo.

Começou a insinuar insistentemente que estava demonizado e concordava
com muita gente que apoia a demonização do crack. Queria até exorcizar a
si próprio para espantar o que reconhecia como uma maldição.

\asterisc{}

Numa próxima consulta ele chegou mais à vontade. Conversamos a respeito
do que se entende por aí como ``vício'' em drogas. Eu lhe disse que o
termo ``vício'' não se emprega mais. Dei a ele algumas explicações
científicas sobre o assunto e insisti que, no combate à adicção, a pior
coisa é fazer da culpa um tratamento. Mesmo porque, trata-se de uma
doença e não de um desvio moral.

Foi quando ele me confessou ser um homem bastante crente. Disse que sua
família toda pertence a uma Congregação Evangélica, e ele também, de
alguma maneira, faz parte da seita, embora estivesse marginalizado por
causa do maldito ``vício'' no crack.

Naquela altura dos acontecimentos percebi no homem um estranho e
hesitante incômodo, quando foi se esboçando nele um constrangimento. De
repente, com o semblante carregado, ele anunciou que ia fazer uma
pergunta bem especial. Pensei tratar-se de mais uma dúvida cruel a
respeito de drogas. Era outra coisa.

Ele revelou que, recentemente, sua mãe morrera e ele, tanto no momento
em que recebeu a notícia quanto no velório, como depois, não derramou
uma lágrima.

Queria saber se era normal, ou se ele teria algum problema por não ter
derramado uma simples lágrima pela morte da mãe.

Eu procurei contornar a situação com cautelosa neutralidade. Afirmei que
lágrimas não são um real testemunho, ou evidência, do que se passa na
mente humana. Lágrimas são exteriorizações, são visibilidades
fisiológicas. Podem ter várias origens, e podem enganar. Não existem,
por acaso, as ``lágrimas de crocodilo''?

Então me lembrei do personagem Mersault, do romance ``O estrangeiro'',
do Camus. Mersault é um homem descrente que não chora durante o velório
e o enterro da mãe, comete pouco tempo depois um homicídio e é na
verdade condenado não pelo homicídio mas pela frieza diante da morte da
mãe.

\asterisc{}

De repente o comentário a respeito do livro do Camus gerou uma situação
curiosa. Ao ter acabado de fazer esse comentário, tomei consciência de
que, sem responder à pergunta incômoda do homem, eu havia exposto a ele
o enredo de uma obra da qual ele nunca ouvira falar.

Mas ele se tornou insistentemente curioso e interessado no assunto.
Anotou o título e o autor da obra.

A princípio eu pensei que este homem crente ficasse apenas chocado com o
enredo do romance de Albert Camus, e não entendesse a postura descrente
e cínica de Mersault, ainda mais pelo fato deste homem crente querer se
proteger contra as sombras demoníacas do crack por meio de um rígido
moralismo de seita cristã apocalíptica. Mesmo que ele se considerasse
temporariamente marginalizado da seita, ainda buscava ser aceito.

O homem, todavia, não ficou nada chocado com o destino trágico de
Mersault. Muito pelo contrário, simpatizou muito com o personagem
principal de ``O estrangeiro''.

\asterisc{}

Numa próxima consulta ele voltou a falar sobre ``O estrangeiro''. E~não
apenas havia comprado como também me mostrou o livro, e disse que o
começava a ler com interesse crescente.

Olhei bem para ele, que tinha o livro nas mãos. De uma maneira estranha,
ele me pareceu naquele momento uma espécie de ator. Ou então --- refleti
melhor --- não seria exatamente esse o caso, e ele me pareceu não
precisamente um ator, mas um homem deslocado de seu mundo e, ao mesmo
tempo, um personagem construído pela família e pela igreja.

Porém tudo ali indicava que ele fazia questão de vender uma imagem. Uma
imagem de que ele fosse um moralista peculiar ou, quem sabe, fosse até
um pastor maldito, um pastor fora dos eixos; ou então, ele parecia se
colocar como uma espécie de ``cruzado'' no combate às drogas. Porque
vinha com discursos pomposos e indignados contra a ``maldição'' das
drogas.

\asterisc{}

Como bem verifiquei nas consultas seguintes, ao contrário de Mersault,
este homem busca se proteger sob um manto religioso de indiscutíveis
crenças arraigadas e algo obsessivas, enquanto Mersault, todo cético,
busca uma questionável liberdade sob um manto de convicto ateísmo.

Se os dois parecem antagônicos, eles seriam, lá no fundo, parecidos!

Após o velório da mãe, Mersault foi ver uma comédia no cinema com uma
nova namorada. Este homem --- cínico, indiferente e distante de sua
igreja --- foi usar crack em algum espaço de repudiado lazer.

Ambos --- estrangeiros de si mesmos, divertiram-se --- e o fizeram sem
choro nem vela.

\begin{center}\asterisc{}\end{center}
\begingroup\small

O estrangeiro\emph{, de Camus, é, para mim, uma obra prima da literatura
universal. Nem tanto pelo seu conteúdo friamente existencialista, sua
secura, seu amargor e sua contundência cruel, mas porque seu
protagonista Mersault nos mostra que o destino injusto do personagem
aponta para um destino de muitas outras pessoas que não são em geral
``flor que se cheira'', porém são profundamente humanas e humanamente
sombrias, e são vítimas de um círculo infernal do qual elas também
escolheram participar.}

\emph{Eis a grande mensagem de} O estrangeiro, \emph{e que fez a ponte
necessária para esta crônica, quando um indivíduo estranho dependente de
crack desejava porcamente conciliar droga pesada com uma vida careta
voltada para um culto evangélico.}

\emph{Ele não tinha nada de ``noia''. Era até preconceituoso contra os
``noias'' e ainda me despejou sua frieza para com a mãe (frieza que
talvez tenha, sabe-se lá, sua razão).}

\emph{Ele mostrou seu vazio existencial e eu, logo depois, fiz uma ponte
entre este vazio e o romance de Camus. Notei como ele procurou
colocar-se em sintonia com a obra quando eu resumi o enredo de \emph{O
Estrangeiro}, e ele insistiu em falar sobre o assunto.}

\emph{Mais tarde, bem a propósito, me veio à mente a frase de Noel Rosa
sem choro nem vela. Eis então --- pensei --- a presença de uma cacetada
da vida: pois aquilo que despenca na existência sem dó nem piedade só
pode ser ``sem choro nem vela''.}

\emph{Quanto ao transtorno, eu, de início, dizia não haver evidências
além de uma dependência pura e simples de droga pesada. Todavia,
surgiram algumas novidades, e eu acabei colhendo dados e evidências
desta pessoa que me permitiram aprofundar um caminho diagnóstico.}

\emph{Cogitei a hipótese de um transtorno de personalidade do tipo
anancástico. Mas, o que é isso? Que termo estranho é esse? Esse
transtorno é a forma sócio-sintônica, ou seja, aceita como ``normal'',
daquilo que, numa manifestação dissintônica, seria considerado
transtorno obsessivo compulsivo e seria motivo de queixa do paciente.}

\emph{É precisamente o caso dele: indivíduo extremamente metódico,
engessado numa aceitação crônica de rituais que acabam fazendo parte de
sua vida e de seu modo de ser. Não seria de estranhar, portanto, que ele
seja um tipo tão adicto quanto religioso e atolado no visgo do que
considera pecado.}

\emph{Não seria de se estranhar que ele seja um tipo metido numa
condição dupla em que, por um lado, existe uma vigilância austera de uma
igreja evangélica e, por outro lado, existe a rotina proibida e profana
do abuso de droga, e ainda mais nos descaminhos do crack.}

\emph{Além dessa possível condição anancástica, restam as mazelas da
criação, resta o território familiar das neuroses, restam os cacos
edípicos mal resolvidos.}

\emph{Faço questão de observar que esta crônica, pela sua referência a \emph{O
Estrangeiro}, chama atenção para a necessidade dos tais adictos não se
enquadrarem demais em diagnósticos pré-moldados.}

\emph{É melhor abrir o leque e olhá-los de um prisma fenomenológico e
existencial. E~perceba você, leitor, o quanto a literatura cumpre sua
função até diagnóstica de alargar ressignificações ao fazer pontes da
arte para com a vida: digo vida como ela é ou parece ser, e também
porque o ``parecer ser'' é uma forma de certas pessoas ressignificarem
suas atormentadas existências para si mesmas.}

\emph{A~universalidade de} O estrangeiro \emph{dá as mãos à particularidade
deste caso incômodo e contraditório. Eis aí um dilema nebuloso que pode
seguir no embalo de rituais suspeitos, por vezes no embalo de caminhos
obsessivos de onde emergem passeios por cracolândias, ou por onde
despontam dilemas religiosos e mais uma ânsia de ser de novo integrado
numa igreja. Tudo junto, em resumo, é uma procura obsessiva de
significados e é sem choro nem vela.}
\endgroup