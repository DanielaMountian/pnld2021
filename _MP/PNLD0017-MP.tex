\documentclass{article}
\usepackage{manualdoprofessor}
\usepackage{fichatecnica}
\usepackage{lipsum,media9,graficos}
\usepackage[justification=raggedright]{caption}
\usepackage{bncc}
\usepackage[arlequino]{logoedlab}

 

\begin{document}


\newcommand{\AutorLivro}{Robert Louis Stevenson}
\newcommand{\TituloLivro}{O médico e o monstro}
\newcommand{\Tema}{Ficção, mistério e fantasia}
\newcommand{\Genero}{Romance}
% \newcommand{\imagemCapa}{PNLD0017-01.png}
\newcommand{\issnppub}{---}
\newcommand{\issnepub}{---}
% \newcommand{\fichacatalografica}{PNLD0017-00.png}
\newcommand{\colaborador}{\textbf{Fulano de Tal} é uma pessoa incrível e vai fazer um bom serviço.}


\title{\TituloLivro}
\author{\AutorLivro}
\def\authornotes{\colaborador}

\date{}
\maketitle
\tableofcontents

\pagebreak

\section{Carta aos professores}

Caro educador / Cara educadora,\\\bigskip

\begin{flushright}
{\footnotesize
O lado mau da minha natureza {[}...{]} era menos robusto e menos
desenvolvido do que o lado bom {[}...{]}. Do mesmo modo, no transcorrer
de minha vida, que tinha sido, apesar de tudo, em noventa por cento uma
vida de esforço, virtude e autocontrole, ele tinha sido muito menos
exercitado e menos desgastado. Daí, creio eu, o fato de que Edward Hyde
era bastante menor, mais leve e mais jovem do que Henry Jekyll. E assim
como o bem reluzia na fisionomia de um, o mal estava escrito de modo
claro e inequívoco no rosto do outro.\\
\textsc{robert louis stevenson}, \emph{O médico e o monstro}
}
\end{flushright}

A leitura de um clássico da literatura sempre nos traz muito mais do que
um enredo romanesco. A literatura mobiliza todo um universo de formas de
cultura: linguagens, comportamentos, conflitos, maneiras de fazer as
coisas necessárias à existência e de compreender essa existência.

Quando lemos \emph{Vidas secas}, de Graciliano Ramos, aprendemos sobre o
Nordeste brasileiro, sobre a pobreza, sobre a seca, a injustiça social e
a imaginação da gente do sertão.

O romance \emph{O médico e o monstro}, de Robert Louis Stevenson, se
baseia em uma fantasia, mas essa fantasia é fortemente reveladora dos
modos de pensar e agir de certa parcela da sociedade do século XIX na
Europa.

As especulações que Stevenson desenvolve se originam amplamente nos
desenvolvimentos científicos daquele século, particularmente da Medicina
e da Psiquiatria, que estava sendo formulada. Nesse sentido, podemos
associar o tema do romance de Stevenson a outras narrativas, como o
romance \emph{Frankenstein}, de Mary Shelley, e os contos \emph{A causa
secreta} e \emph{O alienista}, de Machado de Assis, que é uma sátira do
cientificismo positivista aplicado ao comportamento humano.

O livro de Stevenson, entretanto, não se resume a uma especulação
inspirada na ciência. Sua principal qualidade como narrativa, isto é,
como obra de arte verbal, é questionar a moral que norteia o
comportamento de cada um em uma sociedade. Apesar de tentarmos ser
corretos, honestos, generosos, muitas vezes somos errados, corruptos,
invejosos. Por quê?

A resposta, que fica aberta para cada leitor e leitora, está na
narrativa e aponta para a complexidade de nossas emoções, desejos e
experiências.


\section{Atividades 1}

%\BNCC{EM13LP26}
\subsection{Pré-leitura I}

\paragraph{Tema} Pesquisando sobre gênero literário e contexto histórico.

 %(EM13LGG601; EM13LP01;EM13LP10; EM13LP30)

\paragraph{Conteúdo} Especulação sobre gênero literário.
Como primeira atividade pré-leitura, sugere-se propor uma ação em que
cada aluno e aluna seja convidado a responder, em um pedaço de papel
avulso, à pergunta: ``O que é romance?''

\paragraph{Objetivos}
Os objetivos dessa atividade são promover a interação, a reflexão
crítica, a redação baseada em argumentos, a mobilização dos repertórios
de linguagem e de conhecimentos prévios, a avaliação coletiva das fontes
dos conhecimentos, o reconhecimento da interação entre as fontes
tradicionais e orais e as fontes acadêmicas e escritas dos
conhecimentos, um exercício de refinamento conceitual, pressuposto do
pensamento científico.

\paragraph{Justificativa}
Uma das tarefas do Ensino Médio é aprofundar a complexidade dos
conhecimentos consolidados ao longo do Ensino Fundamental. Quando
convidamos nossos alunos e alunas a demonstrar aquilo que, sem dúvida,
conhecem, muitas vezes ocorrem dificuldades que vem da falta de
linguagem adequada à expressão rigorosa de conceitos científicos. Outras
vezes, ocorre observarmos em atividades como essa que os estudantes
percebem que não conhecem de fato aquilo que imaginavam conhecer e tal
percepção vem do fato de não poderem explicar esse conhecimento a outras
pessoas.

Aqui, seria oportuno comentar a importância do conceito como uma
``ferramenta da reflexão'' e, nesta mesma linha de desenvolvimento, a
reflexão como uma operação que nós podemos e devemos aperfeiçoar. Para
muitas pessoas na fase do Ensino Médio, constitui uma surpresa descobrir
que se pode aprender a pensar e que pensamos melhor e mais refinadamente
quanto mais exercitamos essa habilidade.

Vale utilizar este tema para desencorajar a adoção de dogmas e estigmas
associados à falta de inteligência, que são, infelizmente, muito comuns
no ambiente escolar e devem ser combatidos. É importante valorizar as
diferentes formas de inteligência, não só a inteligência lógica: as
inteligências social, corporal e moral, dentre outras.


\paragraph{Metodologia}
A resposta à pergunta proposta deve ter entre duas e dez linhas e devem ser reservados cinco a dez minutos para redigi-la.

Em seguida, um grupo de alunos e alunas deve recolher as respostas e,
com fita adesiva, fixá-las em uma parede ou quadro (quinze a vinte
minutos).

Após ler as definições, a turma toda pode ser convidada a participar de
uma atividade de agrupamento de definições semelhantes (cerca de quinze
minutos). Quantos significados diferentes foram dados?

O resultado desta atividade é um levantamento dos significados
associados à palavra ``romance'' naquele agrupamento.

\paragraph{Tempo estimado} Uma aula de 50 minutos.

\subsection{Pré-leitura II}


\paragraph{Tema} Definição de romance e dicionário como ferramenta.

%(EM13LGG401; EM13LGG402; EM13LGG704; EM13LP01; EM13LP10; EM13LP30;EM13LP31)

\paragraph{Conteúdo} Aprofundar a compreensão do aluno do gênero romance e estimular a pesquisa no dicionário, mostrando todas as potencialidades dessa ferramenta de pesquisa.


\paragraph{Objetivos}
O resultado desta atividade, do ponto de vista da aprendizagem, diz
respeito à:

\begin{enumerate} 
\item
Aquisição dos conceitos de ``gênero literário'',
``romance'', ``narrativa'', ``ficção'' e ``prosa'', representando um
vocabulário crítico básico de análise literária; 

\item
O recurso ao dicionário como fonte de aperfeiçoamento da leitura e da escrita;

\item
A demonstração das diferentes fontes do saber: as fontes tradicionais, de
tradição e transmissão orais, e as fontes eruditas, acadêmicas e
científicas, de tradição e transmissão escritas. (Lembrar que, na
escola, a principal fonte de transmissão é oral, isto é, fala-se e
ouve-se muito mais do que se escreve e se lê durante a escolarização).
\end{enumerate}

\paragraph{Justificativa}

O trato com conceitos é fundamental como base para a compreensão das
várias complexidades apresentadas por professores e professoras
especialistas ao longo do Ensino Médio. O uso adequado de ferramentas
adequadas é um pressuposto para a boa realização de qualquer trabalho. O
pensamento conceitual é nosso trabalho e nossas ferramentas são os
conceitos. É fundamental, portanto, apresentar o pensamento conceitual
para que alunos e alunas possam lidar com os diferentes conjuntos de
conceitos que orientam as disciplinas: dos esportes e artes até a
biologia e a matemática, pensamos com conceitos.

\paragraph{Metodologia}
A segunda atividade pré-leitura deve iniciar-se com o pedido de que
todos os alunos e alunas consultem o verbete ``romance'' em um
dicionário de uso escolar. Sem excluir a possibilidade de uso de
dicionários eletrônicos, o ideal nessa atividade é usar dicionários de
papel, para mostrar que a pesquisa em papel encontra necessariamente
outras palavras na página, enquanto em verbetes de dicionários
eletrônicos isso não acontece. (Quinze minutos)

Em seguida, oralmente, deve-se estimular os alunos e alunas a comparar
as definições produzidas na Atividade 1 com as encontradas no dicionário
e refletir sobre as relações entre as duas fontes da linguagem, a oral e
a escrita.

A ideia central é comparar as definições feitas pela turma, mostrando
que se trata de repertório válido e relevante, com as definições
encontradas no dicionário a fim de refletir sobre as diferentes relações
de determinação entre o registro coloquial ou falado e o registro
escrito. (Quinze minutos)

Nesse momento, deve ser apresentado o conceito de gênero literário e o
significado de romance: narrativa de ficção em prosa. É importante
discutir por que usamos conceitos e mostrar que o conceito nem sempre é
autoexplicativo. (Quinze minutos)

Sabendo que o gênero romance pode ser definido como ``narrativa de
ficção em prosa'', restaria ainda conferir o sentido de cada um dos
termos da definição (essa parte deve ser adaptada à definição que cada
turma produzir).

Assim, pode-se propor uma segunda atividade de consulta ao dicionário,
ou pesquisa lexical, para as palavras ``narrativa, ``ficção'' e
``prosa''. Nessa etapa, a atividade pode ser individual e silenciosa e o
professor ou professora pode conclui-la com a simples oferta do
esclarecimento de dúvidas que porventura tenham surgido das consultas.
(Quinze minutos)

Cada aluno e aluna deve fazer um glossário com esses cinco conceitos no
caderno (gênero literário, romance, narrativa, ficção e prosa). Para
tanto, o professor ou professora deve explicar o que é um glossário,
quais os seus tipos e quais os seus usos.

\paragraph{Tempo estimado} Uma aula de 60 minutos.

\subsection{Pré-leitura III}

\paragraph{Tema} Desvendando um verbete 

%(EM13LGG104; EM13LGG401; EM13LGG704; EM13LP01; EM13LP10; EM13LP30; EM13LP31)

\paragraph{Conteúdo} Estimular o uso do dicionário como ferramenta de pesquisa, apresentar as características de um verbete de dicionário e explorar a escrita objetiva, impessoal, concisa e precisa.


\paragraph{Objetivos}
Os resultados pretendidos com essa atividade são: 1. a instrumentação
acadêmica de alunos e alunas; 2. o aprofundamento dos estudos sobre a
linguagem e seus vários usos sociais; 3. a familiaridade com o
vocabulário da gramática, da linguística e da lexicografia.

\paragraph{Justificativa}
A disponibilidade dos meios de consulta eletrônicos levou muitos alunos
e alunas, e pessoas em geral, a deixarem de usar os dicionários, cuja
consulta é considerada muito trabalhosa quando comparada a uma consulta
à Internet. Escolarmente, perde-se a possibilidade de apresentar,
valorizar e usar uma tecnologia intelectual aperfeiçoada ao longo de
séculos de estudos linguísticos e humanísticos.

\paragraph{Metodologia}
Para iniciar a terceira atividade, sugere-se que o professor ou
professora, oralmente:

\begin{enumerate}
\item
Valorize o uso do dicionário como instrumento de aperfeiçoamento da
escrita e da leitura e a profissão do lexicógrafo e da lexicógrafa;

\item
Explique o que é um verbete enquanto exibe um para a classe; e

\item
Apresente analiticamente um verbete e todas as suas partes técnicas:
classificação gramatical, divisão silábica, pronúncia, campos semânticos
(por exemplo, medicina, engenharia, história), registros (por exemplo,
popular, familiar, arcaico, regional, gíria etc.) e acepções,
evidenciando que o dicionário traz muitas informações além do
significado. (Cerca de trinta minutos)
\end{enumerate}

Em seguida, coletivamente o professor ou professora pode construir uma
lousa com dez a quinze palavras selecionadas por serem de uso comum e
cotidiano, por exemplo, mochila, óculos, copo, dormir, passear, estudar
etc. (Cerca de cinco minutos)

A seguir, deve-se pedir que cada aluno e aluna escolha uma palavra da
lousa e escreva a definição que imagina que encontrará no dicionário. A
conclusão da atividade consiste em buscar a palavra definida, copiar no
caderno a primeira acepção do verbete e comparar as definições,
observado as diferenças encontradas. (Cerca de quinze minutos)

Acrescentar oralmente que a redação do verbete obedece a alguns
critérios que são comuns ao pensamento e à redação científicos:
objetividade, impessoalidade, concisão, precisão.

\paragraph{Tempo estimado} Uma aula de 60 minutos.

\subsection{Leitura I}

\paragraph{Tema} Roda de leitura

%(EM13LGG601; EM13LGG602)

\paragraph{Conteúdo}
Para essa atividade, seria solicitado que todos e todas trouxessem o
livro \emph{O médico e o monstro}. Se houver possibilidade, o ideal é
dispor as cadeiras em roda. A proposta é fazer uma leitura em voz alta
coletiva e encadeada.

\paragraph{Objetivos}
\begin{enumerate}
\item
Evidenciar a leitura em voz alta como recurso que favorece a
compreensão leitora, a pronúncia correta e ressalta as qualidades
expressivas e dramáticas do texto literário;

\item
Sugerir que a vocalização é uma forma de incorporação e apropriação do texto
literário;

\item
Ilustrar a riqueza de cada vocalização, baseando-se na
particularidade de cada voz - com timbres, alturas, tons e ritmos
diferentes de fala -, que empresta suas características ao texto;

\item
Valorizar a dimensão oral e corporificada da leitura literária.
\end{enumerate}

\paragraph{Justificativa}
A oralidade como meio de expressão, embora seja o mais ostensivamente
usado na escola, deve ter seu valor reiterado, ao lado dos conhecimentos
letrados, adquiridos e transmitidos por escrito. O poder de emocionar
que um texto literário tem é potencializado pela leitura em voz alta.
Lendo em voz alta, aprendemos também a utilizar a pontuação e aprendemos
que a pontuação rege a leitura em voz alta. Aprendemos também a entoar
uma frase, acentuando sua carga expressiva de indiferença, surpresa,
alegria, tristeza, dúvida etc.

\paragraph{Metodologia}
Para dar uma ideia do que seja uma leitura expressiva, isto é, que
enfatize as pausas, os ritmos e a acentuação, o professor ou professora
pode ler um trecho curto de \emph{O médico e o monstro} e pedir que
alguns alunos e alunas o repitam com a mesma entonação.

Pode-se usar o seguinte trecho, situado no início do capítulo 2, ``Em
busca de Mr. Hyde'':

``Até aquele momento, sua indignação tinha sido alimentada pela sua
absoluta ignorância sobre quem pudesse ser o tal Mr. Hyde; e agora, numa
reviravolta súbita, isto lhe era revelado. Tudo aquilo já tinha um mau
aspecto quando ``Hyde'' era apenas um nome do qual ele nada mais sabia.
O que pensar agora, quando tal nome começava a ser revestido dos
atributos mais detestáveis? Por entre a névoa insubstancial que até
aquele momento tinha desfocado sua visão, começou a se delinear a súbita
e nítida presença de um inimigo.''\footnote{STEVENSON, Robert Louis.
  \emph{O estranho caso do Dr. Jekyll e Mr. Hyde}, trad. Bráulio
  Tavares. São Paulo: Hedra, 2011, p. 26.}
(Cerca de quinze minutos)

Em seguida, o professor ou professora pode ouvir com a sala a leitura
dos cinco minutos iniciais do conto ``O gato preto'', de Edgar Allan
Poe, no Youtube (https://www.youtube.com/watch?v=YHyRLgfF2as). Podemos
considerar essa uma boa leitura? A leitura poderia ser mais expressiva?
(Cerca de dez minutos)

Agora, com todos e todas sentados em círculo, a ideia é fazer uma
leitura encadeada. Cada aluno e aluna lê alguns minutos, e o professor
ou professora indica o momento de passar a palavra ao próximo
participante.
(Vinte a trinta minutos)

\paragraph{Tempo estimado} Uma aula de 60 minutos.

\subsection{Leitura II}

\paragraph{Tema} Trabalhando com diálogos e dramatização.

%(EM13LGG203;EM13LGG601; EM13LGG602; EM13LGG603; EM13LP01; EM13LP02; EM13LP16)

\paragraph{Conteúdo} Aprofundar as outras dimensões de um romance que não apenas a narrativa, estimulando os alunos a dramatizarem o texto e trocarem suas experiências de leitura.

\paragraph{Objetivos}
Os objetivos da atividade são:
\begin{enumerate}
\item
Diversificar as formas de abordagem
pedagógica dos textos literários, valorizando seus aspectos dramáticos e
evidenciando seus diferentes recursos: narração, descrição, diálogos,
monólogos etc.; 

\item
Estimular habilidades transversais e de convivência,
como a desinibição, o trabalho em equipe, a interpretação cênica, o
entusiasmo;

\item
Apresentar possibilidades de transposição intermidiática
de textos literários, como podcasts, radionovelas, encenações teatrais,
leituras dramáticas.
\end{enumerate}

\paragraph{Justificativa}
A leitura dramática é uma ferramenta capaz de mobilizar alunos e alunas
por exigir uma presença mais integral dos participantes na performance
(gestualidade, voz, memória, atenção espacial e auditiva). A
apresentação pode envolver vários atores e contar com a participação
ativa do restante da turma, fortalecendo a construção das relações
coletivas e do sentido de grupo. A atuação cênica, que pode envolver
dança, canto e improvisação, mobiliza habilidades e competências
extracurriculares capazes de disparar novas teias de relações sociais,
afetivas e artísticas, além de ações coletivas na comunidade escolar.

\paragraph{Metodologia}
Para esta atividade, a turma precisará fazer uma seleção de trechos da
narrativa em que ocorram predominantemente diálogos e em que haja muitas
trocas de turnos, isto é, em que a interlocução seja intensa. (Cerca de
25 minutos)

O capítulo 7, ``O incidente da janela'', pode ser usado por ser
basicamente dialogado e ter apenas duas páginas. O capítulo narra uma
caminhada de Mr. Utterson e Mr. Enfield, como ocorre no capítulo 1.
Quando eles passam pela rua de trás da casa do Dr. Jekyll, ao comentarem
o estranho comportamento do médico, o avistam na janela e conversam
brevemente. O Dr. Jekyll diz que não está bem e que não durará muito.
Mr. Utterson o convida, então, para vir caminhar com eles, a fim de
reestabelecer sua disposição. Enquanto recusa o convite, o Dr. Jekyll
aparentemente é atacado por uma dor severa, que faz com que sua
fisionomia se transtorne, e a janela se fecha violentamente. Os dois
amigos seguem aterrorizados por aquela visão.

Uma vez separados os trechos dramáticos, alunos e alunas, em pequenos
grupos, escolherão os atores e farão um breve ensaio treinando as falas
dos personagens. (25 minutos)

A etapa final é a apresentação dos mesmos trechos encenados por grupos
diferentes, com uma votação ao final das exibições das propostas para o
mesmo trecho, para eleger a mais convincente interpretação. (Depende do
tamanho da turma, da composição dos grupos e do dinamismo, mas estima-se
de uma a duas horas)

\paragraph{Tempo estimado} Quatro aulas de 50 minutos.

\subsection{Pós-leitura} 

\paragraph{Tema} Enciclopédia de literatura: verbete \emph{O médico e o
monstro}.

%EM13LGG104; EM13LGG401; EM13LGG601; EM13LGG602; EM13LGG704 EM13LP01; EM13LP11; EM13LP12; EM13LP30

\paragraph{Conteúdo} Aproximar o aluno do uso da enciclopédia e do princípio da universalização do saber.

\paragraph{Objetivos}
Nessa atividade, pretende-se:
\begin{enumerate} 
\item
Ativar procedimentos de pesquisa e
valorizar a construção coletiva do conhecimento; 

\item
Propiciar uma atividade de redação coletiva e de uso de padrões científicos, bem como a reflexão e a discussão em busca do consenso em cada grupo. Ao longo de
suas etapas, o objeto literário situou-se no centro da atenção coletiva,
mostrando que os usos sociais da literatura e do conhecimento dependem
da atuação de uma ampla coletividade de indivíduos.
\end{enumerate}

\paragraph{Justificativa}
Uma das tarefas do Ensino Médio é desenvolver a metacognição, isto é, o
conhecimento sobre como o conhecimento é construído. Para isso, é
preciso aprender a aprender, aprender como se estuda ou estudar como se
aprende. Essa sequência, baseada no uso da enciclopédia, pretende
instrumentalizar alunos e alunas, através da apresentação de uma
ferramenta acadêmica, para a metacognição. Ao mesmo tempo, evidenciar a
construção coletiva, histórica e provisória do conhecimento através dos
consensos sociais.

\paragraph{Metodologia}
Após lembrar alunos e alunas da Atividade 1, caso tiver sido realizada,
que estimula o uso do dicionário como ferramenta de estudos, sugerimos
que o professor ou professora apresente uma enciclopédia: o que é, quais
são os seus tipos, quais os seus usos, que enciclopédias foram famosas,
quais as enciclopédias usadas atualmente. Seria muito bom se houvesse
uma enciclopédia para ser consultada na sala ou na biblioteca. (Cerca de
trinta minutos)

Sugerimos que os alunos e alunas leiam o verbete ``enciclopédia'' no
site Wikipedia, de fácil acesso. Nessa leitura, deve-se observar a
estrutura do verbete e as partes em que ele se divide:
\begin{enumerate}
\item
Etimologia;
\item
Características;
\item
História;
\item
Algumas enciclopédias famosas;
\item 
Ver também;
\item
Referências;
\item
Ligações externas. (Cerca de vinte minutos)
\end{enumerate}

A última etapa desta Atividade 6 é redigir, em pequenos grupos, um
verbete de enciclopédia para o livro \emph{O médico e o monstro}. Para
isso, alunos e alunas podem pesquisar verbetes dedicados a outros livros
a fim de conferir sua organização e divisões. O professor ou professora
deve lembrar que ninguém deve ler o verbete do próprio livro antes de
redigir o do grupo.

Importante estabelecer uma extensão mínima e máxima de linhas, páginas
ou palavras, para não resultar em verbetes muito curtos ou muito longos
e obter homogeneidade. (cerca de 25 minutos)

\paragraph{Tempo estimado} Duas aulas de 50 minutos.

\section{Atividades 2}

\subsection{Pré-leitura}

\paragraph{Tema} O século XIX e o desenvolvimento científico

%(EM13LGG303; EM13CHS1020; EM13CHS401; EM13CHS504; EM13LGG602; EM13LP30)

\paragraph{Conteúdo}
Foi com muita controvérsia que a Ciência consolidou seus avanços ao
longo do século XIX, porque muitas questões morais se impunham em seu
caminho: em muitos casos, a Ciência parecia contrariar a moral e a
religião.

Um exemplo seria o uso de corpos humanos em estudos de Anatomia no
século XIX. Crenças religiosas, que consideram o corpo um objeto
sagrado, logo se manifestaram contrariamente a práticas que violariam o
corpo humano.

Ainda no ano de 2021, podemos encontrar debates semelhantes na
sociedade: deve-se permitir a interrupção voluntária da vida em casos de
doenças terminais, conhecida como eutanásia? Deve-se considerar o aborto
um direito de toda mulher que não deseje ou não possa tornar-se mãe ou
deve-se considerar um atentado contra a vida? Durante a Pandemia de
Covid 19 nos anos de 2020 e 2021, pudemos observar a sociedade
mobilizada em torno de questões que associam crenças morais e religiosas
à resistência à vacinação.

Para que as discussões éticas contemporâneas ganhem historicidade,
propomos que o professor ou professora de História faça uma aula sobre a
Revolução Industrial, na qual possa dar um panorama dos avanços
científicos do século XIX. 


\paragraph{Objetivos}
Os objetivos dessa atividade são:

\begin{enumerate}
\item
Inserir a narrativa literária em
seu contexto histórico e ideológico; 

\item
Evidenciar as formas pelas quais
a literatura reflete sobre o mundo a sua volta, aproveitando sugestões
para construir personagens, ambientes, conflitos e enredos;

\item
Evidenciar que a literatura também pode ser usada como fonte de registro
de ideias e hábitos pelas Ciências Humanas, como História, Geografia,
Sociologia, Filosofia, Ciências Sociais, Antropologia, Psicologia,
Economia e Direito, entre outras.
\end{enumerate}

\paragraph{Justificativa}
Essa atividade pretende estimular as associações entre disciplinas a fim
de abordar com maior riqueza os objetos de estudo. Também estimular o
uso de meios audiovisuais como ferramentas pedagógicas que possibilitam
uma aproximação mediada de problemas teóricos valiosos para o
desenvolvimento da capacidade argumentativa dos estudantes. Se bem
sucedida, a sequência promove ainda a ampliação dos repertórios
culturais de alunos e alunas e a percepção de que aprendemos em todos os
momentos, enquanto passeamos, assistimos séries ou nos distraímos.

\paragraph{Metodologia}
A primeira aula, sobre a Revolução Industrial, pode ser ricamente complementada
com colaborações das áreas de Ciências Naturais, como a Biologia, a
Química e a Física. (Uma a duas horas)

O professor ou professora de Língua Portuguesa poderia apresentar uma
aula sobre como a Literatura representa a Ciência no século XIX: para
isso, seria útil comentar os romances \emph{Frankenstein} (1823), de
Mary Shelley, e os contos \emph{O alienista} (1882) e ``A causa
secreta'' (1885), de Machado de Assis. Ao final da aula, remeter a
\emph{O médico e o monstro}, observando que o romance de Stevenson se
insere nesse grupo de obras narrativas que aproveitam os progressos
científicos em seus enredos. (Uma hora)

Para incrementar essa sequência didática, o professor ou professora
poderia convidar suas turmas para assistir a filmes que se integram na
mesma série temática, como \emph{De volta para o futuro} (EUA, Dir.
Robert Zemeckis, 1985) e \emph{A invenção de Hugo Cabret} (EUA, Dir.
Martin Scorsese, 2011). Após assistir ao filme, alunos e alunas seriam
convidados a pensar no conflito entre moral e ciência no mundo
contemporâneo, por exemplo, a partir dos problemas colocados pela
engenharia genética e pela possibilidade de criar clones de seres vivos.

Para estimular esse debate, pode-se exibir a reportagem televisiva
``Polêmica: japonês faz clone da própria cachorrinha após morte de
animal de estimação''. O professor ou professora deve chamar atenção
para o modo possivelmente preconceituoso do título da reportagem do
Youtube, ao caracterizar o protagonista da reportagem por sua
nacionalidade, em detrimento de outros qualificativos, como
``empresário'', ``cidadão'', ``aposentado'' etc. A reportagem pode ser
encontrada no Youtube, no link:
https://www.youtube.com/watch?v=ahb7pE4d3C8\\
(Para exibir o filme e fazer um debate, deve-se reservar cerca de 150
minutos com pelo menos dois intervalos; para introduzir o tema, exibir a
reportagem e instalar um debate, cerca de sessenta minutos).

\paragraph{Tempo estimado} Oito aulas de 50 minutos.

\subsection{Leitura}

\subsection{Pós-leitura I}

\paragraph{Tema} Laboratório de adaptação audiovisual.

%(EM13LGG103; EM13LGG105; EM13LGG601; EM13LGG602; EM13LGG603; EM13LGG701; EM13LGG703; EM13LP02; EM13LP06; EM13LP11; EM13LP17; EM13LP18; EM13LP21)

\paragraph{Conteúdo}
A atividade se propõe a trabalhar com outras linguagens das áreas de
artes, como teatro, vídeo e cinema, na realização de uma transposição do
romance \emph{O médico e o monstro} para o meio audiovisual em um
curta-metragem.

\paragraph{Objetivos}

\begin{enumerate}
\item
Desenvolver um produto material, artístico, reflexivo e criativo a
partir das sequências de estudos em torno de \emph{O médico e o
monstro};

\item
Desenvolver habilidades de organização, convivência,
solução de conflitos, trabalho cooperativo; 

\item
Fomentar as habilidades
artísticas dos grupos escolares, estimulando a expressão autoral e
acolhendo no ambiente escolar atividades que permitam a expressão dos
anseios e pensamentos da juventude.
\end{enumerate}

\paragraph{Justificativa}
A variedade de abordagens da leitura literária favorece o vínculo de
alunos e alunas, neste caso, permitindo que eles e elas aportem
habilidades cênicas, de expressão corporal, musicais e técnicas, caso
das habilidades envolvidas na filmagem e edição. Ao instalar uma
sequência ativa, pluralizamos as vozes presentes na construção do fazer
educacional e criamos ensejo para que os corpos se movimentem em torno
de objetivos comuns, rompendo, em certa medida, a rigidez dos circuitos
de interlocução da sala de aula. Permitir a formação de coletivos
artísticos no espaço escolar e no entorno comunitário, ensejando também
maior porosidade nas relações entre estudantes e comunidade escolar,
caso se criem possibilidades para que a comunidade possa apreciar os
resultados de trabalhos como esse e outros.

\paragraph{Metodologia}
\begin{enumerate}
\item
Introdução

Após apresentar a proposta, sugere-se que o professor ou professora
exiba o curta-metragem mudo \emph{O médico e o monstro}, produzido por
Edwin W. Thanhouser, disponível no site Youtube, no link a seguir:

https://www.youtube.com/watch?v=Ua2ha4doVvg

(Cinquenta minutos)

\item
Roteiro

Em seguida, grupos de quatro a seis alunos e alunas realizarão um
curta-metragem mudo de cerca de cinco minutos adaptando o enredo de
\emph{O médico e o monstro}.

A primeira tarefa é escrever um roteiro com as cenas principais que
serão filmadas. O professor ou professora deve apresentar modelos de
redação de roteiro ou solicitar uma pesquisa para os grupos, com
posterior compartilhamento. No site do Instituto de Cinema, há uma
página com instruções. O link para a página é:

https://www.institutodecinema.com.br/mais/conteudo/como-fazer-um-roteiro-de-cinema.

(Cerca de 150 a trezentos minutos)

\item
Organização do trabalho e filmagem

Depois, o grupo deve escolher dois ou três cenários dentro ou fora da
escola, escolher os atores de acordo com o número de personagens
estabelecido no roteiro, escolher outros membros para filmar e editar o
curta-metragem e fazer a gravação das cenas. (Cerca de 150 a trezentos
minutos)

\item
Socialização

Se bem sucedida, a atividade pode originar um festival com as
apresentações dos filmes e pode-se organizar uma eleição para escolher o
melhor curta-metragem, os melhores cenários, atores, figurinos etc.
\end{enumerate}

\paragraph{Tempo estimado} Treze aulas de 50 minutos.

\subsection{Pós-leitura II}

\paragraph{Tema} ``Corredor do inconsciente'': pesquisa e montagem de
exposição escolar

%(EM13CHS104; EM13LGG602; EM13LGG603; EM13LP11; EM13LP12; EM13LP21; EM13LP30; EM13LP32)

\paragraph{Conteúdo}
Essa atividade pode ser favorecida pela colaboração dos professores ou
professoras das áreas de Filosofia, Sociologia, História e Artes. Com
ela, pretendemos apresentar as contribuições de uma das disciplinas mais
influentes surgidas no século XIX: a psicanálise.

\paragraph{Objetivos}
\begin{enumerate}
\item
Ampliar o repertório de conhecimentos sobre o mundo contemporâneo, no
qual a Psicanálise passou a influenciar, praticamente, todas as áreas do
saber; 

\item
Proporcionar ambientes que favorecerão a organização grupal, a
disciplina, o aprendizado de pesquisa e a formalização de conhecimentos
obtidos; 

\item
Proporcionar uma intervenção ativa no espaço escolar,
originada em sequência estruturada de aprendizagem, que suscite a
socialização das aprendizagens e a sociabilidade no espaço escolar.
\end{enumerate}

\paragraph{Justificativa}
Abrir os campos curriculares a disciplinas em geral excluídas
formalmente da trajetória escolar favorece o diálogo com o mundo
contemporâneo, aponta para possibilidades de atuação social e
profissional, estabelece espaços de reelaboração das subjetividades,
matéria de grande importância no mundo atual, no qual professores e
professoras devem estar preparados para lidar com diferentes identidades
culturais e com identidades híbridas, fluidas ou refratárias a soluções
binárias, em geral simplificadoras.

\paragraph{Metodologia}
\begin{enumerate}
\item
Introdução

Um ponto de partida comum pode ser a leitura do artigo ``O mundo secreto
do inconsciente'', de Sílvia Lisboa e Bruno Garattoni, disponível no
site da revista \emph{Super-interessante}, no link abaixo:

https://super.abril.com.br/ciencia/o-mundo-secreto-do-inconsciente/

(Trinta minutos)

\item
Pesquisa

A continuação da atividade seria uma ampliação da pesquisa sobre
psicanálise, realizada em trios ou quartetos, desenvolvendo os seguintes
subtemas e outros a serem propostos pelos professores e professoras e
pela turma:

\begin{itemize}
\item
Tema 1. Conceitos fundamentais da psicanálise, como personalidade,
caráter, subjetividade, consciência e inconsciente, sonhos, pulsões,
sublimação etc. (selecionar de cinco a dez conceitos e redigir
definições de até cinco linhas);

\item
Tema 2. Grandes personalidades da psicanálise (selecionar de cinco a
quinze nomes de homens e mulheres que se destacaram na pesquisa
psicanalítica e redigir breves biografias de cinco a dez linhas);

\item
Tema 3. Psicanálise e literatura (selecionar dez a quinze obras
literárias que foram influenciadas pela Psicanálise e redigir sinopses
de cinco a dez linhas);

\item
Tema 4. Cinema e psicanálise (selecionar dez filmes que se relacionam à
Psicanálise e redigir sinopses de cinco a dez linhas).
\end{itemize}

Sugere-se as seguintes etapas:

\begin{enumerate}
\item
Seleção de fontes de pesquisa: artigos de revistas, jornais e sites,
entrevistas, aulas, documentários, podcasts etc.;

\item
Ler, ouvir ou ver pelos menos três dessas fontes e fazer, no caderno,
uma síntese dos conceitos explicados;

\item
Fazer a lista de conceitos, pessoas, livros ou filmes e redigir os
textos.

Dicas para uma boa pesquisa podem ser conferidas no link a seguir, no
artigo ``5 etapas para realizar uma boa pesquisa escolar'', de Anderson
Moço, no site da revista \emph{Nova Escola}:

https://novaescola.org.br/conteudo/1463/5-etapas-para-realizar-uma-boa-pesquisa-escolar

(120 a 150 minutos)
\end{enumerate}

\item
Construção de um cartaz acadêmico

A última etapa é a construção de um cartaz acadêmico, usando os textos
já desenvolvidos, e usando também recursos gráficos, como desenhos,
fotos, capas de livros ou cartazes de cinema. Também deve-se estimular o
uso de colagem e caligrafia artística, como recursos ornamentais que
valorizam a apresentação do trabalho.

Os educadores devem estimular os estudantes a buscar modelos e tutoriais
de construção de pôsteres científicos na Internet. (150 minutos)

\item
Montagem da exposição ``Corredor do inconsciente''

Em colaboração com a área de Artes, propomos a montagem de uma exposição
com os cartazes sobre Psicanálise. A turma pode fazer um rodízio de
monitores para apresentar oralmente a exposição aos visitantes em
horários agendados e, se for possível, marcar horários para visitantes
da comunidade escolar. A turma pode escrever coletivamente um texto de
apresentação, contando um pouco do processo de trabalho. (Sessenta a 120
minutos)
\end{enumerate}

\section{Aprofundamento}

TODOROV, Tzvetan. \emph{Introdução à literatura fantástica}. 2ª ed. São
Paulo: Perspectiva, 1992 (Coleção Debates).

O curto livro de Todorov é um manancial de análises de obras que se
valem da fantasia e de teorias que explicam como o fantástico é
construído na literatura através de recursos técnicos. Entre os tópicos
abordados estão os gêneros literários, a definição e os tipos de
fantástico e os temas da narrativa fantástica.

NUNES, Benedito. ``A visão romântica'', em \emph{O romantismo},
organização J. Guinsburg. São Paulo: 1978 (Coleção Stylus).

A literatura fantástica ou gótica, ramo em que se encaixa \emph{O médico
e o monstro}, é uma derivação de certos temas do Romantismo, como a
fixação com a morte, a valorização da fantasia e das formas de
espiritualidade não institucionais. Por isso, a leitura do ensaio do
filósofo e crítico Benedito Nunes é de grande valor para se compreender
as raízes dos movimentos culturais mais amplos que resultam em
narrativas como a que estamos abordando neste Manual. O ensaio é denso e
exige uma leitura lenta e cuidadosa, mas o leitor e a leitora não se
arrependerão, pois poderão colher nele muitas lições.

BOSI, Alfredo. \emph{História concisa da literatura brasileira}. 33ª ed.
São Paulo: Cultrix, 1995.

O livro do professor Alfredo Bosi é uma história panorâmica da
literatura brasileira, com muitas análises gerais sobre movimentos e
períodos literários, além de análises de romances, peças teatrais e
poemas, representando uma fonte riquíssima de subsídios para professores
e professoras.

\section{Sugestões de referências complementares}

\subsection{Artigos científicos}

NETTO, Marcus Vinícius Rezende Fagundes. ``O médico, o analista e o
monstro''. \emph{Psicanálise \& Barroco em revista}. V.9, nº 1: 94-114,
julho 2011. Disponível em
http://seer.unirio.br/index.php/psicanalise-barroco/article/
viewFile/8747/7443. Consultado em 27/02/2021.

O artigo apresenta uma interpretação psicanalítica da narrativa de
Robert Louis Stevenson, comparando os papéis do médico e do psicanalista
e os seus respectivos sabere\emph{s}.

PINHEIRO, Carlos Eduardo Brefore. ``Da literatura ao teatro: a eterna
luta entre o bem e o mal nas figuras do Dr. Jekyll e de Mr. Hyde.''
Revista \emph{Travessias} v.4, n.3, 2010. Disponível em
http://e-revista.unioeste. br/index.php/travessias/article/view/4065.
Consultado em 20/01/2021.

O artigo compara o romance de Stevenson com uma de suas adaptações,
neste caso para o musical, \emph{Jekyll and Hyde}, de Leslie Bricusse e
Frank Wildborn.

\subsection{Filme}

\emph{Fragmentado} (EUA, Dir. M. Night Shyamalan, 2016)

O filme conta a história de um homem chamado Kevin Wendel Crumb, que
sofre de Transtorno Dissociativo de Personalidade (TDI). Kevin faz
consultas com uma psiquiatra que conhece as 23 personalidades que ele
possui. Kevin, uma personalidade perturbada e pervertida, realiza um
sequestro e a história acompanha o desenrolar do crime.

\subsection{Quadrinhos}

Há um personagem da Marvel cuja criação pode ter sido inspirada em parte
no romance \emph{O médico e o monstro}: o incrível Hulk. Trata-se de um
cientista chamado Bruce Banner, que foi acidentalmente exposto à
radiação, tendo como efeito que, quando perde o controle emocional, se
transforma numa criatura de pele verde e proporções animalescas, que tem
um comportamento instintivo e descontrolado.


\section{Bibliografia comentada}

STEVENSON, Robert Louis. \emph{O estranho caso do Dr. Jekyll e Mr.
Hyde}. Organização e tradução Bráulio Tavares. São Paulo: Hedra, 2011.

Esta edição em formato de bolso, além da tradução excelente levada a
cabo por um dos maiores especialistas em literatura fantástica do
Brasil, traz em apêndice um conjunto de textos, selecionados pelo
organizador. Alguns são relatos sobre o processo de escrita de \emph{O
médico e o monstro}, um deles escrito pelo autor e outro por sua esposa,
Fanny Van de Grift-Stevenson. Além desses relatos, encontram-se ainda
nessa edição, dois artigos sobre psicologia experimental do século XIX,
``A personalidade multiplex'' e ``As desintegrações do ego''.

%Códigos da BNCC
%
%EM13LGG601; EM13LP01; EM13LP10; EM13LP30; EM13LGG401; EM13LGG402;
%EM13LGG704; EM13LP31; EM13LGG104; EM13LGG602; EM13LGG203; EM13LGG603;
%EM13LP02; EM13LP16; EM13LP11; EM13LP12; EM13LGG303; EM13CHS1020;
%EM13CHS401; EM13CHS504; EM13LGG103; EM13LGG105; EM13LGG701; EM13LGG703;
%EM13LP06; EM13LP17; EM13LP18; EM13LP21; EM13CHS104; EM13LP12; EM13LP32


\end{document}

