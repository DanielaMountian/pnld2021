\documentclass[12pt]{extarticle}
\usepackage{manualdoprofessor}
\usepackage{fichatecnica}
\usepackage{lipsum,media9,graficos}
\usepackage[justification=raggedright]{caption}
\usepackage[one]{bncc}
\usepackage[hedra]{../edlab}

\begin{document}


\newcommand{\AutorLivro}{João Martins de Athayde}
\newcommand{\TituloLivro}{Tipógrafo, poeta e editor}
\newcommand{\Tema}{Ficção, mistério e fantasia}
\newcommand{\Genero}{Poema}
\newcommand{\imagemCapa}{./images/PNLD0045-01.png}
\newcommand{\issnppub}{---}
\newcommand{\issnepub}{---}
% \newcommand{\fichacatalografica}{PNLD0045-00.png}
\newcommand{\colaborador}{\textbf{Bruno Gradella e Vicente Castro}}


\title{\TituloLivro}
\author{\AutorLivro}
\def\authornotes{\colaborador}

\date{}
\maketitle

\baselineskip=1.15\baselineskip\par

\begin{abstract}\addcontentsline{toc}{section}{Carta ao professor}
Este Manual tem como objetivo fornecer subsídios para o trabalho com a
obra literária \emph{Tipógrafo, poeta e editor}, de João Martins de
Athayde.

João Martins de Athayde nasceu e cresceu no meio do sertão da Paraíba, em 
Cachoeira de Cebolas, povoado de Ingá do Bacamarte, junho de 1880. Não se sabe 
se fugindo de casa ou com o consentimento do pai, já adolescente saiu de Ingá
e foi para o Recife trabalhar no comércio e em algumas fábricas. Em 1908 começou 
a escrever seus primeiros folhetos que vendia nas principais feiras e mercados da 
cidade. Com o apurado das vendas e outros pequenos empregos, conseguiu comprar uma 
pequena impressora manual. Ganha fama com seus poemas e com essa pequena tipografia 
que, ainda no início do século \textsc{xx}, especializou-se na publicação de cordéis. 
Escreveu centenas de folhetos e publicou inúmeros outros, já que comprou os direitos
autorais de diversos poetas.

Em \emph{Tipógrafo, poeta e editor} (2021), coletânea de oito cordeis de sua vasta obra,
o estudante do ensino médio encontrará histórias que retratam temáticas sobre a vida no
sertão, o amor, o grotesco, a aventura e a imaginação, resgatando elementos da cultura e 
imaginários popular, possibilitando uma maior compreensão da diversidade artística
do cordel.

\emph{Como Lampião entrou na cidade de Juazeiro}, cordel que abre a coletânea,
descobrimos que Virgulino tem grande respeito a Padre Cícero, e que havia ajudado 
o tenente Chagas contra um grupo revoltado, em Cipó de Pernambuco. Por essa ajuda
ao exército e por sua devoção ao Padrinho Cícero, Lampião, para o desgosto da
polícia local, tem um salvo-conduto na cidade, contanto que, em respeito ao líder religioso, 
mantenha-se desarmado.

Já em \emph{A sorte de uma meretriz}, acompanhamos a história de uma bela moça
que, para infortúnio de seu pai, um rico fazendeiro, decide tornar-se prostituta. 
A jovem, por sua beleza incomum, recebe muitos presentes de seus clientes, mas, 
seguindo o destino da profissão, cai doente e é abandonada por todos. 

Esperamos que as indicações propostas aqui sejam muito úteis no trabalho em
sala de aula!




\end{abstract}

\tableofcontents

\section{Introdução}

Olá! é com muita alegria que apresentamos a você “tipógrafo, poeta e editor”, de joão martins de athayde.

Esta é uma obra produzida por aquele que é considerado um dos maiores cordelistas brasileiros.

O autor nasceu em mil oitocentos e setenta e sete, na paraíba,  e é considerado  o príncipe dos poetas populares do nordeste do brasil  

Mesmo sem frequentar a escola,  seu maior sonho sempre foi criar histórias e compor versos

Depois de cada dia de trabalho, que às vezes se prolongava até a madrugada, ele tinha uma grande paixão, que era o cinema 

Ninguém sabe, com certeza, quantos folhetos foram escritos e publicados por joão martins de athayde

Sua gráfica, trabalhando a todo vapor, quase que semanalmente lançava um título novo ou mais uma edição de um folheto que estivesse fazendo sucesso

As encomendas recebidas de seus agentes espalhados por todo o nordeste chegavam quase todos os dias, e ele procurava entregar, imprimindo durante as madrugadas

É com athayde que se realizam mudanças na relação entre os poetas e o proprietário da gráfica, e na apresentação dos folhetos

Ele fez surgir os contratos de edição com o pagamento de direitos de propriedade intelectual, o uso de subtítulos e pre mbulos em prosa e o padrão fixo dos folhetos pelo número de páginas em múltiplos de quatro

João martins de athayde contribuiu para o desenvolvimento da arte e da comercialização do folheto popular no recife, para onde se mudou

Foi o desbravador da indústria do folheto de cordel no país, industrializando e comercializando sua produção e a de outros artistas

Criou uma grande rede de atividades lucrativas no nordeste, que se espalhou para outras regiões brasileiras, possibilitando a diversos poetas populares se dedicarem exclusivamente à poesia como atividade profissional

João martins de athayde, no ano de mil novecentos e quarenta e nove, após haver passado por um acidente vascular cerebral, se afastou da atividade de editor e vendeu a sua tipografia 

Athayde foi acusado de comprar originais de dezenas de poetas populares e publicá-los sem mencionar os nomes dos autores

Esse fato dificultou a identificação da autoria de histórias rimadas da literatura de cordel

No entanto, esse fato não diminui a import ncia da sua obra, tampouco sua contribuição para a poesia popular no brasil

Suas obras até hoje são reimpressas, quando seu estilo irônico e jornalístico se revela nos versos que faziam crítica aos costumes modernos

Nesta obra,  você vai encontrar uma seleção de oito cordéis  que abordam o universo do sertão nordestino .

O imaginário da literatura de cordel  é criado a partir de relações entre mundos culturais distintos  

Nesses poemas,  a imagem poética não representa a realidade 

Ela faz parte de um imaginário construído,  partilhado pelo  poeta  com seus ouvintes e leitores 

A poesia popular do nordeste do brasil  tem muita influência da linguagem oral  e  aproveita muito da língua coloquial,  praticada nas ruas e na comunicação cotidiana  

Por isso,  a poesia narrativa do cordel  é compartilhada e desfrutada de forma coletiva  e atinge uma grande resson ncia social  

No cordel,  a métrica e a rima são recursos que favorecem a memorização 

Todo grande cordelista dialoga com uma longa tradição de poetas do passado,  recuperando e recriando  temas,  formas  e imagens 

Assim,  a sabedoria de matriz oral e popular,  acumulada nessa tradição,  é transmitida pelos cordelistas 

O contato com essa literatura  é de extrema import ncia para compreender  a diversidade cultural do nosso país,  além de favorecer a quebra de estereótipos  e ampliar a discussão sobre a produção artística brasileira. 

Muitos temas do cordel  são originários das tradições populares e eruditas da europa medieval e moderna .

Nos folhetos  encontramos temas das novelas de cavalaria da idade média  e das narrativas bíblicas 

A variedade linguística utilizada pelo poeta é caracterizada pelo registro, na escrita, de formas típicas da linguagem oral

O cordel é uma forma poética popular de origem europeia., mas foi incorporado e desenvolvido na cultura brasileira, sobretudo do nordeste do país

Os poemas em cordel costumam ser longos e narram histórias repletas de reviravoltas e aventuras

A linguagem é marcada pelo humor e pelo suspense, com o objetivo de captar a atenção do público

Desde a sua origem, para facilitar a memorização e a recitação falada, esses textos apresentam um esquema próprio de rimas e uma métrica específica

Antigamente, os cordéis eram vendidos nas feiras, em barracas e tendas que os expunham pendurados em varais de barbante, presos por pregadores de roupas ou exibidos em balcões e esteiras

Muitas vezes, performances eram realizadas ali mesmo, pelos próprios autores que declamavam oralmente as narrativas em versos

Havia uma roda de ouvintes, formada por passantes que paravam para ouvir as histórias

Hoje, em vários lugares, essa configuração mudou um pouco e os cordéis são expostos dentro de plásticos transparentes, em bancas de revistas, e acabam sendo mais lidos silenciosamente

Algumas dessas obras podem também ser musicadas e interpretadas por cantores

A matriz oral dos cordéis remonta a tempos ancestrais, quando histórias eram compostas e contadas por rapsodos e aedos para entreter um grupo de ouvintes

Muitas narrativas célebres que conhecemos hoje, como os poemas épicos e as antigas sagas, nasceram nesse contexto de oralidade e, somente depois de muito tempo, ganharam versões escritas

No caso do cordel, temos uma forma poética cujas origens podem ser encontradas nas tradições poéticas orais da antiguidade clássica

Além disso, havia as práticas dos trovadores medievais da europa que, em portugal, na espanha e no sul da frança, compunham e recitavam poemas narrativos

Durante o século dezenove,  no brasil e, em especial, a partir do estado da paraíba, os poemas narrativos começaram a circular em forma de livretos expostos nas feiras populares

No caso de athayde, havia poemas abrangendo vários ciclos, dentre eles o do sertanejo, o heroico, o circunstancial, o da crítica de costumes, o das pelejas e desafios, o das fábulas, o dos repentes

A obra do poeta athayde deve ser observada em conjunto, nos seus vários aspectos

O folclorista luís da c mara cascudo afirmou que athayde foi "o maior poeta, mais tradicionalista do nordeste brasileiro"


\section{Atividades 1}

%\BNCC{EM13LP26}

\subsection{Pré-leitura}

%\BNCC{EM13LGG302}
%\BNCC{EM13LGG704}
%\BNCC{EM13LP10}
%\BNCC{EM13LP19}

% \paragraph{Tema}
% \paragraph{Conteúdo}
% \paragraph{Objetivo}
% \paragraph{Justificativa}
% \paragraph{Metodologia}
% 	\begin{enumerate}
% 	\end{enumerate}
% \paragraph{Tempo estimado}

Antes de se iniciar a leitura, é interessante que os alunos
entendam o universo para o qual estão enveredando. A literatura de
cordel é um patrimônio da cultura brasileira. Fabricada em folhetos,
geralmente trazem para o universo escrito casos e histórias já comuns na
oralidade popular. Também é comum que seu texto seja construído por meio
de rimas. Além disso, uma das características típicas do cordel são as
gravuras, que possuem um tracejado muito próprio. Não obstante, por
serem uma expressão muito típica da cultura de uma região do país, os
diversos títulos produzidos em cordel trazem elementos básicos que são
características intrínsecas muito próprias. Essa atividade sugere que os
alunos façam uma pesquisa com vários títulos de cordel. Procurem olhar a
sinopse da obra, folheiem seu conteúdo e busquem temas e personagens que
frequentemente aparecem nessas obras. Isso posto, é interessante que
cada aluno escreva um relatório com seus pareceres iniciais, colocando
suas suposições do porquê da repetição desses eixos temáticos e
personagens. Posteriormente, o professor pode coletar essas informações
junto aos alunos e abordar as questões durante as aulas sobre o livro.

\Image{Literatura de Cordel (Diego Dacal/Wikimedia Commons; CC-BY-SA 2.0)}{PNLD0045-05.png}

\subsection{Leitura}

%\BNCC{EM13LGG103}
%\BNCC{EM13LP02}
%\BNCC{EM13LP48}

% \paragraph{Tema}
% \paragraph{Conteúdo}
% \paragraph{Objetivo}
% \paragraph{Justificativa}
% \paragraph{Metodologia}
% 	\begin{enumerate}
% 	\end{enumerate}
% \paragraph{Tempo estimado}

Já mais familiarizados com as características do cordel, os
alunos devem passar então à produção de um texto nesse formato. É
possível trabalhar com a redação de uma autobiografia do aluno, contando
sua história nos moldes de um cordel. Entretanto, é possível que alguns
alunos não se sintam confortáveis em relatar experiências próprias.
Diante disso é possível propor que seja uma biografia com elementos
romanceados, elementos do fantástico, ou mesmo a biografia de uma
terceira pessoa, pela qual o aluno nutre admiração. Em certo sentido, os
cordéis promovem isso, personagens admiráveis, de modo que escrever
sobre um ídolo ainda é algo muito produtivo nesta atividade. Além do
mais, só a produção no formato já ajudará os alunos a terem maior
familiaridade com o gênero. As rimas devem ser estimuladas, mas não são
uma obrigatoriedade, posto que podem gerar uma trava produtiva a alguns
estudantes.

\subsection{Pós-leitura}

%\BNCC{EM13LGG102}
%\BNCC{EM13LGG303}
%\BNCC{EM13LGG402}
%\BNCC{EM13LGG703}
%\BNCC{EM13LP13}
%\BNCC{EM13LP14}
%\BNCC{EM13LP28}
%\BNCC{EM13LP29}
%\BNCC{EM13LP52}

\Image{Retrato de João Martins de Athayde (Memórias do Cordel; CC-BY-SA 2.0)}{PNLD0045-03.png}
Com o material produzido na atividade anterior, é possível a
realização de uma grande feira de cordéis na escola, junto ao professor
de artes, sugere-se que os alunos preparem uma decoração adequada,
valorizando as cores e os traços presentes nesse tipo de material tão
distinto. Além da exposição dos cordéis produzidos, sugere-se também a
promoção de encenações de trechos de obras lidas, ou produzidas pelos
alunos, a declamação de poemas, competição de repentes. Também se
estimula a aproximação deste gênero com outras produções artísticas
mundiais. Nada impede que, em uma dessas encenações ou leituras
dramáticas, os alunos podem buscar trechos de peças, filmes ou obras da
literatura mundial e encená-los, valendo-se da estética e do vocabulário
típico aos cordéis.

\section{Atividades 2}

%\BNCC{EM13CNT201}
%\BNCC{EM13CNT303}
%\BNCC{EM13CHS101}
%\BNCC{EM13CHS102}
%\BNCC{EM13CHS106}
%\BNCC{EM13CHS401}

% \paragraph{Tema}
% \paragraph{Conteúdo}
% \paragraph{Objetivo}
% \paragraph{Justificativa}
% \paragraph{Metodologia}
% 	\begin{enumerate}
% 	\end{enumerate}
% \paragraph{Tempo estimado}

A obra \emph{Tipógrafo, poeta e editor} possibilita trabalhos
interdisciplinares e integradores de diferentes campos do saber e áreas
de conhecimento. A seguir, propomos algumas atividades que podem ser
desenvolvidas conjuntamente com professores de outras áreas. Além das
habilidades de Linguagens e suas Tecnologias e de Língua Portuguesa,
indicadas nas etapas da seção anterior e válidas também para esta,
listamos a seguir as habilidades de outras áreas, presentes na abordagem
interdisciplinar:

\subsection{Pré-leitura}

% \paragraph{Tema}
% \paragraph{Conteúdo}
% \paragraph{Objetivo}
% \paragraph{Justificativa}
% \paragraph{Metodologia}
% 	\begin{enumerate}
% 	\end{enumerate}
% \paragraph{Tempo estimado}

Antes da leitura, é possível discutir a situação da seca no
Nordeste. Professores de ciências humanas podem discutir a história
desse fenômeno e suas implicações sociais, bem como as relações que o
homem travou com essa situação ao longo de toda a ocupação da região.
Por outro lado, professores de ciências da natureza podem auxiliar os
alunos na compreensão dos fenômenos geológicos e meteorológicos que
contribuem para a perpetuação da aridez local, bem como explicar o bioma
da caatinga e sua importância para o equilíbrio ecológico. Munidos dessa
informação os alunos podem produzir um jornal, tanto nos moldes
escritos, como em vídeo, construindo uma reportagem sobre todas as
questões aventadas. O material produzido pode ser divulgado no site da
escola. Como sugestão, indica-se utilizar aplicativos gratuitos de
gravação e edição de vídeos, disponíveis em dispositivos digitais

\subsection{Leitura}

% \paragraph{Tema}
% \paragraph{Conteúdo}
% \paragraph{Objetivo}
% \paragraph{Justificativa}
% \paragraph{Metodologia}
% 	\begin{enumerate}
% 	\end{enumerate}
% \paragraph{Tempo estimado}

Entre os poemas do livro, está o intitulado \emph{História
de Aladim e a Lâmpada Maravilhosa}, baseada na famosa história que
compõe as \emph{Mil e Uma Noites}, já tendo recebido várias adaptações
para filmes e desenhos. O que poucas pessoas sabem, é que essa história
passou a integrar o cômputo das \emph{Mil e Uma Noites}, apenas no
século XVIII, após ter aparecido em uma tradução francesa. O tradutor,
Antoine Galland, afirmava que ele havia ouvido a narrativa de um
contador de histórias Sírio. Em geral, estudiosos entendem que a
história de Aladim é de fato uma história árabe, produzida entre os
séculos VII e XIII, tendo permanecido a maior parte de sua existência na
oralidade. Entretanto, há alguns que entendem que se trata de uma
fabricação europeia, imbuída do que o intelectual Edward Said definiu
como Orientalismo. Com auxílio do professor de ciências humanas, peça
para que os alunos levantem informações e debatam acerca desse conceito
e significado.

É importante que alguns pontos sejam levantados, como por exemplo, o que
Said define como caráter generalizante das observações ocidentais em
relação ao Oriente, que colocam culturas distintas como se fossem
homogêneas apenas por uma localização geográfica. Entre esses elementos,
Said indica que os ocidentais vislumbram o Oriente como o lugar dos
déspotas, do esplendor, das especiarias, das sabedorias misteriosa, da
sensualidade, de modo que, muitos, ao viajarem para o Oriente, se
frustram, porque a realidade é totalmente distinta da expectativa. Aqui
se frisa que, apesar de alguns pontos dessa visão serem positivos, ela
ainda é estereotipada.

Com isso em mãos, acrescido da pesquisa dos alunos, promova uma reflexão
e debate acerca da questão, transpondo-a para a realidade nacional.
Indague aos alunos se no próprio Brasil existe, ou não, uma visão
estereotipada de uma região frente a outra.

\Image{Capa do cordel ``História de Roberto do Diabo'' (Fundação Casa de Rui Barbosa; Domínio Público)}{PNLD0045-06.png}

\subsection{Pós-leitura}
% \paragraph{Tema}
% \paragraph{Conteúdo}
% \paragraph{Objetivo}
% \paragraph{Justificativa}
% \paragraph{Metodologia}
% 	\begin{enumerate}
% 	\end{enumerate}
% \paragraph{Tempo estimado}

Na obra lida, é feita a oposição entre campo e cidade, e as
características das pessoas que habitam esses dois ambientes. Nesta
temática, do homem rural em oposição ao homem urbano, proponha aos
alunos, contando com o auxílio do professor de ciências humanas, a
criação de um mapa. Nele, inicialmente divida as regiões em
majoritariamente urbanas e rurais. Feito isso, aconselha-se buscar
notícias e estudos sobre as grandes desigualdades e os fatores do
deslocamento populacional para as grandes cidades. Com essa informação,
é possível indicar de maneira gráfica, valendo-se de setas por exemplos,
caso seja um painel físico, ou por recursos animados, caso o mapa esteja
sendo produzido virtualmente, os fluxos migratórios estudados. Como
fonte, também, sugere-se orientar a pesquisa por meio da obra de
Patativa do Assaré e de outros autores nordestinos.

\Image{Escultura Patativa do Assaré, Fortaleza, Ceará (Autoria Desconhecida; CC-BY-SA 2.0)}{PNLD0045-08.png}

Para complementar a obra, sugere-se criar xilogravuras a partir da
confecção de carimbos artesanais.

\Image{Xilogravura da Artista plástica Yolanda Carvalho (Rafael Nolêto/Wikimedia Commons; CC-BY-SA 3.0)}{PNLD0045-10.png}

\section{Aprofundamento}

Ao chegar ao Ensino Médio, é necessário que os estudantes se aprofundem
na compreensão das múltiplas linguagens e, sobretudo, da linguagem
literária. Em relação à literatura, a BNCC traz as seguintes
considerações:

\begin{quote}
{[}...{]} a leitura do texto literário, que ocupa o centro do trabalho
no Ensino Fundamental, deve permanecer nuclear também no Ensino Médio.
Por força de certa simplificação didática, as biografias de autores, as
características de épocas, os resumos e outros gêneros artísticos
substitutivos, como o cinema e as HQs, têm relegado o texto literário a
um plano secundário do ensino. Assim, é importante não só (re)colocá-lo
como ponto de partida para o trabalho com a literatura, como
intensificar seu convívio com os estudantes. Como linguagem
artisticamente organizada, a literatura enriquece nossa percepção e
nossa visão de mundo. Mediante arranjos especiais das palavras, ela cria
um universo que nos permite aumentar nossa capacidade de ver e sentir.
Nesse sentido, a literatura possibilita uma ampliação da nossa visão do
mundo, ajuda-nos não só a ver mais, mas a colocar em questão muito do
que estamos vendo/vivenciando. (Brasil, 2018, p. 491)
\end{quote}

Nesta seção, desenvolvemos um trabalho de aprofundamento que, em diálogo
com a formação continuada de professores, oferece subsídios para a
abordagem do texto literário. A leitura em sala de aula de
\emph{Patativa do Assaré} pode ser enriquecida pelo aprofundamento no
universo literário.

Levar a literatura de cordel até à escola significa motivar o aluno a
conhecer a formação cultural do povo brasileiro e valorizar a
diversidade regional.

O cordel não trata apenas de ficção, mas dialoga também com fatos
históricos e com o cotidiano brasileiro. As obras da literatura de
cordel podem promover o incentivo à leitura com foco na oralidade e na
memorização.

A escola deve possibilitar o contato com a literatura popular e promover
a ampliação do repertório sobre escritores nem sempre inseridos no
cânone literário.

Assim como ocorre com outros gêneros que operam com a ficção, o poema em
cordel pode contar histórias de amor, casos fantásticos ou assustadores,
causos e anedotas, versões de lendas populares e clássicos da
literatura, ou mesmo narrar fatos históricos marcantes para a vida de
uma comunidade.

Em um contexto de miséria e analfabetismo largamente propagado, em meio
à ausência de estruturas educativas de base, o poeta popular desempenha
um papel importante no despertar da consciência cívica e política.

Trazer leituras que abordem esse universo, mesmo que no âmbito regional,
possibilita mapear questões históricas e sociais. A popularização dessa
literatura em versos por todo o brasil permitiu que, ao longo dos anos,
o cordel fosse um meio de alfabetização de muitas comunidades do
interior do país.

Embora o cordel seja uma fonte de conhecimento do passado, ela ainda é
importante no mundo de hoje, mesmo com todos os avanços dos meios de
comunicação.

Dar ao leitor acesso à literatura de cordel, o possibilita refletir,
sensibilizar e entrar em contato com a vasta identidade cultural
brasileira. Além de patrimônio cultural, o resgate destas manifestações
de tradição oral, propõe um debate sobre a escuta e a fala, a variedade
linguística, a musicalidade, e os diálogos possíveis com a comunicação
contemporânea.

O cordel é valorizado como expressão poética de alta significação por
escritores do porte de Ariano Suassuna, Carlos Drummond de Andrade,
Jorge Amado, Guimarães Rosa, Mario de Andrade, João Cabral de Melo Neto,
motivando estudos e pesquisas nas áreas de Antropologia, Folclore,
Linguística, Literatura, História, entre outras.

A questão da oralidade e escuta, aspectos essenciais na criação e
compartilhamento na literatura de cordel, as temáticas regionais e a
variação linguística possibilita um campo literário muito convidativo
porque cria no leitor uma postura mais ativa em relação ao texto, que
tem em si uma métrica e um ritmo musicalizado. As xilogravuras e
desenhos que ilustram as capas dos folhetos são uma manifestação da
criatividade de artistas populares, com suas soluções plásticas em que
se destaca o traço forte, de bela expressividade.

O contato com a literatura de cordel é de extrema importância para
compreendermos a diversidade cultural do nosso país, além de favorecer a
quebra de estereótipos e ampliar a discussão sobre a produção artística
brasileira.

\Image{Capa do cordel ``Os sofrimentos de Alzira'' (Fundação Casa de Rui Barbosa; Domínio Público)}{PNLD0045-07.png}

\subsection{Importância da Literatura de Cordel em sala}

Com uma produção simples e de grande abrangência, a Literatura de Cordel
ganhou espaço e prestígio na cultura nordestina brasileira, tornando-se,
em 2018, patrimônio cultural imaterial do nosso país -- reconhecido pelo
Conselho Consultivo do Instituto do Patrimônio Histórico e Artístico
Nacional. Assim, o Cordel como gênero do discurso contribui na formação
do aluno possibilitando o domínio de outros conteúdos, além da
descentralização do ensino. O estudo da Literatura de Cordel como forma
de expressão da cultura popular contribui também no aprimoramento das
habilidades de oralidade, escrita, leitura, interpretação, linguagens
artísticas e até na dramatização de peças, auxiliando na
interdisciplinaridade dos temas.

A Educação Literária aparece entre as competências gerais da BNCC (Base
Nacional Comum Curricular), enfatizando a importância vivência do aluno
no aprendizado da literatura e demais manifestações artísticas --
``\emph{Valorizar e fruir as diversas manifestações artísticas e
culturais, das locais às mundiais, e também participar de práticas
diversificadas da produção artístico-cultural\footnote{BRASIL. Ministério da Educação. Base Nacional Comum Curricular.
Brasília, 2018.}'' --},
logo, o Cordel ganha espaço nesse cenário, pois, a partir de seu estudo,
é possível despertar nos alunos o interesse por diversos campos
artísticos.

Entretanto, o uso do Cordel em sala de aula, bem como nos livros
didáticos, ainda é muito restrito, por não ser tão prestigiado quanto os
demais gêneros literários. Diante disso, cabe a nós resgatarmos essa
parte da nossa identidade nacional, contextualizando o aluno no meio
social e cultural de seu país.

\subsection{Variação linguística e oralidade}

O Cordel nasceu da oralidade e da linguagem popular; uma leitura
silenciosa limita seu poder de comunicação, impedindo que seu potencial
seja trabalhado como um todo. O gênero, se bem explorado, pode auxiliar
no aprendizado e desenvoltura dos alunos nessa modalidade de expressão,
devido ao seu ritmo cadenciado e seu linguajar comum, próximo ao
cotidiano do aluno. Para isso, o professor deve promover atividades que
possibilitem a verbalização do aluno, que estimulem a livre expressão,
para que ele possa, a partir dos exercícios, identificar seu local de
fala, além de desenvolver respeito e empatia pela fala do outro,
aprimorando, também, a convivência social.

Ao utilizar a Literatura de Cordel, o professor poderá abordar a questão
do preconceito linguístico da língua portuguesa, ao estimular a leitura
de poemas que fogem do padrão gramaticalmente institucionalizado. É
possível mostrar aos alunos que a linguagem popular é muitas vezes
discriminada, mesmo fazendo parte de uma cultura rica e diversificada,
quebrando a ideia de que o ideal é necessariamente o padrão.

\subsection{Variação cultural e geográfica}

Como explicitado anteriormente, é sabido que a Literatura de Cordel faz
parte de nossa cultura e tradição. Antes mesmo da chegada das grandes
mídias e meios de comunicação, o Cordel funcionou como instrumento de
disseminação de valores, lendas e conhecimento popular da tradição
nordestina. Levá-lo à escola é uma maneira de resgate da nossa cultura,
motivando o aluno a conhecer mais sobre nosso país e seus diferentes
povos e regiões, além de nossa história religiosa, econômica e política,
vez que muitos cordéis abordam realisticamente essas questões.

\subsection{Campo artístico e literário}

A ilustração com xilogravura (gravura em madeira) é uma característica
marcante dos folhetos de Cordel, usada para decorar e dar mais vida aos
poemas, além de oferecer material para as mais variadas interpretações
das obras. O uso da técnica deu-se graças ao baixo custo de produção e
foi fundamental para disseminar a cultura do Nordeste em outras partes
do Brasil. Os traços marcantes da xilogravura de cordel em composição
com os poemas se transformam em uma expressão de linguagem, registrando
a história do nosso povo.

Levar componentes artísticos para a sala de aula é uma forma de chamar à
atenção do aluno, além de proporcionar maior pensamento crítico e
incentivo à expressão artística e literária.

\Image{Capa de ``A Xilogravura Popular e a Literatura de Cordel'', 1985 (Thomas Fisher Rare Book Library/Flickr; CC-BY-NC-SA 2.0)}{PNLD0045-09.png}

\subsection{O autor e a obra}

João Martins de Athayde nasceu em Cachoeira de Cebolas, povoado de Ingá
do Bacamarte, Paraíba, no dia 23 de junho de 1880. Devido à seca de
1898, migrou para Pernambuco, radicando-se no Recife. Publicou o seu
primeiro folheto em 1908.

\Image{Mapa de Ingá na Paraíba, cidade onde nasceu o autor (Marcos Elias de Oliveira Júnior/Wikimedia Commons; CC-BY-SA 3.0)}{PNLD0045-04.png}

Seu pai, o velho Belchior Martins de Luna era um pequeno agricultor e
tirava da terra o sustento de sua família, composta pela mulher e três
filhos. Com a morte da esposa, o velho Belchior casou pela segunda vez e
seus filhos não tiveram sorte com a madrasta, que judiava dos enteados a
ponto de deixá-los quase nus e em geral mal alimentados. Sem que o pai
soubesse, cada menino recebia da madrasta apenas uma xícara de farinha
de mandioca, sem um pedacinho de carne sequer, o que fez com que o
menino João Martins de Athayde, certa vez voltava do roçado, desmaiou de
fome no caminho, onde foi encontrado por conhecidos de seu pai que
ajudaram a trazê-lo para casa.

Mesmo sem frequentar a escola, o poeta andava com uma carta de ABC no
chapéu, pois seu maior sonho era aprender a ler e a escrever. E tanto
era assim, que ele saía perguntando as letras às pessoas e, como não
tinha caderno nem lápis, escrevia no chão, com o dedo.

Não se sabe se ele fugiu de casa ou se teve o consentimento do pai para
vir tentar a vida no Recife, onde tinha um parente ou conhecido que era
pequeno comerciante. No Recife, trabalhou no comércio e, dizem, até em
fábrica não sei de quê.

Sabendo ler e escrever, João Martins de Athayde começou a escrever seus
primeiros versos e imprimir seus primeiros folhetos, que vendia nas
feiras e nos mercados do Recife. Com o dinheiro da venda dos folhetos e
com o que ganhava nos empregos, conseguiu comprar uma pequena impressora
manual, uma guilhotina para cortar o papel dos folhetos, alugar uma
casa, contratar vários empregados, de vez que a procura de seus folhetos
era tão grande e ele procurava atender à freguesia e aos seus
agentes-vendedores em diversas cidades do Nordeste.

Estava, assim, feito o poeta popular João Martins de Athayde, que, com o
apurado dos folhetos que escrevia, comprou a casa onde morava e uma
máquina melhor. Sua fama de poeta popular corria solta pelas feiras e
pelos mercados do Nordeste.

O poeta João Martins de Athayde faleceu às seis horas do dia 7 de agosto
de 1959, na cidade de Limoeiro, Pernambuco, vítima de uma embolia
cerebral, deixando a viúva Sofia Cavalcanti de Athayde e oito filhos:
Josefa Augusta de Athayde Dornelas, João Martins de Athayde Filho,
Manoel Cristiano de Athayde, Maria José de Athayde, Ceci de Athayde
Montenegro, João Oliveira de Athayde, Fernando Oliveira de Athayde,
Carlos Oliveira de Athayde e Marcus VinÍcius de Athayde.

Ninguém sabe, com absoluta certeza, quantos folhetos foram escritos e
publicados por João Martins de Athayde. Sua gráfica, trabalhando a todo
vapor, quase que semanalmente lançava um título novo ou mais uma
edição/impressão de um folheto que, na época, estivesse fazendo sucesso.
As encomendas recebidas de seus agentes-vendedores espalhados por todo o
Nordeste chegavam quase todos os dias, e ele procurava entregar,
imprimindo durante as madrugadas.

Como se vê, João Martins de Athayde não era um poeta cuja temática fosse
o sobrenatural, apesar de alguns de seus folhetos enfocarem o céu, padre
Cícero, o Diabo ou o inferno. Não era, também, o poeta do
circunstancial, de fazer um jornalismo paralelo, como José Costa Leite,
o poeta-repórter. Era, sim, um poeta voltado para o amor, para a
aventura, para o grotesco, para o mundo da imaginação. Mas seus
folhetos, no que se refere à autoria, geraram dúvidas com a venda que o
poeta fez aos herdeiros de José Bernardo da Silva dos direitos autorais
dos seus folhetos, que passaram a trazer impressos na capa, como também
na primeira página, os dizeres: JOÃO MARTINS DE ATHAYDE.

Sobre a autoria dos folhetos de José Bernardo da Silva ou de João
Martins de Athayde, acreditamos que ela só poderia ser elucidada se
fosse feito, a cargo de linguistas e outros especialistas no assunto, um
sério estudo do vocabulário, dos temas de predileção, das rimas e outros
recursos técnicos que fogem ao meu conhecimento. Na realidade, a obra de
João Martins de Athayde, como poeta popular, é uma das mais
significativas e ricas do Nordeste.

O presente livro traz uma coletânea de oito cordéis da vasta trajetória
artística de João Martins de Athayde, poeta popular e um dos mais
significativos artistas deste gênero do Nordeste. Sua obra retrata
temáticas sobre a vida no sertão, o amor, o grotesco, a aventura e a
imaginação, resgatando elementos da cultura e imaginários popular,
possibilitando uma maior compreensão da diversidade artística.

A literatura de cordel, também conhecida como poesia popular, que se
popularizou no Brasil por volta do século 18. Sua grande contribuição é
a difusão das manifestações da cultura popular que caracterizam a
identidade social de um povo e fortalece as identidades regionais. O
texto é escrito com métrica fixa e rimas que fazem a musicalidade dos
versos. A literatura de cordel é muito conhecida por suas xilogravuras
(gravuras em madeira), que ilustram as páginas dos poemas.

O imaginário do cordel é criado a partir de múltiplas relações entre
mundos culturais distintos, o que implica que não se pode tomar a imagem
da poética enquanto imagem do real, mas de um imaginário construído e
partilhado por aqueles que se associam, a partir deste universo poético,
a uma relação que vincula o criador e o receptor do cordel.

\subsection{A estrutura da métrica}

A Literatura de Cordel sofreu, estruturalmente, diversas modificações
com o passar dos anos, por se tratar de uma linguagem oral que foi sendo
transformada, também, em escrita. No início, os repentistas não tinham
compromisso com número de versos ou métrica, entretanto a rima sempre
esteve presente nos poemas -- instrumento utilizado para favorecer a
memorização e facilitar a articulação dos repentistas. Entretanto, a
simplicidade não está atrelada apenas à oralidade, mas também ao alcance
social que uma linguagem acessível pode fornecer.

Em \emph{Uma voz do Nordeste}, estão reunidos cinco poemas dos mais
diversos tipos de métricas e rimas, representando a expressão do canto
do poeta e desvencilhando-se da forma de saber erudita.

\subsubsection{Papel do leitor}

O hábito de leitura auxilia na evolução de diversas habilidades, em
especial se desenvolvido durante a infância e adolescência. Ele trabalha
diretamente com o aprimoramento do vocabulário, criatividade,
imaginação, além das habilidades socioemocionais, como maior habilidade
para estabelecer diálogos, lidar com desafios e sentimentos, ajudando na
formação do indivíduo. Esses são aspectos fundamentais para uma melhor
interação social, além de proporcionar sensações ímpares.

É de extrema importância que o professor incentive o aluno a adquirir o
hábito de leitura pois, através dela, é possível que ele aprenda, viaje
e descubra sobre novos lugares e povos sem sair de casa.

\section{Sugestões de referências complementares}



\subsection{Filmes}

  \textbf{Auto da Compadecida. Direção: Guel Arraes (Brasil}, 2000).

Baseado na peça de Ariano Suassuna, o filme evoca o imaginário popular e
religioso do Nordeste para contar as aventuras de João Grilo e Chicó,
uma dupla de malandros que sobrevive de trapaças.


  \textbf{Patativa do Assaré -- Ave Poesia}. Direção: Rosemberg Cariny
  (Brasil, 2007).

O documentário apresenta a trajetória da vida e da obra do poeta
cearense Patativa do Assaré, que explorou em sua obra a riqueza das
tradições populares. Sua história é contada por meio de depoimentos de
amigos, familiares e admiradores que destacam a relevância do artista
para a cultura brasileira.


  \subsection{\emph{Site}}


\textbf{cordel: literatura popular em verso }

(\url{http://www.casaruibarbosa.gov.br/cordel/acervo.html})

No \emph{site} da Fundação Casa de Rui Barbosa, há informações e
materiais diversos sobre o acervo da instituição, com diversos
exemplares representativos da literatura nacional em cordel.

\section{Bibliografia comentada}


 ANDRADE, Cláudio Henrique Salles; SILVA, João Melquíades Ferreira;
  BARROS Leandro Gomes de. \textbf{Feira de versos: poesia de cordel}.
  São Paulo: Ática, 2019.

Este livro é uma coleção de pérolas do cordel nacional. A obra reúne
textos de três importantes cordelistas: Leandro Gomes de Barros, o
primeiro a editar cordel no Brasil no século XIX, João Melquíades, que
apresenta \emph{O pavão misterioso} e Patativa do Assaré, cujos textos
vêm ganhando reconhecimento internacional.

  ALENCAR, Maria Silvana Militão de. \textbf{A linguagem regional
  popular na obra de Patativa do Assaré}. Fortaleza: Universidade
  Federal do Ceará, 1997.

Este estudo, central para compreender o emprego da variedade regional
nos cordéis de Patativa do Assaré, aproxima a literatura e a análise
sociolinguística.


  BARROSO, Oswald; BARBALHO, Alexandre. \textbf{Letras ao sol --
  antologia da literatura cearense.} Fortaleza: Fundação Demócrito
  Rocha, 1998.

A obra apresenta uma amostra representativa da poesia popular
nordestina, com ênfase na literatura do Ceará.


  FARIAS, Pedro Américo de. \textbf{Nordestinos: coletânea poética do
  Nordeste brasileiro}. Lisboa: Fragmentos, 1994.

A obra apresenta uma recolha da poesia popular nordestina, com temas
oriundos do folclore e da matéria histórica.


  FIGUEIREDO FILHO, J. de. \textbf{Patativa do Assaré: novos poemas
  comentados}. Ceará: Museu do Ceará, 2005.

A obra apresenta uma antologia comentada de poemas populares de Patativa
do Assaré, com rica contextualização sobre o autor e seu universo.


  HAURÉLIO, Marco. \textbf{Antologia do cordel brasileiro.} São Paulo:
  Global, 2012.

Nesta antologia, o leitor tem acesso a um leque variado de cordéis,
desde aqueles inspirados nos contos fantásticos e nos contos de fadas,
até outros em que predominam mitos da Grécia Antiga ou que deitam raízes
nas histórias de animais do fabulário mundial.



 \_\_\_\_\_. \textbf{Literatura de cordel: do sertão à sala de
  aula.} São Paulo: Paulus, 2013.

Declamados ou cantados, os cordéis levaram ao público, da tradição oral
ao contexto escolar, as façanhas dos cangaceiros Lampião e Antônio
Silvino, os milagres do Padre Cícero e outras narrativas populares.


  \_\_\_\_\_. \textbf{Breve história da literatura de cordel.} São
  Paulo: Claridade, 2018.

Esta obra apresenta as origens do Cordel, destaca seus principais
expoentes e mostra o leque de influências dessa tradição na cena
cultural brasileira.


  NASCIMENTO, Lourgeny Damasceno do. \textbf{A importância da literatura
  de cordel no cotidiano dos alunos da EJA}. Monografia apresentada ao
  Departamento de Artes da UNB. Brasília: 2011. (Disponível em:
  \url{https://bdm.unb.br/bitstream/10483/4463/1/2011_LourgenyDamascenodoNascimento.pdf}.
  Acesso em 18 de fevereiro de 2021.)

A autora destaca a relevância do trabalho com poemas em cordel no
trabalho com jovens e adultos, a partir do resgate das tradições
populares e da valorização dos saberes regionais.


 SUASSUNA, Ariano. \textbf{Romance da Pedra do Reino e o Príncipe do
  Sangue do Vai-e-volta}. Rio de Janeiro: Nova Fronteira, 2017.

O romance de Ariano Suassuna, publicado originalmente em 1971, narra a
história de Dom Pedro Dinis Ferreira, o Quaderna, apresentando seu
memorial de defesa perante o corregedor, com ressonâncias da tradição
literária do cordel.

  TAVARES, Braulio. \textbf{Contando histórias em versos: poesia e
  romanceiro popular no Brasil.} São Paulo: Editora 34, 2009.

O autor apresenta os principais recursos expressivos da linguagem
poética popular, enquanto introduz os leitores a rimas, ritmos, temas,
formas literárias e modos narrativos tipicamente brasileiros.

 

\end{document}


