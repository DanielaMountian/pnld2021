\documentclass[12pt]{extarticle}
\usepackage{manualdoprofessor}
\usepackage{fichatecnica}
\usepackage{lipsum,media9,graficos}
\usepackage[justification=raggedright]{caption}
\usepackage{bncc}
\usepackage[iluminuras]{../edlab}

 


\begin{document}


\newcommand{\AutorLivro}{Heinrich Heine}
\newcommand{\TituloLivro}{O Rabi de Bacherach}
\newcommand{\Tema}{Ficção, mistério e fantasia}
\newcommand{\Genero}{Romance}
\newcommand{\imagemCapa}{./images/PNLD0018-01.png}
\newcommand{\issnppub}{---}
\newcommand{\issnepub}{---}
% \newcommand{\fichacatalografica}{PNLD0018-00.png}
\newcommand{\colaborador}{\textbf{Michelle Etienne Florence, Bruno Gradella e Vicente Castro} é uma pessoa incrível e vai fazer um bom serviço.}


\title{\TituloLivro}
\author{\AutorLivro}
\def\authornotes{\colaborador}

\date{}
\maketitle

\baselineskip=1.15\baselineskip\par

\begin{abstract}

Este Manual tem como objetivo fornecer subsídios para o trabalho com a
obra literária \emph{O Rabi de Bacherach}, obra de autoria de Heinrich
Heine.

\textbf{Heinrich Heine}, nascido em 1797, é um dos maiores nomes da literatura alemã,
sendo conhecido como ``o último dos românticos''.
De família judaica, revelou-se inepto as carreiras do comércio e direito ainda 
jovem, interessando-se assim por assuntos literários. Seguiu, então, cursos de 
literatura, encontrando em Berlim um ambiente propício para frequentar os salões 
literários e seguir os cursos de filosofia política de Hegel.
Já famoso pelos poemas e livros de viagens, Heine tentou se firmar em várias
cidades alemãs, mas descontente com a situação antissemita no país, emigrou
para Paris em 1831. Lá encontrou a acolhida dos meios literários e passou a
viver confortavelmente, como correspondente de grandes jornais alemães.

A história de \textbf{O Rabi de Bacherach} se passa em Bacherach, cidade medieval 
de vida intensa e conflituosa, onde vive uma pequena comunidade judaica, liderada pelo
rabino Abraão. A história da opressão e perseguição aos judeus é narrada a
partir da trajetória do Rabi e sua esposa Sara em sua fuga para Frankfurt.
Fazem parte do enredo: nobres, alcaides, clero, cavaleiros, donzelas, assassinos, 
curandeiros, comerciantes, moças itinerantes, monges, rabinos, bêbados e bufões, 
todas personagens que, de alguma forma, permeiam a sociedade até os dias de hoje.
Escrita com a intenção de se contrapor ao crescente sentimento de antissemitismo
na Alemanha de sua época, o autor promove o intercâmbio e o diálogo entre as culturas 
alemã e judaica. Para isso, o narrador remonta também a séculos anteriores para tocar 
nas raízes históricas do antissemitismo no país.

Esperamos que as indicações propostas aqui sejam muito úteis no trabalho em
sala de aula! 



\end{abstract}

\tableofcontents

\section{Introdução}

Seja por uma fatalidade do destino – possível perda de capítulos incendiados, ou pela dificuldade em finalizar o pretensioso projeto – uma obra imortal sobre a grande dor judaica, Heinrich Heine nos instiga com seu “fragmento de romance”.

Desejoso de secularizar a cultura judaica, o jovem escritor propôs-se a criar uma epopeia sobre o povo judeu diante da escalada antissemita na alemanha do século xix.
Publicado em 1840, dezesseis anos após sua concepção inicial, o texto reflete as diversas facetas do autor:
ascendência judaica, lirismo poético, ironia, crítica à religião e índole revolucionária.

Heinrich Heine nasceu numa família judia assimilada. Seu pai era um comerciante que, durante a ocupação francesa, se beneficiou dos novos ideais de igualdade cívica para todos os cidadãos, em particular importante para os judeus, uma minoria discriminada nos territórios da atual alemanha. 
Quando o negócio do pai faliu, Heine foi enviado para Hamburgo, onde um tio Salomon, um rico banqueiro, financiou os estudos e encorajou-o a iniciar uma carreira comercial.

Em breve se tornou evidente que Heine não tinha um interesse na carreira comercial e assim, voltou-se para o estudo de direito. 
Descobriu também que estava menos interessado no direito do que na literatura, apesar de se ter licenciado em direito em 1825, ao mesmo tempo que decidiu converter-se do judaísmo para o cristianismo luterano,nomeando-se a si próprio oficialmente pelo nome de heinrich Heine.

Decidiu-se pela conversão considerando as várias proibições e restrições aos judeus, então vigentes em muitos estados alemães. O exercício de várias profissões e cargos em determinadas instituições, assim como o acesso a certas universidades eram proibidos aos judeus. 

Assim proclamou sua conversão como o ``bilhete de admissão na cultura européia'', apesar de a realidade ter sido bem diferente. Outra grande razão para a conversão de Heine foi o possível acesso que o teria ao mundos dos escritores rom nticos, em que a religião luterana e católica desempenhavam importante papel.

Como poeta, Heine fez a sua estreia em 1821. Heine trocou a alemanha por paris em 1831, onde sofreu influência dos socialistas utópicos, seguidores do conde saint-simon, cujo partido político intitula-se, em português, são-simonistas, um grupo que pregava um paraíso igualitário baseado na meritocracia.

A opção por Paris, a princípio, foi voluntária, pois Heine acreditava que encontraria na capital francesa maior liberdade de expressão e maior compreensão de suas ideias por parte da sociedade francesa, o que de fato aconteceu.

Seus escritos geraram desconforto nas autoridades alemãs e Heine foi tido como um subversivo e sofreu com a censura. Suas obras foram banidas da alemanha, assim como outros escritos associados ao movimento da jovem alemanha de 1835, liderado por Heine. 

O escritor foi então proibido de voltar a viver em sua terra natal e permaneceu exilado na França.

Embora no exílio, Heine sempre manteve uma profunda ligação com a alemanha, que se exercia através da crítica constante da situação política de seu país. A influência dos ideais franceses sobre seu espírito libertaram afinal em uma renovação da literatura alemã.

Heine foi um grande mediador dos aspectos culturais entre frança e alemanha e acreditava também em uma união entre a filosofia alemã e espírito revolucionário francês que culminariam na emancipação política e cultural da europa.

Foi um crítico mordaz das instituições religiosas. A famosa expressão que qualifica a religião como ``ópio do povo'' - havia sido adiantada por Heine. 

De fato, entre os livros queimados pelos nazistas, em 1933, na praça da ópera de berlim estavam as obras de Heine. Como ele próprio dissera, sociedades que queimam livros não demoram a queimar seres humanos.

Heine teve uma influência muito maior ao redor do mundo que na própria alemanha. Na frança, sua obra foi aclamada e o escritor chegou a receber uma pensão do governo francês. Também no japão e china foi admirado e na europa oriental foi tido com uma das grandes influências na formação de uma literatura nacional, assim como Goethe.

O romance histórico, gênero da obra em questão, é uma narrativa ficcional ambientada no passado e marcado pela influência de personagens e eventos históricos.


Heine traz estas referências, sob a forma de guarnições jornalísticas,
que antecedem o bloco de ações das personagens e contextualizam os
acontecimentos da narrativa. Assim, o panorama geral é em seguida
particularizado. Observam"-se tais aspectos no relato sobre a cidade
medieval de Bacherach e sobre as perseguições aos judeus no século \textsc{xiii}
que introduzem o primeiro capítulo; sobre a cidade comercial de Frankurt
e confinamento desse povo em guetos, quando da chegada de Sara e Abraão
a estas localidades; e sobre rituais e leitura do livro sagrado, no
momento em que o casal chega à Sinagoga.

\section{Atividades 1}

%\BNCC{EM13LP26}

\subsection{Pré"-leitura}

%\BNCC{EM13LGG302}
%\BNCC{EM13LGG704}
%\BNCC{EM13LP10}
%\BNCC{EM13LP19}

Para essa atividade, recomenda"-se a exibição de filmes, ou
trechos de filmes que retratem como era a vida dos judeus na Europa ao
longo dos séculos. Filmes como o Mercador de Veneza, Jojo Rabit, Esther,
entre outros ajudam os alunos a compreender as relações de
antissemitismo que o povo judeu sofreu durante boa parte de sua
história. Entretanto, o foco dessa atividade não deve ser unicamente na
história contada, mas também como se conta a história. Recomenda"-se ao
professor que indique aos alunos que prestem atenção nas maquiagens,
vestimentas, paletas de cores e ângulos de câmera. Afinal, todos eles
são elementos que compõe a narrativa, tanto quanto as trilhas sonoras ou
os diálogos. Feito isso, deve ser aberta uma roda para que os alunos
apresentem suas visões e indiquem o que perceberam, devendo também serem
estimulados a especificar porque escolheram aquelas determinadas
passagens.

\subsection{Leitura}

%\BNCC{EM13LGG103}
%\BNCC{EM13LP02}
%\BNCC{EM13LP48}

Ao longo da obra, encontramos diversos momentos em que o
autor se vale de figuras de linguagem, como hipérboles, ou de recursos
textuais como paráfrases, para criar um conteúdo jocoso e provocativo.
Este tipo de linguagem, em geral, tem a intenção de performar uma
crítica a algo ou alguém, representando"-o de forma caricata, exagerando
um ou diversos aspectos desta personalidade. Entretanto, é sabido que há
um limite tênue entre o que pode ser considerado uma crítica, uma
brincadeira, ou o que pode ser tomado por uma fala ofensiva e abusiva.
Recomenda"-se que, durante a leitura, os alunos façam uma análise desses
trechos problemáticos, anotando, segundo a opinião deles, quais
consideraram uma crítica menos problemática e quais consideraram mais
ofensivas.

\subsection{Pós"-leitura}

%\BNCC{EM13LGG102}
%\BNCC{EM13LGG303}
%\BNCC{EM13LGG402}
%\BNCC{EM13LGG703}
%\BNCC{EM13LP13}
%\BNCC{EM13LP14}
%\BNCC{EM13LP28}
%\BNCC{EM13LP29}
%\BNCC{EM13LP52}

Ainda atentos ao enunciado da atividade anterior,
recomenda"-se que os alunos busquem notícias de jornal onde pessoas foram
à justiça em razão de terem se sentido ofendidas por conta de
comentários de uma terceira pessoa. Recomenda"-se que seja realizado um
debate na sala acerca dos limites da liberdade de expressão e, como eles
podem interferir no combate ou promoção de discursos de ódio. Para esta
atividade, recomenda"-se, antes do debate, uma pesquisa e leitura e
interpretação de textos legais que defendem as liberdades individuais e
combatem os abusos.

\section{Atividades 2}

A obra \emph{O Rabi de Bacherach} possibilita trabalhos
interdisciplinares e integradores de diferentes campos do saber e áreas
de conhecimento. A seguir, propomos algumas atividades que podem ser
desenvolvidas conjuntamente com professores de outras áreas. Além das
habilidades de Linguagens e suas Tecnologias e de Língua Portuguesa,
indicadas nas etapas da seção anterior e válidas também para esta,
listamos a seguir as habilidades de outras áreas, presentes na abordagem
interdisciplinar:

%\BNCC{EM13CNT201}
%\BNCC{EM13CNT303}
%\BNCC{EM13CHS101}
%\BNCC{EM13CHS102}
%\BNCC{EM13CHS106}
%\BNCC{EM13CHS401}

\subsection{Pré"-leitura}
Para melhor contextualização, sugere"-se que o professor
estimule os alunos a buscar referências sobre a história do povo judeu,
com ênfase na sua formação, nos êxodos e nas grandes perseguições, desde
a Antiguidade até a Segunda Guerra Mundial.

É importante que os alunos entendam os motivos pelos quais os judeus
foram expulsos de suas terras e como ocorreu sua diáspora para a Europa.
E como lá, eles praticamente perpetuaram uma transumância continental ao
longo da Idade Média.

Além disso, até para melhor compreensão da obra, propomos que seja feita
uma pesquisa da versão bíblica das histórias de Moisés, Abraão e Jacó.
Posto que são muito referenciados no livro.

\subsection{Leitura}

Ampliando a atividade anterior, sugere"-se a pesquisa e
discussão de referências históricas de momentos de tolerância e combate
entre denominações religiosas diversas. Por exemplo: Na Polônia, em
1556, uma jovem judia foi morta sob a acusação de roubar hóstias para
práticas ocultas. Protestantes e católicos evitaram que outros judeus
fossem condenados e, após 10 anos, o parlamento polonês tornou ilegal
qualquer tribunal religioso. Em contrapartida, França e Inglaterra, em
uma época muito próxima, passaram por grandes conflitos religiosos, que
deixaram fortes cicatrizes nos seus respectivos solos, sendo inclusive
motivos pelos quais Voltaire e Locke escreveram tratados sobre a
tolerância. A atividade pode ser acompanhada dos professores de
humanidades, abordando aspectos sociológicos da religião e porque, há na
história, momentos de maior e de menor fanatismo religioso.
Recentemente, viu"-se um recrudescimento de movimentos religiosos
paramilitares. É importante que os alunos entendam a pluralidade de
motivos em torno disso.

\subsection{Pós"-leitura}

Concluída a leitura, sugerimos a pesquisa e discussão sobre
o papel da mulher na cultura judaica ao longo dos tempos. A partir
disso, reavaliar e discutir a trajetória da personagem Sara sob o
aspecto religioso, simbólico e lírico encontrado principalmente no
primeiro capítulo. É válida a comparação entre essa personagem, com as
demais personagens femininas presentes no fragmento de romance, bem como
de outras personagens femininas de outras histórias. Como elas são
apresentadas e como elas se relacionam com outros protagonistas de seus
enredos.

\section{Aprofundamento}

Ao chegar ao Ensino Médio, é necessário que os estudantes se aprofundem
na compreensão das múltiplas linguagens e, sobretudo, da linguagem
literária. Em relação à literatura, a \textsc{bncc} traz as seguintes
considerações:

\begin{quote}
{[}\ldots{}{]} a leitura do texto literário, que ocupa o centro do trabalho
no Ensino Fundamental, deve permanecer nuclear também no Ensino Médio.
Por força de certa simplificação didática, as biografias de autores, as
características de épocas, os resumos e outros gêneros artísticos
substitutivos, como o cinema e as \textsc{hq}s, têm relegado o texto literário a
um plano secundário do ensino. Assim, é importante não só (re)colocá"-lo
como ponto de partida para o trabalho com a literatura, como
intensificar seu convívio com os estudantes. Como linguagem
artisticamente organizada, a literatura enriquece nossa percepção e
nossa visão de mundo. Mediante arranjos especiais das palavras, ela cria
um universo que nos permite aumentar nossa capacidade de ver e sentir.
Nesse sentido, a literatura possibilita uma ampliação da nossa visão do
mundo, ajuda"-nos não só a ver mais, mas a colocar em questão muito do
que estamos vendo/vivenciando. (Brasil, 2018, p. 491)
\end{quote}

Nesta seção, desenvolvemos um trabalho de aprofundamento que, em diálogo
com a formação continuada de professores, oferece subsídios para a
abordagem do texto literário. A leitura em sala de aula de \emph{O Rabi
de Bacherach} pode ser enriquecida pelo aprofundamento no universo
literário em que a obra está inserida.

\Image{Retrato do autor, pintado em 1831 por Moritz Oppenheim (Moritz Oppenheim; Domínio público)}{PNLD0018-03.png}

O jovem escritor propôs"-se a criar uma epopeia sobre o
povo judeu diante da escalada antissemita na Alemanha do século \textsc{xix}.
Bacherach, cidade medieval de vida pujante e conflituosa, abriga uma
pequena comunidade judaica, liderada pelo rabino Abraão. A história da
opressão e perseguição aos judeus é narrada a partir da trajetória do
Rabi e sua esposa Sara. Publicado em 1840, dezesseis anos após sua
concepção inicial, o texto reflete as diversas facetas do autor:
ascendência judaica, lirismo poético, ironia, crítica à religião e
índole revolucionária. O tom exaltador, que descreve os costumes e
posterior fuga do casal judeu para Frankfurt, ganha uma complexidade
questionara dos dogmas, fanatismo, violência e relações de poder.
Nobres, alcaides, clero, cavaleiros, donzelas, assassinos, curandeiros,
comerciantes, moças itinerantes, monges, rabinos, bêbados e bufões são
personagens que, de alguma forma, permeiam a sociedade até os dias de
hoje. Captados por Heine, permitem espelhar ideologias da época do
autor, numa leitura engajada e, sobretudo, da condição humana. Decerto,
uma história que estará sempre por ser concluída.

\subsection{Por que ler \textit{O Rabi de Bacherach}?}

Seria convincente responder a essa questão relatando que Heinrich Heine
foi um autor admirado por Machado de Assis, Castro Alves e Álvares de
Azevedo? Tanto o prosador Heine quanto o poeta romântico mundialmente
reconhecido. Caso os olhos do leitor não tenham ainda brilhado,
argumentamos que o fragmento de romance permite viajar pela saga judaica
com matizes jornalísticas, poéticas, irônicas e paródicas. A prosa do
autor é refinada, rica em imagens e precisa.


Image{Retrato de Heine, desenhado em 1829 (Bibliothek des allgemeinen und praktischen Wissens. Bd. 5; Domínio público)}{PNLD0018-05.png}


Heine considerava"-se ``um pobre rouxinol alemão que fez seu ninho na
peruca de Monsieur Voltaire'' e ``um brioso soldado na luta de
libertação da humanidade''. Escrever sobre a filosofia e a história
judaicas era o objetivo do Iluminismo Judaico, corrente intelectual da
qual o autor fazia parte por volta de 1824. Entretanto, o olhar sobre a
dor do povo judeu ganhou uma perspectiva universal nos dois últimos
capítulos, finalizados em 1840. Comprometido com o projeto de
emancipação dos povos subjugados, o escritor preocupava"-se com a espiral
de violência em que oprimido transforma"-se em opressor. Esse
amadurecimento ideológico transforma as páginas do Rabi, como costumava
chamar sua obra, e é digno de ser apreciado também em sua poética.


\Image{Monumento em homenagem ao autor, localizado em Brocken, na Alemanha (Stefan Schäfer, Lich ; CC-BY-SA-4.0)}{PNLD0018-04.png}



\subsection{O épico e o lírico na exaltação do povo judeu}

O primeiro capítulo do Rabi de Bacherach, escrito em 1824, difere dos
demais por seus traços nitidamente épicos e líricos. Isso se deve ao
estilo do autor à época e à proposta de dar um tratamento literário à
história do povo judeu.

Aproxima"-se do romance épico na medida em que os protagonistas
personificam o caráter heroico dos judeus. Abraão e Sara remetem ao
casal homônimo bíblico, cujos inúmeros descendentes herdariam a terra
prometida. O próprio Abraão das escrituras é considerado o fundador do
judaísmo. O rabino e sua esposa encarnam traços universais de sua
cultura e destinam"-se à perseguição. O êxodo hebreu do Egito é
oportunamente recontado quando da celebração da Páscoa Judaica pelo
Rabi. Notam"-se ainda nessa solenidade descrições eloquentes, ressaltadas
pelo uso de verbos no presente e adjetivações de riqueza e beleza. O
leitor entra, então, em contato narrativas imemoriais e tem empatia pela
dor e luta desse povo.


\Image{Ilustração para o livro, feito por Max Liebermann para edição de 1922. (Max Liebermann; Domínio público)}{PNLD0018-06.png}


O lirismo e a sinestesia dão voz ao estilo romântico que conduz a fuga
de Abraão e Sara pelo rio Reno. Nesse sentido, a paisagem adquire
atributos humanizados e assustadores que potencializam o pavor e a
violência, revelando o íntimo das personagens. A sucessão de composições
pictóricas culmina com uma atmosfera onírica, em que lembranças se
misturam a fragmentos de contos de fadas e mitos, referências muito
resgatadas no Romantismo. Sara toma uma atitude lânguida e devaneadora,
a exemplo das heroínas românticas.

\subsection{A ironia como elemento niilista}

A ironia permeia a obra como um todo, mas obtém destaque nos dois
últimos capítulos, quando instrumentaliza a crítica às incoerências
sociais, religiosas e políticas. Isso se dá sem formar juízo valorativo
e deixando um vazio de significado, traços do niilismo heineano.

No início, a construção de abadias e igrejas em nome de São Werner marca
uma situação irônica em sua base: edificações gloriosas cristãs que
trazem em sua história assassinatos, tanto do menino canonizado, quanto
dos judeus acusados injustamente.

Já em Frankfurt, existem inúmeros exemplos de ironia advindos da
observação à distância pelo narrador: Sara foi compelida a rir da luta
dos judeus com as ratazanas, entretanto estas representam seus próprios
algozes; o desfile de tipos sociais e seus símbolos, paradoxais entre
si, como médicos, cavaleiros, prostitutas, monges, caçadores, com
destaque do profano em relação ao religioso; Hans, o católico tocador de
tambor, cita santos, satanás, catecismo, mas demonstra atitudes
distanciadas do discurso; o comportamento das mulheres, fofoqueiras e
vaidosas, em plena celebração da Páscoa Judaica; o cavaleiro espanhol
elogia a seriedade e determinação do Rabi e ao mesmo tempo revela que
este flertava com uma moça na Espanha, época em que já era um homem
casado. Em tom cômico, um dos guardiões do gueto judeu elogia o nariz
avantajado do companheiro, característica étnica desse povo.

\subsection{A crítica caricaturesca e jocosa }

Os indivíduos encontrados no gueto e cercania são caricatos em suas
vestes, atributos físicos e psíquicos, com a intenção de subverter a
lógica para contestar dogmas e fé, e denunciar o preconceito e a
violência.

Do lado de fora anunciando quem se aproxima, há um cristão tocador de
tambor, que tem a voz de cerveja e um traje com adornos que lembravam
línguas. Da sua boca e, metaforicamente, das línguas teciduais, saem
palavras gargarejadas sempre a ressuscitar os conflitos entre os judeus
e cristãos.

Os soldados, supostamente defensores do local, estão deitados na torre
de vigia. Tratam"-se de Nasenstern (ou Stern), um sujeito medroso, tísico
e narigudo, e Jakel, o Tolo, homenzinho atarracado, de pernas tortas, e
cara vermelha e risonha, que lembra um bufão. Jakel debocha de Stern, ao
sugerir que a linhagem dos parentes do covarde chega até os tempos do
rei Saul, sendo os primeiros a debandarem do campo de batalha.

Ao receber o Abraão e Sara, Jakel canta uma canção infantil repetitiva e
cheia de diminutivos, conferindo um caráter patético à situação. Na
cantiga original o relâmpago, que representa Deus, destrói o anjo da
Morte (turcos) e salva o cabrito (judeus). No entanto, o Tolo interrompe
a música antes disso, no trecho em que o anjo da Morte acaba com o
carniceiro (cristãos), e afirma que o Deus dos judeus é o Deus da
vingança.


\Image{Quadro ``Abraão e Sarah visitados por três anjos'' (National Trust; Domínio público)}{PNLD0018-10.png}


Cabe ao Rabi fazer, então, uma resenha de toda cena, dizendo à esposa:
``é esta a proteção que tem em Israel! Por fora falsos amigos guardam
seus portões e por dentro seus guardiães são a tolice e o temor!''

No terceiro capítulo aparece Don Isaak, cavaleiro espanhol, donjuanesco
e cristão convertido, cujas penas do barrete ``movimentavam"-se mais com
o elegante balanço da cabeça do que pelo sopro do vento''. Ao rodear
Sara, jura a ela sua devoção, misturando flores com cebolas e ervilhas,
e mulas e cabritos com velhos cristãos. Segue com revelações de sua
índole pagã: ``eu amo vossa cozinha muito mais que vossa crença; a esta
falta o tempero certo'' e ``os sombrios nazarenos me são tão repulsivos
como os ressequidos hebreus sem alegria''.


\Image{Ilustração para o livro, feita por Josef Budko para edição de 1921. (Josef Budko; Domínio público)}{PNLD0018-07.png}


\Image{Ilustração de Josef Budko. (Josef Budko; Domínio público)}{PNLD0018-08.png}


\Image{Ilustração de Josef Budko. (Josef Budko; Domínio público)}{PNLD0018-09.png}


\subsection{A paráfrase e a paródia do sagrado}

A utilização de textos tradicionais como base para novas criações
artísticas potencializa o alcance semântico do autor que os utiliza. Na
paródia, há a perversão/contestação do sentido original, e na paráfrase
esta relação é neutra ou similar.

Heine utiliza tais estilizações, principalmente de fatos e trechos dos
livros sagrados judaicos e cristãos, com o intuito de multiplicar a
acepção dos mesmos e questionar a sociedade, a hierarquia e a religião.

Nota"-se o fenômeno parodístico em: Sara, em seu devaneio, vê cenas
bíblicas desvirtuadas, em que Moisés é enfrentado pelo egípcio Mitzri, o
Monte Sinai arde em chamas, o Rei Faraó nada no Mar Vermelho; Stern
choraminga e comprime o rosto contra o muro, numa alusão deturpada da
reza no Muro da Lamentações; Tolo graceja da passagem bíblica sobre o
sacrifício de Isaac, que, se levado a cabo, geraria ``mais cabritos e
menos judeus''; e Dom Isaak pede perdão a deusa sírio"-fenícia Astarte
por se ajoelhar e orar diante ``da dolorosa mãe do crucificado'',
referindo"-se à conhecida imagem da Virgem Maria carregando o corpo de
cristo.

A paráfrase: ``Entre tantos cães o coelho está perdido'' remete à
expressão alemã: ``Muitos cachorros são a morte do coelho'', em que os
coelhos representam os judeus, e os cachorros, os cristãos.

\section{Sugestões de atividades complementares: relações dialógicas e
intertextuais}

%\BNCC{EM13LP03}
%\BNCC{EM13LP04}
%\BNCC{EM13LP49}
%\BNCC{EM13LP51}

No Ensino Médio, da mesma forma que no Ensino Fundamental, a \textsc{bncc}
organiza o trabalho com as práticas de linguagem em cinco \textbf{campos
de atuação social}. São eles: campo da vida pessoal, campo da vida
pública, campo jornalístico"-midiático, campo artístico"-literário e campo
das práticas de estudo e pesquisa.

De acordo com essa divisão, propomos na sequência um trabalho
interdiscursivo e intertextual com a obra \emph{O Rabi de Bacherach}.

\subsection{Campo da vida pessoal}

\begin{quote}
O campo da vida pessoal pretende funcionar como espaço de articulações
e sínteses das aprendizagens de outros campos postas a serviço dos
projetos de vida dos estudantes. As práticas de linguagem privilegiadas
nesse campo relacionam"-se com a ampliação do saber sobre si, tendo em
vista as condições que cercam a vida contemporânea e as condições
juvenis no Brasil e no mundo.

Está em questão também possibilitar vivências significativas de práticas
colaborativas em situações de interação presenciais ou em ambientes
digitais e aprender, na articulação com outras áreas, campos e com os
projetos e escolhas pessoais dos jovens, procedimentos de levantamento,
tratamento e divulgação de dados e informações e o uso desses dados em
produções diversas e na proposição de ações e projetos de natureza
variada, para fomentar o protagonismo juvenil de forma
contextualizada. (\textsc{bncc}, p. 494)
\end{quote}

Perseguições institucionalizadas são um triste capítulo da história da
humanidade. Frequentes no passado, infelizmente, muitas ainda
acontecem até hoje. Entretanto, ainda que documentadas e divulgadas,
elas acabam registrando um fenômeno macro. O desrespeito ao ser humano
acontece também no universo micro, nas relações interpessoais.
Bullying e assédio moral são alguns exemplos de abusos psicológicos
que algumas pessoas sofrem em seus ambientes escolares, residenciais e
de trabalho. Em uma época de grande preocupação com a saúde mental dos
indivíduos, é interessante traçar paralelos, por meio das artes e da
historiografia, entre os registros de experiências de abusos que os
autores registraram no passado\textbf{. Nesse exercício, os alunos
podem constatar a tendência crescente na humanidade de respeito ao
próximo. Mas, também, podem identificar falhas ainda a serem
corrigidas. O resultado dessa pesquisa pode ser divulgado por meio de}
um \textbf{minidocumentário de divulgação científica} sobre os
impactos do comportamento abusivo no psicológico da vítima. Os alunos
podem produzir tanto um texto, quanto um podcast ou um pequeno vídeo.
Para isso, recomenda"-se a utilização de aplicativos gratuitos de
gravação e edição de vídeos, disponíveis em dispositivos digitais,
para montar um vídeo de até 4 minutos sobre o tema. É possível
utilizar desenhos e animações, formato de telejornal ou de
\emph{vlog}, com cuidados para garantir qualidade de áudio e imagem.
Reforce a importância de haver um recorte histórico"-cultural do tema,
passando pelas artes visuais e pela literatura.

\subsection{Campo de atuação na vida pública}

\begin{quote}
No cerne do campo de atuação na vida pública estão a ampliação da
participação em diferentes instâncias da vida pública, a defesa dos
direitos, o domínio básico de textos legais e a discussão e o debate de
ideias, propostas e projetos. {[}\ldots{}{]}

Ainda no domínio das ênfases, indica"-se um conjunto de habilidades que
se relacionam com a análise, discussão, elaboração e desenvolvimento de
propostas de ação e de projetos culturais e de intervenção social.
(\textsc{bncc}, p. 494)
\end{quote}

Conforme dito na atividade anterior, condutas abusivas podem se fazer
presentes não só no plano individual, mas também entre grupos sociais
distintos. Daí, tem"-se perseguições étnicas, religiosas, etc.
Recentemente, o conflito na Síria escancarou esse tipo de situação.
Recomenda"-se orientar os alunos a realizarem uma pesquisa sobre
migrações forçadas nos tempos recentes. Como as ocorridas após e
durante as Grandes Guerras Mundiais, após a descolonização de África e
Ásia, durante as Guerras de Dissolução da Iugoslávia, etc.
Aconselha"-se que os alunos nessa atividade trabalhem com dados e mapas
e procurem obter informações sobre as causas do conflito.
Posteriormente, devem montar um quadro mural com as informações que
pode vir a ser exibido no colégio.

\subsection{Campo jornalístico"-midiático}

\begin{quote}
Em relação ao campo jornalístico"-midiático, espera"-se que os jovens
que chegam ao Ensino Médio sejam capazes de: compreender os fatos e
circunstâncias principais relatados; perceber a impossibilidade de
neutralidade absoluta no relato de fatos; adotar procedimentos básicos
de checagem de veracidade de informação; identificar diferentes pontos
de vista diante de questões polêmicas de relevância social; avaliar
argumentos utilizados e posicionar"-se em relação a eles de forma ética;
identificar e denunciar discursos de ódio e que envolvam desrespeito aos
Direitos Humanos; e produzir textos jornalísticos variados, tendo em
vista seus contextos de produção e características dos gêneros. Eles
também devem ter condições de analisar estratégias
linguístico"-discursivas utilizadas pelos textos publicitários e de
refletir sobre necessidades e condições de consumo.

No Ensino Médio, os jovens precisam aprofundar a análise dos interesses
que movem o campo jornalístico midiático, da relação entre informação e
opinião, com destaque para o fenômeno da pós"-verdade, consolidar o
desenvolvimento de habilidades, apropriar"-se de mais procedimentos
envolvidos na curadoria de informações, ampliar o contato com projetos
editoriais independentes e tomar consciência de que uma mídia
independente e plural é condição indispensável para a democracia.

Como já destacado, as práticas que têm lugar nas redes sociais têm
tratamento ampliado. (\textsc{bncc}, p. 494-495)
\end{quote}

Durante muito tempo na história, os relatos de uma guerra eram aqueles
contados pelos sobreviventes no campo de batalha, frequentemente os
vencedores. É somente no século \textsc{xix} que, com advento da fotografia e
de meios de telecomunicação como o telégrafo, o telefone e, por fim, a
internet, que o mundo passou a receber notícias de conflitos armados
concomitantemente à sua ocorrência. Isso colocou em xeque as
narrativas oficiais, uma vez que as imagens, por vezes, contestavam o
que estava sendo oficialmente descrito. Um exemplo presente é a Guerra
do Paraguai (1864-1870), em que as pinturas encomendadas pelo governo
brasileiro descreviam uma guerra heróica, ao passo que as fotografias
revelavam uma narrativa muito mais nefasta. Para essa atividade, os
alunos devem criar um podcast de poucos minutos (aconselha"-se o máximo
de seis minutos), cujo conteúdo deva ser jornalismo de guerra. Na sua
realização, os alunos devem se valer de notícias e depoimentos de
conflitos recentes. É interessante uma pesquisa também de como esses
relatos do local divergem, por vezes dos relatos oferecidos pelos
governos beligerantes.

\subsection{Campo artístico"-literário}

\begin{quote}
No campo artístico"-literário busca"-se a ampliação do contato e a
análise mais fundamentada de manifestações culturais e artísticas em
geral. Está em jogo a continuidade da formação do leitor literário e do
desenvolvimento da fruição. A análise contextualizada de produções
artísticas e dos textos literários, com destaque para os clássicos,
intensifica"-se no Ensino Médio. Gêneros e formas diversas de produções
vinculadas à apreciação de obras artísticas e produções culturais
(resenhas, vlogs e podcasts literários, culturais etc.) ou a formas de
apropriação do texto literário, de produções cinematográficas e teatrais
e de outras manifestações artísticas (remidiações, paródias,
estilizações, videominutos, fanfics etc.) continuam a ser considerados
associados a habilidades técnicas e estéticas mais refinadas.

A escrita literária, por sua vez, ainda que não seja o foco central do
componente de Língua Portuguesa, também se mostra rica em possibilidades
expressivas. (\textsc{bncc}, p. 495-496).
\end{quote}

Nas atividades desse setor, procurou"-se tratar de abusos, conflitos,
pois essa é uma narrativa presente no livro O Rabi de Bacherach. A
iconografia da dor, do conflito, do êxodo, foi muito narrada,
desenhada, fotografada e roteirizada. Talvez, uma das pinturas mais
famosas de todas seja a obra Güernica do Espanhol Pablo Picasso. Nela,
retrata"-se o bombardeio realizado na pequena cidade de Güernica,
durante a Guerra Civil Espanhola. Apresente essa pintura aos alunos.
Peça para que eles indiquem as emoções que a mesma desperta, que
descrevam o que acontece na cena. Indique que a expressão dos
sentimentos não precisa adotar a forma realista. Posteriormente, peça
para que eles selecionem alguma passagem do livro e produzam um
desenho retratando algum trecho do livro lido. Como leitura
complementar para a busca de referências para essa atividade,
recomenda"-se, além das pinturas dopróprio Picasso, as pinturas de Marc
Chagall e Edvard Much, além da leitura de trechos do livro \emph{O
Exército de Cavalaria}, de Isaac Bábel.

\subsection{Campo das práticas de estudo e pesquisa}

\begin{quote}
O campo das práticas de estudo e pesquisa mantém destaque para os
gêneros e habilidades envolvidos na leitura/escuta e produção de textos
de diferentes áreas do conhecimento e para as habilidades e
procedimentos envolvidos no estudo. Ganham realce também as habilidades
relacionadas à análise, síntese, reflexão, problematização e pesquisa:
estabelecimento de recorte da questão ou problema; seleção de
informações; estabelecimento das condições de coleta de dados para a
realização de levantamentos; realização de pesquisas de diferentes
tipos; tratamento dos dados e informações; e formas de uso e
socialização dos resultados e análises.

Além de fazer uso competente da língua e das outras semioses, os
estudantes devem ter uma atitude investigativa e criativa em relação a
elas e compreender princípios e procedimentos metodológicos que orientam
a produção do conhecimento sobre a língua e as linguagens e a formulação
de regras. (\textsc{bncc}, p. 495-496)
\end{quote}

Durante muito tempo, o direito a cidadania era conferido a poucas
pessoas. Na Atenas Clássica, era oferecida apenas aos homens livres,
maiores de idade, descendentes de jônicos, filhos de pais e mães
atenienses. Na Europa, judeus não eram considerados como pertencentes
ao povo de alguns países onde se encontravam, mas eram vistos como
estrangeiros, ainda que há gerações vivessem no país. Houve, também,
uma decisão judicial da Suprema Corte americana, indicando que o país
não entendia os negros como seus cidadãos, ainda que, também,
estivessem há muitas gerações no país. A questão da cidadania, de como
ela é determinada é poder exclusivo do Estado. Sendo ela um fator que
implica direitos e deveres, é importante que o tema seja debatido em
sala de aula. Para tanto, propõe"-se duas etapas. Primeiramente, os
alunos devem ser divididos em grupo, fazendo uma breve pesquisa sobre
as definições de Nação, Estado, Governo e Território. Feito isso,
devem, em uma segunda etapa, buscarem informações sobre nacionalidade.
As diferenças entre os conceitos de \emph{ius sanguini} e \emph{ius
soli}, o entendimento da Organização das Nações Unidas sobre a
questão. Por fim, munidos disso, o professor deve propor um debate
acerca da situação de refugiados, de apátridas e de grandes nações
desprovidas de Estados. Nessa atividade, os alunos devem pensar em
soluções possíveis, à luz dos direitos humanos, para essas questões.

\section{Referências complementares}

\subsection{Livro}

\begin{itemize}
\item\textsc{boyne}, John. \textit{O menino do pijama listrado}. São Paulo: Editora
Seguinte, 2007.

O menino do pijama listrado é uma fábula sobre amizade em tempos de
guerra, e sobre o que acontece quando a inocência é colocada diante de
um monstro terrível e inimaginável.
\end{itemize}

\subsection{Filme}

\begin{itemize}
\item\textit{Um Violinista no Telhado}. Direção: Norman Jewison (Holanda, 1971).

Um leiteiro judeu pretende arranjar o casamento de suas duas filhas,
porém ambas se recusam e uma delas decide casar com um não judeu. O
leiteiro debate"-se nesta situação delicada quando um decreto do Czar
obriga todos os judeus a abandonar a aldeia, condenando a sua família ao
exílio e à dispersão.
\end{itemize}

\section{Bibliografia comentada}

\begin{itemize}
\item\textsc{carpeaux}, Otto Maria. \textit{História da literatura ocidental}. São
Paulo: Leya, 2019.

Uma coleção que nos leva ao encontro dos mais importantes autores da
literatura ocidental, numa obra imprescindível para qualquer pessoa que
se proponha a estudar verdadeiramente o tema.

\item\textsc{carpeaux}, Otto Maria. \textit{A história concisa da literatura alemã}.
Barueri: Faro Editorial, 2013.

Neste livro encontra"-se uma síntese dos grandes momentos, livros e
autores da literatura alemã, bem como sua importância para o a cultura e
o desenvolvimento do país e do mundo contemporâneo.

\item\textsc{safranski}, Rudiger. \textit{Romantismo: uma questão alemã}. São
Paulo: Estação Liberdade, 2012.

A obra divide"-se em duas partes: ``O romantismo'' e ``O romântico?'' Na
primeira parte, o autor aponta as origens do movimento. Na segunda, há a
investigação do ``romântico como atitude espiritual'' até a chegada do
socialismo nacional alemão.

\item\textsc{hauser}, Arnold. \textit{História social da arte e da literatura}. São
Paulo: Martins Fontes, 2010.

O valor desta obra consiste no fato de que Hauser, fundamentado em um
conhecimento preciso de fontes e literatura especializada, reúne
resultados claros da sociologia da arte, da música e da literatura.
\end{itemize}


\end{document}

