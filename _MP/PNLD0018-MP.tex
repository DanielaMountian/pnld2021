\documentclass[12pt]{extarticle}
\usepackage{manualdoprofessor}
\usepackage{fichatecnica}
\usepackage{lipsum,media9,graficos}
\usepackage[justification=raggedright]{caption}
\usepackage[one]{bncc}
\usepackage[iluminuras]{../edlab}


\begin{document}


\newcommand{\AutorLivro}{Heinrich Heine}
\newcommand{\TituloLivro}{O Rabi de Bacherach}
\newcommand{\Tema}{Ficção, mistério e fantasia}
\newcommand{\Genero}{Romance}
\newcommand{\imagemCapa}{./images/PNLD0018-01.png}
\newcommand{\issnppub}{---}
\newcommand{\issnepub}{---}
% \newcommand{\fichacatalografica}{PNLD0018-00.png}
\newcommand{\colaborador}{{Michelle Etienne Florence, Bruno Gradella e Vicente Castro}}


\title{\TituloLivro}
\author{\AutorLivro}
\def\authornotes{\colaborador}

\date{}
\maketitle

\baselineskip=1.15\baselineskip\par

\begin{abstract}\addcontentsline{toc}{section}{Carta ao professor}

Este manual tem como objetivo fornecer subsídios para o trabalho com a
obra literária \emph{O Rabi de Bacherach}, obra de autoria de Heinrich
Heine.

Heinrich Heine (1797--1856) é um dos maiores nomes da literatura alemã,
sendo conhecido como ``o último dos românticos''.
De família judaica, revelou"-se inepto às carreiras do comércio e direito ainda 
jovem, interessando"-se por assuntos literários. Seguiu, então, cursos de 
literatura, encontrando em Berlim um ambiente propício para frequentar os salões 
literários e seguir os cursos de filosofia política de Hegel.
Já famoso pelos poemas e livros de viagens, Heine tentou se firmar em várias
cidades alemãs, mas descontente com a situação antissemita no país, emigrou
para Paris em 1831. Lá encontrou a acolhida dos meios literários e passou a
viver confortavelmente, como correspondente de grandes jornais alemães.

A história de \textit{O Rabi de Bacherach} se passa em Bacherach, cidade medieval 
de vida intensa e conflituosa, onde vive uma pequena comunidade judaica, liderada pelo
rabino Abraão. A história da opressão e perseguição aos judeus é narrada a
partir da trajetória do Rabi e sua esposa Sara em sua fuga para Frankfurt.
Fazem parte do enredo: nobres, alcaides, clero, cavaleiros, donzelas, assassinos, 
curandeiros, comerciantes, moças itinerantes, monges, rabinos, bêbados e bufões, 
todas personagens que, de alguma forma, permeiam a sociedade até os dias de hoje.
Escrita com a intenção de se contrapor ao crescente sentimento de antissemitismo
na Alemanha de sua época, o autor promove o intercâmbio e o diálogo entre as culturas 
alemã e judaica. Para isso, o narrador remonta também a séculos anteriores para tocar 
nas raízes históricas do antissemitismo no país.

Esperamos que as indicações propostas aqui sejam muito úteis no trabalho em
sala de aula! 


\end{abstract}

\tableofcontents

\section{Introdução}

Seja por uma fatalidade do destino, possível perda de capítulos 
incendiados ou pela dificuldade em finalizar o pretensioso projeto --- uma 
obra imortal sobre a \emph{grande dor judaica} --- Heinrich Heine nos 
instiga com seu ``fragmento de romance''.

Desejoso de secularizar a cultura judaica, o jovem escritor propôs"-se 
a criar uma epopeia sobre o povo judeu diante da escalada antissemita 
na Alemanha do século \textsc{xix}.
Publicado em 1840, dezesseis anos após sua concepção inicial, 
o texto reflete as diversas facetas do autor:
ascendência judaica, lirismo poético, ironia, crítica 
à religião e índole revolucionária.

Heinrich Heine nasceu numa família judia assimilada. Seu pai era 
um comerciante que, durante a ocupação francesa, se beneficiou dos 
novos ideais de igualdade cívica para todos os cidadãos, 
em particular importante para os judeus, uma minoria 
discriminada nos territórios da atual Alemanha. 
Quando o negócio do pai faliu, Heine foi enviado 
para Hamburgo, onde um tio Salomon, um rico banqueiro, 
financiou os estudos e encorajou"-o a iniciar uma carreira comercial.

Em breve se tornou evidente que Heine não tinha um interesse 
na carreira comercial e assim, voltou"-se para o estudo de Direito. 
Descobriu também que estava menos interessado no direito do que 
na literatura, apesar de se ter licenciado em direito em 1825, 
ao mesmo tempo que decidiu converter"-se do judaísmo para o 
cristianismo luterano, nomeando"-se a si próprio oficialmente pelo nome de Heinrich Heine.

Decidiu"-se pela conversão considerando as várias proibições e 
restrições aos judeus, então vigentes em muitos estados alemães. 
O exercício de várias profissões e cargos em determinadas 
instituições, assim como o acesso a certas universidades eram proibidos aos judeus. 

Assim proclamou sua conversão como o ``bilhete de admissão na cultura europeia'', 
apesar de a realidade ter sido bem diferente. Outra grande 
razão para a conversão de Heine foi o possível acesso que teria 
ao mundo dos escritores românticos, em que a religião luterana e 
católica desempenhavam importante papel.

Como poeta, Heine fez a sua estreia em 1821. Ele trocou a Alemanha 
por Paris em 1831, onde sofreu influência dos socialistas 
utópicos, seguidores do Conde Saint"-Simon, cujo partido político 
intitula"-se, em português, são"-simonistas, um grupo que pregava 
um paraíso igualitário baseado na meritocracia.

A opção por Paris, a princípio, foi voluntária, pois Heine 
acreditava que encontraria na capital francesa maior 
liberdade de expressão e maior compreensão de suas ideias 
por parte da sociedade francesa, o que de fato aconteceu.

Seus escritos geraram desconforto nas autoridades alemãs e 
Heine foi tido como um subversivo e sofreu com a censura. 
Suas obras foram banidas da Alemanha, assim como outros 
escritos associados ao movimento da jovem Alemanha de 1835, liderado por Heine. 

O escritor foi então proibido de voltar a viver em sua terra natal e 
permaneceu exilado na França. Embora no exílio, Heine sempre 
manteve uma profunda ligação com a Alemanha, que se 
exercia através da crítica constante da situação política 
de seu país. A influência dos ideais franceses sobre seu 
espírito libertaram afinal em uma renovação da literatura alemã.

Heine foi um grande mediador dos aspectos culturais entre 
França e Alemanha e acreditava também em uma união entre 
a filosofia alemã e espírito revolucionário francês 
que culminariam na emancipação política e cultural da Europa.

Foi um crítico mordaz das instituições religiosas. 
A famosa expressão que qualifica a religião 
como ``ópio do povo'' havia sido adiantada por Heine. 

De fato, entre os livros queimados pelos nazistas, em 
1933, na Praça Bebel, em Berlim, estavam as obras 
de Heine. Como ele próprio dissera, ``sociedades 
que queimam livros não demoram a queimar seres humanos''.

Heine teve uma influência muito maior ao redor do 
mundo que na própria Alemanha. Na França, sua obra foi 
aclamada e o escritor chegou a receber uma pensão 
do governo francês. Também no Japão e na China 
foi admirado e na Europa oriental foi tido com uma 
das grandes influências na formação de uma literatura nacional, assim como Goethe.

O romance histórico, gênero da obra em questão, é 
uma narrativa ficcional ambientada no passado 
e marcado pela influência de personagens e eventos históricos. 
Heine traz estas referências, sob a forma de guarnições jornalísticas,
que antecedem o bloco de ações das personagens e contextualizam os
acontecimentos da narrativa. Assim, o panorama geral é em seguida
particularizado. Observam"-se tais aspectos no relato sobre a cidade
medieval de Bacherach e sobre as perseguições aos judeus no século \textsc{xiii}
que introduzem o primeiro capítulo; sobre a cidade comercial de Frankurt
e confinamento desse povo em guetos, quando da chegada de Sara e Abraão
a estas localidades; e sobre rituais e leitura do livro sagrado, no
momento em que o casal chega à sinagoga.


\Image{Retrato do autor, pintado em 1831 por Moritz Oppenheim (Moritz Oppenheim; Domínio público)}{PNLD0018-03.png}


\section{Propostas de Atividades I}

\subsection{Pré"-leitura}

\paragraph{Tema} O preconceito contra judeus.

\paragraph{Conteúdo} Cine"-debate sobre antissemitismo e a vida 
dos judeus na Europa.

\paragraph{Objetivo} Compreensão do antissemitismo e da figura do judeu 
através da exibição de filmes.

\paragraph{Justificativa} O antissemitismo tem raízes antigas, e a forma mais simples 
de conhecê"-las e compreendê"-las é tendo contato com a cultura judaica e as 
particularidades vividas por judeus durante as épocas --- buscando assim entender a 
tolerância que se alinha no amplo movimento
para a emancipação dos povos oprimidos. A grande perseguição aos judeus 
começou com as cruzadas e recrudesceu da maneira mais feroz por volta 
da metade do século \textsc{xiv}, ao final da grande peste que, como toda 
desgraça pública, teria sido provocada pelos judeus, uma vez que se 
afirmava terem eles desencadeado a ira de Deus e envenenado 
as fontes com ajuda dos leprosos.

O cinema, que articula através de linguagem direta
situações dinâmicas e complexas, é um ótimo 
meio para mergulhar no tema com proximidade suficiente a fim de 
entender essa questão profundas. Por isso indicamos que a primeira 
atividade de pré"-leitura seja através de filmes: reunimos 
excelentes sugestões para você abaixo.

\paragraph{Metodologia}

\begin{enumerate}
\item O professor ou professora deve selecionar um filme que retrate 
a vida dos judeus na Europa, entre os séculos \textsc{xix} e \textsc{xx}. Sugerimos, por exemplo,
\emph{Jojo Rabbit}, \emph{A Vida é Bela}, \emph{O pianista}, 
\emph{Um violinista no telhado} ou \emph{A lista de Schindler}. 

\item Antes da exibição, recomenda"-se ao
professor ou professora que indique aos alunos prestarem atenção nas 
vestimentas, características sociais, locações, trilha sonora e línguas 
faladas no filme. 
Afinal, todos os elementos que compõe a narrativa são importantes, 
ainda mais quando falamos sobre análises culturais. Ou seja, o foco 
dessa atividade não deve ser unicamente na
história contada, mas também em como se conta a história --- e em como é feito 
o retrato de um povo através dos olhos do cineasta. Dessa forma, o estudante
conseguirá compreender e atentar a detalhes importantes do universo judaico.
\BNCC{EM13CHS104}

\item Feito isso, deve ser aberto um debate com os alunos, para que
apresentem suas visões e indiquem o que perceberam. Eles devem também ser
estimulados a especificar porque observaram as tais determinadas
passagens do filme. Você pode se pautar por algumas questões, como:

\begin{itemize}
\item O filme apresenta ações cotidianas específicas ligadas à tradição judaica? 
Como por exemplo alimentação, casamentos, enterros, formas de comunicação e 
línguas utilizadas, relação com o Estado?
\item Quais são os ritos religiosos e culturais observados?
\item O filme retrata o antissemitismo sofrido pelos judeus? De que forma?
\item Em qual época o filme é ambientado? O preconceito contra judeus 
mudou ao longo do tempo?
\item Como as privações afetavam a vida dos judeus? É possível fazer um 
paralelo com outros povos ou culturas?
\end{itemize}
\end{enumerate}
\BNCC{EM13LGG303}

\paragraph{Tempo estimado} Três aulas de 50 minutos, caso a 
exibição do filme seja feita em sala de aula.


\subsection{Leitura}

\paragraph{Tema} O teor provocativo expresso através de recursos literários.

\paragraph{Conteúdo} Análise da obra \emph{O Rabi de Bacherach}, 
com foco na crítica do autor ao antissemitismo com o uso de recursos textuais.

\paragraph{Objetivo} Proporcionar aos estudantes uma leitura ativa do livro
e compreensão dos meios empregados pelo autor para tratar do antissemitismo.

\paragraph{Justificativa} Ao longo da obra, encontramos diversos momentos em que o
autor se vale de figuras de linguagem, como hipérboles, ou de recursos
textuais como paráfrases, para criar um conteúdo jocoso e provocativo.
Este tipo de linguagem, em geral, tem a intenção de performar uma
crítica a algo ou alguém, representando"-o de forma caricata, exagerando
um ou diversos aspectos desta personalidade. É interessante se valer dos recursos
utilizados pelo autor para aprofundar a compreensão da obra e retomar alguns conceitos
de figuras de linguagem e recursos textuais.

\paragraph{Metodologia}

\begin{enumerate} 

\item Introduza a atividade com uma explanação acerca de recursos textuais 
utilizados para problematizar certas questões difíceis de abordar. 

\item Faça uma leitura coletiva, em sala de aula, do início do primeiro capítulo,
em especial do seguinte trecho:

\begin{quote}
Nesse dia de festa, os judeus,
já bastante odiados por causa de sua fé, de suas riquezas e de seus
livros contábeis, encontravam"-se inteiramente nas mãos de seus
inimigos; e com extrema facilidade podiam estes provocar sua desgraça,
bastando para isso espalhar o boato de um tal infanticídio --- talvez até
mesmo introduzissem sorrateiramente um ensanguentado cadáver de criança
na casa proscrita de um judeu para depois, durante a madrugada,
investir de surpresa contra a família judia congregada em oração,
quando então se assassinava, saqueava e batizava, e grandes milagres
aconteciam graças à criança encontrada morta, a qual a Igreja por fim
chegava até mesmo a canonizar.
\end{quote}
\BNCC{EM13LP02}

\item Proponha um debate com os alunos acerca do tom utilizado nesse trecho. 
Podem ser trazidas perguntas como: qual a sensação ao ler esse trecho?
Quais recursos são usados pelo autor para tratar do tema do antissemitismo?
Como a ironia pode ser empregada para escrever sobre acontecimentos dolorosos?
\BNCC{EM13LP02}

\item Peça que os estudantes selecionem outros trechos de teor provocativo da obra
e apontem quais recursos textuais foram utilizados.

\end{enumerate}

\paragraph{Tempo estimado} Duas aulas de 50 minutos. 

\subsection{Pós"-leitura}

\paragraph{Tema} Os limites da liberdade de expressão.

\paragraph{Conteúdo} Pesquisa em meios jornalísticos sobre processos judiciais
relacionados à discursos de ódio.

\paragraph{Objetivo} Ampliar o conhecimento dos estudantes
acerca das liberdades individuais e das iniciativas no combate ao discurso de ódio.

\paragraph{Justificativa} Além de aprofundar a discussão feita
na atividade anterior, os estudantes estarão mais conectados com debates
vigentes na sociedade.

\paragraph{Metodologia} 

\begin{enumerate}

\item Recomenda"-se que os alunos busquem notícias de jornal onde pessoas foram
à justiça em razão de terem se sentido ofendidas por conta de
comentários de uma terceira pessoa.
\BNCC{EM13LP42}

\item Sugerimos que os estudantes façam uma pesquisa de 
textos que defendam as liberdades individuais e
combatam os abusos. É interessante que o professor auxilie
 na leitura e interpretação desses artigos.
 \BNCC{EM13LP12}
 
\item Em seguida, deve ser realizado um
debate na sala acerca dos limites da liberdade de expressão.
Os alunos devem trazer ideias de como eles
podem interferir no combate a discursos de ódio.
\BNCC{EM13LGG102}

\end{enumerate}

\paragraph{Tempo estimado} Uma aula de 50 minutos.


\section{Propostas de Atividades II}

A obra \emph{O Rabi de Bacherach} possibilita trabalhos
interdisciplinares e integradores de diferentes campos do saber e áreas
de conhecimento. A seguir, propomos algumas atividades que podem ser
desenvolvidas conjuntamente com professores de outras áreas. 

\begin{comment}Além das
habilidades de Linguagens e suas Tecnologias e de Língua Portuguesa,
indicadas nas etapas da seção anterior e válidas também para esta,
listamos a seguir as habilidades de outras áreas, presentes na abordagem
interdisciplinar:
\end{comment}

\subsection{Pré"-leitura}

\paragraph{Tema} A história do povo judeu.

\paragraph{Conteúdo} Conteúdo da área de História e Geografia sobre o povo judeu
e pesquisa sobre depoimentos de sobreviventes do Holocausto.

\paragraph{Objetivo} Ampliar o conhecimento dos estudantes acerca da trajetória
do povo judeu ao longo do tempo.

\paragraph{Justificativa} Essa atividade visa preparar os estudantes para
a leitura da obra e deixá"-los mais familiarizados com a história do povo 
judeu e a questão do ódio racial.

\paragraph{Metodologia}

\begin{enumerate}

\item Com o auxílio dos professores da área de ciências humanas,
deve ser feita uma recapitulação histórica da formação do povo judeu, 
os êxodos e perseguições
que sofreram, desde a Antiguidade até a \textsc{ii} Guerra Mundial. 
É importante que os alunos entendam os motivos pelos quais os judeus 
foram expulsos de suas terras e como ocorreu sua diáspora para a Europa. 
\BNCC{EM13CHS204}

\item Em seguida, peça que os estudantes assistam aos depoimentos 
de sobreviventes do Holocausto no site 
do \href{https://www.museudoholocausto.org.br/depoimentos/}{Museu do Holocausto de Curitiba}.
\BNCC{EM13CHS503}

\item Cada estudante deve selecionar um dos depoimentos e 
sistematizar os relatos em forma de uma linha do tempo 
ou de um mapa com o trajeto feito pelo entrevistado.

\end{enumerate}

\paragraph{Tempo estimado} Duas aulas de 50 minutos.


\subsection{Leitura}

\paragraph{Tema} O papel da mulher na cultura judaica.

\paragraph{Conteúdo} Pesquisa e discussão sobre
o papel da mulher na cultura judaica ao longo dos tempos, 
a partir da trajetória da personagem Sara na obra.

\paragraph{Objetivo} Incentivar uma maior atenção no momento de leitura.

\paragraph{Justificativa} Com o suporte de professores da área
de ciências humanas, o debate pode revelar aspectos da personagem
que poderiam passar despercebidos.

\paragraph{Metodologia}

\begin{enumerate}

\item Para introduzir o tema, sugerimos que seja feita uma 
pesquisa por parte dos estudantes, sobre o papel da mulher
na cultura judaica ao longo dos tempos. Podem utilizar 
meios impressos e digitais para realizar a pesquisa.
\BNCC{EM13LP30}

\item Em seguida, proponha uma discussão a partir dos dados
pesquisados pelos alunos. Os professores de ciências
humanas podem contribuir com materiais que alimentem o debate. 

\item A partir disso, os estudantes devem reavaliar e discutir a
trajetória da personagem Sara sob o
aspecto religioso, simbólico e lírico encontrado principalmente no
primeiro capítulo.

\item Peça que os estudantes façam uma comparação entre essa personagem, com as
demais personagens femininas presentes no fragmento de romance, bem como
de  personagens femininas de outras histórias. Um exemplo relacionado à temática judaica é 
a Fânia em \emph{De amor e trevas}, escrito pelo israelense Amós Oz --- mas não 
é necessário trabalhar somente com personagens ligados a esse universo. O importante é 
discutir sobre a forma que elas são apresentadas e como elas se relacionam com outros 
protagonistas de seus enredos.
\BNCC{EM13LP50}

\end{enumerate}

\paragraph{Tempo estimado} Duas aulas de 50 minutos.
   

\subsection{Pós"-leitura}

\paragraph{Tema} Conflitos religiosos nos séculos \textsc{xix} e \textsc{xx}.

\paragraph{Conteúdo} Debate acerca de outros conflitos religiosos
que ocorreram, quais suas causas e consequências. Redação de texto
argumentativo sobre o tema.

\paragraph{Objetivo} Ampliar a discussão feita na atividade de pré"-leitura
e apresentar episódios de conflitos religiosos ao longo do tempo.
Contextualizá"-los historicamente, geopoliticamente e sociologicamente.

\paragraph{Justificativa} Após a leitura, a questão do ódio racial e dos
conflitos que envolvem religião e política será um tema mais próximo dos estudantes.
Agora, podem aprofundar esse conhecimento ao pesquisarem sobre outros conflitos 
religiosos que ocorreram ao redor do mundo.

\paragraph{Metodologia}

\begin{enumerate}

\item Deve ser feita uma pesquisa, por parte dos estudantes, sobre
conflitos religiosos que ocorreram nos séculos \textsc{xix} e \textsc{xx}. 
\BNCC{EM13LP01}

\item A partir dos episódios de intolerância trazidos pelos 
alunos, os professores de ciências humanas devem propor uma discussão,
em que contribuam com referências históricas e o contexto geopolítico.
Além disso, deve"-se tratar da questão de recente aumento do
recrudescimento de movimentos religiosos
paramilitares. 

\item Por fim, sugerimos que os professores abordem
aspectos sociológicos de diferentes
religiões e peçam que os estudantes escolham um 
dos conflitos religiosos estudados e escrevam um
breve texto argumentativo sobre como a religião
e certos temas da Sociologia se entrelaçam nesse acontecimento.

\end{enumerate}

\paragraph{Tempo estimado} Duas aulas de 50 minutos.


\section{Aprofundamento}

Nesta seção, desenvolvemos um trabalho de aprofundamento que, em diálogo
com a formação continuada de professores, oferece subsídios para a
abordagem do texto literário. A leitura em sala de aula de \emph{O Rabi
de Bacherach} pode ser enriquecida pelo aprofundamento no universo
literário em que a obra está inserida.

O jovem escritor propôs"-se a criar uma epopeia sobre o
povo judeu diante da escalada antissemita na Alemanha do século \textsc{xix}.
Bacherach, cidade medieval de vida pujante e conflituosa, abriga uma
pequena comunidade judaica, liderada pelo rabino Abraão. A história da
opressão e perseguição aos judeus é narrada a partir da trajetória do
Rabi e sua esposa Sara. Publicado em 1840, dezesseis anos após sua
concepção inicial, o texto reflete as diversas facetas do autor:
ascendência judaica, lirismo poético, ironia, crítica à religião e
índole revolucionária. O tom exaltador, que descreve os costumes e
posterior fuga do casal judeu para Frankfurt, ganha uma complexidade ao
questionar dogmas, fanatismo, violência e relações de poder.
Nobres, alcaides, clero, cavaleiros, donzelas, assassinos, curandeiros,
comerciantes, moças itinerantes, monges, rabinos, bêbados e bufões são
personagens que, de alguma forma, permeiam a sociedade até os dias de
hoje. Captados por Heine, permitem espelhar ideologias da época do
autor, numa leitura engajada e, sobretudo, da condição humana. Decerto,
uma história que estará sempre por ser concluída.

\subsection{Por que ler \textit{O Rabi de Bacherach}?}

Seria convincente responder a essa questão relatando que Heinrich Heine
foi um autor admirado por Machado de Assis, Castro Alves e Álvares de
Azevedo? Tanto o prosador Heine quanto o poeta romântico mundialmente
reconhecido. Caso os olhos do leitor não tenham ainda brilhado,
argumentamos que o fragmento de romance permite viajar pela saga judaica
com matizes jornalísticas, poéticas, irônicas e paródicas. A prosa do
autor é refinada, precisa e rica em imagens.


Image{Retrato de Heine, desenhado em 1829 (Bibliothek des allgemeinen und praktischen Wissens. Bd. 5; Domínio público)}{PNLD0018-05.png}


Heine considerava"-se ``um pobre rouxinol alemão que fez seu ninho na
peruca de Monsieur Voltaire'' e ``um brioso soldado na luta de
libertação da humanidade''. Escrever sobre a filosofia e a história
judaicas era o objetivo do Iluminismo Judaico, corrente intelectual da
qual o autor fazia parte por volta de 1824. Entretanto, o olhar sobre a
dor do povo judeu ganhou uma perspectiva universal nos dois últimos
capítulos, finalizados em 1840. Comprometido com o projeto de
emancipação dos povos subjugados, o escritor preocupava"-se com a espiral
de violência em que oprimido transforma"-se em opressor. Esse
amadurecimento ideológico transforma as páginas do \emph{Rabi}, como costumava
chamar sua obra, e é digno de ser apreciado também em sua poética.


\Image{Monumento em homenagem ao autor, localizado em Brocken, na Alemanha (Stefan Schäfer, Lich ; CC-BY-SA-4.0)}{PNLD0018-04.png}



\subsection{O \emph{épico} e o \emph{lírico} na exaltação do povo judeu}

O primeiro capítulo de \emph{O Rabi de Bacherach}, escrito em 1824, difere dos
demais por seus traços nitidamente épicos e líricos. Isso se deve ao
estilo do autor à época e à proposta de dar um tratamento literário à
história do povo judeu.

Aproxima"-se do romance épico na medida em que os protagonistas
personificam o caráter heroico dos judeus. Abraão e Sara remetem ao
casal homônimo bíblico, cujos inúmeros descendentes herdariam a terra
prometida. O próprio Abraão das escrituras é considerado o fundador do
judaísmo. O rabino e sua esposa encarnam traços universais de sua
cultura e destinam"-se à perseguição. O êxodo hebreu do Egito é
oportunamente recontado quando da celebração do \emph{Pessach}, a Páscoa Judaica, pelo
Rabi. Notam"-se ainda nessa solenidade descrições eloquentes, ressaltadas
pelo uso de verbos no presente e adjetivações de riqueza e beleza. O
leitor entra, então, em contato narrativas imemoriais e tem empatia pela
dor e luta desse povo.


\Image{Ilustração para o livro, feito por Max Liebermann para edição de 1922. (Max Liebermann; Domínio público)}{PNLD0018-06.png}


O lirismo e a sinestesia dão voz ao estilo romântico que conduz a fuga
de Abraão e Sara pelo rio Reno. Nesse sentido, a paisagem adquire
atributos humanizados e assustadores que potencializam o pavor e a
violência, revelando o íntimo das personagens. A sucessão de composições
pictóricas culmina com uma atmosfera onírica, em que lembranças se
misturam a fragmentos de contos de fadas e mitos, referências muito
resgatadas no Romantismo. Sara toma uma atitude lânguida e devaneadora,
a exemplo das heroínas românticas.

\subsection{A ironia como elemento niilista}

A ironia permeia a obra como um todo, mas obtém destaque nos dois
últimos capítulos, quando instrumentaliza a crítica às incoerências
sociais, religiosas e políticas. Isso se dá sem formar juízo valorativo
e deixando um vazio de significado, traços do niilismo heineano.

No início, a construção de abadias e igrejas em nome de São Werner marca
uma situação irônica em sua base: edificações gloriosas cristãs que
trazem em sua história assassinatos, tanto do menino canonizado, quanto
dos judeus acusados injustamente.

Já em Frankfurt, existem inúmeros exemplos de ironia advindos da
observação à distância pelo narrador. Sara foi compelida a rir da luta
dos judeus com as ratazanas, entretanto estas representam seus próprios
algozes; o desfile de tipos sociais e seus símbolos, paradoxais entre
si, como médicos, cavaleiros, prostitutas, monges, caçadores, com
destaque do profano em relação ao religioso; Hans, o católico tocador de
tambor, cita santos, satanás, catecismo, mas demonstra atitudes
distanciadas do discurso; o comportamento das mulheres, fofoqueiras e
vaidosas, em plena celebração da Páscoa Judaica; o cavaleiro espanhol
elogia a seriedade e determinação do Rabi e ao mesmo tempo revela que
este flertava com uma moça na Espanha, época em que já era um homem
casado. Em tom cômico, um dos guardiões do gueto judeu elogia o nariz
avantajado do companheiro, característica étnica desse povo.

\subsection{A crítica caricaturesca e jocosa}

Os indivíduos encontrados no gueto e cercania são caricatos em suas
vestes, atributos físicos e psíquicos, com a intenção de subverter a
lógica para contestar dogmas e fé, e denunciar o preconceito e a
violência.

Do lado de fora anunciando quem se aproxima, há um cristão tocador de
tambor, que tem a voz de cerveja e um traje com adornos que lembravam
línguas. Da sua boca e, metaforicamente, das línguas teciduais, saem
palavras gargarejadas sempre a ressuscitar os conflitos entre os judeus
e cristãos.

Os soldados, supostamente defensores do local, estão deitados na torre
de vigia. Tratam"-se de Nasenstern (ou Stern), um sujeito medroso, tísico
e narigudo, e Jakel, o Tolo, homenzinho atarracado, de pernas tortas, e
cara vermelha e risonha, que lembra um bufão. Jakel debocha de Stern, ao
sugerir que a linhagem dos parentes do covarde chega até os tempos do
rei Saul, sendo os primeiros a debandarem do campo de batalha.

Ao receber o Abraão e Sara, Jakel canta uma canção infantil repetitiva e
cheia de diminutivos, conferindo um caráter patético à situação. Na
cantiga original o relâmpago, que representa Deus, destrói o anjo da
Morte (turcos) e salva o cabrito (judeus). No entanto, o Tolo interrompe
a música antes disso, no trecho em que o anjo da Morte acaba com o
carniceiro (cristãos), e afirma que o Deus dos judeus é o Deus da
vingança.


\Image{Quadro \emph{Abraão e Sarah visitados por três anjos} (National Trust; Domínio público)}{PNLD0018-10.png}


Cabe ao Rabi fazer, então, uma resenha de toda cena, dizendo à esposa:
``é esta a proteção que tem em Israel! Por fora falsos amigos guardam
seus portões e por dentro seus guardiães são a tolice e o temor!''

No terceiro capítulo aparece Don Isaak, cavaleiro espanhol, \emph{donjuanesco}
e cristão convertido, cujas penas do barrete ``movimentavam"-se mais com
o elegante balanço da cabeça do que pelo sopro do vento''. Ao rodear
Sara, jura a ela sua devoção, misturando flores com cebolas e ervilhas,
e mulas e cabritos com velhos cristãos. Segue com revelações de sua
índole pagã: ``eu amo vossa cozinha muito mais que vossa crença; a esta
falta o tempero certo'' e ``os sombrios nazarenos me são tão repulsivos
como os ressequidos hebreus sem alegria''.


\Image{Ilustração para o livro, feita por Josef Budko para edição de 1921. (Josef Budko; Domínio público)}{PNLD0018-07.png}


\Image{Ilustração de Josef Budko. (Josef Budko; Domínio público)}{PNLD0018-08.png}


\Image{Ilustração de Josef Budko. (Josef Budko; Domínio público)}{PNLD0018-09.png}


\subsection{A paráfrase e a paródia do sagrado}

A utilização de textos tradicionais como base para novas criações
artísticas potencializa o alcance semântico do autor que os utiliza. Na
paródia, há a perversão/contestação do sentido original, e na paráfrase
esta relação é neutra ou similar.

Heine utiliza tais estilizações, principalmente de fatos e trechos dos
livros sagrados judaicos e cristãos, com o intuito de multiplicar a
acepção dos mesmos e questionar a sociedade, a hierarquia e a religião.

Nota"-se o fenômeno parodístico em Sara. Durante seu devaneio, ela vê cenas
bíblicas desvirtuadas, em que Moisés é enfrentado pelo egípcio Mitzri, o
Monte Sinai arde em chamas, o Rei Faraó nada no Mar Vermelho; Stern
choraminga e comprime o rosto contra o muro em uma alusão deturpada da
reza no Muro da Lamentações; Tolo graceja da passagem bíblica sobre o
sacrifício de Isaac, que, se levado a cabo, geraria ``mais cabritos e
menos judeus''; e Dom Isaak pede perdão à deusa sírio"-fenícia Astarte
por se ajoelhar e orar diante ``da dolorosa mãe do crucificado'',
referindo"-se à conhecida imagem da Virgem Maria carregando o corpo de
cristo.

A paráfrase: ``Entre tantos cães o coelho está perdido'' remete à
expressão alemã: ``Muitos cachorros são a morte do coelho'', em que os
coelhos representam os judeus, e os cachorros, os cristãos.

\subsection{Atividades para o aprofundamento da pesquisa}

\subsubsection{Documentário sobre a situação dos refugiados}

Em uma época de grande conturbações políticas e sociais, 
é necessário conscientizar os alunos da situação dos refugiados 
em todo o planeta. Desde 2015 o mundo tem visto um aumento 
significativo do número de refugiados. Assim como os personagens 
do livro, muitas populações forçadas a deixar sua região não eram considerados 
refugiados. E até hoje há uma grande discussão sobre como definir
cada caso a fim de assegurar os direitos civis resguardados pela 
comunidade internacional. 

Segundo a jornalista Erika Sallum:

\begin{quote}
Conflitos e perseguições têm forçado mais gente a deixar suas casas do
que em qualquer outra época desde que a \textsc{acnur}, a agência das Nações
Unidas para refugiados, começou a coletar informações sobre o tema, nos
anos 1950. Dados do fim de 2015 mostraram que, pela primeira vez na
história da \textsc{onu}, o número de ``deslocados'' ultrapassou os 60 milhões.
Mais precisamente, existem hoje cerca de 65,6 milhões de pessoas
``deslocadas'' de suas casas à procura de mais segurança -- isso
equivale, em média, a vinte seres humanos fugindo por minuto, quatro vezes
mais que na década anterior. Dentre esses indivíduos, cerca de 22,5 
milhões enquadram-se na definição internacional de ``refugiados'', dos
quais 51\% têm menos de 18 anos (sendo 75.000 desacompanhados dos pais
ou parentes).

As estatísticas assustadoras não param por aí: ainda segundo a agência,
até 2016, a cada 113 pessoas no planeta, uma é refugiada, deslocada
interna ou solicitante de asilo. Se contabilizarmos o número de
refugiados, deslocados internos e solicitantes de asilo, temos uma
população maior que a de países como a França. Nada menos que 55\% dos
refugiados existentes no mundo hoje vêm de apenas três países: Síria (5,5 milhões),
Afeganistão (2,5 milhões) e Sudão do Sul (1,4 milhão). E, diferentemente do que muita gente é levada a
crer devido à divulgação de fotos e notícias sobre pessoas tentando
chegar à Europa, a vasta maioria dos refugiados não se encontra no
continente: 84\% dos que estão sob o mandato da {UNHCR} vivem em países em
desenvolvimento -- a exemplo do Líbano, onde há um refugiado a cada 6
habitantes, colocando-o na liderança das nações que mais abrigam esse
grupo em relação a sua população\footnote{Edwards, Adrian. \emph{Global forced
  displacement hits record high}.
  {unhcr.org/} (20 de
  junho de 2016) e \emph{{UNHCR} - Global Trends: Forced Displacement 2016;}
  https://goo.gl/86CXwK; Acesso: 4 
  de julho de 2017.}. 
\end{quote}

Propomos então uma pesquisa que pode ser divulgada por meio de
um mini documentário de divulgação científica sobre a situação dos refugiados 
no mundo. Você pode dispor das seguintes perguntas, colocadas pela 
jornalista Erika Sallum:

\begin{enumerate}
\item O que exatamente significa ser um refugiado? 
\item Quem deve protegê-lo quando seu próprio Estado torna"-se o algoz 
ou incapaz de fazê"-lo? Qual a
diferença entre refugiado, deslocado interno e imigrante? 
\item Quando a comunidade internacional se omite e vira as
costas para uma das maiores e mais delicadas crises do mundo contemporâneo, a
quem essas pessoas podem recorrer?
\end{enumerate}

Os alunos podem produzir tanto um texto, quanto um podcast ou um pequeno vídeo.
Para isso, recomenda"-se a utilização de aplicativos gratuitos de
gravação e edição de vídeos, disponíveis em dispositivos digitais,
para montar um vídeo de até 4 minutos sobre o tema. É possível
utilizar desenhos e animações, formato de telejornal ou de
\emph{vlog}, com cuidados para garantir qualidade de áudio e imagem.
Reforce a importância de haver um recorte histórico"-cultural do tema,
passando pelas artes visuais e pela literatura.

\Image{Aumento do número de refugiados por ano. Fonte: UNHCR. Divulgação.}{PNLD0018-20}

\subsubsection{Mural sobre refugiados do passado}

Recentemente, o conflito na Síria escancarou esse tipo de situação.
Como sugestão de atividade complementar e, considerando
o campo de atuação na vida pública, recomenda"-se orientar os 
alunos a realizarem uma pesquisa sobre
migrações forçadas no passado. É possível pensar nas migrações bíblicas ou 
até as mais recentes durante as Grandes Guerras Mundiais, ou 
após a descolonização de África e Ásia, durante o século \textsc{xix}.

Aconselha"-se que os alunos nessa atividade trabalhem com dados e mapas
e procurem obter informações históricas sobre o conflito.

Posteriormente, devem montar um quadro mural com as informações que
pode vir a ser exibido no colégio.

\subsubsection{Um podcast sobre notícias de guerra}

Levando em consideração o campo jornalístico"-midiático, sugerimos
uma atividade complementar que trate dos relatos de guerra.
Durante muito tempo na história, os relatos de uma guerra eram aqueles
contados pelos sobreviventes no campo de batalha, frequentemente os
vencedores. É somente no século \textsc{xix} que, com advento da fotografia e
de meios de telecomunicação como o telégrafo, o telefone e, por fim, a
internet, que o mundo passou a receber notícias de conflitos armados
concomitantemente à sua ocorrência. Isso colocou em xeque as
narrativas oficiais, uma vez que as imagens, por vezes, contestavam o
que estava sendo oficialmente descrito. Um exemplo presente é a Guerra
do Paraguai (1864--1870), em que as pinturas encomendadas pelo governo
brasileiro descreviam uma guerra heroica, ao passo que as fotografias
revelavam uma narrativa muito mais nefasta. Para essa atividade, os
alunos devem criar um podcast de poucos minutos (aconselha"-se o máximo
de seis minutos), cujo conteúdo deva ser jornalismo de guerra. Na sua
realização, os alunos devem se valer de notícias e depoimentos de
conflitos recentes. É interessante uma pesquisa também de como esses
relatos do local divergem, por vezes dos relatos oferecidos pelos
governos beligerantes.

%\Image{Captura de tela do site \href{https://liveuamap.com}{Liveumap}}.{PNLD0018-21.png}

Aconselhamos que eles utilizem a \href{https://liveuamap.com/}{ferramenta gratuita Liveuamap}. 
Nela é possível acompanhar, em um mapa, eventos citados por jornalistas locais 
em tempo real. Ao longo do tempo, esses jornalistas são classificados conforme
suas tendências políticas e as notícias fragmentadas são dispostas em 
um mapa de acordo com a sua natureza. E conforme a evolução do conflito as fronteiras
vão sendo desenhadas. É possível ainda consultar uma linha do tempo e 
verificar o que ocorreu naquele determinado momento. Aconselhamos que 
os alunos escolham por exemplo uma cidade em conflito.  

\subsubsection{Analisando o \emph{Guernica} de Picasso}

Essa atividade complementar está relacionada com campo artístico"-literário,
em que sugerimos tratar da
iconografia da dor, do conflito, do êxodo, que foi muito narrada,
desenhada, fotografada e roteirizada. 

\SideImage{Cidade de Guernica destruída após bombardeio, 1937. Fonte: Wikipedia. CC-BY-SA 3.0}{PNLD0018-22.jpg}

Em conjunto com os professores de
humanidades, apresente os principais fatos da guerra civil espanhola. 
Em seguida, proponha algum filme sobre o conflito. 
\href{https://www.brasildefato.com.br/2020/07/18/no-pasaran-10-filmes-sobre-a-guerra-civil-espanhola}{Nesse link} 
você pode achar uma lista de 10 filmes. Sugerimos dois: \emph{Ay, Carmela!} (1990), 
do cineasta Carlos Saura; e  \textit{Terra e Liberdade} (1995), de Ken Loach.

Em seguida, apresente uma das pinturas mais
do mundo, a obra \emph{Guernica} (1937), do espanhol Pablo Picasso. Nela,
retrata"-se o bombardeio realizado na pequena cidade de Guernica,
durante a Guerra Civil Espanhola. Apresente essa pintura aos alunos.
Peça para que eles indiquem as emoções que a mesma desperta, que
descrevam o que acontece na cena. 

Indique que a expressão dos
sentimentos não precisa adotar a forma realista. Posteriormente, peça
para que eles selecionem alguma passagem do livro e produzam um
desenho retratando algum trecho do livro lido. 

% Como leitura
% complementar para a busca de referências para essa atividade,
% recomenda"-se, além das pinturas do próprio Picasso, as pinturas de Marc
% Chagall e Edvard Much, além da leitura de trechos do livro \emph{O
% Exército de Cavalaria}, de Isaac Bábel. 

\subsubsection{Os conceitos de \emph{ius sanguini} e \emph{ius soli}}

Por fim, considerando o campo das práticas de estudo e pesquisa,
sugere-se tratar do direito à cidadania.
Durante muito tempo, esse direito era conferido a poucas
pessoas. Na Atenas Clássica, era oferecida apenas aos homens livres,
maiores de idade, descendentes de jônicos, filhos de pais e mães
atenienses. Na Europa, judeus não eram considerados como pertencentes
ao povo de alguns países onde se encontravam, mas eram vistos como
estrangeiros, ainda que há gerações vivessem no país. Houve, também,
uma decisão judicial da Suprema Corte americana, indicando que o país
não entendia os negros como seus cidadãos, ainda que, também,
estivessem há muitas gerações no país. A questão da cidadania, de como
ela é determinada é poder exclusivo do Estado. Sendo ela um fator que
implica direitos e deveres, é importante que o tema seja debatido em
sala de aula. Para tanto, propõe"-se duas etapas. Primeiramente, os
alunos devem ser divididos em grupo, fazendo uma breve pesquisa sobre
as definições de Nação, Estado, Governo e Território. Feito isso,
devem, em uma segunda etapa, buscarem informações sobre nacionalidade.
As diferenças entre os conceitos de \emph{ius sanguini} e \emph{ius
soli}, no entendimento da Organização das Nações Unidas sobre a
questão. Por fim, munidos disso, o professor deve propor um debate
acerca da situação de refugiados, de apátridas e de grandes nações
desprovidas de Estados. Nessa atividade, os alunos devem pensar em
soluções possíveis, à luz dos direitos humanos, para essas questões.

\section{Sugestões de referências complementares}\label{sugestoes}

\subsection{Audiovisual}

\begin{itemize}
\item\textit{A Vida é Bela}. Direção: Roberto Benigni, 1997.

Na 2ª Guerra Mundial, um judeu, dono de uma livraria na Itália,
é capturado e levado para um campo de concentração. Esse é um dos 
filmes sugeridos para a atividade I de pré"-leitura, 
o cine debate sobre antissemitismo.

\item\textit{Jojo Rabbit}. Direção: Taika Waititi, 2019.

O filme conta a história de um menino de 10 anos que sonha
em participar da Juventude Hitlerista, quando descobre que sua
mãe está escondendo uma jovem judia no sótão de sua casa.
Esse é um dos 
filmes sugeridos para a atividade I de pré"-leitura, 
o cine debate sobre antissemitismo.

\item\textit{Nada Ortodoxa}. Direção: Maria Schrader, 2020.

A série da Netflix conta a história de Esty, que foge de uma comunidade 
judaica ortodoxa em Nova York para buscar um caminho próprio em Berlim.
A narrativa pode ser um motor de discussão para a atividade
acerca do papel da mulher na cultura judaica.

\item\textit{O Pianista}. Direção: Roman Polanski, 2002.

Um pianista judeu é perseguido e se refugia em um prédio abandonado,
após as decorrências da \textsc{ii} Guerra Mundial e do surgimento do Gueto de Varsóvia.
Esse é um dos 
filmes sugeridos para a atividade I de pré"-leitura,
o cine debate sobre antissemitismo.

\item\textit{Um Violinista no Telhado}. Direção: Norman Jewison, 1971.

Um leiteiro judeu pretende arranjar o casamento de suas duas filhas,
porém ambas se recusam e uma delas decide casar com um não judeu. O
leiteiro debate"-se nesta situação delicada quando um decreto do Czar
obriga todos os judeus a abandonar a aldeia, condenando a sua família ao
exílio e à dispersão. Esse é um dos 
filmes sugeridos para a atividade I de pré"-leitura, 
o cinedebate sobre antissemitismo.
\end{itemize}

\subsection{Museus}

\begin{itemize}
\item Casa do Povo

A \href{https://casadopovo.org.br/}{Casa do Povo} é um espaço 
judaico, aberto também a outras minorias. Localizado no
Bom Retiro, em São Paulo, um bairro de imigrantes que hoje conta com muitos coreanos, 
bolivianos e judeus --- mas já recebeu uma grande comunidade italiana e grega. 
No site, é possível conhecer
os eventos e iniciativas realizados pelos coletivos que ocupam a casa e 
contribuem com a programação.

\item Museu do Holocausto de Curitiba

Esse museu, com acesso ao site 
\href{https://www.museudoholocausto.org.br/}{disponível neste link}, 
é uma das maiores referências de memória do Holocausto no Brasil.
Sugerimos que os depoimentos de sobreviventes do Holocausto presentes no site
sirvam como meio de pesquisa para a atividade II de pré"-leitura.

\item Sinagoga Zahal Kur

Foi a primeira sinagoga oficial dos judeus que habitaram as Américas. 
O piso térreo possui exposições permanentes sobre a cultura judaica. 
Pode ser interessante para conhecer a história do povo judeu no Brasil.

\item Memorial do Holocausto de Jerusalém

O \href{https://www.yadvashem.org/}{site} do memorial possui um 
extenso arquivo de memória do povo judeu
e da história do Holocausto. Pode ser uma fonte interessante de material 
iconográfico para a pesquisa a ser realizada na
atividade II de pré"-leitura.

\item Museu Judaico de Berlim

O \href{https://www.jmberlin.de/en}{site} do museu possui 
várias exposições virtuais que abordam o tema
do passado e do presente dos judeus na Alemanha.

\item Museu do Holocausto dos Estados Unidos

No \href{https://www.ushmm.org/}{site} do museu, que é um dos 
mais importantes locais de registro da história
da perseguição de judeus, é possível encontrar páginas educativas
que explicam sobre o que é antissemitismo, o que é genocídio etc.
\end{itemize}

\section{Bibliografia comentada}

\begin{itemize}
%Sofia: Livro abaixo não está comentado. Tirar da lista?
\item \textsc{bonis}, Gabriel. \textit{Refugiados de Idomeni}. São Paulo: Editora Hedra, 2017.

\item\textsc{boyne}, John. \textit{O menino do pijama listrado}. São Paulo: Editora
Seguinte, 2007.

O menino do pijama listrado é uma fábula sobre amizade em tempos de
guerra, e sobre o que acontece quando a inocência é colocada diante de
um monstro terrível e inimaginável.

\item\textsc{carpeaux}, Otto Maria. \textit{História da literatura ocidental}. São
Paulo: Leya, 2019.

Uma coleção que nos leva ao encontro dos mais importantes autores da
literatura ocidental, numa obra imprescindível para qualquer pessoa que
se proponha a estudar verdadeiramente o tema.

\item\textsc{carpeaux}, Otto Maria. \textit{A história concisa da literatura alemã}.
Barueri: Faro Editorial, 2013.

Neste livro encontra"-se uma síntese dos grandes momentos, livros e
autores da literatura alemã, bem como sua importância para o a cultura e
o desenvolvimento do país e do mundo contemporâneo.

\item\textsc{safranski}, Rudiger. \textit{Romantismo: uma questão alemã}. São
Paulo: Estação Liberdade, 2012.

A obra divide"-se em duas partes: ``O romantismo'' e ``O romântico?'' Na
primeira parte, o autor aponta as origens do movimento. Na segunda, há a
investigação do ``romântico como atitude espiritual'' até a chegada do
socialismo nacional alemão.

\item\textsc{hauser}, Arnold. \textit{História social da arte e da literatura}. São
Paulo: Martins Fontes, 2010.

O valor desta obra consiste no fato de que Hauser, fundamentado em um
conhecimento preciso de fontes e literatura especializada, reúne
resultados claros da sociologia da arte, da música e da literatura.
\end{itemize}


\end{document}

