\documentclass[12pt]{extarticle} 
\usepackage{manualdoprofessor}
\usepackage{fichatecnica} 
\usepackage{lipsum,media9,graficos}
\usepackage[justification=raggedright]{caption} 
\usepackage{bncc}
\usepackage[iluminuras]{../edlab}



\begin{document}


\newcommand{\AutorLivro}{Sidney Rocha}
\newcommand{\TituloLivro}{\textit{Sofia}}
\newcommand{\Tema}{Ficção, mistério e fantasia}
\newcommand{\Genero}{Romance}
\newcommand{\imagemCapa}{./images/PNLD0054-01.png}
\newcommand{\issnppub}{---}
\newcommand{\issnepub}{---}
%\newcommand{\fichacatalografica}{PNLD0054-00.png}
\newcommand{\colaborador}{\textbf{Fulano de Tal} é uma pessoa incrível e vai
fazer um bom serviço.}


\title{\TituloLivro} \author{\AutorLivro} \def\authornotes{\colaborador}

\date{} \maketitle

\baselineskip=1.20\baselineskip\par

\begin{abstract}

Este Manual tem como objetivo fornecer subsídios para o trabalho com a
  literária \textit{Sofia} (2014), de Sidney Rocha.

Sidney Rocha, nascido em Juazeiro do Norte, Ceará, em 1965, é escritor e
  editor, além e roteirista de cinema e televisão. 
  Foi vencedor do Prêmio Jabuti, em
  2012, com o romance \textit{O Destino das Metáforas}, e do Prêmio Osman Lins,
  em 1985, com o presente romance, \textit{Sofia}. Entre 
  seus livros publicados, estão 
  \textit{Matriuska} (2009), 
  \textit{O destino das metáforas} (2011) e
  \textit{A estética da diferença} (2018).


\textit{Sofia} foi reescrito três vezes autor. Trata-se um romance
  em primeira pessoa sobre a paixão de um homem por uma mulher misteriosa, 
  que nunca se apresenta materialmente. As descrições sobre ela nunca são físicas.
  Mesmo assim, causam forte efeito por conta de seus efeitos mágicos. 

Um dos grandes encantos do livro é o fato de ser o
  leitor responsável por responder à questão: quem é \textit{Sofia}? Quem é, para quem
  lê, essa personagem que movimenta as emoções de tudo e todos ao seu redor?



Esperamos, com este manual, poder auxiliar o professor em seu trabalho,
  certamente muito prazeroso, em sala de aula!


\end{abstract}

\tableofcontents



\section{Introdução}

Sidney Rocha é um dos mais importantes e reconhecidos
escritores da atualidade. Entre suas qualidades, podemos destacar a capacidade
de contar uma boa história. Seu estilo de escrita é bastante direto. Suas
frases curtas, quase sem metáforas, envolvem os leitores desde o início da
leitura. Ele desenvolve o assunto com simplicidade e leveza. Outra qualidade de
Sidney Rocha é a maneira como constrói as personagens. São figuras quase
desmaterializadas, mas que exercem um grande fascínio nos leitores e leitoras. 

Seus livros exploram a sensibilidade das pessoas, causando aquelas pausas
meditativas e uma vontade enorme de continuar conhecendo suas obras.  No livro
\textit{Sofia}, a dança é um tema central, podendo ser uma porta de entrada dos alunos
no universo da arte, do movimento e da construção da cidadania. 

Ao explicar seu processo criativo, Sidney Rocha diz em uma 
\href{http://www.vacatussa.com/entrevista-sidney-rocha/}{entrevista}\footnote{
  Revista Vakatusa, por Thiago Corrêa Ramos, 28 de abril de 2014.} para para 
o jornalista Hugo Viana:

\begin{quote}
Então, eu era um romancista de vinte e poucos anos com muitas ideias-fixas (e feitas), todas inúteis, mas descobri logo que, num romance, quase nada do que você acha que sabe, você termina por usar. E a grande maioria do que fica, de fato, na página, tem a ver com escolhas mais duras e menos cosméticas, que tem a ver com a vida de verdade, que é disso que é feito um bom romance. E, paradoxo: ter lido Flaubert nesse tempo foi uma sorte. O bom romance tem mais a ver com perdas que com ganhos. Sofia é sobre essas perdas. 
\end{quote}

O livro foi reescrito e teve três versões distintas ao longo de duas décadas, explica com 
humor o autor:

\begin{quote}
 Não tenho nenhuma dificuldade em buscar melhoras no texto. Alguns livros na minha estante, mesmo, gosto de brincar com eles e retirar uma ou duas palavras de uma frase de Faulkner, por exemplo, e experimentar. Alterar ou acrescentar um parágrafo em James Joyce. Pena que o inverso não possa ocorrer.
\end{quote}


O romance conta a história do relacionamento entre o narrador e e uma
jovem chamada \textit{Sofia}, que se torna
presente gradualemte ao longo da narrativa, na medida em que  
o escritor, cauteloso, revela seus pensamentos.  

A partir da construção de imagens, imaginamos como a garota 
pode é pintada como a própria personificação da música. A dança, como 
um de seus elementos básicos, é o elemento que descreve seus
cabelos. Uma sinfonia que, por meio de uma
sonata retorna após contra-argumentos, derivações e floreios.  
\SideImage{Sidney Rocha (Arquivo do autor}{PNLD0054-03.png}


Como conta o autor, “sempre me senti um escritor inconcluso, 
mesmo quando termino algo. Nesses anos vim alterando o texto, 
e isso já é bem visível já na segunda edição. Agora o romance 
se transformou em um novo romance. Quem o leu antes ‘se’ reconhecerá 
nele tanto quanto o novo leitor, de novíssima geração”

Por onde \textit{Sofia} passa é possível sentir mais do que ver. 
A suas principais características são a dança e a música. 
Ela tem o poder encantatório, que arrebata o narrador. Ao vê-la,
entende que a vitalidade da vida depende de seu corpo. 
\textit{Sofia} é também misteriosa e atraente. A moça 
não passaria de uma invenção do narrador? De um delírio, talvez?  

A trama é simples: um homem que conhece uma mulher, apaixona-se e passa a
carregar essa paixão dentro de si. A diferença é que \textit{Sofia} é uma personificação
da música; é a musa que inspira e o objeto pelo qual a inspiração é
transmitida. 

Como diz o autor, ``tudo que escrevo está mesmo conectado 
a esse mundo interior. Mas digamos que é um mundo interior inteiramente 
voltado para fora, das figuras em ação e convivendo com outras.'' 

A escolha do nome da personagem principal é chave para a interpretação 
do romance. O nome ``sofia'' é de origem grega e significa sabedoria, 
mas que também tem o sentido arcaico ``daquele que domina uma arte 
ou uma técnica, como frequentemente os poetas ou os músicos''. 
O livro apresenta assim a beleza de uma mulher e a arte do amor
associadas à música e a dança, características típicas das 
musas gregas.

Com isso, o autor indica que
a música é a sabedoria máxima, afinal, como a própria \textit{Sofia} personagem, ela
desde muito cedo nos atrai, convida-nos a querer estar próximo a ela,
percebemos seu poder de atração e de compreensão nos momentos de tristeza e a
potência que ela dá às horas de celebração.  

Assim, \textit{Sofia} como música é a
sabedoria máxima, capaz de se comunicar e definir sentimentos
desconhecidos ou que não podem ser traduzidos em fala. Transborda-se, como 
um som universal. É encantadora, afinal, e turva a visão do expectador que 
busca o olhar direto.  

Com isso, a forma com que a paixão aparece no livro não é a mesma
de uma história de amor tradicional. O tom sublime dessa atração é 
marcado.  \textit{Sofia} de carne e osso fica por conta da imaginação
dos leitores e leitoras. Esse é um dos grandes encantos do livro. Aos poucos
mergulhamos nos encantos que \textit{Sofia} enreda o narrador como uma espécie 
de figura marinha, cujos cantos atraíam marinheiros.
Trata-se, simplesmente, de uma mulher fascinante, que provoca crises no
personagem-narrador, após rápidos e fortuitos encontros.  

O narrador por sua vez não tem seu nome revelado. Conhecemos apenas sua
consciência, identificada com o próprio autor. 
que compartilha com os leitores pensamentos contínuos, afetos que sofreu
pela visão de \textit{Sofia}. Como partilhar o que sentimos? 
Essa parece a grande questão inquietante do narrador.  





\section{Proposta de atividades I}

\subsection{Pré-leitura}


%(EM13LGG302) Compreender e posicionar-se criticamente diante de diversas
%visões de mundo presentes nos discursos em diferentes linguagens, levando em
%conta seus contextos de produção e de circulação.%

%(EM13LGG704) Apropriar-se criticamente de processos de pesquisa e busca de
%informação, por meio de ferramentas e dos novos formatos de produção e
%distribuição do conhecimento na cultura de rede.%

%(EM13LP10) Selecionar informações, dados e argumentos em fontes confiáveis,
%impressas e digitais, e utilizá-los de forma referenciada, para que o texto a
%ser produzido tenha um nível de aprofundamento adequado (para além do senso
%comum) e contemple a sustentação das posições defendidas.%

%(EM13LP19) Compartilhar gostos, interesses, práticas culturais, temas/
%problemas/questões que despertam maior interesse ou preocupação, respeitando e
%valorizando diferenças, como forma de identificar afinidades e interesses
%comuns, como também de organizar e/ou participar de grupos, clubes, oficinas e
%afins.

\textbf{1.} NA PRÉ-LEITURA A IDEIA É SENSIBILIZAR OS ALUNOS E ALUNAS DE MANEIRA
POÉTICA. PROPOMOS PARA ESSE MOMENTO ALGUMAS MÚSICAS CALMAS E RELAXANTES,. COMO
A CANÇÃO \emph{ROMÂNTICOS,.} DE VANDER LEE.

PROPOMOS TAMBÉM A LEITURA DO POEMA \emph{ISMÁLIA},. DE ALPHONSUS DE GUIMARÃES.

\subsection{Leitura}

% \begin{longtable}[]{@{}l@{}} \toprule \endhead
% \begin{minipage}[t]{0.97\columnwidth}\raggedright \textbf{De olho na BNCC}

% \begin{quote} \textbf{As atividades sugeridas nesta subseção contemplam as
% seguintes habilidades propostas pela BNCC: }

% (EM13LGG103) Analisar, de maneira cada vez mais aprofundada, o funcionamento
% das linguagens, para interpretar e produzir criticamente discursos em textos
% de diversas semioses.

% (EM13LP02) Estabelecer relações entre as partes do texto, tanto na produção
% como na recepção, considerando a construção composicional e o estilo do
% gênero, usando/reconhecendo adequadamente elementos e recursos coesivos
% diversos que contribuam para a coerência, a continuidade do texto e sua
% progressão temática, e organizando informações, tendo em vista as condições
% de produção e as relações lógico-discursivas envolvidas (causa/efeito ou
% consequência; tese/argumentos; problema/solução; definição/exemplos etc.).

% (EM13LP48) Perceber as peculiaridades estruturais e estilísticas de
% diferentes gêneros literários (a apreensão pessoal do cotidiano nas crônicas,
% a manifestação livre e subjetiva do eu lírico diante do mundo nos poemas, a
% múltipla perspectiva da vida humana e social dos romances, a dimensão
% política e social de textos da literatura marginal e da periferia etc.) para
% experimentar os diferentes ângulos de apreensão do indivíduo e do mundo pela
% literatura.  \end{quote}\strut \end{minipage}\tabularnewline \bottomrule
% \end{longtable}

\textbf{2. }

AO LONGO DA LEITURA DO LIVRO,. PROPOMOS QUE OS PROFESSORES E PROFESSORAS
  INCENTIVEM OS ALUNOS A OBSERVAREM COMO A DANÇA ESTÁ PRESENTE AO LONGO DO
  ROMANCE.


\Image{Espectáculo da Companhia de Dança Deborah Colker (Rosano Mauro; CC BY 2.0)}{PNLD0054-04.png}


\Image{Oficina de dança contemporânea (Mallu Silva/Labfoto; CC BY 2.0)}{PNLD0054-05.png}


PODE SER DESENVOLVIDA,. PARALELAMENTE AO PROCESSO DE LEITURA,. UMA
PESQUISA BIOGRAFICA DE DANÇARINAS FAMOSAS. AS PESQUISAS PODERÃO SER
SOCIALIZADAS POR MEIO DE ATIVIDADES COMO REPORTAGENS EM VÍDEO.


\Image{Pina Bausch foi uma importante dançarina e coreógrafa alemã. (Raphael Labbé; CC BY-SA 3.0)}{PNLD0054-06.png}


\Image{Cecília Kerche é uma famosa bailarina brasileira, que assumiu em 1986 a posição de primeira bailarina do Teatro Municipal do Rio de Janeiro. (Fernando Frazão/Agência Brasil; CC BY 2.0)}{PNLD0054-07.png}


\Image{Ana Botafogo é atriz e uma das mais conhecidas bailarinas brasileiras. (Roberto Filho; CC BY 2.1)}{PNLD0054-08.png}


 
É MUITO IMPORTANTE LEMBRAR QUE SIDNEY ROCHA É UM DOS MAIS IMPORTANTES E
  RECONHECIDOS ESCRITORES DA ATUALIDADE.
 
É UM AUTOR MUITO RECOMENDADO PARA QUEM ESTÁ INICIANDO A LEITURA DE LIVROS DE
  FICÇÃO,. PRINCIPALMENTE DE AUTORES BRASILEIROS CONTEMPORÂNEOS.





\subsection{Pós-leitura}

% \begin{longtable}[]{@{}l@{}} \toprule \endhead
% \begin{minipage}[t]{0.97\columnwidth}\raggedright \textbf{De olho na BNCC}

% \begin{quote} \textbf{As atividades sugeridas nesta subseção contemplam as
% seguintes habilidades propostas pela BNCC: } \end{quote}

% (EM13LGG102) Analisar visões de mundo, conflitos de interesse, preconceitos e
% ideologias presentes nos discursos veiculados nas diferentes mídias como
% forma de ampliar suas as possibilidades de explicação e interpretação crítica
% da realidade.

% (EM13LGG303) Debater questões polêmicas de relevância social, analisando
% diferentes argumentos e opiniões manifestados, para negociar e sustentar
% posições, formular propostas, e intervir e tomar decisões democraticamente
% sustentadas, que levem em conta o bem comum e os Direitos Humanos, a
% consciência socioambiental e o consumo responsável em âmbito local, regional
% e global.

% (EM13LGG402) Empregar, nas interações sociais, a variedade e o estilo de
% língua adequados à situação comunicativa, ao(s) interlocutor(es) e ao gênero
% do discurso, respeitando os usos das línguas por esse(s) interlocutor(es) e
% combatendo situações de preconceito linguístico.

% (EM13LGG703) Utilizar diferentes linguagens, mídias e ferramentas digitais em
% processos de produção coletiva, colaborativa e projetos autorais em ambientes
% digitais.

% (EM13LP13) Planejar, produzir, revisar, editar, reescrever e avaliar textos
% escritos e multissemióticos, considerando sua adequação às condições de
% produção do texto, no que diz respeito ao lugar social a ser assumido e à
% imagem que se pretende passar a respeito de si mesmo, ao leitor pretendido,
% ao veículo e mídia em que o texto ou produção cultural vai circular, ao
% contexto imediato e sócio histórico mais geral, ao gênero textual em questão
% e suas regularidades, à variedade linguística apropriada a esse contexto e ao
% uso do conhecimento dos aspectos notacionais (ortografia padrão, pontuação
% adequada, mecanismos de concordância nominal e verbal, regência verbal etc.),
% sempre que o contexto o exigir.

% (EM13LP14) Produzir e analisar textos orais, considerando sua adequação aos
% contextos de produção, à forma composicional e ao estilo do gênero em
% questão, à clareza, à progressão temática e à variedade linguística
% empregada, como também aos elementos relacionados à fala (modulação de voz,
% entonação, ritmo, altura e intensidade, respiração etc.) e à cinestesia
% (postura corporal, movimentos e gestualidade significativa, expressão facial,
% contato de olho com plateia etc.).

% (EM13LP28) Resumir e resenhar textos, com o manejo adequado das vozes
% envolvidas (do autor da obra e do resenhador), por meio do uso de paráfrases,
% marcas do discurso reportado e citações, para uso em textos de divulgação de
% estudos e pesquisas.

% (EM13LP29) Realizar pesquisas de diferentes tipos (bibliográfica, de campo,
% experimento científico, levantamento de dados etc.), usando fontes abertas e
% confiáveis, registrando o processo e comunicando os resultados, tendo em
% vista os objetivos colocados e demais elementos do contexto de produção, como
% forma de compreender como o conhecimento científico é produzido e
% apropriar-se dos procedimentos e dos gêneros textuais envolvidos na
% realização de pesquisas.

% (EM13LP52) Produzir apresentações e comentários apreciativos e críticos sobre
% livros, filmes, discos, canções, espetáculos de teatro e dança, exposições
% etc. (resenhas, vlogs e podcasts literários e artísticos, playlists
% comentadas, fanzines, e-zines etc.).\strut \end{minipage}\tabularnewline
% \bottomrule \end{longtable}

Por fim, a atividade sugere a proposição de uma oficina de escrita poética. Com
a reprodução de uma playlist composta pelos alunos para ambientação tocando ao
fundo, incentiva-se aproveitar todo o espaço da sala para o exercício. Proponha
que os alunos tentem se valer da sonoridade ouvida para imprimir ritmo e
melodia no texto escrito. O objetivo é explorar a importância da sonoridade
para o gênero lírico.


\section{Proposta de atividades II}

A obra \emph{\textit{Sofia}} possibilita trabalhos interdisciplinares e integradores de
diferentes campos do saber e áreas de conhecimento. A seguir, propomos algumas
atividades que podem ser desenvolvidas conjuntamente com professores de outras
áreas. Além das habilidades de Linguagens e suas Tecnologias e de Língua
Portuguesa, indicadas nas etapas da seção anterior e válidas também para esta,
listamos a seguir as habilidades de outras áreas, presentes na abordagem
interdisciplinar:

% \begin{longta ble}[]{@{}l@{}} \toprule \endhead
% \begin{minipage}[t]{0.97\columnwidth}\raggedright \textbf{De olho na BNCC}

% \begin{quote} \textbf{As atividades sugeridas nesta subseção contemplam as
% seguintes habilidades propostas pela BNCC: } \end{quote}

% (EM13CNT201) Analisar e utilizar modelos científicos, propostos em diferentes
% épocas e culturas para avaliar distintas explicações sobre o surgimento e a
% evolução da Vida, da Terra e do Universo.

% (EM13CNT303) Interpretar textos de divulgação científica que tratem de
% temáticas das Ciências da Natureza, disponíveis em diferentes mídias,
% considerando a apresentação dos dados, a consistência dos argumentos e a
% coerência das conclusões, visando construir estratégias de seleção de fontes
% confiáveis de informações.

% (EM13CHS101) Analisar e comparar diferentes fontes e narrativas expressas em
% diversas linguagens, com vistas à compreensão e à crítica de ideias
% filosóficas e processos e eventos históricos, geográficos, políticos,
% econômicos, sociais, ambientais e culturais.

% (EM13CHS102) Identificar, analisar e discutir as circunstâncias históricas,
% geográficas, políticas, econômicas, sociais, ambientais e culturais da
% emergência de matrizes conceituais hegemônicas (etnocentrismo, evolução,
% modernidade etc.), comparando-as a narrativas que contemplem outros agentes e
% discursos.

% (EM13CHS106) Utilizar as linguagens cartográfica, gráfica e iconográfica e de
% diferentes gêneros textuais e as tecnologias digitais de informação e
% comunicação de forma crítica, significativa, reflexiva e ética nas diversas
% práticas sociais (incluindo as escolares) para se comunicar, acessar e
% disseminar informações, produzir conhecimentos, resolver problemas e exercer
% protagonismo e autoria na vida pessoal e coletiva.

% (EM13CHS401) Identificar e analisar as relações entre sujeitos, grupos e
% classes sociais diante das transformações técnicas, tecnológicas e
% informacionais e das novas formas de trabalho ao longo do tempo, em
% diferentes espaços e contextos.\strut \end{minipage}\tabularnewline
% \bottomrule \end{longtable}

\subsection{Pré-leitura}

 A atividade de pré-leitura sugere uma pesquisa mediada pelos professores de
 ciências humanas. A ideia é realizar uma pesquisa sobre o papel das musas e da
 deusa Afrodite na mitologia. Como essas representantes da arte e das forças de
 atração agem nas lendas antigas influenciando a vida dos humanos. É
 interessante abrir uma roda de conversa, indicando aos educandos os paralelos
 e interpretações que podem se fazer desses contos antigos e como esses são
 consoantes a vida de cada um de nós.


 
 
\Image{Quadro "O Nascimento de Vênus", de Sandro Botticceli representa a deusa do amor, cujo nome grego é Afrodite. (Sandro Botticelli; Domínio Público)}{PNLD0054-09.png}


\Image{Estátua de Afrodite exposta na "Biblioteca Nazionale Marciana", em Veneza (Biblioteca Nazionale Marciana; CC BY 3.0)}{PNLD0054-10.png}


\Image{Estátua de Afrodite exposta no Museu Arqueológico de Nápoles. É uma cópia da estátua grega de 310-200 A.C. (Museu Arqueológico de Nápoles; CC BY 3.1)}{PNLD0054-11.png}


\subsection{Leitura}

\textbf{5.} Em seguida, professores de ciências naturais podem ajudar a
entender o dinamismo dos movimentos corporais dos dançarinos. Os alunos verão
como se originam o equilíbrio e a força física necessários à dança (com a
produção de uma reportagem em vídeo ao final).    Nesse momento, é fundamental
um paralelo entre as linguagens expressivas do corpo, as ciências humanas e as
ciências da natureza. A vida humana na terra só é possível graças ao corpo. É
ele que nos torna vivos e, junto com a mente, concretiza a nossa existência.   
Por isso, a dança está ligada à formação da identidade, a partir da expressão
artística.  O indivíduo se expressa e se torna capaz por meio da arte, que
torna possível o trabalho com as competências socioemocionais.


\subsection{Pós-leitura}

\textbf{6.} Por último, depois da leitura, propomos a realização de um sarau
para apresentação dos poemas e do vídeo feitos na atividade um (1). Assim como
a socialização do programa de divulgação científica feito na atividade 2.

\section{Aprofundamento}


\subsection{A Obra}

Já no título da obra podemos esperar algo como a presença de uma personagem
feminina. Mas durante a leitura, aos poucos, começamos a suspeitar de algo.

\textit{Sofia} é uma mulher vista pelos olhos de um homem apaixonado. Seus cabelos
dançam e deixam um rastro de música. Sua existência é um compêndio harmônico
sem as balizas da tablatura. Ela é um tipo de espírito livre que anda pelo
mundo desde os dois anos.


\textbf{Campo da vida pessoal}

``O campo da vida pessoal pretende funcionar como espaço de articulações
e sínteses das aprendizagens de outros campos postas a serviço dos
projetos de vida dos estudantes. As práticas de linguagem privilegiadas
nesse campo relacionam-se com a ampliação do saber sobre si, tendo em
vista as condições que cercam a vida contemporânea e as condições
juvenis no Brasil e no mundo.

Está em questão também possibilitar vivências significativas de práticas
colaborativas em situações de interação presenciais ou em ambientes
digitais e aprender, na articulação com outras áreas, campos e com os
projetos e escolhas pessoais dos jovens, procedimentos de levantamento,
tratamento e divulgação de dados e informações e o uso desses dados em
produções diversas e na proposição de ações e projetos de natureza
variada, para fomentar o protagonismo juvenil de forma
contextualizada.'' (BNCC, p. 494)

\begin{itemize}
\item
  Nesta atividade recomenda-se uma produção artística conjugada.
  Sugere-se que sejam reproduzidas músicas de escolas e temáticas
  diferentes durante a aula. Cada um dos alunos deverá produzir um
  desenho, tentando reproduzir as sensações que sentiu ao ouvir essa
  música. Não há necessidade de se fazer uma produção trabalhosa, apenas
  lápis e papel bastam. Sugere-se que, após os desenhos concluídos, os
  alunos deverão tentar criar uma narrativa, que aglutine cada um dos
  momentos, paisagens e/ou personagens desenhados.
\end{itemize}

\textbf{Campo de atuação na vida pública}

``No cerne do campo de atuação na vida pública estão a ampliação da
participação em diferentes instâncias da vida pública, a defesa dos
direitos, o domínio básico de textos legais e a discussão e o debate de
ideias, propostas e projetos. {[}...{]}

Ainda no domínio das ênfases, indica-se um conjunto de habilidades que
se relacionam com a análise, discussão, elaboração e desenvolvimento de
propostas de ação e de projetos culturais e de intervenção social.''
(BNCC, p. 494)

\begin{itemize}
\item
  \textit{É possível conhecer, ainda que de forma sutil, a história e a
  cultura de um povo por meio de sua culinária e sua música. Dito isso,
  recomenda-se, para esta outra proposta de atividade que a sala seja
  dividida em grupos e que, cada grupo, deve montar uma Playlist de
  canções típicas de um país, região ou continente, ficando essa
  determinação a critério do professor.}
\end{itemize}

{É interessante uma parceria entre os professores de humanidades
para dar a orientação devida, acerca de elementos culturais, sociais,
históricos, geográficos, econômicos, etc.}

{Deve ser indicado ao aluno estar atento quanto a fluxos
migratórios, influências de minorias étnicas na composição musical.
Elementos pretéritos que perpetuam na música, etc. }

\subsection{Campo jornalístico-midiático}

``Em relação ao campo jornalístico-midiático, espera-se que os jovens
que chegam ao Ensino Médio sejam capazes de: compreender os fatos e
circunstâncias principais relatados; perceber a impossibilidade de
neutralidade absoluta no relato de fatos; adotar procedimentos básicos
de checagem de veracidade de informação; identificar diferentes pontos
de vista diante de questões polêmicas de relevância social; avaliar
argumentos utilizados e posicionar-se em relação a eles de forma ética;
identificar e denunciar discursos de ódio e que envolvam desrespeito aos
Direitos Humanos; e produzir textos jornalísticos variados, tendo em
vista seus contextos de produção e características dos gêneros. Eles
também devem ter condições de analisar estratégias
linguístico-discursivas utilizadas pelos textos publicitários e de
refletir sobre necessidades e condições de consumo.

No Ensino Médio, os jovens precisam aprofundar a análise dos interesses
que movem o campo jornalístico midiático, da relação entre informação e
opinião, com destaque para o fenômeno da pós-verdade, consolidar o
desenvolvimento de habilidades, apropriar-se de mais procedimentos
envolvidos na curadoria de informações, ampliar o contato com projetos
editoriais independentes e tomar consciência de que uma mídia
independente e plural é condição indispensável para a democracia.

Como já destacado, as práticas que têm lugar nas redes sociais têm
tratamento ampliado.'' (BNCC, p. 494-495)

\begin{itemize}
\item
  A dança é um elemento comum a todas as culturas. Entretanto ela
  ultrapassa um mero passatempo, sendo, em muitos casos elemento que
  compõe ritos importantes para a cultura de um povo. Peça para que os
  alunos realizem uma breve pesquisa acerca de danças ritualísticas ao
  redor do mundo. Sugira que montem uma apresentação com o material
  colhido. Nessa atividade é interessante também trazer vídeos onde tais
  danças são registradas.
\end{itemize}

\subsection{Campo artístico-literário}

``No campo artístico-literário busca-se a ampliação do contato e a
análise mais fundamentada de manifestações culturais e artísticas em
geral. Está em jogo a continuidade da formação do leitor literário e do
desenvolvimento da fruição. A análise contextualizada de produções
artísticas e dos textos literários, com destaque para os clássicos,
intensifica-se no Ensino Médio. Gêneros e formas diversas de produções
vinculadas à apreciação de obras artísticas e produções culturais
(resenhas, vlogs e podcasts literários, culturais etc.) ou a formas de
apropriação do texto literário, de produções cinematográficas e teatrais
e de outras manifestações artísticas (remidiações, paródias,
estilizações, videominutos, fanfics etc.) continuam a ser considerados
associados a habilidades técnicas e estéticas mais refinadas.

A escrita literária, por sua vez, ainda que não seja o foco central do
componente de Língua Portuguesa, também se mostra rica em possibilidades
expressivas.'' (BNCC, p. 495-496)

\begin{itemize}
\item
  Retome as passagens em que o narrador descreve Sofia, sublinhe com os
  alunos esses momentos, indicando os recursos que o autor utiliza para
  produzir sensações no leitor. Abra uma roda e pergunte aos alunos se
  eles entendem ser isso proposital, ou não, perguntando se há maneiras
  de se construir essa atmosfera por meio das letras. Recomenda-se,
  então, a leitura de poemas com ritmo marcado. Feito isso, sugira aos
  alunos que redijam um parágrafo no qual utilizarão os recursos
  abordados para provocar sensação no leitor.
\end{itemize}

\subsection{Campo das práticas de estudo e pesquisa }

``O campo das práticas de estudo e pesquisa mantém destaque para os
gêneros e habilidades envolvidos na leitura/escuta e produção de textos
de diferentes áreas do conhecimento e para as habilidades e
procedimentos envolvidos no estudo. Ganham realce também as habilidades
relacionadas à análise, síntese, reflexão, problematização e pesquisa:
estabelecimento de recorte da questão ou problema; seleção de
informações; estabelecimento das condições de coleta de dados para a
realização de levantamentos; realização de pesquisas de diferentes
tipos; tratamento dos dados e informações; e formas de uso e
socialização dos resultados e análises.

Além de fazer uso competente da língua e das outras semioses, os
estudantes devem ter uma atitude investigativa e criativa em relação a
elas e compreender princípios e procedimentos metodológicos que orientam
a produção do conhecimento sobre a língua e as linguagens e a formulação
de regras.'' (BNCC, p. 495-496)

\begin{itemize}
\item
  Para a sedimentação dos conhecimentos musicais, sugerimos que, com
  auxílio dos professores de ciências das naturezas, os alunos procurem
  estudar o som sob sua perspectiva natural e imanente. Isto é, sabemos
  que a música é composta de sons. Mas o que é especificamente um som? E
  como ele é formado?
\end{itemize}

É aconselhável que o professor de ciência das naturezas acompanhe o
aluno em uma investigação acerca da ondulatória, indicando como
comprimento de onda e frequência são determinantes na forma como
percebemos os sons.


\section{Referências complementares}

\begin{quote} \textbf{HOOKS, Bell. Minha dança tem história. São Paulo:
  Boitatá, 2019.}

\textbf{O livro conta a história do pequeno Bibói, um garoto que cresce dentro
  da cultura do hip-hop. Enquanto ele arrasa nas batalhas e nas rimas, vai
  fazendo descobertas sobre masculinidade e sobre ele mesmo.}

\textbf{FARO, Antonio Jose. Pequena história da dança. Rio de Janeiro: Zahar,
  2001.}

\textbf{O livro apresenta uma visão panorâmica da dança, desde o seu surgimento
  até a atualidade, mostrando como ela pode ser usada como forma de arte, de
  entretenimento e de ritual.}

\textbf{BOSI, Ecléa. O tempo vivo da memória: Ensaios de Psicologia. São Paulo:
  Ateliê Editorial, 2013.}

\textbf{Referência em estudos sobre a memória, a autora convida o leitor a
  pensar sobre o que a memória recupera, redime e inspira. Apoiando-se em
  autores clássicos, a obra ajuda a entender o cotidiano das metrópoles, com
  suas contradições entre lembrança e esquecimento.} \end{quote}


\section{Bibliografia comentada}

\begin{quote} \textbf{COELHO, Nelly Novaes. Literatura infantil. Teoria,
análise, didática. São Paulo: Moderna, 2002.} \end{quote}

\textbf{Neste amplo painel das possíveis abordagens e leituras da literatura
infantil e juvenil, a escritora apresenta a necessidade de reflexão e crítica
das questões suscitados por essa produção literária.}

\begin{quote} \textbf{BOSI, Ecléa. Memória e sociedade. São Paulo: Companhia
das Letras, 1994.} \end{quote}

\textbf{Através da fala de pessoas simples, a pesquisadora faz um empolgante
estudo sobre a memória. O livro é recheado de ensinamentos, sensibilidade e
poesia.}

\begin{quote} \textbf{BOURCIER, Paul. História da dança no ocidente. São Paulo:
Martins Fontes, 2001.} \end{quote}

\textbf{A partir de uma documentação rigorosa, o professor de história da dança
mostra a evolução dessa arte desde as primeiras manifestações, há mais de
quinze mil anos, até a nossa época.}

\end{document}


