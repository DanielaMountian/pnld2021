\documentclass{extarticle}
\usepackage{manualdoprofessor}
\usepackage{fichatecnica}
\usepackage{lipsum,media9,graficos}
\usepackage[justification=raggedright]{caption}
\usepackage{bncc}
\usepackage[iluminuras]{../edlab}

\begin{document}


\newcommand{\AutorLivro}{Sidney Rocha}
\newcommand{\TituloLivro}{Sofia}
\newcommand{\Tema}{Ficção, mistério e fantasia}
\newcommand{\Genero}{Romance}
\newcommand{\imagemCapa}{./images/PNLD0054-01.png}
\newcommand{\issnppub}{---}
\newcommand{\issnepub}{---}
% \newcommand{\fichacatalografica}{PNLD0054-00.png}
\newcommand{\colaborador}{\textbf{Fulano de Tal} é uma pessoa incrível e vai fazer um bom serviço.}


\title{\TituloLivro}
\author{\AutorLivro}
\def\authornotes{\colaborador}

\date{}
\maketitle

\begin{abstract}
\lipsum[1-3]
\end{abstract}

\tableofcontents


\begin{abstract}

Este Manual tem como objetivo fornecer subsídios para o trabalho com as
obras literárias \emph{Sofia, de Sidney Rocha.}

Sidney Rocha, nascido em Juazeiro do Norte, Ceará, em 1965, é um escritor 
e editor. Tem vários livros publicados e foi vencedor do Prêmio Jabuti, 
em 2012, pelo romance \textit{O Destino das Metáforas}, e do Prêmio Osman
Lins, em 1985, pelo presente romance, \textit{Sofia}.

Falando nele, \textit{Sofia}, em constante reescrita pelo autor, é um romance
em primeira pessoa de um homem que conhece uma mulher, se apaixona, e passa 
a carregar essa paixão dentro de si. O que os leitores vêem, porém, não é nunca 
a Sofia de carne e osso. As descrições não são físicas, mas sim do efeito que 
essa mulher, de cabelos mágicos, causa por todos os lugares por onde passa, 
sobretudo no coração do narrador apaixonado. 

Tudo fica mais interessante e intrigante quando percebemos que "Sofia", além da 
pessoa física, a mulher real por quem o narrador de nome desconhecido se apaixonou, 
pode também ecoar no significado que esta palavra tem na língua grega antiga: 
"sabedoria". Um dos grandes encantos do livro é o fato de ser o leitor o 
responsável por responder à questão: quem é Sofia? Quem é, para quem lê, essa 
personagem que movimenta as emoções de tudo e todos ao seu redor?

Esperamos, com este manual, poder auxiliar o professor em seu trabalho, certamente
muito prazeroso, em sala de aula!


\end{abstract}

\section{Proposta de atividades I}

\subsection{Pré-leitura}


%(EM13LGG302) Compreender e posicionar-se criticamente diante de diversas
%visões de mundo presentes nos discursos em diferentes linguagens,
%levando em conta seus contextos de produção e de circulação.%

%(EM13LGG704) Apropriar-se criticamente de processos de pesquisa e busca
%de informação, por meio de ferramentas e dos novos formatos de produção
%e distribuição do conhecimento na cultura de rede.%

%(EM13LP10) Selecionar informações, dados e argumentos em fontes
%confiáveis, impressas e digitais, e utilizá-los de forma referenciada,
%para que o texto a ser produzido tenha um nível de aprofundamento
%adequado (para além do senso comum) e contemple a sustentação das
%posições defendidas.%

%(EM13LP19) Compartilhar gostos, interesses, práticas culturais, temas/
%problemas/questões que despertam maior interesse ou preocupação,
%respeitando e valorizando diferenças, como forma de identificar
%afinidades e interesses comuns, como também de organizar e/ou participar
%de grupos, clubes, oficinas e afins.

\textbf{1.} NA PRÉ-LEITURA A IDEIA É SENSIBILIZAR OS ALUNOS E ALUNAS DE
MANEIRA POÉTICA. PROPOMOS PARA ESSE MOMENTO ALGUMAS MÚSICAS CALMAS E
RELAXANTES,. COMO A CANÇÃO \emph{ROMÂNTICOS,.} DE VANDER LEE.

PROPOMOS TAMBÉM A LEITURA DO POEMA \emph{ISMÁLIA},. DE ALPHONSUS DE
GUIMARÃES.

\subsection{Leitura}

% \begin{longtable}[]{@{}l@{}}
% \toprule
% \endhead
% \begin{minipage}[t]{0.97\columnwidth}\raggedright
% \textbf{De olho na BNCC}

% \begin{quote}
% \textbf{As atividades sugeridas nesta subseção contemplam as seguintes
% habilidades propostas pela BNCC: }

% (EM13LGG103) Analisar, de maneira cada vez mais aprofundada, o
% funcionamento das linguagens, para interpretar e produzir criticamente
% discursos em textos de diversas semioses.

% (EM13LP02) Estabelecer relações entre as partes do texto, tanto na
% produção como na recepção, considerando a construção composicional e o
% estilo do gênero, usando/reconhecendo adequadamente elementos e recursos
% coesivos diversos que contribuam para a coerência, a continuidade do
% texto e sua progressão temática, e organizando informações, tendo em
% vista as condições de produção e as relações lógico-discursivas
% envolvidas (causa/efeito ou consequência; tese/argumentos;
% problema/solução; definição/exemplos etc.).

% (EM13LP48) Perceber as peculiaridades estruturais e estilísticas de
% diferentes gêneros literários (a apreensão pessoal do cotidiano nas
% crônicas, a manifestação livre e subjetiva do eu lírico diante do mundo
% nos poemas, a múltipla perspectiva da vida humana e social dos romances,
% a dimensão política e social de textos da literatura marginal e da
% periferia etc.) para experimentar os diferentes ângulos de apreensão do
% indivíduo e do mundo pela literatura.
% \end{quote}\strut
% \end{minipage}\tabularnewline
% \bottomrule
% \end{longtable}

\textbf{2. }

AO LONGO DA LEITURA DO LIVRO,. PROPOMOS QUE OS PROFESSORES E PROFESSORAS
INCENTIVEM OS ALUNOS A OBSERVAREM COMO A DANÇA ESTÁ PRESENTE AO LONGO DO
ROMANCE.
 
PODE SER DESENVOLVIDA,. PARALELAMENTE AO PROCESSO DE LEITURA,. UMA
PESQUISA BIOGRAFICA DE DANÇARINAS FAMOSAS. AS PESQUISAS PODERÃO SER
SOCIALIZADAS POR MEIO DE ATIVIDADES COMO REPORTAGENS EM VÍDEO.
 
É MUITO IMPORTANTE LEMBRAR QUE SIDNEY ROCHA É UM DOS MAIS IMPORTANTES E
RECONHECIDOS ESCRITORES DA ATUALIDADE.
 
É UM AUTOR MUITO RECOMENDADO PARA QUEM ESTÁ INICIANDO A LEITURA DE
LIVROS DE FICÇÃO,. PRINCIPALMENTE DE AUTORES BRASILEIROS CONTEMPORÂNEOS.

\subsection{Pós-leitura}

% \begin{longtable}[]{@{}l@{}}
% \toprule
% \endhead
% \begin{minipage}[t]{0.97\columnwidth}\raggedright
% \textbf{De olho na BNCC}

% \begin{quote}
% \textbf{As atividades sugeridas nesta subseção contemplam as seguintes
% habilidades propostas pela BNCC: }
% \end{quote}

% (EM13LGG102) Analisar visões de mundo, conflitos de interesse,
% preconceitos e ideologias presentes nos discursos veiculados nas
% diferentes mídias como forma de ampliar suas as possibilidades de
% explicação e interpretação crítica da realidade.

% (EM13LGG303) Debater questões polêmicas de relevância social, analisando
% diferentes argumentos e opiniões manifestados, para negociar e sustentar
% posições, formular propostas, e intervir e tomar decisões
% democraticamente sustentadas, que levem em conta o bem comum e os
% Direitos Humanos, a consciência socioambiental e o consumo responsável
% em âmbito local, regional e global.

% (EM13LGG402) Empregar, nas interações sociais, a variedade e o estilo de
% língua adequados à situação comunicativa, ao(s) interlocutor(es) e ao
% gênero do discurso, respeitando os usos das línguas por esse(s)
% interlocutor(es) e combatendo situações de preconceito linguístico.

% (EM13LGG703) Utilizar diferentes linguagens, mídias e ferramentas
% digitais em processos de produção coletiva, colaborativa e projetos
% autorais em ambientes digitais.

% (EM13LP13) Planejar, produzir, revisar, editar, reescrever e avaliar
% textos escritos e multissemióticos, considerando sua adequação às
% condições de produção do texto, no que diz respeito ao lugar social a
% ser assumido e à imagem que se pretende passar a respeito de si mesmo,
% ao leitor pretendido, ao veículo e mídia em que o texto ou produção
% cultural vai circular, ao contexto imediato e sócio histórico mais
% geral, ao gênero textual em questão e suas regularidades, à variedade
% linguística apropriada a esse contexto e ao uso do conhecimento dos
% aspectos notacionais (ortografia padrão, pontuação adequada, mecanismos
% de concordância nominal e verbal, regência verbal etc.), sempre que o
% contexto o exigir.

% (EM13LP14) Produzir e analisar textos orais, considerando sua adequação
% aos contextos de produção, à forma composicional e ao estilo do gênero
% em questão, à clareza, à progressão temática e à variedade linguística
% empregada, como também aos elementos relacionados à fala (modulação de
% voz, entonação, ritmo, altura e intensidade, respiração etc.) e à
% cinestesia (postura corporal, movimentos e gestualidade significativa,
% expressão facial, contato de olho com plateia etc.).

% (EM13LP28) Resumir e resenhar textos, com o manejo adequado das vozes
% envolvidas (do autor da obra e do resenhador), por meio do uso de
% paráfrases, marcas do discurso reportado e citações, para uso em textos
% de divulgação de estudos e pesquisas.

% (EM13LP29) Realizar pesquisas de diferentes tipos (bibliográfica, de
% campo, experimento científico, levantamento de dados etc.), usando
% fontes abertas e confiáveis, registrando o processo e comunicando os
% resultados, tendo em vista os objetivos colocados e demais elementos do
% contexto de produção, como forma de compreender como o conhecimento
% científico é produzido e apropriar-se dos procedimentos e dos gêneros
% textuais envolvidos na realização de pesquisas.

% (EM13LP52) Produzir apresentações e comentários apreciativos e críticos
% sobre livros, filmes, discos, canções, espetáculos de teatro e dança,
% exposições etc. (resenhas, vlogs e podcasts literários e artísticos,
% playlists comentadas, fanzines, e-zines etc.).\strut
% \end{minipage}\tabularnewline
% \bottomrule
% \end{longtable}

\textbf{3.} PROPOMOS UMA OFICINA DE ESCRITA POÉTICA,. COM PLAYLIST
COMPOSTA PELOS ALUNOS PARA AMBIENTAÇÃO. O OBJETIVO É EXPLORAR A
IMPORTÂNCIA DA SONORIDADE PARA O GÊNERO LÍRICO.

\section{Proposta de atividades II}

A obra \emph{Sofia} possibilita trabalhos interdisciplinares e
integradores de diferentes campos do saber e áreas de conhecimento. A
seguir, propomos algumas atividades que podem ser desenvolvidas
conjuntamente com professores de outras áreas. Além das habilidades de
Linguagens e suas Tecnologias e de Língua Portuguesa, indicadas nas
etapas da seção anterior e válidas também para esta, listamos a seguir
as habilidades de outras áreas, presentes na abordagem interdisciplinar:

% \begin{longtable}[]{@{}l@{}}
% \toprule
% \endhead
% \begin{minipage}[t]{0.97\columnwidth}\raggedright
% \textbf{De olho na BNCC}

% \begin{quote}
% \textbf{As atividades sugeridas nesta subseção contemplam as seguintes
% habilidades propostas pela BNCC: }
% \end{quote}

% (EM13CNT201) Analisar e utilizar modelos científicos, propostos em
% diferentes épocas e culturas para avaliar distintas explicações sobre o
% surgimento e a evolução da Vida, da Terra e do Universo.

% (EM13CNT303) Interpretar textos de divulgação científica que tratem de
% temáticas das Ciências da Natureza, disponíveis em diferentes mídias,
% considerando a apresentação dos dados, a consistência dos argumentos e a
% coerência das conclusões, visando construir estratégias de seleção de
% fontes confiáveis de informações.

% (EM13CHS101) Analisar e comparar diferentes fontes e narrativas
% expressas em diversas linguagens, com vistas à compreensão e à crítica
% de ideias filosóficas e processos e eventos históricos, geográficos,
% políticos, econômicos, sociais, ambientais e culturais.

% (EM13CHS102) Identificar, analisar e discutir as circunstâncias
% históricas, geográficas, políticas, econômicas, sociais, ambientais e
% culturais da emergência de matrizes conceituais hegemônicas
% (etnocentrismo, evolução, modernidade etc.), comparando-as a narrativas
% que contemplem outros agentes e discursos.

% (EM13CHS106) Utilizar as linguagens cartográfica, gráfica e iconográfica
% e de diferentes gêneros textuais e as tecnologias digitais de informação
% e comunicação de forma crítica, significativa, reflexiva e ética nas
% diversas práticas sociais (incluindo as escolares) para se comunicar,
% acessar e disseminar informações, produzir conhecimentos, resolver
% problemas e exercer protagonismo e autoria na vida pessoal e coletiva.

% (EM13CHS401) Identificar e analisar as relações entre sujeitos, grupos e
% classes sociais diante das transformações técnicas, tecnológicas e
% informacionais e das novas formas de trabalho ao longo do tempo, em
% diferentes espaços e contextos.\strut
% \end{minipage}\tabularnewline
% \bottomrule
% \end{longtable}

\subsection{Pré-leitura}

\textbf{4.} A ATIVIDADE QUE ELABORAMOS COMEÇA COM UMA PESQUISA MEDIADA
PELOS PROFESSORES DE CIÊNCIAS HUMANAS.
 
A IDEIA É REALIZAR UMA PESQUISA SOBRE O PAPEL DAS MUSAS E DA DEUSA
AFRODITE NA MITOLOGIA.
 
\subsection{Leitura}

\textbf{5.} EM SEGUIDA,. PROFESSORES DE CIÊNCIAS NATURAIS PODEM AJUDAR A
ENTENDER O DINAMISMO DOS MOVIMENTOS CORPORAIS DOS DANÇARINOS.
 
OS ALUNOS VERÃO COMO SE ORIGINAM O EQUILÍBRIO E A FORÇA FÍSICA
NECESSÁRIOS À DANÇA (COM A PRODUÇÃO DE UMA REPORTAGEM EM VÍDEO AO
FINAL).
 
NESSE MOMENTO, É FUNDAMENTAL UM PARALELO ENTRE AS LINGUAGENS EXPRESSIVAS
DO CORPO, AS CIÊNCIAS HUMANAS E AS CIÊNCIAS DA NATUREZA.
 
A VIDA HUMANA NA TERRA SÓ É POSSÍVEL GRAÇAS AO CORPO. É ELE QUE NOS
TORNA VIVOS E, JUNTO COM A MENTE, CONCRETIZA A NOSSA EXISTÊNCIA.~
 
POR ISSO, A DANÇA ESTÁ LIGADA À FORMAÇÃO DA IDENTIDADE, A PARTIR DA
EXPRESSÃO ARTÍSTICA.~
 
O INDIVÍDUO SE EXPRESSA E SE TORNA CAPAZ POR MEIO DA ARTE, QUE TORNA
POSSÍVEL O TRABALHO COM AS COMPETÊNCIAS SOCIOEMOCIONAIS.

\subsection{Pós-leitura}

\textbf{6.} POR ÚLTIMO,. DEPOIS DA LEITURA,. PROPOMOS A REALIZAÇÃO DE UM
SARAU PARA APRESENTAÇÃO DOS POEMAS E DO VÍDEO FEITOS NA ATIVIDADE UM
(1). ASSIM COMO A SOCIALIZAÇÃO DO PROGRAMA DE DIVULGAÇÃO CIENTÍFICA
FEITO NA ATIVIDADE 2.

\section{Aprofundamento}


Ao chegar ao Ensino Médio, é necessário que os estudantes se aprofundem
na compreensão das múltiplas linguagens e, sobretudo, da linguagem
literária. Em relação à literatura, a BNCC traz as seguintes
considerações:

``{[}...{]} a leitura do texto literário, que ocupa o centro do trabalho
no Ensino Fundamental, deve permanecer nuclear também no Ensino Médio.
Por força de certa simplificação didática, as biografias de autores, as
características de épocas, os resumos e outros gêneros artísticos
substitutivos, como o cinema e as HQs, têm relegado o texto literário a
um plano secundário do ensino. Assim, é importante não só (re)colocá-lo
como ponto de partida para o trabalho com a literatura, como
intensificar seu convívio com os estudantes. Como linguagem
artisticamente organizada, a literatura enriquece nossa percepção e
nossa visão de mundo. Mediante arranjos especiais das palavras, ela cria
um universo que nos permite aumentar nossa capacidade de ver e sentir.
Nesse sentido, a literatura possibilita uma ampliação da nossa visão do
mundo, ajuda-nos não só a ver mais, mas a colocar em questão muito do
que estamos vendo/vivenciando.'' (Brasil, 2018, p. 491)

Nesta seção, desenvolvemos um trabalho de aprofundamento que, em diálogo
com a formação continuada de professores, oferece subsídios para a
abordagem do texto literário.

JÁ NO TÍTULO DA OBRA PODEMOS ESPERAR ALGO COMO A PRESENÇA DE UMA
PERSONAGEM FEMININA,. . MAS DURANTE A LEITURA,. AOS POUCOS,. COMEÇAMOS A
SUSPEITAR DE ALGO.


SOFIA É UMA MULHER VISTA PELOS OLHOS DE UM HOMEM APAIXONADO. SEUS
CABELOS DANÇAM E DEIXAM UM RASTRO DE MÚSICA.
 
ELA É UM TIPO DE ESPÍRITO LIVRE QUE ANDA PELO MUNDO DESDE OS DOIS ANOS.
 
O RELACIONAMENTO ENTRE O NARRADOR E SOFIA TORNA-SE PRESENTE NA NARRATIVA
AOS POUCOS,. POIS O ESCRITOR É MUITO CAUTELOSO EM SE ABRIR.
 
A PARTIR DE IMAGENS QUE RETORNAM POR VÁRIAS VEZES,. COMO OS CABELOS QUE
DANÇAM,. VAMOS ADENTRANDO NA HISTÓRIA.
 
POR ONDE SOFIA PASSA,. SEU CABELO DEIXA UM RASTRO DE DANÇA E MÚSICA. TEM
O PODER DE ENCANTAR AS PESSOAS.

A VIDA HUMANA NA TERRA SÓ É POSSÍVEL GRAÇAS AO CORPO. É ELE QUE NOS
TORNA VIVOS E, JUNTO COM A MENTE, CONCRETIZA A NOSSA EXISTÊNCIA.~
 
POR ISSO, A DANÇA ESTÁ LIGADA À FORMAÇÃO DA IDENTIDADE, A PARTIR DA
EXPRESSÃO ARTÍSTICA.
 
O INDIVÍDUO SE EXPRESSA E SE TORNA CAPAZ POR MEIO DA ARTE.
 
PELA DANÇA, O INDIVÍDUO É CAPAZ DE DEMONSTRAR AQUILO QUE PENSA, ENTENDE,
OU SEJA, ELE É CAPAZ DE DEMONSTRAR O SEUS CONHECIMENTOS E HABILIDADES.
 
A DANÇA PODE EXPRESSAR SENTIMENTOS E OUTROS ASPECTOS, SOBRETUDO NA ÁREA
DA EDUCAÇÃO.
 
ATUALMENTE JÁ EXISTE UMA MELHOR COMPREENSÃO A RESPEITO DOS VALORES
FORMATIVOS E CRIATIVOS DA DANÇA, PARA ALÉM DOS MOVIMENTOS CORPORAIS.
 
A DANÇA VEM GANHANDO CADA VEZ MAIS ESPAÇO PELOS BENEFÍCIOS EM RELAÇÃO À
AUTOESTIMA E AO TRABALHO COM AS COMPETÊNCIAS SOCIOEMOCIONAIS E
CONSTRUÇÃO DA CIDADANIA E DA DEMOCRACIA.
 
TRATA-SE DE UMA FORMA MAIS EFETIVA DE APRENDER E EXPRESSAR-SE
CRIATIVAMENTE POR MEIO DO MOVIMENTO. A DANÇA, TEMA CENTRAL DE ``SOFIA'',
PODE SER UMA PORTA DE ENTRADA DOS ALUNOS NO UNIVERSO DA ARTE, DO
MOVIMENTO E DA CONSTRUÇÃO DA CIDADANIA.

SOFIA É MISTERIOSA,. ATRAENTE.
 
SOFIA NÃO PASSARIA DE UMA INVENÇÃO DO NARRADOR? ELA EXISTIU MESMO?
 
AS LEMBRANÇAS PARECEM MAIS SAUDADE INVENTADA DO QUE SEPARAÇÃO REAL.
 
NO FUNDO,. O ROMANCE \emph{SOFIA} APRESENTA ENREDO TRADICIONAL. TRATA-SE
DE HOMEM QUE CONHECE UMA MULHER; SE APAIXONA; PASSA A CARREGAR ESSA
PAIXÃO DENTRO DE SI.
 
MAS A FORMA QUE ESSA PAIXÃO APARECE NO LIVRO NÃO É A MESMA DE UMA
HISTÓRIA DE AMOR TRADICIONAL. É ISSO QUE NOS DEIXA PRESOS AO LIVRO DO
COMEÇO AO FIM.

A SOFIA,. DE CARNE E OSSO,. FICA POR CONTA DA IMAGINAÇÃO DOS LEITORES E
LEITORAS . ESSE É UM DOS GRANDES ENCANTOS DO LIVRO
 
AOS POUCOS VAMOS MERGULHANDO NOS ENCANTOS QUE SOFIA VAI DEIXANDO NO
NARRADOR DA HISTÓRIA.
 
TRATA-SE DE UMA MULHER FASCINANTE,. QUE VAI PROVOCANDO ALGUMAS CRISES NO
PERSONAGEM-NARRADOR,. APÓS RÁPIDOS E FORTUITOS ENCONTROS.
 
O NARRADOR DESSA HISTÓRIA NÃO TEM UM NOME DEFINIDO. ENTRAMOS EM CONTATO
APENAS COM SUA CONSCIÊNCIA.
 
ELE QUER COMPARTILHAR COM OS LEITORES AS COISAS QUE PENSOU,. AS CRISES
QUE SOFREU,. TODAS DESENCADEADAS PELA VISÃO DE SOFIA
 
CONTAR OU NÃO CONTAR OS DETALHES DE UM RELACIONAMENTO? ESSA PARECE A
GRANDE QUESTÃO QUE INQUIETA O NARRADOR.

O NOME DA PROTAGONISTA, ``SOFIA'', REMONTA AO GREGO, NO SÉCULO IV A.C.,
E SIGNIFICA SABEDORIA.
 
TORNOU-SE COMUM NOS PAÍSES ORTODOXOS ORIENTAIS. TORNOU-SE MUITO POPULAR
TAMBÉM NO OCIDENTE. FOI ATÉ COMUM NA EUROPA CONTINENTAL DURANTE A IDADE
MÉDIA E NO INÍCIO DA ERA MODERNA. FOI POPULARIZADO ENTRE A POPULAÇÃO
INGLESA A PARTIR DO NOME DE UMA PERSONAGEM DO ROMANCE
\href{https://pt.wikipedia.org/wiki/The_History_of_Tom_Jones,_a_Foundling}{\emph{TOM
JONES}} DE \href{https://pt.wikipedia.org/wiki/Henry_Fielding}{HENRY
FIELDING} (1794).

..

..

SIDNEY ROCHA É UM DOS MAIS IMPORTANTES E RECONHECIDOS ESCRITORES DA
ATUALIDADE.
 
SUAS QUALIDADES,. PARA QUEM ESTÁ INICIANDO A DESCOBERTA DA LITERATURA,.
SÃO FUNDAMENTAIS.
 
A PRIMEIRA DELAS É A CAPACIDADE DE CONTAR UMA BOA HISTÓRIA,. MESMO QUE O
TEMA NÃO SEJA ORIGINAL OU POLÊMICO.
 
SEU ESTILO DE ESCRITA É BASTANTE DIRETO. SUAS FRASES CURTAS,. QUASE SEM
METÁFORAS,. ENVOLVEM OS LEITORES DESDE O INÍCIO DA LEITURA. ELE
DESENVOLVE O ASSUNTO COM SIMPLICIDADE E LEVEZA.
 
OUTRA QUALIDADE DE SIDNEY ROCHA É A MANEIRA COMO CONSTRÓI OS
PERSONAGENS. SÃO FIGURAS QUASE DESMATERIALIZADAS,. MAS QUE EXERCEM UM
GRANDE FASCÍNIO NOS LEITORES E LEITORAS.~
 
SEUS LIVROS EXPLORAM A SENSIBILIDADE DAS PESSOAS,. CAUSAM AQUELAS PAUSAS
MEDITATIVAS E UMA VONTADE ENORME DE CONTINUAR CONHECENDO SUAS OBRAS.

\section{Referências complementares}

\begin{quote}
\textbf{HOOKS, Bell. Minha dança tem história. São Paulo: Boitatá,
2019.}

\textbf{O livro conta a história do pequeno Bibói, um garoto que cresce
dentro da cultura do hip-hop. Enquanto ele arrasa nas batalhas e nas
rimas, vai fazendo descobertas sobre masculinidade e sobre ele mesmo.}

\textbf{FARO, Antonio Jose. Pequena história da dança. Rio de Janeiro:
Zahar, 2001.}

\textbf{O livro apresenta uma visão panorâmica da dança, desde o seu
surgimento até a atualidade, mostrando como ela pode ser usada como
forma de arte, de entretenimento e de ritual.}

\textbf{BOSI, Ecléa. O tempo vivo da memória: Ensaios de Psicologia. São
Paulo: Ateliê Editorial, 2013.}

\textbf{Referência em estudos sobre a memória, a autora convida o leitor
a pensar sobre o que a memória recupera, redime e inspira. Apoiando-se
em autores clássicos, a obra ajuda a entender o cotidiano das
metrópoles, com suas contradições entre lembrança e esquecimento.}
\end{quote}


\section{Bibliografia comentada}

\begin{quote}
\textbf{COELHO, Nelly Novaes. Literatura infantil. Teoria, análise,
didática. São Paulo: Moderna, 2002.}
\end{quote}

\textbf{Neste amplo painel das possíveis abordagens e leituras da
literatura infantil e juvenil, a escritora apresenta a necessidade de
reflexão e crítica das questões suscitados por essa produção literária.}

\begin{quote}
\textbf{BOSI, Ecléa. Memória e sociedade. São Paulo: Companhia das
Letras, 1994.}
\end{quote}

\textbf{Através da fala de pessoas simples, a pesquisadora faz um
empolgante estudo sobre a memória. O livro é recheado de ensinamentos,
sensibilidade e poesia.}

\begin{quote}
\textbf{BOURCIER, Paul. História da dança no ocidente. São Paulo:
Martins Fontes, 2001.}
\end{quote}

\textbf{A partir de uma documentação rigorosa, o professor de história
da dança mostra a evolução dessa arte desde as primeiras manifestações,
há mais de quinze mil anos, até a nossa época.}

\end{document}


