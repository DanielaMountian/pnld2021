\documentclass[12pt]{extarticle} 
\usepackage{manualdoprofessor}
\usepackage{fichatecnica} 
\usepackage{lipsum,media9,graficos}
\usepackage[justification=raggedright]{caption} 
\usepackage[one]{bncc}
\usepackage[iluminuras]{../edlab}
\usepackage{verse}


\begin{document}


\newcommand{\AutorLivro}{Sidney Rocha}
\newcommand{\TituloLivro}{\textit{Sofia}}
\newcommand{\Tema}{Ficção, mistério e fantasia}
\newcommand{\Genero}{Romance}
\newcommand{\imagemCapa}{./images/PNLD0054-01.png}
\newcommand{\issnppub}{---}
\newcommand{\issnepub}{---}
%\newcommand{\fichacatalografica}{PNLD0054-00.png}
\newcommand{\colaborador}{Vicente Castro e Reniêr Cândido de Vasconcelos Silva}


\title{\TituloLivro} \author{\AutorLivro} \def\authornotes{\colaborador}

\date{} \maketitle

\baselineskip=1.20\baselineskip\par

\begin{abstract}

Este Manual tem como objetivo fornecer subsídios para o trabalho com a obra
  literária \textit{Sofia} (2014), de Sidney Rocha.

Sidney Rocha, nascido em Juazeiro do Norte, Ceará, em 1965, é escritor e
  editor, além e roteirista de cinema e televisão. 
  Foi vencedor do Prêmio Jabuti, em
  2012, com o romance \textit{O Destino das Metáforas}, e do Prêmio Osman Lins,
  em 1985, com o presente romance, \textit{Sofia}. Entre 
  seus livros publicados, estão 
  \textit{Matriuska} (2009), 
  \textit{O destino das metáforas} (2011) e
  \textit{A estética da diferença} (2018).


\textit{Sofia} foi reescrito três vezes pelo autor. Trata-se um romance
  em primeira pessoa sobre a paixão de um homem por uma mulher misteriosa, 
  que nunca se apresenta materialmente. As descrições sobre ela nunca são físicas.
  Mesmo assim, causam forte efeito por conta de seus efeitos mágicos. 

Um dos grandes encantos do livro é o fato de ser o
  leitor responsável por responder à questão: quem é \textit{Sofia}? Quem é, para quem
  lê, essa personagem que movimenta as emoções de tudo e todos ao seu redor?

Esperamos, com este manual, poder auxiliar o professor em seu trabalho,
  certamente muito prazeroso, em sala de aula!


\end{abstract}

\tableofcontents


\section{Proposta de atividades I}

\subsection{Pré-leitura}

\BNCC{EM13LGG302}
\BNCC{EM13LGG704}
\BNCC{EM13LP10}
\BNCC{EM13LP19}

\paragraph{Tema} Abrindo-se para a experiência lírica romântica.

\paragraph{Conteúdo} Sensibilização a obras de conteúdo lírico romântico,
tais quais \emph{Sofia}.

\paragraph{Objetivo} Introduzir os estudantes à experiência artística
partindo de uma abordagem mais focada na sensibilidade estética do que
na elaboração de opiniões sobre a obra. 

\paragraph{Justificativa} Gostaríamos de apresentar ao professor um
ponto de vista sobre a experiência em geral que pode ser usado
também para a experiência estética, como a que será trabalhada em 
sala de aula nessas aulas. 

Para isso, nos utilizamos de alguns excertos do texto \emph{Notas sobre
experiência e o saber de experiência} do filósofo espanhol Jorge
Larrosa Bondía. 

Em suas palavras: ``a experiência é o que nos passa, 
o que nos acontece, o que nos toca. Não o que se passa, não o que acontece, ou o que toca. A cada dia se 
passam muitas coisas, porém, ao mesmo tempo, quase nada nos acontece. 
Dir-se-ia que tudo o que se passa está organizado para que nada nos aconteça.''

Justificando essa falta de experiência nos dias de hoje, 
o autor defende que isso acontece por conta da supervalorização
da informação, como se esta tivesse um valor de sabedoria. Segundo Bondía: 

\begin{quote}
O sujeito da informação sabe
muitas coisas, passa seu tempo buscando informação,
o que mais o preocupa é não ter bastante informação;
cada vez sabe mais, cada vez está melhor informado,
porém, com essa obsessão pela informação e pelo saber (mas saber não no sentido 
de “sabedoria”, mas no
sentido de “estar informado”), o que consegue é que
nada lhe aconteça.

A experiência é cada vez mais
rara por excesso de opinião. O sujeito moderno é um
sujeito informado que, além disso, opina. É alguém
que tem uma opinião supostamente pessoal e supostamente própria e, às vezes, 
supostamente crítica sobre tudo o que se passa, sobre tudo aquilo de que tem
informação. Para nós, a opinião, como a informação,
converteu-se em um imperativo. Em nossa arrogância, passamos a vida opinando sobre qualquer coisa
sobre que nos sentimos informados. E se alguém não
tem opinião, se não tem uma posição própria sobre o
que se passa, se não tem um julgamento preparado
sobre qualquer coisa que se lhe apresente, sente-se em
falso, como se lhe faltasse algo essencial.
\end{quote}

\paragraph{Metodologia}

\begin{enumerate}

  \item
  Afim de introduzir a discussão sobre a experiência, sugerimos
  que o professor pergunte aos estudantes como eles definiriam,
  em um verbo, a experiência estética, ou seja, o ato de ouvir uma 
  música, ir ao museu, ver um quadro ou uma escultura e ler um livro.
  Escreva os verbos na lousa, ressaltando seus significados.

  \item Neste segundo momento, indique aos estudantes que eles 
  terão uma experiência estética. Se possível, apague as luzes da
  sala a fim de criar um clima confortável para todos. Sugerimos
  que o professor declame, primeiro, o poema \textit{Ismália}, de 
  Alphonsus Guimarães transcrito abaixo.

\begin{verse}
%Precisa tirar o recuo do primeiro verso.
Quando Ismália enlouqueceu,\\*
Pôs-se na torre a sonhar...\\*
Viu uma lua no céu,\\*
Viu outra lua no mar.\\!


No sonho em que se perdeu,\\
Banhou-se toda em luar...\\
Queria subir ao céu,\\
Queria descer ao mar...\\!


E, no desvario seu,\\
Na torre pôs-se a cantar...\\
Estava perto do céu,\\
Estava longe do mar...\\!


E como um anjo pendeu\\
As asas para voar...\\
Queria a lua do céu,\\
Queria a lua do mar...\\!


As asas que Deus lhe deu\\
Ruflaram de par em par...\\
Sua alma subiu ao céu,\\
Seu corpo desceu ao mar...
\end{verse}

  Depois, sugerimos que execute a canção homônima, \textit{Ismália}, do rapper Emicida,
  sobretudo o trecho em que a atriz brasileira Fernanda Montenegro declama
  o mesmo poema. E, por fim, a canção \textit{Românticos}, de Vander Lee.

  \item
  Ao fim das apreciações, na medida do possível mantendo o conforto
  da experiência anterior, peça, num exercício de escrita livre,
  que os estudantes expressem como foi esta experiência. Procurando
  não descrever as obras, mas as sensações que elas lhes causaram, 
  como sentimentos, lembranças, cores, pessoas, paisagens... 
  Estas impressões podem ser compartilhadas ao fim da atividade para 
  que percebam como, uma mesma obra, pode suscitar universos de
  significação diferentes, ainda que no mesmo assunto, em diferentes
  pessoas.

\end{enumerate}

\paragraph{Tempo estimado} Duas aulas de 50 minutos.


\subsection{Leitura}

\BNCC{EM13LGG103}
\BNCC{EM13LP02}
\BNCC{EM13LP48}


\paragraph{Tema} A literatura e a dança.

\paragraph{Conteúdo} Compreensão da relação entre as linguagens artísticas
da literatura e da dança e de como elas podem estar ligadas uma à outra. 

\paragraph{Objetivo} Habilitar os e as estudantes a perceber a presença
da dança no decorrer do romance. Junto a isso, introduzir os alunos
no universo da dança a partir de obras que dialoguem com a literatura,
como o espetáculo \textit{Une après midi...}, de Nicolas Le Riche
(2014), inspirado na peça musical \textit{Prélude à l'après-midi
d'un faune} de Claude Debussy (1894), por sua
vez, uma interpretação musical do poema \textit{L'après-midi d'un faune} 
de Stéphane Mallarmé (1876).

\paragraph{Justificativa} Sabemos que desde a antiguidade ocidental existe
uma relação criativa entre as linguagens artísticas, como verificamos
no recurso da écfrase --- grosso modo, quando um poema é escrito 
a partir de uma imagem, quadro, escultura ou paisagem. 

Esta relação criativa não se limita, porem, à palavra-imagem da écfrase. 
Temos o ótimo exemplo do poema publicado em 1876 pelo poeta francês Stéphane Mallarmé intitulado
\textit{L'après-midi d'un faune}, que conta a história de um fauno em meio
a um bosque numa tarde quente de verão que após fracassar em sua investida
às ninfas, dorme e sonha com suas conquistas. 

Inspirado pela experiência artística da leitura do poema, vale dizer, ilustrado 
pelo pintor impressionista Édouard Manet, o músico e compositor também
francês Debussy compôs uma peça musical intitulada \textit{Prélude à l'après-midi
d'un faune}, considerada pela crítica como inauguradora da música moderna. 

Em 2014, o bailarino e coreógrafo Nicolas le Riche, inspirado nas duas obras,
musical e poética, dirigiu o espetáculo \textit{Une après-midi...}, verdadeiro 
sucesso na França, por sua vez seguindo a criação de Vaslav Nijinsky
que em 1912, poucos anos após a morte de Mallarmé, coreografou a peça de homônima
em relação ao poema. 

Podemos concluir, com esse repertório, que havia indicações de dança dentro da
obra literária escrita por Stéphane Mallarmé em 1876, percebidas e desenvolvidas
por Vaslav Nijinsky. Esta apresentação deve deixar
claro aos estudantes que a relação entre as linguagens artísticas pode
ser muito criativa, mesmo que os instrumentos de cada uma --- escrita, no caso
da poesia, e corpo, no caso da dança --- pareçam tão distantes.

\paragraph{Metodologia}

\begin{enumerate}

  \item
  Para introduzir ao assunto, peça que os e as estudantes identifiquem, nas primeiras
  páginas do livro, indicações do universo da dança. Podem ser substantivos, verbos ou adjetivos. 
  Escreva as respostas na lousa. O professor ou a professora pode começar usando como exemplo
  o primeiro parágrafo do livro:

  \textit{Fico querendo contar pra alguém sobre ela,
sobre ter havido criatura tão doce quanto Sofia e falar
dos seus cabelos, quando \emph{dançavam} na minha frente e
eu sorria. Muitas vezes, não podia me conter, então batia
palmas, maravilhado, só porque Sofia se mexeu. Numa
noite dessas, ficamos, eu e ela, a bater palmas um para o
outro. Eu, porque ela se movia. Ela, porque eu batia palmas e 
Sofia batia palmas porque não entendia e adorava
não entender nada. Contar isso para as pessoas é ficar ainda 
mais distante dela e dos seus cabelos \emph{bailarinos}.}

  \item
  Neste segundo momento, pergunte se os e as estudantes já assistiram a um
  espetáculo de dança. Se sim, qual e qual história era contada. 
  Apresente"-os, então, um trecho do espetáculo \textit{Une après-midi...}
  de Nicolas Le Riche, disponível no YouTube. Antes, faça uma curta 
  contextualização sobre o percurso das obras. Se possível, declame
  alguns dos trechos do poema de Mallarmé que deixamos abaixo para introduzí"-los
  ao universo. 




\begin{verse}
Quero perpetuar essas ninfas.\\
                                    \hspace{5cm}             Tão claro\\
Seu ligeiro encarnado a voltear no ar\\
Espesso de mormaço e sonos.\\
                             \hspace{5cm}                    Sonhei ou…?\\!

Borra de muita noite, a dúvida se acaba\\
Em mil ramos sutis a imitar a mata,\\
Prova infeliz de que eu sozinho me ofertava\\
À guisa de triunfo a ausência ideal das rosas.\\!

Reflitamos…\\
        \hspace{2cm}         E se essas moças, minhas glosas,\\
Não passarem de sonho e senso fabulosos?\\
Fauno, dos olhos da mais casta, azuis e frios,\\
Flui a ilusão com uma fonte em prantos, rios:\\
Mas, em contraste, o hálito da outra, arfante,\\
Não é o sopro de um dia quente nos teus pelos?
\end{verse}


  \item
  Por fim, os e as estudantes devem fazer uma pesquisa biográfica
  sobre os principais bailarinos e bailarinas, coreógrafos 
  e coreógrafas do Brasil, bem como as principais obras desta linguagem. 
  Os resultados das pesquisas podem ser apresentados num blog coletivo da
  turma.

\end{enumerate}

\paragraph{Tempo estimado} Duas aulas de 50 minutos.



\Image{Oficina de dança contemporânea (Mallu Silva/Labfoto; CC-BY 2.0)}{PNLD0054-05.png}

\SideImage{Espectáculo da Companhia de Dança Deborah Colker (Rosano Mauro; CC-BY 2.0)}{PNLD0054-04.png}



% \Image{Ana Botafogo é atriz e uma das mais conhecidas bailarinas brasileiras. (Roberto Filho; CC-BY 2.1)}{PNLD0054-08.png}


\subsection{Pós-leitura}

\BNCC{EM13LGG102}
\BNCC{EM13LGG303}
\BNCC{EM13LGG402}

\BNCCC{EM13LGG703
EM13LP13;
EM13LP14;
EM13LP28;
EM13LP29;
EM13LP52.}

\paragraph{Tema} Experimentando os ritmos, na poesia e no corpo.

\paragraph{Conteúdo} Oficina de escrita poética onde os e as estudantes 
articularão o que aprenderam nas últimas aulas.

\paragraph{Objetivo} Proporcionar aos e às estudantes um ambiente 
propício à criação literária --- mais especificamente, poética. Espera"-se
que eles e elas, seguindo os exemplos apresentados nas ultimas aulas,
sintam"-se à vontade para criar suas próprias obras.
\Image{Pina Bausch foi uma importante dançarina e coreógrafa alemã. (Raphael Labbé; CC-BY-SA 3.0)}{PNLD0054-06.png}



\paragraph{Justificativa} Seguindo o que propomos na primeira aula dessa
sessão de atividades quando trouxemos algumas falas do filósofo Jorge 
Larrosa Bondía sobre a experiência, queremos, agora finalizar novamente
com ele.

Em seu texto, Bondía diz que a informação não é o mesmo que sabedoria.
A sabedoria vem da experiência, que ainda segundo ele, é a capacidade
de se expor  aos acontecimentos da vida, às obras de arte, por exemplo.
É preciso, para que haja experiência, paixão. O autor afirma que:

\begin{quote}

Se a experiência é 
o que nos acontece, e se o sujeito da experiência é um território de passagem,
então a experiência é uma paixão. Não se pode captar
a experiência a partir de uma lógica da ação, a partir
de uma reflexão do sujeito sobre si mesmo enquanto
sujeito agente, a partir de uma teoria das condições de
possibilidade da ação, mas a partir de uma lógica da
paixão, uma reflexão do sujeito sobre si mesmo enquanto sujeito passional.
\Image{Cecília Kerche é uma famosa bailarina brasileira, que assumiu em 1986 a posição de primeira bailarina do Teatro Municipal do Rio de Janeiro. (Fernando Frazão/Agência Brasil; CC-BY 2.0)}{PNLD0054-07.png}

\end{quote}

Sabendo que ``Sofia'' no romance aqui trabalhado também faz referência 
ao mito grego da Sabedoria, não podemos deixar de ver uma relação
entre os dois textos. Podemos dizer que a experiência do narrador
é profunda pois ele está apaixonado, entregue e aberto, como percebemos
durante a leitura à sua amada, Sofia. 

Neste sentido, podemos concluir que tanto a experiência amorosa da personagem
do romance produz algo --- um livro e a própria concretização do amor ---
quanto a experiência artística tem suas reverberações. E agora é o momento de 
suscitá"-las nos e nas estudantes de uma forma mais contundente.


\begin{enumerate}
\paragraph{Metodologia}

  \item
  O professor ou a professora deve criar uma \textit{playlist} participativa em seu celular
  e pedir que os e as estudantes insiram músicas de suas escolhas. De quaisquer
  estilos e épocas. Por isso, é importante que o ou a professora 
  organize"-as minimamente por gêneros, para que não haja uma mudança tão abrupta
  na hora da oficina. 


  \item
  O professor ou a professora deve introduzir a atividade pedindo que os e as estudantes
  retomem as produções feitas nas últimas aulas: as reações ao poema \textit{Ismália}
  e à música \textit{Românticos}, na primeira aula, e ao poema \textit{L'après-midi d'un faune},
  em sua versão original e na leitura do espetáculo de dança, da última. Agora,
  tendo a opções de desenvolver algo que foi iniciado em uma destas escritas,
  o professor deve pedi"-los para escrever um poema, levando em conta a
  atmosfera que as músicas, tocadas em volume médio para baixo, irão lhes sugerir. 
  Proponha que eles e elas, espalhados pelo espaço físico da sala, 
  busquem se valer da sonoridade ouvida para imprimir ritmo e
  melodia no texto escrito. O objetivo é explorar a importância da sonoridade
  para o gênero lírico.

  \item
  Ao final das produções, as obras devem ser compartilhadas entre a turma. 
  Uma sugestão que pode ser feita é que, em casa, tentem interpretar o poema
  escrito por meio da dança, com o auxílio das músicas que inspiraram sua criação.
  Aqueles que optarem por fazer esta atividade, serão convidados a compartilhar 
  com a turma suas produções gravadas em vídeo.


\end{enumerate}

\paragraph{Tempo estimado} Duas aulas de 50 minutos.




\section{Proposta de atividades II}

\BNCC{EM13CNT201}
\BNCC{EM13CNT303}
\BNCC{EM13CHS101}
\BNCCC{EM13CHS102;
      EM13CHS106;
      EM13CHS401}

A obra \emph{\textit{Sofia}} possibilita trabalhos interdisciplinares e integradores de
diferentes campos do saber e áreas de conhecimento. A seguir, propomos algumas
atividades que podem ser desenvolvidas conjuntamente com professores de outras
áreas. %Além das habilidades de Linguagens e suas Tecnologias e de Língua
%Portuguesa, indicadas nas etapas da seção anterior e válidas também para esta,
%listamos a seguir as habilidades de outras áreas, presentes na abordagem interdisciplinar:


\subsection{Pré-leitura} 

\paragraph{Tema} A influência dos deuses na vida dos humanos.

\paragraph{Conteúdo} Compreensão acerca do papel das musas e da deusa Afrodite
na mitologia grega clássica e consequentemente na vidas dos humanos que faziam parte dessa cultura.

\paragraph{Objetivo} Provocar a reflexão dos educandos e educandas acerca de
uma realidade diferente das suas, mas que ainda assim guarda alguns padrões 
de comportamento. 
\Image{Quadro "O Nascimento de Vênus", de Sandro Botticceli representa a deusa do amor, cujo nome grego é Afrodite. (Sandro Botticelli; Domínio Público)}{PNLD0054-09.png}

\paragraph{Justificativa} É verdade que desde sempre o ser humano teve suas
crenças e mitologias próprias. Os indígenas das Américas, os africanos das diversas
partes do continentes, os povos judeus e os nórdicos... Cada um deles tem uma
mitologia diferente, composta por divindades com poderes e áreas de atuação
específica na vida dos seres humanos que lhes cultuam. Eis algo que unem todas
estas manifestações: as divindades não existem por si só, mas sim em relação 
àqueles que nelas creem e lhes pedem por ajuda nas dificuldades quotidianas da vida
humana. 







Não é diferente com os povos da Antiguidade clássica grega. Esta cultura, que tanto
influenciou na construção da cultura ocidental, possui um vasto panteão
de deuses, alguns mais conhecidos devido à indústria cultural, e outros
deixados de lado.
\SideImage{Estátua de Afrodite exposta no Museu Arqueológico de Nápoles. 
É uma cópia da estátua grega de 310-200 A.C. 
(Museu Arqueológico de Nápoles; CC-BY 3.1)}{PNLD0054-11.png}

Obras deste período como a \textit{Teogonia} de Hesíodo é uma ótima forma de
ser introduzido a este universo. Lá, pode"-se ter contato com as histórias
de cada uma destas divindades, desde como nasceram até aquilo que lhes
particulariza em relação aos outros. 

Visto que o assunto principal do romance estudado é a paixão de um homem por
uma moça chamada Sofia, não podemos deixar de fazer uma relação com o universo 
helenístico, já que Sofia lhe inspira a escrever tal qual as musas inspiravam os
poetas.

As musas eram as personificações e as patrocinadoras das representações de discursos 
em verso ou \textit{mousike}, ``arte das musas'', de onde provém ``música''. No período 
arcaico, antes de que os livros estivessem amplamente disponíveis, isto incluía quase 
todas as formas de ensinamento: o primeiro livro grego de astronomia, por Tales de Mileto, 
estava escrito em hexametros dactílicos, igual que muitas outras obras da filosofia pré-socrática. 
Tanto Platão com os pitagóricos incluíam explicitamente a filosofia como um subgênero de 
\emph{mousike}. Heródoto, cujo principal meio de expressão era a recitação pública, chamou 
a cada um dos nove livros de suas Histórias com o nome de uma musa diferente.

\paragraph{Metodologia}

\begin{enumerate}

  \item
  O professor ou a professora deve, num primeiro momento, pedir que os e as estudantes façam
  uma pesquisa em dez grupos sobre o papel das musas e da deusa Afrodite na mitologia. 
  Devem se debruçar sobre como essas representantes da arte e das forças de atração agem nas lendas antigas 
  influenciando a vida dos humanos. É interessante abrir uma roda de conversa, 
  indicando aos educandos os paralelos e interpretações que podem se fazer desses 
  contos antigos e como esses são ecoam na vida de cada um de nós.
\SideImage{Estátua de Afrodite exposta na "Biblioteca Nazionale Marciana", em Veneza (Biblioteca Nazionale Marciana; CC-BY 3.0)}{PNLD0054-10.png}

  \item
  Cada grupo deve escolher uma das musas, que contam nove ao todo, e um ficará com a deusa
  Afrodite. Ficarão responsáveis, assim, de organizar as informações e representações
  mais importantes de cada uma. 

  \item
  Num último momento da aula, o professor deve dispor a sala em formato de 
  roda onde cada grupo deve apresentar o resultado das pesquisas já 
  organizados. Ao fim das apresentações, os e as estudantes serão estimados
  a pensar esta mitologia a partir de suas realidades. Apesar de serem tão antigos,
  há, nestes contos, alguma reminiscência nos nossos dias? E em outras culturas
  próximas a nós, há algo parecido? É importante sempre se ressaltar
  a necessidade do respeito à diversidade em atividades com este tom, podendo
  caber ao professor ou professora corrigir algum comentário que infrinja
  este princípio.

\end{enumerate}

\paragraph{Tempo estimado} Duas aulas de 50 minutos.

 
\subsection{Leitura}

\paragraph{Tema} Como funciona a dança?

\paragraph{Conteúdo} Compreensão, nos termos das ciências naturais, da estruturação do movimento.

\paragraph{Objetivo} Habilitar os e as estudantes a compreender, com o auxílio dos professores
de ciências naturais, os mecanismos que permitem aos corpos físicos se movimentar, e, sendo o caso,
dançar.

\paragraph{Justificativa} Partindo do pressuposto de que a dança é um dos elementos estruturantes
do romance \textit{Sofia}, verificado na leitura dos trechos em a personagem é descrita por seus
movimentos corporais, é interessante para os e as estudantes entender, de uma perspectiva
das ciências naturais, como se dá o movimento de corpos físicos. 

Professores e professoras de ciências naturais podem ajudar a
entender o dinamismo dos movimentos corporais dos dançarinos. Os alunos verão
como se originam o equilíbrio e a força física necessários à dança. Nesse momento, é fundamental
um paralelo entre as linguagens expressivas do corpo, as ciências humanas e as
ciências da natureza. 

A vida humana na terra só é possível graças ao corpo. É
ele que nos torna vivos e, junto com a mente, concretiza a nossa existência. Por isso, a dança está ligada à formação da identidade, a partir da expressão
artística.  O indivíduo se expressa e se torna capaz por meio da arte, que
torna possível o trabalho com as competências sociais e emocionais.


\begin{enumerate}
\paragraph{Metodologia}

  \item
  O professor ou professora de ciências naturais deve apresentar os 
  principais fundamentos da mecânica, área das ciências que estuda
  o movimento. É importante que o assunto seja preparado de modo
  a caber na extensão da aula. Esta primeira parte será sobretudo expositiva.

  \item
  Depois de serem expostos os principais conceitos e cálculos,
  os e as estudantes devem fazer exercícios básicos sobre o tema, 
  se possível, usando seus próprios corpos e objetos do contexto
  da sala de aula para tal. 

  \item
  Nesta última etapa da aula, o professor ou a professora deve indicar que a turma inteira
  será responsável pela realização de uma reportagem sobre a dinâmica dos 
  corpos dos dançarinos, partindo dos termos das ciências naturais. A turma deve
  se dividir em áreas de trabalho, como: roteiro, pesquisa, equipe técnica de
  filmagem, \textit{casting} de convidados (se for o caso).

\end{enumerate}

\paragraph{Tempo estimado} Duas aulas de 50 minutos.


\subsection{Pós-leitura}

\paragraph{Tema} Compartilhando os resultados.

\paragraph{Conteúdo} Socialização das produções realizadas nas atividades de leitura de Português e das outras disciplinas.

\paragraph{Objetivo} Fazer circular o conhecimento interdisciplinar de modo
que os e as próprias estudantes tenham autonomia na preparação do evento.


\paragraph{Metodologia}

\begin{enumerate}

  \item
  O professor ou a professora deve informar aos e às estudantes da atividade de conclusão que será feita.
  A preparação do sarau será de sua responsabilidade, desde a organização física
  (decoração e disposição dos projetos no espaço da sala de aula) até a curadoria
  sobre quais são os projetos que serão apresentados e como o serão.

  \item Os e as estudantes
  podem aproveitar esta aula também para fazer alterações em suas produções, bem 
  como propor novas versões delas: transformar um poema numa apresentação de dança, por exemplo.
  As atividades em grupo devem ser incentivadas. 

\end{enumerate}

\paragraph{Tempo estimado} Duas aulas de 50 minutos.


\section{Aprofundamento}

Sidney Rocha é um dos mais importantes e reconhecidos
escritores da atualidade. Entre suas qualidades, podemos destacar a capacidade
de contar uma boa história. Seu estilo de escrita é bastante direto. Suas
frases curtas, quase sem metáforas, envolvem os leitores desde o início da
leitura. Ele desenvolve o assunto com simplicidade e leveza. Outra qualidade de
Sidney Rocha é a maneira como constrói as personagens. São figuras quase
desmaterializadas, mas que exercem um grande fascínio nos leitores e leitoras. 

Seus livros exploram a sensibilidade das pessoas, causando aquelas pausas
meditativas e uma vontade enorme de continuar conhecendo suas obras.  No livro
\textit{Sofia}, a dança é um tema central, podendo ser uma porta de entrada dos alunos
no universo da arte, do movimento e da construção da cidadania. 

Ao explicar seu processo criativo, Sidney Rocha diz em uma 
\href{http://www.vacatussa.com/entrevista-sidney-rocha/}{entrevista}\footnote{
  Revista Vakatusa, por Thiago Corrêa Ramos, 28 de abril de 2014.} para para 
o jornalista Hugo Viana:

%Precisa trocar a nota de rodapé de letra para número.

\begin{quote}
Então, eu era um romancista de vinte e poucos anos com muitas ideias-fixas (e feitas), todas inúteis, mas descobri logo que, num romance, quase nada do que você acha que sabe, você termina por usar. E a grande maioria do que fica, de fato, na página, tem a ver com escolhas mais duras e menos cosméticas, que tem a ver com a vida de verdade, que é disso que é feito um bom romance. E, paradoxo: ter lido Flaubert nesse tempo foi uma sorte. O bom romance tem mais a ver com perdas que com ganhos. Sofia é sobre essas perdas. 
\end{quote}

O livro foi reescrito e teve três versões distintas ao longo de duas décadas, explica com 
humor o autor:

\begin{quote}
 Não tenho nenhuma dificuldade em buscar melhoras no texto. Alguns livros na minha estante, mesmo, gosto de brincar com eles e retirar uma ou duas palavras de uma frase de Faulkner, por exemplo, e experimentar. Alterar ou acrescentar um parágrafo em James Joyce. Pena que o inverso não possa ocorrer.
\end{quote}


O romance conta a história do relacionamento entre o narrador e uma
jovem chamada \textit{Sofia}, que se torna
presente gradualmente ao longo da narrativa, na medida em que  
o escritor, cauteloso, revela seus pensamentos.  

A partir da construção de imagens, imaginamos como a garota é pintada como a própria personificação da música. A dança, como 
um de seus elementos básicos, é o elemento que descreve seus
cabelos. Uma sinfonia que, por meio de uma
sonata, retorna após contra-argumentos, derivações e floreios.  


\Image{Sidney Rocha (Arquivo do autor}{PNLD0054-03.png}


O autor conta que sempre se sentiu um escritor inconcluso, mesmo quando terminava algo:

\begin{quote}

Nesses anos vim alterando o texto, 
e isso já é bem visível já na segunda edição. Agora o romance 
se transformou em um novo romance. Quem o leu antes ‘se’ reconhecerá 
nele tanto quanto o novo leitor, de novíssima geração.

\end{quote}

Por onde \textit{Sofia} passa é possível sentir mais do que ver. 
A suas principais características são a dança e a música. 
Ela tem o poder encantatório, que arrebata o narrador. Ao vê-la,
entende que a vitalidade da vida depende de seu corpo. 
\textit{Sofia} é também misteriosa e atraente. A moça 
não passaria de uma invenção do narrador? De um delírio, talvez?  

A trama é simples: um homem que conhece uma mulher, apaixona-se e passa a
carregar essa paixão dentro de si. A diferença é que \textit{Sofia} é uma personificação
da música; é a musa que inspira e o objeto pelo qual a inspiração é
transmitida. Sidney Rocha afirma que tudo que escreve está conectado à um mundo interior, mas, ao mesmo tempo, inteiramente 
voltado para fora.

A escolha do nome da personagem principal é chave para a interpretação 
do romance. O nome ``Sofia'' é de origem grega e significa sabedoria, 
mas que também tem o sentido arcaico ``daquele que domina uma arte 
ou uma técnica, como frequentemente os poetas ou os músicos''. 
O livro apresenta assim a beleza de uma mulher e a arte do amor
associadas à música e a dança, características típicas das 
musas gregas.

Com isso, o autor indica que
a música é a sabedoria máxima, afinal, como a própria \textit{Sofia} personagem, ela
desde muito cedo nos atrai, convida-nos a querer estar próximo a ela,
percebemos seu poder de atração e de compreensão nos momentos de tristeza e a
potência que ela dá às horas de celebração.  

Assim, \textit{Sofia} como música é a
sabedoria máxima, capaz de se comunicar e definir sentimentos
desconhecidos ou que não podem ser traduzidos em fala. Transborda-se, como 
um som universal. É encantadora, afinal, e turva a visão do expectador que 
busca o olhar direto.  

Com isso, a forma com que a paixão aparece no livro não é a mesma
de uma história de amor tradicional. O tom sublime dessa atração é 
marcado.  \textit{Sofia} de carne e osso fica por conta da imaginação
dos leitores e leitoras. Esse é um dos grandes encantos do livro. Aos poucos
mergulhamos nos encantos que \textit{Sofia} enreda o narrador como uma espécie 
de figura marinha, cujos cantos atraíam marinheiros.
Trata-se, simplesmente, de uma mulher fascinante, que provoca crises no
personagem-narrador, após rápidos e fortuitos encontros.  

O narrador por sua vez não tem seu nome revelado. Conhecemos apenas sua
consciência, identificada com o próprio autor
que compartilha com os leitores pensamentos contínuos e afetos que sofreu
pela visão de \textit{Sofia}. Como partilhar o que sentimos? 
Essa parece a grande questão inquietante do narrador.  



\textbf{Campo da vida pessoal}

\begin{comment}
``O campo da vida pessoal pretende funcionar como espaço de articulações
e sínteses das aprendizagens de outros campos postas a serviço dos
projetos de vida dos estudantes. As práticas de linguagem privilegiadas
nesse campo relacionam-se com a ampliação do saber sobre si, tendo em
vista as condições que cercam a vida contemporânea e as condições
juvenis no Brasil e no mundo.

Está em questão também possibilitar vivências significativas de práticas
colaborativas em situações de interação presenciais ou em ambientes
digitais e aprender, na articulação com outras áreas, campos e com os
projetos e escolhas pessoais dos jovens, procedimentos de levantamento,
tratamento e divulgação de dados e informações e o uso desses dados em
produções diversas e na proposição de ações e projetos de natureza
variada, para fomentar o protagonismo juvenil de forma
contextualizada.'' (BNCC, p. 494)
\end{comment}

\begin{itemize}
\item
  Nesta atividade recomenda-se uma produção artística conjugada.
  Sugere-se que sejam reproduzidas músicas de escolas e temáticas
  diferentes durante a aula. Cada um dos alunos deverá produzir um
  desenho, tentando reproduzir as sensações que sentiu ao ouvir essa
  música. Não há necessidade de se fazer uma produção trabalhosa, apenas
  lápis e papel bastam. Sugere-se que, após os desenhos concluídos, os
  alunos deverão tentar criar uma narrativa, que aglutine cada um dos
  momentos, paisagens e/ou personagens desenhados.
\end{itemize}

\textbf{Campo de atuação na vida pública}

\begin{comment}
``No cerne do campo de atuação na vida pública estão a ampliação da
participação em diferentes instâncias da vida pública, a defesa dos
direitos, o domínio básico de textos legais e a discussão e o debate de
ideias, propostas e projetos. {[}...{]}

Ainda no domínio das ênfases, indica-se um conjunto de habilidades que
se relacionam com a análise, discussão, elaboração e desenvolvimento de
propostas de ação e de projetos culturais e de intervenção social.''
(BNCC, p. 494)
\end{comment}

É possível conhecer, ainda que de forma sutil, a história e a
  cultura de um povo por meio de sua culinária e sua música. Dito isso,
  recomenda-se, para esta outra proposta de atividade que a sala seja
  dividida em grupos e que, cada grupo, deve montar uma \emph{playlist} de
  canções típicas de um país, região ou continente, ficando essa
  determinação a critério do professor.

É interessante uma parceria entre os professores de humanidades
para dar a orientação devida, acerca de elementos culturais, sociais,
históricos, geográficos, econômicos, etc.

{Deve ser indicado ao aluno estar atento quanto a fluxos
migratórios, influências de minorias étnicas na composição musical.
Elementos pretéritos que perpetuam na música, etc. }

\subsection{Campo jornalístico-midiático}

\begin{comment}
``Em relação ao campo jornalístico-midiático, espera-se que os jovens
que chegam ao Ensino Médio sejam capazes de: compreender os fatos e
circunstâncias principais relatados; perceber a impossibilidade de
neutralidade absoluta no relato de fatos; adotar procedimentos básicos
de checagem de veracidade de informação; identificar diferentes pontos
de vista diante de questões polêmicas de relevância social; avaliar
argumentos utilizados e posicionar-se em relação a eles de forma ética;
identificar e denunciar discursos de ódio e que envolvam desrespeito aos
Direitos Humanos; e produzir textos jornalísticos variados, tendo em
vista seus contextos de produção e características dos gêneros. Eles
também devem ter condições de analisar estratégias
linguístico-discursivas utilizadas pelos textos publicitários e de
refletir sobre necessidades e condições de consumo.

No Ensino Médio, os jovens precisam aprofundar a análise dos interesses
que movem o campo jornalístico midiático, da relação entre informação e
opinião, com destaque para o fenômeno da pós-verdade, consolidar o
desenvolvimento de habilidades, apropriar-se de mais procedimentos
envolvidos na curadoria de informações, ampliar o contato com projetos
editoriais independentes e tomar consciência de que uma mídia
independente e plural é condição indispensável para a democracia.

Como já destacado, as práticas que têm lugar nas redes sociais têm
tratamento ampliado.'' (BNCC, p. 494-495)
\end{comment}


  A dança é um elemento comum a todas as culturas. Ela não é um mero passatempo, sendo, em muitos casos elemento que
  compõe rituais importantes para a cultura de um povo. Peça para que os
  alunos realizem uma breve pesquisa acerca de danças ritualísticas ao
  redor do mundo. Sugira que montem uma apresentação com o material
  colhido. Nessa atividade é interessante também trazer vídeos onde tais danças são registradas.


\subsection{Campo artístico-literário}


\begin{comment}
``No campo artístico-literário busca-se a ampliação do contato e a
análise mais fundamentada de manifestações culturais e artísticas em
geral. Está em jogo a continuidade da formação do leitor literário e do
desenvolvimento da fruição. A análise contextualizada de produções
artísticas e dos textos literários, com destaque para os clássicos,
intensifica-se no Ensino Médio. Gêneros e formas diversas de produções
vinculadas à apreciação de obras artísticas e produções culturais
(resenhas, vlogs e podcasts literários, culturais etc.) ou a formas de
apropriação do texto literário, de produções cinematográficas e teatrais
e de outras manifestações artísticas (remidiações, paródias,
estilizações, videominutos, fanfics etc.) continuam a ser considerados
associados a habilidades técnicas e estéticas mais refinadas.

A escrita literária, por sua vez, ainda que não seja o foco central do
componente de Língua Portuguesa, também se mostra rica em possibilidades
expressivas.'' (BNCC, p. 495-496)
\end{comment}


  Retome as passagens em que o narrador descreve Sofia, sublinhe com os
  alunos esses momentos, indicando os recursos que o autor utiliza para
  produzir sensações no leitor. Abra uma roda e pergunte aos alunos se
  eles entendem ser isso proposital, ou não, perguntando se há maneiras
  de se construir essa atmosfera por meio das letras. Recomenda-se,
  então, a leitura de poemas com ritmo marcado. Feito isso, sugira aos
  alunos que redijam um parágrafo no qual utilizarão os recursos
  abordados para provocar sensação no leitor.


\subsection{Campo das práticas de estudo e pesquisa }

\begin{comment}
``O campo das práticas de estudo e pesquisa mantém destaque para os
gêneros e habilidades envolvidos na leitura/escuta e produção de textos
de diferentes áreas do conhecimento e para as habilidades e
procedimentos envolvidos no estudo. Ganham realce também as habilidades
relacionadas à análise, síntese, reflexão, problematização e pesquisa:
estabelecimento de recorte da questão ou problema; seleção de
informações; estabelecimento das condições de coleta de dados para a
realização de levantamentos; realização de pesquisas de diferentes
tipos; tratamento dos dados e informações; e formas de uso e
socialização dos resultados e análises.

Além de fazer uso competente da língua e das outras semioses, os
estudantes devem ter uma atitude investigativa e criativa em relação a
elas e compreender princípios e procedimentos metodológicos que orientam
a produção do conhecimento sobre a língua e as linguagens e a formulação
de regras.'' (BNCC, p. 495-496)
\end{comment}

Para a sedimentação dos conhecimentos musicais, sugerimos que, com auxílio dos professores de ciências das naturezas, os alunos procurem estudar o som sob sua perspectiva natural e imanente. Isto é, sabemos que a música é composta de sons. Mas o que é especificamente um som? E como ele é formado?

É aconselhável que o professor de ciência das naturezas acompanhe o aluno em uma investigação acerca da ondulatória, indicando como comprimento de onda e frequência são determinantes na forma como percebemos os sons.


\section{Referências complementares}

\begin{quote} \textsc{HOOKS}, Bell. \emph{Minha dança tem história}. São Paulo:
  Boitatá, 2019.

O livro conta a história do pequeno Bibói, um garoto que cresce dentro
  da cultura do hip-hop. Enquanto ele arrasa nas batalhas e nas rimas, vai
  fazendo descobertas sobre masculinidade e sobre ele mesmo.

\textsc{FARO}, Antonio Jose. \emph{Pequena história da dança}. Rio de Janeiro: Zahar,
  2001.

O livro apresenta uma visão panorâmica da dança, desde o seu surgimento
  até a atualidade, mostrando como ela pode ser usada como forma de arte, de
  entretenimento e de ritual.

\textsc{BOSI}, Ecléa. \emph{O tempo vivo da memória: Ensaios de Psicologia}. São Paulo:
  Ateliê Editorial, 2013.

Referência em estudos sobre a memória, a autora convida o leitor a
  pensar sobre o que a memória recupera, redime e inspira. Apoiando-se em
  autores clássicos, a obra ajuda a entender o cotidiano das metrópoles, com
  suas contradições entre lembrança e esquecimento. \end{quote}


\section{Bibliografia comentada}

\begin{itemize}

\item  
\textsc{COELHO}, Nelly Novaes. Literatura infantil. \emph{Teoria,
análise, didática}. São Paulo: Moderna, 2002.

Neste amplo painel das possíveis abordagens e leituras da literatura
infantil e juvenil, a escritora apresenta a necessidade de reflexão e crítica
das questões suscitados por essa produção literária.

\item
\textsc{BOSI}, Ecléa. \emph{Memória e sociedade.} São Paulo: Companhia
das Letras, 1994.

Através da fala de pessoas simples, a pesquisadora faz um empolgante
estudo sobre a memória. O livro é recheado de ensinamentos, sensibilidade e
poesia.

\item
\textsc{BOURCIER}, Paul. \emph{História da dança no ocidente.} São Paulo:
Martins Fontes, 2001. 

A partir de uma documentação rigorosa, o professor de história da dança
mostra a evolução dessa arte desde as primeiras manifestações, há mais de
quinze mil anos, até a nossa época.

\end{itemize}

\end{document}


