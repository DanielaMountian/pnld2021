\documentclass[12pt]{extarticle}
\usepackage{manualdoprofessor}
\usepackage{fichatecnica}
\usepackage{lipsum,media9,graficos}
\usepackage[justification=raggedright]{caption}
\usepackage[one]{bncc}
\usepackage[alameda]{../edlab}

\begin{document}


\newcommand{\AutorLivro}{Bernardo Kucinski}
\newcommand{\TituloLivro}{Júlia – Nos campos conflagrados do Senhor}
\newcommand{\Tema}{Ficção, mistério e fantasia}
\newcommand{\Genero}{Romance}
\newcommand{\imagemCapa}{./images/PNLD0053-01.png}
\newcommand{\issnppub}{---}
\newcommand{\issnepub}{---}
% \newcommand{\fichacatalografica}{PNLD0053-00.png}
\newcommand{\colaborador}{{Haroldo Ceravolo}}


\title{\TituloLivro}
\author{\AutorLivro}
\def\authornotes{\colaborador}

\date{}
\maketitle





\begin{abstract}\addcontentsline{toc}{section}{Carta ao professor}

Com este manual, apresentaremos algumas sugestões para abordar o
conteúdo do livro de \emph{Júlia - nos campos conflagrados do Senhor}
com suas alunas e seus alunos. São propostas, que você pode adaptar à
vontade, conforme à sua realidade e a de seus estudantes, para quem, de
fato, coloca o processo educativo em prática e que conhece, melhor que
ninguém, o dia a dia do trabalho de formar leitores.

\emph{Júlia} é um livro que trata de forma ficcional um período difícil
de nossa história, o regime militar instituído em 1964. O fato de ser
uma novela, e não um livro de história, tende a favorecer uma leitura
aberta ao diálogo e, ao mesmo tempo, engajada emocionalmente por parte
de adolescentes e jovens.

Assim, a leitura de \emph{Júlia}, a nosso ver, abre múltiplos caminhos
para pensar tanto processos de linguagem quanto para apresentar questões
de outras disciplinas, como história e geografia, por exemplo, de modo
transdisciplinar. \emph{Júlia} é uma novela, como dissemos, ou, se você
preferir, um romance curto, escrito de modo bastante convidativo à
reflexão sobre a escrita e as marcas do tempo na vida do indivíduo. Tem
como principal protagonista uma jovem que busca entender o passado da
própria família e também como esse passado se articula com a história do
país.

Essas questões, que em \emph{Júlia} estão situadas no período da
ditadura militar, são pertinentes para qualquer leitor e para qualquer
estudante, que também precisa refletir sobre o espaço e o tempo em que
vive. Júlia inicia um processo de investigação que inclui observação
atenta de episódios do cotidiano, conversas com pessoas mais velhas,
pesquisa em documentos pessoais e de arquivo e análise de suas memórias
para compor esse passado. Esse processo é muito rico e produtivo, pois,
durante essa verdadeira reconstrução da própria história, essa jovem
amadurece e se constitui como um indivíduo mais maduro e consciente de
seu passado e do passado do Brasil.

\Image{Depredação da sede do Sindicato dos Metalúrgicos em 1964.; Wikipedia; CC-BY-NC.}{PNLD0053-11}


Como na ficção de tradição realista, muitos dos eventos mencionados em
\emph{Júlia} são reais, escorando a vida de personagens inventadas. Essa
diferenciação nem sempre é clara para os estudantes, e aprender a
identificar os elementos da realidade como forma de compor histórias
ficcionais é apenas uma das possibilidades que a leitura da obra trará.

A ditadura militar é hoje um tema que precisa ser tratado na sala de
aula, mas há muitos bloqueios, impostos a partir de fora da escola, ao
trabalho que os professores e professoras precisam executar, em respeito
à própria função que exercem na sociedade, a de educadores. Acreditamos
que a abordagem do tema a partir de um livro de ficção, de um escritor
consagrado como B. Kucinski, pode quebrar barreiras que os processos
ideológicos criaram artificialmente.

Por fim, gostaríamos de dizer que \emph{Júlia} não foi escrita como uma
obra juvenil, mas desde o primeiro momento em que entregou o texto à
editora, o escritor B. Kucinski, que se aposentou como professor de
jornalismo na Universidade de São Paulo, afirmou imaginar que ela
poderia ser uma leitura de estudantes do ensino médio, por conta de
algumas características, como a temática difícil mas necessária, a
questão da busca pela história pessoal pela personagem e a organização
da narrativa em dois tempos paralelos.

\Image{Anúncio do AI-5; Jornal ÚLtima Hora; CC-BY-NC.}{PNLD0053-04}


Acreditamos que essa análise é precisa e esperamos que este livro seja
uma leitura formativa e, ao mesmo tempo, prazerosa para os alunos e as
alunas.

Boas leituras e bom trabalho!

\end{abstract}


\tableofcontents


\section{Sugestões de Atividades I}

\paragraph{Tema} Como contar uma história? A ideia de verossimilhança. As
estratégias do escritor para criar tempos diferentes numa narrativa. A
ditadura militar como história do Brasil e dos brasileiros. As
estratégias de leitura de uma novela como \emph{Júlia}.

\paragraph{Objetivos e justificativa}

Esse primeiro grupo de habilidades buscas estimular o desenvolvimento de algumas habilidades da BNCC, entre as quais destacamos a de relacionar o texto, tanto na produção quanto na recepção, com suas condições de produção e contexto de circulação. 

\BNCC{EM13LP01}

A primeira atividade que apresentamos foi imaginada prioritariamente
para a condução por professores de língua portuguesa. Buscamos, com ela,
favorecer o mergulho dos estudantes nessa leitura, tanto no tema ou
entrecho da novela em si, quanto nos elementos formais que permitem a
construção da ideia de verossimilhança num romance ficcional.

Uma das exigências de qualquer texto, seja uma placa de trânsito, seja
um romance com milhares de páginas, é a de responder ao que a
linguística chama de ``metarregra da coerência''. Na leitura literária,
sobretudo na ficção de tradição realista, na qual podemos enquadrar
\emph{Júlia}, para que essa metarregra seja observada, o texto deve
responder ao que os críticos costumam chamar de ``verossimilhança'':
mais do que reproduzir o real, o texto deve representar o real de forma
que os elementos ficcionais sejam entendidos com uma representação da
realidade convincente. Dito de outra forma, a história contada não
precisa ter acontecido, mas precisa, sim, parecer ter acontecido.

Nossa leitura de \emph{Júlia} vai apresentar uma série de recursos
usados pelo autor para criar essa verossimilhança, formas de escrita que
servem para a reconstituição de histórias de vida e do país, seja pela
via da ficção, ou seja, imaginando episódios e personagens que não
necessariamente existiram, seja na não ficcional, em que se espera um
firme compromisso bem mais estrito do escritor com os fatos observados.

\subsection{Pré-leitura: o Regime Militar}

A leitura do texto de \emph{Júlia} torna-se imediatamente mais
instigante quando apresentado o contexto do seu enquadramento histórico,
ou seja, quando temos uma dimensão temporal do que significou a ditadura
militar.

O primeiro passo é mostrar que o regime durou 21 anos, mais do que a
idade da maioria dos leitores em sala de aula (exceto se estivermos numa
sala de educação de jovens e adultos).

Isso significa que as alunas e alunos que estão lendo a obra, caso
tivessem nascido no primeiro dia do regime militar, ou seja, 31 de março
de 1964, ainda estariam vivendo sob ele. Isso significa um período longo
do ponto de vista pessoal, e também que esse regime não é totalmente
homogêneo, passando por mudanças durante esses anos todos.


\Image{Artistas protestam contra a ditadura militar em fevereiro de 1968. Na imagem, Tônia Carrero, Eva Wilma, Odete Lara, Norma Bengell e Cacilda Becker; Wikipedia; CC-BY-NC.}{PNLD0053-10}

Por outro lado, o fato de o presidente deixar de ser um militar em 1985
não elimina, imediatamente, sua herança. Um período tão longo como esse
deixa marcas profundas não só no passado, mas também no futuro.

De todo modo, ainda hoje, sabemos que a ditadura militar é uma história
pouco conhecida e que apresenta episódios importantes ainda não bem
estudados ou compreendidos mesmo por especialistas. Daí sua urgência
como tema na escola.

Durante esse exercício de apresentação, o professor pode indicar os
nomes dos presidentes do período, mostrar índices econômicos e sociais
representativos, como inflação e mortalidade infantil, e também
apresentar imagens icônicas de diferentes momentos da ditadura. Também é
possível usar a vasta produção musical do período que faz referência
direta ou indireta ao movimento, em especial de compositores e cantores,
como Gilberto Gil, Elis Regina, Chico Buarque. O importante é dar um
panorama que favoreça o mergulho no período: é uma aula para, digamos,
``abrir o túnel do tempo''.

Aqui, além de habilidades de Língua Portuguesa, essa aula certamente colaborará para o desenvolvimento da habilidade identificar, caracterizar e relacionar a presença do autoritarismo na sociedade brasileira.

\BNCC{EM13CHS602}

\Image{Manifestação das Diretas Já em Brasília, diante do Congresso Nacional.; Wikipedia; CC-BY-NC.}{PNLD0053-09}


\paragraph{Tempo estimado:} uma aula de 50 minutos.

\subsection{Leitura: os tempos das narrativas}

Uma característica muito marcante do texto de B. Kucinski é sua clareza
expositiva. Essa clareza é reconfortante. Quem o lê sente-se diante de
um narrador confiável, comprometido com o entendimento do texto e
respeitoso para com o leitor. O tema difícil encontra, o que é muito
convidativo à leitura, uma escrita direta e sincera.

Assim, quando começa a ler seus romances e contos, o leitor rapidamente
perde o medo do livro e do autor. Nesse pacto de leitura que se
estabelece, o leitor deixa de considerar a figura do autor como alguém
``especial'' ou mesmo ``sagrado'', imagem que muitas vezes o próprio
objeto-livro parece impor.

Para motivar os alunos a lerem integralmente o livro, sugerimos que o
professor ou a professora estimulem os estudantes a se prepararem para a
aula, lendo os dois primeiros capítulos. Sabemos, no entanto, que,
especialmente nesse momento inicial, muitos deles não se sentem
confiantes de que poderem fazê-lo ou têm outras atividades que lhes
parecem mais atrativas ou urgentes. Para quebrar esse ``gelo'',
sugerimos que, mesmo que a maioria dos estudantes tenha lido o texto,
que os trechos iniciais dos primeiros capítulos sejam relidos, em voz
alta, em sala de aula.

Não é preciso, embora se possa, caso o docente prefira, ir além da
primeira página de cada um dos dois capítulos. A ideia é capturar a
atenção dos que ainda estão intimidados e conseguir contagiá-los para a
leitura da novela. Isso porque esses dois capítulos apresentam, de modo
bastante explícito, algo estruturante da obra: sua articulação temporal.

As habilidades relacionadas a esta aula são:

\BNCC{EM13LP02}
\BNCC{EM13LP03}

Os trechos devem ser lidos em voz alta pelo docente ou por alguns alunos
e alunas, aguçando a curiosidade sobre Júlia e sobre Durval. Após a
leitura desses trechos, caberia ao docente abrir um debate chamando a
atenção para os seguintes pontos:

\begin{itemize}
\item No capítulo 1, a situação dramática inicial, em que, na divisão da
  herança, Júlia, a protagonista, inicia seu processo de
  autodescobrimento do passado.
  
\item No capítulo 2, intitulado ``Trinta anos antes'', temos a prisão de
  estudantes do ITA, o Instituto Tecnológico da Aeronáutica, onde
  trabalha Durval.
\end{itemize}

Após chamar a atenção dos alunos, propomos que o professor pergunte aos
alunos: ``O que será que une Júlia a Durval?''; ``É possível supor o ano
em que se passam os dois episódios?''; ``Quais os tempos dessa
história?''

Espera-se que os estudantes discutam esses temas de modo aprofundado,
trazendo informações que já possuem sobre a história do país e
experiências pessoais semelhantes, que favoreçam a leitura da obra.

Mais do que respostas certas, a ideia é colocar para os alunos questões
que vão sendo respondidas ao longo do texto, mostrando que a simples
apresentação do problema pelo escritor sugere caminhos de leitura e
perguntas que estimulam a imaginação do leitor.

Essa aula busca, assim, criar expectativas para que os alunos se sintam
encorajados a seguir com a leitura de modo autônomo, num diálogo aberto
com o texto.

\paragraph{Tempo estimado:} uma aula de 50 minutos.

\subsection{Pós-leitura}

Como atividade pós-leitura, queremos que os estudantes ampliem o domínio
sobre as formas de escrita do passado, por meio da produção de um texto
criativo.

As principais habilidades relacionadas a essa atividade, neste momento, são:
\BNCC{EM13LP09}
\BNCC{EM13LP14}

Assim, o docente deve propor às alunas e aos alunos que busquem uma
história familiar ou de conhecidos e que essa história possa ser contada
em dois tempos, de modo semelhante à forma encontrada por B. Kucinski
nos dois capítulos iniciais para introduzir a trama de \emph{Júlia}.

O aluno deve buscar duas personagens e escrever um texto de uma página
sobre cada uma delas. A história dessas duas pessoas deve, de alguma
forma, conter uma relação a ser, eventualmente, revelada. Mas não é
necessário, nestes textos, deixar evidente essa relação.

Também é uma opção do aluno escrever uma história ficcional (total ou
parcialmente) ou um relato não ficcional. O importante é encontrar dois
focos narrativos ligados às duas personagens, que devem estar distantes
no espaço ou no tempo.

Essas personagens podem ser um avô ou avó, ou um tio ou tia, ou mesmo um
vizinho, um amigo de um conhecido.

A sugestão de uma página por personagem é isso mesmo: só uma sugestão,
um limite indicativo. Se o aluno quiser ir além, a professora ou o
professor devem estimulá-los a escrever o quanto acharem necessário.

Numa atividade como essa, é importante dar um retorno aos alunos sobre a
produção. Esse retorno pode ser colocado coletivamente, com o professor
expondo histórias interessantes e pontos fortes da escrita dos alunos
para a classe em geral. Certamente, os estudantes irão se interessar
pelas histórias uns dos outros, e bem como pelas estratégias narrativas
escolhidas pelos colegas.

Caso o professor considere o caso, além da exposição coletiva, poderá
dar retornos individuais, com sugestões que façam os alunos avançarem na
história. O importante aqui é estimular os alunos a escreverem cada vez
mais!

\paragraph{Tempo estimado:} duas aulas de 50 minutos (uma para a escrita -
com conclusão em casa, caso o aluno ou aluna prefira, outra para uma
conversa sobre os resultados).

\section{Sugestões de Atividades II}

\paragraph{Tema}

Uma história como \emph{Júlia} conta com várias estratégias narrativas.
Uma delas é recorrer a diferentes formas textuais. Aqui, a ideia é
mergulhar numa delas, a escrita de cartas.

\BNCC{EM13LP29} 


\paragraph{Objetivos e justificativa}

Esta atividade é direcionada especificamente para as aulas de língua
portuguesa. Queremos, nesse grupo de atividades, aprofundar os
conhecimentos sobre as cartas e correspondências eletrônicas,
estabelecendo laços e diferenças entre esses métodos de comunicação, mas
também apontar como os escritores de ficção se apropriam desse gênero
para conferir verossimilhança à escrita.

Além disso, queremos aproveitar essa atividade para que os estudantes se
mostrem capazes, ao final dela, de se apropriarem de partes do livro e
serem capazes de relatar o lido a um interlocutor, com um objetivo
específico.

Se a escrita responde a uma necessidade de comunicação que supere os
limites da linguagem oral, a troca de mensagem com maior perenidade é
uma das decorrências mais importantes dessa invenção da humanidade.

Seja na forma escrita, enviada pelo correio, ou digital, o e-mail, as
cartas permanecem sendo importantes meios para trocar informações mais
complexas, que os mensageiros instantâneos do celular e mesmo as
conversas telefônicas não dão conta.

A partir da leitura de \emph{Júlia}, em que as cartas da tia Hortência
têm um papel fundamental na compreensão da trama, buscaremos ampliar o
domínio da escrita dessa forma textual, de modo a produzir uma reflexão
sobre o papel dessa escrita no dia a dia e também na literatura.

Esse grupo de atividades foi pensado, essencialmente, para a disciplina
de Língua Portuguesa.


\subsection{Pré-leitura: Da carta à mensagem eletrônica}

Durante a vida escolar, a escrita de cartas é um tema que atravessa
diversos momentos. Por quê, então, retomá-la neste momento?

A primeira questão é dominar ainda mais a sua escrita. A retomada do
tema permitirá ao estudante se aprofundar nas mudanças que os diferentes
meios e ocasiões permitem. O segundo, e bastante interessante, é notar a
presença dessa forma na produção ficcional.

Assim, como atividade de pré-leitura, sugerimos que às professoras e
professores que apresentem, num sobrevoo, os diferentes tipos de cartas:
a carta de amor (como nas cartas de Abelardo e Heloísa, a
carta-documento, com a de Pero Vaz de Caminha; cartas comerciais etc.).
O importante é mostrar que as cartas se constituíram, ao longo da
história, numa forma fundamental da vida cotidiana. Também é
interessante o professor falar sobre os romances epistolares, ou seja,
obras ficcionais escritas como se fossem trocas de cartas entre as
personagens. O docente pode, por exemplo, citar o caso do livro \emph{Os
sofrimentos do jovem Werther}, de Goethe, imaginado como uma coleção de
cartas de um jovem que narra sua paixão por Carlota a um amigo e que é
considerado um dos grandes marcos da literatura ocidental e do
romantismo alemão.

Ainda nessa exposição, vale adentrar no gênero específico de carta usado
por B. Kucinski no romance, a carta familiar. Por muito tempo, as cartas
de família serviram para trocar informações importantes sobre a vida de
parentes que viviam em diferentes cidades. Até o surgimento do telefone
e, depois, da internet.

Para provocar o envolvimento da turma, sugerimos que o docente se dirija
à turma, ao final de sua exposição, com uma pergunta: ``O que muda na
correspondência quando passamos a usar o e-mail?''

Ao final dos debates, o professor deve lembrar à classe que, para a aula
seguinte, é preciso pedir à turma que leia o capítulo 7 do livro, ``As
cartas da tia Hortência''.

\paragraph{Tempo estimado:} uma aula de 50 minutos.

\subsection{Leitura: Capítulo 7 -- <<As cartas da Tia Hortência>>}

Para essa atividade, sugerimos uma leitura coletiva das cartas que
integram o capítulo 7 do livro \emph{Júlia - nos campos conflagrados do
Senhor}. Novamente, mesmo que os estudantes tenham lido o capítulo, a
leitura coletiva deve funcionar também como um novo estímulo para que as
alunas e alunos se engajem na leitura completa do livro. Além disso, são
uma forma de mergulhar no objeto deste grupo de atividades, a carta como
gênero textual.

As cartas do capítulo 7 são parte dos documentos que Júlia encontrou, no
capítulo 5, e que sugerem uma história sobre o país e sobre ela mesmo
que desconhecia. Esses documentos vão mudar o modo como Júlia encara o
passado. Lidas por ela, adquirem novo sentido. Essas são questões
importantes a se debater com os estudantes.

Após a leitura em voz alta do capítulo, o que pode ser feito pelo
professor ou por diferentes estudantes (sugerimos que um estudante leia
uma carta e outro o trecho em que Júlia avalia o sentido do documento,
envolvendo assim pelo menos oito alunas e alunas diferentes), o
professor pode fazer as seguintes perguntas à classe:

\begin{enumerate}

\item Quando e de onde são enviadas as cartas?

\item A quem são dirigidas as cartas, originalmente?

\item O que muda quando não é o destinatário imaginado pelo autor da carta
  quem a lê?

\end{enumerate}

\paragraph{Tempo estimado:} de uma a duas aulas de 50 minutos.

\subsection{Pós-leitura: Escrevendo uma carta sobre Júlia}

Aqui, vamos fazer um exercício de imaginação. Partiremos da ideia de que
a aluna ou o aluno é uma amiga ou amigo de Júlia, com quem não tem
contato há muitos anos, mas que conhece sua história.

O ano é de 2018 e uma jornalista acaba de descobrir e publicar uma
reportagem que mostra ter havido, no Brasil, uma série de sequestros e
adoções ilegais de crianças filhas de militantes políticos e camponeses
de regiões de guerrilha durante o regime militar.

Diante dessa situação-problema, vamos sugerir às alunas e alunos que se
coloquem na posição de alguém que decide contar a essa jornalista o caso
de Júlia, por meio de uma carta, a ser enviada para a redação do jornal.
Essa carta deve ser escrita à mão e resumir as principais informações do
livro e também detalhar as informações pessoais disponíveis para que a
jornalista busque documentos que comprovem a veracidade do caso e possa,
com as informações possíveis, procurar Júlia nas redes sociais e na
internet.

Para a tarefa, o estudante terá, nessa carta, de apresentar o livro primeiro como se seu relato fosse verídico, ou seja, como se fosse uma obra de não ficção, desenvolvendo a seguinte habilidade:

\BNCC{EM13LP29}

Após esse exercício, abrir a possibilidade para a produção de uma resenha literária. Nesse caso, uma outra habilidade está em pauta:
\BNCC{EM13LP44} 


Vamos lá?

\paragraph{Tempo estimado:} duas aulas de 50 minutos.


% \textbf{EM13LP01, EM13LP02, EM13LP04, EM13LP05, EM13LP06, 

\BNCC{EM13LP07}
\BNCC{EM13LP09}
\BNCC{EM13LP10}
% \BNCC{EM13LP13}
% \BNCC{EM13LP16}
% \BNCC{EM13LP21}
% \BNCC{EM13LP25} %, EM13LP27, EM13LP28, EM13LP29, EM13LP31, EM13LP33, EM13LP38, EM13LP45, EM13LP48}

\section{Aprofundamento}

\textbf{Por que é importante lembrar a ditadura?}

A leitura de \emph{Júlia - Nos campos conflagrados do Senhor}, de B.
Kucinski, nos coloca diante de questões muito diferentes. Porque, como
escreveu o resenhista Camilo Vannuchi, ``\emph{Júlia} é um livro sobre
sequestro de crianças na ditadura. É também um livro sobre a indignação
de um homem comum, um professor universitário, diante da violência e o
arbítrio''.

Ao trazer a questão da violência de Estado, ou seja, do arbítrio, para o
centro da narrativa, Kucinski participa de uma busca, por parte de
muitos escritores de ficção brasileiros contemporâneos, de jogar luz
sobre um período sombrio e proibido. A escolha pela via ficcional, neste
caso, não apenas de ``contar histórias'', embora jamais deixe de ter
essa intenção, mas também de recuperar justamente aquilo que foi
interditado: a reparação aos perseguidos políticos brasileiros.

Nas décadas de 1960 e 1970, e avançando em muitos lugares, não foram
poucos regimes ditatoriais em todo o planeta. Apenas na América do Sul,
o período é marcado por ditaduras no Brasil (1964-1985), Argentina
(1976-1983), Uruguai (1973-1984), Bolívia (1971-1985), Chile
(1973-1990), Paraguai (1954-1989) e Peru (1968-1980). Em muitos
momentos, diversos desses países colaboraram entre si na perseguição a
opositores, na chamada Operação Condor.

Entre os países da região, muitos buscaram instaurar, em alguma medida,
garantir o que chamamos de direito à memória e à verdade. Contabilizando
milhares de opositores mortos e desaparecidos, além de outras formas de
repressão e violência que incluem desde a perda de empregos (como ocorre
com Durval em \emph{Júlia}) ao banimento do país e a perda da cidadania
brasileira, esses regimes deixaram consequências que vão muito além do
período em que ocorreram. Mas nenhum foi tão reticente em fazer uma
autocrítica e dar voz às pessoas que sofreram nas mãos do regime quanto
o Brasil.

Sobre essa questão, o filósofo Edson Teles, professor da Universidade
Federal de São Paulo, afirma que ``uma ditadura não se mede pelo número
de mortos e desaparecidos que produziu, mas pelo impacto que gerou no
país, o que se percebe pela herança autoritária vivida em democracia''.

Desse modo, Teles se coloca entre os que apontam a ditadura brasileira
como uma das mais violentas do período, embora, em comparação com
outras, tenha produzido um número oficial menor de mortos e
desaparecidos políticos. Isso porque ela deixou uma série de ``restos'':
``Poderíamos dizer que a maior delas encontra-se na imposição de uma
cultura de impunidade, que privilegia a violência e os que detêm o poder
político em detrimento da ideia de uma cidadania plena'', afirmou ele em
entrevista à revista \emph{IHU Online}, ligada à Unisinos, universidade
católica situada no Vale dos Sinos, no Rio Grande do Sul. Teles é um
pensador do período, mas também uma vítima: tinha apenas quatro anos de
idade quando foi levado a uma sessão de tortura de sua mãe, Maria Amélia
de Almeida Teles, ou seja, sofreu, ainda na primeira infância, uma forma
de tortura psicológica, e foi usado como forma de pressionar sua mãe.
Sua história está contada em diversos depoimentos, seus e de seus
familiares.

\SideImage{Manifestantes simulam o método de tortura conhecido como pau de arara em Brasília; Wikipedia; CC-BY-NC .}{PNLD0053-06}


Episódios assim provocaram marcas profundas na sociedade brasileira, mas
o Estado brasileiro abriu poucos espaços para restaurar tanto os
indivíduos quanto a sociedade desses crimes. Há uma espécie de
interdição em relação à ditadura, como se esse fosse um assunto superado
e que tudo deve ser esquecido.

Esse caminho não foi o seguido por outros países. A África do Sul, do
apartheid, por exemplo, onde o regime de segregação racial perdurou por
décadas, limitando e perseguindo as pessoas e a cultura dos povos
negros, decidiu encarar, logo após o fim do regime, em 1994, com mais
coragem. Como explica Edson Teles na mesma entrevista, na África do Sul
foi feita ``a opção pela narrativa e publicidade dos crimes do
Apartheid. O Brasil escolheu o silêncio. A anistia sul-africana foi
individual, caso a caso, crime a crime, e só foi concedida depois da
confissão pública do ato criminoso e do esclarecimento do que foi feito
com o corpo das vítimas. No Brasil, como vocês sabem, a anistia foi
genérica e, simbolicamente, acabou por tornar inimputáveis os autores de
crimes bárbaros praticados enquanto eram funcionários do Estado''.

Com evidentes diferenças, também foi o caso da Argentina, onde
diferentes processos levaram lideranças do governo da ditadura militar a
julgamento, incluindo três ex-militares presidentes da República. Também
se procurou ouvir e dar voz aos perseguidos, com a criação de espaços de
memória, tão raros no Brasil.

\textbf{O papel da ficção}

O próprio B. Kucinski, num encontro com críticos de sua obra, avaliou
que o fato de a sociedade brasileira não ter olhado para o passado
colocou um problema para a identidade nacional. Fazendo uma analogia com
a Alemanha, o escritor, que é descendente de judeus que deixaram a
Europa antes da Segunda Guerra Mundial, Kucinski afirmou que tentar
entender como o país chegou ao Holocausto passou a ser um problema da
identidade alemã. Segundo ele, demorou cerca de duas décadas, mas essa
mudança ocorreu e hoje faz parte das políticas públicas estudar o
nazismo e entender como o país produziu tamanho grau de violência. ``O
nazismo se tornou um problema identitário para o alemão'', explica ele.

De certo modo, por meio de sua ficção, B. Kucinski busca colocar o
ocorrido na ditadura militar e sua consequência, com a disseminação da
violência por toda a sociedade, que hoje apresenta altíssimos níveis de
violência policial e de assassinatos de jovens, mesmo fora de um período
claramente ditatorial.

Eurídice Figueiredo, autora do livro \emph{A literatura como arquivo da
ditadura brasileira}, ao tratar de \emph{K. Relato de uma busca},
primeiro romance de B. Kucinski, afirma que ele é um autor de estilo
``enxuto e fragmentário'', que ``atinge a emoção do leitor sem apelar
para o melodramático, pelo contrário, ele usa da ironia e do
despojamento da linguagem para criar o ambiente absurdo, claustrofóbico
e apavorante em que se viu o pai diante do sumiço da filha''. A mesma
estudiosa, em outro momento de seu livro, explica o papel da ficção na
compreensão da ditadura brasileira: ``Só numa dimensão ficcional é
possível entrever nas dobras da história os interditos''.

Tomemos essas duas sugestões de Figueiredo para refletir sobre a força
narrativa de \emph{Júlia}. A opção da novela, por contar a história da
protagonista em dois tempos, e tornar isso explícito inclusive com o uso
de tipos tipográficos diferentes, sugere uma tentativa explícita de B.
Kucinski de colocar o leitor em contato com dois tempos diferentes.

Primeiro, temos, de trás para frente, o tempo da leitura, os anos 2020
(quando o livro foi originalmente publicado). Depois temos os anos 1990,
quando Júlia entra em contato com o seu passado; por fim, temos os anos
1960, quando há o golpe militar que desorganiza a vida social
brasileira.

Pensando retrospectivamente, é como se B. Kucinski apontasse para o
momento em que teríamos tido a grande oportunidade de buscar esse
passado, ou seja, 30 anos após o golpe de 1964, quando já havia a nova
Constituição e o país havia superado os primeiros anos do regime sob
essa nova lei, marcados pela crise do período da presidência de Fernando
Collor de Mello e pela estabilização pós-impeachment dos anos de Itamar
Franco, que o sucedeu.

Mas mais de outras duas décadas se passaram sem que o país tenha olhado
para o passado e buscado essa restauração. Assim, \emph{Júlia} acaba se
tornando uma espécie de proposta de restauração dessa memória, uma nova
tentativa de retomar as oportunidades perdidas. Ainda segundo
Figueiredo, a literatura sobre a ditadura se coloca como ``elemento
ativo na transmissão da memória para que não se apague aquilo que afetou
a vida das pessoas''.

A leitura de \emph{Júlia} e de toda a ficção sobre o regime militar
produzida nos últimos anos mostra que os muitos escritores tomaram para
si uma responsabilidade que, em tese, era a do Estado: manter viva essa
história, para que o passado ditatorial se torne um problema que todo
brasileiro deve percorrer e elaborar, de modo que sua violência seja de
fato superada, e não varrida para debaixo do tapete da história e
reproduzida no cotidiano.

A história do regime militar é hoje uma tópica, ou seja, um assunto
recorrente em nossa literatura, e B. Kucinski um de seus autores mais
emblemáticos. A produção literária e a crítica sobre esse período são
crescentes, e sua recorrência mostra que a ficção tem também esse papel:
o de não deixar apagar a realidade.


\subsection{Atividades para o aprofundamento da pesquisa}

\subsubsection{Geografia, História e Literatura}

\paragraph{Tema} O espaço e o tempo na literatura. Onde se passam os
acontecimentos de uma ficção? Como identificar e compreender o tempo dos
fatos de uma novela como \emph{Júlia}?

\BNCC{EM13LP33} % 
\BNCC{EM13LP16} % 
\BNCC{EM13LP10} % (EM13LP27) (EM13LP38)}

\paragraph{Objetivos e justificativa}

Esta atividade é direcionada especificamente para as aulas de geografia
e história, ainda que possa contar com a colaboração dos docentes de
língua portuguesa. Queremos, nesse grupo de atividades, aprofundar os
conhecimentos sobre as cartas e correspondências eletrônicas,
estabelecendo laços e diferenças entre esses métodos de comunicação, mas
também apontar como os escritores de ficção se apropriam desse gênero
para conferir verossimilhança à escrita. A ideia é fortalecer a habilidade indicada abaixo:

\BNCC{EM13LP33}

Além disso, queremos usar essa atividade para os estudantes se engajarem
numa leitura atenta de \emph{Júlia} e se mostrem capazes, ao final dela,
de se apropriarem da história do livro e serem capazes de relatar o lido
a um interlocutor, com um objetivo específico.

\paragraph{Metodologia}

Acreditamos que a leitura de \emph{Júlia} com os estudantes pode ser
também bastante interessante para o aprendizado não só de língua
portuguesa, mas também de geografia e história. Esse conjunto de
atividades procura, de modo transversal, incorporar conhecimento e
estratégias de aprendizagem dessas áreas, tornando a leitura da novela
mais complexa e completa.

Sabemos que a personagem \emph{Júlia} busca recuperar o passado e,
assim, a professora ou o professor precisa, sempre, contextualizar a
trama da obra.

O capítulo dois do livro menciona a prisão de estudantes do ITA, o
Instituto Tecnológico da Aeronáutica, no navio Raul Soares, em Santos,
cidade litorânea do Estado de São Paulo, onde fica o maior porto
marítimo brasileiro. Mas a trama não se desenvolve só ali: na realidade,
ela percorre inúmeras cidades do Vale do Paraíba, região industrial
situada entre São Paulo e Rio de Janeiro, e também a própria capital
paulista.
\Image{Navio Raul Soares, ca. 1926; Brasil de fato; CC-BY-NC.}{PNLD0053-05}


Uma curiosidade sobre os bastidores dessa publicação: um dos títulos
pensados por B. Kucinski para o livro seria \emph{Júlia -- Brincando nas
planícies conflagradas do Senhor}, justamente pelo fato de o livro se
passar sobretudo numa região com essas características. Por fim, acabou
optando por ``campos''. A possibilidade da utilização desse título, no
entanto, demonstra a importância do espaço na construção da narrativa.

Essas atividades procuram situar em detalhes a novela nos tempos e nos
espaços identificados pelo narrador.

\paragraph{Pré-leitura: verbete, uma forma de escrita}

Para começar essa série de atividades, o professor deve apresentar aos
estudantes um resumo do que o volume 1 do relatório da Comissão Nacional
da Verdade, disponível na internet, diz sobre o navio-prisão Raul
Soares. Sugerimos uma atenção especial para o verbete ``Navio Raul
Soares - São Paulo''. Eventualmente, o professor pode buscar informações
adicionais abundantes na internet sobre esse episódio, em reportagens em
texto e vídeo.

Nessa pré-leitura, o objetivo é mostrar a relação episódica do fato com
a obra e a apresentar, também, uma forma de escrita muito especial: o
verbete.

Um verbete é um apontamento bastante objetivo e, ao mesmo tempo, breve
sobre algo - que pode ser uma palavra, no caso de um dicionário comum,
um fato histórico ou qualquer outra coisa que permita tal abordagem.

O verbete é a forma como se organizam dicionários, enciclopédias e
muitos manuais. Essa forma é bastante útil para a organização e
sistematização do conhecimento, que os alunos de antigamente consultavam
no papel e que hoje, mais comumente, usam a internet.

Os professores podem, havendo tempo, discutir as formas de construção de
um verbete - seja a busca por escritores especializados, no caso dos
dicionários e manuais e, antigamente, das enciclopédias, seja a
construção colaborativa da enciclopédia virtual Wikipedia, provavelmente
a mais conhecida das alunas e alunos.

O docente pode ainda discutir os riscos da construção coletiva, a
possibilidade de manipulação e as formas de controle que a Wikipedia
(https://pt.wikipedia.org) usa para evitar esse problema. Uma
possibilidade é apresentar, justamente, uma leitura crítica do verbete
``Raul Soares (navio)'' da Wikipedia e compará-lo com o verbete da
Comissão Nacional da Verdade.

Outro caminho interessante é mostrar o exemplo da Desciclopédia
(https://desciclopedia.org), uma espécie de Wikipedia humorística, em
que a forma é emulada, mas sem o compromisso com a veracidade das
informações. Ou seja, mostrar como a forma verbete se presta não só para
``coisa séria'', mas também para fazer piada.

\paragraph{Tempo estimado:} uma aula de 50 minutos.

\paragraph{Leitura: uma lista de lugares}

Como atividade de leitura, o professor deve pedir aos estudantes que
identifiquem as cidades e espaços mencionados ao longo do romance. Por
exemplo, o ITA, em São José dos Campos, o Porto de Santos, o apartamento
de Júlia, a Faculdade de Medicina da Santa Casa\ldots{} Há muitos
outros\ldots{} A ideia é que, durante a leitura, os estudantes componham
uma lista, que será utilizada na atividade de pós-leitura.

A realização desse trabalho pode ser feita em grupo. Durante o semestre,
o professor responsável por sua organização, que pode ser o de língua
portuguesa, o de história e o de geografia, deve orientar os alunos a,
durante a leitura ir organizando essa lista. Esse trabalho pode ser
feito em grupo, mas também individualmente, e deve ser compartilhado
pela classe.

\paragraph{Tempo estimado:} duas aulas de 50 minutos.

\paragraph{Pós-leitura: E se \emph{Júlia} virasse um mapa?}

Com o apoio dos professores de geografia e história, ou mesmo com a
condução de um deles, a leitura de \emph{Júlia} pode se transformar num
mapa colaborativo, a ser disponibilizado na Internet, inclusive para
outros leitores do livro!

A ideia é que essa seja uma tarefa semestral de geografia ou história.

Sabemos que, na rede mundial de computadores, é possível valer-se de
ferramentas colaborativas como o Google Maps para a construção de mapas
colaborativos. Dessa forma, pretendemos também estimular o uso de ferramentas e a produção colaborativa em ambientes digitais.
\BNCC{EM13LGG703}

Recomendamos a formação de grupos de até cinco alunos, de modo que cada
aluno seja responsável por um ou no máximo dois verbetes, que devem ter
de cinco a dez linhas. Os alunos podem e devem se ajudar, mas é
importante que todos se engajem nesse processo, pois a produção de verbetes é uma das habilidades da BNCC que a atividade deve estimular.

\BNCC{EM13LP33}

Os professores de geografia e história podem usar o mapa colaborativo
para retomar temas como o desenvolvimento e a industrialização do Vale
do Paraíba, ou características físicas da região. São exemplos. O que
queremos, nessa atividade conjunta, é justamente aproveitar o potencial
transdisciplinar na obra e a possibilidade de a sua leitura gerar uma
nova produção textual.

Ao final dos trabalhos, o mapa colaborativo pode ser disponibilizado
para a comunidade escolar ou mesmo tornar-se totalmente público, a
depender do resultado alcançado.

\paragraph{Tempo estimado:} quatro aulas de 50 minutos, no total, que podem
ser diluídos em partes de aulas das diferentes disciplinas.

\subsubsection{A escrita não ficcional sobre o passado}


\paragraph{Tema}

O objetivo desse conjunto de atividades é permitir aos alunos um
exercício de escrita estritamente não ficcional. Do ponto de vista da
escrita, queremos desenvolver técnicas de redação comprometidas não com
a verossimilhança, mas com a fidelidade ao relato. Como recontar um
passado traumático?

\paragraph{Objetivos e justificativa}

A releitura do passado exige diferentes estratégias discursivas. A busca
pela via ficcional, encontrada por B. Kucinski em \emph{Júlia}, é uma,
mas o relato não ficcional, comprometido com a memória afetiva e factual
dos envolvidos, é outra.

Essa atividade pode ser desenvolvida tanto por professores de língua
portuguesa quanto por professores de histórias ou, ainda, em conjunto
pelos responsáveis das duas disciplinas.

O fundamental aqui é apresentar as dificuldades de retratar o período e
sugerir uma atividade que busque retomar o período de modo respeitoso,
especialmente para aqueles que tanto sofreram em decorrência dele.

\paragraph{Pré-leitura: O fim do regime militar}

Cabe ao docente responsável pela atividade apresentar aos alunos os
últimos anos do regime militar. Assim, sugerimos dois marcos temporais:
a aprovação da lei da anistia, em 1978, e o fim da Assembleia Nacional
Constituinte, em 1988.

Esse é um período de intensa mobilização social e política brasileira.
Com a aprovação da Anistia, muitos exilados políticos puderam retornar
ao Brasil, onde encontraram um movimento sindical em plena atividade e
uma reorganização política em andamento.

Pressionada, a ditadura permite a criação de novos partidos, acabando
com o sistema de apenas dois partidos, um que sustentava o governo,
outro que fazia oposição. Nesse clima de fermentação política acontecem
as primeiras eleições diretas para governadores (1982), a campanha por
eleições diretas para presidente (as Diretas-Já!, em 1984), que acabou
não ocorrendo, a eleição de uma chapa de civis para a presidência da
República em 1985, com Tancredo Neves e José Sarney, ambos do PMDB, como
presidente e vice, a instauração da Assembléia Nacional Constituinte
após as eleições de 1986 e a redação da Carta Constitucional em 1988.

\Image{Ulisses Guimarães; Wikipedia; CC-BY-NC.}{PNLD0053-08}


Apesar dessa vida ativa, a Anistia aos militares que participaram de
crimes como tortura, sequestro e assassinato foi mantida pela
Constituinte, como que colocando uma barreira ao debate de uma releitura
crítica do passado, ao contrário do que estava ocorrendo, por exemplo,
na Argentina, com já se começava a discutir a violência de Estado contra
os seus próprios cidadãos.



Do ponto de vista econômico e social, o país estava numa enorme crise: a
inflação anual passava dos 200\% ao fim da ditadura e o desemprego era
enorme. Não havia um sistema de saúde público universal, como temos hoje
o SUS, e a maioria das crianças deixava a escola muito antes de chegar
ao ensino médio.

A retomada desse cenário deve ser entendida como a abertura de um
caminho para a compreensão desse passado.

\paragraph{Tempo estimado:} uma aula de 50 minutos.

\paragraph{Leitura:}

O professor aqui deve pedir uma leitura especial do último capítulo do
livro \emph{Júlia - nos campos conflagrados do Senhor}. Essa leitura
pode ser feita fora da sala de aula e não é necessária a releitura
coletiva nesse caso.
\SideImage{Monumento Tortura Nunca Mais, no Recife; Wikipedia; CC-BY-NC.}{PNLD0053-07}

Duas habilidades serão desenvolvidas nessa atividade:
\BNCC{EM13LP29} % 
\BNCC{EM13LP31} % (EM13LP38) (EM13LP48)

Entendemos que, a essa altura, as alunos e os alunos já leram toda a
novela e que é hora de discutir como o livro acaba.

Neste epílogo, Júlia procura a Secretaria Nacional de Direitos Humanos
para relatar sua história e inscrevê-la, portanto, na história oficial
do país. Depois, ela conhece mais alguns detalhes, que finalmente
esclarecem o que ainda faltava. Há, portanto, uma retomada de toda a
trajetória do livro e, finalmente, a compreensão da tragédia que
percorre a sua história familiar.

O fim do livro apresenta Júlia finalmente entendendo o seu passado e se
reencontrando com ele. Isso não o muda, mas a faz ter uma visão mais
completa e madura dos acontecimentos.

Diante dessa situação, sugerimos uma aula debate de finalização da
leitura, com uma pergunta orientadora: ``O que Júlia aprendeu durante
essa pesquisa?''

O debate deve ser franco e permitir também que os alunos expressem os
sentimentos e dificuldades que tiveram durante a leitura.

\paragraph{Tempo estimado:} uma aula de 50 minutos.

\paragraph{Pós-leitura:}

Como contar a história de um alguém que viveu o regime militar e cuja
história o Estado brasileiro não reconheceu por décadas? Como pôr no papel um relato oral sobre essa vida?

Nessa última atividade sugerida, propomos que as alunas e alunos visitem
a página de vídeos do Memorial da Resistência de São Paulo, em
\href{http://memorialdaresistenciasp.org.br/acervo/}{Memorial da Resistenciasp}, um dos raros espaços de
Memória da ditadura militar mantidos pelo Estado no Brasil.



Nessa página, há uma sessão dedicada à ``Coleta Regular de
Testemunhos'', em vídeos. É com esses vídeos que vamos desenvolver a
última atividade.

A professora ou o professor responsável pela atividade deve escolher até
três dos vídeos disponíveis e exibi-los em sala de aula. São vídeos
curtos, de poucos minutos. A escolha pode ser feita a partir do perfil
da turma. A ideia é apresentar o relato como uma forma de
reconstruir o passado.

Isso, sempre considerando o desenvolvimento das sequintes habilidades:

\BNCC{EM13LP33}
\BNCC{EM13LP37}

Após a exibição do vídeo, as alunas e alunos devem ser chamados a
escolher, individualmente, um depoimento como um trabalho a ser entregue
posteriormente.

A partir do relato e de uma pesquisa adicional, caberá ao aluno escrever
um texto de uma página, reunindo informações relevantes que coletou e
apresentando essa personagem para um leitor.

Ao professor, cabe dar orientações sobre como escrever um relato
evitando a ficcionalização: ou seja, referenciando as fontes, indicando
o que se sabe e o que não se sabe, identificando o que é fala da própria
personagem e o que encontrou em outros lugares etc.

Esse trabalho precisa, necessariamente, contar com um retorno, de
preferência individualizado, detalhando os sucessos e os problemas da
escrita não ficcional, em seu compromisso factual.

\paragraph{Tempo estimado:} duas aulas de 50 minutos (uma para a
apresentação do trabalho, outra para o retorno da escrita)


\section{Sugestões de referências complementares}

\begin{itemize}

\item \textsc{figueiredo}, Eurídice. \emph{A literatura como arquivo da ditadura
brasileira}. Rio de Janeiro: 7Letras, 2017.\\
Neste livro, Figueiredo organiza a produção literária sobre a ditadura
militar, desde os livros produzidos anos 1970, ainda sob o regime, até a
dos anos 2000, marcada pela influência das Comissão de Anistia e da
Comissão Nacional da Verdade. Figueiredo contrapõe a intensa preocupação
dos escritores do período à lentidão do Estado brasileiro em reconhecer
seus crimes contra os cidadãos brasileiros.

\item \textsc{MEMORIAL DA RESISTÊNCIA}. ``Acervo digital do Memorial da Resistência''. Nesta página online, estão reunidos diversos depoimentos em vídeo e farto material fotográfico organizado pelo museu mantido pelo Estado de São Paulo. Endereço online: \href{http://http://memorialdaresistenciasp.org.br/acervo/}{{memorialdaresistenciasp.org.br/acervo/}}.
Acesso em 2 mar 2021.

\item \textsc{junges}, Márcia. ``A apuração da verdade: grande medo das instituições
militares''. Revista do IHU Online, número 358, p. 16-18. São Leopoldo:
18 abr. 2011. Disponível em
\href{http://www.ihuonline.unisinos.br/media/pdf/IHUOnlineEdicao358.pdf}{{ihuonline.unisinos.br}}.
Acesso em 2 mar 2021.

Nesta entrevista a Márcia Junges, o filósofo Edson Teles discute a
questão da violência da ditadura brasileira e algumas de suas
consequências. Ele também explica porque África do Sul, Argentina e
Chile lidaram melhor com o passado ditatorial que o Brasil.

\item \textsc{vanucchi}, Camilo. ``Com 'Júlia', Kucinski se consagra como genial
romancista da ditadura''. UOL, 03/07/2020. Disponível em
\href{https://noticias.uol.com.br/colunas/camilo-vannuchi/2020/07/03/com-julia-kucinski-se-consagra-como-genial-romancista-da-ditadura.htm}{{uol.com.br}}.
Acesso em 2 mar. 2021.

Nessa resenha de \emph{Júlia}, o jornalista e pesquisador Camilo
Vanucchi apresenta um pouco também do Bernardo Kucinski professor da
Escola de Comunicações e Artes. Ele lembra que, apesar de reconhecido
como jornalista, Kucinski não era reconhecido especialmente pela
qualidade de texto. O jornalista, entre outros aspectos, chama a atenção
para o fato de que o jornalista cujos textos davam ``algum trabalho aos
revisores'' seja hoje um ficcionista consagrado.
\end{itemize}

\section{Bibliografia comentada}

\begin{itemize}

\item \textsc{gervitz}, Roberto. \emph{Feliz Ano Velho.} Filme baseado no relato de Marcelo Rubens Paiva, narra a trajetória de um filho de um importante deputado federal morto pela Ditadura Militar. O garoto sofre um acidente, ao pular de uma cachoeira, e o período do fim do regime que perseguiu e assassinou seu pai se mistura com um drama pessoal. Brasil, 1987. O filme foi lançado em DVD. 

\item \textsc{olivieri-godet}, Rita e \textsc{garcia}, Mireille. \emph{Revista brasileira de
estudos de literatura contemporânea,} nº 60. Brasília: UnB, maio/ago.
2020. Disponível em
\href{periodicos.unb.br/index.php/estudos/issue/view/1995}{periodicos.unb.br}.
Acesso em 17 fev./2021.

Esse número da \emph{Revista brasileira de estudos de literatura
contemporânea} é dedicado inteiramente à produção literária que tem como
tema a ditadura militar brasileira. Nesse sentido, trata-se de uma
excelente leitura para quem for trabalhar com \emph{Júlia, brincando nos
campos conflagrados do Senhor}, pois permite à professora e ao professor
ter acesso a diferentes abordagens sobre esse verdadeiro gênero, hoje,
na produção literária nacional. Dois primeiros artigos e uma resenha são
dedicados a livros de B. Kucinski.

\item \textsc{reina}, Eduardo. \emph{cativeiro sem fim} - as histórias dos bebês,
crianças e adolescentes sequestrados pela ditadura. São Paulo: Alameda,
2019.

Este livro procurou, a partir de diversas pistas deixados por outros
relatos menos atentos e uma longa reportagem, recuperar a trajetória de
crianças tiradas de seus pais e adotadas ilegalmente por militares ou
apoiadores do regime durante a ditadura.

\item \textsc{brasil}. Relatórios da Comissão Nacional da Verdade. Brasil: Comissão
Nacional da Verdade, 2014. Disponível em
\href{http://cnv.memoriasreveladas.gov.br/}{{cnv.memoriasreveladas.gov.br}}.
Acesso em 17 fev. 2021.

Os relatórios da CNV são o mais completo trabalho realizado pelo Estado
brasileiro no sentido de recuperar e reparar o passado ditatorial.

\item \textsc{comissão da verdade do estado de São Paulo Rubens Paiva.} \emph{Infância
roubada}, crianças atingidas pela ditadura militar no Brasil. São Paulo:
Assembleia Legislativa, Comissão da Verdade do Estado de São Paulo,
2014. Disponível em
\href{https://www.al.sp.gov.br/repositorio/bibliotecaDigital/20800_arquivo.pdf}{{al.sp.gov.br}}.
Acesso em 17 fev. 2021.

Produzido por uma comissão designada pela Assembleia Legislativa de São
Paulo, \emph{Infância roubada} recuperou a memória de diversas crianças
que foram atingidas pela ditadura militar. Há aqueles cujos pais seguem
desaparecidos, crianças que foram criadas sem saberem de seu passado e
até mesmo crianças, hoje adultos, que foram mantidos sob custódia ilegal
de agentes da repressão enquanto os pais e mães eram torturados. Esse
livro pode ser usado como material de apoio para os professores.

\end{itemize}
\end{document}


