\documentclass[12pt]{extarticle}
\usepackage{manualdoprofessor}
\usepackage{fichatecnica}
\usepackage{lipsum,media9,graficos}
\usepackage[justification=raggedright]{caption}
\usepackage{bncc}
\usepackage[nexus]{../edlab}



\begin{document}


\newcommand{\AutorLivro}{Orides Fontela}
\newcommand{\TituloLivro}{Poemas escolhidos}
\newcommand{\Tema}{Ficção, mistério e fantasia}
\newcommand{\Genero}{Poema}
\newcommand{\imagemCapa}{./images/PNLD0030-01.png}
\newcommand{\issnppub}{---}
\newcommand{\issnepub}{---}
% \newcommand{\fichacatalografica}{PNLD0030-00.png}
\newcommand{\colaborador}{\textbf{Rodrigo Ribeiro Neves} é uma pessoa incrível e vai fazer um bom serviço.}


\title{\TituloLivro}
\author{\AutorLivro}
\def\authornotes{\colaborador}

\date{}
\maketitle

\begin{abstract}
Este manual tem o objetivo de auxiliá-lo no desenvolvimento de práticas
pedagógicas que estabeleçam o diálogo entre a obra literária aqui
apresentada e os estudantes, de modo a ampliar não apenas a leitura do
texto em si, mas também a sua relação com o mundo.

O livro \emph{Poemas escolhidos} consiste em uma reunião da poesia de
Orides Fontela extraída de \emph{Transposição} (1969), \emph{Helianto}
(1973), \emph{Alba} (1983), \emph{Rosácea} (1986), \emph{Teia} (1996) e
alguns poemas inéditos publicados em sua \emph{Poesia completa} (2015).

Poucas escritoras conseguiram encontrar uma voz tão própria, na
contramão das tendências de seu tempo, como Orides Fontela. Por isso, é
difícil definir a sua poesia em uma palavra ou expressão, mas talvez
possamos arriscar uma definição usando o título de seu último livro:
teia. A poesia de Orides é como uma teia, cada fio e em cada ponto em
que eles se encontram abrem uma infinidade de outros caminhos, onde
seguimos orientados por sua forma bem desenha e construída. E na sua
aparente fragilidade e leveza, nos deparamos com uma força que agarra
tudo aquilo que tenta passar por ela e se ilumina com os primeiros raios
de sol. Ela aprofunda algumas das principais conquistas do modernismo,
mas também dialoga com seu tempo, marcado pelo fim das vanguardas e por
novos olhares no passado.

Orides Fontela é uma das poetas mais importantes da literatura
brasileira da segunda metade do século XX. A modernidade de seu processo
criativo, as reflexões filosóficas, a vitalidade e concisão da forma
poética e a renovação das lições do passado a colocam como uma artista
ainda bastante atual. Portanto, esta coletânea de poemas vem contribuir
para que a escritora ocupe o lugar que merece na história da nossa
literatura e da cultura brasileira, trazendo para a escola a reflexão e
discussão de aspectos da linguagem e sua relação com o mundo que
certamente irão contribuir na formação dos estudantes como leitores
críticos e sensíveis da realidade.


\Image{A autora, Orides Fontela. (Arquivo da autora.)}{PNLD0030-03.png}


Para isso, apresentamos aqui propostas de atividades, aprofundamento,
referências complementares e uma bibliografia comentada, a fim de que o
material possa ser útil nas suas aulas para estimular os estudantes a
desbravar um universo de possibilidades através de uma das escritoras
mais importantes da nossa literatura. Além disso, é ótimo trabalhar com
contos, novelas e poesia em sala de aula, pois são gêneros literários
bastante fecundos, que possibilitam a dinamização das atividades e a
exploração de uma variedade maior de temas para discussão com os
estudantes.

Aproveite bastante este material. Ele foi feito com muita dedicação e
carinho para você! Boa aula!
\end{abstract}

\tableofcontents


\section{Atividades 1}
%\BNCC{EM13LP26}


\subsection{Pré-leitura}

%\textbf{EM13LGG101:} Compreender e analisar processos de produção e
%circulação de discursos, nas diferentes linguagens, para fazer escolhas
%fundamentadas em função de interesses pessoais e coletivos.
%\textbf{EM13LGG102:} Analisar visões de mundo, conflitos de interesse,
%preconceitos e ideologias presentes nos discursos veiculados nas
%diferentes mídias como forma de ampliar suas as possibilidades de
%explicação e interpretação crítica da realidade.
%\textbf{EM13LGG703:} Utilizar diferentes linguagens, mídias e
%ferramentas digitais em processos de produção coletiva, colaborativa e
%projetos autorais em ambientes digitais.
%\textbf{EM13LP04:} Estabelecer relações de interdiscursividade e
%intertextualidade para explicitar, sustentar e qualificar
%posicionamentos e para construir e referendar explicações e relatos,
%fazendo uso de citações e paráfrases devidamente marcadas.
%\textbf{EM13LP29:} Realizar pesquisas de diferentes tipos
%(bibliográfica, de campo, experimento científico, levantamento de dados
%etc.), usando fontes abertas e confiáveis, registrando o processo e
%comunicando os resultados, tendo em vista os objetivos colocados e
%demais elementos do contexto de produção, como forma de compreender como
%o conhecimento científico é produzido e apropriar-se dos procedimentos e
%dos gêneros textuais envolvidos na realização de pesquisas.
%\textbf{EM13LP53:} Criar obras autorais, em diferentes gêneros e mídias
%-- mediante seleção e apropriação de recursos textuais e expressivos do
%repertório artístico --, e/ou produções derivadas (paródias,
%estilizações, fanfics, fanclipes etc.), como forma de dialogar crítica
%e/ou subjetivamente com o texto literário.

\paragraph{Tema} As poetas brasileiras contemporâneas.

\paragraph{Conteúdo} Criação de um panorama com os nomes de poetas
brasileiras contemporâneas, da segunda metade do século XX até a
atualidade. A atividade consiste na apresentação dos trabalhos
utilizando as principais redes sociais, como Twitter, Instagram,
Facebook, TikTok e YouTube.

\paragraph{Objetivo} Capacitar os estudantes a identificar, refletir e
discutir sobre a produção poética contemporânea das mulheres e sua
importância para a compreensão dos diversos usos da linguagem no mundo
virtual.

\paragraph{Justificativa} Não é nenhuma novidade que as redes sociais e
outras plataformas do mundo virtual vêm ocupando cada vez mais o
cotidiano de todos nós, especialmente dos estudantes, que vêm crescendo
e se formando na chamada Web 2.0. Tendências, estilos e pontos de vista
são profundamente influenciados pelo contato dos jovens com essa
realidade (ou, seria multi-realidade?), conferindo novas maneiras de
perceber, sentir, se expressar e se posicionar no mundo.

Mas nem tudo é apenas novidade. Embora os meios em que as informações e
as experiências dos indivíduos tenham sido transformados nas últimas
décadas, há elementos e estruturas que são mais antigos que os nossos
próprios avós. As redes sociais, ainda que sejam uma realidade
completamente diferente daquela que vivíamos há poucas décadas atrás,
conservam práticas, pensamentos e sensibilidades desde épocas
imemoriais, refletindo a sociedade em suas principais dimensões.

A poesia é um dos gêneros mais interessantes para desenvolver as
habilidades dos estudantes nos diversos usos da linguagem e no estímulo
de sua sensibilidade, tão importante nos dias atuais. Um poema não é
apenas uma justaposição de palavras e frases curtas, mas a expressão da
subjetividade do indivíduo e das formas como ele percebe e sente a
realidade que o cerca. Por esta razão, alguns poemas também podem
discutir questões sociais fundamentais de seu tempo, possibilitando que
o leitor reflita sobre o tema discutido e possa lançar mão de ações para
a superação dos problemas envolvidos.

Não são poucas as mulheres que escreveram e escrevem poemas expressando
as inquietações de seus respectivos tempos. A historiografia literária é
marcada por processos de apagamento dessas escritoras, como se sua voz
não tivesse o mesmo prestígio das vozes masculinas, sintomas de uma
sociedade que excluiu (e ainda exclui) as mulheres dos círculos
intelectuais e culturais. Não são poucas as representações pictóricas
das mulheres leitoras no século XIX, mas elas não foram e não são apenas
leitoras, são também poetas, romancistas, contistas, cronistas,
dramaturgas, ensaístas e editoras. Por isso, esta atividade pode
contribuir para que os educandos ampliem a sua perspectiva sobre a
produção intelectual e artística das poetas contemporâneas, que vêm
ocupando cada vez mais o seu espaço, demonstrando suas habilidades
artísticas, expressando suas inquietações e denunciando toda tentativa
que ainda persiste de calar a suas vozes.

\paragraph{Metodologia}
\begin{enumerate}
\item Como ponto de partida, o professor pode começar
perguntando aos estudantes o que é poesia, sem se preocupar com uma
definição teórica ou determinada nesse momento. Anotar na lousa as
definições mais recorrentes dada pelos participantes da aula ou pedir
para que eles as registrem e mantenham para a etapa posterior da
discussão são boas opções. O mais importante, nesse momento, é elaborar,
coletivamente, uma definição do gênero literário tendo em vista as
experiências de cada um. Ainda que ninguém tenha lido um poema na vida
(algo bem difícil de acontecer), em sentido estrito, é fundamental que o
educador os encoraje a dizer o que eles entendem como um texto do
gênero. E essa percepção é bastante útil na presente atividade, pois
estimula os estudantes a pensar no texto levando em conta não a sua
forma acabada e fechada, mas o seu processo de construção e os elementos
estruturais que o compõem, ou seja, da parte para o todo.

\item Proponha que a atividade seja realizada individualmente, em dupla ou,
no máximo, em trio. Um bom ponto de partida é o livro \emph{As 29 poetas
hoje} (2021), organizado por Heloisa Buarque de Hollanda, com poemas de
29 poetas mulheres que escrevem nos dias atuais. Como toda seleção, ela
parte de critérios determinados, assim como você e a turma podem definir
outros para a escolha das poetas para esta atividade.


\Image{Heloísa Buarque de Hollanda é autora de "Pensamento feminista brasileiro" e organizadora do livro "As 29 poetas hoje", lançado em 2021. (Produção Cultural no Brasil; CC BY-SA 2.0)}{PNLD0030-08.png}

\item É importante delimitar um marco temporal. Considerando que este
manual é sobre a edição dos \emph{Poemas escolhidos}, de Orides Fontela,
você pode determinar o período de produção da poeta paulista como
marcador. Na delimitação dessa fase, vale a pena discutir com os
estudantes a própria categoria do que vem a ser contemporânea. Embora
esta seja uma atividade voltada para você, professora ou professor de
língua portuguesa, talvez ela possa também ser feita com a professora ou
professor de história, a fim de enriquecer esse complexo debate sobre o
que é o tempo e o que o determina.

\item A partir das poetas e poemas selecionados, os estudantes devem
apresentar os resultados de sua pesquisa através dos recursos das
principais redes sociais, como Instagram, Twitter, TikTok, YouTube,
entre outras sugeridas pelos educandos ou que você considere adequado.
Discuta com eles de que modo esses recursos virtuais dialogam com a
poesia da artista selecionada. Há casos de produção recente em que a
divulgação se dá justamente por esse meio. 

\Image{A indiana Rupi Kaur é uma das responsáveis por trazer visibilidade para mulheres poetas através das redes sociais. (Baljit Singh; CC BY-SA 4.0)}{PNLD0030-09.png}

É o caso das Instapoetas, como a indiana Rupi Kaur, uma das responsáveis por alavancar o gênero.
No Brasil, temos, por exemplo, Ryane Leão, Eulíricas, Germana Zanettini
e Verena Smit entre as mais conhecidas. Mas é claro que, no universo
inesgotável das redes sociais, há muitas outras mulheres poetas. Esta é
a oportunidade de conhecê-las ou de levar para as redes a potência que
tem vibrado no papel impresso e em outras páginas.
\end{enumerate}

\paragraph{Tempo estimado} Duas aulas de 50 minutos.

\subsection{Leitura}

%\textbf{EM13LGG101:} Compreender e analisar processos de produção e
%circulação de discursos, nas diferentes linguagens, para fazer escolhas
%fundamentadas em função de interesses pessoais e coletivos.
%\textbf{EM13LGG302:} Compreender e posicionar-se criticamente diante de
%diversas visões de mundo presentes nos discursos em diferentes
%linguagens, levando em conta seus contextos de produção e de circulação.
%\textbf{EM13LGG603:} Expressar-se e atuar em processos criativos que
%integrem diferentes linguagens artísticas e referências estéticas e
%culturais, recorrendo a conhecimentos de naturezas diversas (artísticos,
%históricos, sociais e políticos) e experiências individuais e coletivas.
%\textbf{EM13LP01:} Relacionar o texto, tanto na produção como na
%recepção, com suas condições de produção e seu contexto sócio-histórico
%de circulação (leitor previsto, objetivos, pontos de vista e
%perspectivas, papel social do autor, época, gênero do discurso etc.).
%\textbf{EM13LP29:} Realizar pesquisas de diferentes tipos
%(bibliográfica, de campo, experimento científico, levantamento de dados
%etc.), usando fontes abertas e confiáveis, registrando o processo e
%comunicando os resultados, tendo em vista os objetivos colocados e
%demais elementos do contexto de produção, como forma de compreender como
%o conhecimento científico é produzido e apropriar-se dos procedimentos e
%dos gêneros textuais envolvidos na realização de pesquisas.
%\textbf{EM13LP45:} Compartilhar sentidos construídos na leitura/escuta
%de textos literários, percebendo diferenças e eventuais tensões entre as
%formas pessoais e as coletivas de apreensão desses textos, para
%exercitar o diálogo cultural e aguçar a perspectiva crítica.

\paragraph{Tema} Tradição e modernidade na poesia.

\paragraph{Conteúdo} Debate sobre a relação entre os conceitos de tradição
e modernidade a partir dos poemas de Orides Fontela, identificando as
principais questões que eles mobilizam. A atividade consiste na criação
de um blog com os textos de análise crítica dos alunos.

\paragraph{Objetivo} Capacitar os estudantes a identificar e discutir como
os conceitos de tradição e modernidade dialogam na poesia de Orides
Fontela e como eles contribuem para explicar o desenvolvimento da
linguagem artística.

\paragraph{Justificativa} Mário de Andrade, em uma de suas primeiras
cartas ao então jovem Pedro Nava, afirma que a poesia é tradicional
desde o homem das cavernas até a atualidade. Naquela ocasião, a
``atualidade'' era o decênio de 1920, marcado por transformações
culturais, estéticas, políticas e sociais, mas a definição do poeta
paulista ainda pode ser, de alguma maneira, lida hoje, praticamente um
século depois da carta mencionada.

As relações entre a tradição e a modernidade são complexas e nem sempre
se opõem. Cada tempo estabelece um diálogo com uma dada tradição, ou
seja, com um conjunto de noções, hábitos e procedimentos de um período
histórico que podem servir de modelo para a posteridade. O dicionário a
define como ato ou efeito de transmitir ou entregar. A apropriação,
reprodução ou rejeição dessa tradição vai depender dos caminhos tomados
pelas tendências que surgem.

O modernismo, ou seja, um dos movimentos estéticos e culturais da
modernidade, tinha a intenção, em sua fase inicial, de romper com as
estéticas do passado, mas um passado imediatamente anterior, como o
parnasianismo, por exemplo. No entanto, autores como Mário de Andrade
compreendiam o que permanecia e o que se transformava dessas lições do
passado. Uma das características das vanguardas dos anos 1950 foi a
ruptura com os pressupostos do modernismo da primeira fase,
aprofundando, por exemplo, a crise do verso tradicional, entre outras
questões estéticas.

A poesia de Orides Fontela surge no fim dos anos 1960, em um período
pós-vanguardista, quando a poesia passou a tomar outras direções e se
ajustar a novas transformações. Orides dialoga, então, com o modernismo
brasileiro, tanto em sua última fase, por meio de João Cabral de Melo
Neto, quanto em sua fase inicial, não reproduzindo suas lições, e sim
renovando-as de uma maneira muito pessoal. Ao mesmo tempo, Orides
Fontela também lança mão das conquistas das vanguardas dos anos 1950,
como a exploração da dimensão visual. Portanto, esta atividade, a partir
dos poemas de Orides Fontela, é fundamental para entender as complexas
relações entre o tradicional e o moderno na literatura e nas artes em
geral.

\paragraph{Metodologia}
\begin{enumerate}
\item Discuta com a turma os conceitos de tradição e
modernidade, a fim de que possam analisar os poemas considerando essas
perspectivas críticas. Comece perguntando o que cada um entende por
tradicional e moderno com exemplos do cotidiano. O que faz com que algo
seja considerado parte de uma tradição? Ela se opõe ao que entendemos
como moderno? E em que medida esses conceitos não são necessariamente
excludentes e podem dialogar entre si? Encaminhe o debate para o campo
das artes e traga exemplos, como a relação do modernismo com as
estéticas do passado, apontando os aspectos em que ele também dialoga
com elas.

\item Organize a turma em grupos e peça para que selecionem um conjunto de
poemas de Orides Fontela desta antologia. Antes de analisar os poemas,
eles devem apresentar inicialmente as distinções e aproximações entre o
tradicional e o moderno. Depois disso, os poemas servirão como objetos
de análise considerando esses dois conceitos. Antes de escrever o texto
de crítica, sugira que eles analisem os poemas separadamente em um
rascunho, identificando os elementos da forma, como o tipo de estrofe e
o ritmo, e discutindo as imagens produzidas. Opcionalmente, você pode
pedir que eles entreguem os rascunhos ou comentá-los em sala de aula, a
fim de acompanhar o processo de reflexão crítica e analítica dos
educandos.

\item A apresentação dos trabalhos pode ser feita de duas maneiras: a) uma
equipe da turma formada por representantes de cada grupo fica
responsável pela criação do blog e pela edição dos textos que vão ser
publicados. A escrita dos textos críticos pode ser feita pelos demais
membros do grupo, que podem produzi-lo coletiva ou individualmente; b)
ou, cada grupo cria o seu blog, publicando os textos de seus membros,
que podem unificar a análise dos poemas ou analisá-los separadamente.
\end{enumerate}

\paragraph{Tempo estimado} Quatro aulas de 50 minutos.


\subsection{Pós-leitura}

%\textbf{EM13LGG101:} Compreender e analisar processos de produção e
%circulação de discursos, nas diferentes linguagens, para fazer escolhas
%fundamentadas em função de interesses pessoais e coletivos.
%
%\textbf{EM13LGG301:} Participar de processos de produção individual e
%colaborativa em diferentes linguagens (artísticas, corporais e verbais),
%levando em conta seus funcionamentos, para produzir sentidos em
%diferentes contextos.
%
%\textbf{EM13LP13:} Planejar, produzir, revisar, editar, reescrever e
%avaliar textos escritos e multissemióticos, considerando sua adequação
%às condições de produção do texto, no que diz respeito ao lugar social a
%ser assumido e à imagem que se pretende passar a respeito de si mesmo,
%ao leitor pretendido, ao veículo e mídia em que o texto ou produção
%cultural vai circular, ao contexto imediato e sócio-histórico mais
%geral, ao gênero textual em questão e suas regularidades, à variedade
%linguística apropriada a esse contexto e ao uso do conhecimento dos
%aspectos notacionais (ortografia padrão, pontuação adequada, mecanismos
%de concordância nominal e verbal, regência verbal etc.), sempre que o
%contexto o exigir.
%
%\textbf{EM13LP46:} Participar de eventos (saraus, competições orais,
%audições, mostras, festivais, feiras culturais e literárias, rodas e
%clubes de leitura, cooperativas culturais, jograis, repentes, slams
%etc.), inclusive para socializar obras da própria autoria (poemas,
%contos e suas variedades, roteiros e microrroteiros, videominutos,
%playlists comentadas de música etc.) e/ou interpretar obras de outros,
%inserindo-se nas diferentes práticas culturais de seu tempo.
%
%\textbf{EM13LP53:} Criar obras autorais, em diferentes gêneros e mídias
%-- mediante seleção e apropriação de recursos textuais e expressivos do
%repertório artístico --, e/ou produções derivadas (paródias,
%estilizações, fanfics, fanclipes etc.), como forma de dialogar crítica
%e/ou subjetivamente com o texto literário.

\paragraph{Tema} Escrever para pensar.

\paragraph{Conteúdo} Exercitar a escrita literária como forma de
construção e organização do pensamento, a partir da percepção da
realidade e da relação do estudante com ela, tendo a poesia de Orides
Fontela como uma das principais referências criativas. A atividade
consiste na criação de uma oficina de poesia.

\paragraph{Objetivo} Capacitar os estudantes a criar poemas em uma oficina
de criação literária e discutir os principais elementos de seu processo
criativo, estimulando a criatividade, a sensibilidade e a reflexão
crítica.

\paragraph{Justificativa} Escrever é também ler, assim como a leitura
também produz um texto, ainda que não necessariamente escrito. Portanto,
estimular e desenvolver a escrita nos estudantes é fundamental para que
eles possam exercitar não apenas os elementos básicos na construção de
um texto, mas também a sua relação com a realidade que os cercam, em
suas dimensões estética, social, ética e política. Os poemas de Orides
Fontela podem servir de lição, especialmente na relação entre escrita e
pensamento, ou seja, escrever como uma forma de construção do pensar.

Neste sentido, a dinâmica de uma oficina literária permite não apenas
refletir e discutir sobre a concepção de gêneros literários, mas também
aguçar a criatividade, a percepção, a sensibilidade e a visão de mundo
dos educandos. A atividade não é para ser uma espécie de ``agência de
caça-talentos'' literária, claro. Não que não seja possível que exista
entre os alunos uma sementinha de uma grande escritora ou de um grande
escritor. Se houver, ótimo! Contudo, o mais importante é que todos
possam exercitar ativamente sua capacidade de criar e recriar visões de
mundo a partir de um poema, aprendendo que escrever é também uma forma
crítica de leitura.

\paragraph{Metodologia} Esta atividade pode ser o projeto de algo mais
amplo e, quem sabe, permanente dentro da escola, abrangendo outros
gêneros literários e outras turmas.

Para a criação da oficina literária de poesia, você pode seguir os
passos a seguir, com a possibilidade, evidentemente, de ser adaptados de
acordo com o lugar, o tempo e as condições disponíveis para que a
atividade seja desenvolvida:

\begin{enumerate}
\item
  Organize os alunos em uma roda, a fim de que a atividade de produção
  de textos seja, sobretudo, uma atividade de compartilhamento de
  experiências;
\item
  Embora a escrita seja considerada, em princípio, como uma empreitada
  individual, nada impede que também seja coletiva, portanto, permita
  que grupos de dois ou mais estudantes sejam formados. É uma boa
  oportunidade, aliás, para discutirem a questão da autoria na criação
  literária. Apenas limite o número de integrantes, a fim de que a
  atividade seja proveitosa para todos. Um grupo formado por um número
  muito grande de alunos pode acabar dispersando-os e frustrando o
  objetivo da oficina;
\item
  Proponha a descrição de uma paisagem a partir da emoção que ela
  provocou. Pode ser também um objeto, uma construção, uma rua ou
  qualquer outro elemento material. Ele também pode partir de uma
  indignação coletiva, como o problema da desigualdade, mas é preciso
  partir de um exemplo concreto. Além dos poemas de Orides Fontela da
  antologia, mostre também o ``Poema retirado de uma notícia de
  jornal'', de Manuel Bandeira, que se encontra em seu livro
  \emph{Libertinagem} (1930); e ``Procura da poesia'', de Carlos
  Drummond de Andrade, que está em seu livro \emph{A rosa do povo}
  (1945). Discutam o processo de criação dos poemas e os recursos
  utilizados por cada poeta na construção dos versos;
\item
  Em alguns encontros da oficina, proponha discussões sobre filmes;
\item
  À medida que os encontros da oficina avançarem, incentive sempre que
  eles apresentem parte do desenvolvimento de seus textos, ainda que
  estejam incompletos e travados em algum ponto. A discussão em sala de
  aula pode contribuir para que eles solucionem alguns desses problemas;
\item
  A apresentação dos poetas fica a critério dos educandos, ou de acordo
  com o que você e a turma decidirem juntos. Os textos podem ser
  reunidos de diversas maneiras, a depender dos recursos que estiverem
  disponíveis. Se houver a possibilidade de publicação do material
  impresso, não hesite, claro, mas não deixe de estimular a divulgação
  do trabalho através dos meios que o universo digital oferece.
\end{enumerate}

\paragraph{Tempo estimado} Um a dois bimestres.


\section{Atividades 2}

\subsection{Pré-leitura}

%\textbf{EM13LGG101:} Compreender e analisar processos de produção e
%circulação de discursos, nas diferentes linguagens, para fazer escolhas
%fundamentadas em função de interesses pessoais e coletivos.
%
%\textbf{EM13LGG201:} Utilizar adequadamente as diversas linguagens
%(artísticas, corporais e verbais) em diferentes contextos,
%valorizando-as como fenômeno social, cultural, histórico, variável,
%heterogêneo e sensível aos contextos de uso.
%
%\textbf{EM13LP01:} Relacionar o texto, tanto na produção como na
%recepção, com suas condições de produção e seu contexto sócio-histórico
%de circulação (leitor previsto, objetivos, pontos de vista e
%perspectivas, papel social do autor, época, gênero do discurso etc.).
%
%\textbf{EM13LP33:} Produzir textos para a divulgação do conhecimento e
%de resultados de levantamentos e pesquisas -- texto monográfico, ensaio,
%artigo de divulgação científica, verbete de enciclopédia (colaborativa
%ou não), infográfico (estático ou animado), relato de experimento,
%relatório, relatório multimidiático de campo, reportagem científica,
%podcast ou vlog científico, apresentações orais, seminários,
%comunicações em mesas redondas, mapas dinâmicos etc. --, considerando o
%contexto de produção e utilizando os conhecimentos sobre os gêneros de
%divulgação científica, de forma a engajar-se em processos significativos
%de socialização e divulgação do conhecimento.
%
%\textbf{EM13CHS101:} Analisar e comparar diferentes fontes e narrativas
%expressas em diversas linguagens, com vistas à compreensão e à crítica
%de ideias filosóficas e processos e eventos históricos, geográficos,
%políticos, econômicos, sociais, ambientais e culturais.
%
%\textbf{EM13CHS401:} Identificar e analisar as relações entre sujeitos,
%grupos e classes sociais diante das transformações técnicas,
%tecnológicas e informacionais e das novas formas de trabalho ao longo do
%tempo, em diferentes espaços e contextos.
%
%\textbf{EM13CHS504:} Analisar e avaliar os impasses ético-políticos
%decorrentes das transformações científicas e tecnológicas no mundo
%contemporâneo e seus desdobramentos nas atitudes e nos valores de
%indivíduos, grupos sociais, sociedades e culturas.

\paragraph{Tema}{A cultura na sociedade brasileira entre os anos 1960 e 1980.}

\paragraph{Conteúdo} Estudo e pesquisa de representações da cultura na
sociedade brasileira entre as décadas de 1960 e 1980, destacando também
suas dinâmicas sociais e políticas. A atividade consiste em uma
exposição virtual ou física.

\paragraph{Objetivo} Capacitar os estudantes a identificar, refletir e
discutir as dinâmicas da sociedade brasileira entre os anos 1960 e 1980
a partir do seu cenário cultural.

\paragraph{Justificativa} A sociedade brasileira sofreu grandes
transformações políticas, sociais e econômicas entre os anos 1960 e
1980, refletindo em sua vasta e múltipla produção artística e cultural.
A música, o teatro, a literatura, as artes visuais e outras expressões
buscaram expressar as inquietações de um tempo marcado por lutas em
diversas esferas da sociedade. Também foi um tempo de consolidação do
conhecimento produzido pela academia e de ampliação dos meios de
circulação de bens culturais.

Apesar de a poesia de Orides Fontela não estar vinculada a uma geração
ou grupo específico, ela é produto de seu tempo e surge como resultado
das condições que se apresentaram. Inclusive a miséria que a poeta
atravessou no fim da vida é consequência dessas condições, ou seja, dos
rumos tomados pela sociedade brasileira. Portanto, esta atividade pode
contribuir para que os educandos reflitam sobre como a cultura
representou as diversas tendências, mudanças e problemas vivenciados no
país durante esses decênios.

\paragraph{Metodologia}

\begin{enumerate} 
\item Apresente e discuta de forma panorâmica com os
alunos alguns dos principais aspectos históricos dos anos 1960 aos 1980
no Brasil. Pontue os momentos-chave desse intervalo, de modo a ajudar os
educandos a encontrar direções a serem tomadas para o desenvolvimento
dos trabalhos. A ideia é servir de base para o que eles irão aprofundar
nas pesquisas e apresentar na classe.


\Image{O fim da década de 1970 e começo de 1980 no Brasil foi marcado pelo movimento das Diretas Já, que reuniu milhares de pessoas nas ruas se manifestando a favor da volta das eleições. (Arquivo da Agência Brasil; CC-0)}{PNLD0030-10.png}


\item Depois dessa breve exposição, organize as turmas em grupos e proponha
que escolham um tema relacionado ao período. Pode ser o nome de um
artista, de uma obra, de uma estética determinada, de uma tendência, de
uma área do conhecimento etc. A seguir, algumas sugestões temáticas, que
podem, claro, servir apenas como ideia inicial para a escolhe de outros
objetos de pesquisa:

\begin{itemize}
\item
  Contracultura
  
  \Image{A obra "Seja marginal, seja herói" de Hélio Oiticica é um marco do movimento de cultura marginal, que passou a fazer parte do debate cultural brasileiro a partir do final de 1968. (Wally Gobetz; CC BY-NC-ND 2.0)}{PNLD0030-11.png}
  
\item
  Geração mimeógrafo ou poesia marginal
\item
  Cinema novo
  
  \Image{Alguns diretores do Cinema Novo: movimento cinematográfico brasileiro marcante nos anos 1960 e 1970 e que se destacou por sua crítica a desigualdade social e a Ditadura Civil-Militar. (Johnsmith22who; CC BY-SA 4.0)}{PNLD0030-12.png}
  
\item
  Tropicalismo
\item
  Teatro de Arena
  
 \Image{O Teatro de Arena foi um dos mais importantes grupos teatrais brasileiros das décadas de 1950 e 1960. (Henrique Artuni; CC BY-SA 2.0)}{PNLD0030-13.png}
 
\item
  Festival da Canção
  
  \Image{Gilberto Gil e Nana Caymmi no III Festival da Música Popular (Arquivo Nacional; Domínio Público)}{PNLD0030-14.png}
   
  \Image{Caetano Veloso no III Festival da Música Popular (Arquivo Nacional; Domínio Público)}{PNLD0030-15.png}

\end{itemize}




\item Cada grupo deve montar uma exposição física ou virtual sobre o tema
escolhido. A metodologia precisa ser bem definida. Vai ser apenas de
imagens ou de vídeos? Ou ainda uma combinação dos dois? Vai haver
apresentação de alguns objetos, páginas de jornais, revistas ou
ilustrações? Exemplares de livros ou fac-símile de manuscritos?

\item Depois de delimitar a metodologia de pesquisa e apresentação dos
resultados, eles devem dividir a exposição em seções, de maneira que as
partes se articulem como em uma história, afinal, uma exposição não é
apenas a presença pura e simples do objeto relacionado ao tema, mas uma
narrativa sobre ele. Então, eles precisam contar uma história!

\item Os estudantes podem pesquisar tanto nos arquivos físicos quanto nos
virtuais. E são inúmeros! Na rede, eles podem consultar a Hemeroteca
Digital da Biblioteca Nacional, com praticamente todo o acervo de
periódicos mantidos pela instituição, que também disponibiliza imagens
em sua Brasiliana Fotográfica. As páginas de importantes arquivos no
país também presta esse serviço, como o Arquivo Nacional, assim como
instituições privadas, como Instituto Moreira Salles. A Biblioteca
Brasiliana Guita e José Mindlin, da Universidade de São Paulo, também
disponibiliza diversos periódicos e livros raros em seu acervo digital.
É preciso estar atento, pois algumas cobram uma taxa para a reprodução
de imagens. E não se esqueça de avisá-lo de sempre citar a fonte!

\item Cada grupo deve produzir também um texto de curadoria da exposição, a
ser distribuído entre os colegas, informando o tema, a metodologia, os
itens expostos, os acervos consultados e uma análise sobre o conjunto em
diálogo com o tema escolhido.
\end{enumerate}

\paragraph{Tempo estimado} Quatro aulas de 50 minutos.

\subsection{Leitura}

%\textbf{EM13LGG101:} Compreender e analisar processos de produção e
%circulação de discursos, nas diferentes linguagens, para fazer escolhas
%fundamentadas em função de interesses pessoais e coletivos.
%
%\textbf{EM13LGG201:} Utilizar adequadamente as diversas linguagens
%(artísticas, corporais e verbais) em diferentes contextos,
%valorizando-as como fenômeno social, cultural, histórico, variável,
%heterogêneo e sensível aos contextos de uso.
%
%\textbf{EM13LP04:} Estabelecer relações de interdiscursividade e
%intertextualidade para explicitar, sustentar e qualificar
%posicionamentos e para construir e referendar explicações e relatos,
%fazendo uso de citações e paráfrases devidamente marcadas.
%
%\textbf{EM13LP15:} Elaborar roteiros para a produção de vídeos variados
%(vlog, videoclipe, videominuto, documentário etc.), apresentações
%teatrais, narrativas multimídia e transmídia, podcasts, playlists
%comentadas etc., para ampliar as possibilidades de produção de sentidos
%e engajar-se de forma reflexiva em práticas autorais e coletivas.
%
%\textbf{EM13CHS101:} Analisar e comparar diferentes fontes e narrativas
%expressas em diversas linguagens, com vistas à compreensão e à crítica
%de ideias filosóficas e processos e eventos históricos, geográficos,
%políticos, econômicos, sociais, ambientais e culturais.
%
%\textbf{EM13CHS103:} Elaborar hipóteses, selecionar evidências e compor
%argumentos relativos a processos políticos, econômicos, sociais,
%ambientais, culturais e epistemológicos, com base na sistematização de
%dados e informações de natureza qualitativa e quantitativa (expressões
%artísticas, textos filosóficos e sociológicos, documentos históricos,
%gráficos, mapas, tabelas etc.).
%
%\textbf{EM13CHS104:} Analisar objetos da cultura material e imaterial
%como suporte de conhecimentos, valores, crenças e práticas que
%singularizam diferentes sociedades inseridas no tempo e no espaço.

\paragraph{Tema} Diálogos entre a literatura e a filosofia.

\paragraph{Conteúdo} Estudo e pesquisa sobre as relações entre a
literatura e a filosofia a partir dos poemas de Orides Fontela. A
atividade consiste na apresentação dos trabalhos por meio de
representações cênicas, como teatro, cinema, dança, performance ou
outras expressões.

\paragraph{Objetivo} Capacitar os estudantes a identificar e a refletir
sobre as aproximações e as diferenças entre a literatura e a filosofia,
tendo em vista a influência desses dois campos na produção de Orides
Fontela.

\paragraph{Justificativa} As relações entre a literatura e a filosofia são
antigas e suscitam questões fundamentais entre elas até hoje. Na
antiguidade, os filósofos já se ocupavam de conceitos caros à
literatura, como Aristóteles e Platão. Ao longo da história, as
preocupações filosóficas, de certa maneira, sempre foram acompanhadas
pelas transformações operadas nos campos das artes, que buscaram
absorver, ou até mesmo antecipar, o espírito do tempo e tentaram
apresentar, seja por meio da poesia, do texto dramático ou da narrativa
em prosa, os elementos que acabaram se tornando objetos de reflexão e
sistematização por filósofos, influenciando todo o sistema de pensamento
de sua época. Assim também a filosofia, desde a \emph{Poética}, de
Aristóteles, tem papel essencial nas bases teóricas da tradição crítica
dos estudoa literários.

Orides Fontela teve formação em filosofia pela Universidade de São
Paulo, em um momento de crescente prestígio da instituição e de
consolidação de seus cursos de pós-graduação em ciências humanas. Orides
frequentou as aulas e dialogou com importantes críticos e intelectuais,
o que influenciou seu fazer poético. Contemporaneamente, temos exemplos
de outros poetas e escritores que, de certa maneira, se aproximam ou se
afastam dessa influência, como Antonio Cicero, na poesia, e Juliano
Garcia Pessanha, na prosa.

Assim, esta atividade contribui para que os educandos percebam não
apenas a importância de uma disciplina em relação a outra, mas os
caminhos e as condições que valorizam a interdiscisciplinaridade,
discutindo o seu papel na formação intelectual de cada um.

\paragraph{Metodologia} 

\begin{enumerate}
\item Como ponto de partida, você pode apresentar os
seguintes fragmentos do texto de Orides Fontela sobre as relações entre
poesia e literatura, um dos últimos que escreveu antes de falecer:

\begin{quote}
``Alta agonia é ser, difícil prova'' é o primeiro verso de um soneto
meu, escrito aos 23 anos --- um soneto muito importante para mim, pois é
uma espécie de programa de vida, que não renego nunca e nem jamais
conseguirei cumprir, porém é minha tarefa tentar. Difícil prova, sim,
impossível, pois isso constitui propriamente o humano. E, claro, todas
as ferramentas servem, principalmente a religião (sobre o aspecto
místico), a poesia --- intuições básicas e... musicais, que tive de
nascença --- e a bem mais recente, a filosofia. Deixando a religião de
lado (mas fica lá́, por baixo), falemos só́ de poesia e filosofia.

Arcaica como o verbo é a poesia, velha como o cântico. A poesia, como o
mito, também pensa e interpreta o ser, só́ que não é pensamento puro,
lúcido. Acolhe o irracional, o sonho, inventa e inaugura os campos do
real, canta. Pode ser lúcida, se pode pensar --- é um logos --- mas não
se restringe a isso. Não importa: poesia não é loucura nem ficção, mas
sim um instrumento altamente válido para a apreender o real --- ou pelo
menos meu ideal de poesia é isso. Depois é que surgem o esforço para a
objetividade e a lucidez, a filosofia. Fruto da maturidade humana,
emerge lentamente da poesia e do mito, e inda guarda as marcas de
co-nascença, as pegadas vitais da intuição poética. Pois ninguém chegou
a ser cem por cento lúcido e objetivo, nunca. Seria inumano, seria
loucura e esterilidade. Bem, aí já́ temos uma diferença básica entre
poesia e filosofia --- a idade, a técnica, não o escopo. Pois a
finalidade de entender o real é sempre a mesma, é ``alta agonia'' e
``difícil prova'' que devemos tentar para realizar nossa humanidade.
Isso é o que temos a dizer, inicialmente, sobre a filosofia e poesia.

Bem, fazer poesia fiz sempre, e curiosa sempre fui. ``Que bicho é
esse?'' era minha pergunta de aluninha. ``Ti esti'', ``que é'', pergunta
o filósofo. É pergunta igual... Aos dezesseis anos fiz os seguintes
versos:

\begin{verse}
\emph{Pensar dói}\\
\emph{e não adianta nada.}
\end{verse}

Maus versos, mas intuição válida. Pensar dói mesmo, faz cócegas, pode
ser tão irreprimível como a curiosidade da aluninha. E de que adianta?
Bem, o caso é que eu não engolia, nem engulo, respostas já prontas,
quero ir lá eu mesma, tentar. Tentava pela poesia. Ora, uma intuição
básica de minha poesia é o ``estar aqui'' --- auto-descoberta e
descoberta de tudo, problematizando tudo ao mesmo tempo. Só que este
``estar aqui'' é, também, estar ``a um passo'' --- de meu espírito, do
pássaro, de Deus --- e este um passo é o ``impossível'' com que luto. É
o paradoxo que exprimo num poemeto.

\begin{verse}
\emph{Próxima: mas ainda estrela}\\
\emph{muito mais estrela que próxima.}
\end{verse}

Ora, esta posição existencial básica de meus poemas já é filosó- fica,
isto é, seria possível desenvolvê-la em filosofia, e daí veio meu
interesse pela filosofia propriamente dita. Eu vivia a intuição quase
inefável de estar só ``a um passo'', que bastava erguer um só véu.
Mocidade! E aí entra na minha vida a filosofia explícita. Entrou em
aulas da Escola Normal, entrou pelos livros que procurei conseguir
(Pascal, Gilson, Maritain, e até alguns não tão ortodoxos), e
misturou-se a um interesse pela mística --- Huxley, Sta. Tereza, São
João da Cruz. Salada de que resultou meu livro ``Transposição'', muito
``abstrato'' e ``pensado'' --- no sentido poético de tais termos. Girava
em torno do problema do ser e da lucidez, e abusava do termo ``luz''. Um
livro estranho, que só recentemente percebi como estava na contramão da
poesia brasileira, sensual e sentimental. Parecia até meio cabralino
devido a um vezo analítico, mas nunca foi, claro. Era um livro escrito
no interior, tramado pelas tendências já levantadas, e onde já poesia e
filosofia tentavam se irmanar, como possível.

Não preciso explicar, agora, porque meu interesse por filosofia era
quase inato, como a poesia. Assim, agarrei a oportunidade de fazer
realmente filosofia. Talvez desse em algo prático (não deu), mas o que
me interessava era, acreditem ou não, a Verdade. Ingenuidade? Hoje sei
que era, mas era a própria ingenuidade nobre sem a qual não se cria. E
lá parti eu para tentar a filosofia, continuando com a poesia
naturalmente. E o curioso é que estas águas não se mesclaram mais do que
já estavam, senão a poesia poderia se tornar seca e não espontânea. Mas
dei sorte (!) de não me tornar filósofa... Aliás, o mais que conseguiria
seria ser uma professora de filosofia, isto é, uma técnica no assunto
--- e, bom, não era essa a finalidade. Nem dava; faltava base econômica
e cultural. Pobre e vindo apenas do Normal só consegui terminar o curso.
Mas me diverti muito.
\end{quote}

\item O texto foi publicado no livro \emph{Poesia (e) filosofia: por poetas
filósofos em atuação no Brasil} (2018). Depois da leitura e discussão do
texto, você pode desdobrar as questões que eles trazem e pedir que os
alunos apresentem as relações que eles encontram entre a filosofia e a
poesia, a filosofia e o romance, a filosofia e o conto, ou ainda entre a
filosofia e a personagem, a filosofia e a metáfora, ou seja, buscar
estabelecer diálogos possíveis entre a reflexão filosófica e os
elementos fundamentais da literatura e dos estudos literários.

\item Organize a turma em grupos e proponha que eles desenvolvam uma
apresentação cênica que discuta essa relação. Eles podem partir de um
dos poemas de Orides Fontela ou de episódios de sua biografia.

\item Eles podem apresentar o trabalho por meio de qualquer linguagem
cênica: teatro, dança, cinema, performance, \emph{Slams} ou até mesmo
animação. O documentário sobre a vida e a obra de Orides Fontela,
\emph{A um passo do pássaro}, com direção de Ivan Marques, pode ser uma
referência interessante. Além de fragmentos de entrevistas da poeta e de
críticos, o filme apresenta cenas de uma performance com a declamação de
alguns de seus versos feita por atrizes.

\item Não esqueça de pedir a eles o roteiro do trabalho antes de
apresentá-lo, a fim de que possam comentar o texto, o conteúdo e
metodologia utilizada. Depois de apresentarem os trabalhos e realizarem
os devidos ajustes no texto, eles devem entregar o roteiro final, com
todas as alterações e um roteiro de pesquisa, relacionando as
referências bibliográficas, as fontes e os meios utilizados para chegar
ao trabalho final.
\end{enumerate}

\textbf{Tempo estimado:} Entre um a dois bimestres.

\subsection{Pós-leitura}

\paragraph{Tema} Arte, economia e políticas culturais.

%\textbf{EM13LGG101:} Compreender e analisar processos de produção e
%circulação de discursos, nas diferentes linguagens, para fazer escolhas
%fundamentadas em função de interesses pessoais e coletivos.
%\textbf{EM13LGG201:} Utilizar adequadamente as diversas linguagens
%(artísticas, corporais e verbais) em diferentes contextos,
%valorizando-as como fenômeno social, cultural, histórico, variável,
%heterogêneo e sensível aos contextos de uso.
%\textbf{EM13LP01:} Relacionar o texto, tanto na produção como na
%recepção, com suas condições de produção e seu contexto sócio-histórico
%de circulação (leitor previsto, objetivos, pontos de vista e
%perspectivas, papel social do autor, época, gênero do discurso etc.)
%\textbf{EM13LP33:} Produzir textos para a divulgação do conhecimento e
%de resultados de levantamentos e pesquisas -- texto monográfico, ensaio,
%artigo de divulgação científica, verbete de enciclopédia (colaborativa
%ou não), infográfico (estático ou animado), relato de experimento,
%relatório, relatório multimidiático de campo, reportagem científica,
%podcast ou vlog científico, apresentações orais, seminários,
%comunicações em mesas redondas, mapas dinâmicos etc. --, considerando o
%contexto de produção e utilizando os conhecimentos sobre os gêneros de
%divulgação científica, de forma a engajar-se em processos significativos
%de socialização e divulgação do conhecimento.
%\textbf{EM13CHS102:} Identificar, analisar e discutir as circunstâncias
%históricas, geográficas, políticas, econômicas, sociais, ambientais e
%culturais da emergência de matrizes conceituais hegemônicas
%(etnocentrismo, evolução, modernidade etc.), comparando-as a narrativas
%que contemplem outros agentes e discursos.
%\textbf{EM13CHS404:} Identificar e discutir os múltiplos aspectos do
%trabalho em diferentes circunstâncias e contextos históricos e/ou
%geográficos e seus efeitos sobre as gerações, em especial, os jovens e
%as gerações futuras, levando em consideração, na atualidade, as
%transformações técnicas, tecnológicas e informacionais.
%\textbf{EM13CHS504:} Analisar e avaliar os impasses ético-políticos
%decorrentes das transformações científicas e tecnológicas no mundo
%contemporâneo e seus desdobramentos nas atitudes e nos valores de
%indivíduos, grupos sociais, sociedades e culturas.

\paragraph{Conteúdo} Estudo e pesquisa sobre os desafios e a importância
de iniciativas públicas ou privadas para incentivar a produção artística
e cultural, a fim de contribuir para o desenvolvimento da sociedade e a
democratização do acesso aos bens resultantes dessa produção. A
atividade consiste na apresentação dos trabalhos como texto
jornalístico, impresso ou digital.

\paragraph{Objetivo} Capacitar os estudantes a refletir e discutir sobre a
importância, para a sociedade, do incentivo à produção artística e
cultural por meio de programas do governo ou de entidades civis.

\paragraph{Justificativa} Orides Fontela costumava dizer em suas
entrevistas que a única coisa que sabia fazer na vida era escrever
poesia. Formada em filosofia pela Universidade de São Paulo, ela foi
professora primária e bibliotecária, mas sofreu sérias restrições
financeiras nos últimos anos de vida, sendo, por isso, despejada de seu
apartamento, o que a levou a morar na Casa do Estudante.

Nunca foi fácil viver de literatura ou de qualquer outra expressão
artística no Brasil. Poucos foram os escritores que conseguiram, e
conseguem, viver de seu próprio ofício. Os fatores são inúmeros,
portanto, esta atividade pode contribuir para que os estudantes os
identifiquem e debatam possíveis soluções para esse problema. Afinal, é
possível viver de literatura no Brasil? Qual é a importância da
literatura e de outras artes para o desenvolvimento de um país? Qual
seria o papel do estado no estímulo das atividades artísticas e
culturais? Estas e outras perguntas podem mobilizar a turma a se
aprofundar no tema.

\paragraph{Metodologia} 
\begin{enumerate}
\item Apresente um breve panorama da relação entre
arte, cultura e desenvolvimento socioeconômico do país. Aproveite a
oportunidade para ouvir dos alunos suas impressões sobre as políticas
culturais e a sua importância. Traga dados que possam servir de base
para as discussões e demonstre que esse debate merece ser levado
adiante.

\item Organize a turma em grupos, em que cada um formará uma pequena
redação de jornal. Proponha que as tarefas sejam divididas por função,
mas todos devem estar envolvidos na pesquisa e na divulgação dos
resultados.

\item Cada grupo deve pesquisar temas que discutam a relação entre arte,
cultura e desenvolvimento socioeconômico. Eles podem pesquisar sobre os
programas de incentivo já existentes, seja do governo ou de entidades da
sociedade civil. Também há diversos projetos sociais que fazem uso da
cultura como instrumento de redução das desigualdades. Ou ainda podem se
concentrar em uma figura específica que tenha atuação de destaque nesse
cenário. É um tema amplo e que pode render trabalhos muito
interessantes!

\item Depois de definir o tema, os estudantes jornalistas devem fazer um
levantamento de informações e coletar imagens. A primeira parte do grupo
fica responsável pela escrita, revisão e edição do texto, a segunda
cuida da produção, coleta e tratamento das imagens e uma outra se
encarrega de diagramar o material e publicar o jornal. Dependendo do
número de alunos, evidentemente essas tarefas podem se misturar. O
importante é que todos participem da execução do trabalho.

\item Durante a pesquisa, estimule-os a consultar os arquivos na rede, como
os periódicos mantidos pela Biblioteca Nacional em sua Hemeroteca
Digital, assim como os acervos disponibilizados on-line pelo Arquivo
Nacional, caso eles precisem realizar uma investigação histórica sobre o
tema. É fundamental que eles não deixem de citar as fontes de pesquisa.
\end{enumerate}

\paragraph{Tempo estimado} Um bimestre de aulas.

\section{Aprofundamento}

Selvagem, assim como o silêncio que cresceu difícil de um de seus
versos, nasceu Orides Fontela, em 21 de abril de 1940, na cidade de São
João da Boa Vista, interior de São Paulo. Filha de pai analfabeto,
recebeu da mãe suas lições ainda cedo, chegando a escrever seus
primeiros versos entre os seis e sete anos de idade. Em entrevistas,
Orides comentou que compunha, nessa época, umas quadrinhas infantis, a
mesma fonte de que muitos grandes poetas beberam, como Manuel Bandeira,
um de seus mestres na fase adulta. Na sua cidade natal, frequentou o
ginásio e a escola normal, formando-se como professora primária, em
1955.

No ano seguinte, publicou seus primeiros poemas, no jornal O Município.
Alguns anos mais tarde, se mudou para a capital paulista, para estudar
filosofia na USP, formação que influenciou muito na seu fazer poético.
Os poemas iniciais, publicados naquele periódico paulista, chamaram a
atenção de um conterrâneo, o crítico Davi Arrigucci Junior, de quem
Orides se tornou amiga. Arrigucci Junior foi um dos principais
responsáveis por trazer a obra de Orides Fontela para uma comunidade
maior de leitores, além de intelectuais e críticos paulistas,
coordenando a publicação de seu primeiro livro, \emph{Transposição}, de
1969. Seu terceiro livro, Alba, de 1983, contou com prefácio de Antonio
Candido e ganhou o prêmio Jabuti, o que sugeria a consolidação de uma
carreira literária. Muitos a viam como uma continuidade do modernismo de
João Cabral de Melo Neto, outros, como uma renovadora da fase heroica do
movimento, e outros ainda, como uma voz necessária em um período em que
já se esboçava um esgotamento das vanguardas dos anos 1950.


\Image{Orides Fontela (1940-1998) (Inez Guerreiro; Inez Guerreiro)}{PNLD0030-04.png}


Na verdade, Orides Fontela aparece como uma voz única e dissonante das
tendências que se formavam em seu tempo, por mais dispersivas e
múltiplas que, em geral, elas sejam caracterizadas. A influência do
modernismo na formação intelectual e artística de Orides Fontela foi
fundamental, no entanto, ela também dialogou com preceitos de seu tempo,
como a visualidade na construção dos poemas, à maneira dos poetas
concretos. A dimensão dos símbolos com o qual ela trabalhou fez com que
alguns a enxergassem como neossimbolista, na esteira de uma Cecília
Meireles, mas os vestígios dessa estética a aproximam mais de outra
poeta modernista, a mineira Henriqueta Lisboa. Entre os poetas de sua
geração, ela não se alinha a nenhum grupo.

E essa ausência de relação reflete não apenas a inserção de sua obra,
mas também o seu lugar como intelectual. Infelizmente as dificuldades
financeiras e o seu temperamento considerado difícil a lançou em
situações de privação. Morou de favor em uma residência estudantil na
região central de São Paulo e se isolou cada vez mais dos amigos. Passou
os últimos dias de sua vida em um sanatório, em Campos do Jordão, onde
faleceu em 2 de novembro de 1998. Dois anos antes, havia publicado sua
última obra, Teia, conquistando o prêmio da APCA, a Associação Paulista
de Críticos de Arte. Uma obra, assim como todas as demais, tecida com os
fios do tempo, que amadurece diante da nossa espera.

Seu livro de estreia, \emph{Transposição}, contém poemas escritos entre
1966 e 1967. A sua publicação teve a colaboração do crítico Davi
Arrigucci Junior, conterrâneo de Orides e de quem se tornou grande
incentivador e amigo. O livro se divide em quatro seções, ``Base'',
``(--)'', ``(+)'' e ``Fim'', numerada com algarismos romanos. A primeira
sugere o estabelecimento dos fundamentos construtivos de sua poética,
como se fossem os alicerces de uma edificação. A partir do primeiro
poema, que intitula a obra, somos levados a acompanhar um movimento
entre o real e o símbolo, em um jogo de perda e de alcance do que é
possível. Os sinais negativo e positivo indicam essa tendência: ora o
impasse e a dor, ora a lucidez e a contemplação. É a transposição do
ser, daquilo ele que é e daquilo que falta, para a dimensão simbólica,
que o sustenta e o resgata.

\emph{Helianto} traz poemas de 1973. Enquanto no primeiro, havia uma
noção de deslocamento de um ponto a outro, aqui surge a noção de
movimento circular, também expresso na epígrafe com uma cantiga de roda.
A palavra que intitula a obra é uma designação, em Botânica, para se
referir às plantas do gênero Helianthus, das quais, uma das mais
conhecidas é o girassol, portanto, os signos de lucidez permanecem como
elementos de sua poética, mas, desta vez, eles giram, atando as pontas
entre a palavra e o ser que ela busca representar. É um dos livros em
que Orides mais explora a geometria e materialidade visual de seus
poemas.

Os textos de \emph{Alba} são de 1983 e acompanham um prefácio de Antonio
Candido, conferindo uma estatura consolidada na recepção crítica de sua
poesia. Não por acaso, o livro rendeu a Orides o Prêmio Jabuti. O título
tanto remete à primeira claridade da manhã quanto a um gênero lírico
medieval em que há a despedida de dois amantes nesse mesmo instante do
dia. De uma certa maneira, podemos pensar nesses dois amantes como o ser
e o símbolo, a claridade é a poesia. O instante da despedida, no
entanto, não é a separação consumada, mas o que a antecede e a suspende,
tornando-os conscientes, lúcidos, de que o momento seguinte os divide,
mas aquele em que expressam isso, o presente, ainda os mantém. Não por
acaso, \emph{Alba} é um livro mais maduro em relação à linguagem poética
e à destinação da própria poesia como uma atividade do espírito. A
epígrafe de San Juan de la Cruz dialoga com o aspecto místico-religioso
que também aparece nos poemas, não de forma temática, mas na sobriedade
e gravidade na representação das paixões.

A composição de \emph{Rosácea}, de 1986, é mais heterogênea em relação
aos anteriores e se divide em seções, ``Novos'', ``Lúdicos'',
``Bucólicos'', ``Mitológicos'' e ``Antigos''. As questões que permeiam
sua obra até aqui estão presentes, mas os movimentos circulares e de
transposição possuem referências intelectuais, culturais e afetivas, com
nomes de poetas, filósofos, familiares, entre outros. A sigla de Carlos
Drummond de Andrade (CDA), por exemplo, uma das grandes influências para
a poeta, é uma das mais recorrentes. Essas referências não são um mero
repositório de informações biográficas e bibliográficas. A poesia de
Orides Fontela rejeita qualquer tom confessional ou de reconstituição
nostálgica de algum passado. O tempo é o da experiência poética, que
atualiza essas referências em face do inevitável destino da poesia.


\Image{Parte da frente de bilhete escrito à mão por Orides Fontela. (Arquivo da autora.)}{PNLD0030-06.png}


\Image{Verso de bilhete escrito à mão por Orides Fontela. (Arquivo da autora.)}{PNLD0030-07.png}


Por fim, Orides lançou \emph{Teia}, de 1996, premiado pela APCA, a
Associação Paulista de Críticos de Arte. Com prefácio de Marilena Chauí,
o livro se divide nas seções ``Fala'', ``Axiomas'', ``O antipássaro'',
``Galo {[}Noturnos{]}'', ``Figuras'' e ``Vésper {[}Finais{]}''. Neste
seu último livro, a poeta retoma as questões e as imagens que a
acompanharam em seu fazer poético, expressão do silêncio extraído dos
símbolos em seu encontro com o real. A imagem da teia reforça a
atividade literária, artifício de quem tece o seu texto à espera de
agarrar o que resta do tempo, das coisas e da vida. Ou de quem sabe da
permanência da própria espera, de quem está à espreita, a um passo do
pássaro e de acontecer.


\Image{Orides Fontela (1940-1998) (Arquivo da autora.)}{PNLD0030-05.png}


\section{Referências complementares}

\subsection{Audiovisual}

\begin{enumerate}
\item
  \emph{Orides: a um passo do pássaro} (2000), direção de Ivan Marques;
\item
  \emph{Orides, onde ninguém mais} (2018), direção de David Ribeiro;
\item
  \emph{No intenso agora} (2017), direção de João Moreira Salles;
\item
  Qualquer versão em vídeo de 4'33" (1952), de John Cage.
\end{enumerate}

\subsection{Sonora}

\begin{enumerate}
\item
  ``Sebastião'', canção do álbum \emph{Boca do mundo}, do projeto Música
  Extemporânea Brasileira (MEB), criado e dirigido pelo compositor Zé
  Luiz Rinaldi;
\item
  \emph{Orides Fontela: o grito tornado silêncio} (2008), peça sonora,
  direção de Thiago Eustáquio de Oliveira e Lázara Luzia Fausto Alves.
\end{enumerate}

\subsection{Artes Visuais}

\begin{enumerate}
\item
  \emph{Rochedos em l'Estaque} (1882-1885), de Paul Cézanne;
\item
  \emph{A árvore vermelha} (1908), de Piet Mondrian.
\item
  Série fotográfica ``Sol Negro'' (2017-), de Søren Solkær.
\end{enumerate}

\section{Bibliografia comentada}

\begin{enumerate}
\item
  Bosi, Alfredo. \emph{O ser e o tempo da poesia}. São Paulo: Companhia
  das Letras, 2000. Um dos livros mais importantes da crítica de poesia
  e dos estudos literários em geral. Surgido nos anos 1970, ele consiste
  em seis ensaios: ``Imagem, Discurso'', ``O som no signo'', ``Frase:
  música e silêncio'', ``O encontro dos tempos'', ``Poesia Resistência''
  e ``Leitura de Vico''. Trata-se de um conjunto sofisticado de
  reflexões críticas em torno das categorias do ser e do tempo na poesia
  no processo de construção da linguagem poética.
\item
  Castro, Gustavo de. \emph{O enigma Orides}. São Paulo: Hedra, 2015. É
  a biografia de Orides Fontela, em que somos apresentados aos eventos
  que a tornaram em uma das poetas mais importantes da nossa literatura.
  Além de alguns poemas inéditos, depois incluídos em sua \emph{Poesia
  completa}, publicado pela mesma editora, o livro conta ainda com
  fac-símiles de documentos e manuscritos.
\item
  Lavelle, Patrícia et al. \emph{Poesia e filosofia}: homenagem a Orides
  Fontela. Belo Horizonte: Relicário, 2019. O livro é a reunião de
  ensaios de diversos pesquisadores sobre aspectos da obra de Orides
  Fontela, tendo, como objeto central, a discussão em torno das relações
  entre a poesia e a filosofia. Os textos são resultantes de um simpósio
  dedicado ao tema.
\item
  Paz, Octavio. \emph{O arco e a lira}. Trad. Ari Roitman. São Paulo:
  Cosac Naify, 2014. Um dos ensaios mais importantes da crítica
  literária latino-americana, o livro de Paz explora os diversos
  elementos constitutivos da relação entre a experiência poética e o
  mundo. A escrita fascinante de Paz se alia a uma exposição rigorosa e
  erudita sobre a natureza da poesia como uma das atividades humanas
  mais fundamentais.
\item
  Massi, Augusto (org.). \emph{Artes e ofícios da poesia}. Porto Alegre:
  Artes e Ofícios, 1991. Depoimentos de 29 poetas contemporâneos sobre
  seus respectivos processos criativos. Entre os nomes selecionados,
  estão Adélia Prado, Alice Ruiz, José Paulo Paes, Manoel de Barros e
  Sebastião Uchoa Leite. O depoimento de Orides Fontela incluído no
  livro se intitula ``Na trilha do trevo''.
\item
  Matos, Nathan (org.). \emph{Orides Fontela}: toda palavra é crueldade.
  Belo Horizonte: Moinhos, 2019. O livro é uma coletânea de depoimentos,
  entrevistas e resenhas de Orides Fontela, que nos ajudam na
  compreensão de sua poética e do valor que conferia à palavra em seu
  ofício como poeta. Através desses textos, temos a oportunidade de
  conhecer a formação intelectual de Orides e um pouco dos elementos que
  influenciaram em seu processo criativo.
\end{enumerate}


\end{document}

