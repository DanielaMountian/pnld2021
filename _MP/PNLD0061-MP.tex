\documentclass[12pt]{extarticle}
\usepackage{manualdoprofessor}
\usepackage{fichatecnica}
\usepackage{lipsum,media9,graficos}
\usepackage[justification=raggedright]{caption}
\usepackage[one]{bncc}
\usepackage[glac]{../edlab}

\begin{document}


\newcommand{\AutorLivro}{Claudio Medeiros}
\newcommand{\TituloLivro}{História da experiência das epidemias no Brasil}
\newcommand{\Tema}{Diálogos com a sociologia e com a antropologia}
\newcommand{\Genero}{Narrativa histórica}
\newcommand{\imagemCapa}{./images/PNLD0061-01.png}
\newcommand{\issnppub}{978-65-86598-07-0}
\newcommand{\issnepub}{978-65-86598-08-7}
% \newcommand{\fichacatalografica}{PNLD0061-00.png}
\newcommand{\colaborador}{{Rafael Zacca Fernandes}}


\title{\TituloLivro}
\author{\AutorLivro}
\def\authornotes{\colaborador}

\date{}
\maketitle

\begin{abstract}\addcontentsline{toc}{section}{Carta ao professor}

Este manual apresenta a \emph{História da experiência das epidemias no
Brasil} de modo a oferecer alguns pontos de contato entre o livro e a
prática no processo educativo com os seus educandos e educandas.

Foi o professor Claudio Medeiros, que é também filósofo e poeta, quem
escreveu a obra. Nela, somos apresentados a uma narrativa histórica não
apenas das epidemias no Brasil, como também dos efeitos que os discursos
médicos higienistas tiveram na gestão e na organização do espaço
público, bem como dos processos de subjetivação de nossas populações
(ou, em outras palavras, no que poderíamos chamar de macro e
micropolíticas).

Para isso, o autor escolhe como recorte temporal um momento-chave na
história do Brasil: desde a primeira epidemia de febre amarela no
Brasil, em 1850, até 1888, às portas dos eventos que trariam a
instauração da primeira República, em meio às resistências das
comunidades afrodescendentes ao isolamento e à marginalização. O período
é emblemático também porque tem como pano de fundo a ascensão do
discurso higienista em paralelo com a abolição da escravidão, com os
privilégios das elites garantidos por meio de dispositivos médicos e
legais (o mais famoso deles, a Lei de Terras, que inviabilizou o acesso
dos libertos à terra) que se radicalizariam nas décadas seguintes. A
Corte Imperial foi alvo de catástrofes epidêmicas ao longo do século
\textsc{xix}, e durante elas vimos a transformação das práticas de cuidado dos
corpos e políticas de saúde em direção ao modelo da medicina
positivista.

Também o recorte geográfico é significativo: o Rio de Janeiro, que
serviu como capital do Império, e décadas depois despontaria como
laboratório da \emph{Belle Époque} tropical, com as reformas urbanas de
Pereira Passos e a tentativa de aproximar a aparência da cidade daquela
de Paris, com seus grandes bulevares e avenidas, cafés e praças, e a
tentativa de banimento dos ``marginais'' do perímetro urbano, que
deveriam se higienizar ou sair de cena.

Por tudo isso, você poderá trabalhar com os seus educandos e educandas a
relação entre medicina e história, discurso e subjetividade, território
e poder, urbanização e racismo, e, é claro, poderá trabalhar em suas
atividades a epidemia como fenômeno de \emph{experiência} muito além dos
seus efeitos nocivos sobre os corpos.

Em nossa época, que enfrenta uma das maiores pandemias que temos
notícia, a do \textsc{covid}-19, esta \emph{História da experiência das epidemias
no Brasil} nos oferece a chance de trabalhar com a complexidade dos
atores sociais, das forças institucionais e anti-institucionais que
esses atores convocam, das suas relações no tempo e no espaço, e suas
implicações geográficas, historiográficas, linguísticas e filosóficas no
exercício do saber e da cidadania.

Aqui serão fornecidas uma série de questões, propostas de atividades e
materiais suplementares que permitirão elaborar esses temas de forma ao
mesmo tempo aprofundada e dinâmica. Esperamos que ache este material
útil no seu trabalho em sala e fora dela!

Boa aula!

\end{abstract}

\tableofcontents


\section{Propostas de Atividades I}

\subsection{Pré"-Leitura}

% \paragraph{Habilidades}

% \begin{itemize}
% \item
%   EM13LGG704: Apropriar-se criticamente de processos de pesquisa e busca
%   de informação, por meio de ferramentas e dos novos formatos de
%   produção e distribuição do conhecimento na cultura de rede.
% \item
%   EM13LP10: Analisar o fenômeno da variação linguística, em seus
%   diferentes níveis (variações fonético-fonológica, lexical, sintática,
%   semântica e estilístico-pragmática) e em suas diferentes dimensões
%   (regional, histórica, social, situacional, ocupacional, etária etc.),
%   de forma a ampliar a compreensão sobre a natureza viva e dinâmica da
%   língua e sobre o fenômeno da constituição de variedades linguísticas
%   de prestígio e estigmatizadas, e a fundamentar o respeito às
%   variedades linguísticas e o combate a preconceitos linguísticos.
% \item
%   EM13LP45: Analisar, discutir, produzir e socializar, tendo em vista
%   temas e acontecimentos de interesse local ou global, notícias,
%   fotodenúncias, fotorreportagens, reportagens multimidiáticas,
%   documentários, infográficos,~\emph{podcasts}~noticiosos, artigos de
%   opinião, críticas da mídia,~\emph{vlogs}~de opinião, textos de
%   apresentação e apreciação de produções culturais (resenhas, ensaios
%   etc.) e outros gêneros próprios das formas de expressão das culturas
%   juvenis
%   (\emph{vlogs}~e~\emph{podcasts}~culturais,~\emph{gameplay}~etc.), em
%   várias mídias, vivenciando de forma significativa o papel de repórter,
%   analista, crítico, editorialista ou articulista, leitor, vlogueiro
%   e~\emph{booktuber}, entre outros.
% \end{itemize}

% \includegraphics[width=3.81651in,height=5.99083in]{media/image1.jpg}

\Image{Liberato de Castro Carreira.~Descripção da epidemia da febre amarella
na Provincia do Ceará em 1851 e 1852.~Rio de Janeiro, RJ: Typ. de N.L.
Vianna Junior, 1853. {[}5{]}, ii, 91, {[}2{]} p. Acervo da Biblioteca Nacional.}{PNLD0061-11.jpg}

\paragraph{Tema} Como ler uma epidemia

\paragraph{Conteúdo}

Nessa atividade investigaremos as diferentes formas como
podemos ler uma epidemia e uma pandemia, a partir da pesquisa
etimológica e da imaginação coletiva das tentativas de significação em
torno desse evento que acompanha a nossa história.

\paragraph{Objetivo}

Introduzir os educandos e educandas no tema das epidemias a partir da
relação do adoecimento em massa com a perda de sentido que essa situação
acarreta, ao mesmo tempo em que se produz práticas de atribuição de
sentido a um conjunto de fenômenos em torno de um mesmo significado: o
de epidemia e, eventualmente, o de pandemia.

Além disso, espera-se que essas palavras sejam objeto de análise e
reflexão, propiciando, no processo pedagógico, a capacidade crítica
necessária para a posterior avaliação dos casos particulares e a
argumentação crítica dos educandos e educandas em debates orais ou
dissertativos.

\paragraph{Justificativa}

Do ponto de vista pedagógico, lidar primeiro com a dimensão linguística,
e com um debate mais geral sobre epidemias e pandemias, pode ser uma
entrada ao mesmo tempo mais sutil e significativa para os educandos e
educandas, uma vez que eles poderão, posteriormente, fazer a ligação dos
significados mais gerais aventados nessa etapa do processo pedagógico
com os conteúdos particulares que aprenderão com a história das
epidemias no Brasil.

A análise e investigação das forças, das causas e dos efeitos da
pandemia do \textsc{covid}-19, a partir de 2019, ou da febre amarela e da varíola
no século \textsc{xix} brasileiro, pode envolver uma série de capacidades
cognitivas, entre elas, sem dúvida, está a de se construir uma
legibilidade, uma possibilidade de leitura, de um fenômeno aparentemente
ilegível. Uma vez que epidemias adoecem e ceifam muitas vidas de uma só
vez, é comum que as sociedades passem por grandes crises de sentido
diante desses eventos, e percam a capacidade de articulá-los em
significados. Trabalhar com os educandos e educandas desde os processos
de significação de termos como ``epidemia'', ``pandemia'', ``surto'',
``endemia'', e outros congêneres ajuda a acompanhar esses processos de
perda e ganho de sentido que tais eventos acarretam.

Além disso, a pesquisa dos educandos e educandas e a reflexão acerca
disso pode levar a conclusões de alto valor para o processo pedagógico:
a partir das leituras que se constroem para essas palavras, que
interesses e jogos de força se revelam nessas classificações? Quem
define o sentido de um evento que se alastra ``sobre o povo'' (conforme
a etimologia de ``epidemia'')? Quando a explicação é mágica, quando é
científica, quando é religiosa ou quando é de outra ordem, que política
se pode inferir dessas significações?

Ora, segundo Medeiros, um conjunto de regras que surge num contexto
epidêmico permite ``que se deixe subsistir uma multiplicidade de
discursos sobre procedimentos de prevenção ou terapêutica da epidemia,
cada qual empenhado em sua polêmica singularidade, cada qual investido
nas correlações de força que animam os jogos.'' Essas correlações de
força são construídas junto com os entendimentos de como se deve ler uma
epidemia, com os sentidos que se atribuem a ela.

Por fim, trata-se de uma oportunidade para refletir com educandos e
educandas sobre como todos esses discursos sobre as epidemias jogam com
a imaginação coletiva e com a individual de cada um que tenta construir
uma leitura desses eventos.

\paragraph{Metodologia}

Essa atividade possui três etapas. 

\begin{enumerate}
\item Na primeira delas, a mais breve,
estabeleça um ambiente de perguntas sobre o significado da palavra
``epidemia''. Deve-se estimular que os alunos falem a partir de suas
impressões mais gerais e imediatas. Para isso, você pode tanto partir de
perguntas mais genéricas, sobre o contexto da \textsc{covid}-19, por exemplo,
quanto levar frases que circulam na imprensa ou nas redes sociais, e
pedir que os educandos e educandas expliquem o significado dessa palavra
em sala de aula. Ao fim desse pequeno debate, em que eles podem
concordar ou discordar uns dos outros (as perguntas podem estimular essa
diversidade de ideias), tem início uma investigação.

\item Peça, nessa segunda atividade, que os educandos e educandas procurem
diferentes definições da palavra ``epidemia''. Opcionalmente, você pode
pedir que se dividam em diversos grupos que investigarão diferentes
significados (científico, religioso, mágico, de outras épocas,
ficcionais, provenientes de diferentes instituições, etc.). Mas todos
devem acrescentar às suas pesquisas uma outra, que servirá como base da
atividade: o significado dicionarizado e a origem etimológica de
``epidemia''. A essa investigação, some outras palavras, como
``pandemia'', ``endemia'', ``surto'' e, a depender do seu processo com
eles, inclua também a investigação de palavras como ``crise'',
``vírus'', ``bactéria'', ``infecção'', assim como ``quarentena'',
``isolamento'' e ``cura''.
\BNCC{EM13LGG101}

\item Na terceira etapa, os educandos e educandas devem debater os
significados recolhidos e as eventuais discordâncias e inconsistências
entre um significado e outro para a mesma palavra (uma vez que,
ocasionalmente, eles investigarão fontes distintas). Também é preciso
debater, nesse momento, a relação dos significados entre as diferentes
palavras investigadas. Certas perguntas podem ajudar nesse momento, tais
como: por que a palavra ``epidemia'' se parece com ``pandemia''? Por que
a ideia de ``demos'', de ``povo'', aparece nas duas? Etc. Devem refletir
também de que maneira o sentido dicionarizado e etimológico dessas
palavras conversa com as opiniões que tinham antes, com os discursos que
enxergam na imprensa e nas redes sociais.
\BNCC{EM13LGG102}

Por fim, peça para que eles criem a sua própria definição para essas
palavras, com base na pesquisa que fizeram: esta finalização da etapa
pode ser feita também em grupos, de modo a criar um pequeno dicionário
da turma.
\end{enumerate}

\subsection{Leitura}

% \textbf{Habilidades}

% \begin{itemize}
% \item
%   EM13LGG103: Analisar o funcionamento das linguagens, para interpretar
%   e produzir criticamente discursos em textos de diversas semioses
%   (visuais, verbais, sonoras, gestuais).
% \item
%   EM13LP02: Estabelecer relações entre as partes do texto, tanto na
%   produção como na leitura/escuta, considerando a construção
%   composicional e o estilo do gênero, usando/reconhecendo adequadamente
%   elementos e recursos coesivos diversos que contribuam para a
%   coerência, a continuidade do texto e sua progressão temática, e
%   organizando informações, tendo em vista as condições de produção e as
%   relações lógico-discursivas envolvidas (causa/efeito ou consequência;
%   tese/argumentos; problema/solução; definição/exemplos etc.).
% \item
%   EM13LP48: Identificar assimilações, rupturas e permanências no
%   processo de constituição da literatura brasileira e ao longo de sua
%   trajetória, por meio da leitura e análise de obras fundamentais do
%   cânone ocidental, em especial da literatura portuguesa, para perceber
%   a historicidade de matrizes e procedimentos estéticos.
% \end{itemize}

\paragraph{Tema} As epidemias e o Brasil

\paragraph{Conteúdo}

Nesta atividade leremos três diferentes depoimentos acerca de epidemias
que ocorreram no Brasil, visando, com isso, aproximar os educandos e
educandas da ideia de peste como uma forma de experiência.

\paragraph{Objetivo}

O objetivo desta atividade é mostrar como o entendimento do que é uma
epidemia e do que se pode fazer em uma situação como essa depende da
leitura que produzimos dela. Nesse sentido, a leitura de uma epidemia
como um texto suporta as possibilidades de ação diante dela.

Com isso, objetivamos também desenvolver a capacidade de identificação
de diferentes gêneros textuais em articulação com a produção de sentido
sobre a realidade a que se referem.

Como objetivo secundário e complementar, situa-se a experiência
histórica de leituras da epidemia. Compreender que nem sempre o corpo
foi tomado como um organismo ``invadido'' por outros organismos,
entender que as epidemias já foram apreendidas, por exemplo, em
conjunção com a leitura dos astros ou da atmosfera, em uma era
pré-moderna, ajuda a perceber que a produção de significado em torno
desses eventos atua não somente na sua leitura, como também é
condicionada pela visão de mundo, pelo conceito de vida e pela ideia de
corpo em jogo em cada cultura.

\paragraph{Justificativa}

A vivência das epidemias é uma experiência de leitura. Desde a
classificação de um conjunto de eventos como epidemia, até a reunião de
suas causas e consequências, essa leitura determina as reflexões e as
ações que podemos ter diante de tais eventos catastróficos.

A própria ideia de que a epidemia é um evento de certa forma
meteorológico, como uma nuvem infecciosa que ``está aí'', ``no ar'',
conjuga duas leituras opostas do que ela significa: a medicina de
Hipócrates, por exemplo, determinava que o corpo humano estava em
relações de analogia e semelhança com os astros, com o que acontecia na
atmosfera e no clima acima dos seres humanos; por outro lado, a medicina
pós-hipocrática, da ciência positiva, determinaria o corpo com
organismo, que é invadido, de fora para dentro, por micro-organismos que
causam defeitos no corpo.

Dessa forma, as epidemias contemporâneas tomadas como uma ``nuvem
escura'' que precisamos ``atravessar'' e em certa medida ``combater''
parecem combinar a forma como dois momentos históricos distintos
produziram leituras diferentes das epidemias. Ser capaz de identificar
que uma epidemia é também produto de uma leitura dos eventos, de uma
forma de ler esses eventos, é também dotar os alunos de capacidade
crítica e argumentativa diante das medidas que são tomadas (ou das
medidas que não se tomam) diante de cenários deste tipo.

Educandos e educandas devem ser capazes de compreender as narrativas
acerca das epidemias como discursos, e reconhecer os agentes dos mesmos:
os sanitaristas, médicos, religiosos, científicos, político-partidários
etc., para que possam, em meio a essa massa discursiva, formar seus
próprios juízos, de maneira a exercer a cidadania. Cidadania não apenas
no momento de leitura das epidemias, mas também no momento de reconhecer
diferentes leituras que são feitas delas e as relações de poder que elas
mobilizam. Trata-se do gesto crítico por excelência, a capacidade de
enxergar, como afirma Cláudio Medeiros, os regimes de verdade que
determinam as ações dos corpos numa definida realidade.

\paragraph{Metodologia}

\begin{enumerate}
\item Distribua o primeiro texto referente a esta atividade, um fragmento de
Joaquim Ferreira da Rosa, do seu \emph{Tratado único da constituição
pestilencial de Pernambuco}, de fins do século \textsc{xvii}, citado no livro
de Claudio Medeiros:

\begin{quote}
Tendo nós já dado notícia que o ar se pode influenciar pelos Astros
(quaisquer que sejam) e principalmente pelos eclipses do Sol e da Lua,
podemos entender que não faltaram estas causas: pois no ano de 1685, a
dez de Dezembro (conforme Argolo) houve eclipse da Lua às seis horas
para sete neste hemisfério, estando a Lua na cabeça do Dragão no Signo
de Gêmeos, e o Sol na cauda do Dragão no Signo de Sagitário, e conjunção
com Mercúrio, e oposição com a Lua. Precedeu algum tempo antes outro
eclipse do Sol, a quem um insigne Matemático Padre da Companhia Valentim
Estancel chamava Aranha do Sol; e conforme a calculação, e juízo, que
formou dos movimentos dos Planetas, além de outros infortúnios,
prognosticava doenças. E em um Tratado manuscrito diz nesta forma:
Durarão os efeitos de seus venenos influxos (se a Divina Misericórdia
não se compadecer de suas criaturas) até o ano de 1691. Oxalá não passem
a mais anos nossas calamidades.

(Joaquim Ferreira da Rosa, \emph{Tratado único da constituição
pestilencial de Pernambuco}.)
\end{quote}

O objetivo desta leitura é incitar os alunos a responderem dois tipos de
questão: 1. de que forma os problemas epidêmicos ou de insalubridade são
compreendidos neste texto; 2. que tipo de texto se apresenta nesta
escrita.

Para isto o educador ou a educadora pode dirigir as suas perguntas tanto
ao título do texto apresentado, quanto se referir a características ou
trechos específicos. Este texto é de astrologia ou medicina? Fala de
signos ou de pandemia? Fala dos astros? Descreve o céu? É um texto
ficcional? A partir dessas perguntas, deve-se introduzir os educandos e
educandas à historicidade própria deste texto, à sua vinculação a uma
medicina hipocrática (a este propósito, cf. o material de aprofundamento
no final deste manual) e aos preconceitos culturais que nos levam a crer
que outras épocas ou cultura fazem ``ficção'', por oposição ao suposto
saber mais verdadeiro da ``ciência''.
\BNCC{EM13LP02}

Pode"-se também aproveitar este momento para introduzir os educandos e
educandas, gradativamente, na ideia de que nem sempre se leem as doenças
da mesma forma, e que isso acarreta outra rotina de cuidados com os
corpos, reforçando, dessa maneira, a ligação das questões linguísticas
com diversas outras áreas do conhecimento, como a medicina, a biologia e
a antropologia, por exemplo.

\item Em seguida, saltamos para o século \textsc{xx}, e apresentamos um trecho escrito
pelo dramaturgo Antonin Artaud, também presente no livro:

\begin{quote}
Pode ser que o veneno teatro lançado no corpo social o desagregue, como
diz Santo Agostinho, mas à maneira de uma peste, um flagelo vingador,
uma epidemia salvadora na qual épocas crédulas quiseram ver o dedo de
Deus e que nada mais é senão a aplicação de uma lei da natureza pela
qual todo gesto é compensado por outro gesto e toda ação por uma reação
(...) O teatro, assim como a peste, é uma crise que se resolve pela
morte ou pela cura. E a peste é um mal superior por ser uma crise
completa, não sobrando nada depois dela a não ser a morte ou a
purificação.

(Antonin Artaud, \emph{O teatro e a peste}.)
\end{quote}

Questões semelhantes devem ser direcionadas aos educandos e educandas. A
partir daqui, o educador ou a educadora tem também a oportunidade de
trazer textos mais informativos, do cotidiano dos educandos e educandas
(como as notícias sobre a \textsc{covid}-19, por exemplo), para destacar as
diferenças entre os tipos de texto, aproveitando para pensar: a
finalidade dos gêneros textuais; a forma da imaginação em cada gênero; o
encadeamento lógico ou analógico (metafórico ou por semelhança) das
ideias em cada texto; etc. 
\BNCC{EM13LP49}

\item Por fim, chegamos ao terceiro texto desta
atividade, da própria lavra de Medeiros, e que se refere a um tratado do
século \textsc{xix}:

\begin{quote}
O referido \emph{Tratado de Higiene Geral}, de Adolphe Motard, foi
publicado em 1868. Ele é uma voz entre um conjunto de livros e artigos
científicos dedicados a um tema profícuo, no Brasil, sobretudo a partir
da segunda metade do XIX: a Higiene. Motard nos oferece, além de lições
de profilaxia, anatomia e fisiologia, um amplo arcabouço médico-teórico
dedicado ao clima, à geografia médica e física em geral, à nutrição, às
habitações, aos banhos, roupas e ginástica, às condições higiênicas
atribuídas a diferentes formas de trabalho, à \emph{higiene das
necessidades morais} -- ou seja, às consequências higiênicas boas e más
resultantes das principais instituições sociais, o casamento, o
celibato, a educação, os alienados, a prisão etc. Árdua tarefa de fazer
repassar uma a uma todas as funções humanas para reportá-las, segundo
seu autor, às determinações higiênicas que as modificam.

Assim, a finalidade da higiene geral deve ser a satisfação das
necessidades físicas e morais, na medida de o que melhor convém ao
desenvolvimento individual e social. A Higiene, segundo o Motard, um
código que a ``natureza do homem'' revelou pela decisão do instinto,
chegará enfim à sua etapa filosófica. Ocupando aí um posto entre as
ciências positivas, transcenderá os ramos tradicionais da medicina e,
impulsionado pelo entusiasmo do higienista, consagrar-se-á a si própria
uma Moral.

(Claudio Medeiros, \emph{História da experiência das epidemias no Brasil}.)
\end{quote}

Para este terceiro texto, procure investigar com seus educandos e
educandas o modo como Medeiros lê criticamente a sua fonte -- o
\emph{Tratado de Higiene Geral}, de Adolphe Motard, do século \textsc{xix}, e
como a partir dele podemos ler o conceito de \emph{Higiene} não apenas dentro
da competência médica e dos objetos da ciência biológica e dos cuidados
com a vida, mas também como uma espécie de categoria moral. Um texto que
é capaz de ler outro trabalho textual criticamente é um escrito que
produz uma leitura: é isso que se espera também que os educandos e
educandas consigam fazer com as suas fontes.
\BNCC{EM13CNT306}

Seja como for, tente se lembrar também do conceito de Michel Foucault de
biopolítica: para o filósofo, a partir dos discursos médicos e
sanitaristas, derivados dos discursos positivos da biologia como ciência
da vida, essa ciência passará não apenas a descrever o estatuto de seu
objeto, como passará a determinar, a partir do conceito de vida, também
os modos como se faz política, principalmente nas grandes cidades. A
política, aqui, é compreendida muito além da relação entre as
instituições do Estado e os governados, para ser compreendida como algo
que é tecido pelas diversas instituições, mesmo as aparentemente
neutras, que compõem o corpo social.

Após realizar o debate sobre este terceiro texto, em função do que se
conversou sobre os outros dois anteriores, apresente para os educandos e
educandas gêneros textuais distintos que eles encontram em um livro como
este: os que dizem respeito às fontes históricas (autos de processos,
registros médicos, tratados de medicina antigos etc.); os que dizem
respeito às fontes literárias (poemas, ensaios, romances, contos); os
que dizem respeito às referências teóricas; e assim por diante. Isto
ajudará no processo de compreensão das argumentações mobilizadas na obra
de Medeiros.
\BNCC{EM13LP01}
\end{enumerate}

\subsection{Pós"-Leitura}

% \textbf{Habilidades}

% \begin{itemize}
% \item
%   EM13LGG303: Posicionar-se criticamente diante de diversas visões de
%   mundo presentes nos discursos em diferentes linguagens, levando em
%   conta seus contextos de produção e de circulação.
% \item
%   EM13LGG402: Empregar, nas interações sociais, a variedade e o estilo
%   de língua adequados à situação comunicativa, ao(s) interlocutor(es) e
%   ao gênero do discurso, respeitando os usos das línguas por esse(s)
%   interlocutor(es) e sem preconceito linguístico.
% \item
%   EM13LGG703: Utilizar diferentes linguagens, mídias e ferramentas
%   digitais em processos de produção coletiva, colaborativa e projetos
%   autorais em ambientes digitais.
% \item
%   EM13LP28: Organizar situações de estudo e utilizar procedimentos e
%   estratégias de leitura adequados aos objetivos e à natureza do
%   conhecimento em questão.
% \item
%   EM13LP29: Resumir e resenhar textos, por meio do uso de paráfrases, de
%   marcas do discurso reportado e de citações, para uso em textos de
%   divulgação de estudos e pesquisas.
% \item
%   EM13LP50: Analisar relações intertextuais e interdiscursivas entre
%   obras de diferentes autores e gêneros literários de um mesmo momento
%   histórico e de momentos históricos diversos, explorando os modos como
%   a literatura e as artes em geral se constituem, dialogam e se
%   retroalimentam.
% \item
%   EM13LP52: Analisar obras significativas das literaturas brasileiras e
%   de outros países e povos, em especial a portuguesa, a indígena, a
%   africana e a latino-americana, com base em ferramentas da crítica
%   literária (estrutura da composição, estilo, aspectos discursivos) ou
%   outros critérios relacionados a diferentes matrizes culturais,
%   considerando o contexto de produção (visões de mundo, diálogos com
%   outros textos, inserções em movimentos estéticos e culturais etc.) e o
%   modo como dialogam com o presente.
% \end{itemize}

\paragraph{Tema} Falando sobre a própria experiência

\paragraph{Conteúdo}

Nesta atividade os educandos e educandas são convocados a explorar as
reflexões adquiridas nas atividades de Pré-leitura e de Leitura,
descritas anteriormente, a partir de sua própria escrita da experiência
da epidemia, buscando diferentes argumentos e explicações para o
fenômeno epidêmico em gêneros literários distintos.

\paragraph{Objetivo}

Esta atividade pretende capacitar para a elaboração de sentidos diversos
para experiências epidêmicas a partir do uso da imaginação e da
argumentação lógica. Também o desenvolvimento da capacidade crítica e de
argumentação em torno de situações de calamidade pública e/ou estados de
emergência. Além disso, o fornecimento de recursos linguísticos variados
para o estabelecimento da autonomia de opinião e pensamento do educando
e educanda. E ainda, intenciona-se estabelecer o fomento a debates a
partir de diferentes formas de pensamento e significação, estimulando o
respeito às diversas formações culturais de elaboração de eventos
variados, bem como em consideração às distintas maneiras de saber-fazer
que povoam a nossa sociedade.

\paragraph{Justificativa}

Ler e escrever são reconhecidamente um par indispensável não apenas no
letramento para a interpretação de textos escritos, como também para o
``texto'' contido nos fenômenos do mundo. Fazer com que os educandos e
educandas escrevam sobre aquilo que leem, refletem e debatem é crucial
para que se complete o ciclo do processo de ensino-aprendizagem. No caso
da apreensão dos fenômenos que englobam epidemias e os discursos que as
cercam, experimentar diferentes registros textuais, com variações nas
formas de argumentação e imaginação nos gêneros literários é fundamental
para ampliar o escopo das capacidades críticas de cada um diante dos
acontecimentos sociais.

Dessa maneira, esta atividade se empenha em possibilitar a educandos e
educandas a escrita de diferentes tipos de texto para que aprendam
também a respeitar, nos outros e em si mesmos, as diversas formas com
que a elaboração verbal situa os fenômenos sociais. Isso contribui para
uma cultura mais plural, tolerante e, ainda assim, crítica.

Por fim, a transformação do ambiente da sala de aula em um laboratório
textual possibilita que os educandos e educandas conquistem,
consequentemente, patamares cada vez mais livres e competentes nos
diversos domínios da escrita. Quando são colocados diante dos textos que
seus colegas de classe também fizeram, esse processo é potencializado.

\Image{Marc Ferrez. {[}Laboratório{]}. {[}S.l.: s.n.{]}, {[}1886-1889{]}. 19 x
25cm em c. 31,5 x 47,5. Disponível em:
acervo.bndigital.bn.br.
Acesso em 2 mar.\,2021.}{PNLD0061-02}

\paragraph{Metodologia}

Após realizar as atividades de pré"-leitura e de leitura do livro, chegou
o momento de trabalhar com os educandos e educandas a elaboração do que
refletiram em textos de sua autoria.

\begin{enumerate}
\item Num primeiro momento, apresente a eles materiais diversos sobre a
pandemia do \textsc{covid}-19: matérias de jornais, discursos da comunidade
científica, de diferentes comunidades religiosas, e eventualmente de
alguns dos diversos pensadores e pensadoras que refletiram sobre esta
pandemia. É possível recorrer até mesmo a narrativas literárias, memes e
outras formas de ficção. Essa pluralidade deve englobar um pequeno
debate recuperando a multiplicidade dos diferentes gêneros do discurso e
textuais debatida na atividade de leitura anterior.
\BNCC{EM13LGG101}

\item Agora, com a turma dividida em grupos ou cada educando e educanda
individualmente, peça para que cada um produza um texto sobre a
COVID"-19. Eles devem ser estimulados a testar diferentes formas
textuais: alguns podem arriscar fazer alguma peça literária ou musical,
outros um texto mais argumentativo, outros ainda uma redação
informativa, ou mesmo de explicação mágica ou religiosa da pandemia. O
que é importante, nesta atividade, é a exploração das diferentes
possibilidades verbais acerca da pandemia.
\BNCC{EM13LP53}

\item Por fim, realize um debate com a turma, discutindo os resultados, e que
efeitos são produzidos por cada gênero mobilizado. Tente demonstrar
quais gêneros convocam mais a criatividade, e quais se atêm mais à
lógica. Motive um debate que seja capaz de enxergar a coerência nos
gêneros criativos, e a criatividade do pensamento. Quanto de imaginação
existe na narrativa histórica das diferentes épocas? Quanto de verdade
pode existir em um poema? Quanta criatividade é exigida na nomeação de
doenças? E assim por diante.
\BNCC{EM13LP49}
\end{enumerate}


\section{Propostas de Atividades II}


\subsection{Pré"-Leitura}

% \textbf{Habilidades}

% \begin{itemize}
% \item
%   EM13CHS101: Identificar, analisar e comparar diferentes fontes e
%   narrativas expressas em diversas linguagens, com vistas à compreensão
%   de ideias filosóficas e de processos e eventos históricos,
%   geográficos, políticos, econômicos, sociais, ambientais e culturais.
% \item
%   EM13CHS102: Identificar, analisar e discutir as circunstâncias
%   históricas, geográficas, políticas, econômicas, sociais, ambientais e
%   culturais de matrizes conceituais (etnocentrismo, racismo, evolução,
%   modernidade, cooperativismo/desenvolvimento etc.), avaliando
%   criticamente seu significado histórico e comparando-as a narrativas
%   que contemplem outros agentes e discursos.
% \item
%   EM13CHS104: Analisar objetos e vestígios da cultura material e
%   imaterial de modo a identificar conhecimentos, valores, crenças e
%   práticas que caracterizam a identidade e a diversidade cultural de
%   diferentes sociedades inseridas no tempo e no espaço.
% \item
%   EM13CHS106: Utilizar as linguagens cartográfica, gráfica e
%   iconográfica, diferentes gêneros textuais e tecnologias digitais de
%   informação e comunicação de forma crítica, significativa, reflexiva e
%   ética nas diversas práticas sociais, incluindo as escolares, para se
%   comunicar, acessar e difundir informações, produzir conhecimentos,
%   resolver problemas e exercer protagonismo e autoria na vida pessoal e
%   coletiva.
% \item
%   EM13CHS401: Identificar e analisar as relações entre sujeitos, grupos,
%   classes sociais e sociedades com culturas distintas diante das
%   transformações técnicas, tecnológicas e informacionais e das novas
%   formas de trabalho ao longo do tempo, em diferentes espaços (urbanos e
%   rurais) e contextos.
% \end{itemize}

\paragraph{Tema} Da vivência individual à pública

\paragraph{Conteúdo}

Esta atividade deve trabalhar com a imaginação e o sentimento de
urgência diante do estabelecimento de uma pandemia no corpo social. O
principal nela é a dimensão de complexidade do corpo social (não apenas
do corpo individual) ao ser afetado por uma situação pandêmica.

\paragraph{Objetivo}

Esta atividade tem como objetivo dar a dimensão da complexidade do corpo
social em situações de pandemia: como se conjugam os discursos médico,
legal, científico, religioso, do senso comum, etc. diante de calamidades
públicas. Além disso, visa a estabelecer uma crítica sobre os jogos de
forças sociais, buscando estimular a capacidade de argumentação e tomada
de decisão diante de situações que exigem cooperação da sociedade civil
nos processos de saúde e política pública.

\paragraph{Justificativa}

As situações epidêmicas convocam as correlações de força do corpo
social. Segundo Medeiros, o próprio corpo é uma espécie de encruzilhada
das forças históricas. Seu livro nos apresenta um momento crucial na
história do Brasil, em que o discurso médico-sanitarista passou a
integrar cada vez mais os campos de decisão do poder no planejamento
urbano e nas práticas de cuidado e política pública: o que significa
dizer que a categoria de higiene, ao mesmo tempo em que se tornava uma
categoria moral, revelou um jogo de forças que se estabelecia entre
medicina e ciência positivistas, racismo e processos de marginalização
de diversas classes que passariam a ser consideradas ``perigosas''.

É importante que os educandos e educandas compreendam as diferentes
correlações de força que determinam o planejamento urbano, os complexos
meios com que as decisões de cuidado de si e cuidado dos outros são
mobilizados e as diversas formas de saber-fazer de múltiplos campos
sociais que atuam na governabilidade das cidades diante de calamidades
públicas.

Por isso também são importantes as atividades pedagógicas que estimulem
a imaginação e a compreensão acerca desses complexos de força histórica
e geograficamente determinados. Nesse sentido, as noções de saber,
território, poder, cultura, ciência, mito, misturam-se no intricado
corpo social.

\paragraph{Metodologia}

\begin{enumerate}
\item Em diálogo com as disciplinas de Geografia e História, 
introduza à turma a seguinte narrativa: o ano é 2050, e uma nova pandemia se
manifestou no mundo. Divida a turma em grupos, e informe que eles
tratarão de discutir como o corpo social irá lidar com esta nova
calamidade.

O importante é garantir que diferentes papéis sejam representados por
esses atores sociais: isso ajuda a compreender a complexidade política
de uma cidade, de um Estado, de um agrupamento qualquer.
\BNCC{EM13LGG202}

\begin{itemize}
\item O primeiro grupo é o dos cientistas, que deve nomear a doença, informar
como ela se transmite e identificar como e onde o vírus surgiu.

\item O segundo grupo é o de governantes, que deve oferecer um comunicado
compreensível para toda a população a respeito da doença, determinando
planos de ação para conter seus avanços.

\item O terceiro grupo é o de trabalhadores da saúde (médicos, enfermeiros,
psicólogos, etc.) que lidam diretamente com as consequências da doença.
Eles comunicam-se diretamente com o primeiro grupo, e de alguma maneira
devem decidir como se comportam de acordo com as decisões do segundo.
Esse grupo também é responsável por oferecer uma explicação do que
ocorre com o corpo, a psiquê e com os agrupamentos humanos ao longo da
difusão da doença, e o que pode ser feito no âmbito de um tratamento
mais imediato. O grupo dos governantes, evidentemente, pode ou não ouvir
as suas recomendações.

\item O quarto grupo é o da sociedade civil, que pode ser subdividido em
grupos menores, conforme o tamanho da turma e a disposição dos educandos
e educandas. Com opiniões distintas e conflitantes, a sociedade civil,
ora acolhendo as informações de alguns grupos, ora a de outros, debate
as dificuldades encontradas e os planos de contenção elaborados pelas
diferentes autoridades.
\end{itemize}

\item Uma regra pode servir de organização para garantir a criatividade e
imaginação dos educandos e educandas nesta atividade: evitar, ao máximo,
semelhanças com aquilo que se conhece sobre a pandemia da \textsc{covid}-19 ou da
febre amarela (já que esta é estudada no livro de Medeiros).

\item Determinadas perguntas podem facilitar o processo dos educandos e
educandas: qual é o nome do vírus? O que se deve fazer para contê-lo?
Como garantir a segurança e a saúde social? O que cada cidadão,
individualmente, pode fazer para ajudar na situação pandêmica? A
sociedade civil pode criar suas próprias organizações ou deve
submeter-se às determinações das autoridades?

\item Por fim, peça aos educandos e educandas que registrem, em uma pequena
narrativa ficcional, o que eles imaginam que iria ocorrer nos anos
seguintes à instalação desta pandemia, de acordo com as determinações
dos diferentes grupos. Compare os resultados com a turma.
\BNCC{EM13LP53}
\end{enumerate}

\subsection{Leitura}

% \textbf{Habilidades}

% \begin{itemize}
% \item
%   EM13CNT303: Interpretar textos de divulgação científica que tratem de
%   temáticas das Ciências da Natureza, disponíveis em diferentes mídias,
%   considerando a apresentação dos dados, tanto na forma de textos como
%   em equações, gráficos e/ou tabelas, a consistência dos argumentos e a
%   coerência das conclusões, visando construir estratégias de seleção de
%   fontes confiáveis de informações.
% \item
%   EM13CHS101: Identificar, analisar e comparar diferentes fontes e
%   narrativas expressas em diversas linguagens, com vistas à compreensão
%   de ideias filosóficas e de processos e eventos históricos,
%   geográficos, políticos, econômicos, sociais, ambientais e culturais.
% \item
%   EM13CHS102: Identificar, analisar e discutir as circunstâncias
%   históricas, geográficas, políticas, econômicas, sociais, ambientais e
%   culturais de matrizes conceituais (etnocentrismo, racismo, evolução,
%   modernidade, cooperativismo/desenvolvimento etc.), avaliando
%   criticamente seu significado histórico e comparando-as a narrativas
%   que contemplem outros agentes e discursos.
% \item
%   EM13CHS104: Analisar objetos e vestígios da cultura material e
%   imaterial de modo a identificar conhecimentos, valores, crenças e
%   práticas que caracterizam a identidade e a diversidade cultural de
%   diferentes sociedades inseridas no tempo e no espaço.
% \item
%   EM13CHS106: Utilizar as linguagens cartográfica, gráfica e
%   iconográfica, diferentes gêneros textuais e tecnologias digitais de
%   informação e comunicação de forma crítica, significativa, reflexiva e
%   ética nas diversas práticas sociais, incluindo as escolares, para se
%   comunicar, acessar e difundir informações, produzir conhecimentos,
%   resolver problemas e exercer protagonismo e autoria na vida pessoal e
%   coletiva.
% \end{itemize}

\paragraph{Tema} Os discursos da sociedade em crise

\paragraph{Conteúdo}

Esta atividade combina a leitura do último capítulo do livro, a
propósito das resistências das populações marginalizadas aos processos
autoritários de higienismo urbano -- mobilizados a partir do discurso
médico"-sanitarista -- e a capacidade dos alunos em refletir,
criticamente, sobre as ambiguidades e ambivalências que envolvem as
políticas de saúde pública na modernidade.

\paragraph{Objetivo}

Os educandos e educandas devem desenvolver, aqui, sua capacidade de ler
criticamente as políticas públicas e os discursos que atravessam a
sociedade em momentos epidêmicos. Com isso, devem ser capazes de
reconhecer como certas atitudes aparentemente despropositadas ou
apolíticas carregam em si potencial de resistência a medidas
autoritárias ou ao excesso de intervenção das elites na determinação dos
modos de vida das classes subalternizadas.

\paragraph{Justificativa}

Desde a transformação da medicina hipocrática em medicina positiva --
quando o corpo passou a ser compreendido pelo discurso médico como um
organismo, que pode ser invadido, segundo condições adversas, por um
organismo exterior que o danifica -- e a partir do momento que esta
medicina se fundiu com o discurso sanitarista e higienista, as
recomendações de planejamento do espaço urbano sofreram cada vez mais
influência da instituição médica e das ciências da vida.

Nesse sentido, torna-se difícil compreender o que são práticas de
cuidado que beneficiam de fato todo o corpo social, e o que são
discursos higienistas e sanitaristas que repudiam formas de vida e
organizações sociais consideradas ``imorais'' pela categoria moral de
higiene vigente.

Este entendimento é capaz de tornar legível a Revolta da Vacina, tal
como descrita na célebre obra de José Murilo de Carvalho, \emph{Os
bestializados}, em que o historiador mostra como o ato de resistência à
vacinação forçada no Rio de Janeiro no início da Nova República foi
também uma resposta política aos abusos de autoridade cometidos em nome
da saúde pública.

A complexidade das situações epidêmicas, portanto, exige dos educandos e
educandas que saibam distinguir o que, em cada época, é política pública
e o que é repressão e marginalização promovidas em nome dessa política.
Com isso, é possível aguçar o entendimento crítico e a capacidade de
reconhecimento de táticas de resistência adotadas por populações que, de
outra forma, seriam lidas meramente como ``ignorância'' social.

\paragraph{Metodologia}

\begin{enumerate}
\item Num primeiro momento, com o apoio dos professores das Ciências Humanas, 
solicite a leitura do terceiro capítulo do livro
de Medeiros, ``Os abusos da liberdade e as vésperas do presente''. Este
capítulo explora o caráter artificial que a medicina clínica vai tomando
no fim do século \textsc{xix}, às vésperas da Primeira República do Brasil e
contemporaneamente à abolição, com a racialização do discurso médico,
que passará cada vez mais a condenar uma série de práticas medicinais e
de cuidado de si pertencentes aos povos afrodescendentes, como o
curandeirismo, os erveiros e os benzedeiros.

O capítulo relata como essas formas de cura, que existiam anteriormente
aos discursos higienistas, começam a ser banidas, junto com as formas de
habitação coletiva, dos grandes centros urbanos. A cultura do corpo
higiênico acarreta um processo de adestramento que incide com violência
sobre os corpos recém alforriados e seus descendentes, o que Medeiros
nos mostra ao deflagrar os movimentos de resistência a esse novo estado
de coisas.

\item Como etapa facultativa, recomenda-se também a leitura do posfácio do
livro, que atualiza a discussão a partir de um panorama da pandemia de
\textsc{covid}-19, do isolamento social e do que o autor chama de ``dialética da
quarentena''. Esse curto-circuito histórico possibilitará uma
perspectiva mais ampla sobre as continuidades e rupturas históricas
presentes na história do Brasil, em conjunção com a evolução dos
discursos sanitaristas e os caminhos do racismo no período pós-abolição.

\item Após a leitura, peça aos estudantes que pesquisem e criem um inventário
de filmes e séries de ficção científica a propósito de epidemias ou
pandemias ficcionais, que têm lugar em algum futuro distópico da
humanidade. Outra opção, ainda, é permitir a adesão, ao inventário, de
jogos eletrônicos que se possam acrescentar a essa lista.
\BNCC{EM13LGG604}

\item De posse dos enredos dos objetos culturais inventariados, incentive os
educandos e educandas a identificar práticas de resistência às doenças
tais como aparecem nestes enredos. Eventualmente, cenas podem ser
exibidas em sala de aula, ou mesmo um filme. Como por exemplo o
\emph{Ensaio sobre a cegueira}, filme de Fernando Meirelles adaptado da
obra de José Saramago, que conta a história da epidemia de ``cegueira
branca'' e das práticas de quarentena subjacentes.
\BNCC{EM13LP28}

\item Peça para que os educandos e educandas comparem e diferenciem processos
de movimentação política da sociedade civil no período imperial e nos
filmes: estabelecer semelhanças e distinções é crucial, para que o
debate estimule a percepção crítica e a imaginação política dos
estudantes. Sublinhe, ainda, processos que se repetem na história e na
ficção, e identifique de que maneira eles representam repetições de
erros recorrentes ou inovações capazes de subverter o encadeamento das
histórias.
\BNCC{EM13LP01}
\end{enumerate}


\subsection{Pós"-Leitura}

% \textbf{Habilidades}

% \begin{itemize}
% \item
%   EM13CHS101: Identificar, analisar e comparar diferentes fontes e
%   narrativas expressas em diversas linguagens, com vistas à compreensão
%   de ideias filosóficas e de processos e eventos históricos,
%   geográficos, políticos, econômicos, sociais, ambientais e culturais.
% \item
%   EM13CHS102: Identificar, analisar e discutir as circunstâncias
%   históricas, geográficas, políticas, econômicas, sociais, ambientais e
%   culturais de matrizes conceituais (etnocentrismo, racismo, evolução,
%   modernidade, cooperativismo/desenvolvimento etc.), avaliando
%   criticamente seu significado histórico e comparando-as a narrativas
%   que contemplem outros agentes e discursos.
% \item
%   EM13CHS104: Analisar objetos e vestígios da cultura material e
%   imaterial de modo a identificar conhecimentos, valores, crenças e
%   práticas que caracterizam a identidade e a diversidade cultural de
%   diferentes sociedades inseridas no tempo e no espaço.
% \item
%   EM13CHS106: Utilizar as linguagens cartográfica, gráfica e
%   iconográfica, diferentes gêneros textuais e tecnologias digitais de
%   informação e comunicação de forma crítica, significativa, reflexiva e
%   ética nas diversas práticas sociais, incluindo as escolares, para se
%   comunicar, acessar e difundir informações, produzir conhecimentos,
%   resolver problemas e exercer protagonismo e autoria na vida pessoal e
%   coletiva.
% \end{itemize}

\Image{Rue de Ouvidor\textbf{.~}{[}S.l.: s.n.{]}, c. 1890. 1 foto, gelatina e
prata, p\&b, 36,2 x 30cm. Biblioteca nacional}{PNLD0061-03}

\paragraph{Tema} A relação com o corpo

\paragraph{Conteúdo}

Esta atividade propõe um percurso epidêmico em dois sentidos: tanto uma
caminhada com os educandos e educandas, quanto um itinerário pela ideia
de epidemia na história e na literatura.

\paragraph{Objetivo}

Criar condições para que uma experiência das epidemias seja percebida e
elaborada pelos estudantes. Introduzir os educandos e educandas a uma
forma de pensar que inclui uma outra relação com o corpo (a partir
também da compreensão de como as situações epidêmicas estabelecem
dinâmicas que tiram o corpo de sua zona de conforto). Integrar a
reflexão feita pelos estudantes com o entorno da comunidade escolar,
propiciando uma ocasião para o conhecimento que transcende os muros da
escola e aproxime o saber do lugar mais disputado e tensionado em
situações pandêmicas: a rua.

\paragraph{Justificativa}

O livro de Claudio Medeiros chama-se \emph{História da experiência das
epidemias no Brasil.} Isso significa que há algo mais do que a simples
listagem de situações epidêmicas ao longo dos séculos no país. Há uma
experiência da epidemia, vivenciada pelos corpos que o habitam, mas
também pelo próprio corpo social, bem como pelo corpo das cidades. Não
por acaso o livro de Medeiros constrói sua narrativa ao redor da cidade
do Rio de Janeiro, sede da Corte Imperial e posteriormente laboratório
das reformas urbanas que tiveram o seu auge no início da Primeira
República, sob a égide dos discursos sanitaristas. Dessa maneira, esta
atividade busca convocar não somente o entendimento, mas o próprio corpo
dos estudantes para uma reflexão sobre cidade, liberdade e cidadania no
contexto das epidemias.

\paragraph{Metodologia}

Esta atividade é uma sugestão de caminhada, à semelhança do filósofo
alemão Friedrich Nietzsche, que no século \textsc{xix} recuperou um método
peculiar do pensamento, exercido na antiguidade: em suas caminhadas, o
filósofo se deixava levar pelas divagações a partir da desorientação no
espaço, frequentemente tendo dificuldades no retorno à casa (um chalé no
alto da montanha). Vale lembrar que Nietzsche vivia em um isolamento
radical, envolto em solidão.

O convite para a caminhada visa restabelecer essa relação entre
divagação, desorientação e pensamento. Convide os educandos e educandas
a não recorrerem a seus aparelhos eletrônicos nem a outras distrações.
Seja a caminhada dentro ou fora da escola (de preferência fora), o
importante é que eles não conversem nem se comuniquem.
\BNCC{EM13LGG503}

Eles devem tentar responder a uma pergunta em seu silêncio: o que se
pode fazer com a liberdade? Ou ainda: o que eu faço com a minha
liberdade?

\begin{enumerate}
\item Durante a caminhada, escolha um deles em segredo e comunique a ele a sua
decisão --- posteriormente, ele será revelado para o restante da turma
como alguém que foi, ficticiamente, é claro, infectado por uma doença
altamente letal e de fácil transmissão. Isto é, durante a caminhada, um
jogo ficcional já estará em curso, mas por enquanto apenas o educador ou
a educadora e o estudante escolhido sabem disso.

\item Esse estudante escolhido deve ser encorajado a esbarrar discretamente na
maior quantidade de colegas sem que seja notado ou descoberto em seu
intuito. Após o encerramento da caminhada silenciosa, todos regressam à
sala de aula, e tem início uma conversa com os estudantes sobre suas
experiências. Todos compartilham as suas conclusões, o que sentiram
durante o percurso, o que pensaram sobre a liberdade e sobre si mesmos.

\item No meio da atividade, num momento apropriado, convide o estudante
escolhido secretamente para que ele informe em quem esbarrou durante a
caminhada. Conforme a situação ficcional é relatada, dê início a uma
nova conversa sobre como a resposta à pergunta ``o que eu faço com a
minha liberdade?'' poderia se modificar com o reconhecimento de uma
epidemia fatal, de um simples contágio por uma infecção rápida, entre
eles.

\item A conversa pode ser pontuada, descontraidamente, por eventos relatados
pela literatura ficcional ao redor das doenças, pestes e epidemias em
geral. Lembre"-se, por exemplo, das tragédias antigas de Sófocles,
Eurípedes, Ésquilo, sempre pontuadas por epidemias que assolavam a
população; os poemas homéricos, como a Ilíada, que começa com a ira de
Aquiles, e também com uma epidemia descrita por Homero como uma chuva de
flechas enviada pelo deus Apolo sobre os aqueus. Você pode trazer à
discussão a própria forma como os gregos, nesse primeiro canto, se
organizaram em assembleia, para decidir o que fariam com a epidemia, de
modo tão distinto como essa discussão ocorre na modernidade e em nossas
sociedades.

\item Você pode trazer à tona a possibilidade de uma pesquisa para saber sobre
como guerreiros \emph{vikings} encaravam as doenças e os estados de morbidade
em geral. Segundo as sagas islandesas, ninguém morre de alguma doença,
pois todos lutamos contra a morte em nossos sonhos e morremos ``de morte
matada e não de morte morrida'', como se diz.

\item Outras referências podem ser mobilizadas. A ideia é justamente
demonstrar como diferentes culturas pontuam diversamente o conceito de
liberdade de acordo com o modo como lidam com a morte ou com estágios de
morbidade e doença do corpo.

\item Por fim, é possível ainda reintroduzir os educandos no tema do livro de
Medeiros, o Rio de Janeiro do Brasil Imperial, a partir do relato e
eventual leitura de trechos de \emph{O cortiço} de Aluísio Azevedo e de intersecções 
com outras disciplinas das Ciências Humanas. Esse
último passo pode ser crucial para a compreensão do que está em jogo nos
limites impostos à liberdade das populações recém-alforriadas no início
da Primeira República, uma vez que os cortiços representam esse espaço
de habitação popular comum no Rio de Janeiro, e que passou a ser
transformado em objeto de imoralidade no discurso das elites sustentadas
pelas narrativas médico-sanitaristas.

Uma pequena redação sobre liberdade pode ser requisitada aos
educandos, como finalização da atividade e elaboração final da reflexão
sobre a vida, a morte e o uso dos corpos.
\BNCC{EM13LP53}
\end{enumerate}

\section{Aprofundamento}

% \textbf{Habilidades}

% \begin{itemize}
% \item
%   EM13LGG102: Analisar visões de mundo, conflitos de interesse,
%   preconceitos e ideologias presentes nos discursos veiculados nas
%   diferentes mídias, ampliando suas possibilidades de explicação,
%   interpretação e intervenção crítica da/na realidade.
% \item
%   EM13LGG302: Posicionar-se criticamente diante de diversas visões de
%   mundo presentes nos discursos em diferentes linguagens, levando em
%   conta seus contextos de produção e de circulação.
% \item
%   EM13LGG303: Debater questões polêmicas de relevância social,
%   analisando diferentes argumentos e opiniões, para formular, negociar e
%   sustentar posições, frente à análise de perspectivas distintas.
% \item
%   EM13LP45: Analisar, discutir, produzir e socializar, tendo em vista
%   temas e acontecimentos de interesse local ou global, notícias,
%   fotodenúncias, fotorreportagens, reportagens multimidiáticas,
%   documentários, infográficos,~\emph{podcasts}~noticiosos, artigos de
%   opinião, críticas da mídia,~\emph{vlogs}~de opinião, textos de
%   apresentação e apreciação de produções culturais (resenhas, ensaios
%   etc.) e outros gêneros próprios das formas de expressão das culturas
%   juvenis
%   (\emph{vlogs}~e~\emph{podcasts}~culturais,~\emph{gameplay}~etc.), em
%   várias mídias, vivenciando de forma significativa o papel de repórter,
%   analista, crítico, editorialista ou articulista, leitor, vlogueiro
%   e~\emph{booktuber}, entre outros.
% \item
%   EM13LP48: Identificar assimilações, rupturas e permanências no
%   processo de constituição da literatura brasileira e ao longo de sua
%   trajetória, por meio da leitura e análise de obras fundamentais do
%   cânone ocidental, em especial da literatura portuguesa, para perceber
%   a historicidade de matrizes e procedimentos estéticos.
% \item
%   EM13CHS101: Identificar, analisar e comparar diferentes fontes e
%   narrativas expressas em diversas linguagens, com vistas à compreensão
%   de ideias filosóficas e de processos e eventos históricos,
%   geográficos, políticos, econômicos, sociais, ambientais e culturais.
% \item
%   EM13CHS102: Identificar, analisar e discutir as circunstâncias
%   históricas, geográficas, políticas, econômicas, sociais, ambientais e
%   culturais de matrizes conceituais (etnocentrismo, racismo, evolução,
%   modernidade, cooperativismo/desenvolvimento etc.), avaliando
%   criticamente seu significado histórico e comparando-as a narrativas
%   que contemplem outros agentes e discursos.
% \item
%   EM13CHS104: Analisar objetos e vestígios da cultura material e
%   imaterial de modo a identificar conhecimentos, valores, crenças e
%   práticas que caracterizam a identidade e a diversidade cultural de
%   diferentes sociedades inseridas no tempo e no espaço.
% \end{itemize}

\subsection{O autor e a obra}

\Image{Marc Ferrez, ~{[}Avenida Central --- Vista para o Norte{]}.~Rio de
Janeiro, RJ: {[}s.n.{]}, 15 nov. 1906. 1 reprod. fotom., gelatina, p\&b,
17,5 x 27,5 cm. Acervo da Biblioteca Nacional.}{PNLD0061-07}

Claudio Vinicius Felix Medeiros é professor de Filosofia Geral na
Universidade Federal Fluminense, com atuação nos campos de Ética,
Filosofia Política, Filosofia da História e Filosofia Contemporânea,
tendo escrito artigos sobre os temas da liberdade, do racismo, da
descolonização e da ``biopolítica'' -- conceito extraído da obra do
filósofo francês Michel Foucault, no qual Medeiros apoia grande parte de
seu pensamento. Além disso, Medeiros dedica parte de sua pesquisa ao
pensamento dos abolicionistas afro-brasileiros do século \textsc{xix} (como André
Rebouças, Luiz Gama e José do Patrocínio).

\emph{História da experiência das epidemias no Brasil} é um livro
derivado de sua tese de doutorado, e se destaca especialmente pela
riqueza de descrições e multiplicidade de fontes. Ora, Medeiros é também
escritor e poeta (autor de \emph{Mármore e Barbárie} e \emph{Zumbimalê
Pivete}), e isto dá a seu texto um caráter fronteiriço, entre literatura
e filosofia, narrativa histórica e experiência do pensamento. Para isso
mobiliza fontes literárias modernas e contemporâneas e rigor de pesquisa
arquivística.

Esta obra pode ser caracterizada como uma tentativa de extrair uma
legibilidade das epidemias e das práticas a elas associadas no Brasil --
desde as mais antigas, de que se tem notícia ainda no período colonial,
na Bahia do século \textsc{xv}, até a mais recente pandemia do coronavírus e da
\textsc{covid}-19. Isto é reforçado no posfácio, assinado por Medeiros e Victor
Galdino, dedicado exclusivamente a esta última. Realizado a partir de
uma leitura do período imperial tardio: as quatro últimas décadas do
século \textsc{xix}, às portas da Primeira República (e da Revolta da Vacina de
1904). Também desde o que acontecia na cidade imperial do Rio de
Janeiro, com seus cortiços e outras formas de coabitação que foram
historicamente marcadas, pelos discursos médico e policial, para
desaparecer, junto com o advento dos pobres e dos descendentes de
africanos como ``classes perigosas'' para as elites herdeiras da
colonização. Graças a esse recorte, conseguimos ver esta ampla narrativa
histórica das epidemias no Brasil por meio de uma narrativa cuidadosa e
rigorosa de um espaço-tempo preciso.

Por outro lado, a \emph{História da experiência das epidemias no Brasil}
também pode ser lida como uma história das relações de força que
constituem aquilo que chamamos de ``corpo'' em nosso país. Assim, é
também uma arqueologia, no sentido que o filósofo Michel Foucault dava a
esse conceito, a esse método filosófico-historiográfico, dos
\emph{regimes de verdade} que determinam a ``objetividade'' dos corpos e
a ``subjetividade'' que lhes é subjacente. Dessa forma, a leitura desta
obra possibilita também ao seu leitor uma reflexão mais profunda sobre a
relação entre os modos de se dizer um corpo e os modos de se compreender
o eu.

Nesse sentido, as duas possibilidades de leitura da obra de Medeiros
coincidem em uma reflexão sobre a reestruturação do corpo da cidade
(isto é, do governo e da política pública para o espaço urbano) a partir
de uma caracterização do corpo sob o signo da higiene -- uma categoria
que surge no discurso médico não apenas no âmbito da saúde biológica,
como também da moral. A passagem do século \textsc{xix} para o \textsc{xx} é também,
assim, um momento de consolidação dos discursos marginalizantes que irão
associar às ``classes perigosas'' o foco tanto das epidemias quanto dos
males sociais.

\subsection{Corpo como microcosmo e corpo como organismo}

Segundo Medeiros, ``um corpo é uma encruzilhada histórica de relações de
força.'' Isso significa que em sua obra a ideia de corpo, e a própria
materialidade social (a forma das cidades, por exemplo), é determinada
por relações de forças consolidadas historicamente. Quando se debruça
sobre a experiência das epidemias no Brasil, o autor propõe
conjuntamente a reconstituição dessas relações que determinam a
constituição dos corpos.

É interessante, nesse sentido, acompanhar a transformação da compreensão
médica pré-moderna do corpo como ``microcosmo'' para a do corpo da
ciência positiva como ``organismo''. Nem sempre os corpos foram
compreendidos como uma organização mecânica de órgãos. E, igualmente,
nem sempre a cura, a medicina, foi exercida segundo as mesmas bases.

Nos séculos \textsc{xvii} e \textsc{xviii}, por exemplo, a medicina imperial se fazia com
o domínio do Latim, da \emph{Física} de Aristóteles, do
\emph{Tetrabiblos} de Ptolomeu e dos trabalhos de Avicena, parte do
currículo na Universidade de Coimbra, junto à leitura e comentário dos
textos de Hipócrates e Galeno. Essa continuidade de uma medicina
hipocrática se dava segundo um entendimento do corpo como uma
imagem-semelhança do Universo. Hipócrates postulava que o corpo humano e
sua saúde eram determinados pelo clima, pela atmosfera, pelos astros.
Com a sua teoria das quatro substâncias do corpo, explicava, por
exemplo, a melancolia por um excesso de ``bile negra'', com a passagem
de Saturno sobre o céu terrestre. Daí o entendimento milenar do
melancólico como saturnino, como alguém sob o signo de Saturno.

Essa medicina hipocrática, exercida no início do Brasil Império, não
era, portanto, totalmente inimiga de outras práticas de cura animistas.
Na modernidade europeia, o corpo passa a ser entendido como mecanismo
desencantado (como nas descrições que Hobbes faz do coração como mola e
dos nervos como cordas, ou nas de Descartes, que compara o coração a um
relógio). Esse mecanismo desencantado, que passaremos a entender mais
como organismo que como microcosmo (ou seja, mais como tecnologia do que
como receptáculo da alma ou do espírito) irá compor o dispositivo
sanitarista que tentará banir as práticas de cura e cuidado não apenas
hipocráticas, como também dos erveiros e benzedeiros indígenas e
afrodescendentes. Aos poucos, o surgimento da ideia de higiene como
categoria moral serviria de amparo para a classificação racial e
positivista dos saberes afrodescendentes e ameríndios enquanto saberes
perigosos, pertencentes a classes tidas como anti-higiênicas, e que
deveriam, por extensão, serem expulsas (ou escondidas nas margens) do
espaço urbano.

\subsection{Higiene e reformulação do espaço geográfico da cidade}

Medeiros nos mostra em seu livro como a primeira epidemia de febre
amarela na Corte do Rio de Janeiro, entre 1849 e 1850, desembocou, junto
à importação dos discursos e saberes da ciência médica positiva, em uma
mudança na forma como o espaço público era pensado e organizado. Essa
epidemia causou, em uma população de 266.000 habitantes, 90.658
infectados e fez falecer mais de 4 mil vidas. A partir desse momento, os
higienistas não seriam apenas convocados em casos excepcionais, mas o
seu discurso passaria a integrar as decisões macropolíticas no que se
refere às questões de urbanismo e moradia.

O dispositivo médico-higienista, como o classifica Medeiros, passará,
aos poucos e ao longo das décadas, de um regime combativo a um regime
preventivo-combativo. Proliferará, num primeiro momento, técnicas de
quarentena, de emergência. Em seguida, o discurso médico passará a
reorganizar a repartição urbana dos espaços de isolamento e
quadriculamento da morte.

Por fim, integrará a completa reformulação, com o ponto de culminância
sendo a passagem para o século seguinte, desembocando nas reformas
urbanas de Pereira Passos no Rio de Janeiro, nas demolições e remoções
em massa de cortiços e habitações coletivas, e marginalização da
população afrodescendente no espaço público.

Com tudo isso, a pergunta fundamental que Medeiros tenta responder é:
como a condenação do estado sanatório da cidade resulta, nos discursos
de poder, na ideia de que as habitações coletivas são imorais e
antiestéticas? Ou seja, como o conceito de saúde se torna um dispositivo
do racismo no final e no período posterior à escravidão? Como tudo isso
acompanhou, ao mesmo tempo, a remodelação radical das cidades, com a
condenação dos hábitos ``coloniais'', e sobretudo a normatização
higienista dos corpos?

A resposta a essas perguntas ajuda a compreender a \emph{experiência}
das epidemias na história do Brasil que a obra de Medeiros explora. Essa
experiência compreende ao mesmo tempo a história da continuidade do
racismo na Primeira República e nos dias de hoje, bem como aquilo que
Medeiros classificará como regime de visibilidade das cidades, condução
contra o qual as populações em permanente estado de marginalização
social resistiram e resistem, com seus saberes-fazeres atacados pelas
instituições modernas, importadas para o Brasil no período analisado
pelo autor.

Por fim, é também por isso que Medeiros traz uma outra história a
contrapelo: a história dos fantasmas e espíritos, das ruas, das vielas,
dos cortiços, dos quilombos, das práticas dos corpos improdutivos e dos
curandeiros, que trazem uma outra forma de organizar o espaço, viver a
cidade e a rua, e de cuidar dos corpos.



\section{Sugestões de referências complementares}


\subsection{Filmes}

\begin{itemize}
\item \emph{Epidemia}, 1995. Direção: Wolfgang Petersen, 1995.

Um filme que, de modo fictício, antecipa muitos dos processos e problemas que 
seriam enfrentados pela população mundial em 2020 e 2021 com o alastramento do 
\textsc{covid}-19. Dirigido por Wolfgang Petersen, o longa retrata a saga de um coronel 
e médico do exército norte americano, que também é responsável pelo departamento 
de pesquisas epidemiológicas do país. A narrativa inicia com a investigação 
do coronel sobre uma doença contagiosa nova e logo se expande para uma epidemia 
muito parecida com a doença em uma pequena cidade dos Estados Unidos da América, 
que foi contaminada por conta de um macaco contrabandeado ao local.  

\item \emph{Ensaio sobre a cegueira}. Direção: Fernando Meirelles, 2008.

Para imaginar o contexto de descontrole político e social que uma doença 
viral pode causar, sugerimos assistir o filme \emph{Ensaio sobre a cegueira}, 
que é uma adaptação do romance homônimo do escritor português José Saramago. 
Nele, uma epidemia apelidada de “cegueira branca” atinge de repente uma cidade 
que passa a conviver repentinamente com um crescente estado de exceção. As técnicas 
de quarentena exercidas pelo Estado, logo evoluem para uma situação em que os 
diferentes grupos quarentenados começam a lutar por sobrevivência, alternando 
entre cenas de competitividade e de cooperação. O filme servirá também ao 
debate sobre a questão da ideologia, sobre a forma como nós tendemos 
a reproduzir, durante os momentos de exceção e calamidade, a mesma ideologia com 
que conduzimos as nossas vidas em situações de aparente normalidade.

\item \emph{Pandemia}, 2019.

Uma série documental muito atual, intitulada \emph{Pandemia}, de 2019, produzida 
pouco antes da chegada do \textsc{covid}-19 no mundo. Dividida em seis episódios, 
ela mostra a empreitada de médicos e demais profissionais da saúde ao redor 
do mundo. Por exemplo, no continente africano, trabalhadores da Organização 
Mundial da Saúde lutam para controlar o vírus do ebola. Já em três diferentes 
países do Oriente, alguns especialistas da área da saúde lidam com a 
precariedade para combater as variações da gripe aviária. E nos Estados Unidos, 
uma médica de uma pequena cidade se esforça para conter uma epidemia de influenza. 
No plano de fundo disso tudo, veterinários espalhados pelo mundo fiscalizam 
diferentes animais com a finalidade de prevenir o mundo de uma próxima pandemia.

\item \emph{Zona de separação}, 2020.

Como colaboração na análise do livro \emph{História da experiência das epidemias no Brasil}, 
a série de ficção científica \emph{Zona de separação}, de 2020, criada por Daniel Écija, 
apresenta uma cidade distópica, contextualmente Madri, capital da Espanha, em que uma 
família batalha pela sobrevivência, pois um muro foi criado pelo governo espanhol a 
fim de separar, de um vírus letal, os cidadãos mais pobres dos cidadãs mais ricos, 
criando com isso um abismo social quase irreparável.
\end{itemize}

\SideImage{\textit{Sair de grande noite: Ensaio sobre a África
descolonizada}, por A. Mbembe}{PNLD0061-08}

\section{Bibliografia comentada}

\begin{itemize}
\item \textsc{mbembe}, Achille. \emph{Sair de grande noite: Ensaio sobre a África
descolonizada} {[}2010{]}, trad. Narrativa Traçada (Luanda, Edições
Mulemba, 2014).

Nesta obra, o filósofo camaronês Achille Mbembe
reflete sobre a aparente impossibilidade da comunidade que chamamos de
``humanidade'' a partir de considerações sobre racismo, colonialidade e
poder na modernidade. O ensaio apresenta também questões que se referem
à possibilidade da democracia em nosso tempo.


\item \textsc{azevedo}, Aluísio. \emph{O cortiço} {[}1890{]} (São Paulo, Círculo do
Livro, n/d). 

A história do português João Romão em busca de
enriquecimento e poder acarreta também uma narrativa em que Aluísio
Azevedo descreve uma série de personagens que oscilam permanentemente de
uma habitação coletiva a outra, tendo como pano de fundo as demolições e
os despejos da segunda metade do século \textsc{xix}, e um protagonista com
suspeita de febre amarela, em plena ascensão do discurso higienista.

\item \textsc{simas}, Luiz Antonio, \textsc{rufino} Luiz. \emph{Fogo no mato: a ciência
encantada das macumbas} (Rio de Janeiro, Mórula, 2018). 

Luiz Antonio
Simas e Luiz Rufino interpretam o Brasil a partir de um saber"-fazer
distinto daquele característico da modernidade e das ciências positivas:
o saber"-fazer das macumbas e dos saberes populares, banidos
principalmente do discurso médico higienista que prosperou a partir do
final do Brasil Império.

\item \textsc{foucault}, Michel. \emph{Microfísica do poder}, trad. Roberto Machado (Rio
de Janeiro, Graal, 1979). 

Esta obra de Foucault atualiza o conceito
de poder muito além da sua concepção tradicional, que previa o estudo
das relações de poder exclusivamente pautada no exercício da autoridade
ou da violência por meio de figuras hierarquicamente ``superiores'' na
cadeia social. Com a \emph{Microfísica do poder}, passamos a compreender
como o poder é uma espécie de tecido social composto por um
atravessamento de discursos e práticas que podem ser destrinchados em
uma escala ``micrológica''.

\item \textsc{foucault}, Michel. \emph{Nascimento da biopolítica} {[}1978-79/2004{]},
trad. Eduardo Brandão (São Paulo, Martins Fontes, 2008). 

Em suas
aulas ministradas no Collège de France, e registradas nesta obra,
Foucault analisa os desdobramentos da política e da governamentalidade
liberal a partir da consolidação, no século \textsc{xviii}, de uma razão de
Estado pautada pela organização disciplinar da sociedade, com auxílio
dos discursos policial, médico, escolar, em suma, das instituições
nascentes da modernidade liberal.

\item \textsc{chalhoub}, Sidney. \emph{Cidade febril: cortiços e epidemias na Corte
Imperial} (São Paulo, Cia. das Letras, 1996). 

Neste trabalho, o
historiador Sidney Chalhoub recupera a história da atuação dos médicos
sanitaristas no Rio de Janeiro na segunda metade do século \textsc{xix}, a partir
do seu combate aos cortiços, às epidemias de febre amarela e das
tentativas de solução para o problema da varíola, com a sua vacinação.
Ao mesmo tempo, Chalhoub busca recuperar a história das resistências
populares, em particular dos povos afrodescendentes, ao processo de sua
identificação como ``classes perigosas'' tanto pelo discurso policial
quanto pelo discurso médico.
\end{itemize}

\end{document}


