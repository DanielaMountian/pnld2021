\documentclass[12pt]{extarticle}
\usepackage{manualdoprofessor}
\usepackage{fichatecnica}
\usepackage{lipsum,media9,graficos}
\usepackage[justification=raggedright]{caption}
\usepackage[one]{bncc}
\usepackage[ayllon]{../edlab}
 
\begin{document}

\newcommand{\AutorLivro}{Daniel Defoe}
\newcommand{\TituloLivro}{Robinson Crusoé}
\newcommand{\Tema}{Ficção, mistério e fantasia}
\newcommand{\Genero}{Romance}
\newcommand{\imagemCapa}{./images/PNLD0013-01.png}
\newcommand{\issnppub}{978-65-994304-3-5}
\newcommand{\issnepub}{978-65-994304-2-8}
% \newcommand{\fichacatalografica}{PNLD0013-00.png}
\newcommand{\colaborador}{{Iuri Pereira}}


\title{\TituloLivro}
\author{\AutorLivro}
\def\authornotes{\colaborador}

\date{}
\maketitle

\baselineskip=1.20\baselineskip\par


\begin{abstract}\addcontentsline{toc}{section}{Carta ao professor}

O romance é um gênero moderno. Antes da consolidação do romance, houve
muitas formas de narração em prosa, mas, a partir do século XVII, o
romance vai ganhando importância até se tornar, no século XIX, a forma
literária com mais ressonância social. Se fizermos uma lista com dez
escritores, com sorte lembraremos de um ou dois poetas ou dramaturgos.

Uma das características importantes do romance moderno é o realismo.
Mesmo um romance que contém muitas fantasias engraçadas e maravilhosas,
como \emph{Dom Quixote}, ou até mesmo \emph{Harry Potter}, mantém sua
marca de registro de realidade. Nesse sentido, o tempo verbal narrativo
é o passado, ``era uma vez...'', mas o tempo do romance como gênero é o
presente. O narrador é sempre uma testemunha das formas de pensar e
viver de seu tempo.

\emph{Robinson Crusoé} é um romance fundamental na história do
desenvolvimento desse gênero. Publicado em 1719, na Inglaterra, e
escrito por Daniel Defoe, é um romance de aventuras que hoje em dia
podem parecer fantasiosas. Para uma pessoa que vivesse na Europa dos
séculos XVI e XVII, entretanto, aventuras como aquelas vividas por
Robinson eram relativamente comuns.

Isso aconteceu por causa das grandes navegações, que resultaram na
descoberta da América em 1492 e do Brasil em 1500. Muita gente vinha
conhecer e descobrir as novas terras em busca de riquezas e aventuras
exóticas em viagens marítimas longas e arriscadas.

\emph{Robinson Crusoé} é o romance mais representativo da aventura que
os navegadores e exploradores europeus enfrentaram para colonizar as
Américas. Mesmo não sendo uma história real, embora tenha sido inspirado
por uma história supostamente real, o enredo do romance é uma síntese da
experiência coletiva europeia na exploração do Novo Mundo.

Neste Manual, professores e professoras encontrarão uma ampla informação
contextual e um conjunto de atividades pedagógicas pensadas para
favorecer e ampliar a experiência escolar de leitura.
\end{abstract}

\tableofcontents



\pagebreak\section{Introdução}

Neste Manual queremos oferecer um conjunto de informações e propostas de
atividades para que você, seus alunos e alunas possam enriquecer a
experiência de leitura de um grande clássico moderno: \emph{Robinson
Crusoé}, de Daniel Defoe.

\paragraph{Quem foi Daniel Defoe?}
Daniel Defoe nasceu na Inglaterra em 1660 e viveu até 1731. Começou sua
vida adulta como comerciante, tornando-se mais tarde autor de livros. É
considerado um dos primeiros romancistas modernos e foi também pioneiro
como escritor profissional. Essa profissão se estabeleceu no tempo de
Defoe, apoiada na difusão do jornal diário como novo meio de circulação
de notícias de grande alcance.

\SideImage{Pintura de Daniel Defoe por Godfrey Kneller (National Maritime Museum; Domínio Público)}{PNLD0013-03.png}

\paragraph{A história de Robinson}
\emph{Robinson Crusoé} conta a história de um jovem inglês homônimo,
oriundo da pequena burguesia comercial da cidade de York, onde seu pai
alemão se estabeleceu após deixar sua cidade natal, a nortista Bremen.

Apesar dos conselhos do pai, que o aconselha a se estabelecer no meio
social em que nasceu, ``o mediano, ou o que se podia chamar de casta
superior do povo miúdo'',\footnote{DEFOE, Daniel. \emph{Robinson
  Crusoé}, trad. Bruno Gambarotto. São Paulo: Hedra, 2021, p. 28.}
Robinson deixa-se levar por seu espírito inquieto e aventureiro, e se
alista como tripulante de um navio para expandir os próprios horizontes.

Após uma série de peripécias, que incluem tormentas, encarceramento, um
período reduzido à condição de servo no Marrocos, fuga e resgate por uma
embarcação portuguesa, ele chega a Salvador, na Bahia. Estabelece-se
naquele Estado, torna-se proprietário de engenho, aprende português e
prospera.

Entretanto, por causa de uma proposta comercial que incluía o tráfico de
escravos da Guiné, Crusoé volta a se aventurar, contra sua própria
impressão de que estava dando um passo que poderia arruiná-lo. É nessa
viagem que o marinheiro sofre o naufrágio que resultaria em uma estadia
de 28 anos em uma ilha desabitada, 24 deles em completa solidão.

\Image{Robinson Crusoé parte e seu barco naufraga (Alexander Frank Lydon (A. F. Lydon); Domínio Público)}{PNLD0013-10.png}

Após 24 anos sozinho, Crusoé surpreende um grupo de homens que,
aparentemente, se prepara para sacrificar e devorar um deles. O
habitante solitário consegue atacar o grupo e libertar o prisioneiro.
Era um indígena que não falava inglês e que Crusoé batizou de
Sexta-feira.

A relação de Crusoé com Sexta-feira é uma relação de dominação e
exploração. Ele tenta ensinar o indígena a falar inglês e convertê-lo ao
cristianismo, mas terá pouco sucesso. Chame a atenção para as relações
entre Robinson e Sexta-feira, porque elas imitam as relações de domínio
e posse entre os conquistadores europeus e os povos nativos das
Américas.

Finalmente, ambos seriam resgatados por um navio espanhol e voltariam
para a Europa.



\paragraph{Características do estilo literário}
\emph{Robinson Crusoé} é um romance realista. As aventuras ali contadas
são extraordinárias, mas não têm nada de sobrenatural. Robinson é uma
pessoa comum, diferente dos heróis mitológicos. Ele não tem nada de
especial, sofre muitas dificuldades criadas pelo acaso das situações e
escapa delas com recursos que qualquer pessoa possui. Essas são algumas
características que garantem ao romance de Defoe um lugar especial na
história da literatura.

\paragraph{Desafios para os leitores e leitoras}
A leitura de \emph{Robinson Crusoé} representa um desafio para leitores
e leitoras em formação. Por isso, raramente as pessoas o leram em edição
integral, mas em versões adaptadas, que retiram as passagens mais
monótonas. Robinson passa longos períodos imerso em si mesmo,
conhecendo-se. Sua trajetória é marcada pela ousadia, aventura, coragem
e pelo amor à liberdade, mesmo sem saber direito o que é ser livre.
Esses temas da narrativa a tornam atraente para leitores e leitoras
adolescentes.

\paragraph{As morais da história: uma análise do prefácio de Robinson Crusoé}
Quase todos e todas conhecem a expressão ``moral da história''. Ela faz
parte do gênero das fábulas, histórias curtas, muitas vezes com animais
como personagens, que ilustram uma regra de comportamento, de prudência
ou de moral.

Uma das funções sociais históricas da literatura foi fornecer exemplos
de comportamentos honrosos, corajosos, virtuosos e heroicos para serem
imitados. Há autores e autoras que chegam até mesmo a compreender uma
obra como a \emph{Odisseia}, de Homero, como um tipo de cartilha para
educar pessoas na Grécia Antiga.

Ao longo do século XVIII, houve uma intensa discussão para saber se a
literatura, e o romance em particular, devia representar personagens
imorais ou criminosos. A sociedade temia que os maus comportamentos dos
personagens fossem imitados por leitores e leitoras.

\emph{Robinson Crusoé} contém uma mensagem moral forte, que afirma a
autonomia do indivíduo, portanto, o individualismo. Esse indivíduo é
caracterizado por seu racionalismo, que o faz duvidar de dogmas e
desejar conhecer as coisas por si mesmo.

Para entender as intenções de formação moral presentes no romance, vamos
fazer uma análise de seu curto prefácio. O primeiro parágrafo diz:

\begin{quote}
Se a relação dos trabalhos de qualquer homem particular no mundo é
digna de ser dada ao público, sem quanto a desabone em publicada, o
Editor desta história pensa ser este o caso.
\end{quote}

Vale lembrar que a palavra ``relação'' no prefácio significa
``narração''. ``Relação'' é o resultado da ação de relatar. E o que será narrado? ``Os trabalhos de qualquer homem particular no
mundo''. Isso é uma novidade porque o trabalho não era uma atividade
muito valorizada. Um ``homem particular'' significa uma pessoa comum,
sem nada de distintivo. O trabalho de uma pessoa qualquer é o trabalho
cotidiano, aquilo que fazemos para sobreviver. Em seguida o autor
justifica seu livro qualificando-o de digno e define o seu destinatário,
``o público'', acrescentando que ele foi considerado digno por seu
editor. Tanto o ``público'', no sentido de classe urbana ou nacional que
opina sobre as coisas, quanto o editor, são realidades novas que surge
com o Período Moderno, após o século XVI.

O segundo parágrafo diz: 

\begin{quote}
As maravilhas da vida deste homem excedem
(assim pensa este Editor) tudo quanto exista; sendo raramente capaz que
a vida de um homem agasalhe tão grande variedade.
\end{quote}

O autor faz uma autopropaganda chamando a atenção para o fato de o livro
narrar uma história espantosa. A opinião elogiosa é atribuída ao Editor,
uma figura social que ganha importância com o crescimento do público
leitor e a popularização do jornal de notícias.

O terceiro parágrafo diz:

\begin{quote}
A história é contada com modéstia, com gravidade, e com útil aplicação
religiosa dos acontecimentos àquelas coisas às quais os sábios sempre as
reservam, ou seja, à instrução das gentes pelo exemplo e à confirmação e
honra da sabedoria da Providência em toda a nossa abundância de
circunstâncias, manifestem-se elas como for.
\end{quote}

Agora, o autor diz que seu estilo é modesto e grave, ou, em outras
palavras, humilde e sério. Além disso, não contraria os ensinamentos
religiosos oferecidos pela Igreja e é útil para esses ensinamentos.

Finalmente, o quarto parágrafo diz:

\begin{quote}
O editor crê que se trata de honesta história de fatos; não se
verificando nela qualquer feição de fingimento; e que o bom uso dela,
sendo este objeto de disputa, quer para divertir, quer para instruir, é
um só; e que como tal, sem mais palavras, sua publicação será de grande
serventia ao mundo.
\end{quote}

O autor garante que conta de forma honesta os acontecimentos, sem fingir
ou falsificar apenas para melhorar seu relato. Diz ainda que o bom uso
da obra é dirigido tanto à instrução quanto à diversão. O prefácio faz uma justificativa da obra e procura convencer o
leitor e a leitora de suas boas qualidades para a formação moral,
inclusive religiosa.

É muito importante discutirmos com nossos alunos e alunas o sentido
moral da literatura. O que aprendemos quando lemos a história de um
personagem? Aprendemos como agir para alcançar objetivos e como não
agir. A literatura fornece modelos de comportamentos nos quais nos
espelhamos para ser parecidos e parecidas com personagens que
adquirimos. É assim que funciona a literatura na formação moral de
leitores e leitoras. Desejamos imitar as atitudes de personagens que
admiramos e ter algumas de suas qualidades.

\Image{Robert Paton Gibbs, como Robinson Crusoé, no filme mudo americano "Robinson Crusoé" de 1916. (Henry W. Savage Prod.; Domínio Público)}{PNLD0013-07.png}

\pagebreak\section{Propostas de Atividades I}

%

\subsection{Pré-leitura}

%ATIVIDADE 1:  (EM13LP54)

\paragraph{Tema} Mensagem na garrafa: cartas da minha ilha.

\paragraph{Conteúdo}
Atividade de criação literária baseada no tema da ilha deserta.
Nesta atividade, pretendemos estimular alunos e alunas para a leitura de
\emph{Robinson Crusoé}. Para isso, a ideia é propor uma atividade que
poderia ser disparada pela seguinte proposta: ``Faça em seu caderno uma
lista de dez objetos que você levaria para uma ilha deserta''.
\BNCC{EM13LP54}

\Image{Rota de Robinson Crusoé pelas montanhas dos Pirenéus (Alan Mattingly; CC-BY-SA 4.0)}{PNLD0013-06.png}

\paragraph{Objetivos}
Proporcionar uma atividade de criação e escrita literária baseada em um
tema que estimula a imaginação de todas as pessoas, especialmente dos
habitantes de grandes centros urbanos e que se relaciona diretamente ao
enredo de \emph{Robinson Crusoé}.

\paragraph{Justificativa}
Com essa atividade as turmas poderão:

\begin{enumerate}
\item
Fazer uma atividade oral baseada numa produção individual; 

\item
Exercer a imaginação artística e a escrita literária;

\item
Criar contexto para a leitura do livro de Daniel Defoe.
\end{enumerate}

\paragraph{Metodologia}
Após a criação da lista de objetos, o professor ou professora, faria uma
socialização oral, a fim de perceber que tipo de objetos foram
escolhidos e se eles seriam úteis de fato em uma ilha deserta.

Em seguida, o professor ou professora explicaria a atividade proposta:
escrever um diário de uma semana na ilha deserta. Para isso, seria
importante imaginar com muitos detalhes uma estadia na ilha deserta:
onde dormir, como se proteger, o que comer etc.

Para estimular a imaginação, alunos e alunas podem criar um personagem e
um ilha imaginárias, mas com detalhes de biografia e geografia.

\paragraph{Tempo estimado} Duas aulas de 50 minutos.

\subsection{Leitura I}

\paragraph{Tema} Jornal de Robinson

\paragraph{Conteúdo} Nessa atividade, as turmas serão convidadas a transpor trechos do enredo de \emph{Robinson Crusoé} em notícias de jornal, se valendo do estilo de
reportagem.

\paragraph{Objetivos}
O objetivo dessa atividade é trabalhar com transposição de conteúdo para
outras linguagens, evidenciar o jornal como o grande portador de uma
imagem atual do mundo, fornecer subsídios para uma leitura crítica das
mídias impressas, por meio da desnaturalização de seus discursos.
\BNCC{EM13LGG103}

\paragraph{Justificativa}
As atividades de exploração de mídias contemporâneas e de transposições
de linguagem são fundamentais para desenvolver habilidades de
comunicação e expressão crítica. O jornalismo ainda é uma área
profissional que exerce atração sobre jovens e desse modo a atividade
também corresponde ao desejo de que o Ensino Médio dialogue com as
formas de comunicação do mundo contemporâneo.

\Image{Na legenda da ilustração: ``Robinson Crusoé e Sexta-feira atacando locais'' (Alexander Frank Lydon (A. F. Lydon); Domínio Público)}{PNLD0013-09.png}

\paragraph{Metodologia}
Antes da leitura, o professor ou professora irá dividir o romance em
seguimentos e a turma em grupos atribuindo em seguida para cada grupo um
segmento do enredo.

Após a leitura, cada grupo deverá trabalhar para transformar o segmento
que recebeu em uma ou mais de uma notícia de jornal.

O professor ou professora deve oferecer subsídios sobre os gêneros
jornalísticos e suas marcas textuais: objetividade, impessoalidade,
concisão, precisão.

Há vários subgêneros presentes em um jornal. Uma atividade que seria
útil para a formação cultural geral seria trabalhar com jornais em sala
de aula para investigar que diferentes gêneros textuais o jornal possui.

\paragraph{Tempo estimado} Duas aulas de 50 minutos.

\subsection{Leitura II}

%EM13LGG401; EM13LGG601

\paragraph{Tema} Registro e síntese do enredo em forma de tópicos.

\paragraph{Conteúdo} Atividade de registro e síntese de leitura, em que
o professor ou professora irá estabelecer um trecho do
livro, de preferência do início dele, de cerca de vinte a trinta
páginas, e determinar que alunos e alunas façam uma leitura anotada.
\BNCC{EM13LP28}

\paragraph{Objetivos}
Ensinar e praticar formas de registro, resumo e memorização da leitura
literária, por meio de procedimentos de anotação e formalização de
registros.


\paragraph{Justificativa}
Desenvolver a metacognição, explorar métodos de anotação, aprender a
analisar romances, retirando deles aquilo que constitui o enredo, como
forma de mostrar que o enredo é uma base, mas não basta ter um bom
enredo para escrever um grande romance.

\paragraph{Metodologia}
As anotações devem ser feitas na margem das páginas e cada página deve
ter de três a cinco anotações. O que será anotado? Cenas, locais,
acontecimentos, descrições de personagens.

Com a leitura anotada, cada aluno e aluna deve digitar sua lista. Após a correção, o professor ou professora deve apresentar uma análise
das listas, mostrando quais foram os tópicos indicados em maior ou menor quantidade.

A partir dessa atividade, pretende-se demonstrar que quando duas pessoas leem
um livro, as duas leem as mesmas palavras, a mesma
história, mas cada uma \emph{interpreta} de maneira própria. Isso ocorre
porque a interpretação aproveita e mistura elementos do texto e
elementos da experiência de cada leitor e leitora. Por isso, é importante e oportuno compartilhar na escola as experiências
de leitura.

\subsection{Pós-leitura}

\paragraph{Tema} Em busca de uma história real

%EM13LGG101; EM13LGG102

\paragraph{Conteúdo} Atividade de pesquisa sobre as fontes da narrativa de Robinson Crusoé a partir da questão: \emph{Robinson Crusoé} é uma história verdadeira?

\paragraph{Objetivos}
O objetivo desta atividade é criar condições para uma reflexão
estruturada e contextualizada sobre os limites da realidade e da ficção,
levando alunas e alunos a perceberem que em uma história de ficção há
muitas coisas reais, mesmo que nunca tenham acontecido.

\paragraph{Justificativa}
Essa atividade vai poder mostrar para os leitores e leitoras que a
oposição entre verdadeiro e falso não é suficiente para avaliar a
ficção. Na ficção, muitas vezes um enredo fantasioso contém lições muito
reais, como provam as fábulas e os contos de fada. O lobo de
\emph{Chapeuzinho Vermelho}, por exemplo, é uma forma de mostrar para uma criança que
o mundo esconde perigos e ameaças. \emph{Romeu e Julieta} é uma forma
simbólica de mostrar as dificuldades das relações amorosas.

\SideImage{Fotografia da Escultura de Alexander Selkirk, personagem que inspirou o livro \emph{Robinson Crusoé}, 2004. (Felipe Henríquez González; CC-BY-SA 3.0)}{PNLD0013-05.png}

\paragraph{Metodologia}
Divididos em grupos, alunos e alunas pesquisariam, 
em fontes abertas e confiáveis\footnote{Verifique 
	a seção \ref{sugestoes}, ``Sugestões de referências complementares'', 
	na página \pageref{sugestoes} desse manual.}
para descobrir quais
foram as fontes da história de Daniel Defoe. Essa pesquisa vai encontrar provavelmente alguns artigos e reportagens
que revelam a informação.
\BNCC{EM13LP30} % Fazer pesquisa

O objetivo é que eles e elas se deparem com a história do navegador
escocês Alexander Selkirk, que ficou sozinho em uma ilha por quatro anos
e que teria inspirado o romance \emph{Robinson Crusoé}.

A partir deste resultado da pesquisa e da socialização dos resultados, o
professor ou professora retomaria a questão inicial, mas com uma pequena
diferença: quanto da história de Robinson é real e quanto é inventada? Vale dar o exemplo de \emph{Harry Potter} e perguntar que partes da
história são reais e que partes são fantasiosas.
\BNCC{EM13LGG302}

\paragraph{Tempo estimado} Duas aulas de 50 minutos.

\pagebreak\section{Propostas de Atividades II}

\subsection{Pré-leitura}

\paragraph{Tema} Robinson Crusoé, a Conquista do Novo Mundo e o período das
grandes navegações. Com História e Ciências (Revolução Científica).

%EM13LGG601; EM13LGG602; EM13CNT201; EM13CHS102; EM13CHS104

\paragraph{Conteúdo}
Nesta atividade exploraremos a comparação entre as aventuras narradas no
romance e a história da conquista do Novo Mundo pelos europeus.

\paragraph{Objetivos}
O objetivo é mostrar que, em uma medida muito ampla, o enredo de
\emph{Robinson Crusoé} é fruto das mudanças sociais e culturais e do
contexto de investigação que existiu no século XVI.
\BNCC{EM13LP01}

É como se o enredo fosse uma síntese ficcional de histórias reais, como
a da conquista da América do Norte pelos ingleses e, em geral, da
colonização do Novo Mundo.

\paragraph{Justificativa}
Essa atividade visa valorizar a abordagem interdisciplinar, mostrar que
podemos aprender História através da literatura e que conhecer História
ajuda a atribuir sentido à literatura, associar o enredo ao contexto
social, político e histórico e mostrar que a narrativa, apesar de ser de
ficção, não é uma invenção livre, sem relação com a realidade.

\paragraph{Metodologia}
Como se sabe, no romance de Defoe ocorrem muitas coisas que faziam parte
da vida dos exploradores marítimos do século XVI. Os naufrágios eram tão
frequentes que existe uma antologia de histórias de naufrágio em língua
portuguesa, conhecida como ``História trágico-marítima''.

No início da colonização da América Portuguesa, a viagem entre Lisboa e
Salvador demorava cerca de sessenta dias e era repleta de ameaças, desde
borrascas, tempestades e calmarias, passando pela fome e pelas doenças e
chegando aos ataques de piratas.

Parte da história do período das Grandes Navegações está representada no
romance: a busca de aventureiros europeus em terras distantes, o
naufrágio, o contato entre europeus e povos nativos.

Professores ou professoras de História e de Ciências Naturais, seja
Biologia, Química ou Física, proporiam para a turma, dividida em grupos,
uma pesquisa sobre o século XVI voltada a dois aspectos: a descoberta e
conquista das Américas e as conquistas científicas da chamada Revolução
Científica.

A ideia é pesquisar em grupo fontes eletrônicas e socializar uma síntese
oralmente sobre os principais aspectos da exploração do Novo Mundo e
sobre as principais conquistas da Revolução Científica.

\Image{Mapa xilografado e aquarelado do mundo antigo, 1513 (Arquivo Digital da Biblioteca Nacional; Domínio Público)}{PNLD0013-04.png}


\subsection{Leitura}

\paragraph{Tema} Cartografia de territórios terrestres e marítimos.

%EM13LGG103; EM13LGG601;EM13CHS106; EM13CHS203;EM13CHS204; EM13CHS205; EM13CHS206

\paragraph{Conteúdo}
A ideia desta atividade a ser desenvolvida na área de Geografia é
estudar a cartografia como tecnologia de reconhecimento e registro
territorial e ferramenta de exploração.
\BNCC{EM13CHS103}

\paragraph{Objetivos}

Essa atividade pretende aprofundar a leitura do romance, explorando um
segmento do enredo, as viagens marítimas de Robinson. Para isso, insere
uma tecnologia histórica, a cartografia, cujo desenvolvimento se
relaciona com as Grandes Descobertas. A atividade também pretende
envolver a Geografia na leitura literária, estimulando reflexões
interdisciplinares.

\paragraph{Justificativa}
Promover sequências de aprendizagem nas quais a leitura literária
estimule um mergulho na realidade, seja na história do passado, seja no
mundo contemporâneo. Promover sequências ativas, como o laboratório de
exploração do \emph{Google Earth}, a atividade de criação de mapa e a colagem.
Promover interação entre as disciplinas.
\BNCC{EM13LGG703}

\paragraph{Metodologia}
Inicialmente, pode ser apresentada para a turma a história da cartografia,
diferentes tipos de mapas e alguns exemplos de mapas historicamente
importantes.

Se houver possibilidade, propomos uma atividade de exploração da
ferramenta digital \emph{Google Earth} como forma de mostrar a permanência da
cartografia em ambiente digitais. A proposta geral seria responder à
pergunta: que possibilidade de reconhecimento de territórios a
ferramenta apresenta?

A partir dessa atividade de estímulo, o professor convidaria a turma a
produzir mapas da trajetória de Robinson Crusoé. Cada aluno e aluna
faria um mapa individualmente, com planejamento e execução em sala de
aula.

A finalização da atividade seria colocar o mapa desenhado no verso da capa
ou da quartacapa do livro lido, deixando uma marca de leitura para os
próximos leitores ou leitoras.

\paragraph{Tempo estimado} Duas aulas de 50 minutos.

\subsection{Pós-leitura}

\paragraph{Tema} A ciência do pão (química, física, biologia).

%EM13LGG601; EM13CNT307; EM13CHS301

\paragraph{Conteúdo}
Essa atividade se baseia na abordagem interdisciplinar de um objeto que
pode ser chamado de universal, o pão. No enredo de \emph{Robinson Crusoé}, a panificação mostra-se um grande
desafio, porque exige o cultivo de grãos, a produção de farinha, a
fermentação e o cozimento.

\paragraph{Objetivos} Extrair um tema transversal da narrativa e
levá-lo ao olhar de outras disciplinas, promovendo interação
curricular. Proporcionar uma atividade que combine pesquisa,
leitura, estudos e laboratório.

\paragraph{Justificativa}

Além de ser uma atividade que mostra como a ciência pode ser vista na vida
diária, evidenciando que ela está muito além dos laboratórios, essa
atividade promove a reflexão sobre o universo do consumo, práticas de
sustentabilidade e os impactos da industrialização na saúde e no meio
ambiente.

\paragraph{Metodologia}
Essa atividade pode começar com a seguinte pergunta: quanto tempo
demorou para esse pão que você come de manhã chegar até sua mão? A
maioria vai responder que demorou algumas horas, mas e se tivéssemos que
calcular desde a semeadura do trigo? E se tivéssemos que construir o
forno e as formas?

A partir desta abertura, cada professor podia mostrar como a Biologia, a
Química e a Física veem o pão, isto é: que processos biológicos,
químicos e físicos estão implicados na panificação?
\BNCC{EM13CNT307}

A finalização ideal dessa atividade seria uma oficina escolar de
panificação, organizada pela turma, com um lanche coletivo no final com
pão artesanal escolar!

\pagebreak\section{Aprofundamento}

\subsection{Robinson: senhor de engenho}

Pode ser interessante para o leitor brasileiro o fato de que Robinson
foi senhor de engenho na Bahia do século XVII, a Bahia em que circularam
Gregório de Matos e Antonio Vieira.

Isso porque a passagem relativa ao Brasil, embora breve, é muito bem
embasada a respeito das práticas civis na colônia, como o monopólio do
tráfico de escravos pela Coroa portuguesa, que obstava a expansão do
negócio açucareiro, suscitando a proposta, na narrativa, da viagem à
Guiné.

Ora, é necessário, portanto, considerar a presença indireta de relatos
produzidos na Bahia e que circularam na Europa, às vezes fundidos às
obras sobre a exploração do Novo Mundo.

\subsection{Relatos do Brasil no século XVI}

São muito escassos os relatos seiscentistas impressos, reduzindo"-se a
cerca de duzentas e cinquenta cartas, escritas por jesuítas a serviço na
colônia, e quatro crônicas, a de Gandavo, a de Fernão Cardim, a de
Gabriel Soares de Sousa e a de Jean de Léry, além da própria carta de
Pero Vaz de Caminha, contando ao rei português sobre as novas terras
achadas.

\subsection{Fontes de Daniel Defoe}

Uma das mais prováveis fontes de Defoe seriam os \emph{Tratados da terra
e gente do Brasil}, de Fernão Cardim, que circularam em inglês num
célebre volume de viagens marítimas de Samuel Purchas:

\begin{quote} Samuel Purchas, que adquirira estes manuscritos por bom preço, depois
de os mesmos terem sido confiscados do padre Fernão Cardim, após ter
sido capturado por corsários ingleses e expropriado de seus bens,
considerou-os de grande qualidade e os mais completos que já tinha visto
sobre o Brasil, atribuindo-os a um ``frade ou jesuíta português'', de
quem o corsário inglês Francis Cook, de Dartmouth, se tinha apoderado,
em uma viagem ao Brasil, em 1601, e que os tinha vendido por vinte
xelins a um certo mestre Hackett.''\end{quote}\footnote{AZEVEDO, Ana Maria de.
  ``Introdução'', em CARDIM, Fernão. \emph{Tratados da terra e gente do
  Brasil}. São Paulo: Hedra, 2009, p. 24-25.}

\subsection{Robinson Crusoé e os indígenas do Brasil}

Um episódio notável do enredo de \emph{Robinson Crusoé} é diretamente
calcado em uma prática dos indígenas do Brasil, a construção de canoas
escavadas em um único tronco, conhecidas como \emph{pirocas} ou
\emph{pirogas}:

\begin{quote}
``Ao longo do tempo, isso me levou a pensar se não era possível
fabricar-me uma canoa, ou piroga, à guisa das feitas pelos nativos
daqueles climas, mesmo desprovidos de utensílios adequados; ou, como eu
poderia dizer, sem ajuda, a partir do tronco de uma grande árvore
somente; e isso eu não apenas julguei ser possível, como fácil, e
agradaram-me muitíssimo os pensamentos sobre seu fabrico; e tendo eu
muito mais conveniências para produzi-la do que quaisquer negros e
índios; mas sem considerar os especiais inconvenientes a que estava
submetido, maiores do que os dos índios; a saber, a falta de braços que
a arrastassem à água quando pronta, sendo esse um obstáculo de superação
muito mais difícil para mim do que todas as consequências que decorriam,
a eles, da falta de ferramentas; pois de que me serviria, depois de ter
escolhido uma árvore de grande porte na mata e com muito esforço a ter
cortado, ser capaz de talhar e alisar a parte exterior do tronco, para
que tivesse a forma apropriada de um barco, e de queimar ou cortar o
lado de dentro para torná-lo oco, de maneira a fazer dele um barco; de
que me serviria tudo isso se, ao fim e ao cabo, precisasse deixá-lo onde
o encontrei por não ter força para lançá-lo à água? (\emph{Robinson
Crusoé}, p. 156).
\end{quote}


Essa notícia dos costumes indígenas foi destacada tanto por Fernão
Cardim, quanto por Gabriel Soares de Sousa, em seu \emph{Tratado
descritivo do Brasil em 1587}:

\begin{quote}
Pelo sertão da Bahia se criam umas árvores muito grandes em comprimento
e grossura, a que os índios chamam \emph{ubiragara}, das quais fazem umas
embarcações para pescarem pelo rio e navegarem, de sessenta e setenta
palmos de comprido, que são facílimas de fazer, porque se cortam estas
árvores muito depressa, por não ter dura mais que a casca e o âmago é
muito mole e tanto que dois índios em três dias tiram com suas foices o
miolo todo a estas árvores, e fica a casca só, que lhes serve de canoas,
tapadas as cabeças, em que se embarcam vinte e trinta
pessoas.\end{quote}\footnote{SOUSA, Gabriel Soares de. \emph{Tratado descritivo
  do Brasil em 1587}. Org. Fernanda Trindade Luciano. São Paulo: Hedra,
  2010, p. 213-214.}

A adoção do método de construção nativo, as referências às medidas e à
capacidade da piroga atestam a concorrência dessas fontes, mesmo que de
forma indireta, isto é, por meio de autores que compilaram relatos e os
incluíram em suas antologias, como foi o caso de Purchas com Cardim.

\subsection{Circulação de informações sobre o Brasil}

Sabe-se que o tratado de Gabriel Soares, que só seria impresso no século
XIX, circulou em cópias manuscritas largamente, pois informações que ele
franqueou pela primeira vez, as mais precisas e minudentes dentre os
quatro cronistas, recorrem em outros papéis sobre o Novo Mundo.

Note-se por último a referência ao tempo de construção da piroga nos
dois textos. Enquanto Robinson demora três meses, no texto de Soares
dois índios fariam a mesma carpintaria em três dias. Defoe se vale
possivelmente de uma referência precisa obtida em uma suposta
``pesquisa'' para representar, de forma verossímil, a estadia de seu
personagem na costa brasileira distorcendo-a para amplificar a
dificuldade de Robinson no estabelecimento de seu novo mundo.

\Image{Harry Myers no filme ``As Aventuras de Robinson Crusoé'' (1922) (Universal Film Manufacturing Company/ Internet Archive; Domínio Público)}{PNLD0013-08.png}


\pagebreak\section{Sugestões de referências complementares}\label{sugestoes}


\subsection{Filmes}

\begin{enumerate}

\item
\emph{O náufrago} (EUA, dir. Robert Zemecks, 2001)


O filme conta a história de um homem, interpretado pelo ator Tom Hanks,
que, após sofrer um acidente aéreo, fica perdido em uma ilha deserta.
Muitas cenas do filme parecem diretamente inspiradas no romance inglês,
como se fosse uma adaptação da história para o século XX.

\item
\emph{Na natureza selvagem} (EUA, dir. Sean Penn, 2008)


Christopher McCandless é um jovem recém-formado que decide partir em uma
viagem sem dinheiro e sem destino, em busca de autoconhecimento. Seu
objetivo era chegar ao Alasca e viver sozinho junto à natureza. A
história atualiza um dos mitos presentes na narrativa de Defoe: o desejo
de fugir da cidade em busca de uma vida mais simples e feliz.

\item
\emph{Robinson Crusoé} (EUA, dir. Rod Hardy e George Miller, 1997)

O filme, baseado na obra original, possui atuação de Pierce Brosnan como Robinson Crusoé.


\end{enumerate}

\subsection{Séries}

\begin{enumerate}
\item
\emph{As Aventuras de Robinson Crusoé} (EUA, dir. Robert F. Hill, 1922)

Seriado de 18 capítulos que narra a história de Robinson Crusoé.

\item
\emph{Lost} (EUA, dir. J.J. Abrams, 2004)

Diversas pessoas sofrem um acidente de avião e são forçados a conviver e tentar sobreviver em uma ilha tropical.
\end{enumerate}

\subsection{Artigo}

\begin{enumerate}
\item
\textsc{evers}, Marcos. ``Cientistas pesquisam o verdadeiro Robinson Crusoé''.
\emph{Der Spiegel}. Disponível em \href{https://noticias.uol.com.br/midiaglobal/derspiegel/2009/02/07/ult2682u1076.jhtm}{Uol Notícias}. Consultado em
12/02/2021.


O artigo fala da busca realizada por dois pesquisadores de vestígios do
marinheiro Alexander Selkirk na ilha atualmente chamada Robinson Crusoé,
na costa do Chile.

\end{enumerate}

\subsection{Músicas}

\begin{enumerate}

\item \emph{Into the Wild} (Eddie Vedder, 2007)

Trilha sonora original do filme Na Natureza Selvagem, possui canções de Eddie Vedder, com
letras que exploram o tema da fuga para a natureza e inconformismo com a sociedade. 

\item \emph{Robinson Crusoé} (Bert Appermont, 2008)

Obra musical baseada no livro, foi composta pelo belga Bert Appermont e executada pela orquestra Harmonie.

\end{enumerate}

\pagebreak\section{Bibliografia Comentada}

\begin{itemize}

\item \textsc{defoe}, Daniel. \emph{Robinson Crusoé}. São Paulo: Hedra, 2021.


A edição é apresentada com tradução de Bruno Gambarotto, que também
assina o prefácio, com informações e análises críticas que auxiliam a
leitura e a interpretação da narrativa.


\item \textsc{watt}, Ian. ``Robinson Crusoé'' e ``Crusoé, ideologia e teoria'', em
\emph{Mitos do individualismo moderno}. Rio de Janeiro: Jorge Zahar,
1997, pp. 147-195.


O crítico inglês analisa o romance de Defoe em suas relações com o
individualismo, econômico e religioso. Também analisa a interpretação
que autores como Rousseau e Marx fizeram do romance.


\item \textsc{watt}, Ian. ``Robinson Crusoé, o individualismo e o romance'', em \emph{A
ascensão do romance}. Estudos sobre Defoe, Richardson e Fielding. São
Paulo: Companhia de Bolso, 2010, pp. 63-99.


Neste conhecido estudo, Ian Watt analisa a consolidação do gênero
romance e a formação de um público leitor ``especializado''. Em sua
interpretação, Robinson Crusoé simboliza os processos relacionados com
o advento do individualismo econômico.


\item \textsc{ferreira}, Fernanda Durão. \emph{As fontes portuguesas de Robinson
Crusoé}. s.l.: Fim de Século, 1996.


Nesta breve pesquisa, a autora identifica episódios do livro de Daniel
Defoe que podem ter sido inspirados na leitura da extensa literatura de
viagens produzida em língua portuguesa desde o século \textsc{xv}.
\end{itemize}
\end{document}

