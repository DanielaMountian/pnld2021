teste.tex

ex:399:\Image{Quadro ``Abolição da Escravatura", de Victor Meirelles (1888) (Brasiliana Iconográfica; Domínio Público)}{PNLD0016-06.png}
PNLD0027-MP.tex:68:Já para escrever ``Sobrevivente de Troia", último capítulo, autor colheu relatos de 
PNLD0030-MP.tex:249:\SideImage{Heloísa Buarque de Hollanda é autora de ``Pensamento feminista brasileiro`` e organizadora do livro ''As 29 poetas hoje", lançado em 2021. (Produção Cultural no Brasil; CC-BY-SA 2.0)}{PNLD0030-08.png}
PNLD0030-MP.tex:532:\SideImage{A obra ``Seja marginal, seja herói" de Hélio Oiticica é um marco do movimento de cultura marginal, que passou a fazer parte do debate cultural brasileiro a partir do final de 1968. (Wally Gobetz; CC-BY-NC-ND 2.0)}{PNLD0030-11.png}
PNLD0035-MP.tex:108:mundo moderno, aquela poesia subjetiva onde o ``coração fala", mas \emph{lírica}
PNLD0038-MP.tex:54:pelos seus ``donos" a fim de mantê-lo na propriedade, além de garantir
PNLD0038-MP.tex:56:ocultos, Brown dissimula e foge do ``combinado".
PNLD0039-MP.tex:58:conflitos entre habitantes e o representante dos ``novos tempos". A Igreja 
PNLD0039-MP.tex:63:declara para o povo que é ``cada um por si e Deus por todos", lema do 
PNLD0039-MP.tex:64:egoísmo predador do ``novo mundo". Acabará, no entanto, vítima do que 
PNLD0040-MP.tex:146:\SideImage{Quadro ``Thor luta com os gigantes", pintado por Mårten Eskil Winge em 1872. (Mårten Eskil Winge; Domínio Público)}{PNLD0040-06.png}
PNLD0043-MP.tex:947:mudar com rapidez e de forma imprevisível" em que vivemos, traz consigo
PNLD0054-MP.tex:450:\Image{Quadro ``O Nascimento de Vênus", de Sandro Botticceli representa a deusa do amor, cujo nome grego é Afrodite. (Sandro Botticelli; Domínio Público)}{PNLD0054-09.png}
PNLD0054-MP.tex:506:\SideImage{Estátua de Afrodite exposta na ``Biblioteca Nazionale Marciana", em Veneza (Biblioteca Nazionale Marciana; CC-BY 3.0)}{PNLD0054-10.png}
PNLD0059-MP.tex:596:silêncio, vazio também o ventre do céu".
PNLD0060-MP.tex:865:Série animada do \href{https://www.youtube.com/watch?v=d_4hLpzRh1A}{Canal USP}, parte da pergunta ``de onde viemos?" para estimular crianças e jovens a se interessarem por ciência, arqueologia, antropologia e evolução humana.
	