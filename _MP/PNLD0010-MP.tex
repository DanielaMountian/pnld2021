\documentclass[12pt]{extarticle}
\usepackage{manualdoprofessor}
\usepackage{fichatecnica}
\usepackage{lipsum,media9,graficos}
\usepackage[justification=raggedright]{caption}
\usepackage{bncc}
\usepackage[acorde]{../edlab}


\begin{document}


\newcommand{\AutorLivro}{Charles Baudelaire}
\newcommand{\TituloLivro}{Pequenos poemas em prosa}
\newcommand{\Tema}{Ficção, mistério e fantasia}
\newcommand{\Genero}{Poema}
\newcommand{\imagemCapa}{./images/PNLD0010-01.png}
\newcommand{\issnppub}{---}
\newcommand{\issnepub}{---}
% \newcommand{\fichacatalografica}{PNLD0010-00.png}
\newcommand{\colaborador}{\textbf{Bruno Gradella e Vicente Castro} é uma pessoa incrível e vai fazer um bom serviço.}


\title{\TituloLivro}
\author{\AutorLivro}
\def\authornotes{\colaborador}

\date{}
\maketitle


\begin{abstract}
Este Manual tem como objetivo fornecer subsídios para o trabalho com as
obras literárias \emph{Pequenos poemas em prosa}, de Charles Baudelaire.

Caro professor,

É com prazer que lhe apresentamos este manual a fim de auxiliar seu 
trabalho com os educandos do Ensino Médio. Contamos com você para 
mediar o contato destes jovens com este livro, tão importante na 
história da literatura mundial. 

\emph{Pequenos poemas em prosa}, publicado a primeira vez em 1869, 
é, junto a \emph{Flores do Mal}, a obra mais difundida do poeta 
francês Charles Baudelaire. 
Considerado o grande precursor da poesia moderna, foi ele quem inaugurou 
o olhar sobre as cidades que cresciam com a industrialização, e a vida 
em sociedade que começava a mudar junto. Figuras como o flâneur e o dândi, 
o caminhante aleatório e o excêntrico, aparecem em seus poemas dissolvidos 
na paisagem da Paris do final do século XIX. 

Com a proposta de representar, com beleza, a feiúra da modernidade, 
Baudelaire influenciou não só gerações de poetas franceses como poetas de 
muitas nacionalidades, inclusive brasileiros. É difícil citar um, posterior 
a ele, que não dialogue em algum nível com sua obra. 

Seguindo os parâmetros da BNCC, sugerimos, neste manual, atividades que 
contemplem a interpretação dos poemas fazendo uso do conhecimento de outras 
disciplinas como a História, a Geografia e as Artes, por exemplo, essenciais 
para uma compreensão profunda deste poeta. Esperamos que os educandos do 
Ensino Médio possam construir, com sua ajuda, uma relação próxima com a obra, 
o que lhes permitirá o desenvolvimento de um olhar poético e propositivo sobre 
suas realidades, seguindo o primoroso exemplo do autor francês.

Desejamos, com isso, um prazeroso e produtivo trabalho em sala de aula!


\end{abstract}

\tableofcontents

\section{Atividades 1}


\subsection{Pré"-leitura}

%\BNCC{EM13LGG302}
%\BNCC{EM13LGG704}
%\BNCC{EM13LP10}
%\BNCC{EM13LP19}

Na pré"-leitura,. aproxime os alunos do contexto
histórico"-cultural de Baudelaire, com ênfase na literatura e nas artes
visuais.
Promova um jogo da memória, baseado em poemas franceses do século
dezenove, e estimule posteriormente a gravação em vídeo da declamação
de alguns desses textos.

\subsection{Leitura}

%\BNCC{EM13LGG103}
%\BNCC{EM13LP02}
%\BNCC{EM13LP48}

Durante a leitura, proponha, juntamente com professores de
artes, uma atividade de escrita criativa baseada na deriva poética.

Adotando perfis de caminhantes urbanos, sobretudo do \textit{flaneur}, os
estudantes podem percorrer ruas da cidade.

Por meio da escrita, do desenho, da música e da fotografia, podem,
sobretudo com recursos digitais, mapear criativamente o percurso e
experimentar as vivências de urbanas Baudelaire.

\subsection{Pós"-leitura}

%\BNCC{EM13LGG102}
%\BNCC{EM13LGG303}
%\BNCC{EM13LGG402}
%\BNCC{EM13LGG703}
%\BNCC{EM13LP13}
%\BNCC{EM13LP14}
%\BNCC{EM13LP28}
%\BNCC{EM13LP29}
%\BNCC{EM13LP52}

Depois da leitura, amplie o repertório cultural dos alunos
por meio de um sarau de poesia, com obras do romantismo, do simbolismo
e do decadentismo de diferentes países.


\section{Atividades 2}

As obras de Baudelaire possibilitam trabalhos interdisciplinares e
integradores de diferentes campos do saber e áreas de conhecimento. A
seguir, propomos algumas atividades que podem ser desenvolvidas
conjuntamente com professores de outras áreas. Além das habilidades de
Linguagens e suas Tecnologias e de Língua Portuguesa, indicadas nas
etapas da seção anterior e válidas também para esta, listamos a seguir
as habilidades de outras áreas, presentes na abordagem interdisciplinar:

%\BNCC{EM13CNT201}
%\BNCC{EM13CNT303}
%\BNCC{EM13CHS101}
%\BNCC{EM13CHS102}
%\BNCC{EM13CHS106}
%\BNCC{EM13CHS401}

\subsection{Pré"-leitura}

Na pré"-leitura, professores da área de ciências humanas
podem propor um seminário sobre a história da cidade de Paris, com
ênfase nas mudanças ocorridas no tempo de Baudelaire.

\subsection{Leitura}

Durante a leitura, é possível contextualizar historicamente
os sentimentos de tédio e de melancolia, assim como as figuras do dândi
e do \textit{flaneur}, tipos humanos recorrentes na literatura europeia do
século dezenove.

\subsection{Pós"-leitura}

Após a leitura, sugerimos diálogos, em conjunto com
professores de ciências humanas, com artistas que ele observou: a arte
caricatural de Daumier, a música tempestuosa de Wagner e os textos
febris de Poe, com discussões sobre o contexto histórico"-cultural.

\section{Aprofundamento}

Ao chegar ao Ensino Médio, é necessário que os estudantes se aprofundem
na compreensão das múltiplas linguagens e, sobretudo, da linguagem
literária. Em relação à literatura, a \textsc{bncc} traz as seguintes
considerações:

\begin{quote}
{[}\ldots{}{]} a leitura do texto literário, que ocupa o centro do trabalho
no Ensino Fundamental, deve permanecer nuclear também no Ensino Médio.
Por força de certa simplificação didática, as biografias de autores, as
características de épocas, os resumos e outros gêneros artísticos
substitutivos, como o cinema e as \textsc{hq}s, têm relegado o texto literário a
um plano secundário do ensino. Assim, é importante não só (re)colocá"-lo
como ponto de partida para o trabalho com a literatura, como
intensificar seu convívio com os estudantes. Como linguagem
artisticamente organizada, a literatura enriquece nossa percepção e
nossa visão de mundo. Mediante arranjos especiais das palavras, ela cria
um universo que nos permite aumentar nossa capacidade de ver e sentir.
Nesse sentido, a literatura possibilita uma ampliação da nossa visão do
mundo, ajuda"-nos não só a ver mais, mas a colocar em questão muito do
que estamos vendo/vivenciando. (Brasil, 2018, p. 491)
\end{quote}

Nesta seção, desenvolvemos um trabalho de aprofundamento que, em diálogo
com a formação continuada de professores, oferece subsídios para a
abordagem do texto literário.

Charles Baudelaire, escritor francês, é ainda hoje reverenciado como
um dos paradigmas máximos da criação poética.
Dono de uma imagética original, Baudelaire foi também um influente
crítico de arte e um tradutor de grande envergadura.


\Image{Charles Baudelaire em 1863. (Etienne Carjat; CC BY 2.0)}{PNLD0010-03.png}


De alma inquieta e conturbada, Baudelaire via com desconfiança a era do
progresso, mesmo sendo o precursor da poesia lírica moderna.
Ele via na modernidade uma morbidez oculta que sua sensibilidade não
tolerava.


\Image{Retrato de Baudelaire pintado por Émile Deroy, em 1844 (Émile Deroy; CC-BY-SA-4.0)}{PNLD0010-05.png}
 

Em mil oitocentos e cinquenta e sete, publicou \textit{As flores do mal},
sua obra"-prima, na qual ofende a moral burguesa. Acabou sofrendo um
processo em que foi obrigado a pagar uma multa considerável, além de
suprimir sete poemas do livro.

A obra é tão importante quanto \textit{As flores do mal} na obra de
Baudelaire e está entre as obras de poesia mais importantes da
literatura universal.
Baudelaire ataca convenções e é um verdadeiro artesão da palavra.

Mal compreendida por seus contemporâneos, a poesia de Baudelaire está
marcada pela contradição. de um lado, revela o romantismo de Allan Poe e
Gérard de Nerval, e de outro, o poeta crítico que se opôs aos excessos
sentimentais e retóricos do romantismo francês.

Baudelaire afirmava que a finalidade de sua poesia era ``extrair a
beleza do mal'' e comunicar aos homens a tragédia essencial do ser
humano, dividido entre deus e o demônio.
Segundo o crítico alemão Erich Auerbach, o poeta criou a poesia moderna
ao incorporar à literatura a realidade grotesca. O escritor André Breton
considerava Baudelaire o primeiro dos surrealistas.


\Image{Desenho de Baudelaire feito pelo artista Édouard Manet (Édouard Manet; Domínio Público)}{PNLD0010-06.png}


O livro de Baudelaire perfaz um todo, apesar da aparente dispersão dos
contos.
O conto, gênero breve cuja ação tende diretamente ao final, com uma
unidade dramática, foi admirado por Baudelaire e Edgar Allan Poe.


\Image{O escritor Edgar Allan Poe, grande inspiração de Baudelaire. (Autor desconhecido, fotografia restaurada por Yann Forget e Adam Cuerden ; Domínio Público)}{PNLD0010-08.png}


Já nos poemas, Baudelaire explora recursos de imagem e sonoridade que
evocam o onírico e o fantástico, para além da realidade cotidiana.
Mas o que é o poema em prosa, que praticamente se autonomizou como um
gênero?.

A ``prosa poética'' utiliza um grupo de recursos mais comuns na poesia,
como aliterações, ecos (rimas), um ritmo demarcado mais claramente.
Só que os ``pequenos poemas em prosa'' são antes de mais nada poemas,
escritos utilizando aquela outra forma, a prosa, como veículo.

Baudelaire será seguido nessa forma dos poemas em prosa por dois dos
mais importantes poetas franceses do fim do século: Mallarmé e
Rimbaud.
Além deles, todos os decadentistas e simbolistas derivam muita coisa
das descobertas e da sensibilidade de Baudelaire, assim como o
modernismo internacional.

Além de ser evidentemente, um precursor de todos os grandes poetas
simbolistas, Baudelaire é considerado pela maior parte dos críticos como
marco fundador da poesia moderna. Isso se deve ao fato de que, pela
percepção do real, chegava sempre a um equivalente do
sentimento que desejasse expressar.

Isso acontece, por exemplo, no poema ``correspondências'', de
\emph{As flores do mal}, onde Baudelaire expõe a origem de seu ``projeto simbólico''.
Desta forma, sua poesia tendeu para a expressão de imagens cotidianas, o
``visto pelo autor'', tendo, o poeta, sido quem melhor, em sua época,
intuiu a mudança radical provocada pela
metrópole sobre a
sensibilidade, tratando de tipos sociais urbanos, como o \textit{flaneur} --
caminhante das galerias de grandes cidades -- e o dandi -- o excentrico
admirador de obras de arte e cultivador de luxos e hábitos excêntricos.


\Image{Estátua "O Flaneur": um tipo social retratado pelo autor, que representa o caminhante de galerias de grandes cidades. (PVT Pauline; CC BY-SA 3.0)}{PNLD0010-07.png}


É uma ideia até certo ponto bizarra dizer que esse quase misantropo
fosse um ``homem das multidões'', mas a imersão na massa das cidades
grandes está presente em sua experiência poética.
Quem ressaltou esse aspecto de sua obra foi o pensador Walter Benjamin,
na obra \textit{Charles Baudelaire: um lírico no auge do capitalismo}, em
que descreve a atmosfera de Paris a partir da figura do \textit{flaneur}, um
tipo de caminhante urbano que surge com a modernização das cidades e o
apogeu das galerias, cafés e boulevares parisienses.


\Image{"Carnaval parisiense": a sensação de estar imerso na multidão na cidade é um dos assuntos tratados pelo autor.  (Anônimo; CC BY-SA 3.0)}{PNLD0010-10.png}


O leitor moderno precisa fazer um exercício de imaginação para apreender
o que terá sido alguém desvinculado da preocupação imediata com as
coisas cotidianas e materiais.
O dândi do século dezenove é um personagem excêntrico e um crédito
ilimitado poderia lhe bastar para viver luxuosamente e distante dos
mortais, como um aristocrata.

Baudelaire tinha o sentido desse primado absoluto da estética, do
desejo da perfeição na beleza e no horror, mas sabia também que ``o
estudo do belo é um duelo em que o artista grita de pavor antes de ser
vencido''.
Ao esteta e ao dândi, na figura de Baudelaire, junta"-se o \textit{flaneur},
que é aquele que flana, ou vive a deambular, caminhar sem rumo pela
cidade.

Isso é importante:. o \textit{flâneur} não está indo a lugar algum, ele está de
passagem, e isso se reflete no modo como esses textos devem ser lidos,
pois eles resvalam na ideia da cidade, que figura inúmeras sugestões ao observador atento.
Baudelaire ainda dinamiza aquela incrível e insuspeita força moral sobre
seus objetos de atenção.
São cenas passageiras, na rua ou dentro de um teatro da imaginação, e
ao mesmo tempo exibem eternidade, numa analogia com a recém"-surgida
fotografia, que nascia com o objetivo de recortar e congelar o
momento.

Já ``spleen'' é uma palavra que designa um ``tédio existencial''.
Podemos definir o spleen como uma espécie de melancolia pensante.
Ele se aplica, como vemos, diretamente à vida nas então recentes
metrópoles e, em particular, em Paris.


\Image{A atmosfera de Paris é retratada pelo autor, especialmente no tema da solidão. (Stockholm Transport Museum; Domínio Público)}{PNLD0010-09.png}


E paris, no livro de Baudelaire, é mais que uma cidade. É um estado
mental e um espaço personificado.
``Pequenos poemas em prosa'' ou ``o spleen de paris'', obra póstuma,
publicada em mil oitocentos e sessenta e nove, consumiu mais de dez
anos até sua forma definitiva.
Muitos dos poemas já haviam aparecido em jornais e receberam de pronto a
admiração da crítica e do público.

\section{Referências complementares}

\begin{itemize}
\item\textsc{novaes}, Adauto. \textit{Poetas que pensaram o mundo}. São Paulo:
Companhia das Letras, 2005.

Os poetas"-filósofos que participam desta coletânea tratam não apenas de
poesia, mas da história do pensamento através da poesia. Diante dos
múltiplos caminhos que percorrem este livro, o leitor começa por onde
quiser.

\item\textsc{troyat}, Henri. \textit{Baudelaire}. São Paulo: Editora Pagina Aberta Ltda Scritta, 1995.

A obra de Henri Troyat relata a biografia do famoso poeta francês
Charles Baudelaire, contextualizando sua obra e seu tempo.
\end{itemize}

\section{Bibliografia comentada}

\begin{itemize}
\item\textsc{baudelaire}, Charles. \textit{As flores do mal}. São Paulo:
Penguin"-Companhia das Letras, 2019.

Este volume bilíngue reúne toda a poesia de Baudelaire: \textit{As flores do mal}
tal como publicado em 1861 e os poemas acrescidos à edição póstuma de
1868.

\item\textsc{benjamin}, Walter. \textit{Charles Baudelaire: um lírico no auge do
capitalismo}. São Paulo: Brasiliense, 1989.

Ao decifrar os nexos entre a obra de Baudelaire e as relações sociais
que se afirmavam na Europa Ocidental do final do século passado Benjamin
inova com relação à crítica literária e à análise sociológica
tradicionais.

\item\textsc{bosi}, Alfredo. \textit{O ser e o tempo da poesia}. São Paulo: Companhia
das Letras, 2000.

São ensaios no sentido mais nobre do gênero: jogo criativo da
inteligência a mover"-se, alerta e sensível, no espaço que vai do geral
ao particular; dos parâmetros da essência às formas de sua atualização
histórica; do ser ao tempo da poesia.

\item\textsc{friedrich}, Hugo. \textit{Estrutura da lírica moderna}. São Paulo: Duas
Cidades, 1978.

A anormalidade da lírica moderna diz respeito às composições deslocadas
da realidade e autônomas em seus significados, ou seja, a anormalidade
da lírica moderna prescreve seus efeitos na união do irreal com
elementos logicamente limitados.

\item\textsc{teles}, Gilberto Mendonça. \textit{Vanguarda europeia e modernismo
brasileiro}. Rio de Janeiro: José Olympio, 2012.

Vanguarda europeia \& modernismo brasileiro reúne poemas, manifestos e
conferências vanguardistas publicados entre 1857 e 1972, reunindo tais
textos numa única fonte de consulta.

\item\textsc{starobinski}, Jean. \textit{A melancolia diante do espelho: três leituras de Baudelaire}. São Paulo: Editora 34, 2014.

A melancolia diante do espelho examina com sensibilidade e minúcia três
poemas das Flores do Mal, de Charles Baudelaire. O foco do autor são os
caminhos pelos quais Baudelaire renova o tema da melancolia na arte
ocidental.
\end{itemize}


\end{document}

