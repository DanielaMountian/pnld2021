\documentclass[12pt]{extarticle}
\usepackage{manualdoprofessor}
\usepackage{fichatecnica}
\usepackage{lipsum,media9,graficos}
\usepackage[justification=raggedright]{caption}
\usepackage[one]{bncc}
\usepackage[araucaria]{../edlab}


\begin{document}


\newcommand{\AutorLivro}{Fábio Atui}
\newcommand{\TituloLivro}{Mare nostrum: Paranã Tipi}
\newcommand{\Tema}{Ficção, mistério e fantasia}
\newcommand{\Genero}{Romance}
\newcommand{\imagemCapa}{./images/PNLD0008-01.png}
\newcommand{\issnppub}{978-65-994481-0-2}
\newcommand{\issnepub}{978-65-994481-3-3}
% \newcommand{\fichacatalografica}{PNLD0008-00.png}
\newcommand{\colaborador}{Bruno Gradella e Vicente Castro}


\title{\TituloLivro}
\author{\AutorLivro}
\def\authornotes{\colaborador}

\date{}
\maketitle

\baselineskip=1.20\baselineskip\par


\begin{abstract}\addcontentsline{toc}{section}{Carta ao professor}


O livro \textit{Mare nostrum: Paranã Tipi} conta a história de Theo, jovem
  estudante de medicina, afeito às tecnologias e que recentemente ganha
  dinheiro com investimentos no mercado financeiro.

Theo também é capaz de se valer do \emph{mare nostrum}, uma espécie de poder
  psíquico que permite com que se conecte com outras pessoas, mesmo a
  distância.

Entretanto, esse poder parece gerar o efeito contrário na vida da personagem
  principal, uma vez que se encontra cada vez mais distante de seus amigos
  e colegas, parecendo perder o interesse nas situações cotidianas. Fica claro 
  que Theo vale"-se desse dom de forma frívola, usando"-o
  para ganhar vantagens e habilidades indiferentes ao seu crescimento pessoal.
  Em uma dessas situações, à beira de sua formatura, seu desempenho em uma
  avaliação é colocado em xeque. Seus professores, pressupondo uma ``cola'', ou
  acesso à informação privilegiada, indicam que não se pode tratar a medicina
  de maneira tão fútil. Afinal,  deve"-se ter em mente que o objeto da medicina
  é a vida do ser humano, dedicando-o toda atenção possível. 
  Assim, diante da suspeita de fraude por parte de Theo, 
  seus professores decidem puni"-lo, enviando à selva amazônica,
  esperando, com isso, incutir no aluno o verdadeiro significado de ser médico.

O romance foi escrito por Fábio Atui, que trabalha desde 2004 como
  cirurgião voluntário e coordenador de cirurgia dos Expedicionários da Saúde,
  um grupo de médicos voluntários que levam atendimento especializado a
  populações indígenas em regiões isoladas. 




O autor, inspirado em sua própria história de vida, apresenta a jornada heroica
  da personagem principal, um jovem médico com dons extraordinários,
  confrontado pela necessidade de se responsabilizar por seu poder e usá"-lo
  para um fim digno. Certamente este enredo vai encontrar terreno fértil na
  cabeça dos educandos do ensino médio. 

% Esperamos que as indicações propostas aqui sejam muito úteis no trabalho em
% sala de aula!  

\end{abstract}

\tableofcontents

\hspace{5cm}


\section{Introdução}

Forçado a realizar uma grande jornada rumo à Amazônia, ele é confrontado com
uma realidade muito distante da sua. Em um primeiro momento, sente muita raiva e inconformismo.
Entretanto, numa breve epifania, Theo observa a capacidade que possui de
tornar a vida de outra pessoa menos miserável.
E nessa jornada ao desconhecido, ele acaba se conhecendo melhor.

A história da personagem principal corresponde à narrativa da jornada do herói, em 
que este se constrói saindo de seu ambiente de conforto, encontrando as forças
que deve enfrentar para se realizar e, então, tornar-se digno de retornar, 
refeito, fortalecido e enobrecido.
 
O tema do médico da cidade enviado a uma província distante, ou a um lugar
inóspito é recorrente na literatura.
Tais vivências são narradas sempre como um desabrochar de humanidade, uma
reflexão sobre os pontos importantes da vida e sobre a beleza contida nos
detalhes.
\Image{O autor da obra, Fabio Atui. (Acervo pessoal)}{PNLD0008-03.png}

A obra \emph{Mare nostrum: Paranã Tipi} não fica para trás nesse sentido.
Nela também está presente a riqueza de se desenvolver em solo nacional,
desvelando situações e características muito próprias à nossa realidade.
 
Outro tema comum apresentado no livro é o da viagem de iniciação,
com a narrativa de um jovem que decide
viajar em busca de algum sentido, presente também no gênero cinematográfico 
denominado ``filmes de estrada'' ou \textit{road movies}, em inglês.

Apesar de ficcional, a história traz narrativas reais amalgamadas no fio
condutor do enredo.
Alguns nomes na história são pessoas reais e, mesmo as personagens que
aparecem são, por vezes, um mosaico de diversas outras.  
O próprio protagonista resulta de experiências do autor e de seus colegas. 

A leitura desse livro é um convite à pausa, e a consequente reflexão sobre
nossos limites. Isto é, até que ponto eu vou? Onde começa o que é externo? 
O que está fora de mim?

É apenas depois de uma trajetória áspera que Theo vai percebendo o que
realmente importa, ao mesmo tempo que vai desenvolvendo um trato mais humano
com seus pacientes.

Eles não representam mais números como os ativos da bolsa, mas são verdadeiros
universos que precisam ser cuidadosamente tratados para a manutenção de sua boa
saúde.
 
De início, o autor faz parecer que o \emph{mare nostrum} é, de certo modo, uma alusão
à tecnologia. Entretanto, depois vemos que não, que na verdade é um acesso ao próprio
potencial humano que Theo detinha, mas que até então usava de forma
equivocada. Apenas com o choque de realidade que vivenciou ele torna"-se capaz de compreender
isso.

Assim, o escritor nos abre os olhos, convidando a sermos escafandristas de
nós mesmos, para, de dentro, vermos como realmente estamos tocando nossas
vidas. O próprio autor afirma que: 

\begin{quote}
``Não quero dizer com isso que precisamos ir para
o meio da floresta  para encontrar nosso verdadeiro eu. 
Cada um tem o seu caminho próprio, mas afirmo sem medo de errar que sair de
seu cotidiano, abrir mão do conforto e ter uma perspectiva externa de si mesmo
faz milagres para o crescimento pessoal e para estabelecer as prioridades na
vida. Foi o que aconteceu com o Theo, comigo e com muitos outros expedicionários.''
\end{quote}

\section{Propostas de Atividades I}

\subsection{Pré"-leitura I}

% Vicente 1.1
%\BNCC{EM13LGG704} \BNCC{EM13LP10} \BNCC{EM13LP19}

Comumente se enxerga o médico como aquele que remove a doença e outros males do
paciente, o que não deixa de ser uma imagem verdadeira. Entretanto, na história da literatura,
temos muitas personagens complexas que exercem a profissão da medicina,
oferecendo ao público uma visão menos tecnicista do profissional da medicina.
Isso posto, sugira aos alunos que pesquisem sobre personagens da literatura que
estavam, de algum modo, ligados à medicina, observando as características da
profissão descritas e aspectos de sua personalidade. Por fim, cada aluno
realizará uma leitura dramatizada de trechos da obra que escolheu, em que haja
certo destaque a essa personagem.
\BNCC{EM13LP16} % leitura dramatizada 

\subsection{Pré-leitura II}
% Atividades para o aluno não cair de paraquedas Uma tarefa de casa ou vinte
% minutos de aula
Abra uma roda na sala de aula e siga, integralmente ou não, o seguinte roteiro de discussão:

\begin{itemize}
\item Levantar conhecimentos prévios dos alunos acerca do gênero ficção
científica.

\item Abordar a noção de metaciência, i.e., a área do conhecimento que aborda o
funcionamento da própria ciência, a construção de suas explicações e verdades.

\item Reunir conhecimentos preexistentes dos alunos a respeito dos sonhos.

\item Orientar os alunos a pesquisar o que é o inconsciente e que teorias
originaram essa nomenclatura.

\item Orientar os alunos a pesquisar por que sonhamos de acordo com a ciência e de
acordo com teorias místicas ou espiritualistas.

\item Contextualizar a ideia de ciclos de sono e de REM (\textit{Rapid Eyes Movement}).

\Image{``O sonho da razão produz monstros'', de Francisco Goya, 1796. (Wikimedia Commons; CC BY 4.0)}{PNLD0008-20.jpg}

\end{itemize}
\BNCC{EM13LP28} % orientar os alunos a pesquisar

\subsection{Leitura}

% Vicente 1.2
% \BNCC{EM13LGG103} \BNCC{EM13LP02} \BNCC{EM13LP48}

Retome o seguinte trecho do livro em que o personagem principal, Theo, se
prepara para sua viagem e recebe conselhos sobre como deve aprender.  

\begin{quote} ``Sua navegação pode não funcionar bem na Amazônia, Theo.''

O monte Roraima estava com uma névoa branca tão espessa que os impedia de ver
  as árvores e os inúmeros rios que cortavam a paisagem abaixo, e que garantiam
  um espetáculo à parte nos dias claros. Theo se cansou daquela cena e fez um
  movimento com a mão direita, limpando as nuvens, franqueando para ambos a sua
  vista preferida. Ele conhecera aquele planalto em uma reportagem, e ficou
  encantado com seus quase três mil metros de altura e a estonteante paisagem
  em seu entorno. Desde então, aquele se tornou o lugar oficial de seu
  treinamento com Alcântara.

``Mestre, não se preocupe, eu já sou um navegador experiente. Posso superar
  qualquer dificuldade de navegação que aparecer, até mesmo na Amazônia.''

``Não estamos falando da sua capacidade, Theo, mas de como as coisas funcionam
  no \emph{mare nostrum}. Para navegar e ter acesso a essa grande biblioteca de
  conhecimento da humanidade, precisamos de uma conexão entre as mentes. Cada
  mente é como um servidor que vai nos interligando, sucessivamente, uns aos
  outros. Como a Amazônia é muito esparsamente povoada, a conexão ocorre em
  bolsões, por isso você vai ter dificuldades de navegar de maneira ampla e
  talvez se restrinja apenas a um grupo geograficamente delimitado.''

``Mas como é possível sobreviver sem o \emph{mare nostrum}? Sabemos que ele é
  importantíssimo para a sanidade mental das pessoas.''

``Fica tranquilo, você vai entender uma porção de coisas assim que estiver
  verdadeiramente conectado com aquele mundo. Você está muito dependente do
  \emph{mare nostrum}, Theo, um choque de realidade vai te fazer muito bem.''

\end{quote}

Discuta com os alunos de que tipo de conhecimento o personagem está falando e o
que seria o ``mare nostrum''. Em seguida aborde algumas dessas questões
sugeridas:
\BNCC{EM13LP02} % Discuta com os alunos de que tipo de conhecimento: Parte do discurso

\begin{itemize} 
\item Que temas são levantados pelo livro por meio da navegação
    onírica e das investigações do protagonista?

\item Quem é o protagonista? Por que ele decidiu estudar medicina?

\item Quem são as outras personagens? Qual a importância delas para o desenvolvimento da trama?

\item Theo entende o \emph{mare nostrum} como uma espécie de ``internet do Cosmos''. O que você entende como sendo o \emph{mare nostrum}? 

\item Como a navegação avançada no \emph{mare nostrum} permitiria o acesso à mente de outras pessoas? Quais as consequências disso?

\item Qual é o espaço"-tempo da trama?

\item Qual é o momento de maior suspense do enredo?

\item Quais consequências são pressupostas pelo clímax do romance? Quais os perigos?

 \end{itemize}

Então, sugere"-se que os alunos busquem mitos e histórias indígenas brasileiras. Selecionada a história, cada aluno deverá adaptá"-la para o formato de texto literário. É interessante também que, para complementar o trabalho, os alunos façam um prefácio, abordando características, histórias, costumes e idioma do povo indígena que produziu o conto a transcrito pelo
estudante.

\subsection{Pós"-leitura}

% Vicente 1.3
%\BNCC{EM13LGG102} \BNCC{EM13LGG303} \BNCC{EM13LGG402} 
%\BNCCC{EM13LGG703; EM13LP13; EM13LP14; EM13LP28; EM13LP29; EM13LP52.}

A obra apresenta o relato de viagem como rito de passagem entre dois mundos e a descoberta de um jovem que percebe a grandeza e a diferença entre eles. O livro pode ser assim apresentado como um \textit{road movie}, uma história comum em que o jovem decide partir em uma viagem iniciática para conhecer e se reconhecer.

Traga para os estudantes a passagem conhecida como ``Telemaquia'', da Odisseia de Homero. Trata"-se da viagem que Telêmaco fez em busca de seu pai, Ulisses. Essa história pode ser um ponto de partida para que a reflexão acerca de quantos filmes comuns sobre viagens de jovens eles já assistiram.

\SideImage{Folha de rosto da primeira edição de <<As aventuras de Telêmaco>>, por François Fénelon (1699) (Wikimedia Commons; Domínio Público)}{PNLD0008-21.jpg}

% Sugira aos alunos 
% que peguem a passagem do texto desenvolvida
% na atividade anterior ou aquela em que eles identificarem como um
% momento crucial da obra e peça para eles transformarem em uma tira de quadrinhos de três quadros somente. Convide os alunos que não queiram desenhar a escrever o trecho como se fossem roteiristas. O ideal é que a cena escolhida seja recontada de memória. 

Peça para eles pesquisarem sobre ``ritos de passagem''
na internet ou na biblioteca, com uma atenção especial para informações sobre jovens indígenas. Discuta com eles em seguida sobre a noção de viagem iniciática. O objetivo aqui é fazê"-los perceber a noção de iniciação no mundo e a relação desse momento da vida com os lugares da literatura como 
a Telemaquia, mencionada acima, ou os roteiros de filmes comerciais
que apresentam jovens em viagens cruciais. Associe a viagem de Telêmaco, filho de Ulisses, 
e outros tipos de ritos de passagens indígenas com as ``aventuras'' 
e os riscos aos quais os jovens costumam se expor durante um período da vida.
\BNCC{EM13CNT207} % pesquisarem sobre ``ritos de passagem na internet ou na biblioteca % vida do jovem:: desafios

\section{Propostas de Atividades II}

A obra \emph{Mare nostrum: Paranã Tipi} possibilita trabalhos
interdisciplinares e integradores de diferentes campos do saber e áreas de conhecimento. A seguir, propomos algumas atividades que podem ser desenvolvidas conjuntamente com professores de outras áreas.

% Vicente 2.1
% \BNCC{EM13LGG302}
% \BNCC{EM13CNT201}
% \BNCC{EM13CNT303}
% \BNCCC{EM13LP45; EM13LP48; EM13CHS101; EM13CHS102; EM13CHS106; EM13CHS401.}

\subsection{Pré"-leitura}

Na pré"-leitura, convém um mergulho na questão das culturas e
saberes tradicionais dos povos indígenas. Para isso, convide professores da área de ciências humanas para essa atividade. É interessante observar como a cultura indígena foi abordada pelos não"-indígenas ao longo da história. É sabido que durante muito tempo, as populações indígenas foram vistas como primitivas. Essa visão, felizmente, tem mudado. Procure, 
nos \textit{sites} principais jornais brasileiros 
matérias sobre povos indígenas nos últimos trinta anos e discuta como
essa percepção mudou.

\Image{Marcha das Mulheres Indígenas, Brasília, 2019 (\textsc{apib}; Domínio Público)}{PNLD0008-25.jpg}

\BNCC{EM13LGG102} % tradicionais dos povos indígenas.

\subsection{Leitura}

Durante a leitura, é possível que os alunos adentrem no
universo da pesquisa acadêmica médica, por meio da busca de notícia sobre profissionais que realizam estudos inovadores, ofertando um melhor
tratamento aos pacientes do mundo. Assim, conjuntamente com o auxílio de
professores de ciências da natureza para o auxílio com a terminologia,
os estudantes podem buscar informações a respeito de uma descoberta
científica no campo da medicina e dissertar sobre ela e seus impactos. Também é interessante que deixem um pequeno espaço do texto para informar o leitor acerca do cientista que realizou a descoberta e quais foram suas fontes
inspiradoras.

% CNT: apoderar-se dos meios de pesquisa

\subsection{Pós"-leitura}

Por fim, já instrumentalizados no mote da obra, sugere"-se
que, com os estudantes em roda, seja proposto um debate que aborde as
diferenças entre o pensamento mítico e o lógico"-científico. A etimologia
dos termos mito e lógica, em grego arcaico, significam, ambas,
``palavra''. É apenas em Platão, na obra \textit{A República}, que encontramos a
cisão simbólica entre sinônimos. Na oportunidade, Platão pergunta a um
ouvinte se ele deseja receber a explicação por meio do raciocínio
lógico, ou de um mito. Nessa seara, temos a clarificação de que, a
partir de então, \emph{logos} passou a definir a palavra que vem do raciocínio
humano, e o mito, a palavra que vem dos deuses. Ampliando a definição,
Platão também teria dito que o mito é uma mentira que contém muitas
verdades.

\SideImage{Folha de rosto de <<A República>> de Platão, edição de 1713 (Columbia University; Domínio Público)}{PNLD0008-24.jpg}

Nesse sentido, convém contar com o auxílio do professor de ciências
humanas para indicar a importância sociológica e cultural dos mitos. O debate deve ser feito com uma abordagem do aspecto psicológico que a ligação com o metafísico ocasiona no ser humano. É adequado trazer também uma reflexão sobre os perigos
da exclusividade do tecnicismo em contraposição aos perigos de uma fé irracional.
% sociológica e cultural dos mitos: outras temporalidades

\section{Aprofundamento}


\subsection{Sobre o autor}

Não são poucos os médicos (atuantes ou não na prática da medicina) que escrevem literatura -– célebres nomes como Anton Tchekhov, Guimarães Rosa, Moacyr Scliar e Arthur Conan Doyle figuram nessa lista. Fabio Atui é também um desses clínicos"-literatos -– médico gastroenterologista e proctologista, formou"-se em 1995 pela Faculdade de Medicina da USP. É cirurgião"-geral do Hospital das Clínicas de São Paulo, onde opera diversas enfermidades que assolam o trato digestivo. Também é responsável pelo ambulatório de doenças sexualmente transmissíveis e proctologia do mesmo hospital, onde atende semanalmente. Ainda no Hospital das Clínicas, Fabio é membro ativo do Navis (Núcleo de Assistência à Vítima de Violência Sexual), amparando e assistindo pessoas sexualmente agredidas.

Desde 2004, Fabio atua também como cirurgião voluntário e coordenador de cirurgia dos Expedicionários da Saúde, um grupo de médicos voluntários que levam atendimento especializado a populações indígenas em regiões isoladas. 


\Image{Bacia Amazônica com a localização do rio Negro (Kmusser; CC-BY-SA 3.0)}{PNLD0008-05.png}


À toda essa vivência em contato direto com pessoas, soma"-se a crença de Fabio de que a disseminação de informações de qualidade é a receita certa para que as pessoas vivam melhor. O conhecimento humano adquirido em tantos anos de dedicação à área da saúde agrega"-se ao fazer artístico: Fabio assina, junto com dois outros autores, o roteiro do documentário \emph{Expedicionários}, longa"-metragem narrativo do trabalho da equipe de médicos voluntários para com indígenas que habitam regiões remotas do país, onde a infraestrutura e o trabalho clínico especializado não chegam. O filme foi dirigido por Octávio Cury e lançado em 2012. 

Mas a vivência com o campo das artes acompanha Atui desde a infância: Fabio é sobrinho de Fauzi Arap, um dos principais nomes da dramaturgia brasileira, de quem toma emprestado o título \emph{Mare nostrum} e a inspiração e ideia inicial do romance.

\subsection{Justificativa de leitura}
% Atualidade do texto
\textit{Mare nostrum: Paranã Tipi} é um texto atual, publicado em 2019, num
gênero de ficção que se aproxima da ficção científica, esbarrando nos
subgêneros da ficção metacientífica e psicocientífica. A ficção científica é um
gênero pouco explorado na escola, mas de muita identificação com o universo do
adolescente, principalmente por meio de produções audiovisuais estrangeiras.
Pouco se menciona de ficção com caráter científico produzida e ambientada no
Brasil.

O livro parte de uma potência comum a toda a humanidade: a capacidade de
sonhar. A trama ficcional não se desenvolve em torno de complexos aparatos
tecnológicos de vocabulário intrincado ou especialista, dificultando a
compreensão do adolescente. O leitor é convidado a ter reflexões acerca do próprio
conhecimento, dos mistérios em torno da mente humana, daquilo que é pressuposto
como verdade científica e da importância do acesso à informação.

A obra é ainda bastante interdisciplinar. Por partir de pressupostos
científicos, o livro permite trabalhar em paralelo os conhecimentos do aluno
sobre a Teoria da Evolução das Espécies, conceitos físicos de onda, eletromagnetismo 
e eletrostática, além da teoria psicanalítica e de um perene teor
filosófico que permeia o enredo do início ao fim da trama.

Em uma sociedade cada vez mais tecnicizante, cuja hipervalorização da
tecnologia e da produção e consumo desenfreado alienam o jovem do instinto de
percepção do seu entorno e da indagação filosófica, 
\emph{Mare nostrum} vai levantar os pilares de que conhecimento e imaginação se alimentam.

\subsection{Sobre o gênero}
% Gênero e categoria
O gênero da ficção científica é um tipo específico de ficção, que constrói o
enredo a partir da ciência -– real ou imaginada -– e do seu impacto em
determinada sociedade. Surgiu a partir do século \textsc{xix}, principalmente em razão
do desenvolvimento científico e tecnológico nas áreas da Física, Química,
Geologia, Astronomia e Informática, que introduziram descobertas, invenções e
aparatos tecnológicos que modificaram a vivência da sociedade humana.

Muito popular em obras estrangeiras, especialmente em filmes e séries para a
televisão, o gênero ainda tem poucos autores de referência no Brasil.
\textit{Mare nostrum: Paranã Tipi} poderia ser categorizado como ficção
psicocientífica, por ficcionalizar algumas teorias psicanalíticas do
inconsciente, do material onírico, daquilo que é o sonhado, ou ainda como
ficção metacientífica, por orbitar a esfera da indagação da ciência acerca da
própria ciência, lançando mão de pressupostos e teorias científicas para a
construção do enredo.


\Image{Floresta Amazônica, vista do alto (Lúcia Barreiros; CC-BY 3.0 br)}{PNLD0008-10.png}


\subsection{Superficial versus o essencial}
Na leitura do livro, vê"-se o desenvolver da personagem, que demonstra
ser um rapaz pouco empático, autocentrado, para alguém atento ao outro e
desejoso de fazer"-se útil a um maior número de pessoas e em
diversas frentes.

Ao longo da história também se percebe como o protagonista vai deixando
o apreço a tecnologia em segundo plano. O apreço ao apara to tecnológico é
gritante já no início da obra, onde o narrador descreve a situação em que
Theo, toma conta do pai adoentado por meio de um
sistema de domótica, controlado por aplicativo.

Isso também ocorre com seu \emph{mare nostrum}, inicialmente utilizado
pela personagem de forma mesquinha, para fugir das frustrações
cotidianas. Depois, Theo o oferece a pessoas em desalento, como quando o
faz em auxílio do índio Carú.

O próprio tom da escrita evolui ao longo da obra, deixando um início
truncado e arredio para uma fluência maior no caminhar da
história, sendo um importante fio"-condutor da história, revelando
ao leitor que o distanciamento de superficialidade pode conduzir a
caminhos mais brandos.

\Image{Vista aérea da comunidade de Assunção do Içana, no município de São Gabriel da Cachoeira (AM), na região do Alto do Rio Negro (Marcelo Camargo/Agência Brasil; CC-BY 3.0 br)}{PNLD0008-09.png}

\subsection{Mergulhar em si no oceano cósmico: a viagem para longe e a
jornada interior}

É curioso pensar que Theo detinha um mecanismo de se conectar com as pessoas do mundo, mas mesmo assim era alguém distante, incapaz de priorizar o outro no lugar de si próprio.

Forçado a realizar a grande jornada rumo à Amazônia, é confrontado com
uma realidade muito distante da sua. Inconformismo, raiva, menosprezo,
são todos sentimentos que lhe envolvem a alma em um primeiro momento.
Entretanto, numa breve epifania, o
protagonista observa a capacidade que possui, tomando pequenas atitudes e com boa vontade, de
tornar a vida de outra pessoa menos miserável.

E nessa jornada ao desconhecido, acaba por se conhecer melhor, detendo
de si uma visão mais clara, construída a partir de seu reflexo na retina
do outro.

\Image{Praia no Alto Rio Negro, uma das regiões em que estão os aprox. 19 060 falantes de Nheengatu, de onde se origina o termo ``Paranã Tipi''. (Isabelle Allet-Coche; CC-BY-SA 4.0)}{PNLD0008-04.png}

\subsection{De muitos, um, e de um, muitos}

O tema de um médico de um grande centro encaminhado a uma província
distante, ou a um lugar inóspito é recorrente na literatura. Como
exemplo, podemos retomar as cartas que Tchekhov escreveu em sua viagem à
Sacalina, ou a experiência que Carlo Levi viveu nas vilas de Grassano e
Aliano. E tais vivências são narradas sempre como um desabrochar de
humanidade, uma reflexão sobre os pontos importantes da vida e sobre a
beleza contida nos detalhes.

A obra \emph{Mare nostrum: Paranã Tipi} não fica para trás nesse sentido,
havendo também nela a riqueza de se desenvolver em solo nacional,
desvelando situações e características muito próprias à nossa realidade.

Apesar de ficcional, a história traz narrativas reais amalgamadas no fio
condutor do enredo. Alguns nomes na história são pessoas reais e, mesmo
as personagens que aparecem são, por vezes, um mosaico de diversas
outras. O próprio protagonista resulta de experiências do autor e de
seus colegas.

O próprio autor afirma que a comunidade que compõe boa parte do enredo é
ficcional, sendo na verdade uma colcha de retalhos de muitas outras,
essas sim, existentes.

Assim, \emph{Mare nostrum: Paranã Tipi} representa a união em poucas
personagens das experiências de diversos seres humanos. Mas também, ao
se pensar em termos de humanidade, denota o quão próximo podemos ser um
dos outros e como, apesar de diferentes, somos similares em nossos
âmagos.

\subsection{Jogo de palavras}

Logo ao ler o título, já somos postos em confronto com dois universos: o do
latim: na expressão  \emph{mare nostrum}, e o do nheengatu: em  \emph{paranã tipi}.

De um lado, a menção ao mundo romano, do \emph{imperium} da \emph{civitas}, do idioma do
lácio, a partir do qual a língua portuguesa surgiu.
Do outro, a língua geral indígena, a língua da terra, a língua que remete a
uma vida ligada à natureza. Ambos os termos denotam água. Em latim, temos 
o ``nosso mar'', e em nheengatu, temos ``rio profundo''.

Essas referências se entremeiam pela obra, afinal, o \emph{mare nostrum} é o
ambiente onde Theo consegue se conectar com outras pessoas.
Porém, quando descobre que esse mesmo lugar já fora chamado de \emph{paranã tipi}  o
protagonista passa a entrar em contato profundo consigo mesmo.
A oposição entre a \textit{civitas} (cidade) e a selva também é pensada pelo próprio
protagonista, que passa a contestar a importância da tecnologia, e passa a
ver sentimentos e ações com mais apreço.


\Image{Palavras de origem nheengatu (Ilustração de Paola Saliby/\,Babbel; Divulgação)}{PNLD0008-23.png}

A oposição entre a \emph{civitas} e a selva também é pesada pelo próprio
protagonista, que passa a contestar a necessidade de todo o aparato
tecnológico que se encontra envolto, passando a ver com mais apreço
sentimentos e ações.
\Image{Localização de São Gabriel da Cachoeira, cidade em que 74\% dos habitantes são indígenas. As línguas oficiais no município, ao lado do português, são o nheengatu, o  tucano e o baníua. (Marcos Elias de Oliveira Júnior; CC-BY 2.5)}{PNLD0008-06.png}


A ideia de profundidade das águas é muito explorada dentro da
psicologia. Faz alusão a um mergulho profundo dentro de si, da busca
pelos naufrágios e tesouros há muito escondidos. Havia também, entre os
gregos, o mito de Nereu, deidade marinha que tinha a resposta para todas
as perguntas. Porém habitava as águas profundas, sendo necessária uma
árdua busca para encontrá"-lo. Quando isso se dava, contudo, era
necessário capturar Nereu que, para fugir, assumia as formas mais
assustadoras. Era preciso ter resiliência e domá"-lo, para então obter as
respostas buscadas.

Tal qual a captura de Nereu, o herói da história tem de enfrentar
monstros, mas não de um ser marinho, e sim os medos de sua
própria profundidade. E nesse conflito sai, como suas dúvidas e seus pacientes, sanado e pleno.




%.........................



\subsection{Atividades para o aprofundamento da pesquisa}

\subsubsection{<<História de uma reviravolta na minha vida>>}

O obra lida, vê"-se que o encaminhamento que a personagem principal
  recebeu logo no início estava longe daquilo que eram seus verdadeiros
  sonhos. Entretanto, ao final, o protagonista percebe que foi uma rica
  experiência, que o tornou uma pessoa melhor. Observando isso, peça 
  para os alunos
  escreverem uma narrativa autobiográfica
  em que apareça uma reviravolta como a vivida
  por Theo.
  \BNCC{EM13LP19}

\subsubsection{História de reviravoltas familiares}

Sugere"-se que os alunos tragam histórias de suas
  famílias que possam ser consideradas mitos pessoais. Como uma história
  que os avós ouviam quando eram novos, a saga de uma família imigrante,
  a narrativa dos pais ao irem ao um jogo de futebol paradigmático ou a
  um show de rock marcante. Peça para os alunos fazerem um questionário
  de perguntas para seus familiares.
\BNCC{EM13LP33}


\Image{Até meados de 1870, o idioma nheengatu era mais falado que o português nos Estados do Amazonas e Pará (Ilustração de Paola Saliby/\,Babbel; Divulgação)}{PNLD0008-22.png}

\subsubsection{Cultura material indígena}
Retome as discussões acerca da mitologia ameríndia. A seguir,
  solicite aos alunos que busquem itens da cultura material que ilustrem
  os mitos. Cada aluno deverá trazer imagens, réplicas de objetos ou
  trechos de vídeo de danças e/ou músicas que narrem esses mitos. Com o
  material coletado, sugere"-se formar um acervo que poderá ser exposto
  no colégio ou então, virtualmente, na página da escola.
\BNCC{EM13CHS104}

\Image{O Festival Cultural dos Povos Indígenas do Alto Rio Negro/AM, 
realizado anualmente em São Gabriel da Cachoeira. (Hans Denis Schneider; CC-BY-NC-SA)}{PNLD0008-07.png}


% \BNCC{EM13LP51}
Por fim, com relação ao campo das práticas de estudo e pesquisa,
proponha aos alunos que realizem um estudo investigativo sobre a
metodologia e as políticas governamentais para a demarcação de terras
indígenas, observando como esta variou ao longo da história.
Sugere"-se também que seja indicado aos alunos problematizar a real
eficácia dessas medidas e se as mesmas atendem as populações indígenas
as quais elas são voltadas.


\Image{Apresentação das agremiações no Festival Cultural dos Povos Indígenas do Alto Rio Negro/AM. (Hans Denis Schneider; CC-BY-NC-SA)}{PNLD0008-08.png}


\section{Sugestões de referências complementares}\label{sugestoes}

\subsection{Filmes}

\begin{itemize}
\item\emph{Xingu}. Direção: Cao Hamburger (Brasil, 2012).

O filme conta a história dos três irmãos Villas Bôas, que resolveram
trocar o conforto da cidade grande pelas aventuras na floresta e
acabaram se tornando personagens centrais na criação do atual Parque
Nacional do Xingu.

\item\emph{Mad Max}. Direção: (George Miller (1979).

O filme é um clássico dos \textit{road movies}, filmes de estrada, e pode servir para 
apresentar aos alunos também a noção de distopia, o contrário da utopia apresentada no livro. 

\item\emph{Central do Brasil}. Direção: Walter Salles (1998).

Trata"-se de outro \textit{road movie}, filme de estrada, que apresenta
o Brasil pela ótica de um menino perdido. A narrativa de viagem juntamente com o desafio 
do enfrentamento da diferença absoluta na primeira idade pode servir 
de um grande paralelo para a apresentação do gênero do romance. 

\item\emph{As hipermulheres}. Direção: Carlos Fausto, Leonardo Sette (2011).

Por fim, sugerimos esse belíssimo filme sobre como as jovens mulheres da etnia 
Kuikuro, do Mato Grosso, que encampam a missão de fazer um ritual, mas contam 
com apenas uma cantora sagrada. Toda a aldeia se coloca então em função do 
ritual. Trata-se de uma excelente introdução ao pensamento pan-amazônico e
também à noção de ritos e mitologias indígenas. 
\end{itemize}


\subsection{Para visitar}

\begin{itemize}

\item Memorial dos Povos Indígenas

Construído em 1987, o \href{http://www.cultura.df.gov.br/memorial-dos-povos-indigenas/}{Memorial dos Povos Indígenas} foi projetado por Oscar Niemeyer em forma de espiral que remete a uma maloca redonda dos índios Yanomami. 

\item Museu do Índio

O \href{http://www.museudoindio.gov.br}{Museu do Índio} possui um acervo com 
milhares de peças, além de biblioteca, galeria de arte e espaços agradáveis para 
receber os visitantes. Fica no bairro do Botafogo, no Rio de Janeiro.

\end{itemize}

\subsection{\emph{Sites}}

\begin{itemize}

\item \href{https://beirasdagua.org.br/}{Beiras D'Água}

Site com acervo colaborativo de produções audiovisuais do Rio São Francisco, possui filmes produzidos por
moradores da região, incluindo quilombolas, indígenas e ribeirinhos. 

\item \textsc{funai}

O site da \href{https://www.gov.br/funai}{Fundação Nacional do Índio} é um importante banco de dados
sobre povos e terras indígenas. Também informa sobre meio ambiente e
direitos sociais.

\item \href{https://www.socioambiental.org/pt-br}{socioambiental.org}
O Instituto Socioambiental (ISA) é uma ONG sem fins lucrativos, fundada em 1994, com foco da defesa do meio ambiente, do patrimônio cultural, dos direitos humanos e de comunidades ribeirinhas, quilombolas e indígenas.
No site, é possível encontrar diversas notícias e iniciativas sobre o assunto.

\item \href{https://trabalhoindigenista.org.br/}{Biblioteca Digital do Centro de Trabalho Indigenista}

O site da Biblioteca Digital do Centro de Trabalho Indigenista possui um acervo de documentos, artigos e fotos acerca das políticas de atuação em Terras Indígenas, feita por meio de projetos elaborados a partir de demandas locais, visando contribuir para que os povos indígenas tenham garantia de seus direitos constitucionais. O Centro de Trabalho Indigenista é uma associação sem fins lucrativos, fundada em março de 1979 por antropólogos e indigenistas.


\end{itemize}

\section{Bibliografia comentada}

\begin{itemize}

\item\textsc{oz}, Amós. \emph{De repente nas profundezas do bosque}. São Paulo:
Seguinte, 2007.

Habitantes de uma aldeia onde não há animais, os jovens Mati e Maia
resolvem encarar o bosque frondoso que circunda a cidade, desafiando a
proibição de nunca entrar naquele território.

\item\textsc{furtado}, Maria Cristina. \emph{O Guardião das Florestas}. São Paulo:
Editora do Brasil Literatura, 2007.

Na Amazônia, Jaciara vive uma aventura que mudará sua vida. Ao lado do
tio Jari, que tem o dom de conversar com animais, de um macaco e do
Curupira, defenderá a floresta daqueles que querem lucro a qualquer
custo.

\item\textsc{carvalho}, Bernardo. \emph{Nove noites}. São Paulo: Companhia de Bolso, 2006.

Fruto de profunda pesquisa, o livro narra a história do antropólogo
americano Buell Quain, que se matou em 1939, aos 27 anos, enquanto
tentava voltar para a civilização, vindo de uma aldeia indígena no
interior do Brasil.

\item\textsc{chatwin}, Bruce. \emph{O rastro dos cantos}. São Paulo: Companhia das
Letras, 1996.

O escritor vai atrás do rastro dos cantos, ligado aos mitos de
aborígines da Austrália Central, sobre seres legendários que
atravessaram o continente no tempo da criação, cantando o que viam e
dando existência ao mundo através do canto.


\item\textsc{eliade}, Mircea. \emph{Cosmos e história: o mito do eterno retorno}. São Paulo: Mercúryo, 2004.

Este trabalho fundador da história das religiões aborda as expressões e
atividades de uma grande variedade de culturas religiosas arcaicas e
``primitivas''.

\item\textsc{krenak}, Ailton. \emph{Ideias para adiar o fim do mundo}. São Paulo:
Companhia das Letras, 2019.

O conhecido líder indígena recusa a ideia de humanidade como algo
separado da natureza e do humano como superior aos demais seres.

\item\textsc{lévi"-strauss}, Claude. \emph{Tristes trópicos}. São Paulo: Companhia das Letras, 1996.

Com um texto que se posiciona entre o ensaio e a narrativa de viagem, o
renomado antropólogo desloca parâmetros consagrados, questionando
viajantes e cientistas.


\item\textsc{munduruku}, Daniel. \emph{Contos indígenas brasileiros}. São Paulo:
Global Editora, 2004.

Os oito contos selecionados pelo autor, a partir de um critério
linguístico, retratam através de seus mitos a caminhada de povos
indígenas de Norte a Sul do Brasil.


\item\textsc{werá}, Kaká. \emph{A terra dos mil povos: história indígena do Brasil}. São
Paulo: Editora Peirópolis, 2020.

Nesta obra, diversos antropólogos se debruçam sobre a questão de quem
eram e como pensavam os primeiros habitantes do Brasil.
\end{itemize}

\end{document}

