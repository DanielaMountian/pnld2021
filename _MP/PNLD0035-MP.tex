\documentclass[12pt]{extarticle}
\usepackage{manualdoprofessor}
\usepackage{fichatecnica}
\usepackage{lipsum,media9,graficos}
\usepackage[justification=raggedright]{caption}
\usepackage{bncc}
\usepackage[escoladepais]{../edlab}

\begin{document}


\newcommand{\AutorLivro}{Safo de Lesbos}
\newcommand{\TituloLivro}{Hino a Afrodite e outros poemas}
\newcommand{\Tema}{Ficção, mistério e fantasia}
\newcommand{\Genero}{Poema}
\newcommand{\imagemCapa}{./images/PNLD0035-01.png}
\newcommand{\issnppub}{---}
\newcommand{\issnepub}{---}
% \newcommand{\fichacatalografica}{PNLD0035-00.png}
\newcommand{\colaborador}{\textbf{Fulano de Tal} é uma pessoa incrível e vai fazer um bom serviço.}


\title{\TituloLivro}
\author{\AutorLivro}
\def\authornotes{\colaborador}

\date{}
\maketitle

\begin{abstract}
Este Manual tem como objetivo fornecer subsídios para o trabalho com a
obra literária \emph{Hino a Afrodite e outros poemas}, de Safo de Lesbos.

É difícil afirmar com certeza alguma coisa sobre a vida de Safo de Lesbos.
Convém se dizer que ela nasceu de uma família aristocrática em Êresos, atual região da
Grécia, na ilha de Lesbos, por volta de 630~a.C. Seu nome é famoso desde seu tempo entre 
os maiores poetas gregos, tendo sido referência para muitos deles. Suas composições fazem
parte de um dos gêneros mais importantes dessa poesia, a \textit{mélica} ou \textit{lírica}.
Safo é o único nome feminino no conjunto de poetas da Grécia antiga e se tornou famosa por 
compor versos celebrando o amor que eram cantados ao som da lira.

\emph{Hino a Afrodite e outros poemas} reune vinte e sete poemas de tradição oral
que sobreviveram ao tempo traduzidos e anotados por Giuliana Ragusa, especialista em 
poesia grega arcaica. Antes dos poemas, o professor contará com uma introdução sobre Safo, 
sua poesia e o contexto em que se produziu e circulou, além do gênero mélico, a fortuna 
crítica sobre a poeta, a transmissão de sua obra, e as outras poetas mulheres de que se 
tem notícia. Com isso, poderá facilitar a aproximação dos estudantes com os fragmentos 
dos poemas.

Feitas para serem executadas em simpósios, eventos públicos onde as pessoas bebiam vinho
e ouviam os poemas cantados, os poemas-canções de Safo aqui presentes carregam uma ideia 
de solenidade, caráter religioso, e de elogio. A mitologia grega, sobretudo Afrodite, a 
a deusa do Amor, estará presente em muitos momentos, como no caso do poema que dá nome 
à antologia, "Hino a Afrodite". 

Esperamos que, por meio das indicações propostas, o professor possa encontrar elementos 
para aproximar os estudantes do Ensino Médio dos elementos do universo sáfico na poesia grega arcaica!





\lipsum[1-3]
\end{abstract}

\tableofcontents




\section{Atividades 1}

%\BNCC{EM13LP26}

\subsection{Pré-leitura}
\subsection{Leitura}
\subsection{Pós-leitura}



\section{Atividades 2}

\subsection{Pré-leitura}
\subsection{Leitura}
\subsection{Pós-leitura}

\lipsum
\end{document}

