\documentclass[12pt]{extarticle}
\usepackage{manualdoprofessor}
\usepackage{fichatecnica}
\usepackage{lipsum,media9,graficos}
\usepackage[justification=raggedright]{caption}
\usepackage[one]{bncc}
\usepackage[escoladepais]{../edlab}

\begin{document}

\newcommand{\AutorLivro}{Safo de Lesbos}
\newcommand{\TituloLivro}{Hino a Afrodite e outros poemas}
\newcommand{\Tema}{Ficção, mistério e fantasia}
\newcommand{\Genero}{Poema}
\newcommand{\imagemCapa}{./images/PNLD0035-01.png}
\newcommand{\issnppub}{---}
\newcommand{\issnepub}{---}
% \newcommand{\fichacatalografica}{PNLD0035-00.png}
\newcommand{\colaborador}{\textbf{Vicente Castro e Bruno Gradella}}


\title{\TituloLivro}
\author{\AutorLivro}
\def\authornotes{\colaborador}

\date{}
\maketitle

\baselineskip=1.2\baselineskip\par


\begin{abstract}\addcontentsline{toc}{section}{Carta ao professor}
Este Manual tem como objetivo fornecer subsídios para o trabalho com a
obra literária \emph{Hino a Afrodite e outros poemas}, de Safo de Lesbos.

É difícil afirmar com certeza alguma coisa sobre a vida de Safo de Lesbos.
Convém se dizer que ela nasceu de uma família aristocrática em Êresos, atual região da
Grécia, na ilha de Lesbos, por volta de 630~a.C. Seu nome é famoso desde seu tempo entre 
os maiores poetas gregos, tendo sido referência para muitos deles. Suas composições fazem
parte de um dos gêneros mais importantes dessa poesia, a \textit{mélica} ou \textit{lírica}.
Safo é o único nome feminino no conjunto de poetas da Grécia antiga e se tornou famosa por 
compor versos celebrando o amor que eram cantados ao som da lira.

\emph{Hino a Afrodite e outros poemas} reune vinte e sete poemas de tradição oral
que sobreviveram ao tempo traduzidos e anotados por Giuliana Ragusa, especialista em 
poesia grega arcaica. Antes dos poemas, o professor contará com uma introdução sobre Safo, 
sua poesia e o contexto em que se produziu e circulou, além do gênero mélico, a fortuna 
crítica sobre a poeta, a transmissão de sua obra, e as outras poetas mulheres de que se 
tem notícia. Com isso, poderá facilitar a aproximação dos estudantes com os fragmentos 
dos poemas.

Feitas para serem executadas em simpósios, eventos públicos onde as pessoas bebiam vinho
e ouviam os poemas cantados, os poemas-canções de Safo aqui presentes carregam uma ideia 
de solenidade, caráter religioso, e de elogio. A mitologia grega, sobretudo Afrodite, a 
a deusa do Amor, estará presente em muitos momentos, como no caso do poema que dá nome 
à antologia, "Hino a Afrodite". 

Esperamos que, por meio das indicações propostas, o professor possa encontrar elementos 
para aproximar os estudantes do Ensino Médio dos elementos do universo sáfico na poesia grega arcaica!

\end{abstract}


\section{Introdução}


Mito e realidade se misturam quando falamos em Safo de Lesbos. Primeiro porque
esse olhar biografizante sobre obras literárias é uma característica mais nossa,
moderna, do que da Grécia arcaica. Ainda que fosse comum o interesse pela pessoa
que enunciava as canções, não havia uma associação direta, como tornou-se costume
fazer, entre o conteúdo do poema com a vida real da pessoa que o compôs.
Safo é o único nome feminino da primeira fase da poesia grega antiga, conhecida 
como fase arcaica. Esse período se estende de 800 a 480 a.C.

Safo nasceu em 630 a.C. em uma família aristocrática. Ela viveu em Mitilene e foi 
contemporânea do poeta Alceu, que viveu na mesma região. O que conhecemos de sua
obra são fragmentos. Completo, há apenas um poema, e então, dez fragmentos, uma centena 
de citações breves de autores antigos e cerca de cinquenta peças de textos em papiro
encontradas no Egito.

No ano de 593 a.C., Lesbos estava sob a ditadura de Pitaco, instalado no poder pelos 
comerciantes e membros menos prósperos da sociedade. Os aristocratas, inconformados 
com a perda do poder, tentam resgatá-lo, conspirando contra o estadista, mas são 
novamente subjugados e seus líderes são enviados para o exílio, entre eles Safo e 
o poeta Alceu, que se tomou de amores pela poetisa, a se levar em conta a ardente, 
mas puritana troca de cartas entre ambos. Este falso recato ocultava a total ausência 
de pudor dos dois artistas, e a total liberalidade da jovem.

Depois do exílio em Pirra, Safo é deportada para a Sicília, bem distante de Lesbos, 
pois Pitaco sentia-se ameaçado por seus escritos. Lá, ela contrai matrimônio com um 
afortunado industrial de Andros, com quem tem uma filha, Cleis, e depois a deixa 
viúva e rica. Após permanecer cinco anos fora de sua terra natal, ela retorna e 
assume a liderança da comunidade local na esfera cultural. Elegante, bela e livre 
de compromissos, passa a viver sem preceitos morais, com absoluta liberdade.

Safo e Alceu são nomes importantes de um gênero poético da mélica. Suas canções 
eram destinadas à performance em solo, no simpósio, ou coral, nos festivais cívicos 
e religiosos. Os poemas eram acompanhados pela lira e por outros instrumentos musicais 
de acordo com cada gênero. A poesia grega é de matriz oral e está ligada a uma 
“cultura da canção” recitada ou cantada na performance. Essa poesia, sempre com uma 
função, transmitia ideias morais, políticas e sociais.

Mélica ou lírica designava o verso que era cantado para a música e para a 
dança, e era composto de elementos de ritmos e tamanhos variados. A mélica era, 
na literatura grega arcaica, o poema lírico. Mas não o lírico que conhecemos no
mundo moderno, aquela poesia subjetiva onde o ``coração fala", mas lírica
pois vem de \textit{lýra}, palavra grega que deu em ``lira", o instrumento musical
que acompanhava a \textit{performance}.

A coralidade da mélica (ou lírica) grega arcaica de Safo é o aspecto que tem sido 
repensado e revalorizado para compreender sua poesia e de seu universo, e sobretudo, 
da relação entre a poeta e seu grupo de moças não casadas que viviam ao seu redor.
Combinou-se a esse estimulante fato o enfoque intensificado na performance da mélica 
grega arcaica, e o resultado tem sido a avaliação mais precisa da produção de Safo, 
e a ampliação da distância das leituras biografizantes e romantizadas, que tanto 
buscam na poeta o que é estranho à sua poesia e, em verdade, à poesia antiga como 
um todo: a voz pessoal, o subjetivismo, o intimismo, a privacidade, os sentimentos 
confessos do “eu”. 

Sob a ótica do Romantismo, a poesia de Safo tem sido lida como uma expressão de 
seus sentimentos íntimos. De acordo com esse modelo, ela aparece como uma solista, 
cantando suas canções e tocando sua lira em um espaço privado, num círculo 
pequeno de companheiras do mesmo sexo.

A propósito dessa ideia que por tanto tempo perdurou --- e, de certo modo, em
certas leituras modernizantes vinculadas a questões de gênero e sexualidade, 
ainda perdura ---, vale recordar, como bem afirma Franco Ferrari, que não há 
indicação da existência de composições poéticas da Grécia arcaica e clássica, 
ao menos de Homero até Eurípides, que não fossem feitas tendo em vista um 
contexto específico, para uma situação convencional oral e para uma audiência definida. 

Isso vale para a mélica de Safo, ao contrário do que amiúde se imagina, por 
aproximação equívoca com a lírica moderna. O diálogo está presente nos gêneros 
poéticos da grécia antiga, desde a épica de Homero. Em Safo, há um componente 
de diálogo que remete à cultura oral.
 
Por fim, é importante salientar que o universo masculino e o feminino eram 
realidades separadas na Grécia antiga em geral. No entanto, no que diz respeito
à poesia, a linguagem e as imagens que Safo emprega no tratamento da paixão 
são as mesmas que encontramos em Arquíloco e em outros poetas homens, antes 
ou depois dela, encontramos na poesia um espaço de convivência entre estes
universos na sociedad grega arcaica.

\SideImage{Escultura de Safo, Museu Bourdelle, em Paris (Jean-Pierre Dalbéra; CC-BY-SA 2.0)}{PNLD0035-03}

\tableofcontents
\section{Proposta de atividades I}

\subsection{Pré-leitura}

%De olho na BNCC}

% \textbf{As atividades sugeridas nesta subseção contemplam as seguintes
% habilidades propostas pela BNCC:}

% (EM13LGG302) Compreender e posicionar-se criticamente diante de diversas
% visões de mundo presentes nos discursos em diferentes linguagens,
% levando em conta seus contextos de produção e de circulação.

% (EM13LGG704) Apropriar-se criticamente de processos de pesquisa e busca
% de informação, por meio de ferramentas e dos novos formatos de produção
% e distribuição do conhecimento na cultura de rede.

% (EM13LP10) Selecionar informações, dados e argumentos em fontes
% confiáveis, impressas e digitais, e utilizá-los de forma referenciada,
% para que o texto a ser produzido tenha um nível de aprofundamento
% adequado (para além do senso comum) e contemple a sustentação das
% posições defendidas.

% (EM13LP19) Compartilhar gostos, interesses, práticas culturais, temas/
% problemas/questões que despertam maior interesse ou preocupação,
% respeitando e valorizando diferenças, como forma de identificar
% afinidades e interesses comuns, como também de organizar e/ou participar
% de grupos, clubes, oficinas e afins.\strut
% \end{minipage}\tabularnewline
% \bottomrule
% \end{longtable}


\paragraph{Tema} A mudança na concepção de poesia lírica no decorrer do tempo.

\paragraph{Conteúdo} Compreensão das diferenças entre a melopeia grega arcaica
utilizada por Safo e o lirismo romântico dos poetas modernos.

\paragraph{Objetivo} Estimular e habilitar os estudantes a identificar as 
diferenças entre os dois universos poéticos construindo relações entre
produções atuais que fazem parte de suas realidades e as concepções 
apresentadas sobre o mundo grego arcaico.

\paragraph{Justificativa} "Hino a Afrodite e outros poemas" reúne textos traduzidos 
e anotados remanescentes da mélica grega arcaica, ou seja, das canções para 
\textit{performance} ao som da lira. Aliás, é daí que vem o nome ``lírica'' 
para se referir a poesia --- os poemas cantados ao som da \textit{lýra}, em grego.

O que o mundo moderno chama de poesia lírica é não tem muito a ver com
o jeito que ela era vista na época de Safo. Foi sobretudo no auge do
Romantismo europeu, no século \textsc{xviii} que poesia lírica passou a ser
associada à composição que expressava o \textit{cri du coeur}, ou seja,
o grito do coração, da alma do poeta. Quanto mais subjetivo e autêntico
em relação às suas emoções, melhor era sua qualidade enquanto poema
lírico. O que o poema dizia era o que o poeta sentia. 

Muito dessa concepção romântica de poesia lírica resiste até nossos dias
de hoje, apesar dos esforços de movimentos literários posteriores ao Romantismo
em deixar ela para trás. Para termos uma experiência mais honesta com a mélica 
sáfica, então, teremos que levar em conta alguns aspectos desse momento da 
história da literatura.

Os poemas na Grécia arcaica tinham um propósito para serem feitos e apresentados. 
Eles só existiam no momento da execução, cantada em solo ou em coral, 
declamada, acompanhada de uma variedade de instrumentos de acordo com cada ocasião. 
A\textit{performance} poderia ser em situações privadas, como os simpósios, tipos 
de festas com bebidas e música/poesia, nos palácios aristocráticos, ou em situações 
públicas, como as celebrações de matrimônios ou os louvores a algum deus, como é o
caso de boa parte dos poemas sáficos. 

As produções da poesia se organizavam em quatro grandes divisões canônicas conforme
seu propósito: épica, jâmbico, elegia e lírica, sendo \emph{grosso modo} a épica 
para a guerra; o jâmbico para a disputa; a elegia para o lamento da morte ou amor 
não correspondido; e finalmente a lírica, para celebrar a vitalidade na sua multiplicidade.

\Image{Ânfora tirrena de produção ateniense, período arcaico, 570-560 a.C (Wikimedia Commons; CC-BY-SA 3.0)}{PNLD0035-04}

\paragraph{Metodologia} 

\begin{enumerate}
\item
Como ponto de partida, o professor pode perguntar à turma o que é poesia,
e, logo em seguida, o que é um poeta. Outra pergunta possível é: desde quando
as pessoas fazem poesia? E em quais partes do mundo? 
Anotando na lousa algumas definições dadas pelos alunos e pedindo que eles
também as registrem em seus cadernos, é importante que o professor estimule 
uma construção coletiva, onde cada um traga seu repertório e se possa alcançar,
em conjunto, um senso comum sobre poesia.

\item
Depois dessa primeira construção coletiva, segue-se uma parte expositiva onde 
o professor introduzirá elementos da sociedade grega arcaica, como a cultura
da oralidade, a importância da vida pública e a presença da mitologia na religiosidade. 
Neste ensejo, deve apresentar as características da poesia produzida neste período, 
enfatizando seu caráter pragmático e indissociável da \textit{performance} musical.
É importante que as características dos gêneros fiquem bem definidas, se possível
escritas na lousa para facilitar a atividade que se seguirá. 

\item
Na última parte da atividade, preferencialmente com a turma dividida em pequenos 
grupos, apresente-os alunos trechos de quatro canções diferentes, se possível numa 
plataforma audiovisual onde eles possam ter acesso à execução da música, além da letra. 
Em seguida, peça que relacionem cada uma das músicas com um dos gêneros da poesia grega 
arcaica, indicando onde, na música, percebem a característica apresentada na aula. 

Nossas sugestões de músicas são, para a épica: ``Um comunista'', de Caetano Veloso, 
onde o poeta e cantor descreve a vida do guerrilheiro brasileiro Carlos Mariguella;
para o jâmbico: ``Ó Quem Chega", de Djonga, uma canção de afronta; para a elegia, 
``Freguês da meia"-noite'', de Criolo, um bolero de desilusão amorosa; e, por fim,
para a elegia, ``Louvação a Oxum'', de Roberto Mendes, interpretada por Maria Bethânia,
uma homenagem ao orixá yorubá do amor e da beleza.
O professor pode ficar à vontade para experimentar outras canções desde que siga os
mesmos critérios de escolha.  

\end{enumerate}

\paragraph{Tempo estimado} Duas aulas de 50 minutos. 

\subsection{Leitura}

%De olho na BNCC}

% As atividades sugeridas nesta subseção contemplam as seguintes


% (EM13LGG103) Analisar, de maneira cada vez mais aprofundada, o
% funcionamento das linguagens, para interpretar e produzir criticamente
% discursos em textos de diversas semioses.

% (EM13LP02) Estabelecer relações entre as partes do texto, tanto na
% produção como na recepção, considerando a construção composicional e o
% estilo do gênero, usando/reconhecendo adequadamente elementos e recursos
% coesivos diversos que contribuam para a coerência, a continuidade do
% texto e sua progressão temática, e organizando informações, tendo em
% vista as condições de produção e as relações lógico-discursivas
% envolvidas (causa/efeito ou consequência; tese/argumentos;
% problema/solução; definição/exemplos etc.).

% (EM13LP48) Perceber as peculiaridades estruturais e estilísticas de
% diferentes gêneros literários (a apreensão pessoal do cotidiano nas
% crônicas, a manifestação livre e subjetiva do eu lírico diante do mundo
% nos poemas, a múltipla perspectiva da vida humana e social dos romances,
% a dimensão política e social de textos da literatura marginal e da
% periferia etc.) para experimentar os diferentes ângulos de apreensão do
% indivíduo e do mundo pela literatura.
% \end{quote}\strut
% \end{minipage}\tabularnewline
% \bottomrule
% \end{longtable}


% Relação entre pintura e poesia 
% Uma descrição de uma pintura poderia ser fonte de poesia (ecfrase)
% Escolher dois ou tres quadros para descreverem em poema

\paragraph{Tema} A lírica sáfica e as outras artes

\paragraph{Conteúdo} Conhecimento e articulação das diferentes linguagens
artísticas na representação dos deuses gregos por meio do recurso da écfrase.

\paragraph{Objetivo} Estimular e habilitar os estudantes a perceber a 
correlação entre as diferentes linguagens artísticas, neste caso, entre a 
poesia e as artes visuais, na criação de obras de arte. 

\paragraph{Justificativa} Na história da literatura antiga, nos clássicos gregos
e romanos, a écfrase sempre esteve presente. A \textit{ekphrasis} pode ser 
explicada como uma descrição detalhada de uma imagem, paisagem, ou obra de 
arte visual. É um recurso retórico que busca, por meio da linguagem verbal, 
fazer aquele que lê ou ouve, ter uma impressão extraordinária sobre o objeto 
descrito, como se estivesse vendo cada um de seus detalhes sem os estar de fato vendo.

Uma descrição técnica, porém não é uma écfrase, mas talvez uma legenda ou uma
bula de remédio. A écfrase é um recurso de criação artística pois habita o lugar 
da palavra-imagem. É a reverberação verbal, em palavras, da imagem, mas também 
o contrário: pode-se escrever um poema a partir de uma pintura ou uma escultura, 
ou pintar um quadro e esculpir uma estátua a partir de um poema que se ouviu. 

Em \textit{Hino a Afrodite}, único texto totalmente preservado de Safo, exceto 
uma pequena lacuna no verso 19, começa com uma evocação à deusa do amor que
nos remete a algumas obras de artes visuais, da antiguidade ou do Renascimento,
onde Afrodite, ou sua versão romana, Vênus, está representada. 

\Image{À direita acima na ânfora, Afrodite está sentada de luto, Fthonos encostado nela. Abaixo está Oineus, cetro na mão esquerda. Abaixo da sala, Teseu e Peleu estão de luto. (Digital LIMC; Domínio Público)}{PNLD0035-05}

\paragraph{Metodologia} 

\begin{enumerate}

\item
Para introduzir a discussão sobre as diferentes linguagens artísticas, sugerimos
que o professor pergunte à turma se conhecem obras da literatura que tenham
uma versão audiovisual, como um filme ou uma série, e, então, instigue comentários
dos estudantes, inicialmente suas preferências, e então uma comparação sobre o que
tem em uma e não na outra. 

\item
Depois da discussão, o professor deve introduzir o tema da écfrase, dizendo que
no caso de muitas obras da Antiguidade não há como saber qual veio primeiro,
a escultura, por exemplo, ou o poema. Deve se ressaltar que o que mais importa
é que há uma relação entre as duas obras e as duas linguagens, e que uma inspira
a criação da outra. Mesmo que alguns prefiram mais uma à outra, o ideal é que 
possamos apreciar as obras em suas particularidades, pois sendo diferentes, 
proporcionam experiências diferentes a quem vê ou ouve. 

\item
Agora, o professor irá, conforme o costume na Grécia arcaica, recitar o ``Hino
a Afrodite'', de Safo, duas vezes de preferência, e então mostrar as imagens
abaixo, uma escultura e uma pintura representando a deusa. 

\Image{O desenho mostra Afrodite nua segurando a pequena figura de Eros perto dela enquanto ele tenta atirar uma flecha de seu arco. (Britannica; Domínio Público)}{PNLD0035-06}

Agora, repetirá o mesmo procedimento com os poemas ``Hino a Ártemis'' e 
``Hino a Dioniso'', ambos do poeta Anacreonte, e em seguir, mostre os quadros 
e esculturas representando os deuses.

``Hino a Ártemis''

Abraço teus joelhos, ó caçadora de cervos
loira filha de Zeus, senhora 
das feras selvagens, ó Ártemis,
que agora, em algum lugar perto
das correntes do Leteu, regozijando-se,
fixa os olhos na cidade dos homens
de ousados corações, pois não são indomados
os cidadãos que pastoreias...

``Hino a Dioniso''

Ó senhor com quem o domador Eros
e também as Ninfas de escuros olhos 
e a purpúrea Afrodite
brincam juntos, enquanto vagueias 
pelos altivos picos das montanhas:
os joelhos abraço-te, e tu, propício,
vem a nós, e aceitável
prece escuta:
a Cleóbulo sê bom
conselheiro, para ele minha paixão,
ó Dioniso, aceitar. 

\Image{Estátua da deusa Ártemis. Cópia romana de uma estátua original grega de Leocares. Localizada no Museu do Louvre, em Paris.}{PNLD0035-16}

\item
Por fim, disponha as seguintes imagens aos estudantes e peça que, seguindo os exemplos
mostrados, eles experimentem o recurso da écfrase e produzam um poema em relação
com a obra ou deus da mitologia grega de sua escolha. Ao fim da atividade, 
devem ser convidados a compartilhar oralmente seus poemas com o restante da turma.

%inserir imagens 

\end{enumerate}

\paragraph{Tempo estimado} Duas aulas de 50 minutos.

\subsection{Pós-leitura}

%De olho na BNCC}

% As atividades sugeridas nesta subseção contemplam as seguintes


% (EM13LGG102) Analisar visões de mundo, conflitos de interesse,
% preconceitos e ideologias presentes nos discursos veiculados nas
% diferentes mídias como forma de ampliar suas as possibilidades de
% explicação e interpretação crítica da realidade.

% (EM13LGG303) Debater questões polêmicas de relevância social, analisando
% diferentes argumentos e opiniões manifestados, para negociar e sustentar
% posições, formular propostas, e intervir e tomar decisões
% democraticamente sustentadas, que levem em conta o bem comum e os
% Direitos Humanos, a consciência socioambiental e o consumo responsável
% em âmbito local, regional e global.

% (EM13LGG402) Empregar, nas interações sociais, a variedade e o estilo de
% língua adequados à situação comunicativa, ao(s) interlocutor(es) e ao
% gênero do discurso, respeitando os usos das línguas por esse(s)
% interlocutor(es) e combatendo situações de preconceito linguístico.

% (EM13LGG703) Utilizar diferentes linguagens, mídias e ferramentas
% digitais em processos de produção coletiva, colaborativa e projetos
% autorais em ambientes digitais.

% (EM13LP13) Planejar, produzir, revisar, editar, reescrever e avaliar
% textos escritos e multissemióticos, considerando sua adequação às
% condições de produção do texto, no que diz respeito ao lugar social a
% ser assumido e à imagem que se pretende passar a respeito de si mesmo,
% ao leitor pretendido, ao veículo e mídia em que o texto ou produção
% cultural vai circular, ao contexto imediato e sócio histórico mais
% geral, ao gênero textual em questão e suas regularidades, à variedade
% linguística apropriada a esse contexto e ao uso do conhecimento dos
% aspectos notacionais (ortografia padrão, pontuação adequada, mecanismos
% de concordância nominal e verbal, regência verbal etc.), sempre que o
% contexto o exigir.

% (EM13LP14) Produzir e analisar textos orais, considerando sua adequação
% aos contextos de produção, à forma composicional e ao estilo do gênero
% em questão, à clareza, à progressão temática e à variedade linguística
% empregada, como também aos elementos relacionados à fala (modulação de
% voz, entonação, ritmo, altura e intensidade, respiração etc.) e à
% cinestesia (postura corporal, movimentos e gestualidade significativa,
% expressão facial, contato de olho com plateia etc.).

% (EM13LP28) Resumir e resenhar textos, com o manejo adequado das vozes
% envolvidas (do autor da obra e do resenhador), por meio do uso de
% paráfrases, marcas do discurso reportado e citações, para uso em textos
% de divulgação de estudos e pesquisas.

% (EM13LP29) Realizar pesquisas de diferentes tipos (bibliográfica, de
% campo, experimento científico, levantamento de dados etc.), usando
% fontes abertas e confiáveis, registrando o processo e comunicando os
% resultados, tendo em vista os objetivos colocados e demais elementos do
% contexto de produção, como forma de compreender como o conhecimento
% científico é produzido e apropriar-se dos procedimentos e dos gêneros
% textuais envolvidos na realização de pesquisas.

% (EM13LP52) Produzir apresentações e comentários apreciativos e críticos
% sobre livros, filmes, discos, canções, espetáculos de teatro e dança,
% exposições etc. (resenhas, vlogs e podcasts literários e artísticos,
% playlists comentadas, fanzines, e-zines etc.).\strut
% \end{minipage}\tabularnewline
% \bottomrule
% \end{longtable}

% Quatro pinturas e exercicio criativo de criacao de poemas atraves de uma descrição
\paragraph{Tema} É hora de ouvir a Musa...  

\paragraph{Conteúdo} Exercícios de escrita criativa de poemas líricos a partir
do que os gregos arcaicos entendiam como ``lírico'' e a concepção moderna
desse tipo de poesia. 

\paragraph{Objetivo} Orientar a produção de poemas a partir da experiência
de leitura, escuta e visualização das obras de arte apresentadas nas últimas
aulas, articulando o que foi estudado sobre o recurso de écfrase.

\paragraph{Justificativa} Toda experiência significativa gera reverberações
dentro de nós. Vimos isso no decorrer das últimas aulas com a écfrase: o 
encanto causado por um quadro ou uma escultura pode desencadear a realização
de uma obra poética. Bem como a leitura ou escuta de um poema ou música 
também pode nos inspirar para compor um. Este é, afinal, um dos recursos
utilizado pelos artistas para buscar inspiração: ``consumir'' arte para
então produzir. 

É neste sentido que, chegando à última etapa desse ciclo de atividades sobre
``Hino a Afrodite'', de Safo de Lesbos, propomos uma oficina de escrita criativa
no gênero da poesia lírica aos estudantes do Ensino Médio. A ideia central
é que estes estudantes desenvolvam um trabalho ao mesmo tempo prazeroso ---
a escrita de um poema --- mas também uma forma de verificar a apreensão do
conteúdo passado nas aulas. Não haverá uma norma pré-estabelecida de
produção certa ou errada. O que se espera é que eles sejam capazes de fazer
a leitura de obras de arte de linguagem verbal e não"-verbal e, a partir
das reverberações dessa leitura, criar um poema de sua autoria. 

É importante ressaltar que, nesta atividade, estará sendo trabalhada uma 
competência socio-cultural que diz respeito à criação artística, em nossa
sociedade moderna ainda associada à inspiração, algo estritamente individual
que nasce de algum lugar desconhecido de dentro do artista ou que ele ouve
de mundos espirituais. Ao mesmo tempo, estará sendo articulada a noção da ``Musa''
que canta nos ouvidos no poeta oferecendo"-lhe inspiração, como diz a tradição
antiga. Ora, o que pretendemos é mostrar que esta ``Musa que canta aos nossos
ouvidos'' pode ser a sua encarnação em obra de arte. Ou seja, ouvir à Musa
pode ser ler poesia, ver quadros e esculturas, ouvir música, enfim, ter 
experiências com obras de arte pode ser o desencadeador de uma inspiração
que poderá, caso o inspirado possua alguma técnica, uma obra de arte.

\Image{Busto de Afrodite, princípio do século IV a. C, Museu Arqueológico Nacional de Atenas (Jerónimo Roure Pérez; CC-BY-SA 4.0)}{PNLD0035-07}

\paragraph{Metodologia}

\begin{enumerate}

\item
Iniciar a aula pedindo que os estudantes retomem  o exercício de écfrase do
último encontro. Este momento inicial pode ser propício para uma roda de troca
sobre esta experiência. No momento oportuno, quando um número razoável de 
estudantes tiver falado, o professor deve avisar que agora eles farão
outro exercício, mas diferente do outro, já que não precisarão seguir 
o modelo da poesia arcaica, mas estarão livres para compor um poema
articulando tudo o que foi dito e visto nas últimas aulas sobre poesia,
da mais antiga à mais atual.

\item
Partindo do pressuposto de que as imagens podem ser motivo de inspiração
para a escrita --- pressuposto da écfrase, palavra"-imagem ---, o professor
deve dispor aos alunos quatro fotos distintas que a turma escolherá e 
produzirá um poema a partir delas. O professor pode ficar livre nesta escolha,
mas sugerimos aqui um repertório: ``A Noite Estrelada'', de Van Gogh; Aglomeração
de pessoas numa estação de metrô de São Paulo; ``Monumento às Bandeiras'', 
de Victor Brecheret; e ``Os Amantes'', de René Magritte.

\end{enumerate}

\Image{``Noite estrelada sobre o Ródano'' (1888), Vincent Van Gogh. Localizado no Museu de Orsay, em Paris.}{PNLD0035-11}

\Image{``Os Amantes'' (1928), René Magritte. Localizado no Museu de Arte Moderna de Nova York.}{PNLD0035-12}

\Image{``Monumento às Bandeiras'' (1953), Victor Brecheret. Cidade de São Paulo (Domínio público)}{PNLD0035-13}

\Image{Metrô de São Paulo em horário de pico}{PNLD0035-14}

\paragraph{Tempo estimado} Duas aulas de 50 minutos.



\section{Proposta de atividades II}


As obras de Safo possibilitam trabalhos interdisciplinares e
integradores de diferentes campos do saber e áreas de conhecimento. A
seguir, propomos algumas atividades que podem ser desenvolvidas
conjuntamente com professores de outras áreas. Além das habilidades de
Linguagens e suas Tecnologias e de Língua Portuguesa, indicadas nas
etapas da seção anterior e válidas também para esta, listamos a seguir
as habilidades de outras áreas, presentes na abordagem interdisciplinar:

% %De olho na BNCC}
% As atividades sugeridas nesta subseção contemplam as seguintes

% (EM13CNT201) Analisar e utilizar modelos científicos, propostos em
% diferentes épocas e culturas para avaliar distintas explicações sobre o
% surgimento e a evolução da Vida, da Terra e do Universo.

% (EM13CNT303) Interpretar textos de divulgação científica que tratem de
% temáticas das Ciências da Natureza, disponíveis em diferentes mídias,
% considerando a apresentação dos dados, a consistência dos argumentos e a
% coerência das conclusões, visando construir estratégias de seleção de
% fontes confiáveis de informações.

% (EM13CHS101) Analisar e comparar diferentes fontes e narrativas
% expressas em diversas linguagens, com vistas à compreensão e à crítica
% de ideias filosóficas e processos e eventos históricos, geográficos,
% políticos, econômicos, sociais, ambientais e culturais.

% (EM13CHS102) Identificar, analisar e discutir as circunstâncias
% históricas, geográficas, políticas, econômicas, sociais, ambientais e
% culturais da emergência de matrizes conceituais hegemônicas
% (etnocentrismo, evolução, modernidade etc.), comparando-as a narrativas
% que contemplem outros agentes e discursos.

% (EM13CHS106) Utilizar as linguagens cartográfica, gráfica e iconográfica
% e de diferentes gêneros textuais e as tecnologias digitais de informação
% e comunicação de forma crítica, significativa, reflexiva e ética nas
% diversas práticas sociais (incluindo as escolares) para se comunicar,
% acessar e disseminar informações, produzir conhecimentos, resolver
% problemas e exercer protagonismo e autoria na vida pessoal e coletiva.

% (EM13CHS401) Identificar e analisar as relações entre sujeitos, grupos e
% classes sociais diante das transformações técnicas, tecnológicas e
% informacionais e das novas formas de trabalho ao longo do tempo, em
% diferentes espaços e contextos.\strut
% \end{minipage}\tabularnewline
% \bottomrule
% \end{longtable}

\subsection{Pré-leitura}

% Trazer alguma explicação histõrica do texto da Raguzza

\paragraph{Tema} Voltando no tempo...

\paragraph{Conteúdo} Introdução e contextualização do período histórico
da Grécia Antiga a partir de um mapa interativo das cidades e colônias 
deste período.

\paragraph{Objetivo} Ambientalizar os estudantes na realidade vivida pela
poeta no contexto de produção das obras a serem estudadas em sala de aula.
Identificar as diferenças em relação ao período atual e as mudanças no decorrer
do tempo.

\paragraph{Justificativa} Partindo do pressuposto de que uma obra de arte, ainda
que não seja estritamente um documento histórico, reflete direta ou indiretamente
o mundo onde vive o seu autor, é de suma importância para uma leitura honesta
dela que haja uma contextualização, por parte do leitor, sobre esta realidade. 
O leitor deve fazer um esforço de, ainda que se mantenha no presente, com os
gostos e valores da sociedade contemporânea, deixar"-se ser transportado
pela própria obra a um mundo que é diferente do seu. Esta abertura à experiência
artística, de se deixar ser contaminado por ela e o que ela tem a mostrar,
vai facilitar para que esta seja uma experiência de maior profundidade e, 
consequentemente, qualidade. 

O contexto histórico de produção dos ``Hinos a Afrodite e outros poemas'' 
é o da Grécia arcaica. O Período Arcaico da história da civilização grega, 
ocorrido entre os séculos \textsc{III} e \textsc{VI} a.C., foi caracterizado 
principalmente pelo fim da comunidade gentílica, que se formou no Período 
Homérico, e pelo fortalecimento da organização dos membros da sociedade em 
torno das pólis, as cidades-Estado, comandadas por uma aristocracia proprietária de terras.

Aliado a esse processo de formação de um grupo social superior, com a aristocracia 
dominando e explorando o trabalho das demais pessoas da sociedade, houve a consolidação 
da propriedade privada das terras e dos instrumentos de trabalho, criando uma situação 
diferente do período anterior, quando havia a propriedade coletiva. Passaram a desenvolver 
uma economia de trocas de produtos entre as diversas cidades-Estado, criando mercados para 
além dos limites das comunidades.

A pólis (cidade, em grego) era uma organização social politicamente independente. 
Era uma organização social por estabelecer formas de convivência das pessoas que 
habitavam a pólis e politicamente independente por contar com um governo próprio 
e com leis que diziam respeito apenas à pólis, formando assim um Estado independente. 
Em razão disso, a pólis grega era também conhecida como cidade-Estado. Os habitantes 
da pólis que tinham direitos políticos eram chamados de cidadãos. Os habitantes de outras 
cidades-Estado, apesar de serem gregos e falarem a mesma língua, eram considerados 
estrangeiros caso estivessem em uma pólis diferente da que havia nascido.

\Image{Afrodite em esculturas gregas antigas (Projekt Runeberg; Domínio Público)}{PNLD0035-08}

\paragraph{Metodologia}

\begin{enumerate}

	\item Para início de conversa, pergunte aos estudantes o que eles sabem sobre a Grécia
	antiga. Faça perguntas sobre organizações sociais da sociedade ocidental 
	contemporânea, como divisão dos poderes, papel da religião na sociedade etc.,
	para instigá"-los sobre as possíveis diferenças em relação ao contexto da Grécia
	antiga.

	\item Neste segundo momento, é hora de uma abordagem mais expositiva. Apresente
	as organizações sociais e políticas da Grécia antiga, sobretudo a pólis, os espaços
	públicos compartilhados, como a ágora, a importância dos deuses na sociedade e 
	o lugar dos poetas em eventos como os simpósios, festas em palácios aristocráticos
	onde as elites se reuniam para comer, beber vinho e apreciar apresentações
	de música-poesia, e nos eventos cívicos de louvação aos deuses. 

	\item Agora, num espaço com acesso à Internet, proponha aos estudantes que explorem 
	um, num recurso educacional aberto, um mapa interativo da Grécia em seu período 
	arcaico identificando as principais cidades e colônias que comportavam o território
	em toda sua expansão. Indique, antes de tudo, a posição da ilha de Lesbos, local de 
	nascimento de Safo. A atividade deve ser feita até que todas as localizações
	tenham sido identificadas.

%inserir o link para o jogo: https://mapasinteractivos.didactalia.net/pt/comunidade/mapasflashinteractivos/recurso/o-mundo-grego-colonias-e-cidades/3c914586-634a-676b-7a96-d64728e0491a

\end{enumerate}


\paragraph{Tempo estimado} Duas aulas de 50 minutos.


% Arrumar um site na internet 
% duas ou tres páginas

\subsection{Leitura}

% História e Geografia 

%??????????????????????????????????


\paragraph{Tema} Os deuses na poesia.

\paragraph{Conteúdo} Compreensão das divindades do panteão grego clássico a partir
de suas representações nas linguagens artísticas, sobretudo pintura, escultura e poesia.

\paragraph{Objetivo} Familiarizar os estudantes com os deuses e deusas do panteão grego
a partir da leitura de alguns poemas de Safo e Anacreonte e suas respectivas representações
em quadros e esculturas. Os estudantes devem conseguir identificar as principais simbologias
de cada divindade e propor soluções para cada problema apresentado.

\paragraph{Justificativa} A mélica nas modalidades solo e coral está representada 
no \textit{corpus} sobrevivente de Safo.

Feita para a \textit{performance} no festival cívico"-religioso, patrocinado
pelos governos e aristocracias das cidades gregas, no qual se
desenvolviam muitas atividades --- como o \textit{agṓn} (“competição”) poético
para cada gênero poético, para a música e para o corpo (os jogos) \mbox{---,} 
a canção coral tinha no poeta o compositor das palavras e
da música, e o diretor do coro de cidadãos, que, liderado por um de seus
membros, conduzia a apresentação, cantando e dançando ao som dos instrumentos.
Nessa atmosfera festiva e de celebração pública e coletiva, o canto tinha por
tônica dominante “o \textit{prazer}, humano e divino”, pois eram homens e deuses 
homenageados a um só tempo.

Mas, para além do festival, também podiam servir de ocasião à
\textit{performance} da mélica coral os grandes funerais e as grandes bodas,
menos públicos e mais privados. São notáveis nessas três ocasiões a ideia da
solenidade, o caráter religioso e a comemoração em chave de elogio; e esses
três elementos permeiam a construção das canções corais, refletindo"-se
em sua linguagem altamente elaborada em registro elevado, em seu forte
componente mítico, no canto autodramático do coro, e nas máximas de tom moral
e ético que pontuam seus versos, a (re)validar e reiterar valores e
ensinamentos compartilhados pela comunidade, pela audiência, pelos
\textit{performers}, pelo poeta cuja voz ganha dimensão mais
pública do que privada, na medida em que sua poesia tem papel vital na autoconsciência 
pública da cidade.

A partir disso, entendemos que não há como entrar em contato com os fragmentos 
de poemas de Safo e de outros poetas da antiguidade grega sem ter conhecimento 
do panteão das divindades deste povo.

\Image{Representação da poetisa grega segurando a carta que vai enviar ao amante Phaon, antes de se matar. (Museu do Louvre; Domínio Público)}{PNLD0035-09}

\paragraph{Metodologia}

\begin{enumerate}

	\item
	Comece a aula perguntando aos estudantes o que eles conhecem sobre 
	a mitologia grega. Quais os deuses eles conhecem? Quais histórias,
	ou poderes de cada um, e como são suas aparências físicas? Faça um apanhado
	com as respostas e com os nomes que mais aparecerem. É importante, também,
	indicar, neste primeiro momento, a relação entre os deuses gregos e romanos,
	como Afrodite e Vênus, e Ares e Marte. Os estudantes devem saber que, uma escultura
	romana de Vênus foi feita seguindo o padrão das esculturas gregas de Afrodite,
	o que lhe confere um estatudo de equidade. 

	\item
	Num momento mais expositivo, apresente ao menos uma imagem de quadro ou escultura
	para cada um dos deuses do panteão clássico.  Junto a isso, apresente os poemas
	``Hino a Afrodite'', de Safo, ``Hino a Ártemis'' e ``Hino a Dioniso'' de Anacreonte.
	Chame a atenção às descrições físicas dessas divindidades realizadas pelos poetas
	e o correspondente nas representações imagéticas. Faça"-os perceber que há uma
	relação de equidade entre ambas as linguagens. 

 	``Hino a Ártemis''

	Abraço teus joelhos, ó caçadora de cervos
	loira filha de Zeus, senhora 
	das feras selvagens, ó Ártemis,
	que agora, em algum lugar perto
	das correntes do Leteu, regozijando-se,
	fixa os olhos na cidade dos homens
	de ousados corações, pois não são indomados
	os cidadãos que pastoreias...

	``Hino a Dioniso''

	Ó senhor com quem o domador Eros
	e também as Ninfas de escuros olhos 
	e a purpúrea Afrodite
	brincam juntos, enquanto vagueias 
	pelos altivos picos das montanhas:
	os joelhos abraço-te, e tu, propício,
	vem a nós, e aceitável
	prece escuta:
	a Cleóbulo sê bom
	conselheiro, para ele minha paixão,
	ó Dioniso, aceitar. 

	\Image{``Baco e Midas'' (1629), Nicolas Poussin. Localizado na Pinacoteca de Munique.}{PNLD0035-15}

	\item
	Utilizando um recurso educacional aberto como o Kahoot, prepare uma atividade onde os
	estudantes, de seus próprios aparelhos eletrônicos, celulares ou compuadores, irão 
	associar a imagem apresentada à legenda correta, neste caso, o nome do deus ou deusa 
	da mitologia grega. 

\end{enumerate}

\paragraph{Tempo estimado} Duas aulas de 50 minutos.




\subsection{Pós-leitura}

\paragraph{Tema} Por que cantamos aos deuses?

\paragraph{Conteúdo} Discussão sobre o papel assumido pela deusa Afrodite no hino que 
Safo dedica a ela e comparação com a realidade cultural do mundo ocidental moderno.
Também nós cantamos aos nossos deuses?

\paragraph{Objetivo} Estimular e habilitar os estudantes a apreender a relação
entre os cantos e hinos a deuses do panteão grego nos hinos de Safo e Anacreonte 
e as necessidades humanas que justificam as súplicas realizadas em cada uma
dessas obras. 

\paragraph{Justificativa} Nenhuma outra divindade grega aparece nas canções de Safo
com a mesma frequência, nem com a mesma importância que Afrodite, o que se explica
quando percebemos que as três principais linhas de força da mélica sáfica são:
a paixão erótica, a beleza e o universo feminino, justamente os campos de atuação da deusa.

Ora, a partir disso podemos constatar que Safo canta a Afrodite pois os assuntos
que lhe interessam e que lhe inspiração a composição são justamente os preferidos
da deusa. 

Quando pensamos na cultura do mundo ocidental contemporâneo, no Brasil mais
especificamente, fica difícil fazer uma associação básica entre Afrodite e apenas
um deus, deusa ou divindade correspondente. Primeiro, porque a religiosidade
tem um papel bem diferente em nossa sociedade em relação à sociedade grega
arcaica. Depois, porque somos povos bem mais complexos e multiculturais do que 
o povo grego àquele momento. No entanto, dentro desta vastidão de panteões,
encontramos o mesmo mecanismo em boa parte dessas expressões culturais: 
quando precisamos de ajuda divina num campo específico de nossas vidas, 
cantamos, rezamos ou pedimos às divindades. No caso do catolicismo, os fiéis
que precisem de ajuda na saúde, entoarão canticos a um santo ou santa que
tenha esta especialidade; no amor, será outra a divindade. O mesmo acontece
com outras religiosidades, as de matriz africana, por exemplo, onde cada
orixá irá reger um aspecto da vida humana e, consequentemente, auxiliar 
aqueles que lhes pedem ajuda também com cantos e outras formas de entrar
em contato com eles.

A partir da leitura do ``Hino a Afrodite'', percebemos que Safo clama
à deusa pois sente ``angústias'' em seu coração devido a um amor
que deseja alcançar, e para isso conta com a ajuda, já recorrente,
de Afrodite, especialista neste assunto. 

\Image{Pintura a óleo de Safo e Alceu, 1881. (Google Art Project; Domínio Público)}{PNLD0035-10}

\paragraph{Metodologia}

\begin{enumerate}

	\item
	Retomando os outros dois poemas trabalhados na última aula, os hinos a 
	Ártemis e Dioniso, além do Hino a Afrodite, peça que os estudantes 
	determinem, a partir dos poemas, para a resolução de quais tipos de 
	problemas os humanos clamam a estes deuses e deusas. 

	\item
	Agora, como proposta de atividade final, peça que os estudantes, articulando
	os conhecimentos apreendidos durante as aulas sobre contexto histórico e cultural
	da sociedade grega arcaica, além das leituras dos poemas, que eles escrevam 
	uma história de um grego do período arcaico que precise da ajuda de alguma
	divindade. Pode ser no formato de uma carta em primeira pessoa, assumindo o lugar
	desta personagem, ou um relato em terceira pessoa, como se contanto a história
	de um indivíduo de uma época e sociedade distantes. É importante que as informações 
	sobre o contexto histórico e social estejam presentes, como o lugar da Grécia antiga 
	onde esta pessoa vive, sua condição na sociedade, e a divindade à qual ela deverá
	prestar homenagens. 
	Ao fim da atividade, os estudantes devem compartilhar suas histórias por meio de um
	\textit{blog} criado pela turma ou alguma das plataformar virtuais de redes sociais,
	onde todos possam entrar em contato.


\end{enumerate}

\paragraph{Tempo estimado} Duas aulas de 50 minutos.




\section{Aprofundamento}


% Ao chegar ao Ensino Médio, é necessário que os estudantes se aprofundem
% na compreensão das múltiplas linguagens e, sobretudo, da linguagem
% literária. Em relação à literatura, a BNCC traz as seguintes
% considerações:

% ``{[}...{]} a leitura do texto literário, que ocupa o centro do trabalho
% no Ensino Fundamental, deve permanecer nuclear também no Ensino Médio.
% Por força de certa simplificação didática, as biografias de autores, as
% características de épocas, os resumos e outros gêneros artísticos
% substitutivos, como o cinema e as HQs, têm relegado o texto literário a
% um plano secundário do ensino. Assim, é importante não só (re)colocá-lo
% como ponto de partida para o trabalho com a literatura, como
% intensificar seu convívio com os estudantes. Como linguagem
% artisticamente organizada, a literatura enriquece nossa percepção e
% nossa visão de mundo. Mediante arranjos especiais das palavras, ela cria
% um universo que nos permite aumentar nossa capacidade de ver e sentir.
% Nesse sentido, a literatura possibilita uma ampliação da nossa visão do
% mundo, ajuda-nos não só a ver mais, mas a colocar em questão muito do
% que estamos vendo/vivenciando.'' (Brasil, 2018, p. 491)

% Nesta seção, desenvolvemos um trabalho de aprofundamento que, em diálogo
% com a formação continuada de professores, oferece subsídios para a
% abordagem do texto literário.

% Os versos de Safo foram imitados por poetas gregos e romanos. Ela foi
% chamada de ``a décima musa'' e se tornou referência para poetas homens
% de toda a Antiguidade.

Pouco se sabe sobre a vida de Safo. Conhecemos fragmentos de sua obra.
Na ilha em que nasceu, mantinha uma escola só para mulheres. Muitos outros 
dados sobre sua vida podem ser colhidos nos testemunhos antigos. 
Vistos de perto, porém, eles se mostram demasiado frágeis,
contraditórios, anedóticos, quase como peças de uma biografia ficcional,
sempre em reconstrução, baseada no que nos restou da obra sáfica.

Safo compôs versos celebrando o amor e tornou-se famosa por poemas
cantados ao som da lira. Na tradição oral da poesia antiga, a
performance era coral e não havia a visão intimista que temos da lírica
hoje.

A oralidade marca profundamente a circulação dessa poesia, muito ligada
à vida cotidiana da comunidade. Por isso, a voz nesse gênero de poema 
está sempre em diálogo: em vez do ``eu lírico'' falando consigo mesmo, 
aqui a voz se dirige sempre ao outro, ao interlocutor. A oralidade precisa 
da memória e, mesmo que a escrita já existisse na Grécia no tempo de Safo, 
a fala era o principal meio de circulação dos poemas. Suas composições são 
basicamente canções, composições literárias e musicais, com sonoridade ritmada. 

Nos poemas desta coleção, há a presença de Eros. Longe de nomear o amor
romântico, ele designa a força controlada pela deusa Afrodite. Mas Afrodite 
é quem toma de assalto sua vítima e se apodera de seu corpo e de sua
mente, como uma invasão, uma doença, uma insanidade, um animal
selvagem, um desastre natural. Esses sentimentos, presentes no poema, 
descrevem o ser humano tomado pelo amor e pelo desejo.


Para finalizar, podemos lembrar que Safo fazia parte de uma comunidade 
formada por uma mulher adulta e por meninas aguardando o casamento com
seus noivos. Os epitalâmicos eram canções ligadas justamente a estas ocasiões,
como o banquete inicial na casa do pai da noiva, os sacrifícios aos deuses do casamento 
durante a festa, e a apresentação da noiva ao noivo. As dança, cantos, brincadeiras
e tochas procuravam diminuir a tensão dos noivos, geralmente desconhecidos.


\section{Referências complementares}

\begin{itemize}
\item \textsc{funari}, Pedro Paulo. Grécia e Roma. São Paulo: Editora Contexto,
2001.

Este livro é uma boa porta de entrada para os estudos de antiguidade
clássica. Ele segue uma sequência clara e trata de aspectos essenciais
do período.

\item \textsc{vernant}, Jean-Pierre. O universo, os deuses, os homens. São
Paulo: Companhia das Letras, 2005.

Como um avô que conta histórias para o neto, o autor narra episódios
clássicos como a luta de Zeus contra os Titãs, a fuga de Ulisses das
sereias e o roubo do fogo do Olimpo por Prometeu.

\item \textsc{vieira}, Trajano. Lírica grega hoje. São Paulo: Editora
Perspectiva, 2020.

Consagrado tradutor, o autor traz fragmentos de poetas da Grécia
antiquíssima, como Álcman, Alceu, Safo, Estesícoro, Íbico, Anacreonte e
Simônides.
\end{itemize}
\subsection{Filme}

\begin{itemize}

\item O feitiço de Áquila. Direção: Richard Donner. (Alemanha, 1985).

Na Europa do século XII, o bispo Áquila fica enciumado ao saber que sua
amada está apaixonada por outro. Ele lança uma maldição sobre o casal:
de dia ela sempre será um falcão e de noite ele será um lobo.
\end{itemize}

\subsection{Lugar para visitar}

\begin{itemize}
\item Museu de Arqueologia e Etnologia (http://mae.usp.br/)

O site reúne notícias, ferramenta de busca em acervo, informações sobre
exposições, cursos e outros assuntos de interesse da área.
\end{itemize}

\section{Bibliografia comentada}

\begin{itemize}

\item \textsc{achcar}, Francisco. Lírica e lugar-comum. São Paulo: Edusp,
2015.

A partir da análise meticulosa de poemas, o autor examina a relação
entre os motivos da fugacidade e da perenização, bem como as convenções
de gênero em que se integram.

\item \textsc{budelmann}, F. (org.). The Cambridge Companion to Greek Lyric.
Cambridge: Cambridge University Press, 2009.

A obra fornece uma introdução acessível à lírica grega e sua recepção
posterior, levando em conta novos achados de papiro e novas abordagens
críticas.

\item \textsc{fontes}, Joaquim Brasil. Eros, tecelão de mitos. São Paulo:
Iluminuras, 2000.

Neste ensaio sobre a vida e obra da poetisa clássica, o autor realiza
também um estudo sobre o próprio saber clássico-humanístico.

\item \textsc{guerrero}, Gustavo. Teorías de la lírica. Cidade do México: Fondo
de Cultura Económica, 1998.

A partir de três ensaios, o autor traça a trajetória do gênero lírico,
apresentando ao leitor a evolução do conceito de poesia lírica, das
origens gregas à poética pré-romântica.

\item \textsc{jaeger}, Werner. Paideia: a formação do homem grego. São Paulo:
Martins Fontes, 2016.

O autor faz um estudo profundo sobre os ideais de educação da Grécia
antiga e mostra como foi possível chegar a um entendimento único da
criação educativa.

\item \textsc{lesky}, Albin. História da literatura grega. Lisboa: Fundação
Calouste Gulbenkian, 1995.

Esta obra é um bom ponto de partida para realizar pesquisas sobre
literatura grega, sobretudo acadêmicas. O autor levanta as principais
questões em torno de cada autor e dá extensa referência bibliográfica.

\item \textsc{ragusa}, Giuliana. Fragmentos de uma deusa: representação de
Afrodite na lírica de Safo. Campinas: Editora da Unicamp, 2005. 

A autora faz um denso estudo sobre a representação que Safo, a célebre
poetisa da ilha de Lesbos, faz da deusa Afrodite, revelada como um ser
complexo e multifacetado.

\item \textsc{ragusa}, Giuliana. Lira grega: antologia de poesia arcaica. São
Paulo: Hedra, 2014.

Em traduções diretas e inéditas, a autora apresenta os nove mais
importantes poetas gregos do Período Arcaico (séculos IX a VI a. C.),
que não escreviam para o papel, mas para a voz, o que fazia de seus
poemas verdadeiras canções.

\item \textsc{snell}, Bruno. A cultura grega e as origens do pensamento
europeu. São Paulo: Perspectiva, 2009.

Acessível e erudita, a obra reflete sobre as mais importantes etapas da
produção helênica e seu papel no processo de formação dos ditos espírito
e cultura da Europa.

\item \textsc{vernant}, Jean-Pierre. As origens do pensamento grego. Rio de
Janeiro: Bertrand Brasil, 2002.

O autor problematiza como a razão, mesmo atrelada ao mito, não apenas se
desvinculou dele, mas o ultrapassou, constituindo o que hoje conhecemos
como a Filosofia.
\end{itemize}

\end{document}

