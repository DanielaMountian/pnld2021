\documentclass[12pt]{extarticle}
\usepackage{manualdoprofessor}
\usepackage{fichatecnica}
\usepackage{lipsum,media9,graficos}
\usepackage[justification=raggedright]{caption}
\usepackage{bncc}
\usepackage[escoladepais]{../edlab}

\begin{document}


%\Image{Legenda}{PNLD0035-XX}

\newcommand{\AutorLivro}{Safo de Lesbos}
\newcommand{\TituloLivro}{Hino a Afrodite e outros poemas}
\newcommand{\Tema}{Ficção, mistério e fantasia}
\newcommand{\Genero}{Poema}
\newcommand{\imagemCapa}{./images/PNLD0035-01.png}
\newcommand{\issnppub}{---}
\newcommand{\issnepub}{---}
% \newcommand{\fichacatalografica}{PNLD0035-00.png}
\newcommand{\colaborador}{\textbf{Fulano de Tal} é uma pessoa incrível e vai fazer um bom serviço.}


\title{\TituloLivro}
\author{\AutorLivro}
\def\authornotes{\colaborador}

\date{}
\maketitle

\begin{abstract}
Este Manual tem como objetivo fornecer subsídios para o trabalho com a
obra literária \emph{Hino a Afrodite e outros poemas}, de Safo de Lesbos.

É difícil afirmar com certeza alguma coisa sobre a vida de Safo de Lesbos.
Convém se dizer que ela nasceu de uma família aristocrática em Êresos, atual região da
Grécia, na ilha de Lesbos, por volta de 630~a.C. Seu nome é famoso desde seu tempo entre 
os maiores poetas gregos, tendo sido referência para muitos deles. Suas composições fazem
parte de um dos gêneros mais importantes dessa poesia, a \textit{mélica} ou \textit{lírica}.
Safo é o único nome feminino no conjunto de poetas da Grécia antiga e se tornou famosa por 
compor versos celebrando o amor que eram cantados ao som da lira.

\emph{Hino a Afrodite e outros poemas} reune vinte e sete poemas de tradição oral
que sobreviveram ao tempo traduzidos e anotados por Giuliana Ragusa, especialista em 
poesia grega arcaica. Antes dos poemas, o professor contará com uma introdução sobre Safo, 
sua poesia e o contexto em que se produziu e circulou, além do gênero mélico, a fortuna 
crítica sobre a poeta, a transmissão de sua obra, e as outras poetas mulheres de que se 
tem notícia. Com isso, poderá facilitar a aproximação dos estudantes com os fragmentos 
dos poemas.

Feitas para serem executadas em simpósios, eventos públicos onde as pessoas bebiam vinho
e ouviam os poemas cantados, os poemas-canções de Safo aqui presentes carregam uma ideia 
de solenidade, caráter religioso, e de elogio. A mitologia grega, sobretudo Afrodite, a 
a deusa do Amor, estará presente em muitos momentos, como no caso do poema que dá nome 
à antologia, "Hino a Afrodite". 

Esperamos que, por meio das indicações propostas, o professor possa encontrar elementos 
para aproximar os estudantes do Ensino Médio dos elementos do universo sáfico na poesia grega arcaica!

\end{abstract}


\section{Introdução}

Olá! 
É com muita alegria que apresentamos a você “Hino a Afrodite e outros poemas”, 
de Safo de Lesbos. 
Você lerá poemas produzidos no período arcaico da literatura grega antiga.
Mélica ou lírica designava o verso que era cantado para a música e para a 
dança, e era composto de elementos de ritmos e tamanhos variados. A mélica era, 
na literatura grega arcaica, o poema lírico. Mas não o lírico que conhecemos no
mundo moderno, aquela poesia subjetiva onde o ``coração fala", mas lírica
pois vem de \textit{lýra}, palavra grega que deu em ``lira", o instrumento musical
que acompanhava a \textit{performance}.

Mito e realidade se misturam quando falamos em Safo de Lesbos. Primeiro porque
esse olhar biografizante sobre obras literárias é uma característica mais nossa,
moderna, do que da Grécia arcaica. Ainda que fosse comum o interesse pela pessoa
que enunciava as canções, não havia uma associação direta, como tornou-se costume
fazer, entre o conteúdo do poema com a vida real da pessoa que o compôs.
Safo é o único nome feminino da primeira fase da poesia grega antiga,/ conhecida como fase arcaica/
esse período se estende de oitocentos a quatrocentos e oitenta antes de cristo//
safo nasceu em seiscentos e trinta antes de cristo,/ em uma família aristocrática/
ela viveu em mitilene e foi contemporanea do poeta alceu,/ que viveu na mesma região/
o que conhecemos da obra de safo são fragmentos/

há apenas um poema completo,/ dez fragmentos,/ uma centena de citações breves de autores antigos e cerca de cinquenta peças de textos em papiro,/ encontradas no egito//

no ano de 593 a.c., lesbos estava sob a ditadura de pitaco, instalado no poder pelos comerciantes e membros menos prósperos da sociedade. os aristocratas, inconformados com a perda do poder, tentam resgatá-lo, conspirando contra o estadista. mas eles são novamente subjugados e seus líderes são enviados para o exílio, entre eles safo e o poeta alceu, que se tomou de amores pela poetisa, a se levar em conta a ardente, mas puritana troca de cartas entre ambos. este falso recato ocultava a total ausência de pudor dos dois artistas, e a total liberalidade da jovem.
depois do exílio em pirra, safo é deportada para a sicília, bem distante de lesbos, pois pitaco sentia-se ameaçado por seus escritos. lá ela contrai matrimônio com um afortunado industrial de andros, com quem tem uma filha, cleis, e depois a deixa viúva e rica. após permanecer cinco anos fora de sua terra natal, ela retorna e assume a liderança da comunidade local na esfera cultural. elegante, bela e livre de compromissos, passa a viver sem preceitos morais, com absoluta liberdade.
 //
//
se safo não é a única poeta grega antiga,/ é a única do período arcaico/
aos que procuram a poesia de safo,/ restarão seus fragmentos,/ pulsantes há séculos// são os fragmentos dessa experiência estética que oferecemos nesta antologia//
 //
//
safo e alceu são nomes importantes de um gênero poético: a mélica/ suas canções eram destinadas à performance em solo,/  no simpósio,/ ou coral,/ nos festivais cívicos e religiosos/
os poemas eram acompanhados pela lira e por outros instrumentos musicais//
a poesia grega é de matriz oral e está ligada a uma “cultura da canção”/recitada ou cantada na performance,/ essa poesia transmitia ideias morais,/ políticas e sociais/
a coralidade da mélica (ou lírica) grega arcaica de safo é o aspecto que tem sido repensado e revalorizado para compreender sua poesia e de seu universo, e sobretudo, da relação entre a poeta e seu grupo de moças não casadas que viviam ao seu redor.
combinou-se a esse estimulante fato o enfoque intensificado na performance da mélica grega arcaica, e o resultado tem sido a avaliação mais precisa da produção de safo, e a ampliação da dist ncia das leituras biografizantes e romantizadas, que tanto buscam na poeta o que é estranho à sua poesia e, em verdade, à poesia antiga como um todo: a voz pessoal, o subjetivismo, o intimismo, a privacidade, os sentimentos confessos do “eu”. 

sob a ótica do romantismo e de uma visão rom ntica, a poesia de safo tem sido lida como uma expressão dos sentimentos íntimos da poeta. de acordo com esse modelo, safo aparece como uma solista, cantando suas canções e tocando sua lira em um espaço privado, num círculo pequeno de companheiras do mesmo sexo.

a propósito dessa ideia que por tanto tempo perdurou – e de certo modo, em certas leituras modernizantes vinculadas a questões de gênero e sexualidade, ainda perdura –, vale recordar, como bem afirma franco ferrari, que não há indicação da existência de composições poéticas da grécia arcaica e clássica, ao menos de homero até eurípides, que não fossem feitas tendo em vista um contexto específico, para uma situação convencional oral e para uma audiência definida. 

isso vale para a mélica de safo, ao contrário do que amiúde se imagina, por aproximação equívoca com a lírica moderna.
o diálogo está presente nos gêneros poéticos da grécia antiga,/ desde a épica de homero/ em safo,/ há um componente de diálogo que remete à cultura oral//
 
por trás do rótulo moderno de poesia lírica,/ os gêneros da poesia grega eram bem distintos/ 
 
na era arcaica,/ existiam a elegia,/ o jambo e a mélica/
a linguagem literária elabora artisticamente os sentimentos/  a experiência ou a emoção do poeta recebem um tratamento literário/
 
essa era uma poesia feita para a performance em viva voz/

no mundo de safo,/ como na grécia antiga em geral,/ o universo masculino e o feminino eram realidades separadas/

a linguagem e as imagens que safo emprega no tratamento da paixão são as mesmas que encontramos em arquíloco e em outros poetas homens,/ antes ou depois dela//

muitas canções eram executadas publicamente em simpósios ou banquetes/

o simpósio é,/ portanto,/ coletivo do ponto de vista do evento,/ mas restrito do ponto de vista da classe e do gênero que o frequentava/

relaxados e tomando vinho,/ os aristocratas gregos ouviam e cantavam poemas líricos,/ competindo uns com os outros/ poderia haver versos em que o “eu” poético fosse de uma mulher,/ admitindo-se,/ nesse contexto,/ a representação de um papel feminino// 

para além do festival,/ também podiam servir de ocasião,/ à performance da mélica coral,/ os grandes funerais e as grandes bodas,/ menos públicos e mais privados/

estavam presentes a ideia da solenidade,/ o caráter religioso e a comemoração em forma de elogio/

esses tres elementos notáveis aparecem nas canções corais,/ de linguagem elaborada e registro elevado/

havia também um forte componente mitológico e máximas morais e éticas do coro/

as composições de safo são basicamente canções,/ composições literárias e musicais,/ com sonoridade ritmada//

 //
//
 
para mais informações sobre as habilidades trabalhadas nesta obra,/ consulte o material digital do professor. obrigado e até a próxima!


\section{Proposta de atividades I}


\subsection{Pré-leitura}

%De olho na BNCC}

% \textbf{As atividades sugeridas nesta subseção contemplam as seguintes
% habilidades propostas pela BNCC:}

% (EM13LGG302) Compreender e posicionar-se criticamente diante de diversas
% visões de mundo presentes nos discursos em diferentes linguagens,
% levando em conta seus contextos de produção e de circulação.

% (EM13LGG704) Apropriar-se criticamente de processos de pesquisa e busca
% de informação, por meio de ferramentas e dos novos formatos de produção
% e distribuição do conhecimento na cultura de rede.

% (EM13LP10) Selecionar informações, dados e argumentos em fontes
% confiáveis, impressas e digitais, e utilizá-los de forma referenciada,
% para que o texto a ser produzido tenha um nível de aprofundamento
% adequado (para além do senso comum) e contemple a sustentação das
% posições defendidas.

% (EM13LP19) Compartilhar gostos, interesses, práticas culturais, temas/
% problemas/questões que despertam maior interesse ou preocupação,
% respeitando e valorizando diferenças, como forma de identificar
% afinidades e interesses comuns, como também de organizar e/ou participar
% de grupos, clubes, oficinas e afins.\strut
% \end{minipage}\tabularnewline
% \bottomrule
% \end{longtable}

% Falar o que é poesia lírica: 
% Olhar texto inspirado no João Angelo
% https://github.com/hedra-editora/cabalat_shabat/blob/master/TEXTO.tex
% Música rap (para jâmbico) 
% Bolero (elegia) 
% Lírica 

\textbf{1.}
"Hino a Afrodite e outros poemas" reúne textos traduzidos e anotados remanescentes
da mélica grega arcaica, ou seja, das canções para \textit{performance} 
ao som da lira. Aliás, é daí que vem o nome ``lírica'' para se referir 
a poesia --- os poemas cantados ao som da \textit{lýra}, em grego.

O que o mundo moderno chama de poesia lírica é não tem muito a ver com
o jeito que ela era visto na época de Safo. Foi sobretudo no auge do
Romantismo europeu, no século \textsc{xviii} que poesia lírica passou a ser
associada à composição que expressava o \texit{cri du coeur}, ou seja,
o grito do coração, da alma do poeta. Quanto mais subjetivo e autêntico
em relação às suas emoções, melhor era sua qualidade enquanto poema
lírico. O que o poema dizia era o que o poeta sentia. 

Muito dessa concepção romântica de poesia lírica resiste até nossos dias
de hoje, apesar dos esforços de movimentos literários posteriores ao Romantismo
em deixar ela para trás. Para termos uma experiência mais honesta com a mélica 
sáfica, então, teremos que levar em conta alguns aspectos desse momento da 
história da literatura.

Os poemas na Grécia arcaica tinham um propósito para serem feitos e apresentados. 
Eles só existentiam no momento da execução, cantada em solo ou em coral, declamada,
acompanhada de uma variedade de instrumentos de acordo com cada ocasião. 
A\textit{performance} poderia ser em situações privadas, como os simpósios, tipos 
de festas com bebidas e música/poesia, nos palácios aristocráticos, ou em situações 
públicas, como as celebrações de matrimônios ou os louvores a algum deus, como é o
caso de boa parte dos poemas sáficos. 

As produções da poesia se organizavam em quatro grandes divisões canônicas conforme
seu propósito: épica, jâmbico, elegia e lírica, sendo \emph{grosso modo} a épica 
para a guerra; o jâmbico para a disputa; a elegia para o lamento da morte ou amor 
não correspondido; e finalmente a lírica, para celebrar a vitalidade na sua multiplicidade.

Para melhor nos aproximarmos desse universo, vamos tentar entender, na nossa 
cultura dos dias de hoje, o que seriam exemplos de cada uma dessas divisões.
Para isso, depois de apresentadas essas informações, apresente os quatro excertos 
seguintes de textos ou canções atuais e peça que os alunos indiquem dentro de 
qual das divisões cada um melhor se encaixaria: épica, jâmbico, elegia ou lírica. 






\subsection{Leitura}

%De olho na BNCC}

% As atividades sugeridas nesta subseção contemplam as seguintes


% (EM13LGG103) Analisar, de maneira cada vez mais aprofundada, o
% funcionamento das linguagens, para interpretar e produzir criticamente
% discursos em textos de diversas semioses.

% (EM13LP02) Estabelecer relações entre as partes do texto, tanto na
% produção como na recepção, considerando a construção composicional e o
% estilo do gênero, usando/reconhecendo adequadamente elementos e recursos
% coesivos diversos que contribuam para a coerência, a continuidade do
% texto e sua progressão temática, e organizando informações, tendo em
% vista as condições de produção e as relações lógico-discursivas
% envolvidas (causa/efeito ou consequência; tese/argumentos;
% problema/solução; definição/exemplos etc.).

% (EM13LP48) Perceber as peculiaridades estruturais e estilísticas de
% diferentes gêneros literários (a apreensão pessoal do cotidiano nas
% crônicas, a manifestação livre e subjetiva do eu lírico diante do mundo
% nos poemas, a múltipla perspectiva da vida humana e social dos romances,
% a dimensão política e social de textos da literatura marginal e da
% periferia etc.) para experimentar os diferentes ângulos de apreensão do
% indivíduo e do mundo pela literatura.
% \end{quote}\strut
% \end{minipage}\tabularnewline
% \bottomrule
% \end{longtable}


% Relação entre pintura e poesia 
% Uma descrição de uma pintura poderia ser fonte de poesia (ecfrase)
% Escolher dois ou tres quadros para descreverem em poema

\textbf{2.} durante a leitura, explorem pinturas e estátuas retratando
eros e afrodite, as divindades antigas do amor. observem como a lírica
de safo se relaciona com a mitologia grega.



\subsection{Pós-leitura}

%De olho na BNCC}

% As atividades sugeridas nesta subseção contemplam as seguintes


% (EM13LGG102) Analisar visões de mundo, conflitos de interesse,
% preconceitos e ideologias presentes nos discursos veiculados nas
% diferentes mídias como forma de ampliar suas as possibilidades de
% explicação e interpretação crítica da realidade.

% (EM13LGG303) Debater questões polêmicas de relevância social, analisando
% diferentes argumentos e opiniões manifestados, para negociar e sustentar
% posições, formular propostas, e intervir e tomar decisões
% democraticamente sustentadas, que levem em conta o bem comum e os
% Direitos Humanos, a consciência socioambiental e o consumo responsável
% em âmbito local, regional e global.

% (EM13LGG402) Empregar, nas interações sociais, a variedade e o estilo de
% língua adequados à situação comunicativa, ao(s) interlocutor(es) e ao
% gênero do discurso, respeitando os usos das línguas por esse(s)
% interlocutor(es) e combatendo situações de preconceito linguístico.

% (EM13LGG703) Utilizar diferentes linguagens, mídias e ferramentas
% digitais em processos de produção coletiva, colaborativa e projetos
% autorais em ambientes digitais.

% (EM13LP13) Planejar, produzir, revisar, editar, reescrever e avaliar
% textos escritos e multissemióticos, considerando sua adequação às
% condições de produção do texto, no que diz respeito ao lugar social a
% ser assumido e à imagem que se pretende passar a respeito de si mesmo,
% ao leitor pretendido, ao veículo e mídia em que o texto ou produção
% cultural vai circular, ao contexto imediato e sócio histórico mais
% geral, ao gênero textual em questão e suas regularidades, à variedade
% linguística apropriada a esse contexto e ao uso do conhecimento dos
% aspectos notacionais (ortografia padrão, pontuação adequada, mecanismos
% de concordância nominal e verbal, regência verbal etc.), sempre que o
% contexto o exigir.

% (EM13LP14) Produzir e analisar textos orais, considerando sua adequação
% aos contextos de produção, à forma composicional e ao estilo do gênero
% em questão, à clareza, à progressão temática e à variedade linguística
% empregada, como também aos elementos relacionados à fala (modulação de
% voz, entonação, ritmo, altura e intensidade, respiração etc.) e à
% cinestesia (postura corporal, movimentos e gestualidade significativa,
% expressão facial, contato de olho com plateia etc.).

% (EM13LP28) Resumir e resenhar textos, com o manejo adequado das vozes
% envolvidas (do autor da obra e do resenhador), por meio do uso de
% paráfrases, marcas do discurso reportado e citações, para uso em textos
% de divulgação de estudos e pesquisas.

% (EM13LP29) Realizar pesquisas de diferentes tipos (bibliográfica, de
% campo, experimento científico, levantamento de dados etc.), usando
% fontes abertas e confiáveis, registrando o processo e comunicando os
% resultados, tendo em vista os objetivos colocados e demais elementos do
% contexto de produção, como forma de compreender como o conhecimento
% científico é produzido e apropriar-se dos procedimentos e dos gêneros
% textuais envolvidos na realização de pesquisas.

% (EM13LP52) Produzir apresentações e comentários apreciativos e críticos
% sobre livros, filmes, discos, canções, espetáculos de teatro e dança,
% exposições etc. (resenhas, vlogs e podcasts literários e artísticos,
% playlists comentadas, fanzines, e-zines etc.).\strut
% \end{minipage}\tabularnewline
% \bottomrule
% \end{longtable}

% Quatro pinturas e exercicio criativo de criacao de poemas atraves de uma descrição

\textbf{3.} depois da leitura, sugira exercícios de escrita criativa de
poemas líricos, unindo o repertório clássico e as concepções atuais de
lirismo.

\section{Proposta de atividades II}


As obras de Safo possibilitam trabalhos interdisciplinares e
integradores de diferentes campos do saber e áreas de conhecimento. A
seguir, propomos algumas atividades que podem ser desenvolvidas
conjuntamente com professores de outras áreas. Além das habilidades de
Linguagens e suas Tecnologias e de Língua Portuguesa, indicadas nas
etapas da seção anterior e válidas também para esta, listamos a seguir
as habilidades de outras áreas, presentes na abordagem interdisciplinar:

% %De olho na BNCC}

% As atividades sugeridas nesta subseção contemplam as seguintes

% (EM13CNT201) Analisar e utilizar modelos científicos, propostos em
% diferentes épocas e culturas para avaliar distintas explicações sobre o
% surgimento e a evolução da Vida, da Terra e do Universo.

% (EM13CNT303) Interpretar textos de divulgação científica que tratem de
% temáticas das Ciências da Natureza, disponíveis em diferentes mídias,
% considerando a apresentação dos dados, a consistência dos argumentos e a
% coerência das conclusões, visando construir estratégias de seleção de
% fontes confiáveis de informações.

% (EM13CHS101) Analisar e comparar diferentes fontes e narrativas
% expressas em diversas linguagens, com vistas à compreensão e à crítica
% de ideias filosóficas e processos e eventos históricos, geográficos,
% políticos, econômicos, sociais, ambientais e culturais.

% (EM13CHS102) Identificar, analisar e discutir as circunstâncias
% históricas, geográficas, políticas, econômicas, sociais, ambientais e
% culturais da emergência de matrizes conceituais hegemônicas
% (etnocentrismo, evolução, modernidade etc.), comparando-as a narrativas
% que contemplem outros agentes e discursos.

% (EM13CHS106) Utilizar as linguagens cartográfica, gráfica e iconográfica
% e de diferentes gêneros textuais e as tecnologias digitais de informação
% e comunicação de forma crítica, significativa, reflexiva e ética nas
% diversas práticas sociais (incluindo as escolares) para se comunicar,
% acessar e disseminar informações, produzir conhecimentos, resolver
% problemas e exercer protagonismo e autoria na vida pessoal e coletiva.

% (EM13CHS401) Identificar e analisar as relações entre sujeitos, grupos e
% classes sociais diante das transformações técnicas, tecnológicas e
% informacionais e das novas formas de trabalho ao longo do tempo, em
% diferentes espaços e contextos.\strut
% \end{minipage}\tabularnewline
% \bottomrule
% \end{longtable}

\subsection{Pré-leitura}

% Trazer alguma explicação histõrica do texto da Raguzza

\textbf{4.} na pré-leitura, professores de ciencias humanas podem
contextualizar o período arcaico da grécia/ construam mapas, linhas do
tempo e pesquisem informações sobre o contexto histórico de safo/

% Arrumar um site na internet 
% duas ou tres páginas

\subsection{Leitura}

% História e Geografia 

%??????????????????????????????????

\textbf{5.} durante a leitura, comparem a figura de afrodite com outras
personagens da mitologia grega/ construam um jogo da memória com as
personagens pesquisadas/


\subsection{Pós-leitura}

\textbf{6.} discutam também o papel assumido por afrodite no hino que
safo dedica à deusa/


nenhuma outra divindade grega aparece nas canções de safo com a mesma
frequência, nem do mesmo modo: afrodite é a mais presente/ o fato não se
explica facilmente, mas três são as linhas de força da mélica sáfica
são a paixão erótica, a beleza e o universo feminino/

\section{Aprofundamento}


Ao chegar ao Ensino Médio, é necessário que os estudantes se aprofundem
na compreensão das múltiplas linguagens e, sobretudo, da linguagem
literária. Em relação à literatura, a BNCC traz as seguintes
considerações:

``{[}...{]} a leitura do texto literário, que ocupa o centro do trabalho
no Ensino Fundamental, deve permanecer nuclear também no Ensino Médio.
Por força de certa simplificação didática, as biografias de autores, as
características de épocas, os resumos e outros gêneros artísticos
substitutivos, como o cinema e as HQs, têm relegado o texto literário a
um plano secundário do ensino. Assim, é importante não só (re)colocá-lo
como ponto de partida para o trabalho com a literatura, como
intensificar seu convívio com os estudantes. Como linguagem
artisticamente organizada, a literatura enriquece nossa percepção e
nossa visão de mundo. Mediante arranjos especiais das palavras, ela cria
um universo que nos permite aumentar nossa capacidade de ver e sentir.
Nesse sentido, a literatura possibilita uma ampliação da nossa visão do
mundo, ajuda-nos não só a ver mais, mas a colocar em questão muito do
que estamos vendo/vivenciando.'' (Brasil, 2018, p. 491)

Nesta seção, desenvolvemos um trabalho de aprofundamento que, em diálogo
com a formação continuada de professores, oferece subsídios para a
abordagem do texto literário.

os versos de safo foram imitados por poetas gregos e romanos. ela foi
chamada de ``a décima musa'' e se tornou referência para poetas homens
de toda a antiguidade.

pouco se sabe sobre a vida de safo. conhecemos fragmentos de sua obra.na ilha em que nasceu, mantinha uma escola só para mulheres.

muitos outros dados sobre sua vida podem ser colhidos nos testemunhos
antigos. vistos de perto, porém, eles se mostram demasiado frágeis,
contraditórios, anedóticos, quase como peças de uma biografia ficcional,
sempre em reconstrução, baseada no que nos restou da obra sáfica.

safo compôs versos celebrando o amor e tornou-se famosa por poemas
cantados ao som da lira. na tradição oral da poesia antiga, a
performance era coral e não havia a visão intimista que temos da lírica
hoje..

a oralidade marca profundamente a circulação dessa poesia, muito ligada
à vida cotidiana da comunidade.

por isso, a voz nesse gênero de poema está sempre em diálogo: em vez do
``eu lírico'' falando consigo mesmo, aqui a voz se dirige sempre ao
outro, ao interlocutor..

a oralidade precisa da memória e, mesmo que a escrita já existisse na
grécia no tempo de safo, a fala era o principal meio de circulação dos
poemas..

as composições de safo são basicamente canções,. composições literárias
e musicais,. com sonoridade ritmada..

nos poemas desta coleção,. há a presença de éros. longe de nomear o amor
romântico,. designa a força controlada pela deusa afrodite.

nos poemas,. há a presença de éros. longe de nomear o amor romântico,.
designa a força controlada pela deusa afrodite.


afrodite toma de assalto sua vítima e se apodera de seu corpo e de sua
mente,. como uma invasão,. uma doença,. uma insanidade,. um animal
selvagem,. um desastre natural.


esses sentimentos,. presentes no poema,. descrevem o ser humano tomado
pelo amor e pelo desejo..


no mundo de safo,. como na grécia antiga em geral,. o universo masculino
e o feminino eram realidades separadas.


muitas canções eram executadas publicamente em simpósios ou banquetes.


o simpósio é,. portanto,. coletivo do ponto de vista do evento,. mas
restrito do ponto de vista da classe e do gênero que o frequentava.


relaxados e tomando vinho,. os aristocratas gregos ouviam e cantavam
poemas líricos,. competindo uns com os outros. poderia haver versos em
que o ``eu'' poético fosse de uma mulher,. admitindo-se,. nesse
contexto,. a representação de um papel feminino..

para além do festival,. também podiam servir de ocasião,. à performance
da mélica coral,. os grandes funerais e as grandes bodas,. menos
públicos e mais privados.


estavam presentes a ideia da solenidade,. o caráter religioso e a
comemoração em forma de elogio.


esses tres elementos notáveis aparecem nas canções corais,. de linguagem
elaborada e registro elevado.


havia também um forte componente mitológico e máximas morais e éticas do
coro.


no caso de safo,. existe uma comunidade formada por uma mulher adulta e
por meninas aguardando o casamento com seus noivos.


os epitalâmios eram composições ligadas a casamentos e revelavam
características da cerimônia: o banquete inicial na casa do pai da
noiva,. os sacrifícios aos deuses do casamento durante a festa,. a
apresentação da noiva ao noivo.


as danças,. cantos,. brincadeiras e tochas procuravam diminuir a tensão
dos noivos,. geralmente desconhecidos.

as composições de safo são basicamente canções,. composições literárias
e musicais,. com sonoridade ritmada..


nos poemas desta coleção,. há a presença de éros. longe de nomear o amor
romântico,. designa a força controlada pela deusa afrodite.


os versos de safo foram imitados por poetas gregos e romanos. ela foi
chamada de ``a décima musa'' e se tornou referência para poetas homens
de toda a antiguidade.


\section{Referências complementares}

\textbf{FUNARI, Pedro Paulo. Grécia e Roma. São Paulo: Editora Contexto,
2001.}

Este livro é uma boa porta de entrada para os estudos de antiguidade
clássica. Ele segue uma sequência clara e trata de aspectos essenciais
do período.

\textbf{VERNANT, Jean-Pierre. O universo, os deuses, os homens. São
Paulo: Companhia das Letras, 2005.}

Como um avô que conta histórias para o neto, o autor narra episódios
clássicos como a luta de Zeus contra os Titãs, a fuga de Ulisses das
sereias e o roubo do fogo do Olimpo por Prometeu.

\textbf{VIEIRA, Trajano. Lírica grega hoje. São Paulo: Editora
Perspectiva, 2020.}

Consagrado tradutor, o autor traz fragmentos de poetas da Grécia
antiquíssima, como Álcman, Alceu, Safo, Estesícoro, Íbico, Anacreonte e
Simônides.

\textbf{Filme}:

\textbf{O feitiço de Áquila. Direção: Richard Donner. (Alemanha, 1985).}

Na Europa do século XII, o bispo Áquila fica enciumado ao saber que sua
amada está apaixonada por outro. Ele lança uma maldição sobre o casal:
de dia ela sempre será um falcão e de noite ele será um lobo.

\textbf{Lugar para visitar:}

\textbf{Museu de Arqueologia e Etnologia} (http://mae.usp.br/)

O site reúne notícias, ferramenta de busca em acervo, informações sobre
exposições, cursos e outros assuntos de interesse da área.

\textbf{Organizando a estante:} Bibliografia comentada

\textbf{ACHCAR, Francisco. Lírica e lugar-comum. São Paulo: Edusp,
2015.}

A partir da análise meticulosa de poemas, o autor examina a relação
entre os motivos da fugacidade e da perenização, bem como as convenções
de gênero em que se integram.

\textbf{BUDELMANN, F. (org.). The Cambridge Companion to Greek Lyric.
Cambridge: Cambridge University Press, 2009.}

A obra fornece uma introdução acessível à lírica grega e sua recepção
posterior, levando em conta novos achados de papiro e novas abordagens
críticas.

\textbf{FONTES, Joaquim Brasil. Eros, tecelão de mitos. São Paulo:
Iluminuras, 2000.}

Neste ensaio sobre a vida e obra da poetisa clássica, o autor realiza
também um estudo sobre o próprio saber clássico-humanístico.

\textbf{GUERRERO, Gustavo. Teorías de la lírica. Cidade do México: Fondo
de Cultura Económica, 1998.}

A partir de três ensaios, o autor traça a trajetória do gênero lírico,
apresentando ao leitor a evolução do conceito de poesia lírica, das
origens gregas à poética pré-romântica.

\textbf{JAEGER, Werner. Paideia: a formação do homem grego. São Paulo:
Martins Fontes, 2016.}

O autor faz um estudo profundo sobre os ideais de educação da Grécia
antiga e mostra como foi possível chegar a um entendimento único da
criação educativa.

\textbf{LESKY, Albin. História da literatura grega. Lisboa: Fundação
Calouste Gulbenkian, 1995.}

Esta obra é um bom ponto de partida para realizar pesquisas sobre
literatura grega, sobretudo acadêmicas. O autor levanta as principais
questões em torno de cada autor e dá extensa referência bibliográfica.

\textbf{RAGUSA, Giuliana. Fragmentos de uma deusa: representação de
Afrodite na lírica de Safo. Campinas: Editora da Unicamp, 2005. }

A autora faz um denso estudo sobre a representação que Safo, a célebre
poetisa da ilha de Lesbos, faz da deusa Afrodite, revelada como um ser
complexo e multifacetado.

\textbf{RAGUSA, Giuliana. Lira grega: antologia de poesia arcaica. São
Paulo: Hedra, 2014.}

Em traduções diretas e inéditas, a autora apresenta os nove mais
importantes poetas gregos do Período Arcaico (séculos IX a VI a. C.),
que não escreviam para o papel, mas para a voz, o que fazia de seus
poemas verdadeiras canções.

\textbf{SNELL, Bruno. A cultura grega e as origens do pensamento
europeu. São Paulo: Perspectiva, 2009.}

Acessível e erudita, a obra reflete sobre as mais importantes etapas da
produção helênica e seu papel no processo de formação dos ditos espírito
e cultura da Europa.

\textbf{VERNANT, Jean-Pierre. As origens do pensamento grego. Rio de
Janeiro: Bertrand Brasil, 2002.}

O autor problematiza como a razão, mesmo atrelada ao mito, não apenas se
desvinculou dele, mas o ultrapassou, constituindo o que hoje conhecemos
como a Filosofia.


\end{document}

