\documentclass[12pt]{extarticle}
\usepackage{manualdoprofessor}
\usepackage{fichatecnica}
\usepackage{lipsum,media9,graficos}
\usepackage[justification=raggedright]{caption}
\usepackage[one]{bncc}
\usepackage[maisemelhores]{../edlab}


%Gerais
% \BNCC{EM13CHS102} % Circunstâncias históricas e pol
% \BNCC{EM13CHS101} % Humanidades
% \BNCC{EM13CHS401} % Novas tecnologia


\begin{document}


\newcommand{\AutorLivro}{Julia Lopes de Almeida}
\newcommand{\TituloLivro}{Contos e novelas}
\newcommand{\Tema}{Ficção, mistério e fantasia}
\newcommand{\Genero}{Conto e novela}
\newcommand{\imagemCapa}{./images/PNLD0002-01.png}
\newcommand{\issnppub}{---}
\newcommand{\issnepub}{---}
% \newcommand{\fichacatalografica}{PNLD0002-00.png}
\newcommand{\colaborador}{Rodrigo Jorge Ribeiro Neves}


\title{\TituloLivro}
\author{\AutorLivro}
\def\authornotes{\colaborador}

\date{}
\maketitle



\begin{abstract}
Este manual tem o objetivo de auxiliá-lo no desenvolvimento de práticas
pedagógicas que estabeleçam o diálogo entre a obra literária aqui
apresentada e os estudantes, de modo a ampliar não apenas a leitura do
texto em si, mas também a sua relação com o mundo.

O livro \emph{Contos e novelas} consiste em uma antologia de narrativas
curtas de Júlia Lopes de Almeida, extraídas de duas de suas obras,
\emph{Ânsia eterna} (1903) e \emph{A isca} (1922), em que a escritora
apresenta alguns dos principais elementos que caracterizam sua
literatura, como a presença do insólito, o lugar da mulher na sociedade
patriarcal, os conflitos familiares, as marcas da escravidão e os
contrastes sociais, políticos e econômicos, resultantes da modernização.

Júlia Lopes de Almeida é uma das escritoras brasileiras mais importantes
da virada do século XIX para o XX, mas infelizmente ainda é pouco lida
se comparada aos escritores, em face da invisibilidade sofrida pelas
escritoras mulheres na formação do nosso cânone literário. Portanto,
esta coletânea de suas narrativas curtas vem contribuir para que a
escritora ocupe o lugar que merece na história da nossa literatura e da
cultura brasileira, trazendo para a escola a reflexão e discussão de
temas ainda tão atuais e que certamente irão contribuir na formação dos
estudantes como leitores críticos da realidade.

Para isso, apresentamos aqui propostas de atividades, aprofundamento,
referências complementares e uma bibliografia comentada, a fim de que o
material possa ser útil nas suas aulas para estimular os estudantes a
desbravar um universo de possibilidades através de uma das escritoras
mais importantes da nossa literatura. Além disso, é ótimo trabalhar com
contos e novelas em sala de aula, pois são gêneros literários bastante
fecundos e que, pela sua forma curta, possibilitam a dinamização das
atividades e a exploração de uma variedade maior de temas para discussão
com os estudantes.

Aproveite bastante este material. Ele foi feito com muita dedicação e
carinho para você! Boa aula!
\end{abstract}

\tableofcontents


\section{Propostas de Atividades I}



\subsection{Pré-leitura}

\BNCC{EM13LGG704}
\BNCC{EM13LP30}
\BNCC{EM13LGG202}

\paragraph{Tema} A relação do conto e da novela com as narrativas produzidas nas redes sociais. 

\paragraph{Conteúdo} Compreensão das relações do conto e da novela,
enquanto gêneros literários, com as narrativas construídas nas redes
sociais pelos usuários, como, por exemplo, Twitter, Instagram, TikTok e
YouTube.

\paragraph{Objetivo} Estimular e habilitar os estudantes a reconhecer e
compreender os principais elementos e as estratégias estéticas e
narrativas do conto e da novela como gêneros literários, tendo em vista
o funcionamento das histórias que eles acompanham e/ou constroem nas
principais redes sociais.

\paragraph{Justificativa} Não é nenhuma novidade que as redes sociais e
outras plataformas do mundo virtual vêm ocupando cada vez mais o
cotidiano de todos nós, especialmente dos estudantes, que vêm crescendo
e se formando na chamada Web 2.0. Tendências, estilos e pontos de vista
são profundamente influenciados pelo contato dos jovens com essa
realidade (ou, seria multi-realidade?), conferindo novas maneiras de
perceber, sentir, se expressar e se posicionar no mundo.

Mas nem tudo é apenas novidade. Embora os meios em que as informações e
as experiências dos indivíduos tenham sido transformadas nas últimas
décadas, há elementos e estruturas que são mais antigos que os nossos
próprios avós. O escritor modernista Mário de Andrade, em resposta aos
críticos do suposto fim dos ``assuntos poéticos'' no modernismo, disse
que o amor ainda existia, só que passou a andar de automóvel.
Parafraseando o escritor paulista, diríamos, então, que o amor ainda
existe, mas agora chega pelo WhatsApp. Ou seja, há situações que nunca
mudam, apenas são atualizadas as formas como são apresentadas. As redes
sociais, ainda que sejam uma realidade completamente diferente daquela
que vivíamos há poucas décadas atrás, conservam práticas, pensamentos e
sensibilidades desde épocas imemoriais, refletindo a sociedade em suas
principais dimensões. A nossa necessidade de contar histórias é uma delas.

O conto é um gênero literário moderno e, pela sua extensão curta, pode
ser facilmente incorporado, por meio de seus componentes estruturais
narrativos, ao funcionamento da comunicação pelas redes sociais, não
apenas pela sua instantaneidade, mas também por se concentrar em
determinados indivíduos ou situações. A novela está entre o conto e o
romance em termos de extensão. Enquanto o primeiro se caracteriza pela
presença de poucos personagens, uma única ação, um só conflito e um
drama, além de uma limitação do tempo e do espaço, a novela é mais ampla
e plural quanto a esses elementos. Por isso, é fundamental que o
estudante, com a proposição e adequada mediação do professor, os
identifique e os analise, a partir da sua experiência como usuário das
principais redes sociais e outras plataformas do mundo virtual, para que
se familiarize e compreenda os principais aspectos do conto e suas
diversas possibilidades.

\paragraph{Metodologia} 

\begin{enumerate}
\item
Como ponto de partida, o professor pode
perguntar aos estudantes o que é um conto e o que é uma novela, sem se
preocupar com uma definição teórica ou determinada nesse momento. Anotar
na lousa as definições mais recorrentes dadas pelos participantes da aula
ou pedir para que eles as registrem e as mantenham para a etapa posterior
da discussão são boas opções. O mais importante, nessa etapa, é
elaborar, coletivamente, uma definição do gênero literário tendo em
vista as experiências de cada um. Ainda que ninguém tenha lido um conto
ou uma novela na vida (algo bem difícil de acontecer), em sentido
estrito, é fundamental que o educador os encoraje a dizer o que eles
entendem como um texto do gênero, pois, muitas vezes, os contatos se
deram por meio de gêneros fronteiriços ou híbridos, que incorporam os
procedimentos narrativos do conto ou da novela em suas estruturas. E
essa percepção é bastante útil na presente atividade, pois estimula os
estudantes a pensar no texto levando em conta não a sua forma acabada e
fechada, mas o seu processo de construção e os elementos estruturais que
o compõem, ou seja, da parte para o todo.

\item
É muito importante que, durante essa discussão, o educador procure
sempre exemplificar com textos, de preferência outros que não sejam
desta coletânea, a fim de que o processo de aprendizagem seja ampliado.
Claro que é imprescindível que a conversa se desenvolva com textos
conhecidos dos estudantes, principalmente levando em contas as sugestões
dadas por eles, mas não se deve perder, de maneira alguma, a
oportunidade de sugerir novas obras, incutindo, sempre, a curiosidade e
o gosto pela leitura.

\item
Após essa etapa introdutória, o professor pode sugerir aos estudantes
que tentem, a partir dos resultados anotados na conversa sobre a
definição dos gêneros conto e novela, relacionar cada item com os
recursos das redes sociais que eles conhecem e/ou utilizam. Naturalmente
as definições vão ser reformuladas de acordo com a plataforma comparada,
de maneira que elas se adequem às suas respectivas funcionalidades, por
isso, o educador precisa ficar atento para que não se perca de vista o
objetivo da atividade, que é reconhecer e compreender como os elementos
compositivos do conto e da novela, enquanto gêneros literários, podem
estar presentes nos recursos das redes sociais, ou seja, elas também
podem ser espaços de narração de histórias curtas.

\item
Peça para que os estudantes formem grupos ou duplas e proponha que
utilizem os recursos de qualquer rede social para narrar um conto, uma
novela ou um fragmento deles, se o texto escolhido for um pouco mais
complexo. Nesse caso, quem sabe o grupo ou dupla proponha o piloto de
uma série? Todas as propostas são bem-vindas. Na aula em que os
trabalhos vão ser apresentados, incentive que todos discutam os contos e
as novelas adaptados para as plataformas virtuais, apontando os
elementos que os aproximam dos gêneros literários e os que os afastam
e/ou os colocam em xeque, testando os limites não apenas da narrativa
curta, mas também das redes sociais como espaços de produção e
compartilhamento de histórias. Comece pelo mais básico em análise
literária, como enredo, personagens, tempo, ambiente e outros elementos
importantes que compõem uma narrativa ficcional, então, deixe que os
estudantes ampliem, a partir da recepção dos textos e do processo de
criação do grupo, para a adaptação da história.
\end{enumerate}

\paragraph{Tempo estimado} Duas aulas de 50 minutos.



\subsection{Leitura}

\BNCC{EM13LGG103}
\BNCC{EM13LP15}
\BNCC{EM13LP09}

\paragraph{Tema} Quem conta um conto, aumenta um ponto?

\paragraph{Conteúdo} Compreensão do conto como gênero literário,
identificando suas especificidades a partir das experiências individual
e coletiva de leitura.

\paragraph{Objetivo} Estimular e habilitar os estudantes, por meio da
leitura dos textos de Júlia Lopes de Almeida, a identificar o conto como
gênero literário por meio de suas principais características, mudanças e
possibilidades de criação.

\paragraph{Justificativa} Por ser de extensão mais curta que o romance, o
conto costuma ser visto, por alguns, como um gênero mais ``fácil'' de
ser praticado. Não necessariamente. As facilidades que podemos encontrar
em um conto são de outra ordem, como o tempo de dedicação à leitura, a
identificação de todos os personagens ou a utilização em atividades na
sala de aula. Mas o processo criativo que envolve a produção de um
conto, ainda que ele sirva de laboratório para a elaboração de uma obra
mais extensa, também requer domínio dos seus principais elementos
compositivos, aliás, como em toda criação literária.

É comum alguns escritores começarem suas carreiras com os textos de
narrativa curta antes de se dedicarem ao romance, por exemplo. No
entanto, isso não significa que um conto não tenha autonomia e não
apresente complexidade em seu desenvolvimento. Elas são apenas de outra
ordem. Para compreender isso, é fundamental que o educador apresente e
discuta com os estudantes os principais componentes que formam o que
podemos chamar de conto e como a sua estrutura se desenvolve para que
uma história seja narrada dentro de um espaço mais curto. Portanto, a
experiência que cada um tem com a leitura pode ser interessante,
ampliando a percepção dos estudantes acerca da gama de possibilidades
que o gênero pode comportar.

A pergunta que intitula esta atividade é, mais do que uma provocação, um
convite para entender o funcionamento do conto por meio de suas
características mais elementares e refletir sobre nossa relação com a
arte de contar histórias, pois é delas, segundo Eduardo Galeano, que
somos feitos.

\paragraph{Metodologia} 

\begin{enumerate}
\item
O professor pode iniciar a aula pedindo que os
estudantes falem sobre sua experiência com a leitura, seja de contos ou
de outro gênero. O próprio educador também deve contribuir com uma
história, talvez iniciando a roda de conversa, a fim de incentivar os
demais. Conte a ``história'' que o levou àquela história, ou seja,
descreva a origem do seu contato com o texto escolhido. Foi o presente
de alguém especial? Era seu aniversário ou algo circunstancial? Fazia
parte da bibliografia de uma determinada matéria da escola? Ou um
professor comentou, rapidamente, em aula e o texto despertou seu
interesse? A capa do livro chamou a sua atenção ou foram os títulos dos
contos? Você se recorda onde estava quando leu a história e o que ela
lhe provocou? Pense nestas e em outras questões relacionadas à
experiência de leitura, que não se resume ao ato de ler propriamente
dito, mas abarca também as condições em que ela se realiza. Dessa
maneira, você está inserindo os educandos no universo do que constitui a
arte de contar histórias, permitindo que a discussão sobre os elementos
específicos do conto, principalmente em sua forma moderna, seja ainda
mais proveitosa.

\item
Organize a turma em grupos com duas ou mais pessoas. Escolham um dos contos
da antologia organizada de Júlia Lopes de Almeida. Caso os estudantes
queiram propor a inclusão de outras narrativas curtas, fora do material
literário, deixe aberta a possibilidade, desde que eles façam uma
comparação ou mesclem com as histórias da coletânea.

\item
Cada grupo é livre para escolher a forma como apresentará o conto em
sala de aula. Pode ser por meio de leitura dramatizada, cartazes,
fotografias, música, mímica ou como fazem os contadores de histórias.
Não importa o meio que eles utilizem, mas a atenção aos principais
aspectos do conto, que serão discutidos pela turma ao final de cada
apresentação. Eles devem identificar os elementos que caracterizam a
história adaptada nos diversos formatos de apresentação e o que faz dela
um conto.

\item
Você pode orientá-los a observarem a construção das personagens, as
ações que a definem na história que está sendo narrada, o tempo e o
espaço em que elas ocorrem, os recursos narrativos e a elaboração dos
diálogos. Como cada grupo lidou com a adaptação desses elementos na
forma escolhida para apresentar o conto, quais foram as suas
dificuldades e por quê. A partir desta análise, alguns pontos em comum
devem aparecer, contribuindo para a apreensão dos principais componentes
de um conto.

\item
A atividade também deve ressaltar as transformações que ocorrem no
texto quando ele é submetido ao trabalho coletivo. É possível contar uma
história sem alterá-la? E por que ocorrem as modificações na forma como
cada um constrói o relato? A perspectiva do ficcionista que a compôs é
complementada ou deslocada com a leitura da história? O que essas
alterações revelam sobre o gênero conto e sobre o ato de leitura? É uma
chance de o educador abordar as relações entre a necessidade do ser
humano em contar e ouvir histórias e a dimensão social da atividade
proposta, buscando, por meio dessa experiência, não apenas a mera
distração ou entretenimento, mas a aprendizagem de novas maneiras de
enxergar a realidade e o fortalecimento dos laços que unem um grupo de
pessoas envolvido em algo comum.
\end{enumerate}

\paragraph{Tempo estimado} Duas aulas de 50 minutos.



\subsection{Pós-leitura}

\BNCC{EM13LGG303}
\BNCC{EM13LGG601}
\BNCC{EM13LP51}

\paragraph{Tema} Aproximações e diferenças do conto e da novela com
  outros gêneros em prosa 

\paragraph{Conteúdo} Compreensão das relações estruturais entre o conto, a
novela e os demais gêneros literários em prosa, como a crônica e o
romance, sem deixar de evidenciar a inserção de cada um na realidade
representada pela obra literária.

\paragraph{Objetivo} Estimular e habilitar os estudantes a perceber os
vínculos estruturais entre os principais gêneros literários em prosa, a
partir do conto e da novela, a fim de que compreendam as relações entre
a realidade e a forma que ela demanda para que seja representada.

\paragraph{Justificativa} A literatura, assim como toda obra de arte, é
representação da realidade social por meio da expressão subjetiva, a fim
de que possamos compreender essa realidade e, de alguma maneira,
transformá-la. Para isso, existe uma forma que busca corresponder às
questões suscitadas por ela. Às vezes, é um poema, uma canção, um
quadro; outras, uma performance, um vídeo, um romance, uma série de TV.
Claro que a formação e as preferências individuais do artista, moldadas
pelas suas origens e história de vida, vão determinar os meios que ele
irá escolher para expressar sua inquietação diante do mundo. No entanto,
o próprio mundo ao qual ele se debruça exigirá um olhar particular para
que seja alcançado. Dentro de cada obra de arte, há uma forma de
expressão; dentro de cada forma de expressão, um conjunto de elementos
que indicam o modo de funcionamento daquela linguagem para expressar o
mundo a ser representado.

A escolha de um gênero literário nunca é arbitrária. Há escritores que
produziram contos impressionantes e romances razoáveis, outros que
criaram romances brilhantes e contos não tão bons, há ainda aqueles que
foram geniais nos dois gêneros, como, por exemplo, Machado de Assis,
Guimarães Rosa e Clarice Lispector, três dos nossos maiores. As
fronteiras que dividem esses e outros gêneros em prosa vão além das mais
óbvias, como a extensão. Fica fácil saber a diferença entre uma crônica
e um romance tomando esse critério. Quando o aspecto estrutural não é
tomado como referência para estabelecer essa distinção, as fronteiras se
tornam ainda mais tênues. Tudo bem que uma novela costuma ser mais
extensa que um conto, mas há outros elementos que se confundem.

A discussão sobre essas relações pode ajudar o estudante a compreender o
projeto envolvido na concepção dos contos e das novelas de Júlia Lopes
de Almeida.

\paragraph{Metodologia} 

\begin{enumerate}
\item
A partir dos contos e das novelas selecionados
desta coletânea, peça para que os estudantes sugiram qual (ou quais)
poderia ser um bom material para romance e por quê. Quais elementos
contidos na narrativa curta escolhida podem ser ampliados para que se
desenvolvam em uma narrativa mais longa. São necessários mais
personagens? Quais e como eles caracterizariam esses personagens
adicionais? De que modo eles se relacionariam com os personagens do
conto ou da novela e em que medida estes seriam modificados para que
componham o romance? Eles podem também tentar investigar as relações
entre o conto, a novela e outros gêneros em prosa, como a crônica.

\item
Organize a turma em grupos e proponha uma discussão sobre a
intertextualidade entre os gêneros em prosa, considerando,
especialmente, o conto. Você pode continuar com o conto ou novela que
acabaram de analisar, ou com outro texto que considere mais adequado às
discussões que pretenda conduzir em sala de aula. Inicialmente, os
grupos podem fazer uma roda de conversa sobre o conto escolhido, como em
um clube de leitura, expondo os principais elementos da história. Depois
cada um pode apresentar uma proposta de outro gênero em prosa a partir
desse conto ou novela. A linguagem escolhida para a apresentação fica a
critério dos educandos. Não há limites para a imaginação! O importante é
convencer os demais que a história em formato de narrativa curta pode
ser transformada em um texto em prosa de outro gênero literário.

\item
A novela ``O laço azul'', por exemplo, pode servir de interessante
ponto de partida, pois ela já apresenta alguns elementos que podem ser
ampliados para a construção de um romance. Os conflitos das duas irmãs
gêmeas diante do noivado de uma delas e a partida do noivo para a
guerra, que passa se comunicar por cartas, são alguns episódios que
poderiam ser ampliados com outros espaços, tempos e conflitos, tornando a narrativa mais complexa.

\item
O conto ``O futuro presidente'', pela linguagem e pela temática,
também pode ser pensado como crônica. Uma mulher pobre, costureira,
cuida do seu filho recém-nascido enquanto trabalha. Ela reflete sobre a
vida difícil das pessoas de sua classe e sonha com um futuro melhor para
a criança. A descrição de uma cena cotidiana é feita em tons poéticos,
ao mesmo tempo em que situa sua crítica social à percepção das
desigualdades profundas existentes no país, matéria fértil para uma boa
crônica.
\end{enumerate}

Depois da atividade, você, professor, pode retomar as perguntas
iniciais. Os alunos podem responder em aula ou entregar por escrito, em
uma ou duas páginas no máximo. O importante é que todos participem!

\paragraph{Tempo estimado} Duas aulas de 50 minutos.

\section{Propostas de Atividades II}



\subsection{Pré-leitura}

\BNCC{EM13LP12}
\BNCC{EM13LP32}


\paragraph{Tema} As representações culturais da \emph{Belle Époque} carioca


\paragraph{Conteúdo} Pesquisa e estudo das representações culturais da
modernização do Rio de Janeiro no período da chamada \emph{Belle
Époque}. A atividade consiste na apresentação dos trabalhos por meio de
uma exposição física ou virtual.

\paragraph{Objetivo} Habilitar os estudantes a compreender os aspectos
sociais, políticos e culturais do Rio de Janeiro na virada do século XIX
para o XX, em um período chamado de \emph{Belle Époque} carioca.


\SideImage{Theatro Municipal do Rio de Janeiro (Elvis Boaventura; CC-BY-3.0)}{PNLD0002-08.png}


\paragraph{Justificativa} Entre o final do século XIX e início do XX, o
Rio de Janeiro, então capital federal, atravessava profundas e intensas
transformações urbanas, sociais e culturais, inspirada nas reformas da
capital francesa entre os anos de 1850 e 1870 promovidas pelo prefeito,
o barão de Haussmann. O prefeito carioca Pereira Passos foi um dos
principais responsáveis pelas mudanças por aqui, abrindo avenidas,
alargando ruas, construindo edificações culturais, como o Theatro
Municipal, e cuidando dos problemas sanitários e de saneamento.


\Image{Arco do Triunfo no Rio de Janeiro, final do século XIX. (Arquivo Nacional; Domínio Público)}{PNLD0002-05.png}


\Image{Palácio Itamaraty, Rio de Janeiro, final do século XIX (Arquivo Nacional; Domínio Público)}{PNLD0002-06.png}


\Image{Paço Municipal ou Intendência Municipal do Rio de Janeiro, final do século XIX. O prédio foi demolido para a abertura da Avenida Presidente Vargas na década de 1940. (Arquivo Nacional; Domínio Público)}{PNLD0002-07.png}


No entanto, essas mudanças também tiveram consequências sociais,
aprofundando desigualdades e contribuindo para o aumento da população
mais pobre, que ocupava os morros e os subúrbios. A literatura e as
outras artes produzidas no período buscaram representar essas
transformações, bem como os crescentes contrastes que as
acompanhavam. Lima Barreto foi um dos escritores que mais combativamente
apontou os problemas de uma cidade que se modernizava, mas excluía
grande parte da população de seus benefícios. Os registros feitos por
fotógrafos como Augusto Malta também são fonte importante para analisar
essas medidas. No caso dos textos de Júlia Lopes de Almeida, os
problemas são expostos pelas relações de seus personagens com suas
condições dentro da sociedade e do meio familiar, principalmente a
mulher e o negro, ainda presos às estruturas arcaicas de privação e
exploração.

\paragraph{Metodologia} 


\SideImage{Edifício Art Nouveau na cidade de Riga, capital da Letônia (Bruno Menetrier; Domínio Público)}{PNLD0002-09.png}


\SideImage{Bonde virado na Revolta da Vacina (Autor desconhecido; Domínio Público)}{PNLD0002-10.png}


\begin{enumerate}

\item
Apresente e discuta com os alunos os aspectos
históricos, sociais, políticos e culturais da \emph{Belle Époque} no Rio
de Janeiro e em outras capitais do país onde a modernização no período
foi igualmente importante e com mudanças estruturais profundas, como São
Paulo, Manaus e Belém. Faça uma breve exposição sobre as diferenças
entre esses processos e relacione as principais características da
reforma urbana liderada por Pereira Passos, no Rio de Janeiro.

\item
Um material interessante para ilustrar essa etapa da atividade é a
série de vídeos-documentários \emph{Rio de Janeiro da Belle Époque:
ciência, lazer e educação}, produzida pelo Laboratório História da
Ciência, do CEFET/RJ, disponível no YouTube. Com participação dos
alunos do colégio, os vídeos oferecem um bom panorama histórico da
cidade no período dessas transformações, com registros sonoros,
audiovisuais e fotográficos coletados de arquivos.

\item
Depois dessa breve exposição, organize as turmas em grupos e proponha
que escolham um tema sobre a \emph{Belle Époque}. Pode ser o nome de um
artista, de uma obra, de uma estética determinada, de uma tendência, de
uma área do conhecimento etc. A seguir, algumas sugestões temáticas que
podem servir como ideia inicial para a escolhe de outros
objetos de pesquisa:



\begin{itemize}
\item
  Imprensa e literatura
\item
  Indumentária
\item
  Surgimento dos subúrbios
\item
  O desenvolvimento técnico: fotografia e cinema
\item
  Migrações urbanas
\item
  Art Nouveau
\item
  Arquitetura cultural
\item
  Urbanismo e saneamento básico
\item
  A Revolta da Vacina
\end{itemize}



\item
Cada grupo deve montar uma exposição física ou virtual sobre o tema
escolhido. A metodologia precisa ser bem definida. Vai ser composta por
imagens ou vídeos? Ou ainda uma combinação dos dois? Vai haver
apresentação de alguns objetos, páginas de jornais, revistas ou
ilustrações? Exemplares de livros ou fac-símile de manuscritos?

\item
Depois de delimitar a metodologia de pesquisa e apresentação dos
resultados, eles devem dividir a exposição em seções, de maneira que as
partes se articulem como em uma história, afinal, uma exposição não é
apenas a presença pura e simples do objeto relacionado ao tema, mas uma
narrativa sobre ele. Então, eles precisam contar uma história!

\item
Os estudantes podem pesquisar tanto nos arquivos físicos quanto nos
virtuais. E são inúmeros! Na rede, eles podem consultar a Hemeroteca
Digital da Biblioteca Nacional, com praticamente todo o acervo de
periódicos mantidos pela instituição, que também disponibiliza imagens
em sua Brasiliana Fotográfica. As páginas de importantes arquivos no
país também presta esse serviço, como o Arquivo Nacional, assim como
instituições privadas, como Instituto Moreira Salles. A Biblioteca
Brasiliana Guita e José Mindlin, da Universidade de São Paulo, também
disponibiliza diversos periódicos e livros raros em seu acervo digital.
É preciso estar atento, pois algumas instituições cobram taxa para a reprodução
de imagens. E não se esqueça de avisá-los de sempre citar a fonte!

\item
Cada grupo deve produzir também um texto de curadoria da exposição, a
ser distribuído entre os colegas, informando o tema, a metodologia, os
itens expostos, os acervos consultados e uma análise sobre o conjunto em
diálogo com o tema escolhido.
\end{enumerate}

\paragraph{Tempo estimado} Quatro aulas de 50 minutos.



\subsection{Leitura I}

\BNCC{EM13CHS601}
\BNCC{EM13LP02}

\paragraph{Tema} As condições do negro na sociedade escravista

\paragraph{Conteúdo} Pesquisa e estudo das condições do negro durante o
período da escravidão no Brasil, a partir das narrativas curtas de Júlia
Lopes de Almeida. A atividade consiste na apresentação dos trabalhos por
meio de representações cênicas, como teatro, cinema, dança ou
performance.

\paragraph{Objetivo} Habilitar os estudantes a refletir e discutir sobre
as condições vividas pelo negro em um período de escravidão, a fim de
compreender suas ressonâncias posteriores.

\paragraph{Justificativa} O conselheiro Aires, personagem do último
romance de Machado de Assis, anota no dia 13 de 1888: ``Ainda bem que
acabamos com isto. Era tempo. Embora queimemos todas as leis, decretos e
avisos, não poderemos acabar com os atos particulares, escrituras e
inventários, nem apagar a instituição da história, ou até da poesia''.
Ironicamente, consta que Rui Barbosa, de fato, mandou incendiar arquivos
relacionados à escravidão que eram mantidos por órgãos públicos
vinculados ao Ministério da Fazenda, pouco depois de ser proclamada a
República. Existe, no entanto, um consenso de que eram documentos
fiscais e que a medida do jurista e abolicionista visava evitar que os
antigos escravizadores pleiteassem indenização junto ao governo.

As marcas estão aí e, até hoje, permanecem. A inexistência de documentos,
fiscais ou não, apenas poderiam atrasar um pouco as tentativas de
reparação histórica, mas a forma como a sociedade atual lida com as
questões raciais, reforçando, cada vez mais, a existência do racismo
estrutural, colocam em evidência essas marcas que não foram e não serão
apagadas. A história, a literatura, os arquivos e os estudos em diversos
outros campos do saber expõem, cada um à sua maneira, a permanência
dessa barbárie de séculos. Assim, a articulação dessas áreas a partir da
leitura dos textos de Júlia Lopes de Almeida pode contribuir para a
compreensão dos mecanismos que propiciaram a escravidão no Brasil.

\paragraph{Metodologia}

\begin{enumerate}

\item
Apresente e discuta com os alunos em sala de
aula os aspectos históricos, sociais, econômicos e políticos que
estruturaram a escravidão no Brasil. Além dos textos de Júlia Lopes de
Almeida, comente outras representações literárias e artísticas sobre o
período, como, por exemplo, os quadros de Johann Moritz Rugendas e
Jean-Baptiste Debret. Mostre algumas delas em sala de aula, apontando as
relações entre a linguagem pictórica e o discurso histórico,
problematizando as possibilidades e os limites de cada um.


\Image{Quadro "Navio Negreiro" de Johann Moritz Rugendas (1830) (Johann Moritz Rugendas; Domínio Público)}{PNLD0002-11.png}


\Image{Quadro "O Jantar" de Jean Baptiste Debret (1839) (Jean Baptiste Debret; Domínio Público)}{PNLD0002-12.png}


\item
Organize a turma em grupos e proponha que selecionem um dos textos de
Júlia Lopes para a montagem de sua apresentação cênica. É importante
ressaltar que as personagens negras da escritora são do período
republicano. Os contos que podem ser trabalhados são:

\begin{itemize}
\item
  ``A rosa branca'';
\item
  ``Os porcos'';
\item
  ``A morte da velha'';
\item
  ``Perfil de preta (Gilda)'';
\item
  ``Nevrose da cor''.
\end{itemize}

\item
Cada grupo pode partir de um personagem específico vinculado ao tema
desta atividade e desenvolver uma narrativa paralela, desde que esteja
relacionada ao texto de origem. Se preferirem, os alunos podem adaptar o
conto integralmente, sem perder de vista a
intertextualidade e interdisciplinaridade que as atividades desta seção
propõem. Uma outra sugestão é dialogar comparativamente com obras e
questões contemporâneas, demonstrando as ressonâncias da sociedade
escravocrata nos dias atuais.

\item
Eles podem apresentar o trabalho por meio de qualquer linguagem
cênica: teatro, dança, cinema, performance, \emph{Slams} ou até mesmo
animação. Um ótimo exemplo de uso de várias linguagens cênicas para
discutir o tema é o documentário do rapper Emicida, \emph{AmarElo: É
tudo pra ontem}, dirigido por Fred Ouro Preto. É possível que os
estudantes já tenham assistido, mas se você achar oportuno, leve para a
escola e vejam o filme juntos. Pode ser uma boa fonte de inspiração para
os estudantes.

\item
Não esqueça de pedir a eles o roteiro do trabalho antes de
apresentá-lo, a fim de que possam discutir o texto, o conteúdo e
metodologia utilizada. Depois de apresentarem os trabalhos e realizarem
os devidos ajustes no texto, eles devem entregar o roteiro final, com
todas as alterações e um roteiro de pesquisa, relacionando as
referências bibliográficas, as fontes e os meios utilizados para chegar
ao trabalho final.
\end{enumerate}

\paragraph{Tempo estimado} Entre um a dois bimestres.



\subsection{Leitura II}


\BNCC{EM13LGG401}
\BNCC{EM13LP52}


\paragraph{Tema} O lugar da mulher na sociedade na virada do século XIX para o XX



\paragraph{Conteúdo} Pesquisa e estudo do papel da mulher na sociedade
patriarcal no período da virada do século XIX para o XX, a partir das
narrativas curtas de Júlia Lopes de Almeida. A atividade consiste na
apresentação dos trabalhos por meio de representações cênicas, como
teatro, cinema, dança ou performance.

\paragraph{Objetivo} Habilitar os estudantes a refletir e discutir o lugar
da mulher na sociedade brasileira na virada do século XIX e XX, a fim de
compreender sua relação com os dias atuais.

\paragraph{Justificativa} Apesar das profundas mudanças na sociedade na
virada do século XIX para o XX, como o fim da escravidão, a modernização
das grandes capitais e a proclamação de um novo regime político, o lugar
reservado à mulher permanecia praticamente o mesmo, voltado para as
tarefas domésticas e funções relacionadas a ela, como ser esposa, mãe e
dona de casa. As representações das mulheres em pinturas do século XIX
como leitoras, em seus ambientes familiares e posição passiva, são bem
ilustrativas quanto ao papel delegado a elas na vida cultural e
intelectual de um país que se dizia livre. Mas livre para quem?


\Image{Bertha Lutz ao lado do avião do qual se lançaram panfletos de propaganda pelo voto feminino (Arquivo Nacional; Domínio Público)}{PNLD0002-13.png}



Não foram poucas as escritoras e os escritores que denunciaram essa
posição em que as mulheres eram colocadas. Assim como Júlia Lopes de
Almeida, o escritor Lima Barreto também foi um defensor ativo dos
direitos das mulheres em suas crônicas e em seus romances. Júlia Lopes
participou de muitas conferências em associações femininas, sempre
preocupada com emancipação das mulheres, em especial através da educação
e também da independência financeira.

\SideImage{A "Lei do Feminicídio" foi aprovada em 2015, fruto de diversas manifestações de mulheres contra a violência machista. (Ramiro Furquim; CC-BY-NC-SA 3.0)}{PNLD0002-14.png}

A legislação do período, apesar de
alguns avanços, mantinha o lugar subalterno da mulher na constituição
familiar e a sua educação era desestimulada. O direito ao voto feminino
só foi conquistado na década de 1930, depois de intensos debates e lutas
dos movimentos sufragistas. Condições, entre outras, que influenciaram
decisivamente na formação do nosso cânone literário, artístico e
intelectual.


\Image{Em ato público, mulheres denunciam feminicídio no DF (Wilson Dias/Agência Brasil; CC-BY-3.0)}{PNLD0002-15.png}


\paragraph{Metodologia}

\begin{enumerate}
\item
Apresente e discuta com os alunos em sala de
aula os aspectos históricos, sociais, econômicos e políticos que
estruturaram o lugar da mulher e o processo de apagamento sofrido pelas
escritoras na virada do século XIX para o XX. Além dos textos de Júlia
Lopes de Almeida, comente outras representações literárias e artísticas
nesse e em outros períodos da história.

\item
Mostre algumas imagens em sala de aula, apontando as relações entre a
linguagem pictórica e o discurso histórico, problematizando as
possibilidades e os limites de cada um. Uma ótima referência para ser
trabalhada nesse ponto é o livro \emph{Imagens da mulher no Ocidente
moderno} (2019), de Isabelle Anchieta.

\item
Organize a turma em grupos e proponha que selecionem um dos textos de
Júlia Lopes para a montagem de sua apresentação cênica. As histórias que
podem ser trabalhadas são:

\begin{itemize}
\item
  ``O caso de Rute'';
\item
  ``A rosa branca'';
\item
  ``Os porcos'';
\item
  ``A caolha'';
\item
  ``As três irmãs'';
\item
  ``O laço azul''.
\end{itemize}

\item
Cada grupo pode partir de uma personagem específica vinculada ao tema
desta atividade e desenvolver uma narrativa paralela, desde que esteja
relacionada ao texto de origem. Se preferirem, os alunos podem adaptar o
conto ou a novela integralmente, mas não se pode perder de vista a
intertextualidade e interdisciplinaridade que as atividades desta seção
propõem. Uma outra sugestão é dialogar comparativamente com obras e
questões contemporâneas, demonstrando as ressonâncias da sociedade
patriarcal nos dias atuais. O livro de poesia de Angélica Freitas,
\emph{Um útero é do tamanho de um punho} (2012), é uma ótima referência
atual para a comparação.

\item
Eles podem apresentar o trabalho por meio de qualquer linguagem
cênica: teatro, dança, cinema, performance, \emph{Slams} ou até mesmo
animação. É possível que os estudantes já tenham assistido, mas se você
achar oportuno, leve para a escola e vejam o filme juntos. Pode ser uma
boa fonte de inspiração para os estudantes.

\item
Cada grupo deve entregar o roteiro do trabalho antes de apresentá-lo,
a fim de que possam discutir o texto, o conteúdo e metodologia utilizada
em sala de aula. Depois de apresentarem os trabalhos montados e
realizarem os devidos ajustes no texto, eles devem entregar o roteiro
final, com todas as alterações e um roteiro de pesquisa, relacionando as
referências bibliográficas, as fontes e os meios utilizados para chegar
ao trabalho final.
\end{enumerate}

\paragraph{Tempo estimado} Entre um a dois bimestres.



\subsection{Pós-leitura}

\BNCC{EM13LGG102}
\BNCC{EM13LGG104}
\BNCC{EM13LP53}

\paragraph{Tema} Precursoras da educação.

\paragraph{Conteúdo} Pesquisa e estudo da atuação das mulheres em favor da
educação no Brasil, em especial para a formação de mulheres e de outras
minorias. A atividade consiste na apresentação dos trabalhos no formato
de texto jornalístico, na versão impressa ou digital.

\paragraph{Objetivo} Habilitar os estudantes a refletir e discutir a
importância das mulheres para o acesso universal à educação e o seu
desenvolvimento.

\paragraph{Justificativa} A história da educação no Brasil é marcada por
uma trajetória de avanços fundamentais, mas também de manutenção de
desigualdades profundas, especialmente nas categorias de cor, gênero e
classe. A exclusão das mulheres do sistema educacional durou séculos e
foi revertida a muito custo. Mesmo assim, ainda no século XXI, há muito
a ser discutido e mudado para que a igualdade entre os gêneros seja
plena, não apenas na educação evidentemente. No entanto, é através dela
que as grandes conquistas da sociedade podem ser alcançadas.

Mesmo escritores que dedicaram páginas dos jornais na defesa dos
direitos da mulher e de outras minorias, como Olavo Bilac, reagiam à
tentativa delas em se tornar também escritoras e intelectuais. Em carta
a sua noiva, Amélia de Oliveira, Bilac demonstra desagrado ao saber que
ela havia publicado versos e avisa que não queria que ela parasse, ao
contrário, ela deveria continuar a escrevê-los, mas para os irmãos,
paras as amigas e, principalmente, para ele. Assim também outras
mulheres foram desencorajadas, bem como os indivíduos pertencentes a
outras minorias.

\paragraph{Metodologia}

\begin{enumerate}
\item
Apresente um breve panorama da história da
educação no Brasil, citando seus avanços, problemas e principais
personagens. Aproveite a oportunidade para ouvir dos alunos suas
impressões sobre a educação pública brasileira e quais medidas eles
consideram importantes para que ela melhore. Não deixe de perguntar a
eles sobre os avanços que eles perceberam. Se puder, leve alguns dados
de forma bem sintética para que eles possam acompanhar, junto com você,
os principais números que refletem a situação do ensino no país. Compare
com outros sistemas educacionais, mas sempre com cuidado e rigor
metodológico.

\item
Organize a turma em grupos, em que cada um formará uma pequena
redação de jornal. Proponha que as tarefas sejam divididas por função,
mas todos devem estar envolvidos na pesquisa e na divulgação dos
resultados.

\item
O grupo deve pesquisar os nomes de precursoras da educação no Brasil,
principalmente de figuras pouco conhecidas e citadas nesse campo. Uma
delas, por exemplo, pode ser a escritora Júlia Lopes de Almeida que,
assim como outras escritoras e intelectuais de seu tempo, se dedicou a
promover a educação para todos, principalmente das mulheres. Maria
Firmina dos Reis, a nossa primeira romancista negra, também teve
importante atuação na área educacional. A ela é atribuída a primeira
escola para meninas e meninos no Brasil. O período histórico pode se
estender até os dias atuais, em perspectiva comparada ou não, citando
exemplos de figuras anônimas que atuam, a contrapelo dos limites
impostos diariamente, para o desenvolvimento da nossa educação.

\item
Depois de definir a educadora, os estudantes jornalistas devem fazer
um levantamento de informações sobre ela e coletar imagens. A primeira
parte do grupo fica responsável pela escrita, revisão e edição do texto,
a segunda cuida da produção, coleta e tratamento das imagens e uma outra
se encarrega de diagramar o material e publicar o jornal. Dependendo do
número de alunos, evidentemente essas tarefas podem se misturar. O
importante é que todos participem da execução do trabalho.

\item
Durante a pesquisa, estimule-os a consultar os arquivos na rede, como
os periódicos mantidos pela Biblioteca Nacional em sua Hemeroteca
Digital, assim como os acervos disponibilizados on-line pelo Arquivo
Nacional. É fundamental que eles não deixem de citar as fontes de
pesquisa.
\end{enumerate}

\paragraph{Tempo estimado} Um bimestre de aulas.

\section{Aprofundamento}

Júlia Lopes de Almeida nasceu no dia 24 de setembro de 1862 em um
casarão na Rua do Lavradio, no Centro do Rio de Janeiro. Filha de
Antonia Adelina Pereira Lopes e Valentim José da Silveira Lopes,
portugueses emigrados para o Brasil, Júlia tornou-se romancista,
contista, cronista e dramaturga. Ainda criança, mudou-se com a família
para São Paulo e com apenas 19 anos, em 1881, publicou seus primeiros
textos no jornal de sua cidade, \emph{A Gazeta de Campinas}. Aos 22
anos, em 1884, começou a escrever para um dos principais periódicos
brasileiros, \emph{O País}, colaboração que se estendeu por mais de
trinta anos. A atividade literária e jornalística em importantes
veículos da imprensa da época teve grande influência na sua produção
intelectual e artística.

Júlia mudou-se, em 1886, para Lisboa, onde deu início à sua carreira de
escritora. No ano seguinte, com a irmã Adelina Lopes Vieira, publicou
\emph{Contos infantis}. Em 1887, casou-se com Filinto de Almeida, também
escritor e diretor do periódico carioca \emph{A Semana Illustrada}, que
contou com a frequente participação de Júlia Lopes. De volta ao Brasil,
no ano de 1888, publicou seu primeiro romance, \emph{Memórias de Marta,}
por meio de folhetins em \emph{O País}. Através de uma escrita prolífica
e engajada, Júlia Lopes de Almeida abordou temas como a República, a
escravidão e o papel da mulher nas esferas pública e privada da
sociedade, colocando o Rio de Janeiro como um de seus principais
cenários. Dentre seus livros, destaca-se o romance \emph{A falência}, de
1901, retrato incisivo de um país que mudava de regime e se modernizava,
mas permanecia preso às estruturas arcaicas de exploração e
desigualdades.

Júlia Lopes de Almeida foi uma das escritoras mais importantes da virada
do século XIX para o XX, sendo um dos principais nomes da \emph{Belle
Époque} carioca conhecida também como \emph{Belle Époque} tropical.
Esteve entre os idealizadores da Academia Brasileira de Letras, mas não
foi escolhida para assumir uma das cadeiras entre os fundadores por ser
mulher, já que a maioria dos membros decidiu acompanhar a tradição da
Academia Francesa de Letras, modelo seguido pela agremiação no Brasil,
que contava apenas com homens no quadro. Seu marido, Filinto de Almeida,
ao contrário, ocupou a cadeira de número 3, embora reconhecesse, em
entrevista a João do Rio, que quem deveria estar na Academia era Júlia
e não ele.


%\SideImage{Júlia Lopes de Almeida, sem data. (Arquivo Nacional; Domínio Público)}{PNLD0002-03.png}


A escritora chegou a morar novamente em Portugal, onde publicou suas
primeiras peças teatrais e depois na França, onde sua obra foi
traduzida e divulgada. Participou ativamente de diversas associações
femininas e discutiu temas relacionados ao Brasil e à mulher em
conferências no país e no exterior, bem como em alguns de seus livros.
Faleceu no Rio de Janeiro em 30 de maio de 1934, pouco antes da
promulgação da nova constituição.

Mesmo sendo uma das autoras mais importantes de seu tempo e admirada
pelos seus pares, o nome de Júlia Lopes de Almeida não resistiu aos
mecanismos de apagamento do cânone. No entanto, sua obra vem sendo
resgatada nos últimos anos por estudiosos de diversas áreas das
humanidades, com reedições de seus principais livros. Além disso, a
atualidade das questões discutidas em sua obra e a moderna sofisticação
de sua escrita são também fatores determinantes para que sua leitura
seja cada vez mais necessária.


%\SideImage{Busto de Júlia Lopes de Almeida, Lisboa (PESP/ Wikimedia; CC-BY-SA 4.0)}{PNLD0002-04.png}


Marcada pelos contos de Guy de Maupassant (1850-1893), assim como pelos
romances de Émile Zola (1840-1902), Júlia Lopes imprime em suas obras
uma forte influência do naturalismo e do realismo francês. Algumas das
características presentes na sua produção literária são a objetividade,
em contraposição ao sentimentalismo, o antropocentrismo, as duras
críticas à sociedade brasileira e o cientificismo na análise de seus
personagens, influenciados pelo meio, raça e contexto histórico, de
acordo com o determinismo e cujo comportamento é associado a causas
biológicas, segundo o biologismo. A zoomorfização também é um elemento
recorrente nas obras de Júlia Lopes, atribuindo características animais
a seres humanos. No entanto, a escritora não deixou de escrever aquilo
considerado mais adequado para uma mulher da época, como \emph{O Livro
das Noivas e Maternidade}.

É importante ressaltar o contexto histórico no qual Júlia Lopes de
Almeida está inserida, a começar pelo ano de seu nascimento. Em 1862 o
Brasil rompe relações com o Reino Unido na Questão Christie, como
consequência de tensões entre as coroas, principalmente por conta da
persistência da~escravidão~no Brasil. Após a Proclamação da República em
1889, importantes transformações políticas, econômicas, sociais e
culturais marcaram o país na virada do século. A Primeira República,
também conhecida como República Velha, vivenciou graves~crises devido às
disputas geradas pelas forças políticas ainda fragmentadas e à
desvalorização da moeda acompanhada do súbito crescimento da inflação.
Júlia Lopes viveu um período de consolidação das instituições
republicanas, de uma~economia agroexportadora, de revoltas populares,
civis e militares, contra o sistema político e social, e de entrada no
século XX, a chamada Era dos Extremos.

Esta coletânea reúne algumas narrativas curtas de Júlia Lopes de
Almeida, dividida em duas seções: ``Contos'' e ``Novelas''. Os livros de
onde foram extraídos os textos são, respectivamente, \emph{Ânsia eterna}
(1903) e \emph{A isca} (1922). Embora não sejam os únicos volumes de
narrativas curtas da escritora, foram selecionados por apresentarem
algumas das características da narrativa de Júlia Lopes e dos temas que
permeiam sua obra. Por isso, o livro não se propõe a ser uma síntese ou
um panorama da multifacetada e expressiva produção literária da autora,
mas um convite à discussão sobre questões presentes em suas temáticas,
bem como um estímulo a conhecer suas demais obras.

\emph{Ânsia eterna} foi publicado pela primeira vez no Rio de Janeiro,
pela H. Garnier. Em 1938, foi lançada uma reedição póstuma pela editora
A Noite, com correções feitas pela autora. Embora os textos de
\emph{Ânsia eterna} fujam um pouco do universo da obra de Júlia Lopes,
ao abordar o insólito e o fantástico, a começar pelo título do volume,
eles não deixam de discutir as questões fundamentais à escritora, como o
papel da mulher e o retrato da sociedade escravocrata. Para esta
coletânea, foram selecionados dez contos: ``O caso de Rute'', ``A rosa
branca'', ``Os porcos'', ``A caolha'', ``Incógnita'', ``A morte da
velha'', ``Perfil de preta (Gilda)'', ``A nevrose da cor'', ``As três
irmãs'' e ``O futuro presidente''. Muitos deles são dedicados a
escritores e intelectuais de sua geração, como Arthur Azevedo e Machado
de Assis.

Já a edição de \emph{A isca} foi um trabalho da Livraria Leite Ribeiro,
também no Rio de Janeiro. O livro é constituído de quatro novelas, das
quais selecionamos duas para esta coletânea, ``O laço azul'' e ``O dedo
do velho''. Com o subtítulo ``novela romântica'', a primeira traz à tona
o lugar da mulher na constituição familiar, sua posição em tempos de
guerra e as dinâmicas das relações entre seus membros. E isto através da
questão do duplo, representada por duas irmãs gêmeas, um dos temas
recorrentes da prosa de ficção moderna. A segunda novela foi publicada
pela primeira vez em \emph{A Illustração Brazileira}, em 1909, com o
subtítulo ``romance''. Assim como em alguns contos de \emph{Ânsia
eterna}, ``O dedo do velho'' também se reveste do insólito no
desenvolvimento de sua história, além de apresentar alguns índices da
modernidade, nas referências ao automóvel e à urbanização.

\section{Referências complementares}

\subsection{Audiovisual}

\begin{enumerate}
\item
  \emph{Que Horas Ela Volta?} (2015), dirigido por Anna Muylaert.
\item
  \emph{Memórias Póstumas Brás Cubas} (2001), dirigido por André
  Klotzel.
\item
  \emph{Madame Bovary} (1991), Claude Chabrol.
\item
  \emph{Germinal} (1993), Claude Berri.
\end{enumerate}

\subsection{Literária}

\begin{enumerate}
\item
  NEEDELL, Jeffrey D..~\emph{Belle époque tropical:} sociedade
  e cultura de elite no Rio de Janeiro na virada do século. São
  Paulo: Companhia das Letras, 1993.
\item
  BERTA, Veronica. \textit{Ânsia Eterna}. São Paulo: SESI-SP,
  2018.
\item
  PONTES, Heloisa. Modas e modos. Uma leitura enviesada de~\emph{O
  espírito das roupas}.~\textit{Cadernos Pagu}~(22), Campinas-SP, Núcleo
  de Estudos de Gênero - Pagu, jan/jul 2004, pp.13-46.
\item
  MICELI, Sergio.~\textit{Imagens negociadas}: retratos da elite
  brasileira (1920-1940). São Paulo, Companhia das Letras, 2006.~
\item
  COUTINHO, Afrânio. \textit{A literatura no Brasil -- Vol. V}:
  Realismo/Naturalismo. Direção de Afrânio Coutinho. Co-direção de
  Eduardo de Faria Coutinho -- 3ª edição -- Rio de Janeiro: José Olympio
  Ed. Niterói: UFF -- Universidade Federal Fluminense, 1986.
\end{enumerate}

\subsection{Musical}

\begin{enumerate}
\item
  ``Maria da Vila Mathilde'', em \emph{A Mulher do Fim do Mundo}
  (2015), Elza Soares.
\item
  ``Desconstruindo Amélia'', em \emph{Chiaroscuro} (2009), Pitty.
\item
  ``100\% Feminista'', em \emph{Bandida} (2016), MC Carol.
\end{enumerate}

\subsection{Artes visuais}

\begin{enumerate}
\item
  BORELLI, Ana. \emph{Rio de Janeiro Perdido}. Rio de Janeiro: TIX
  Editora, 2019.
\item
  Fotografia do I Congresso Internacional no Rio de Janeiro, 12/1922.
  Rio de Janeiro (RJ)/Arquivo Nacional. Disponível em:
  \url{http://brasilianafotografica.bn.br/brasiliana/handle/20.500.12156.1/4946}
\end{enumerate}

\subsection{Outros}

\begin{enumerate}
\item
  LABELLE -- Laboratório de estudos de literatura e cultura da \emph{Belle Époque,} disponível em: \url{http://labelleuerj.com.br/}
\end{enumerate}



\section{Bibliografia Comentada}

\begin{enumerate}
\item
  Anchieta, Isabelle. \emph{Imagens da mulher no Ocidente moderno}. São
  Paulo: Edusp, 2019. Organizado em três volumes, o livro apresenta uma
  interessante análise sociológica das formas de representação visual da
  mulher ao longo da história Ocidental, do século XV até os dias
  atuais.
\item
  Azevedo, André Nunes de. \emph{Grande Reforma Urbana do Rio de
  Janeiro}: Pereira Passos, Rodrigues Alves e as ideias de civilização e
  progresso. Rio de Janeiro: Mauad; PUC-Rio, 2016. O historiador expõe
  as contradições em torno do projeto de urbanização empreendido no Rio
  de Janeiro na virada do século XIX para o XX, questionando
  interpretações já consagradas sobre a atuação do prefeito Pereira
  Passos.
\item
  Del Priori, Mary (org.). \emph{História das Mulheres no Brasil}, 9.
  ed. 2ª reimpressão. São Paulo: Contexto, 2009. Voltado para todos os
  tipos de leitoras e leitores, a obra delineia a trajetória das
  mulheres do Brasil com a participação de vinte historiadores que
  tratam de diferentes espaços e extratos sociais, derruba mitos e
  estimula reflexões. Dentre as personagens está a escritora carioca
  Júlia Lopes de Almeida.
\item
  Faedrich, Anna; Fanini, Michele. Entrevista com os netos de Júlia
  Lopes de Almeida: Claudio e Fernanda Lopes de Almeida.~\emph{Aletria:
  Revista de Estudos de Literatura},~30(4), p. 315--328, 2020.
  Disponível em:
  \textless{}https://doi.org/10.35699/2317-2096.2020.24495\textgreater{}.
  Entrevista estruturada em duas partes com os netos de Júlia Lopes de
  Almeida, Claudio e Fernanda Lopes de Almeida. Passagens biográficas
  pouco conhecidas a respeito da avó e da família Almeida são
  rememoradas, incluindo seu não ingresso na Academia Brasileira de
  Letras.
\item
  Gotlib, Nádia Battella. \emph{Teoria do conto}. São Paulo: Ática,
  1985. Excelente introdução aos principais conceitos e discussões
  teóricas sobre o gênero conto, abordando as concepções de Vladimir
  Propp, Edgar Allan Poe, Júlia Cortázar e outros textos fundamentais.
\item
  Propp, Vladimir. \emph{Morfologia do conto maravilhoso}. Trad. Jasna
  Paravich Sarhan. Rio de Janeiro: Forense Universitária, 2006. A obra
  de Vladimir Propp se dedica a definir o conceito de conto maravilhoso
  a partir de estruturas narrativas observadas na análise de contos
  populares russos.
\end{enumerate}


\end{document}


