\documentclass[11pt]{extarticle}
\usepackage{manualdoprofessor}
\usepackage{fichatecnica}
\usepackage{lipsum,media9,graficos}
\usepackage[justification=raggedright]{caption}
\usepackage[one]{bncc}
\usepackage[alameda]{../edlab}

\begin{document}

\newcommand{\AutorLivro}{Irineu Franco Perpétuo}
\newcommand{\TituloLivro}{História concisa da música clássica brasileira}
\newcommand{\Tema}{História da música}
\newcommand{\Genero}{Diário, biografia, autobiografia, relatos, memórias}
\newcommand{\imagemCapa}{./images/PNLD0052-01}
\newcommand{\issnppub}{---}
\newcommand{\issnepub}{---}
% \newcommand{\fichacatalografica}{PNLD0052-00}
\newcommand{\colaborador}{Irineu F.~Perpétuo}


\title{\TituloLivro}
\author{\AutorLivro}
\def\authornotes{\colaborador}

\date{}
\maketitle

\baselineskip=1.2\baselineskip\par

\begin{abstract}\addcontentsline{toc}{section}{Carta ao professor}

Este manual estabelecerá pontos de contato entre a matéria apresentada
neste livro e você, que é quem de fato coloca em prática o processo
educativo.

Em \emph{História Concisa da Música Clássica Brasileira,} o crítico
musical Irineu Franco Perpetuo narra, em linguagem simples e acessível,
a trajetória da música nos mais de 500 anos que se seguiram após a
chegada ao território em que estamos dos conquistadores portugueses.

Não é segredo para ninguém que o Brasil possui uma produção musical das
mais ricas e exuberantes. A música brasileira conquistou o planeta com
seu colorido, seu vigor e sua originalidade. Talvez não seja exagero
afirmar que a manifestação artística do Brasil mais conhecida no
exterior é a música.

Por outro lado, quando se fala em música brasileira, a primeira coisa que vem
à cabeça de todo mundo é a música popular. Do samba, choro e maxixe do
passado, ao funk, axé e hip-hop de hoje, nossa música popular vem se
reinventando com inesgotável criatividade.

Só que o Brasil tem também uma outra música, em constante diálogo com
essa. A música de Villa-Lobos, Carlos Gomes, Chiquinha Gonzaga, Ernesto
Nazareth. Uma música que vem sendo escrita em partitura há séculos, e
acompanhando cada passo de nossa história. Nosso livro conta a história
dessa música. E, se a música é uma das maneiras mais lúdicas de se
estudar, podemos dizer que \emph{História Concisa da Música Clássica no
Brasil} propõe a trilha sonora que vem acompanhando nossos cinco séculos
de desenvolvimento histórico e cultural.

``Música clássica'' pode ser uma expressão intimidadora para quem não
pertence a este universo. O livro, contudo, evita termos técnicos e o
jargão especializado. Além disso, tem vocação transversal e
multidisciplinar. O autor não encara a música como fenômeno isolado,
porém relaciona-a constantemente com outras esferas do conhecimento. Com
o título inspirado na \emph{História Concisa da Literatura Brasileira,}
de Alfredo Bosi, o livro traça paralelos constantes entre o mundo dos
sons e o mundo das letras. O papel da música nos momentos-chave da nossa
História também é contemplado. A música popular, embora não seja o foco
principal da obra, também é abordada, especialmente quando sua
trajetória acaba se cruzando com os caminhos da música erudita. E, por
fim, a obra almeja não apenas variedade estilística, como também a
regional, a de gênero e a racial. Aborda-se não apenas a música que veio
com os portugueses, como trata também da sonoridade dos povos
originários; fala-se dos inúmeros compositores negros do Brasil; e
também da luta constante das mulheres para adquirirem protagonismo em
nosso cenário musical.

Aqui será fornecida uma série de questões, propostas de atividades e
materiais suplementares que permitirão, através da música, conhecer a
história, a realidade social e a evolução cultural de nosso país ao
longo dos séculos -- chegando, inclusive, aos dias de hoje. Vale lembrar
que os recursos da internet tornam extremamente simples o acesso a
gravações em vídeo e áudio das obras e compositores apresentados no
livro. Ouvindo a música brasileira do passado, podemos não apenas
compreender melhor o momento histórico em que ela foi produzida, como
refletir sobre sua evolução até se tornar a música que ouvimos e
produzimos hoje.

Esperamos que você ache útil e divertido este material para o bom
trabalho em sala de aula!

\end{abstract}

\tableofcontents


\section{Atividade \textsc{i (a)}}

\subsection{``Polcas, maxixes e tangos brasileiros''}

\paragraph{Tema} As fronteiras entre popular e erudito. A luta das
mulheres por emancipação, no exemplo pioneiro de Chiquinha Gonzaga.


\BNCC{EM13LP45}
\BNCC{EM13LP47}
\BNCC{EM13LP50}
\BNCCC{EM13LP51;
EM13CHS101;
EM13CHS102;
EM13CHS104;
EM13CHS105.}

\paragraph{Objetivos e justificativa}

Para abordar a produção das mulheres na música clássica brasileira,
tomamos o caso emblemático de Chiquinha Gonzaga. Filha de um militar
próximo ao Duque de Caxias (1803-1880), que participaria da guerra
contra o Paraguai, Francisca Edwiges Gonzaga mereceu o repúdio da
família depois de abandonar o marido, com o qual se casara aos 16 anos
de idade. Viveu sempre de música e, aos 52 anos, uniu-se a um homem de
16, que apresentava a todos como seu filho.
\SideImage{Chiquinha Gonzaga (1847--1935) aos 18 anos de idade.}{PNLD0052-03}

O caso de Chiquinha permite abordar as tensões entre ``popular'' e
``erudito'', entre ``respeitável'' e ``indecente'' -- entre ``brega'' e
``chique''. Ela ganhou a alcunha de ``Offenbach de saias'' pela
facilidade com que criava partituras para as diversas formas de teatro
musical ligeiro que, por influência francesa, floresciam no Rio de
Janeiro a partir da inauguração, em 1861, do café-cantante Alcázar
Lyrique (mais tarde, Teatro Lírico Francês). Esse gênero teatral causava
sensação e escândalo com o humor de duplo sentido e o apelo sensual das
vedetes. Era o tempo da opereta, da mágica (assim chamada devido à
utilização de temas fantásticos), da burleta e do teatro de revista --
que levava esse nome por ``passar em revista'' os principais
acontecimentos do ano.
\SideImage{Partitura de Odeon, de Ernesto Nazareth (Wikipedia; Domínio público)}{PNLD0052-33}

Nessa época fértil, que marca o florescimento de gêneros fundamentais da
nossa música popular, como o choro e o samba, as tensões latentes na
produção de Chiquinha Gonzaga tornam-se ainda mais evidentes quando
entendidas ao lado de seu contemporâneo Ernesto Nazareth. Sem recursos
para perseguir o sonho de se tornar pianista de concerto, Nazareth
lecionava ao teclado e vivia do ofício que então era chamado, algo
pejorativamente, de \emph{pianeiro}: dedilhando o instrumento em casas
de venda de música (como demonstrador, executando as partituras
negociadas a seus eventuais compradores) e em cinemas, como o Odeon, o
que daria título a um de seus mais célebres tangos.

% \paragraph{Metodologia}


\subsubsection{Antes da leitura}
\SideImage{Piano.}{PNLD0052-07}

Pedir que os alunos procurem no dicionário e anotem os significados das
palavras \emph{polca, maxixe} e \emph{tango.} Promover um bate-papo com
os estudantes, perguntando a diferença entre música popular e música
clássica, e se já tiveram alguma experiência com música clássica.

As questões se multiplicam: na sua época, o maxixe era tido como
``maldito''. Poderia ser feito um paralelo com o que acontece hoje com o
funk? Existem gêneros musicais ``nobres'' e outros ``respeitáveis''? Um
tipo de música é melhor do que outro? Quem decide o que é música
``boa''? A música ``maldita'' de hoje é a música ``boa'' de amanhã?

\subsubsection{Durante a leitura}

Analisar a trajetória de Chiquinha Gonzaga e seu pioneirismo na
sociedade brasileira da segunda metade do século \textsc{xix} e início do século
\textsc{xx}. Como o escândalo que ela causava entre as mentes mais conservadoras
de sua época revela a posição subordinada a que estavam sujeitas as
mulheres no Brasil de então. É possível traçar um paralelo entre
Chiquinha Gonzaga, a primeira maestrina brasileira, e sua contemporânea
Nair de Teffé, pioneira do cartum no Brasil, que, quando era a
primeira-dama da nação, causou alvoroço ao tocar a música de Chiquinha
em uma recepção diplomática.

\Image{Uma família brasileira no Rio de Janeiro, 1839. (Jean-Baptiste Debret -- Wilkipedia; Domínio público.)}{PNLD0052-39}

Perceber, no exemplo do compositor francês Darius Milhaud, que visitou o
Brasil, que a elite brasileira de então atrás estava mais interessada
pelo que acontecia na França do que em seu próprio país. Trocando França
por Estados Unidos, seria possível fazer uma analogia com a realidade
brasileira de hoje?

Enfatizar as tensões entre ``popular'' e ``erudito'', entre ``maldito''
e ``respeitável'', no caso de Ernesto Nazareth. Nazareth entrou para a
História como um notável autor de obras de inspiração popular, mas
padeceu sempre da frustração de não ter sido um pianista de concerto,
que tocava Chopin no Teatro Municipal. Essa questão pode ser aprofundada
com a leitora do conto \emph{Um Homem Célebre,} de Machado de Assis,
cujo protagonista é um músico que enfrenta dilemas similares aos de
Nazareth.

Chamar ainda a atenção dos alunos para a existência de um ``tango
brasileiro'', bastante diferente do célebre tango argentino.

\subsubsection{Depois da leitura}

Propor que os alunos ouçam algumas das obras mencionadas no capítulo,
como o \emph{Corta-Jaca,} de Chiquinha Gonzaga, e \emph{Odeon,} de
Ernesto Nazareth. Ambas existem em versão puramente instrumental e
cantada, com letra. Os alunos podem cotejá-las e decidirem se as letras
refletem o espírito das músicas. E ainda discutirem a relação entre
música e letra. Você gosta de uma música porque se identifica com a
letra, ou porque a melodia é agradável?

Chiquinha Gonzaga inaugurou o cancioneiro brasileiro de Carnaval com
\emph{Ó Abre Alas.} Os alunos podem ouvir essa obra e compará-la com as
músicas de Carnaval de hoje, refletindo sobre as modificações que esse
festejo sofreu ao longo dos tempos.

\paragraph{Tempo estimado:} Duas aulas de 50 minutos.



\section{Atividade \textsc{i (b)}}
\subsection{“Dois séculos sem partituras”}

\paragraph{Tema} A musicalidade dos povos originários. A estratégia de catequização dos
indígenas pelos jesuítas brasileiros, tomando a música como elemento
estratégico.  

\BNCC{EM13LP45}
\BNCC{EM13CHS101}
\BNCC{EM14LP48}
\BNCCC{EM13CHS104;
		EM13CHS105;
		EM13LP51.}

\paragraph{Objetivos e justificativa}

Na música, a fábula das “três raças” que constituiriam a nacionalidade
brasileira costuma, no senso comum, ser reduzida a duas. Fala-se em uma
“influência negra” sobre a música que teria sido “trazida” pelos brancos. Se
esse esquema já é problemático, pior ainda é o fato de silenciar-se sobre a
musicalidade dos povos originários.

No livro, fica demonstrado que a falta de conhecimento da música destes povos
não se deve, obviamente, a qualquer carência deles nesse aspecto. Muito pelo
contrário: irá se verificar que a exuberante musicalidade dos nativos foi
utilizada pela Companhia de Jesus como ferramenta na conquista das almas do
Novo Mundo para a cristandade. Assim, o apagamento desse aspecto da cultura
indígena pode ser visto como um traço simbólico do genocídio físico que os
povos originários brasileiros sofreram a partir da chegada dos colonizadores
europeus. 

%\paragraph{Metodologia}


\subsubsection{Antes da leitura}

Pedir que os alunos procurem ilustrações e definições de instrumentos musicais
empregados pelos indígenas brasileiros, como inúbia, maracá, tacapu, torokaná,
catacá etc. Buscar descobrir que povos indígenas utilizam os instrumentos
pesquisados e, a partir daí, refletir sobre a diversidade étnica e linguística
desses povos. Designados genericamente como “índios”, na verdade eles não
constituem um todo homogêneo, formando antes um mosaico variegado de diversas
nações.

\SideImage{Índios Tupinambás (Wikipedia/\,Jean de Léry; Domínio público)}{PNLD0052-14}

Refletir ainda sobre as diferenças entre a função da música nas diversas
sociedades – e o que essas diferenças dizem sobre as próprias sociedades.
Vivemos em ambientes sonoramente saturados, em que a música é tocada ao fundo
em praticamente todos os espaços de circulação e convívio e está acessível a
um clique. Há uma distinção marcada entre o artista, que se apresenta, e o
público, que ouve; e a execução de música é sobretudo um ato estético, de
fruição sonora. Contrastar essas práticas modernas a uma sociedade em que o
fazer musical é coletivo, envolve toda a comunidade, e cumpre funções
ritualísticas e divinatórias.

\subsubsection{Durante a leitura}

  A leitura deve ser começada pelos três últimos parágrafos do capítulo
anterior, ``Da ocidental praia lusitana''. Especificamente, a partir da frase
“depois desse panorama da bagagem musical dos lusos, procuremos adivinhar como
era a música dos indígenas que eles encontraram na Terra de Santa Cruz”. No
mesmo capítulo, pode-se ainda recuar um pouco e ler a descrição, contida na
carta de Pero Vaz de Caminha, do momento em que Diogo Dias desce à terra
acompanhado de um gaiteiro e dança com os nativos. Observar a dificuldade dos
europeus em reconhecerem as sonoridades dos povos originários como musicais e
destacar o exemplo do francês Jean de Léry, que, ao visitar a França Antártica,
anota cânticos dos nativos em partituras – uma notável exceção.

Na leitura propriamente dita do capítulo Dois séculos sem
partituras, sublinhar a habilidosa evangelização pela música promovida pelos
jesuítas. Assinalar também as diferenças entre a colonização na América
Espanhola e na Portuguesa, entre uma coroa que fundava universidades e imprimia
livros em seus territórios ultramarinos e outra que reprimia esse tipo de
atividade – o que ajuda a explicar a ausência de partituras que dá nome ao
capítulo. O livro \emph{Raízes do Brasil}, de Sérgio Buarque de Holanda, pode fornecer
subsídios a essa analogia. No caso específico dos nativos, o fato de os
indígenas brasileiros estarem submetidos ao sistema de repartição – e, assim,
serem “repartidos” entre colonos ou funcionários da Coroa – reflete-se em uma
instabilidade populacional que resultará em uma atividade musical bem menos
pujante do que a que ocorria nas missões do lado espanhol. 

\subsubsection{Depois da leitura}


Propor que os alunos ouçam gravações de música indígena brasileira original,
dos trabalhos de Marlui Miranda, profunda pesquisadora da cultura dos índios, e
de Villa-Lobos – que se intitulava “índio de casaca”. No caso de Villa-Lobos,
especificamente, seria instrutivo ouvir \emph{Canide Ioune}, sua reelaboração de um
dos cânticos recolhidos no século \textsc{xvi} por Jean de Léry.


A audição pode estimular uma série de discussões. O que é “autêntico”
e o que se perde quando a música de um povo é relida e filtrada por outro?
Músicos “brancos” podem executar corretamente a música “índia”? Há mais de um
século, estamos acostumados a uma cultura de gravações e, antes disso, a música
era conservada e transmitida através de um documento escrito, a partitura. Em
uma cultura oral, como a dos povos originários, como se dá esse processo?
Quanto há da música indígena de 1500 na música executada hoje por esses mesmos
povos? Teria havido “contaminação” ou influências da música europeia? Ou
simplesmente as modificações inerentes à transmissão da cultura por via oral?

\paragraph{Tempo estimado} Duas aulas de 50 minutos.




\section{Atividade \textsc{ii (a)}}
\subsection{``Villa-Lobos''}

\paragraph{Tema} As relações entre Villa-Lobos e Getúlio Vargas como
emblema das relações entre arte e poder. A educação musical. A questão
nacional.

\Image{Heitor Villa-Lobos (1887--1959)}{PNLD0052-04}

\BNCC{EM13LP45}
\BNCC{EM13LP47}
\BNCC{EM13LP50}
\BNCCC{EM13LP51;
EM13LP53;
EM13CHS101;
EM13CHS102;
EM13CHS104;
EM13CHS105.}

\paragraph{Objetivos e justificativa}

Se os alunos já ouviram falar no nome de algum dos compositores citados
em \emph{História Concisa da Música Clássica Brasileira,} a maior
possibilidade é que esse nome seja o de Villa-Lobos\emph{.} Não apenas
no exterior, mas mesmo por aqui, Villa-Lobos é o único compositor
clássico brasileiro que tem merecido \emph{performances} e gravações
regulares de suas obras. Fundador da Academia Brasileira de Música,
conta, no Rio de Janeiro, com um eficiente museu dedicado a seu legado,
e sua efígie já foi parar até em cédula de dinheiro brasileiro, após o
Plano Cruzado, de 1986.

\Image{Heitor Villa-Lobos em uma nota de cruzado, moeda brasileira dos anos 1980.}{PNLD0052-05}

Embora ainda não cubra a totalidade de sua produção (não só por seu
caráter gigantesco, mas também devido pelo estado surpreendentemente
precário das edições de suas partituras), a discografia disponível do
compositor é bastante representativa, bem como a bibliografia sobre ele,
na qual proliferam tanto biografias quanto estudos críticos sobre
aspectos específicos de sua obra.

Gerard Béhague\footnote{BÉHAGUE, Gerard. \emph{Heitor Villa-Lobos: the
  search for Brazil's musical soul.} Austin: Institute of Latin American
  Studies, 1994.} sintetiza: ``em muitas maneiras, sua personalidade,
sua carreira e sua produção refletiram diversos traços típicos
brasileiros, como grandiosidade, exuberância, inquietação, falta de
unidade orgânica, disparidade, e gabolice, ao lado de outros, como
individualidade, espontaneidade, fascinação e sofisticação''.

Vale citar ainda a análise de Fabio Zanon: ``Não é o fato de se servir
da cultura popular que faz dele um compositor singular; é a maneira como
ele conseguiu encontrar nessa cultura os fios mais adequados e os
entrelaçou à sua complexa trama composicional. Nesse processo, ele urdiu
uma maneira brasileira de se expressar em música, que imprimiu uma marca
forte a outros artefatos culturais, desde as trilhas sonoras de filmes
de Glauber Rocha à ambientação de reportagens sobre o Brasil rural,
passando pelos arranjos orquestrais da era do rádio e pelos rastros
deixados no estilo de ícones da música popular como Tom Jobim, Milton
Nascimento ou Egberto Gismonti. É possível dizer que Villa-Lobos criou
uma possibilidade de música brasileira ao invés de ser criado por ela.
Ele `tornou-se' folclore''.

A ideia não é expor um retrato idealizado de um ``super-herói''
nacional, mas utilizar as contradições na vida e na obra de Villa-Lobos
para expor e refletir as contradições de sua época -- algumas das quais
continuam a ser pertinentes nos dias de hoje.


\subsubsection{Antes da leitura}
%\SideImage{}{PNLD0052-8}

Pedir aos alunos que pesquisem a respeito da presença de Villa-Lobos e
Getúlio Vargas na sua cidade. Há alguma instituição ou logradouro com o
nome de algum desses personagens históricos?

Pesquisar no dicionário o significado das palavras \emph{orfeão} e
\emph{orfeônico.} Como o canto orfeônico foi um programa de educação
musical, pode-se abrir um bate-papo com os alunos sobre este tema. As
escolas devem ensinar música? Ou devem se ater a matérias de ``utilidade
prática''? Existem matérias ``úteis'' e ``inúteis''?

\subsubsection{Durante a leitura}

Heitor Villa-Lobos é o único compositor a merecer um capítulo exclusivo
em \emph{História Concisa da Música Clássica Brasileira.} Além de dar a
conhecer uma figura-chave da cultura em nosso país, a leitura do
capítulo abre a possibilidade de discutir algumas questões de nossa
História e sociedade.

Uma delas é a tensão entre ``popular'' e ``erudito'', entre a prática
artística aceita e a que é socialmente condenada. Embora a força da
produção de Villa-Lobos consista justamente na fusão e na superação
dessas dicotomias, a sociedade brasileira de seu tempo era fortemente
marcada por essas clivagens. Na vida pessoal de Villa-Lobos, isso se
traduz nos mundos em que ele circulava com seu violão, instrumento então
``maldido'', e com seu violoncelo, o instrumento ``respeitável''. Essa
contradição é ilustrada pelo episódio em que Villa-Lobos faz a corte à
sua futura esposa, a pianista Lucília -- ele primeiro aparece na casa
dela de violão, mas depois se oferece para voltar com o violoncelo,
demonstrando a seriedade de suas intenções. Uma leitura complementar
enriquecedora é ``~A~Lição~de~Violão~'', primeiro capítulo de
\emph{Triste Fim de Policarpo Quaresma}, de Lima Barreto, que retrata
com agudeza o estigma contra o instrumento na sociedade brasileira do
começo do século \textsc{xx}.

Outra questão é a relação entre arte e poder. Com o chamado canto
orfeônico, Villa-Lobos implantou um pioneiro programa de educação
musical em massa, em âmbito nacional. Esse programa, porém, ocorreu
durante a ditadura do Estado Novo, de Getúlio Vargas, e o regime
aproveitava as grandes demonstrações de canto coral ao ar livre de
Villa-Lobos como oportunidades de propaganda. É ético servir uma
ditadura, ainda que com propósitos ``nobres''? Quem estava usando quem?
\Image{Propaganda do Estado Novo (c.\,1938) (Wikipedia; Domínio público)}{PNLD0052-35}

Por fim, o caso Villa-Lobos também serve para discutir a questão da
brasilidade. Villa-Lobos foi um nome de destaque da Semana de Arte
Moderna de 1922 e definia-se como um compositor da escola nacionalista.
Mas o que constitui o nacional? O que faz com que alguém seja
brasileiro, ou com que alguma obra de arte o seja? São questões que
podem ser sugeridas aos alunos como temas de reflexão e debate.

\subsubsection{Depois da leitura}

Assistir ao filme \emph{Villa-Lobos, uma Vida de Paixão,} de Zelito
Viana. Cotejar o retrato do compositor nas telas e nas páginas do livro 
e convidar os alunos a uma reflexão sobre os limites da elaboração
artística de eventos da realidade.

O filme de Zelito Viana não é um documentário com imagens de época.
Trata-se de uma cinebiografia, de uma dramatização de fatos e
personagens reais. Qual a liberdade do artista ao abordar eventos e
pessoas que realmente existiram? Os alunos podem ser estimulados a se
lembrarem de outros filmes, séries ou telenovelas basadas em eventos
reais e compararem os resultados.

\paragraph{Tempo estimado:} Duas aulas de 50 minutos.


\section{Atividade \textsc{ii (b)}}
\subsection{“Mulatismo musical”}

\paragraph{Tema} A presença negra na sociedade brasileira e sua manifestação pela música.

%(Habilidades BNCC: 
\BNCC{EM13LP45} 
\BNCC{EM13LP47} 
\BNCC{EM13CHS101}

\BNCC{EM14LP48;
EM13LP51;
EM13CHS102;
EM13CHS104;
EM13CHS105.}

\paragraph{Objetivos e justificativa}

Após aquilatar a musicalidade dos povos originários, é a vez de travar
conhecimento com a participação dos negros na construção da música brasileira.
Em princípio, esta seria uma tarefa mais simples. Pois, enquanto a música
indígena, como vimos, foi sistematicamente apagada de nosso imaginário, a negra
foi historicamente tida como fundamental para a construção de nossa identidade
musical.

Aqui vamos, porém, para além dos estereótipos que reduzem a “influência” negra
apenas aos gêneros de música popular urbana conhecidos e praticados a partir do
século \textsc{xix}. Não se trata de negar a importância ou o valor destes gêneros, bem
como o protagonismo dos negros envolvidos em seu processo de criação. Mas sim
de demonstrar que a importância dos negros na música brasileira é bastante
anterior, e sua contribuição é ainda mais variada do que se imagina. A questão
não é invalidar Pixinguinha, Cartola, Criolo e Emicida, mas apontar para seus
tataravós musicais.

\SideImage{Emicida.}{PNLD0054-21}

Talvez, fora da África, não haja país com maior número de compositores
clássicos negros do que o Brasil. Se formos falar em apagamento, poderíamos
talvez nos referir ao apagamento geral da música clássica brasileira, pouco
tocada mesmo em nosso país. Levando o raciocínio até o fim, esse apagamento é
também o apagamento de uma das esferas da cultura brasileira na qual os negros
sempre tiveram uma importância histórica incontornável e decisiva.


\subsubsection{Antes da leitura}

Fazer um levantamento da história do tráfico negreiro para o Brasil: quando
começou, quando terminou, números de escravos traficados, condições em que eram
transportados etc. O livro \emph{O Trato dos Viventes: Formação do Brasil no
Atlântico Sul}, de Luiz Felipe de Alencastro, pode ser uma boa fonte de
informações objetivas.

Promover entre os alunos uma conversa sobre protagonismo negro. Quantos negros
famosos, do presente ou do passado, eles conhecem? Eles são todos da área da
música ou do esporte, ou também atuam em outros campos? Há ruas ou instituições
com nomes de negros em sua cidade?  Sua cidade já foi governada por um prefeito
negro? Eles conhecem negros em posição de chefia ou comando? Assistiram a
muitos filmes, novelas, animações ou séries com heróis ou protagonistas negros?
Como os negros são normalmente retratados nas obras que leram ou assistiram?

\subsubsection{Durante a leitura}

  Observar como as peculiares condições históricas de Minas Gerais no século
\textsc{xviii} deram origem a um florescimento cultural até então inusitado no Brasil. A
descoberta de metais preciosos deu origem a um fluxo migratório impressionante
e à primeira sociedade urbana de um país até então predominantemente agrícola e
rural.

Notar ainda como a necessidade de extração do ouro requeria braços para o
trabalho pesado, representados pelos escravos negros. E como o medo de
contrabando por parte da Coroa portuguesa levou à expulsão das ordens
religiosas e à organização da vida espiritual por meio de irmandades e ordens
terceiras, que competiam entre si através do luxo das igrejas e das festas
religiosas, das quais a música era elemento central.

\SideImage{Suposto retrato póstumo de Aleijadinho feito no século \textsc{xix} (Euclásio Ventura/ Wikpedia; Dominio público)}{PNLD0052-36}
Falar em igrejas mineiras é falar no Aleijadinho, e vale aí procurar
ilustrações de suas obras espalhadas pelas diversas cidades históricas
mineiras. A importância do Aleijadinho é tão grande que a ele se deve,
indiretamente, a redescoberta do repertório musical de Minas Gerais do século
\textsc{xviii}. Progressivamente esquecida após a decadência da mineração, essa música
só foi reencontrada após a década de 1940, quando o musicólogo germano-uruguaio
Curt Lange visitou Minas Gerais e deduziu que igrejas tão suntuosas deveriam
ter servido de palco a cerimônias religiosas não menos exuberantes,
acompanhadas de muita música.

Também é o momento de recordar a pujante produção literária mineira desta
época e a ligação dos literatos locais – como Basílio da Gama, Cláudio Manuel
da Costa e Tomás Antônio Gonzaga – com a música. À guisa de ilustração, pode
ser lida a ``Lira I'', de \emph{Marília de Dirceu}, de Gonzaga, em que este se gaba da
“destreza” com que toca a “sanfoninha”. Deve-se ainda ressaltar a diferença de
classe entre esses poetas – bacharéis, membros da elite branca –  e os músicos,
de sangue negro.

\subsubsection{Depois da leitura}

Propor que os alunos assistam escutem a pelo menos uma das obras de um
compositor listado no capítulo. Por exemplo, o \emph{Salve Regina}, de Lobo de
Mesquita. E, em seguida, que ouçam várias obras de músicos negros. Daí não deve
haver fronteiras geográficas, cronológicas ou de gênero: a ideia é que seja a
\textit{playlist} mais eclética possível. Talvez seja interessante, por exemplo, ter
algo de Milton Nascimento, que é negro, e de Minas Gerais. Mas a ideia é que os
alunos também possam sugerir músicas e músicos de sua preferência, seja de
funk, axé, \textit{rap}, \textit{hip-hop}, \textit{black music}, enfim, o que calhar. As reflexões aí são
basicamente duas. Uma é o que faz a música de Lobo de Mesquita, ou qualquer
outro compositor estudado no capítulo, ser tão diferente das outras obras
ouvidas. É apenas o afastamento temporal de mais de dois séculos? Ou tem também
a ver com a funcionalidade da obra, ou seja, com uma peça ser feita para
execução litúrgica, e as outras terem finalidade profana? Outra questão é a da
negritude em música. Existe música negra? Caso exista, quais são seus
elementos? O que é negritude em música? Um negro pode fazer música branca? Um
branco pode fazer música negra?

\paragraph{Tempo estimado:} Duas aulas de 50 minutos. 

\section{Atividade \textsc{ii (c)}}

\subsection{“Versalhes tropical” e “Modinhas, lundus e um
imperador-compositor”}

\paragraph{Tema} O profundo impacto da vinda da corte portuguesa em 1808 e suas
consequências diretas na Independência. O Brasil visitado, visto e documentado
por viajantes estrangeiros. 

\BNCC{EM13LP45} 
\BNCC{EM13LP51} 
\BNCC{EM13CHS102}
 
\BNCC{EM13CHS104;
EM13CHS105.}

\paragraph{Objetivos e justificativa}

Difícil exagerar o impacto de 1808 na vida brasileira. Evento único na História
das Américas, a transformação do Rio de Janeiro, subitamente, na capital de um
reino europeu virou do avesso a vida da Colônia. Um choque repentino de
urbanidade operou aquilo que Gilberto Freyre, em \emph{Sobrados e Mucambos}, chamou de
“reeuropeização” do Brasil. A construção de uma “Versalhes tropical” incluía
dotar o Rio de Janeiro de tudo que então era negado aos súditos coloniais –  e
a música foi agente decisivo desse processo “civilizador”.

Além de todas as inovações advindas desse transplante lisboeta na Baía de
Guanabara, a presença da Corte colocou o Rio de Janeiro em um circuito
internacional de visitantes europeus. Nomes como Spix, Martius, Debret e
\SideImage{Dança de bebida dos cordados, do atlas de viagem de Spix e Martius (Wilipedia; Domínio público)}{PNLD0052-38}
Neukomm fizeram um mapeamento e uma documentação sem paralelos da vida natural,
social e cultural do país em nascimento. A familiarização com os trabalhos
destes pioneiros fornecerá ao aluno não apenas elementos para melhor
conhecimento de sua época, como ainda subsídios para refletir sobre sua
reverberação e influência nos dias de hoje.
\Image{Mapa francês da Baía de Guanabara, c. 1555 (Wikipedia; Domínio público)}{PNLD0052-15}

\subsubsection{Antes da leitura}

Fazer um levantamento da jornada pelo Brasil dos naturalistas alemães Johann
Baptist von Spix, biólogo, de Carl Friedrich Martius, botânico. Entre 1817 e
1820, eles percorreram 14 mil quilômetros, em diversos estados brasileiros,
lançando as bases para a divisão dos biomas brasileiros, além de catalogarem
quase metade das espécies de nossas plantas. O resultado consta do alentado
volume \emph{Viagem pelo Brasil}, de 1823. Pode valer a pena, por exemplo, identificar
os cinco biomas de nosso país, que nascem justamente do trabalho destes
naturalistas.

Um ano antes da vinda de Spix e Martius ao Brasil, chegou aqui a assim chamada
Missão Francesa, que incluía, entre outros, o pintor Jean-Baptiste Debret, cuja
Viagem Pitoresca e Histórica ao Brasil (1834-1839) trazia 151 pranchas
retratando o cotidiano brasileiro. Os alunos devem ser estimulados a
pesquisarem as gravuras brasileiras de Debret e a refletirem sobre a pioneira
história da vida privada no país que delas emerge.

\subsubsection{Durante a leitura}

Dimensionar o impacto da transferência súbita e repentina da corte portuguesa
para o Rio de Janeiro. A começar do ponto de vista puramente demográfico: entre
os censos de 1799 e 1821, a população da cidade salta de 43 mil para 79 mil
habitantes, e o contingente de livres (ou seja, de brancos de origem europeia)
mais do que dobra, de 20 mil para 43 mil. 

Isso é acompanhado, obviamente, de um impacto cultural. Se (como lemos no
capítulo ``Dois séculos sem partituras'') a tipografia que Antônio Isidoro da
Fonseca abriu na cidade em 1747 fora prontamente fechada pelas autoridades,
agora funda-se uma Imprensa Régia, periódicos, a Real Biblioteca. Em \emph{Formação
da literatura brasileira: momentos decisivos, 1750-1880}, Antonio Candido
identifica nessa época o “aparecimento dos primeiros públicos consumidores
regulares de arte e literatura; a definição da posição social do intelectual; a
aquisição, por parte dele, de hábitos e características mentais que o marcariam
até nossos dias”. De acordo com o mesmo autor, “termina a hegemonia dos
conventos e se organiza o pensamento livre”; “progresso decisivo é a fundação
de cursos técnicos e superiores”. A Imprensa Régia e as tipografias publicam
“principalmente trabalhos oficiais e de utilidade para o ensino, bem como os
periódicos”, e a entrada de livros, que até então era clandestina, aumenta com
a abertura dos portos em 1810. Entre 1807 e 1817, o Rio de Janeiro é dotado de
“quatro livrarias mal fornecidas”, número que dobraria em 1821, um cenário, de
qualquer forma, ainda superior ao que ocorre no resto da Colônia. No Recife,
por exemplo, em 1815, “havia apenas uma porta onde se vendiam livros
religiosos”.

Para uma dinastia tão intimamente ligada à música quanto à dos Bragança, esse
seria um dos principais instrumentos “civilizadores” do Novo Mundo. Não custa
lembrar que D. Pedro I, primeiro imperador do Brasil, era músico, e chegou a
ter uma obra executada em Paris. Em 1813, inaugurou-se o Real Teatro de São
João, réplica do Teatro de São Carlos, de Lisboa. E veio para cá também a
Capela Real, formada por cantores e instrumentistas cujo nível era tido como
equivalente ao que melhor havia na Europa.

As disputas em torno da chefia da instituição no Brasil ajudam a revelar as
tensões raciais do país na época. Ao aqui chegarem, os músicos foram colocados
sob a chefia do Padre José Maurício, um talentosíssimo regente, compositor e
tecladista, que tivera que pedir dispensa do “defeito de cor” para ingressar na
carreira eclesiástica, pois era negro. Seu mandato, contudo, durou apenas três
anos: ele teve que ceder o cargo quando chegou, da Europa, Marcos Portugal,
branco e português.

Testemunha desses eventos é o compositor que pode ser considerado como uma
espécie de equivalente musical de Spix, Martius e Debret: o austríaco Sigismund
Neukomm, discípulo de Haydn. Neukomm escreveu cerca de 60 obras em seus cinco
anos de permanência em solo nacional e deixou relatos valiosos sobre a vida
musical brasileira desta época.

É digno de ser ressaltado o fato de que Spix e Martius, além de seu
impressionante levantamento da fauna e da flora nacional, incluíram em seu
trabalho um anexo musical multiétnico, que abarca melodias indígenas e um
exemplar da dança afrobrasileira conhecida como lundu (chamado pelos
naturalistas alemães de landum). Se o lundu pode ser considerado uma espécie de
avô do samba, florescia nessa época também aquele que talvez tenha sido o
primeiro gênero musical nascido no Brasil e exportado para o exterior: a
modinha, canção de salão com temática normalmente sentimental, no meio do
caminho entre o popular e o erudito, que talvez possa ser vista como a
ancestral do cancioneiro amoroso da MPB.

\subsubsection{Depois da leitura}

Ouvir o lundu (ou landum) coletado por Spix e Martius, e uma modinha do Padre
José Maurício – como, por exemplo, \emph{Beijo a mão que me condena}. Discutir se é
possível indicar influência dessas obras ou desses gêneros na música brasileira
posterior, ou se trata-se simplesmente de documentos de suas épocas, já muito
afastados de nossa realidade contemporânea. Ouvir ainda \emph{Amor Brasileiro}, de
Neukomm, em perspectiva com o que foi visto dos trabalhos de Debret, Spix e
Martius. Afinal, também trata-se do Brasil visto sob a perspectiva de um
europeu. Essa obra soa brasileira para nós? Existe um jeito brasileiro de se
fazer música? Um austríaco, ou qualquer estrangeiro, pode fazer música
brasileira? E isso poderia ser extrapolado para outras áreas do conhecimento?
Os trabalhos de Debret, Spix e Martius sobre o Brasil são menos autênticos por
terem sido fruto de mentes europeias?

\paragraph{Tempo estimado} Duas aulas de 50 minutos. 

 
\section{Atividade \textsc{ii (d)}}
\subsection{“A ópera nacional e Carlos Gomes”}


\paragraph{Tema} A realização de uma ópera nacional como o produto estético que melhor
traduz as aspirações do Segundo Império no Brasil 


\BNCC{EM13LP45}
\BNCC{EM13LP51} 
\BNCC{EM13CHS102} 
\BNCC{EM13CHS104;
EM13CHS105.}

\paragraph{Objetivos e justificativa}

Se hoje a ópera é vista como uma manifestação artística essencialmente “de
elite”, no século \textsc{xix} ela era o principal entretenimento urbano – cumprindo
funções que, posteriormente, seria assumidas pelo rádio, televisão, cinema,
internet e esportes de massas. A ópera, portanto, era mais do que ópera – era
uma arena pública em que estava em jogo o próprio imaginário nacional.

Se, ao longo do século \textsc{xviii}, ópera era, essencialmente, ópera em italiano, a
partir do século \textsc{xix} diversas nações passaram a se empenhar em produzir óperas
em suas próprias línguas – como uma afirmação de sua identidade nacional. No
Brasil do século \textsc{xix} não foi diferente, e a iniciativa de criação da Ópera
Nacional deu origem à grande figura pública de nosso país na época, o primeiro
brasileiro a virar uma figura de destaque internacional: o compositor Carlos
Gomes.
\SideImage{Estátua de Carlos Gomes na Cinelândia, Rio de Janeiro (Wikipedia; Domínio púlbico)}{PNLD0052-30}


Assim, ao compreender o real significado da ópera no Segundo Império, o aluno
não apenas adquire uma compreensão mais profunda desse período de nossa
história, como ainda se capacita para discutir a constante questão da
identidade nacional.


\subsubsection{Antes da leitura}

Pesquisar a definição de ópera e sua história. Notar que a ópera surge, por
volta de 1600, como uma manifestação artística aristocrática, feita apenas no
âmbito das cortes. Contudo, rapidamente transforma-se em um espetáculo encenado
em teatros, com cobrança de ingressos, adquirindo uma feição extremamente
comercial. Era preciso vender entradas, encher as salas de espetáculos, atrair
o público: a ópera caçava audiência, exatamente como a televisão de hoje. E
assim ganhou o planeta. Capítulos de \emph{História Concisa da Música
Clássica Brasileira} (como ``As Primeiras Obras'' e ``Mulatismo Musical'') mostram como,
no período colonial, havia teatros chamados de “casa de ópera” em todo o
território nacional, de Belém a Porto Alegre, passando por Cuiabá, Vila Rica
(hoje Ouro Preto), São Paulo e Rio de Janeiro.
\Image{Detalhe de Ouro Preto com a Igreja do Carmo (Wikipedia; Domínio público)}{PNLD0052-37}

Como os personagens das óperas de Carlos Gomes são índios que cantam em
italiano, é importante observar ainda que o italiano foi durante muito tempo a
língua franca da ópera. Assim, no século \textsc{xviii}, em Londres, Händel era um
alemão compondo óperas para um público inglês. Contudo, essas óperas não eram
cantadas no idioma nem do autor, nem no da plateia, mas em italiano. Uma
analogia pode ser feita com a música pop dos séculos \textsc{xx} e \textsc{xxi}: de acordo com
seu gosto musical (ou sugestões dos alunos), o professor pode apontar diversas
bandas que cantaram em inglês, embora este não fosse seu idioma nativo, como
Abba (sueca), Scorpions (alemã), A-ha (norueguesa), Sepultura (brasileira), BTS
(coreana) etc. 

Ao longo do século \textsc{xix}, como parte do processo de construção de identidade
nacional, os países passam a promover a composição de óperas em seus próprios
idiomas, bem como a encenar títulos estrangeiros em tradução. Pode ser feito um
paralelo com o atual cenário dos musicais estrangeiros, que, quando encenados
no Brasil, têm seus textos vertidos para o português. Versões de canções
estrangeiras são relativamente comuns no universo da música popular, e a
comparação de algumas com seus equivalentes em português pode gerar ricas
discussões não apenas sobre fidelidade da tradução (afinal, para além de
transmitir o significado das palavras, a versão em outra língua deve se
submeter às exigências da música), como o quanto a sonoridade de uma melodia se
modifica (com eventuais perdas ou ganhos) quando se altera o idioma no qual ela
originalmente foi composta. 

\subsubsection{Durante a leitura}

Mostrar a força que a ópera adquire no Brasil, especialmente, a partir da
década de 1840, com a superação do clima de instabilidade política do período
da Regência e a coroação de D. Pedro II.

Sublinhar a reação apaixonada do público que comparecia aos espetáculos, que
não seria exagero talvez comparar aos modernos estádios de futebol brasileiros.
Destacar o exemplo do regulamento da Companhia do Teatro de São Luiz,
estabelecida em 1855 na capital maranhense, que previa prisão de oito dias para
quem lançasse pedras, moedas ou laranjas em direção ao palco, ou partindo
deste. Ou da administração do Teatro da Paz, de Belém, que proibia ``as
assoadas, gritos ou quaisquer outros atos contrários à ordem, sossego e
decência'', prevendo ainda que o espectador que infringisse as normas fosse
admoestado, podendo ser retirado do teatro em consequência de seu
comportamento. Jornais paraenses da época relatam “guardas urbanos, de sabres
desembainhados […] fazendo evacuar os camarotes e lançando fora do teatro
pessoas que haviam pago para entrar”.

Evidenciar ainda o impacto da nova tecnologia da navegação a vapor na inserção
do Brasil no circuito global da ópera, com companhias estrangeiras visitando
não apenas o Rio de Janeiro (nossa capital à época), mas também cidades do
Nordeste, como Salvador, Recife e São Luiz. Essa inserção chegou a chamar a
atenção na cidade que ditava os padrões de comportamento do Brasil do século
\textsc{xix}: Paris, onde um dos maiores compositores de todos os tempos, Berlioz,
escreveu um artigo sobre a vida operística do Rio. Seria algo como, hoje, o
cinema brasileiro virar notícia em Nova York.

Destacar ainda como esse protagonismo da ópera surge nos escritos do principal
escritor brasileiro da época, Machado de Assis. Os alunos podem ler ``A Ópera'',
capítulo \textsc{ix} do icônico romance machadiano \emph{Dom Casmurro}, e lá encontrarão toda
uma concepção da existência humana amparada nas convenções da ópera de seu
tempo.

Machado de Assis, inclusive, atuou fazendo versões para o português de óperas
estrangeiras na Imperial Academia de Música e Ópera Nacional, a iniciativa que
tornou conhecido o talento de um dos principais compositores brasileiros de
todos os tempos: Carlos Gomes.

Chamar a atenção dos estudantes para como o exemplo pessoal de Carlos Gomes
revela a complexidade das relações raciais no Brasil do Segundo Império. Embora
a tez escura evidenciasse antepassados negros (seu pai, na infância, era
descrito como “pardo”), ele oficialmente se dizia descendente de um nobre
espanhol de Pamplona e de “uma belíssima índia, filha de um chefe Guarany”,
segundo escreveu sua filha, Ítala. E os heróis de suas óperas mais conhecidas
seriam índios cantando em italiano. Não apenas \emph{Il Guarany}, mas também \emph{Lo
Schiavo} – ou seja, \emph{O Escravo}, ópera supostamente abolicionista, estreada no Rio
de Janeiro em 1889, após a proclamação da Lei Áurea, cujo protagonista,
contudo, não é um escravo negro, mas sim um índio.
\SideImage{Desenho para a capa do libreto de  (Wikipedia; Domínio público)}{PNLD0052-31}

Índio também é Pery, o mais célebre herói das óperas de Carlos Gomes, em Il
Guarany, inspirada no romance homônimo de José de Alencar. Vale ressaltar que
uma das características do romantismo brasileiro era a idealização do indígena,
que correspondia, grosso modo, à glorificação da Idade Média que marcou o
movimento análogo na Europa. Os alunos podem ler trechos de algumas das obras
literárias  típicas dessa manifestação, como os poemas \emph{épicos I-Juca-Pirama}
(1851), de Gonçalves Dias, e A Confederação dos Tamoios (1857), de Gonçalves de
Magalhães.

Batizando hoje um time de futebol na cidade natal do compositor – Campinas –, a
ópera mais famosa de Carlos Gomes, . Ela estreou no supremo palco de ópera da Itália no
século \textsc{xix}: o Teatro alla Scala, de Milão. Notar que, para uma nação nascente,
como o Brasil, ainda em busca de afirmação internacional, o impacto desse
triunfo podia ser considerado equivalente, no século \textsc{xx}, à conquista da
primeira Copa do Mundo pela Seleção brasileira. Mais do que um músico, Carlos
Gomes pode ser visto talvez como o precursor de alguém como Pelé, ou Ayrton
Senna: um brasileiro que tem sucesso reconhecido no mundo inteiro. Na
observação de Lorenzo Mammì, “se o Segundo Reinado se caracteriza justamente
pela tentativa de construir um perfil cultural nacional, cimentando traços
locais com uma linguagem internacional mais ou menos atualizada, pode se dizer
que \emph{Il Guarany} é seu produto artístico mais bem-sucedido”.

\subsubsection{Depois da leitura}

A abertura da ópera \textit{Il Guarany} chegou a ser considerada uma espécie de
“segundo hino nacional brasileiro”, devido à sua utilização como prefixo do
programa radiofônico A Voz do Brasil. Propor aos alunos que ouçam ou assistam a
trechos da ópera  (há uma 
\href{https://www.youtube.com/watch?v=XTIpAyXyvFA&t=3746s}{versão completa no YouTube}. 
com legendas em
português).
E que cotejem
com excertos do romance de José de Alencar. A partir daí, há algumas linhas
possíveis de reflexão. Uma é a adaptação de uma obra de um meio a outro – no
caso, de um livro para ópera. E paralelos podem ser feitos com um tipo de
adaptação bem mais familiar aos alunos: quando uma obra literária é filmada
(seja longa-metragem, série ou novela televisiva). 

  Outra linha seria a reflexão sobre a identidade nacional. Os brasileiros do
século \textsc{xix} se sentiram representados por uma ópera como . Ao ouvir
hoje esses índios cantando em italiano, sentimos essa música como nossa? Mesmo
falando em português, os personagens de Alencar soam próximos ao leitor de
hoje? E é possível colocar em pauta até a questão do lugar de fala. O quanto
alguém como José de Alencar ou Carlos Gomes estava apto para falar dos
indígenas? É preciso pertencer a um grupo ou comunidade para poder traduzi-los
esteticamente com propriedade?

\paragraph{Tempo estimado} Duas aulas de 50 minutos.


\section{Aprofundamento}

A estratégia de aproximação do leitor leigo por parte do jornalista e
escritor Irineu Franco Perpetuo evidencia-se já na escolha do título de
\emph{História Concisa da Música Clássica Brasileira}. Embora, em seus
textos jornalísticos, o autor já tenha empregado inúmeras vezes a
expressão ``música erudita'', esta lhe soa algo distanciadora e pedante.
Assim, ele resolveu adotar a fórmula ``música clássica'', que tem
aceitação internacional e, ao mesmo tempo, ajuda a delimitar o objeto da
obra.

Pois, no Brasil, há muitos livros cujos títulos variam em torno da
expressão \emph{História da Música Brasileira,} sem especificação do
estilo. Porém, quando o leitor se aproxima, acaba descobrindo que ali
está contemplada, essencialmente, a música clássica. Como se essa fosse
a única forma legítima ou possível de musicalidade, enquanto, na
verdade, quando se fala simplesmente de ``música brasileira'', sem
nenhum outro adjetivo, é a produção popular que imediatamente vem à
cabeça da maioria, seja no Brasil, seja no exterior.

\SideImage{Violino.}{PNLD0052-09}

Não que isso seja injusto. Nossa música popular não apenas merece
destaque dentre as outras ``músicas populares'' do planeta por sua
especial riqueza, sofisticação e variedade, como ainda se tornou uma das
mais divulgadas e respeitadas manifestações da arte brasileira em todo o
mundo. A produção musical do Brasil conquistou o planeta com seu
colorido, seu vigor e sua originalidade. Do samba, choro e maxixe do
passado ao funk, axé e hip hop de hoje, nossa música popular vem se
reinventando com inesgotável criatividade.

Essa produção, por sinal, sempre esteve em diálogo constante com nossa
música de concerto, e isso transparece nas páginas da \emph{História
Concisa da Música Clássica Brasileira.} O livro trata do nascimento da
modinha e do lundu, passa pelas origens do choro, fala do primeiro samba
escrito em partitura, visita a efervescente Era dos Festivais, da década
de 1960, mapeia a contribuição fundamental de arranjadores de formação
erudita para o Tropicalismo, conta como Tom Jobim virou o autor de uma
sinfonia para a nova capital do Brasil e descreve o tipo de música
eletrônica que se fazia de forma experimental antes desse gênero ganhar
as pistas de dança de nossas cidades. Afinal, em uma época em que a
técnica permite compor diretamente o som, em estúdio, parece cada vez
mais difusa a fronteira entre uma tradição de uma música dita
``letrada'', produzida, conservada e interpretada por meio de
partituras, e uma outra mais ligada a práticas um dia transmitidas
oralmente (mas cada vez mais cristalizadas em livros, gravações e também
partituras, estudadas e ensinadas em universidade por eruditos na plena
acepção do termo).

De qualquer maneira, a música popular brasileira aparece aqui
pontualmente. O foco do livro não é, digamos, o forró, o jongo ou os
violeiros caipiras, mas sim as missas, sinfonias, quartetos de cordas e
óperas de autores brasileiros. Trata-se da história da ``outra'' música
brasileira. A ``clássica''. A impopular -- não no sentido de ser contra
o povo, mas de não ser conhecida.

Pois a magnitude e a difusão do nome de Villa-Lobos constitui antes
exceção do que regra no cenário musical brasileiro. Em \emph{O Resto é
Ruído -- Escutando o Século \textsc{xix}}, Alex Ross chama os compositores
clássicos norte-americanos de ``homens invisíveis'', lutando pela
visibilidade de sua arte. ``Cada geração tem de refazer todo o trabalho.
Os compositores têm uma eterna carência de apoio do Estado. Falta-lhes
um público mais amplo; faltam-lhes séculos de tradição'', escreve Ross,
em linhas que também parecem ter validade do lado de baixo do Equador.

Nossos compositores que não se chamam Villa-Lobos vivem eternamente
tendo que ser ``resgatados'' por musicólogos e intérpretes abnegados,
que teimam em não deixar apagar a memória de uma produção contínua e de
qualidade, que começou na época colonial, chegando até os nossos tempos.
Em maior ou menor grau, poderiam todos repetir a indagação que João
Silvério Trevisan coloca na boca de Alberto Nepomuceno no romance
\emph{Ana em Veneza:} ``continuarei sendo punido ou quem sabe terei
ouvintes no futuro?''

Além do título, outra estratégia do autor para tornar a obra acessível
ao leitor ``leigo'', não-familiarizado com o jargão musical específico,
foi minimizar o uso deste jargão, e não inserir exemplos musicais em
pentagrama -- de resto, com as facilidades tecnológicas de hoje, é
relativamente simples entrar na internet e ouvir pelo menos trechos das
obras citadas. Trata-se de um recurso bastante recomendável para ser
adotado em sala de aula, por sinal, e o Caderno de Atividades traz
algumas sugestões de escuta. Ademais, o autor não possui pretensões de
exaurir o tema. Entraram os compositores e obras que ele considerou
``fundamentais'' (procurando colocar alguma objetividade dentro de
escolhas fadadas inexoravelmente à subjetividade); contudo, omissões (e
as injustiças que elas acarretam) serão sempre inevitáveis em trabalhos
dessa natureza.

Sendo cada partitura um dado cultural, e não uma mera coleção de folhas
soltas no tempo e no espaço, o autor procurou relacionar nossa música
``clássica'' com os fatos históricos e culturais de sua época (inclusive
a música popular) e situar seus criadores no ambiente intelectual e
social brasileiro. E isso é que torna o livro especialmente adequado
para a utilização em sala de aula.

Com vocação transversal e multidisciplinar, \emph{História Concisa da
Música Clássica Brasileira} apresenta personagens que o aluno já conhece
de outras disciplinas sob novos prismas. Ele vai aprender, por exemplo,
que D. Pedro I, nosso primeiro imperador, foi também compositor, e que
até teve uma obra tocada em Paris. Que Machado de Assis, além de
escrever contos e romances fundamentais, como \emph{O Alienista, Dom
Casmurro} e \emph{Memórias Póstumas de Brás Cubas,} esteve envolvido na
atividade operística de seu tempo, traduzindo óperas para o português.
Que os alemães Spix e Martius, fundamentais no mapeamento da fauna e da
flora de nosso país, e na definição de nossos biomas, também tiveram um
papel decisivo na documentação de nosso folclore musical. E que Mário de
Andrade, ademais de poeta e autor de \emph{Macunaíma,} foi o ``guru'' e
inspirador de alguns dos principais compositores brasileiros do século
\textsc{xx}.

Em outras palavras, o livro encara a música como um fenômeno geral da
cultura brasileira. Assim, contar a História de nossa música é também
contar a História de nossa cultura. Com o título inspirado na
\emph{História Concisa da Literatura Brasileira,} de Alfredo Bosi
(embora sem a pretensão de emular esse livro tão conhecido dos
professores de literatura), a obra traça paralelos inescapáveis entre o
mundo dos sons e o mundo das letras. Gregório de Mattos, o ``Boca do
Inferno'', aparece, por exemplo, como um de nossos primeiros cronistas
musicais -- e são citados trechos de poemas de sua autoria que
documentam a música brasileira de sua época. A música que enchia as
suntuosas igrejas mineiras do Aleijadinho é vista como inseparável da
frutífera atividade literária dos poetas da Inconfidência. Descobre-se
que a ópera italiana era o modelo estético de escritores como Tomás
Antônio Gonzaga, Cláudio Manuel da Costa e Brasílio da Gama, e que os
libretos de ópera italiana do século \textsc{xviii} circulavam do lado de cá do
Atlântico em adaptação, como literatura de cordel. Isso para não falar
da importância decisiva da música na Semana de Arte Moderna de 1922.

Claro que um livro que traz a palavra ``História'' no nome também trata
bastante de perto os acontecimentos históricos que marcaram a vida de
nossa nação ao longo dos séculos. Podemos dizer que \emph{História
Concisa da Música Clássica Brasileira} conduz e propõe a ``trilha
sonora'' da História do Brasil. Ao contar, por exemplo, a história da
composição de nossos hinos oficiais -- o \emph{Hino à Independência,
Hino Nacional, Hino da Proclamação da República} e \emph{Hino à Bandeira
--,} a obra revisita o contexto em que essas partituras foram
encomendadas, escritas, modificadas e executadas.

\emph{História Concisa da Música Clássica Brasileira} tem preocupação
não apenas com diversidade estética, mas também regional, de gênero e
racial. Assim, ao abordar os eventos de 1500, mostra não apenas a música
que os portugueses trouxeram ao chegar por aqui, mas também o que é
possível saber da musicalidade dos povos originários. Reflete também
sobre o que há de racial no apagamento da música brasileira de concerto.
Pois talvez não seja exagerado afirmar que, fora da África, o país com
maior número de compositores clássicos negros em sua História é o
Brasil. Nossa exuberante vida musical dos tempos da Colônia foi tocada,
basicamente, por compositores negros, cujas obras caíram no esquecimento
ao longo do século \textsc{xix}, e só foi ser resgatada paulatinamente a partir
da década de 1940.

Descobrir a música erudita brasileira é conhecer mais um aspecto do rico
fazer artístico dos negros no Brasil. O livro também demonstra como, a
partir do pioneirismo de Chiquinha Gonzaga, as mulheres compositoras
passaram a lutar por protagonismo em nossa vida musical. Mulheres
compositoras ainda são minoritárias nas programações de concerto, não
apenas no Brasil, como em todo o planeta. O livro faz um esforço de
mapear e aquilatar a produção de nossas compositoras. E busca não se
restringir ao eixo Rio-São Paulo, ou à região Sudeste, empenhando-se em
documentar nossa música em âmbito nacional.

Música erudita contemporânea é uma espécie de gueto dentro do gueto, um
enclave minoritário em um território minúsculo. \emph{História Concisa
da Música Clássica Brasileira} tenta contemplar os compositores vivos,
de hoje, em todo o país, bem como a pujança de nossa produção crítica
universitária. O número de estudos acadêmicos relevantes e profundos
sobre a música brasileira cresceu de maneira avassaladora nas primeiras
décadas do terceiro milênio, e apenas uma lista de todos já superaria em
muito o tamanho de uma história que ser pretende concisa. Nessa área,
não foi possível fazer mais do que roçar a ponta de um iceberg de
dimensões incomensuráveis.

Não custa lembrar, ainda que o livro foi viabilizado graças a uma
campanha de financiamento coletivo, na internet. Além da contribuição
pecuniária, muitos estudiosos, compositores e intérpretes aproveitaram a
ocasião para botar à disposição do autor seus ricos acervos e
conhecimentos, aportando dicas inestimáveis, que acabaram por enriquecer
o trabalho de forma decisiva.

\section{Sugestões de referências complementares}

\begin{itemize}
\item 
\emph{Villa-Lobos -- uma vida de paixão}. Direção de Zelito Viana. Rio
de Janeiro, Mapa Filmes, 2000.

Além de comparar o Villa-Lobos das telas com o compositor das
páginas do livro, os estudantes poderão fazer um exercício crítico de
reflexão sobre as possibilidades e limites da dramatização de eventos
reais. O filme de Zelito Viana é documentário? É ficção? Qual o limite
entre esses gêneros? Disponível \href{https://www.youtube.com/watch?v=5WFFxGjxUv4
}{neste link}.
\end{itemize}

\section{Bibliografia comentada}

\begin{itemize}

\item 
\textsc{azevedo}, Luiz Heitor Corrêa de. \emph{150 anos de música no Brasil
(1800-1950).} 2\textsuperscript{a} Edição. Rio de Janeiro: FBN,
Coordenadoria de editoração, 2016.


Um dos principais musicólogos brasileiros, Luiz Heitor escreveu,
na década de 1950, esta obra pioneira, que continua servindo como
referência na área.

\item 
\textsc{diniz}, Edinha. \emph{Chiquinha Gonzaga: uma história de vida}. Rio de
Janeiro: Jorge Zahar, 2009.

Um apanhado de fôlego da trajetória da maestrina pioneira do
Brasil -- Chiquinha Gonzaga.

\item 
\textsc{machado}, Cacá. \emph{O enigma do homem célebre: ambição e vocação de
Ernesto Nazareth}. São Paulo: Instituto Moreira Salles, 2007.

Uma reflexão sobre os impasses e dilemas que marcaram a carreira
de Ernesto Nazareth.

\item 
\textsc{zanon}, Fabio. \emph{Folha explica Villa-Lobos.} São Paulo: Publifolha,
2009.

Uma excelente síntese resumida e acessível da vida e obra de
Villa-Lobos, por um de seus mais destacados intérpretes: o violonista
Fabio Zanon, que gravou a obra do compositor com enorme êxito
internacional.
\end{itemize}

\end{document}
