\documentclass[12pt]{extarticle}
\usepackage{manualdoprofessor}
\usepackage{fichatecnica}
\usepackage{lipsum,media9,graficos}
\usepackage[justification=raggedright]{caption}
\usepackage[one]{bncc}
\usepackage[alameda]{../edlab}

\begin{document}


\newcommand{\AutorLivro}{Irineu Franco Perpétuo}
\newcommand{\TituloLivro}{História concisa da música clássica brasileira}
\newcommand{\Tema}{Diálogos com a sociologia e com a antropologia}
\newcommand{\Genero}{Diário, biografia, autobiografia, relatos, memórias}
\newcommand{\imagemCapa}{./images/PNLD0052-01.png}
\newcommand{\issnppub}{---}
\newcommand{\issnepub}{---}
% \newcommand{\fichacatalografica}{PNLD0052-00.png}
\newcommand{\colaborador}{\textbf{Fulano de Tal} é uma pessoa incrível e vai fazer um bom serviço.}


\title{\TituloLivro}
\author{\AutorLivro}
\def\authornotes{\colaborador}

\date{}
\maketitle

\baselineskip=1.2\baselineskip\par

\begin{abstract}

Este manual estabelecerá pontos de contato entre a matéria apresentada
neste livro e você, que é quem de fato coloca em prática o processo
educativo.

Em \emph{História Concisa da Música Clássica Brasileira,} o crítico
musical Irineu Franco Perpetuo narra, em linguagem simples e acessível,
a trajetória da música nos mais de 500 anos que se seguiram após a
chegada ao território em que estamos dos conquistadores portugueses.

Não é segredo para ninguém que o Brasil possui uma produção musical das
mais ricas e exuberantes. A música brasileira conquistou o planeta com
seu colorido, seu vigor e sua originalidade. Talvez não seja exagero
afirmar que a manifestação artística do Brasil mais conhecida no
exterior é a música.

Por outro lado, quando se fala em música brasileira, a primeira que vem
à cabeça de todo mundo é a música popular. Do samba, choro e maxixe do
passado, ao funk, axé e hip-hop de hoje, nossa música popular vem se
reinventando com inesgotável criatividade.

Só que o Brasil tem também uma outra música, em constante diálogo com
essa. A música de Villa-Lobos, Carlos Gomes, Chiquinha Gonzaga, Ernesto
Nazareth. Uma música que vem sendo escrita em partitura há séculos, e
acompanhando cada passo de nossa história. Nosso livro conta a história
dessa música. E, se a música é uma das maneiras mais lúdicas de se
estudar, podemos dizer que \emph{História Concisa da Música Clássica no
Brasil} propõe a trilha sonora que vem acompanhando nossos cinco séculos
de desenvolvimento histórico e cultural.

``Música clássica'' pode ser uma expressão intimidadora para quem não
pertence a este universo. O livro, contudo, evita termos técnicos e o
jargão especializado. Além disso, tem vocação transversal e
multidisciplinar. O autor não encara a música como fenômeno isolado,
porém relaciona-a constantemente com outras esferas do conhecimento. Com
o título inspirado na \emph{História Concisa da Literatura Brasileira,}
de Alfredo Bosi, o livro traça paralelos constantes entre o mundo dos
sons e o mundo das letras. O papel da música nos momentos-chave da nossa
História também é contemplado. A música popular, embora não seja o foco
principal da obra, também é abordada, especialmente quando sua
trajetória acaba se cruzando com os caminhos da música erudita. E, por
fim, a obra almeja não apenas variedade estilística, como também a
regional, a de gênero e a racial. Aborda-se não apenas a música que veio
com os portugueses, como trata também da sonoridade dos povos
originários; fala-se dos inúmeros compositores negros do Brasil; e
também da luta constante das mulheres para adquirirem protagonismo em
nosso cenário musical.

Aqui será fornecida uma série de questões, propostas de atividades e
materiais suplementares que permitirão, através da música, conhecer a
história, a realidade social e a evolução cultural de nosso país ao
longo dos séculos -- chegando, inclusive, aos dias de hoje. Vale lembrar
que os recursos da internet tornam extremamente simples o acesso a
gravações em vídeo e áudio das obras e compositores apresentados no
livro. Ouvindo a música brasileira do passado, podemos não apenas
compreender melhor o momento histórico em que ela foi produzida, como
refletir sobre sua evolução até se tornar a música que ouvimos e
produzimos hoje.

Esperamos que você ache útil e divertido este material para o bom
trabalho em sala de aula!

\end{abstract}

\tableofcontents


\section{Atividade I: ``Polcas, maxixes e tangos brasileiros''}

\paragraph{Tema:} As fronteiras entre popular e erudito. A luta das
mulheres por emancipação, no exemplo pioneiro de Chiquinha Gonzaga.


\BNCC{EM13LP45}
\BNCC{EM13LP47}
\BNCC{EM13LP50}
\BNCCC{EM13LP51;
EM13CHS101;
EM13CHS102;
EM13CHS104;
EM13CHS105.}

\paragraph{Objetivos e justificativa}

Para abordar a produção das mulheres na música clássica brasileira,
tomamos o caso emblemático de Chiquinha Gonzaga. Filha de um militar
próximo ao Duque de Caxias (1803-1880), que participaria da guerra
contra o Paraguai, Francisca Edwiges Gonzaga mereceu o repúdio da
família depois de abandonar o marido, com o qual se casara aos 16 anos
de idade. Viveu sempre de música e, aos 52 anos, uniu-se a um homem de
16, que apresentava a todos como seu filho.
\SideImage{Chiquinha Gonzaga (1847--1935) aos 18 anos de idade.}{PNLD0052-03.png}

O caso de Chuiquinha permite abordar as tensões entre ``popular'' e
``erudito'', entre ``respeitável'' e ``indecente'' -- entre ``brega'' e
``chique''. Ela ganhou a alcunha de ``Offenbach de saias'' pela
facilidade com que criava partituras para as diversas formas de teatro
musical ligeiro que, por influência francesa, floresciam no Rio de
Janeiro a partir da inauguração, em 1861, do café-cantante Alcázar
Lyrique (mais tarde, Teatro Lírico Francês). Esse gênero teatral causava
sensação e escândalo com o humor de duplo sentido e o apelo sensual das
vedetes. Era o tempo da opereta, da mágica (assim chamada devido à
utilização de temas fantásticos), da burleta e do teatro de revista --
que levava esse nome por ``passar em revista'' os principais
acontecimentos do ano.

Nessa época fértil, que marca o florescimento de gêneros fundamentais da
nossa música popular, como o choro e o samba, as tensões latentes na
produção de Chiquinha Gonzaga tornam-se ainda mais evidentes quando
entendidas ao lado de seu contemporâneo Ernesto Nazareth. Sem recursos
para perseguir o sonho de se tornar pianista de concerto, Nazareth
lecionava ao teclado e vivia do ofício que então era chamado, algo
pejorativamente, de \emph{pianeiro}: dedilhando o instrumento em casas
de venda de música (como demonstrador, executando as partituras
negociadas a seus eventuais compradores) e em cinemas, como o Odeon, o
que daria título a um de seus mais célebres tangos

% \paragraph{Metodologia}


\subsection{Antes da leitura}
\SideImage{}{PNLD0052-07}
Pedir que os alunos procurem no dicionário e anotem os significados das
palavras \emph{polca, maxixe} e \emph{tango.} Promover um bate-papo com
os estudantes, perguntando a diferença entre música popular e música
claśsica, e se já tiveram alguma experiência com música clássica.

As questões se multiplicam: na sua época, o maxixe era tido como
``maldito''. Poderia ser feito um paralelo com o que acontece hoje com o
funk? Existem gêneros musicais ``nobres'' e outros ``respeitáveis''? Um
tipo de música é melhor do que outro? Quem decide o que é música
``boa''? A música ``maldita'' de hoje é a música ``boa'' de amanhã?

\subsection{Durante a leitura}

Analisar a trajetória de Chiquinha Gonzaga e seu pioneirismo na
sociedade brasileira da segunda metade do século XIX e início do século
XX. Como o escândalo que ela causava entre as mentes mais conservadoras
de sua época revela a posição subordinada a que estavam sujeitas as
mulheres no Brasil de então. É possível traçar um paralelo entre
Chiquinha Gonzaga, a primeira maestrina brasileira, e sua contemporânea
Nair de Teffé, pioneira do cartum no Brasil, que, quando era a
primeira-dama da nação, causou alvoroço ao tocar a música de Chiquinha
em uma recepção diplomática.

Perceber, no exemplo do compositor francês Darius Milhaud, que visitou o
Brasil, que a elite brasileira de então atrás estava mais interessada
pelo que acontecia na França do que em seu próprio país. Trocando França
por Estados Unidos, seria possível fazer uma analogia com a realidade
brasileira de hoje?

Enfatizar as tensões entre ``popular'' e ``erudito'', entre ``maldito''
e ``respeitável'', no caso de Ernesto Nazareth. Nazareth entrou para a
História como um notável autor de obras de inspiração popular, mas
padeceu sempre da frustração de não ter sido um pianista de concerto,
que tocava Chopin no Teatro Municipal. Essa questão pode ser aprofundada
com a leitora do conto \emph{Um Homem Célebre,} de Machado de Assis,
cujo protagonista é um músico que enfrenta dilemas similares aos de
Nazareth.

Chamar ainda a atenção dos alunos para a existência de um ``tango
brasileiro'', bastante diferente do célebre tango argentino.

\subsection{Depois da leitura}

Propor que os alunos ouçam algumas das obras mencionadas no capítulo,
como o \emph{Corta-Jaca,} de Chiquinha Gonzaga, e \emph{Odeon,} de
Ernesto Nazareth. Ambas existem em versão puramente instrumental e
cantada, com letra. Os alunos podem cotejá-las e decidirem se as letras
refletem o espírito das músicas. E ainda discutirem a relação entre
música e letra. Você gosta de uma música porque se identifica com a
letra, ou porque a melodia é agradável?

Chiquinha Gonzaga inaugurou o cancioneiro brasileiro de Carnaval com
\emph{Ó Abre Alas.} Os alunos podem ouvir essa obra e compará-la com as
músicas de Carnaval de hoje, refletindo sobre as modificações que esse
festejo sofreu ao longo dos tempos.

\paragraph{Tempo estimado:} Duas aulas de 50 minutos.

\section{Atividade II: ``Villa-Lobos''}

\paragraph{Tema:} As relações entre Villa-Lobos e Getúlio Vargas como
emblema das relações entre arte e poder. A educação musical. A questão
nacional.

\Image{Heitor Villa-Lobos (1887--1959)}{PNLD0052-04.png}

\BNCC{EM13LP45}
\BNCC{EM13LP47}
\BNCC{EM13LP50}
\BNCCC{EM13LP51;
EM13LP53;
EM13CHS101;
EM13CHS102;
EM13CHS104;
EM13CHS105.}

\paragraph{Objetivos e justificativa }

Se os alunos já ouviram falar no nome de algum dos compositores citados
em \emph{História Concisa da Música Clássica Brasileira,} a maior
possibilidade é que esse nome seja o de Villa-Lobos\emph{.} Não apenas
no exterior, mas mesmo por aqui, Villa-Lobos é o único compositor
clássico brasileiro que tem merecido \emph{performances} e gravações
regulares de suas obras. Fundador da Academia Brasileira de Música,
conta, no Rio de Janeiro, com um eficiente museu dedicado a seu legado,
e sua efígie já foi parar até em cédula de dinheiro brasileiro, após o
Plano Cruzado, de 1986.

\Image{Heitor Villa-Lobos em uma nota de cruzado desvalorizada.}{PNLD0052-05}

Embora ainda não cubra a totalidade de sua produção (não só por seu
caráter gigantesco, mas também devido pelo estado surpreendentemente
precário das edições de suas partituras), a discografia disponível do
compositor é bastante representativa, bem como a bibliografia sobre ele,
na qual proliferam tanto biografias quanto estudos críticos sobre
aspectos específicos de sua obra.

Gerard Béhague\footnote{BÉHAGUE, Gerard. \emph{Heitor Villa-Lobos: the
  search for Brazil's musical soul.} Austin: Institute of Latin American
  Studies, 1994.} sintetiza: ``em muitas maneiras, sua personalidade,
sua carreira e sua produção refletiram diversos traços típicos
brasileiros, como grandiosidade, exuberância, inquietação, falta de
unidade orgânica, disparidade, e gabolice, ao lado de outros, como
individualidade, espontaneidade, fascinação e sofisticação''.

Vale citar ainda a análise de Fabio Zanon: ``Não é o fato de se servir
da cultura popular que faz dele um compositor singular; é a maneira como
ele conseguiu encontrar nessa cultura os fios mais adequados e os
entrelaçou à sua complexa trama composicional. Nesse processo, ele urdiu
uma maneira brasileira de se expressar em música, que imprimiu uma marca
forte a outros artefatos culturais, desde as trilhas sonoras de filmes
de Glauber Rocha à ambientação de reportagens sobre o Brasil rural,
passando pelos arranjos orquestrais da era do rádio e pelos rastros
deixados no estilo de ícones da música popular como Tom Jobim, Milton
Nascimento ou Egberto Gismonti. É possível dizer que Villa-Lobos criou
uma possibilidade de música brasileira ao invés de ser criado por ela.
Ele `tornou-se' folclore''.

A ideia não é expor um retrato idealizado de um ``super-herói''
nacional, mas utilizar as contradições na vida e na obra de Villa-Lobos
para expor e refletir as contradições de sua época -- algumas das quais
continuam a ser pertinentes nos dias de hoje.

%\paragraph{Metodologia}

\subsection{Antes da leitura}
\SideImage{}{PNLD0052-08}

Pedir aos alunos que pesquisem a respeito da presença de Villa-Lobos e
Getúlio Vargas na sua cidade. Há alguma instituição ou logradouro com o
nome de algum desses personagens históricos?

Pesquisar no dicionário o significado das palavras \emph{orfeão} e
\emph{orfeônico.} Como o canto orfeônico foi um programa de educação
musical, pode-se abrir um bate-papo com os alunos sobre este tema. As
escolas devem ensinar música? Ou devem se ater a matérias de ``utilidade
prática''? Existem matérias ``úteis'' e ``inúteis''?

\subsection{Durante a leitura}

Heitor Villa-Lobos é o único compositor a merecer um capítulo exclusivo
em \emph{História Concisa da Música Clássica Brasileira.} Além de dar a
conhecer uma figura-chave da cultura em nosso país, a leitura do
capítulo abre a possibilidade de discutir algumas questões de nossa
História e sociedade.

Uma delas é a tensão entre ``popular'' e ``erudito'', entre a prática
artística aceita e a que é socialmente condenada. Embora a força da
produção de Villa-Lobos consista justamente na fusão e na superação
dessas dicotomias, a sociedade brasileira de seu tempo era fortemente
marcada por essas clivagens. Na vida pessoal de Villa-Lobos, isso se
traduz nos mundos em que ele circulava com seu violão, instrumento então
``maldido'', e com seu violoncelo, o instrumento ``respeitável''. Essa
contradição é ilustrada pelo episódio em que Villa-Lobos faz a corte à
sua futura esposa, a pianista Lucília -- ele primeiro aparece na casa
dela de violão, mas depois se oferece para voltar com o violoncelo,
demonstrando a seriedade de suas intenções. Uma leitura complementar
enriquecedora é \emph{A Lição de Violão,} primeiro capítulo de
\emph{Triste Fim de Policarpo Quaresma,} de Lima Barreto, que retrata
com agudeza o estigma contra o instrumento na sociedade brasileira do
começo do século XX.

Outra questão é a relação entre arte e poder. Com o chamadotcanto
orfeônico, Villa-Lobos implantou um pioneiro programa de educação
musical em massa, em âmbito nacional. Esse programa, porém, ocorreu
durante a ditadura do Estado Novo, de Getúlio Vargas, e o regime
aproveitava as grandes demonstrações de canto coral ao ar livre de
Villa-Lobos como oportunidades de propaganda. É ético servir uma
ditadura, ainda que com propósitos ``nobres''? Quem estava usando quem?

Por fim, o caso Villa-Lobos também serve para discutir a questão da
brasilidade. Villa-Lobos foi um nome de destaque da Semana de Arte
Moderna de 1922, e definia-se como um compositor da escola nacionalista.
Mas o que constitui o nacional? O que faz com que alguém seja
brasileiro, ou com que alguma obra de arte o seja? São questões que
podem ser sugeridas aos alunos como temas de reflexão e debate.

\subsection{Depois da leitura}

Assistir ao filme \emph{Villa-Lobos, uma Vida de Paixão,} de Zelito
Viana. Cotejar o retrato do compositor nas telas e nas páginas do livro,
e convidar os alunos a uma reflexão sobre os limites da elaboração
artística de eventos da realidade.

O filme de Zelito Viana não é um documentário com imagens de época.
Trata-se de uma cinebiografia, de uma dramatização de fatos e
personagens reais. Qual a liberdade do artista ao abordar eventos e
pessoas que realmente existiram? Os alunos podem ser estimulados a se
lembrarem de outros filmes, séries ou telenovelas basadas em eventos
reais e compararem os resultados.

\paragraph{Tempo estimado:} Duas aulas de 50 minutos.

\section{Aprofundamento}
\SideImage{}{PNLD0052-09}

A estratégia de aproximação do leitor leigo por parte do jornalista e
escritor Irineu Franco Perpetuo evidencia-se já na escolha do título de
\emph{História Concisa da Música Clássica Brasileira}. Embora, em seus
textos jornalísticos, o autor já tenha empregado inúmeras vezes a
expressão ``música erudita'', esta lhe soa algo distanciadora e pedante.
Assim, ele resolveu adotar a fórmula ``música clássica'', que tem
aceitação internacional e, ao mesmo tempo, ajuda a delimitar o objeto da
obra.

Pois, no Brasil, há muitos livros cujos títulos variam em torno da
expressão \emph{História da Música Brasileira,} sem especificação do
estilo. Porém, quando o leitor se aproxima, acaba descobrindo que ali
está contemplada, essencialmente, a música clássica. Como se essa fosse
a única forma legítima ou possível de musicalidade, enquanto, na
verdade, quando se fala simplesmente de ``música brasileira'', sem
nenhum outro adjetivo, é a produção popular que imediatamente vem à
cabeça da maioria, seja no Brasil, seja no exterior.

Não que isso seja injusto. Nossa música popular não apenas merece
destaque dentre as outras ``músicas populares'' do planeta por sua
especial riqueza, sofisticação e variedade, como ainda se tornou uma das
mais divulgadas e respeitadas manifestações da arte brasileira em todo o
mundo. A produção musical do Brasil conquistou o planeta com seu
colorido, seu vigor e sua originalidade. Do samba, choro e maxixe do
passado ao funk, axé e hip hop de hoje, nossa música popular vem se
reinventando com inesgotável criatividade.

Essa produção, por sinal, sempre esteve em diálogo constante com nossa
música de concerto, e isso transparece nas páginas da \emph{História
Concisa da Música Clássica Brasileira.} O livro trata do nascimento da
modinha e do lundu, passa pelas origens do choro, fala do primeiro samba
escrito em partitura, visita a efervescente Era dos Festivais, da década
de 1960, mapeia a contribuição fundamental de arranjadores de formação
erudita para o Tropicalismo, conta como Tom Jobim virou o autor de uma
sinfonia para a nova capital do Brasil e descreve o tipo de música
eletrônica que se fazia de forma experimental antes desse gênero ganhar
as pistas de dança de nossas cidades. Afinal, em uma época em que a
técnica permite compor diretamente o som, em estúdio, parece cada vez
mais difusa a fronteira entre uma tradição de uma música dita
``letrada'', produzida, conservada e interpretada por meio de
partituras, e uma outra mais ligada a práticas um dia transmitidas
oralmente (mas cada vez mais cristalizadas em livros, gravações e também
partituras, estudadas e ensinadas em universidade por eruditos na plena
acepção do termo).

De qualquer maneira, a música popular brasileira aparece aqui
pontualmente. O foco do livro não é, digamos, o forró, o jongo ou os
violeiros caipiras, mas sim as missas, sinfonias, quartetos de cordas e
óperas de autores brasileiros. Trata-se da história da ``outra'' música
brasileira. A ``clássica''. A impopular -- não no sentido de ser contra
o povo, mas de não ser conhecida.

Pois a magnitude e a difusão do nome de Villa-Lobos constitui antes
exceção do que regra no cenário musical brasileiro. Em \emph{O Resto é
Ruído -- Escutando o Século XX}, Alex Ross chama os compositores
clássicos norte-americanos de ``homens invisíveis'', lutando pela
visibilidade de sua arte. ``Cada geração tem de refazer todo o trabalho.
Os compositores têm uma eterna carência de apoio do Estado. Falta-lhes
um público mais amplo; faltam-lhes séculos de tradição'', escreve Ross,
em linhas que também parecem ter validade do lado de baixo do Equador.

Nossos compositores que não se chamam Villa-Lobos vivem eternamente
tendo que ser ``resgatados'' por musicólogos e intérpretes abnegados,
que teimam em não deixar apagar a memória de uma produção contínua e de
qualidade, que começou na época colonial, chegando até os nossos tempos.
Em maior ou menor grau, poderiam todos repetir a indagação que João
Silvério Trevisan coloca na boca de Alberto Nepomuceno no romance
\emph{Ana em Veneza:} ``continuarei sendo punido ou quem sabe terei
ouvintes no futuro?''

Além do título, outra estratégia do autor para tornar a obra acessível
ao leitor ``leigo'', não-familiarizado com o jargão musical específico,
foi minimizar o uso deste jargão, e não inserir exemplos musicais em
pentagrama -- de resto, com as facilidades tecnológicas de hoje, é
relativamente simples entrar na internet e ouvir pelo menos trechos das
obras citadas. Trata-se de um recurso bastante recomendável para ser
adotado em sala de aula, por sinal, e o Caderno de Atividades traz
algumas sugestões de escuta. Ademais, o autor não possui pretensões de
exaurir o tema. Entraram os compositores e obras que ele considerou
``fundamentais'' (procurando colocar alguma objetividade dentro de
escolhas fadadas inexoravelmente à subjetividade); contudo, omissões (e
as injustiças que elas acarretam) serão sempre inevitáveis em trabalhos
dessa natureza.

Sendo cada partitura um dado cultural, e não uma mera coleção de folhas
soltas no tempo e no espaço, o autor procurou relacionar nossa música
``clássica'' com os fatos históricos e culturais de sua época (inclusive
a música popular) e situar seus criadores no ambiente intelectual e
social brasileiro. E isso é que torna o livro especialmente adequado
para a utilização em sala de aula.

Com vocação transversal e multidisciplinar, \emph{História Concisa da
Música Clássica Brasileira} apresenta personagens que o aluno já conhece
de outras disciplinas sob novos prismas. Ele vai aprender, por exemplo,
que D. Pedro I, nosso primeiro imperador, foi também compositor, e que
até teve uma obra tocada em Paris. Que Machado de Assis, além de
escrever contos e romances fundamentais, como \emph{O Alienista, Dom
Casmurro} e \emph{Memórias Póstumas de Brás Cubas,} esteve envolvido na
atividade operística de seu tempo, traduzindo óperas para o português.
Que os alemães Spix e Martius, fundamentais no mapeamento da fauna e da
flora de nosso país, e na definição de nossos biomas, também tiveram um
papel decisivo na documentação de nosso folclore musical. E que Mário de
Andrade, ademais de poeta e autor de \emph{Macunaíma,} foi o ``guru'' e
inspirador de alguns dos principais compositores brasileiros do século
XX.

Em outras palavras, o livro encara a música como um fenômeno geral da
cultura brasileira. Assim, contar a História de nossa música é também
contar a História de nossa cultura. Com o título inspirado na
\emph{História Concisa da Literatura Brasileira,} de Alfredo Bosi
(embora sem a pretensão de emular esse livro tão conhecido dos
professores de literatura), a obra traça paralelos inescapáveis entre o
mundo dos sons e o mundo das letras. Gregório de Mattos, o ``Boca do
Inferno'', aparece, por exemplo, como um de nossos primeiros cronistas
musicais -- e são citados trechos de poemas de sua autoria que
documentam a música brasileira de sua época. A música que enchia as
suntuosas igrejas mineiras do Aleijadinho é vista como inseparável da
frutífera atividade literária dos poetas da Inconfidência. Descobre-se
que a ópera italiana era o modelo estético de escritores como Tomás
Antônio Gonzaga, Cláudio Manuel da Costa e Brasílio da Gama, e que os
libretos de ópera italiana do século XVIII circulavam do lado de cá do
Atlântico em adaptação, como literatura de cordel. Isso para não falar
da importância decisiva da música na Semana de Arte Moderna de 1922.

Claro que um livro que traz a palavra ``História'' no nome também trata
bastante de perto os acontecimentos históricos que marcaram a vida de
nossa nação ao longo dos séculos. Podemos dizer que \emph{História
Concisa da Música Clássica Brasileira} conduz e propõe a ``trilha
sonora'' da História do Brasil. Ao contar, por exemplo, a história da
composição de nossos hinos oficiais -- o \emph{Hino à Independência,
Hino Nacional, Hino da Proclamação da República} e \emph{Hino à Bandeira
--,} a obra revisita o contexto em que essas partituras foram
encomendadas, escritas, modificadas e executadas.

\emph{História Concisa da Música Clássica Brasileira} tem preocupação
não apenas com diversidade estética, mas também regional, de gênero e
racial. Assim, ao abordar os eventos de 1500, mostra não apenas a música
que os portugueses trouxeram ao chegar por aqui, mas também o que é
possível saber da musicalidade dos povos originários. Reflete também
sobre o que há de racial no apagamento da música brasileira de concerto.
Pois talvez não seja exagerado afirmar que, fora da África, o país com
maior número de compositores clássicos negros em sua História é o
Brasil. Nossa exuberante vida musical dos tempos da Colônia foi tocada,
basicamente, por compositores negros, cujas obras caíram no esquecimento
ao longo do século XIX, e só foi ser resgatada paulatinamente a partir
da década de 1940.

Descobrir a música erudita brasileira é conhecer mais um aspecto do rico
fazer artístico dos negros no Brasil. O livro também demonstra como, a
partir do pioneirismo de Chiquinha Gonzaga, as mulheres compositoras
passaram a lutar por protagonismo em nossa vida musical. Mulheres
compositoras ainda são minoritárias nas programações de concerto, não
apenas no Brasil, como em todo o planeta. O livro faz um esforço de
mapear e aquilatar a produção de nossas compositoras. E busca não se
restringir ao eixo Rio-São Paulo, ou à região Sudeste, empenhando-se em
documentar nossa música em âmbito nacional.

Música erudita contemporânea é uma espécie de gueto dentro do gueto, um
enclave minoritário em um território minúsculo. \emph{História Concisa
da Música Clássica Brasileira} tenta contemplar os compositores vivos,
de hoje, em todo o país, bem como a pujança de nossa produção crítica
universitária. O número de estudos acadêmicos relevantes e profundos
sobre a música brasileira cresceu de maneira avassaladora nas primeiras
décadas do terceiro milênio, e apenas uma lista de todos já superaria em
muito o tamanho de uma história que ser pretende concisa. Nessa área,
não foi possível fazer mais do que roçar a ponta de um iceberg de
dimensões incomensuráveis.

Não custa lembrar, ainda que o livro foi viabilizado graças a uma
campanha de financiamento coletivo, na internet. Além da contribuição
pecuniária, muitos estudiosos, compositores e intérpretes aproveitaram a
ocasião para botar à disposição do autor seus ricos acervos e
conhecimentos, aportando dicas inestimáveis, que acabaram por enriquecer
o trabalho de forma decisiva.

\section{Referências complementares}

\begin{itemize}

\item 
\emph{Villa-Lobos -- Uma Vida de Paixão.} Direção de Zelito Viana. Rio
de Janeiro, Mapa Filmes, 2000.
https://www.youtube.com/watch?v=5WFFxGjxUv4

\emph{Além de comparar o Villa-Lobos das telas com o compositor das
páginas do livro, os estudantes poderão fazer um exercício crítico de
reflexão sobre as possibilidades e limites da dramatização de eventos
reais. O filme de Zelito Viana é documentário? É ficção? Qual o limite
entre esses gêneros? }
\end{itemize}

\section{Bibliografia comentada}

\begin{itemize}
	
\item 
AZEVEDO, Luiz Heitor Corrêa de. \emph{150 anos de música no Brasil
(1800-1950).} 2\textsuperscript{a} Edição. Rio de Janeiro: FBN,
Coordenadoria de editoração, 2016.


\emph{Um dos principais musicólogos brasileiros, Luiz Heitor escreveu,
na década de 1950, esta obra pioneira, que continua servindo como
referência na área. }

\item 
DINIZ, Edinha. \emph{Chiquinha Gonzaga: uma história de vida}. Rio de
Janeiro: Jorge Zahar, 2009.


\emph{Um apanhado de fôlego da trajetória da maestrina pioneira do
Brasil -- Chiquinha Gonzaga. }

\item 
MACHADO, Cacá. \emph{O enigma do homem célebre: ambição e vocação de
Ernesto Nazareth}. São Paulo: Instituto Moreira Salles, 2007.


\emph{Uma reflexão sobre os impasses e dilemas que marcaram a carreira
de Ernesto Nazareth. }

\item 
ZANON, Fabio. \emph{Folha explica Villa-Lobos.} São Paulo: Publifolha,
2009.


\emph{Uma excelente síntese resumida e acessível da vida e obra de
Villa-Lobos, por um de seus mais destacados intérpretes: o violonista
Fabio Zanon, que gravou a obra do compositor com enorme êxito
internacional. }
\end{itemize}

\end{document}


