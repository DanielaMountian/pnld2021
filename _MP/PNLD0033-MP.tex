\documentclass[12pt]{extarticle}
\usepackage{manualdoprofessor}
\usepackage{fichatecnica}
\usepackage{lipsum,media9,graficos}
\usepackage[justification=raggedright]{caption}
\usepackage[one]{bncc}
\usepackage[blooks]{../edlab}

% Ler depoimento dele sobre a propria vida 
% Ver documentário https://www.youtube.com/watch?v=CR_SJkeUN9o
% Patativa era sobretudo um declamador. 
% Paul Zunthor (tipos de culturas orais)
% Poeta cego
% As cantoras cegas de campina grande
% Cego aderaldo https://pt.wikipedia.org/wiki/As_Ceguinhas_de_Campina_Grande
% Tirésias https://pt.wikipedia.org/wiki/Tir%C3%A9sias

\begin{document}


\newcommand{\AutorLivro}{Patativa do Assaré}
\newcommand{\TituloLivro}{Ispinho e Fulô}
\newcommand{\Tema}{Ficção, mistério e fantasia}
\newcommand{\Genero}{Poema}
\newcommand{\imagemCapa}{./images/PNLD0033-01.png}
\newcommand{\issnppub}{---}
\newcommand{\issnepub}{---}
% \newcommand{\fichacatalografica}{PNLD0033-00.png}
\newcommand{\colaborador}{\textbf{Vicente Castro e Bruno Gradella} Sofia Boldrini (edição)}


\title{\TituloLivro}
\author{\AutorLivro}
\def\authornotes{\colaborador}

\date{}
\maketitle


\begin{abstract}\addcontentsline{toc}{section}{Carta ao professor}
Este Manual tem como objetivo fornecer subsídios para o trabalho com a
obra literária \emph{Ispinho e Fulô}, de Patativa do Assaré.

Antônio Gonçalves da Silva, o Patativa do Assaré, foi um dos mais importantes
poetas brasileiros.  Patativa do Assaré carrega em suas obras a essência da 
cultura de seu povo, relatando suas alegrias e sofrimentos. Em 1995, o então 
presidente do Brasil, Fernando Henrique Cardoso, prestou uma homenagem pública 
ao poeta, dedicando-lhe a medalha José de Alencar, por suas grandes contribuições 
em favor do desenvolvimento cultural do estado do Ceará.

De sua obra, destacam-se os livros \textit{Inspiração nordestina} (1956), 
\textit{Cante lá que eu canto cá} (1978), \textit{Ispinho e fulô} (1988), 
\textit{Balceiro} (1991), \textit{Aqui tem coisa} (1994) e \textit{Digo e 
não peço segredo} (2001); e os discos \textit{Poemas e canções} (1979), 
\textit{A terra é naturá} (1981) e \textit{Canto nordestino} (1989).

\emph{Ispinho e Fulô} reúne setenta e oito poemas que recontam a trajetória 
de Patativa do Assaré, passando por diferentes temas e formas, apresentando as 
principais características da poesia popular. O poeta contribuiu muito para a 
construção e divulgação da identidade nordestina, utilizando, em seus poemas, 
imagens marcantes da tradição popular e uma grande variedade de imagens que 
simbolizam o nordeste brasileiro. É um poeta atento ao seu meio.

Vemos isso em ``Padre Henrique e o dragão da maldade'', poema que narra o assassinato 
de um padre pelos agendes de repressão da Ditadura Militar. Outro exemplo é a 
quantidade de histórias e lendas que o poeta vai buscar na tradição popular 
para compor seus poemas. É o caso do poema ``Brosogó, Militão e o diabo'', história 
protagonizada por personagens típicos e que em geral traz uma lição, uma moral. 
No poema temos Brosogó, um simples e decente vendedor ambulante, que, um dia, entra 
na casa de um coronel chamado Militão. Sem conseguir vender nada, quer comprar meia 
dúzia de ovos de Militão. Como o coronel não tem troco, diz a Brosogó que leve os ovos 
e volte depois para pagar. Brosogó prospera e conquista muitas posses. Um ano e 
sete meses depois, volta para pagar a dívida e é enganado por Militão, que faz as 
contas dizendo que os ovos teriam sido chocados e que as galinhas nascidas poriam 
novos ovos e teriam novas crias, e assim por diante. Brosogó fica desiludido, mas 
é ajudado pelo Diabo, a quem teria acendido velas em um dia em que não lembrava de 
mais nenhum santo a quem agradecer.

Esperamos que as indicações propostas aqui sejam muito úteis no trabalho em
sala de aula!



\end{abstract}

\tableofcontents

\SideImage{Escultura de Patativa do Assaré, Fortaleza, Ceará (Autoria Desconhecida; CC-BY-SA 2.0)}{PNLD0033-03.png}

\section{Introdução}

Baseada na vida e obra desse poeta, o livro dialoga com a proposta do educador Paulo Freire, 
discute adversidades da vida no sertão e conta histórias do Brasil que não constam dos acervos 
oficiais. Patativa do Assaré, cujo verdadeiro nome é Antônio Gonçalves da Silva, nasceu no dia 
05 de março de 1909 na Serra de Santana, pequena propriedade rural da prefeitura de Assaré, ao 
sul do estado do Ceará. Por meio da improvisação e da oralidade, aliadas à técnica tradicional 
dos cordéis, Patativa cantou os sofrimentos e as alegrias do sertanejo, sendo ele próprio uma 
personificação das lutas por igualdade social.

A obra de Patativa do Assaré traz a tona o debate sobre as desigualdades nas diferentes regiões 
do Brasil e revela as condições de vida precária do trabalhador rural no sertão nordestino. Esse 
olhar sobre o mundo, numa perspectiva espacial, recupera também uma oposição entre:

Passado x Presente
Tradição x Modernidade

Para escapar da seca o sertanejo é obrigado a abandonar sua terra em direção às cidades do litoral 
ou às cidades do sul e Patativa retrata a delicada passagem de um mundo rural a um mundo urbano em 
que impera o anonimato. Assim, o universo descrito por Patativa do Assaré é percebido como um espelho 
da realidade e expõe os conflitos do sujeito diante dos desafios do vasto mundo. 

Em um contexto de miséria e analfabetismo largamente propagado e em meio à ausência de estruturas educativas 
de base, o poeta popular tem um papel importante na consciência cívica e política. Ao defender a principal 
reivindicação dos habitantes do sertão, ele se torna verdadeiramente a voz do nordeste e o símbolo dos direitos 
humanos fundamentais .

``Ispinho e Fulô'' é o terceiro livro de Patativa do Assaré, publicado originalmente em 1988. A variedade linguística 
utilizada pelo poeta é caracterizada pelo registro de formas típicas da linguagem oral: os substantivos \textit{ispinho} 
e \textit{fulô} correspondem, no dicionário, a espinho e flor. Na década de 70, Patativa declarou:

``Eu sou um caboclo roceiro que, como poeta, canto sempre a vida do povo! {}\ldots{} O meu problema é cantar a vida do 
povo, o sofrimento do meu nordeste, principalmente daqueles que não tem terra.''
 
A obra vai descrevendo os vários estágios da vida, respectivamente, a infância e a adolescência. Sendo a primeira estrofe do 
poema a que apresenta o ciclo da vida como um fato que deve ser aceito, com momentos de dor maiores que os de alegria.

O cordel é uma forma poética popular de origem europeia e ele foi incorporado à cultura brasileira, sobretudo do nordeste do 
país. Os poemas em cordel costumam ser longos e narram histórias repletas de reviravoltas e aventuras. A linguagem é marcada 
pelo humor e pelo suspense, com o objetivo de captar a atenção do público. 
 
Desde a sua origem, para facilitar a memorização e a recitação, esses textos apresentam um esquema próprio de rimas e uma métrica 
específica. Antigamente, os cordéis eram vendidos nas feiras, pendurados em varais e exibidos em balcões e esteiras, e muitas vezes, 
as performances eram realizadas ali mesmo, pelos autores que declamavam as narrativas em versos. Havia uma roda formada por passantes 
que paravam para ouvir as histórias. Hoje, em vários lugares, essa configuração mudou um pouco e os cordéis são expostos dentro de 
plásticos, em bancas de revistas
 
A poesia matuta reflete o pensamento e a linguagem do homem do campo, como nos versos abaixo:  

\begin{verse}
Vendo os ispinho omentando\\
E as fulô diminuindo.\\
\end{verse}
 
Observe também que a concordância nominal de número não corresponde à norma padrão. 

Este livro traz um conjunto de cordéis com elementos de denúncia e de contestação pelos direitos dos sertanejos e fica explícita a ausência 
do Estado na época como principal motivador de atraso no sertão. A população sofria com a seca e Patativa do Assaré, assim, se destaca como 
porta voz do povo, vivenciando a dramática realidade da região nordestina. A obra literária \emph{Ispinho e Fulô} apresenta um panorama da 
vida do sertanejo e a oposição entre mundo urbano e mundo rural, construída a partir de diferenças socioculturais e do sistema de valores. 
 
A situação do sertanejo que abandona sua terra é uma posição delicada, ele passa sem transição de um mundo rural a um mundo urbano onde 
impera o anonimato. 

\section{Proposta de atividades I}

\subsection{Pré-Leitura}

\paragraph{Tema} O universo do cordel

\paragraph{Conteúdo} Compreensão das múltiplas narrativas de cordel, por meio da pesquisa 
dos títulos das obras, de suas temáticas e personagens. 

\paragraph{Objetivo} Aproximar a perspectiva sertanista, do cordel e das culturas orais, ao universo acadêmico, 
compartilhando as narrativas de maneira didática.

\paragraph{Justificativa} A literatura de
cordel é um patrimônio da cultura brasileira. Fabricada em folhetos,
geralmente trazem para o universo escrito casos e histórias já comuns na
oralidade popular.

Também é comum que seu texto seja construído por meio
de rimas. Além disso, uma das características típicas do cordel são as
gravuras, que possuem um tracejado muito próprio. Não obstante, por
serem uma expressão muito típica da cultura de uma região do país, os
diversos títulos produzidos em cordel trazem elementos básicos que são
características intrínsecas muito próprias.

\paragraph{Metodologia}

\begin{enumerate}
\item
Antes do adentrar na obra, como atividade de pré"-leitura,
em uma roda de conversa com os estudantes, debata sobre o que eles 
sabem acerca da palavra ``cordel''.  

\item
Em seguida, proponha aos estudantes que façam uma pesquisa, em duplas, escolhendo 
um título de uma obra, buscando atentar"-se à sinopse do cordel selecionado, entendendo 
a temática da história e como se apresentam suas personagens. Depois, sugira que cada dupla
apresente brevemente para a sala o que encontraram. 

\item
Isso posto, peça então às duplas que escrevam um pequeno texto concatenando o que descobriram 
em suas pesquisas, em uma espécie de relatório/fichamento, mas agora colocando suas suposições do 
porquê da repetição desses eixos temáticos e personagens. 

Posteriormente, o professor pode coletar essas informações
junto aos alunos e abordar as questões durante as aulas sobre o livro.

\item
Na última parte da atividade, apresente aos alunos alguma das músicas abaixo, se possível, em uma 
plataforma audiovisual onde eles possam ter acesso à letra. 
Em seguida, peça que relacionem a canção à forma poética do cordel, indicando onde, na música, 
percebem as características apresentadas na aula. 

Nossas sugestões de músicas são, \href{https://www.youtube.com/watch?v=S1eE0lGpi-c&ab_channel=CordeldoFogoEncantado}{``Os Oim do Meu Amor''} 
do grupo Cordel do Fogo Encantado e \href{https://www.youtube.com/watch?v=pPmQYbTE21M&ab_channel=CaetanoVeloso-Topic}{``Circulado de Fulo''} de Caetano Veloso.

O professor pode ficar à vontade para experimentar outras canções desde que siga os
mesmos critérios de escolha. 


\end{enumerate}

\paragraph{Tempo estimado} Duas aulas de 50 minutos. 

\subsection{Leitura}

\paragraph{Tema} Escrevendo com Patativa do Assaré

\paragraph{Conteúdo} Exercícios de escrita criativa de poemas no formato de cordel, 
por meio da interpretação de trechos da obra.

\paragraph{Objetivo} Orientar a produção de poemas a partir da experiência
de leitura, escuta e visualização da obra \emph{Ispinho e Fulô}, articulando o que foi estudado sobre 
suas características formais.

\paragraph{Justificativa} Os grandes poemas de cordel são perfeitamente 
metrificados e rimados. A métrica e a rima são recursos que
favorecem a memorização e tradicionalmente se costuma
dizer que são resquícios de uma cultura oral, na qual toda a
tradição e sabedoria são sabidas de cor.

Uma das grandes forças da poesia popular do Nordeste
se origina em sua forma muito própria de falar, com um
ritmo muito diferente dos falares do sul, e também muito
diferentes entre si, pois percebe"-se a diferença entre os
falares de um baiano, um cearense e um pernambucano,
por exemplo.

Além desse aspecto rítmico, quase sempre também há
palavras peculiares a certas regiões.

Reposicionar"-se como escritor/escritora desse gênero literário que cantam um Nordeste em transformação, 
permitiria, ainda que como desafio, aos estudantes 
verificar a apreensão do conteúdo da obra lida e as possíveis reverberações dessa leitura.

Assim, operando traduções para multiplicar seus mundos e visões.

\paragraph{Metodologia}

\begin{enumerate}
\item
Para dar início a essa atividade de leitura e enriquecer a experiência com outros recursos, 
proponha a exibição do próprio Patativa do Assaré lendo alguns
de seus poemas no \href{https://www.youtube.com/watch?v=RTEfYnMNNpc&ab_channel=IrandiHolanda}{Youtube}

\item
Em seguida, já mais familiarizados com as características do cordel, os
alunos devem passar então à produção de um texto nesse formato. Sugira a 
leitura de ``O padre Henrique e o dragão da maldade'' (pp.205) como 
inspiração para diálogo com a escrita que irão produzir. 

Esse poema conta a história de um padre que vivia na cidade
do Recife. Por seu engajamento nos movimentos sociais, de
esquerda, e por seus vínculos com o movimento estudantil,
o padre foi assassinado. O poema celebra a biografia do
padre, dizendo que foi um homem que lutou pela melhoria
das condições de vida dos mais pobres. É a crônica de
um acontecimento que chocou a população pela violência
dirigida a um personagem caridoso.

\item
Permita aos alunos, se eles preferirem, redigir uma autobiografia, contando
sua história nos moldes de um cordel. Entretanto, é possível que alguns
alunos não se sintam confortáveis em relatar experiências próprias e, então,
sugira que remontem uma história fictícia ou real, aproximando"-a de 
``O padre Henrique e o dragão da maldade''.

Diante disso é possível propor que seja uma biografia com elementos
romanceados, elementos do fantástico, ou mesmo a biografia de uma
terceira pessoa, pela qual o aluno nutre admiração. 

\end{enumerate}

\paragraph{Tempo estimado} Duas aulas de 50 minutos. 

\Image{Literatura de Cordel (Diego Dacal/Wikimedia Commons; CC-BY-SA 2.0)}{PNLD0033-06.png}

\subsection{Pós-Leitura}

\paragraph{Tema} Feira de cordéis

\paragraph{Conteúdo} Exercício de compilação sobre o que aprenderam ao longo da leitura da obra e das outras 
atividades correlacionadas. 

\paragraph{Objetivo} Retomar junto com os alunos os pontos importantes da cultura do cordel e identificar 
nos espaços dos sertões, das paisagens, das produções artísticas e dos costumes as reverberações do que nos 
é apresentado por Patativa do Assaré em seu livro. 

\paragraph{Justificativa} O cordel, mesmo sendo escrito e impresso para ser lido, costumava
ser lido em voz alta e desfrutado por outros ouvintes
além do leitor. A poesia popular, praticada principalmente
no Nordeste do Brasil, tem muita influência da linguagem
oral, aproveita muito da língua coloquial praticada nas ruas
e na comunicação cotidiana.

Assim, revela"-se pertinente, como recapitulação do que foi lido e estudado, recriar os espaços 
cantados pelo autor, em seus próprios termos de promoção dos cordéis, aprofundando esses estudos.

\paragraph{Metodologia}

\begin{enumerate}
\item
Para essa atividade de pós"-leitura, a partir do material produzido na atividade anterior (de leitura), 
sugira, se possível, a realização de uma feira de cordéis da turma, que junto com outros professores, 
preparem uma decoração adequada, valorizando as cores e os traços presentes nesse tipo de material tão
distinto. 

\item
Além da exposição dos cordéis produzidos, propõe"-se também a
promoção de encenações de trechos de obras lidas, ou produzidas pelos
alunos, a declamação de poemas e até mesmo competição de repentes. 

\item
Permita aos alunos, se assim desejarem, buscar aproximações desse gênero 
com outras produções artísticas mundiais. Nada impede que, em uma dessas encenações 
ou leituras dramáticas, os alunos podem buscar trechos de peças, filmes ou obras da
literatura mundial e encená"-los, valendo"-se da estética e do vocabulário
típico aos cordéis.

\end{enumerate}

\paragraph{Tempo estimado} Duas aulas de 50 minutos. 

\section{Proposta de atividades II}
A obra \emph{Ispinho e Fulô} possibilita trabalhos interdisciplinares e
integradores de diferentes campos do saber e áreas de conhecimento. A
seguir, propomos algumas atividades que podem ser desenvolvidas
conjuntamente com professores de outras áreas. Além das habilidades de
Linguagens e suas Tecnologias e de Língua Portuguesa, indicadas nas
etapas da seção anterior e válidas também para esta, listamos a seguir
as habilidades de outras áreas, presentes na abordagem interdisciplinar:


\subsection{Pré-Leitura}

\paragraph{Tema} Escrita que canta um Nordeste 

\paragraph{Conteúdo} Introdução e contextualização da atmosfera do sertão 
e suas características geomorfológicas, climáticas e culturais. 

\paragraph{Objetivo} Ambientar os estudantes no universo que é tema na obra de
Patativa do Assaré.

\paragraph{Justificativa} O sertão tem mitos culturais próprios. 
Contemporaneamente, o sertão evoca principalmente o sofrimento resignado
daqueles que padecem a falta de chuva e de boas safras
na lavoura. 

Evoca a experiência histórica de uma região
empobrecida, embora tenha sido geradora de riquezas, como
o cacau e cana de açúcar, ambos bens muito valiosos.

O sertão formou também o seu imaginário por meio de
grandes personalidades e uma pujante expressão artística.
Além do cordel, o sertão viu nascer ritmos tão importantes
quanto o forró e o baião. Produziu artistas expressivos na 
gravura popular, que sempre acompanha os folhetos de cordel,
também floresceu em diversos pontos e ficou mais famosa
em Juazeiro do Norte, no Ceará, e em Caruaru, no estado
de Pernambuco.

Faz"-se, assim, necessário contextualizar a obra e aprofundar questões particulares ao universo 
sertanejo para que se tenha uma leitura situada histórica"-geográfica"-culturalmente, 
permitindo uma maior compreensão de \textit{Ispinho e fulô}. 

\paragraph{Metodologia}

\begin{enumerate}
\item
Antes da leitura, é possível discutir a situação da seca no
Nordeste. Professores de ciências humanas podem discutir a história
desse fenômeno e suas implicações sociais, bem como as relações que o
homem travou com essa situação ao longo de toda a ocupação da região.

Sugerimos abordar a questão da transposição do rio São Francisco como 
solução para a seca em lugares mais áridos. O que os alunos pensam à respeito 
dessa iniciativa? 

\item
Nessa atividade de pré"-leitura" também os professores de ciências da natureza 
podem auxiliar os alunos na compreensão dos fenômenos geológicos e meteorológicos que
contribuem para a perpetuação da aridez local, bem como explicar o bioma
da caatinga e sua importância para o equilíbrio ecológico. 

\item
Munidos dessa informação os alunos podem produzir um jornal, tanto nos moldes
escritos e, se possível, em vídeo, construindo uma reportagem sobre todas as
questões levantadas. 

\item
O material produzido pode ser divulgado no site da
escola. Como sugestão, indica"-se utilizar aplicativos gratuitos de
gravação e edição de vídeos, disponíveis em dispositivos digitais.

\end{enumerate}

\paragraph{Tempo estimado} Duas aulas de 50 minutos. 

\Image{Flor de mandacaru registrada em trilha na Bahia. (Wellscte/Wikimedia Commons; CC-BY-SA 4.0)}{PNLD0033-08.png}

\subsection{Leitura}

\paragraph{Tema} Os esteriótipos da identidade sertaneja 

\paragraph{Conteúdo} Compreensão e aprofundamento das características e temáticas da poesia popular
que trata a obra.

\paragraph{Objetivo} Familiarizar os estudantes com as características envoltas na construção da 
identidade sertaneja, a partir da leitura de alguns poemas de \emph{Ispinho e fulô}.
Os alunos devem conseguir identificar os principais retratos que constituem esse imaginário, 
questionando se existe uma visão estereotipada. 

\paragraph{Justificativa} A obra \emph{Ispinho e fulô} é baseada na vida e na obra de
Patativa do Assaré, passando por diferentes temas e versificações,
apresentando as principais características da poesia popular -- poesia
essa que, muitas vezes, é deixada de lado por se distanciar do erudito.
No entanto, Patativa do Assaré contribuiu ativamente para a construção e
divulgação da identidade nordestina, utilizando, em seus poemas, imagens
marcantes da tradição popular e uma grande variedade de personagens que
representam o nordeste brasileiro.

Este livro apresenta a identidade sertaneja e histórias do sertão muitas
vezes não registradas pela historiografia oficial. A realidade vivida e
retratada por Patativa do Assaré é mostrada com tamanha fidelidade, que
fez com que ele recebesse o título de ``poeta da oralidade'', inspirando
poetas, cantadores e músicos.

\paragraph{Metodologia}

\begin{enumerate}
\item
Para essa atividade de leitura, proponha um primeiro debate com os alunos,
em diálogo com as disciplinas e professores de humanidades, a partir da seguinte pergunta: 
Existiria no Brasil uma visão estereotipada de uma região frente a outra? 
Qual é esse imaginário? Como se dá e a quem se vincula? Essas visões se aproximam 
ou se afastam de uma identidade nacional brasileira?

\item
Em seguida, proponha a leitura do poema ``Nordestino, sim, nordestinado, não'' (pp. 41) e 
``A triste partida'' (pp. 49), e atente aos alunos para o fato que ambos poemas relatam as 
transformações geográficas e sociais que afetam as regiões do Nordeste, principalmente pelas 
secas, enumerando as consequências e reverberações para os animais, a vegetação e, principalmente, 
para as pessoas em sua luta cotidiana. 

Aproveite para mostrar também a versão de 1964, gravada por Luiz Gonzaga de \href{https://www.youtube.com/watch?v=Yu0bvuK8s_k&ab_channel=EllisStoffel}{``A triste partida''}

\item
Como produto final para essa atividade, promova a escrita de um artigo de opinião 
defendendo a valorização das identidades regionais, com atenção especial à nordestina, 
procurando articular trechos da obra lida nessa produção de texto. 

\end{enumerate}

\paragraph{Tempo estimado} Duas aulas de 50 minutos.  

\subsection{Pós-Leitura}

\paragraph{Tema} Onde dá o \textit{ispinho}?

\paragraph{Conteúdo} Exercício e articulação da leitura da obra \emph{Ispinho e fulô} com a realidade 
dos êxodos rurais e urbanos na região Nordeste.

\paragraph{Objetivo} Estimular e habilitar os estudantes, por meio da elaboração de um mapa, 
a refletir acerca dos fluxos migratórios e suas motivações entre os estados do Nordeste e do Sul 
e Sudeste do país. 

\paragraph{Justificativa} Na obra lida, é retratada a oposição entre campo e cidade, e as
características das pessoas que habitam esses dois ambientes.

Um outro poema de Patativa do Assaré também é representativo do movimento migratório, que 
caracterizam as regiões do Nordeste do Brasil. Intitulado ``Emigração e as consequências'', 
o poema defende que a seca no Nordeste obriga as famílias a abandonarem suas terras natais e 
migrarem para o Sul do país. Ao chegarem ao Sul, as famílias não conseguem empregos, e, quando 
conseguem, seus salários são insuficientes. Logo a mãe também vai trabalhar. Os filhos pequenos, 
expostos a privações, se envolvem com outros menores, eventualmente já introduzidos na delinquência, 
e algumas vezes ingressam na vida criminosa e comprometem para sempre suas vidas. Situação essa expressada 
nas estrofes de abaixo:

\begin{verse}
Por causa desta inclemência\\
Viajam pelas estradas\\
Na mais cruel indigência\\
Famílias abandonadas\\
Deixando o céu lindo e azul\\
Algumas vão para o Sul\\
E outras para o Maranhão\\
Cada qual com sua cruz\\
Se valendo de Jesus\\
E do padre Cícero Romão\\

Nestes medonhos consternos\\
Sem meios para a viagem\\
Muitas vezes os governos\\
Para o Sul dão a passagem\\
E a faminta legião\\
Deixando o caro torrão,\\
Entre suspiros e ais,\\
O martírio inda mais cresce\\
Porque quem fica padece\\
E quem parte sofre mais\\
\end{verse}

Dessa maneira, entender o contexto cultural e geomorfológico que influencia as principais personagens 
de \emph{Ispinho e fulô} a migrarem, além de outros interlocutores que dialogam com a obra e com a 
realidade Nordestina retratada por Patativa do Assaré, revela"-se de grande valia para maior compreensão 
de sua produção.

\paragraph{Metodologia}

\begin{enumerate}
\item
Para essa atividade de pós"-leitura, seguindo a temática do homem rural em oposição ao homem urbano, 
proponha aos alunos, contando com o auxílio do professor de ciências humanas, a produção de um mapa 
que situe o fluxo migratório Nordestino no país. Nesse mapa, sinalize majoritariamente as zonas urbanas e rurais. 

\item
Em paralelo à produção cartográfica, aconselha"-se buscar notícias e estudos sobre as grandes desigualdades 
e os fatores do deslocamento populacional para as grandes cidades.
Como fonte, também, sugere"-se orientar a pesquisa por meio da obra lida de
Patativa do Assaré, \emph{Ispinho e fulô}, e de outros autores nordestinos.

Sugerimos também para esse exercício escutar a música de Luiz Gonzaga \href{https://www.youtube.com/watch?v=L9WSaMi2QhA&ab_channel=LuizGonzaga-Topic}{Pobreza por pobreza} 
que também retrata a situação Nordestina da migração e seus anseios. 

\item
Com essa informação, é possível indicar de maneira gráfica, valendo"-se de setas por exemplos,
caso seja um painel físico, ou por recursos animados, caso o mapa esteja
sendo produzido virtualmente, os movimentos migratórios estudados. 

\item
Para a produção do produto final e como sugestão gráfica, sugere"-se criar xilogravuras a partir da
confecção de carimbos artesanais, possibilitando a criação de uma identidade visual, para esse mapa, vinculada ao cordel.

\end{enumerate}

\paragraph{Tempo estimado} Duas aulas de 50 minutos. 

\Image{Xilogravura da Artista plástica Yolanda Carvalho (Rafael Nolêto/Wikimedia Commons; CC-BY-SA 3.0)}{PNLD0033-10.png}

\section{Aprofundamento}

% Ao chegar ao Ensino Médio, é necessário que os estudantes se aprofundem
% na compreensão das múltiplas linguagens e, sobretudo, da linguagem
% literária. Em relação à literatura, a BNCC traz as seguintes
% considerações:

%Vicente: Aprofundamento

Nesta seção, desenvolvemos um trabalho de aprofundamento que, em diálogo
com a formação continuada de professores, oferece subsídios para a
abordagem do texto literário. A leitura em sala de aula de
\emph{Patativa do Assaré} pode ser enriquecida pelo aprofundamento no
universo literário.

\subsection{O romance da maturidade}

O texto literário, presente no currículo escolar desde o Ensino
Fundamental, deve também permanecer como base de formação no Ensino
Médio, para que os alunos possam se aprofundar na compreensão não só da
linguagem literária, mas também das múltiplas linguagens, estimulando a
criatividade e imaginação. Segundo Paulo Freire (1989){[}1{]}, o ato de
ler implica na percepção crítica, interpretação, reescrita e
reelaboração do que lemos, agindo assim, diretamente, com diferentes
esferas de aprendizagem.

Este manual oferece subsídios para a abordagem do texto literário em
diálogo com a formação continuada de professores, apresentando
propostas, com diferentes abordagens, para o trabalho de \emph{Vocação
de Cantador, de Oliveira de Panelas,} dentro e fora sala, incentivando
os alunos a adquirirem o hábito de leitura e pensamento crítico.

\subsection{Importância da literatura de cordel em sala}

Com uma produção simples e de grande abrangência, a Literatura de Cordel
ganhou espaço e prestígio na cultura nordestina brasileira, tornando"-se,
em 2018, patrimônio cultural imaterial do nosso país -- reconhecido pelo
Conselho Consultivo do Instituto do Patrimônio Histórico e Artístico
Nacional. Assim, o Cordel como gênero do discurso contribui na formação
do aluno possibilitando o domínio de outros conteúdos, além da
descentralização do ensino. O estudo da Literatura de Cordel como forma
de expressão da cultura popular contribui também no aprimoramento das
habilidades de oralidade, escrita, leitura, interpretação, linguagens
artísticas e até na dramatização de peças, auxiliando na
interdisciplinaridade dos temas.

A Educação Literária aparece entre as competências gerais da BNCC (Base
Nacional Comum Curricular), enfatizando a importância vivência do aluno
no aprendizado da literatura e demais manifestações artísticas --
``\emph{Valorizar e fruir as diversas manifestações artísticas e
culturais, das locais às mundiais, e também participar de práticas
diversificadas da produção artístico-cultural\textbf{{[}2{]}}'' --},
logo, o Cordel ganha espaço nesse cenário, pois, a partir de seu estudo,
é possível despertar nos alunos o interesse por diversos campos
artísticos.

Entretanto, o uso do Cordel em sala de aula, bem como nos livros
didáticos, ainda é muito restrito, por não ser tão prestigiado quanto os
demais gêneros literários. Diante disso, cabe a nós resgatarmos essa
parte da nossa identidade nacional, contextualizando o aluno no meio
social e cultural de seu país.

\Image{Ilustração da capa do cordel ``O ABC da Cachaça'' (Thomas Fisher Rare Book Library/Flickr; CC-BY-NC-SA 2.0)}{PNLD0033-07}

\subsection{Variação linguística e oralidade}

O Cordel nasceu da oralidade e da linguagem popular; uma leitura
silenciosa limita seu poder de comunicação, impedindo que seu potencial
seja trabalhado como um todo. O gênero, se bem explorado, pode auxiliar
no aprendizado e desenvoltura dos alunos nessa modalidade de expressão,
devido ao seu ritmo cadenciado e seu linguajar comum, próximo ao
cotidiano do aluno. Para isso, o professor deve promover atividades que
possibilitem a verbalização do aluno, que estimulem a livre expressão,
para que ele possa, a partir dos exercícios, identificar seu local de
fala, além de desenvolver respeito e empatia pela fala do outro,
aprimorando, também, a convivência social.

Ao utilizar a Literatura de Cordel, o professor poderá abordar a questão
do preconceito linguístico da língua portuguesa, ao estimular a leitura
de poemas que fogem do padrão gramaticalmente institucionalizado. É
possível mostrar aos alunos que a linguagem popular é muitas vezes
discriminada, mesmo fazendo parte de uma cultura rica e diversificada,
quebrando a ideia de que o ideal é necessariamente o padrão.

\subsection{Variação cultural e geográfica}

Como explicitado anteriormente, é sabido que a Literatura de Cordel faz
parte de nossa cultura e tradição. Antes mesmo da chegada das grandes
mídias e meios de comunicação, o Cordel funcionou como instrumento de
disseminação de valores, lendas e conhecimento popular da tradição
nordestina. Levá"-lo à escola é uma maneira de resgate da nossa cultura,
motivando o aluno a conhecer mais sobre nosso país e seus diferentes
povos e regiões, além de nossa história religiosa, econômica e política,
vez que muitos cordéis abordam realisticamente essas questões.

\subsection{Intercâmbio artístico e literário}

A ilustração com xilogravura (gravura em madeira) é uma característica
marcante dos folhetos de Cordel, usada para decorar e dar mais vida aos
poemas, além de oferecer material para as mais variadas interpretações
das obras. O uso da técnica deu"-se graças ao baixo custo de produção e
foi fundamental para disseminar a cultura do Nordeste em outras partes
do Brasil. Os traços marcantes da xilogravura de cordel em composição
com os poemas se transformam em uma expressão de linguagem, registrando
a história do nosso povo.

Levar componentes artísticos para a sala de aula é uma forma de chamar à
atenção do aluno, além de proporcionar maior pensamento crítico e
incentivo à expressão artística e literária.

\Image{Capa de ``A Xilogravura Popular e a Literatura de Cordel'', 1985 (Thomas Fisher Rare Book Library/Flickr; CC-BY-NC-SA 2.0)}{PNLD0033-09.png}

\subsection{Sobre a obra }

\emph{Ispinho e Fulô} reúne poemas que recontam a trajetória de Patativa
do Assaré, passando por diferentes temas e versificações, apresentando
as principais características da poesia popular -- poesia essa que,
muitas vezes, é deixada de lado por se distanciar do erudito. No
entanto, Patativa do Assaré contribuiu ativamente para a construção e
divulgação da identidade nordestina, utilizando, em seus poemas, imagens
marcantes da tradição popular e uma grande variedade de imagens que
simbolizam o nordeste brasileiro.

\subsection{A estrutura da métrica}

A Literatura de Cordel sofreu, estruturalmente, diversas modificações
com o passar dos anos, por se tratar de uma linguagem oral que foi sendo
transformada, também, em escrita. No início, os repentistas não tinham
compromisso com número de versos ou métrica, entretanto a rima sempre
esteve presente nos poemas -- instrumento utilizado para favorecer a
memorização e facilitar a articulação dos repentistas. Entretanto, a
simplicidade não está atrelada apenas à oralidade, mas também ao alcance
social que uma linguagem acessível pode fornecer.

Em \emph{Uma voz do Nordeste}, estão reunidos cinco poemas dos mais
diversos tipos de métricas e rimas, representando a expressão do canto
do poeta e desvencilhando-se da forma de saber erudita.

\subsection{O legado do autor}

Em décadas de cantador, grande parte da obra de Patativa do Assaré ficou
apenas na oralidade, sem ser transcrita. Esse é um dos grandes aspectos
da poesia oral e popular: feita pensando em atender a todos os públicos,
sem empecilhos linguísticos ou sociais, transmitindo a tradição e os
acontecimentos da vida cotidiana. É importante ressaltar também a
temática das obras de Assaré que, sob aparente sutilidade e ingenuidade,
aborda temas relevantes para a população nordestina -- população essa
que, sem os cordelistas, nem sempre conseguiam ter acesso à informação.

Além de representar a voz do povo nordestino, Patativa do Assaré também
inspirou muitos escritores e músicos, contribuindo para o crescimento da
cultura popular nordestina.

\subsubsection{{Sobre o autor}}

Antônio Gonçalves da Silva, mais conhecido como Patativa do Assaré,
nasceu em 5 de março de 1909 em Assaré -- Ceará. Dono de um dos maiores
legados da Literatura de Cordel nordestina, Patativa do Assaré carrega
em suas obras a essência da cultura de seu povo, relatando suas alegrias
e sofrimentos.

\Image{Município de Assaré em vermelho, 2006. (Raphael Lorenzeto de Abreu/Wikimedia Commons; CC-BY-SA 3.0)}{PNLD0033-05.png}

Em 1995, o atual presidente do Brasil, Fernando Henrique Cardoso,
prestou uma homenagem pública ao poeta, dedicando"-lhe a medalha José de
Alencar, por suas grandes contribuições em favor do desenvolvimento
cultural do estado do Ceará.

Sua popularidade é notável. Além da premiação recebida por Fernando
Henrique Cardoso, Patativa do Assaré também sempre foi tratado com muito
carinho pelos demais sertanejos e cordelistas, que, frequentemente,
escrevem poemas e canções em sua homenagem.

\Image{Memorial Patativa do Assaré, em Assaré, Ceará. (Allan Patrick/Flickr; CC-BY-SA 2.0)}{PNLD0033-04.png}

\subsubsection{{Papel do leitor}}

O hábito de leitura auxilia na evolução de diversas habilidades, em
especial se desenvolvido durante a infância e adolescência. Ele trabalha
diretamente com o aprimoramento do vocabulário, criatividade,
imaginação, além das habilidades socioemocionais, como maior habilidade
para estabelecer diálogos, lidar com desafios e sentimentos, ajudando na
formação do indivíduo. Esses são aspectos fundamentais para uma melhor
interação social, além de proporcionar sensações ímpares.

É de extrema importância que o professor incentive o aluno a adquirir o
hábito de leitura pois, através dela, é possível que ele aprenda, viaje
e descubra sobre novos lugares e povos sem sair de casa.

{[}1{]} FREIRE, Paulo. A importância do ato de ler: em três artigos que
se completam, 23 ed, São Paulo: Autores Associados: Cortez, 1989.

{[}2{]} BRASIL. Ministério da Educação. Base Nacional Comum Curricular.
Brasília, 2018.

\section{Sugestões de referências complementares}\label{sugestoes}

\subsection{Filmes:}

\textit{Auto da Compadecida.} Direção: Guel Arraes (Brasil, 2000).

Baseado na peça de Ariano Suassuna, o filme evoca o imaginário popular e
religioso do Nordeste para contar as aventuras de João Grilo e Chicó,
uma dupla de malandros que sobrevive de trapaças.

\textit{Patativa do Assaré -- Ave Poesia}. Direção: Rosemberg Cariny
(Brasil, 2007).

O documentário apresenta a trajetória da vida e da obra do poeta
cearense Patativa do Assaré, que explorou em sua obra a riqueza das
tradições populares. Sua história é contada por meio de depoimentos de
amigos, familiares e admiradores que destacam a relevância do artista
para a cultura brasileira.

\subsection{\emph{Site}:}

\href{http://www.casaruibarbosa.gov.br/cordel/acervo.html}{Casa de Rui Barbosa}

No \emph{site} da Fundação Casa de Rui Barbosa, há informações e
materiais diversos sobre o acervo da instituição, com diversos
exemplares representativos da literatura nacional em cordel.

\section {Bibliografia comentada}

\textsc{alencar}, Maria Silvana Militão de. \textIT{A linguagem regional
popular na obra de Patativa do Assaré}. Fortaleza: Universidade
Federal do Ceará, 1997.

Este estudo, central para compreender o emprego da variedade regional
nos cordéis de Patativa do Assaré, aproxima a literatura e a análise
sociolinguística.

\textsc{andrade}, Cláudio Henrique Salles; \textsc{silva}, João Melquíades Ferreira;
BARROS Leandro Gomes de. \textit{Feira de versos: poesia de cordel}.
São Paulo: Ática, 2019.

Este livro é uma coleção de pérolas do cordel nacional. A obra reúne
textos de três importantes cordelistas: Leandro Gomes de Barros, o
primeiro a editar cordel no Brasil no século XIX, João Melquíades, que
apresenta \emph{O pavão misterioso} e Patativa do Assaré, cujos textos
vêm ganhando reconhecimento internacional.

\text{barroso}, Oswald; \textsc{barbalho}, Alexandre. \textit{Letras ao sol --
antologia da literatura cearense.} Fortaleza: Fundação Demócrito
Rocha, 1998.

A obra apresenta uma amostra representativa da poesia popular
nordestina, com ênfase na literatura do Ceará.

\textsc{farias}, Pedro Américo de. \textit{Nordestinos: coletânea poética do
Nordeste brasileiro}. Lisboa: Fragmentos, 1994.

A obra apresenta uma recolha da poesia popular nordestina, com temas
oriundos do folclore e da matéria histórica.

\textsc{figueiredo filho}, J. de. \textit{Patativa do Assaré: novos poemas
comentados}. Ceará: Museu do Ceará, 2005.

A obra apresenta uma antologia comentada de poemas populares de Patativa
do Assaré, com rica contextualização sobre o autor e seu universo.

\textsc{haurélio}, Marco. \textit{Antologia do cordel brasileiro.} São Paulo:
Global, 2012.

Nesta antologia, o leitor tem acesso a um leque variado de cordéis,
desde aqueles inspirados nos contos fantásticos e nos contos de fadas,
até outros em que predominam mitos da Grécia Antiga ou que deitam raízes
nas histórias de animais do fabulário mundial.

\item \text{\_\_\_\_\_.} \textit{Literatura de cordel: do sertão à sala de
aula.} São Paulo: Paulus, 2013.

Declamados ou cantados, os cordéis levaram ao público, da tradição oral
ao contexto escolar, as façanhas dos cangaceiros Lampião e Antônio
Silvino, os milagres do Padre Cícero e outras narrativas populares.

\item \text{\_\_\_\_\_.} \textit{Breve história da literatura de cordel.} São
Paulo: Claridade, 2018.

Esta obra apresenta as origens do Cordel, destaca seus principais
expoentes e mostra o leque de influências dessa tradição na cena
cultural brasileira.

\textsc{nascimento}, Lourgeny Damasceno do. \textit{A importância da literatura
de cordel no cotidiano dos alunos da EJA}. Monografia apresentada ao
Departamento de Artes da UNB. Brasília: 2011. (Disponível em:
\href{https://bdm.unb.br/bitstream/10483/4463/1/2011_LourgenyDamascenodoNascimento.pdf}{Biblioteca Digital da Produção Intelectual Discente}.
Acesso em 18 de fevereiro de 2021.)

A autora destaca a relevância do trabalho com poemas em cordel no
trabalho com jovens e adultos, a partir do resgate das tradições
populares e da valorização dos saberes regionais.

\textsc{suassuna}, Ariano. \textit{Romance da Pedra do Reino e o Príncipe do
Sangue do Vai-e-volta}. Rio de Janeiro: Nova Fronteira, 2017.

O romance de Ariano Suassuna, publicado originalmente em 1971, narra a
história de Dom Pedro Dinis Ferreira, o Quaderna, apresentando seu
memorial de defesa perante o corregedor, com ressonâncias da tradição
literária do cordel.

\textsc{tavares}, Braulio. \textit{Contando histórias em versos: poesia e
romanceiro popular no Brasil.} São Paulo: Editora 34, 2009.

O autor apresenta os principais recursos expressivos da linguagem
poética popular, enquanto introduz os leitores a rimas, ritmos, temas,
formas literárias e modos narrativos tipicamente brasileiros.


\end{itemize}

\end{document}

