\documentclass[12pt]{extarticle}
\usepackage{manualdoprofessor}
\usepackage{fichatecnica}
\usepackage{lipsum,media9,graficos}
\usepackage[justification=raggedright]{caption}
\usepackage[one]{bncc}
\usepackage[blooks]{../edlab}

% Ler depoimento dele sobre a propria vida 
% Ver documentário https://www.youtube.com/watch?v=CR_SJkeUN9o
% Patativa era sobretudo um declamador. 
% Paul Zunthor (tipos de culturas orais)
% Poeta cego
% As cantoras cegas de campina grande
% Cego aderaldo https://pt.wikipedia.org/wiki/As_Ceguinhas_de_Campina_Grande
% Tirésias https://pt.wikipedia.org/wiki/Tir%C3%A9sias

\begin{document}


\newcommand{\AutorLivro}{Patativa do Assaré}
\newcommand{\TituloLivro}{Ispinho e Fulô}
\newcommand{\Tema}{Ficção, mistério e fantasia}
\newcommand{\Genero}{Poema}
\newcommand{\imagemCapa}{./images/PNLD0033-01.png}
\newcommand{\issnppub}{---}
\newcommand{\issnepub}{---}
% \newcommand{\fichacatalografica}{PNLD0033-00.png}
\newcommand{\colaborador}{\textbf{Fulano de Tal} é uma pessoa incrível e vai fazer um bom serviço.}


\title{\TituloLivro}
\author{\AutorLivro}
\def\authornotes{\colaborador}

\date{}
\maketitle


\begin{abstract}
Este Manual tem como objetivo fornecer subsídios para o trabalho com a
obra literária \emph{Ispinho e Fulô}, de Patativa do Assaré.

Antonio Gonçalves da Silva, o Patativa do Assaré, foi um dos mais importantes
poetas brasileiros.  Patativa do Assaré carrega em suas obras a essência da 
cultura de seu povo, relatando suas alegrias e sofrimentos. Em 1995, o então 
presidente do Brasil, Fernando Henrique Cardoso, prestou uma homenagem pública 
ao poeta, dedicando-lhe a medalha José de Alencar, por suas grandes contribuições 
em favor do desenvolvimento cultural do estado do Ceará.

De sua obra, destacam-se os livros \textit{Inspiração nordestina} (1956), 
\textit{Cante lá que eu canto cá} (1978), \textit{Ispinho e fulô} (1988), 
\textit{Balceiro} (1991), \textit{Aqui tem coisa} (1994) e \textit{Digo e 
não peço segredo} (2001); e os discos \textit{Poemas e canções} (1979), 
\textit{A terra é naturá} (1981) e \textit{Canto nordestino} (1989).

\emph{Ispinho e Fulô} reúne setenta e oito poemas que recontam a trajetória 
de Patativa do Assaré, passando por diferentes temas e formas, apresentando as 
principais características da poesia popular. O poeta contribuiu muito para a 
construção e divulgação da identidade nordestina, utilizando, em seus poemas, 
imagens marcantes da tradição popular e uma grande variedade de imagens que 
simbolizam o nordeste brasileiro. É um poeta atento ao seu meio.

Vemos isso em “Padre Henrique e o dragão da maldade”, poema que narra o assassinato 
de um padre pelos agendes de repressão da Ditadura Militar. Outro exemplo é a 
quantidade de histórias e lendas que o poeta vai buscar na tradição popular 
para compor seus poemas. É o caso do poema “Brosogó, Militão e o diabo”, história 
protagonizada por personagens típicos e que em geral traz uma lição, uma moral. 
No poema temos Brosogó, um simples e decente vendedor ambulante, que, um dia, entra 
na casa de um coronel chamado Militão. Sem conseguir vender nada, quer comprar meia 
dúzia de ovos de Militão. Como o coronel não tem troco, diz a Brosogó que leve os ovos 
e volte depois para pagar. Brosogó prospera e consquista muitas posses. Um ano e 
sete meses depois, volta para pagar a dívida e é enganado por Militão, que faz as 
contas dizendo que os ovos teriam sido chocados e que as galinhas nascidas poriam 
novos ovos e teriam novas crias, e assim por diante. Brosogó fica desiludido, mas 
é ajudado pelo Diabo, a quem teria acendido velas em um dia em que não lembrava de 
mais nenhum santo a quem agradecer.

Esperamos que as indicações propostas aqui sejam muito úteis no trabalho em
sala de aula!



\end{abstract}

\tableofcontents

\Image{Escultura de Patativa do Assaré, Fortaleza, Ceará (Autoria Desconhecida; CC-BY-SA 2.0)}{PNLD0033-03.png}

\section{Introdução}

OLÁ! É COM MUITA ALEGRIA QUE APRESENTAMOS A VOCÊ A OBRA / “ISPINHO E FULÔ” / DE PATATIVA DO ASSARÉ / UM DOS MAIORES CORDELISTAS BRASILEIROS. // 
 
BASEADA NA VIDA E OBRA DESSE POETA, / O LIVRO DIALOGA COM A PROPOSTA DO EDUCADOR PAULO FREIRE, / DISCUTE ADVERSIDADES DA VIDA NO SERTÃO / E CONTA HISTÓRIAS DO BRASIL / QUE NÃO CONSTAM DOS ACERVOS OFICIAIS. //
 //
//
PATATIVA DO ASSARÉ,/ CUJO VERDADEIRO NOME É ANTONIO GONÇALVES DA SILVA,/ NASCEU NO DIA CINCO DE MARÇO DE MIL NOVECENTOS E NOVE NA SERRA DE SANTANA,/ PEQUENA PROPRIEDADE RURAL DA PREFEITURA DE ASSARÉ,/ AO SUL DO ESTADO DO CEARÁ //

POR MEIO DA IMPROVISAÇÃO E DA ORALIDADE,/ ALIADAS À TÉCNICA TRADICIONAL DOS CORDÉIS,/ PATATIVA CANTOU OS SOFRIMENTOS E AS ALEGRIAS DO SERTANEJO,/ SENDO ELE PRÓPRIO UMA PERSONIFICAÇÃO DAS LUTAS POR IGUALDADE SOCIAL //
//
//
A OBRA DE PATATIVA DO ASSARÉ / TRAZ A TONA O DEBATE SOBRE AS DESIGUALDADES NAS DIFERENTES REGIÕES DO BRASIL / E REVELA AS CONDIÇÕES DE VIDA PRECÁRIA DO TRABALHADOR RURAL / NO SERTÃO NORDESTINO //
//
ESSE OLHAR SOBRE O MUNDO,/ NUMA PERSPECTIVA ESPACIAL,// RECUPERA TAMBÉM UMA OPOSIÇÃO ENTRE: //

PASSADO  E PRESENTE; //
TRADIÇÃO E MODERNIDADE//

PARA ESCAPAR DA SECA / O SERTANEJO É OBRIGADO A ABANDONAR SUA TERRA / EM DIREÇÃO ÀS CIDADES DO LITORAL / OU ÀS CIDADES DO SUL //

PATATIVA RETRATA A DELICADA PASSAGEM DE UM MUNDO RURAL / A UM MUNDO URBANO EM QUE IMPERA O ANONIMATO //
//
ASSIM,/ O UNIVERSO DESCRITO POR PATATIVA DO ASSARÉ É PERCEBIDO COMO UM ESPELHO DA REALIDADE / E EXPÕE OS CONFLITOS DO SUJEITO DIANTE DOS DESAFIOS DO VASTO MUNDO //
//
EM UM CONTEXTO DE MISÉRIA E ANALFABETISMO LARGAMENTE PROPAGADO / E EM MEIO À AUSÊNCIA DE ESTRUTURAS EDUCATIVAS DE BASE, / O POETA POPULAR TEM UM PAPEL IMPORTANTE NA CONSCIÊNCIA CÍVICA E POLÍTICA //
//
AO DEFENDER A PRINCIPAL REIVINDICAÇÃO DOS HABITANTES DO SERTÃO,/ ELE SE TORNA VERDADEIRAMENTE A VOZ DO NORDESTE / E O SÍMBOLO DOS DIREITOS HUMANOS FUNDAMENTAIS //

“ISPINHO E FULO” É O TERCEIRO LIVRO DE PATATIVA DO ASSARÉ, /  PUBLICADO ORIGINALMENTE EM MIL NOVECENTOS E OITENTA E OITO. //
 
A VARIEDADE LINGUÍSTICA UTILIZADA PELO POETA / É CARACTERIZADA PELO REGISTRO DE FORMAS TÍPICAS DA LINGUAGEM ORAL //
 
OS SUBSTANTIVOS ISPINHO E FULO CORRESPONDEM,/ NO DICIONÁRIO,/ A ESPINHO E FLOR.//
 
NA DÉCADA DE SETENTA,/ PATATIVA DECLAROU:

>  “EU SOU UM CABOCLO ROCEIRO QUE,/ COMO POETA,/ CANTO SEMPRE A VIDA DO POVO! ….O MEU PROBLEMA É CANTAR A VIDA DO POVO,/ O SOFRIMENTO DO MEU NORDESTE,/ PRINCIPALMENTE DAQUELES QUE NÃO TEM TERRA.<< ”
 
A OBRA VAI DESCREVENDO OS VÁRIOS ESTÁGIOS DA VIDA,/ RESPECTIVAMENTE, A INFANCIA E A ADOLESCENCIA. //
 
A PRIMEIRA ESTROFE DO POEMA / APRESENTA O CICLO DA VIDA COMO UM FATO QUE DEVE SER ACEITO,/ COM MOMENTOS DE DOR MAIORES QUE OS DE ALEGRIA
//
AGORA VAMOS ENTENDER UM POUCO DE ONDE VEM OS CORDÉIS/

O CORDEL É UMA FORMA POÉTICA POPULAR DE ORIGEM EUROPEIA. //
 
ELE FOI INCORPORADO A CULTURA BRASILEIRA, / SOBRETUDO DO NORDESTE DO PAÍS. //
 
OS POEMAS EM CORDEL COSTUMAM SER LONGOS / E NARRAM HISTÓRIAS REPLETAS DE REVIRAVOLTAS E AVENTURAS. //
 
A LINGUAGEM É MARCADA PELO HUMOR E PELO SUSPENSE,/ COM O OBJETIVO DE CAPTAR A ATENÇÃO DO PÚBLICO. //
 
DESDE A SUA ORIGEM,/ PARA FACILITAR A MEMORIZAÇÃO E A RECITAÇÃO / ESSES TEXTOS APRESENTAM UM ESQUEMA PRÓPRIO DE RIMAS E UMA MÉTRICA ESPECÍFICA.//
 
ANTIGAMENTE, OS CORDÉIS ERAM VENDIDOS NAS FEIRAS,/ PENDURADOS EM VARAIS / E EXIBIDOS EM BALCÕES E ESTEIRAS.//
 
MUITAS VEZES,/ PERFORMANCES ERAM REALIZADAS ALI MESMO,/ PELOS AUTORES QUE DECLAMAVAM AS NARRATIVAS EM VERSOS //
 
HAVIA UMA RODA FORMADA POR PASSANTES / QUE PARAVAM PARA OUVIR AS HISTÓRIAS. //
 
HOJE,/ EM VÁRIOS LUGARES,/ ESSA CONFIGURAÇÃO MUDOU UM POUCO E OS CORDÉIS SÃO EXPOSTOS DENTRO DE PLÁSTICOS,/ EM BANCAS DE REVISTAS //
 
A POESIA MATUTA REFLETE O PENSAMENTO E A LINGUAGEM DO HOMEM DO CAMPO…./       “VENDO OS ISPINHO OMENTANDO/ E AS FULOO DIMINUINDO”... //
//
 
OBSERVE QUE A CONCORDANCIA NOMINAL DE NÚMERO / NÃO CORRESPONDE À NORMA PADRÃO. //
//
ESTE LIVRO TRAZ UM CONJUNTO DE CORDÉIS / COM ELEMENTOS DE DENÚNCIA E DE CONTESTAÇÃO / PELOS DIREITOS DOS SERTANEJOS. //
//
 
FICA EXPLÍCITA A AUSENCIA DO ESTADO NA ÉPOCA / COMO PRINCIPAL MOTIVADOR DE ATRASO NO SERTÃO. //
//
A POPULAÇÃO SOFRIA COM A SECA.// 

PATATIVA DO ASSARÉ SE DESTACA COMO PORTA VOZ DO POVO,/ VIVENCIANDO A DRAMÁTICA REALIDADE DA REGIÃO NORDESTINA//
 
A OBRA APRESENTA UM PANORAMA DA VIDA DO SERTANEJO E A OPOSIÇÃO ENTRE MUNDO URBANO E MUNDO RURAL,/ CONSTRUÍDA A PARTIR DE DIFERENÇAS SOCIOCULTURAIS E DO SISTEMA DE VALORES. //
//
 
A SITUAÇÃO DO SERTANEJO QUE ABANDONA SUA TERRA / É UMA POSIÇÃO DELICADA.//

ELE PASSA SEM TRANSIÇÃO / DE UM MUNDO RURAL / A UM MUNDO URBANO ONDE IMPERA O ANONIMATO. //


\section{Proposta de atividades I}

\subsection{Pré-Leitura}

% \paragraph{Tema}
% \paragraph{Conteúdo}
% \paragraph{Objetivo}
% \paragraph{Justificativa}
% \paragraph{Metodologia}
% 	\begin{enumerate}
% 	\end{enumerate}
% \paragraph{Tempo estimado}

Antes de se iniciar a leitura, é interessante que os alunos
entendam o universo para o qual estão enveredando. A literatura de
cordel é um patrimônio da cultura brasileira. Fabricada em folhetos,
geralmente trazem para o universo escrito casos e histórias já comuns na
oralidade popular. Também é comum que seu texto seja construído por meio
de rimas. Além disso, uma das características típicas do cordel são as
gravuras, que possuem um tracejado muito próprio. Não obstante, por
serem uma expressão muito típica da cultura de uma região do país, os
diversos títulos produzidos em cordel trazem elementos básicos que são
características intrínsecas muito próprias. Essa atividade sugere que os
alunos façam uma pesquisa com vários títulos de cordel. Procurem olhar a
sinopse da obra, folheiem seu conteúdo e busquem temas e personagens que
frequentemente aparecem nessas obras. Isso posto, é interessante que
cada aluno escreva um relatório com seus pareceres iniciais, colocando
suas suposições do porquê da repetição desses eixos temáticos e
personagens. Posteriormente, o professor pode coletar essas informações
junto aos alunos e abordar as questões durante as aulas sobre o livro.

\subsection{Leitura}

% \paragraph{Tema}
% \paragraph{Conteúdo}
% \paragraph{Objetivo}
% \paragraph{Justificativa}
% \paragraph{Metodologia}
% 	\begin{enumerate}
% 	\end{enumerate}
% \paragraph{Tempo estimado}


Já mais familiarizados com as características do cordel, os
alunos devem passar então à produção de um texto nesse formato. É
possível trabalhar com a redação de uma autobiografia do aluno, contando
sua história nos moldes de um cordel. Entretanto, é possível que alguns
alunos não se sintam confortáveis em relatar experiências próprias.
Diante disso é possível propor que seja uma biografia com elementos
romanceados, elementos do fantástico, ou mesmo a biografia de uma
terceira pessoa, pela qual o aluno nutre admiração. Em certo sentido, os
cordéis promovem isso, personagens admiráveis, de modo que escrever
sobre um ídolo ainda é algo muito produtivo nesta atividade. Além do
mais, só a produção no formato já ajudará os alunos a terem maior
familiaridade com o gênero. As rimas devem ser estimuladas, mas não são
uma obrigatoriedade, posto que podem gerar uma trava produtiva a alguns
estudantes.

\Image{Literatura de Cordel (Diego Dacal/Wikimedia Commons; CC-BY-SA 2.0)}{PNLD0033-06.png}

\subsection{Pós-Leitura}

% \paragraph{Tema}
% \paragraph{Conteúdo}
% \paragraph{Objetivo}
% \paragraph{Justificativa}
% \paragraph{Metodologia}
% 	\begin{enumerate}
% 	\end{enumerate}
% \paragraph{Tempo estimado}

 Com o material produzido na atividade anterior, é possível a
realização de uma grande feira de cordéis na escola, junto ao professor
de artes, sugere-se que os alunos preparem uma decoração adequada,
valorizando as cores e os traços presentes nesse tipo de material tão
distinto. Além da exposição dos cordéis produzidos, sugere-se também a
promoção de encenações de trechos de obras lidas, ou produzidas pelos
alunos, a declamação de poemas, competição de repentes. Também se
estimula a aproximação deste gênero com outras produções artísticas
mundiais. Nada impede que, em uma dessas encenações ou leituras
dramáticas, os alunos podem buscar trechos de peças, filmes ou obras da
literatura mundial e encená-los, valendo-se da estética e do vocabulário
típico aos cordéis.

\section{Proposta de atividades II}
A obra \emph{Ispinho e Fulô} possibilita trabalhos interdisciplinares e
integradores de diferentes campos do saber e áreas de conhecimento. A
seguir, propomos algumas atividades que podem ser desenvolvidas
conjuntamente com professores de outras áreas. Além das habilidades de
Linguagens e suas Tecnologias e de Língua Portuguesa, indicadas nas
etapas da seção anterior e válidas também para esta, listamos a seguir
as habilidades de outras áreas, presentes na abordagem interdisciplinar:


\subsection{Pré-Leitura}

% \paragraph{Tema}
% \paragraph{Conteúdo}
% \paragraph{Objetivo}
% \paragraph{Justificativa}
% \paragraph{Metodologia}
% 	\begin{enumerate}
% 	\end{enumerate}
% \paragraph{Tempo estimado}

Antes da leitura, é possível discutir a situação da seca no
Nordeste. Professores de ciências humanas podem discutir a história
desse fenômeno e suas implicações sociais, bem como as relações que o
homem travou com essa situação ao longo de toda a ocupação da região.
Por outro lado, professores de ciências da natureza podem auxiliar os
alunos na compreensão dos fenômenos geológicos e meteorológicos que
contribuem para a perpetuação da aridez local, bem como explicar o bioma
da caatinga e sua importância para o equilíbrio ecológico. Munidos dessa
informação os alunos podem produzir um jornal, tanto nos moldes
escritos, como em vídeo, construindo uma reportagem sobre todas as
questões aventadas. O material produzido pode ser divulgado no site da
escola. Como sugestão, indica-se utilizar aplicativos gratuitos de
gravação e edição de vídeos, disponíveis em dispositivos digitais.

\Image{Flor de mandacaru registrada em trilha na Bahia. (Wellscte/Wikimedia Commons; CC-BY-SA 4.0)}{PNLD0033-08.png}

\subsection{Leitura}

% \paragraph{Tema}
% \paragraph{Conteúdo}
% \paragraph{Objetivo}
% \paragraph{Justificativa}
% \paragraph{Metodologia}
% 	\begin{enumerate}
% 	\end{enumerate}
% \paragraph{Tempo estimado}

A obra Ispinho e Fulô é baseada na vida e na obra de
Patativa do Assaré, passando por diferentes temas e versificações,
apresentando as principais características da poesia popular -- poesia
essa que, muitas vezes, é deixada de lado por se distanciar do erudito.
No entanto, Patativa do Assaré contribuiu ativamente para a construção e
divulgação da identidade nordestina, utilizando, em seus poemas, imagens
marcantes da tradição popular e uma grande variedade de personagens que
representam o nordeste brasileiro.

Este livro apresenta a identidade sertaneja e histórias do sertão muitas
vezes não registradas pela historiografia oficial. A realidade vivida e
retratada por Patativa do Assaré é mostrada com tamanha fidelidade, que
fez com que ele recebesse o título de ``poeta da oralidade'', inspirando
poetas, cantadores e músicos. Indague aos alunos se no próprio Brasil
existe, ou não, uma visão estereotipada de uma região frente a outra, e
promova a escrita de um artigo de opinião defendendo a valorização das
identidades regionais.

\subsection{Pós-Leitura}

% \paragraph{Tema}
% \paragraph{Conteúdo}
% \paragraph{Objetivo}
% \paragraph{Justificativa}
% \paragraph{Metodologia}
% 	\begin{enumerate}
% 	\end{enumerate}
% \paragraph{Tempo estimado}

Na obra lida, é feita a oposição entre campo e cidade, e as
características das pessoas que habitam esses dois ambientes. Nesta
temática, do homem rural em oposição ao homem urbano, proponha aos
alunos, contando com o auxílio do professor de ciências humanas, a
criação de um mapa. Nele, inicialmente divida as regiões em
majoritariamente urbanas e rurais. Feito isso, aconselha-se buscar
notícias e estudos sobre as grandes desigualdades e os fatores do
deslocamento populacional para as grandes cidades. Com essa informação,
é possível indicar de maneira gráfica, valendo-se de setas por exemplos,
caso seja um painel físico, ou por recursos animados, caso o mapa esteja
sendo produzido virtualmente, os fluxos migratórios estudados. Como
fonte, também, sugere-se orientar a pesquisa por meio da obra de
Patativa do Assaré e de outros autores nordestinos.

Para complementar a obra, sugere-se criar xilogravuras a partir da
confecção de carimbos artesanais.

\Image{Xilogravura da Artista plástica Yolanda Carvalho (Rafael Nolêto/Wikimedia Commons; CC-BY-SA 3.0)}{PNLD0033-10.png}

\section{Aprofundamento}

Ao chegar ao Ensino Médio, é necessário que os estudantes se aprofundem
na compreensão das múltiplas linguagens e, sobretudo, da linguagem
literária. Em relação à literatura, a BNCC traz as seguintes
considerações:

``{[}...{]} a leitura do texto literário, que ocupa o centro do trabalho
no Ensino Fundamental, deve permanecer nuclear também no Ensino Médio.
Por força de certa simplificação didática, as biografias de autores, as
características de épocas, os resumos e outros gêneros artísticos
substitutivos, como o cinema e as HQs, têm relegado o texto literário a
um plano secundário do ensino. Assim, é importante não só (re)colocá-lo
como ponto de partida para o trabalho com a literatura, como
intensificar seu convívio com os estudantes. Como linguagem
artisticamente organizada, a literatura enriquece nossa percepção e
nossa visão de mundo. Mediante arranjos especiais das palavras, ela cria
um universo que nos permite aumentar nossa capacidade de ver e sentir.
Nesse sentido, a literatura possibilita uma ampliação da nossa visão do
mundo, ajuda-nos não só a ver mais, mas a colocar em questão muito do
que estamos vendo/vivenciando.'' (Brasil, 2018, p. 491)

Nesta seção, desenvolvemos um trabalho de aprofundamento que, em diálogo
com a formação continuada de professores, oferece subsídios para a
abordagem do texto literário. A leitura em sala de aula de
\emph{Patativa do Assaré} pode ser enriquecida pelo aprofundamento no
universo literário.

\subsection{o romance da maturidade}

O texto literário, presente no currículo escolar desde o Ensino
Fundamental, deve também permanecer como base de formação no Ensino
Médio, para que os alunos possam se aprofundar na compreensão não só da
linguagem literária, mas também das múltiplas linguagens, estimulando a
criatividade e imaginação. Segundo Paulo Freire (1989){[}1{]}, o ato de
ler implica na percepção crítica, interpretação, reescrita e
reelaboração do que lemos, agindo assim, diretamente, com diferentes
esferas de aprendizagem.

Este manual oferece subsídios para a abordagem do texto literário em
diálogo com a formação continuada de professores, apresentando
propostas, com diferentes abordagens, para o trabalho de \emph{Vocação
de Cantador, de Oliveira de Panelas,} dentro e fora sala, incentivando
os alunos a adquirirem o hábito de leitura e pensamento crítico.

\subsection{importância da literatura de cordel em sala}

Com uma produção simples e de grande abrangência, a Literatura de Cordel
ganhou espaço e prestígio na cultura nordestina brasileira, tornando-se,
em 2018, patrimônio cultural imaterial do nosso país -- reconhecido pelo
Conselho Consultivo do Instituto do Patrimônio Histórico e Artístico
Nacional. Assim, o Cordel como gênero do discurso contribui na formação
do aluno possibilitando o domínio de outros conteúdos, além da
descentralização do ensino. O estudo da Literatura de Cordel como forma
de expressão da cultura popular contribui também no aprimoramento das
habilidades de oralidade, escrita, leitura, interpretação, linguagens
artísticas e até na dramatização de peças, auxiliando na
interdisciplinaridade dos temas.

A Educação Literária aparece entre as competências gerais da BNCC (Base
Nacional Comum Curricular), enfatizando a importância vivência do aluno
no aprendizado da literatura e demais manifestações artísticas --
``\emph{Valorizar e fruir as diversas manifestações artísticas e
culturais, das locais às mundiais, e também participar de práticas
diversificadas da produção artístico-cultural\textbf{{[}2{]}}'' --},
logo, o Cordel ganha espaço nesse cenário, pois, a partir de seu estudo,
é possível despertar nos alunos o interesse por diversos campos
artísticos.

Entretanto, o uso do Cordel em sala de aula, bem como nos livros
didáticos, ainda é muito restrito, por não ser tão prestigiado quanto os
demais gêneros literários. Diante disso, cabe a nós resgatarmos essa
parte da nossa identidade nacional, contextualizando o aluno no meio
social e cultural de seu país.

\Image{Ilustração da capa do cordel ``O ABC da Cachaça'' (Thomas Fisher Rare Book Library/Flickr; CC-BY-NC-SA 2.0)}{PNLD0033-07}

\subsection{variação linguística e oralidade}

O Cordel nasceu da oralidade e da linguagem popular; uma leitura
silenciosa limita seu poder de comunicação, impedindo que seu potencial
seja trabalhado como um todo. O gênero, se bem explorado, pode auxiliar
no aprendizado e desenvoltura dos alunos nessa modalidade de expressão,
devido ao seu ritmo cadenciado e seu linguajar comum, próximo ao
cotidiano do aluno. Para isso, o professor deve promover atividades que
possibilitem a verbalização do aluno, que estimulem a livre expressão,
para que ele possa, a partir dos exercícios, identificar seu local de
fala, além de desenvolver respeito e empatia pela fala do outro,
aprimorando, também, a convivência social.

Ao utilizar a Literatura de Cordel, o professor poderá abordar a questão
do preconceito linguístico da língua portuguesa, ao estimular a leitura
de poemas que fogem do padrão gramaticalmente institucionalizado. É
possível mostrar aos alunos que a linguagem popular é muitas vezes
discriminada, mesmo fazendo parte de uma cultura rica e diversificada,
quebrando a ideia de que o ideal é necessariamente o padrão.

\subsection{variação cultural e geográfica}

Como explicitado anteriormente, é sabido que a Literatura de Cordel faz
parte de nossa cultura e tradição. Antes mesmo da chegada das grandes
mídias e meios de comunicação, o Cordel funcionou como instrumento de
disseminação de valores, lendas e conhecimento popular da tradição
nordestina. Levá-lo à escola é uma maneira de resgate da nossa cultura,
motivando o aluno a conhecer mais sobre nosso país e seus diferentes
povos e regiões, além de nossa história religiosa, econômica e política,
vez que muitos cordéis abordam realisticamente essas questões.

\subsection{campo artístico e literário}

A ilustração com xilogravura (gravura em madeira) é uma característica
marcante dos folhetos de Cordel, usada para decorar e dar mais vida aos
poemas, além de oferecer material para as mais variadas interpretações
das obras. O uso da técnica deu-se graças ao baixo custo de produção e
foi fundamental para disseminar a cultura do Nordeste em outras partes
do Brasil. Os traços marcantes da xilogravura de cordel em composição
com os poemas se transformam em uma expressão de linguagem, registrando
a história do nosso povo.

Levar componentes artísticos para a sala de aula é uma forma de chamar à
atenção do aluno, além de proporcionar maior pensamento crítico e
incentivo à expressão artística e literária.

\Image{Capa de ``A Xilogravura Popular e a Literatura de Cordel'', 1985 (Thomas Fisher Rare Book Library/Flickr; CC-BY-NC-SA 2.0)}{PNLD0033-09.png}

\subsection{sobre a obra }

\emph{Ispinho e Fulô} reúne poemas que recontam a trajetória de Patativa
do Assaré, passando por diferentes temas e versificações, apresentando
as principais características da poesia popular -- poesia essa que,
muitas vezes, é deixada de lado por se distanciar do erudito. No
entanto, Patativa do Assaré contribuiu ativamente para a construção e
divulgação da identidade nordestina, utilizando, em seus poemas, imagens
marcantes da tradição popular e uma grande variedade de imagens que
simbolizam o nordeste brasileiro.

\subsection{a estrutura da métrica}

A Literatura de Cordel sofreu, estruturalmente, diversas modificações
com o passar dos anos, por se tratar de uma linguagem oral que foi sendo
transformada, também, em escrita. No início, os repentistas não tinham
compromisso com número de versos ou métrica, entretanto a rima sempre
esteve presente nos poemas -- instrumento utilizado para favorecer a
memorização e facilitar a articulação dos repentistas. Entretanto, a
simplicidade não está atrelada apenas à oralidade, mas também ao alcance
social que uma linguagem acessível pode fornecer.

Em \emph{Uma voz do Nordeste}, estão reunidos cinco poemas dos mais
diversos tipos de métricas e rimas, representando a expressão do canto
do poeta e desvencilhando-se da forma de saber erudita.

\subsection{o legado do autor}

Em décadas de cantador, grande parte da obra de Patativa do Assaré ficou
apenas na oralidade, sem ser transcrita. Esse é um dos grandes aspectos
da poesia oral e popular: feita pensando em atender a todos os públicos,
sem empecilhos linguísticos ou sociais, transmitindo a tradição e os
acontecimentos da vida cotidiana. É importante ressaltar também a
temática das obras de Assaré que, sob aparente sutilidade e ingenuidade,
aborda temas relevantes para a população nordestina -- população essa
que, sem os cordelistas, nem sempre conseguiam ter acesso à informação.

Além de representar a voz do povo nordestino, Patativa do Assaré também
inspirou muitos escritores e músicos, contribuindo para o crescimento da
cultura popular nordestina.

\subsubsection{{Sobre o autor}}

Antônio Gonçalves da Silva, mais conhecido como Patativa do Assaré,
nasceu em 5 de março de 1909 em Assaré -- Ceará. Dono de um dos maiores
legados da Literatura de Cordel nordestina, Patativa do Assaré carrega
em suas obras a essência da cultura de seu povo, relatando suas alegrias
e sofrimentos.

\Image{Município de Assaré em vermelho, 2006. (Raphael Lorenzeto de Abreu/Wikimedia Commons; CC-BY-SA 3.0)}{PNLD0033-05.png}

Em 1995, o atual presidente do Brasil, Fernando Henrique Cardoso,
prestou uma homenagem pública ao poeta, dedicando-lhe a medalha José de
Alencar, por suas grandes contribuições em favor do desenvolvimento
cultural do estado do Ceará.

Sua popularidade é notável. Além da premiação recebida por Fernando
Henrique Cardoso, Patativa do Assaré também sempre foi tratado com muito
carinho pelos demais sertanejos e cordelistas, que, frequentemente,
escrevem poemas e canções em sua homenagem.

\Image{Memorial Patativa do Assaré, em Assaré, Ceará. (Allan Patrick/Flickr; CC-BY-SA 2.0)}{PNLD0033-04.png}

\subsubsection{{Papel do leitor}}

O hábito de leitura auxilia na evolução de diversas habilidades, em
especial se desenvolvido durante a infância e adolescência. Ele trabalha
diretamente com o aprimoramento do vocabulário, criatividade,
imaginação, além das habilidades socioemocionais, como maior habilidade
para estabelecer diálogos, lidar com desafios e sentimentos, ajudando na
formação do indivíduo. Esses são aspectos fundamentais para uma melhor
interação social, além de proporcionar sensações ímpares.

É de extrema importância que o professor incentive o aluno a adquirir o
hábito de leitura pois, através dela, é possível que ele aprenda, viaje
e descubra sobre novos lugares e povos sem sair de casa.

{[}1{]} FREIRE, Paulo. A importância do ato de ler: em três artigos que
se completam, 23 ed, São Paulo: Autores Associados: Cortez, 1989.

{[}2{]} BRASIL. Ministério da Educação. Base Nacional Comum Curricular.
Brasília, 2018.

\section{Referências complementares}


\subsection{Livros}

\begin{itemize}
\item \textsc{andrade}, Cláudio Henrique Salles; SILVA, João Melquíades Ferreira;
BARROS Leandro Gomes de. \textbf{Feira de versos: poesia de cordel}.
São Paulo: Ática, 2019.

Este livro é uma coleção de pérolas do cordel nacional. A obra reúne
textos de três importantes cordelistas: Leandro Gomes de Barros, o
primeiro a editar cordel no Brasil no século XIX, João Melquíades, que
apresenta \emph{O pavão misterioso} e Patativa do Assaré, cujos textos
vêm ganhando reconhecimento internacional.


\item \textsc{haurélio}, Marco. \textbf{Antologia do cordel brasileiro.} São Paulo:
Global, 2012.

Nesta antologia, o leitor tem acesso a um leque variado de cordéis,
desde aqueles inspirados nos contos fantásticos e nos contos de fadas,
até outros em que predominam mitos da Grécia Antiga ou que deitam raízes
nas histórias de animais do fabulário mundial.


\item \textsc{suassuna}, Ariano. \textbf{Romance da Pedra do Reino e o Príncipe do
Sangue do Vai-e-volta}. Rio de Janeiro: Nova Fronteira, 2017.

O romance de Ariano Suassuna, publicado originalmente em 1971, narra a
história de Dom Pedro Dinis Ferreira, o Quaderna, apresentando seu
memorial de defesa perante o corregedor, com ressonâncias da tradição
literária do cordel.
\end{itemize}

\subsection{Filmes}

\begin{itemize}
\item \textbf{Auto da Compadecida. Direção: Guel Arraes (Brasil}, 2000).

Baseado na peça de Ariano Suassuna, o filme evoca o imaginário popular e
religioso do Nordeste para contar as aventuras de João Grilo e Chicó,
uma dupla de malandros que sobrevive de trapaças.


\item \textbf{Patativa do Assaré -- Ave Poesia}. Direção: Rosemberg Cariny
(Brasil, 2007).

O documentário apresenta a trajetória da vida e da obra do poeta
cearense Patativa do Assaré, que explorou em sua obra a riqueza das
tradições populares. Sua história é contada por meio de depoimentos de
amigos, familiares e admiradores que destacam a relevância do artista
para a cultura brasileira.
\end{itemize}


  \subsection{\emph{Site}}


\subsection{cordel: literatura popular em verso }

(\url{http://www.casaruibarbosa.gov.br/cordel/acervo.html})

No \emph{site} da Fundação Casa de Rui Barbosa, há informações e
materiais diversos sobre o acervo da instituição, com diversos
exemplares representativos da literatura nacional em cordel.

\subsection{organizando a estante:} Bibliografia comentada

\begin{itemize}

\item \textsc{alencar}, Maria Silvana Militão de. \textbf{A linguagem regional
popular na obra de Patativa do Assaré}. Fortaleza: Universidade
Federal do Ceará, 1997.

Este estudo, central para compreender o emprego da variedade regional
nos cordéis de Patativa do Assaré, aproxima a literatura e a análise
sociolinguística.


\item \text{barroso}, Oswald; BARBALHO, Alexandre. \textbf{Letras ao sol --
antologia da literatura cearense.} Fortaleza: Fundação Demócrito
Rocha, 1998.

A obra apresenta uma amostra representativa da poesia popular
nordestina, com ênfase na literatura do Ceará.


\item \text{farias}, Pedro Américo de. \textbf{Nordestinos: coletânea poética do
Nordeste brasileiro}. Lisboa: Fragmentos, 1994.

A obra apresenta uma recolha da poesia popular nordestina, com temas
oriundos do folclore e da matéria histórica.


\item \text{figueiredo filho}, J. de. \textbf{Patativa do Assaré: novos poemas
comentados}. Ceará: Museu do Ceará, 2005.

A obra apresenta uma antologia comentada de poemas populares de Patativa
do Assaré, com rica contextualização sobre o autor e seu universo.


\item \text{haurélio}, Marco. \textbf{Literatura de cordel: do sertão à sala de
aula.} São Paulo: Paulus, 2013.

Declamados ou cantados, os cordéis levaram ao público, da tradição oral
ao contexto escolar, as façanhas dos cangaceiros Lampião e Antônio
Silvino, os milagres do Padre Cícero e outras narrativas populares.


\item \text{\_\_\_\_\_.} \textbf{Breve história da literatura de cordel.} São
Paulo: Claridade, 2018.

Esta obra apresenta as origens do Cordel, destaca seus principais
expoentes e mostra o leque de influências dessa tradição na cena
cultural brasileira.


\item \textsc{nascimento}, Lourgeny Damasceno do. \textbf{A importância da literatura
de cordel no cotidiano dos alunos da EJA}. Monografia apresentada ao
Departamento de Artes da UNB. Brasília: 2011. (Disponível em:
\url{https://bdm.unb.br/bitstream/10483/4463/1/2011_LourgenyDamascenodoNascimento.pdf}.
Acesso em 18 de fevereiro de 2021.)

A autora destaca a relevância do trabalho com poemas em cordel no
trabalho com jovens e adultos, a partir do resgate das tradições
populares e da valorização dos saberes regionais.


\item \textsc{tavares}, Braulio. \textbf{Contando histórias em versos: poesia e
romanceiro popular no Brasil.} São Paulo: Editora 34, 2009.

O autor apresenta os principais recursos expressivos da linguagem
poética popular, enquanto introduz os leitores a rimas, ritmos, temas,
formas literárias e modos narrativos tipicamente brasileiros.
\end{itemize}

\end{document}

