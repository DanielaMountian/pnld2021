\documentclass[11pt]{extarticle}
\usepackage{manualdoprofessor}
\usepackage{fichatecnica}
\usepackage{lipsum,media9,graficos}
\usepackage[justification=raggedright]{caption}
\usepackage[one]{bncc}
\usepackage[circuito]{../edlab}
 


\begin{document}


\newcommand{\AutorLivro}{Mário de Andrade}
\newcommand{\TituloLivro}{Eu tudoamo: antologia de poemas}
%antologia de poemas de Mário de Andrade}
\newcommand{\Tema}{Ficção, mistério e fantasia}
\newcommand{\Genero}{Poema}
\newcommand{\imagemCapa}{./images/PNLD0005-01.png}
\newcommand{\issnppub}{978-65-86974-19-5}
\newcommand{\issnepub}{978-65-86974-23-2}
% \newcommand{\fichacatalografica}{PNLD0005-00.png}
\newcommand{\colaborador}{Rogério Barreiros}


\title{\TituloLivro}
\author{\AutorLivro}
\def\authornotes{\colaborador}

\date{}
\maketitle


\begin{abstract}\addcontentsline{toc}{section}{Carta ao professor}

É com muita alegria que lhe apresentamos a obra \emph{Eu tudoamo: antologia de poemas},	
de Mário de Andrade. Cada um desses textos tem um
interesse específico e, reunidos, eles compõem um conjunto bastante
fértil para análise em sala de aula. Mas, antes de falar de cada um
deles especificamente, é necessário avaliar a importância de ler Mário
de Andrade no Ensino Médio e justificar, de forma geral, nossa escolha
pelos poemas desse autor consagrado.

Primeiramente, acreditamos que os poemas de Mário de Andrade podem ser
um ótimo ponto de partida para os estudantes conhecerem outros poetas.
Acreditamos também que tudo depende da escolha dos textos: para ser
porta de entrada para outras obras, eles têm de ser do interesse de
nossos alunos -- por isso é que selecionamos os poemas desta antologia:
neles é possível identificar temas atuais, que merecem discussão. Além
disso, como pretendemos mostrar a seguir, o interesse de Mário de
Andrade pelas manifestações orais de poesia e pela canção popular
dialoga diretamente com os saraus, \emph{slams} e raps das periferias
brasileiras do século \textsc{xxi} -- que pertencem ao repertório de nossos
alunos.

Além disso, também vale destacar a atuação de Mário de Andrade no
serviço público e a vasta correspondência pessoal desse autor. Essas
duas atividades aparentemente diferentes, aparentemente opostas, revelam
um aspecto comum muito significativo para a compreensão do conjunto da
obra desse autor. Quando observamos as iniciativas de Mário -- como a
Missão de Pesquisas Folclóricas, quando ele era Diretor do Departamento
de Cultura do Município de São Paulo -- e analisamos o diálogo que ele
mantinha com outros artistas e intelectuais de sua geração, em extensas
cartas, pelas quais ele também ficou famoso, percebemos por trás de
todas essas ações um sentido profundo de serviço ao Brasil. Mário de
Andrade tinha a obsessão do colecionador não apenas pelo desfrute das
obras -- fossem elas livros, pinturas, esculturas, registros escritos ou
gravados em áudio e vídeo de danças e canções --, mas também pelo desejo
de produzir uma mediação e um diálogo entre as muitas culturas
brasileiras, que ele pesquisava de forma incansável. É notável no
conjunto da produção de Mário de Andrade esse intuito \emph{mediador},
sem disposição de conciliar e apagar as diferenças, mas com o firme
propósito de produzir traduções entre as diferentes expressões das
culturas brasileiras, sem forçar as aproximações e criando uma linguagem
própria. É o que se observa, por exemplo, na sua obra-prima,
\emph{Macunaíma}.

\Image{Epitáfio da mãe de Macunaíma, 1928 (Biblioteca Digital de Literatura de Países Lusófonos; Domínio Público)}{PNLD0005-06.png}

\Image{Retrato de Mário de Andrade, intitulado ``Ridícula Aposta em Teffé'', 12 de junho de 1927. (Instituto de Estudos Brasileiros; Domínio Público)}{PNLD0005-03.png}

Dizendo de forma simples, na poesia e na prosa, o ambicioso projeto de
Mário de Andrade era converter a experiência de ser brasileiro em texto
literário, criando uma \emph{forma literária} que nos caracterizasse e
com a qual nos identificássemos, especialmente a partir de nossa
musicalidade popular, manifesta nas danças, nas canções e nas festas que
esse escritor pesquisou de forma incansável.

Somente grandes escritores têm essa habilidade -- e é nossa função
mostrá-la aos estudantes, para que eles possam avaliar melhor os
escritores contemporâneos à luz dessa experiência acumulada. A leitura
de Mário de Andrade nas nossas salas de aula é sempre um bom farol para
iluminar as avaliações que fazemos da produção literária atual, porque
poucos escritores entenderam o Brasil de forma tão aprofundada e
precisa.

No caso específico de Mário de Andrade, destaca-se, no contexto da
Semana de Arte Moderna, a valorização das matrizes orais e musicais de
nossas culturas, associada à ideia de serviço público ao Brasil, com o
objetivo de superar os atrasos de ordem econômica e avançar no sentido
da modernização, sem abrir mão das muitas vozes e suas culturas que se
sedimentaram por aqui -- uma proposta ousada, quase utópica, mas
certamente inspiradora e muito atual. Dizendo de maneira simples, a
liberdade formal dos primeiros modernistas e sua expressão específica na
obra de Mário de Andrade abriram as portas para todos os outros poetas
que vieram depois. E é por todos esses motivos que convidamos você,
professora, professor, a mergulhar mais uma vez na obra dele.
\end{abstract}

\pagebreak
\tableofcontents

\section{Propostas de Atividades I}

\subsection{A linguagem moderna de Mário de Andrade}

\subsubsection{Pré-leitura}



\paragraph{Tema} Vida e obra de Mário de Andrade


\paragraph{Conteúdo} Análise da vida e da obra de Mário de Andrade

\paragraph{Objetivos} 1. Introduzir conhecimentos gerais a respeito da
vida e da obra de Mário de Andrade; 2. Contextualizar os poemas da
coletânea no conjunto dessa obra e em seu momento histórico,
especialmente o Modernismo e a Semana de Arte Moderna; 3. Avaliar a
atualidade da obra de Mário de Andrade e seu alcance entre os leitores
do século \textsc{xxi}.

\paragraph{Justificativa} Contextualizar a leitura é motivar os alunos a
efetuá-la, daí a proposta de apresentar o autor e o conjunto da sua obra
antes de partir para os textos da antologia. No caso específico de Mário
de Andrade, essa atividade talvez seja ainda mais necessária, devido à
dificuldade de leitura de alguns poemas desse escritor.
\BNCC{EM13LP52}

\paragraph{Metodologia}


1. Dividir os estudantes em grupos de pesquisa que analisarão diferentes
passagens da vida e da obra de Mário de Andrade e as apresentarão aos
colegas e ao professor. Sugerimos a divisão a seguir:

\begin{enumerate}
\item a família e a infância do autor, especialmente sua formação musical,
e as primeiras obras;

\item a Semana de Arte Moderna de 1922 e as amizades desse período: Oswald
de Andrade, Tarsila do Amaral e Anita Malfatti;

\item \emph{Macunaíma}, a obra-prima de Mário de Andrade, publicada em
1928;

\item a atuação de Mário de Andrade como servidor público, em São Paulo e
Rio de Janeiro (aqui, podem ser destacadas iniciativas específicas, como
a Missão de Pesquisas Folclóricas, a fundação do SPHAN, Serviço do
Patrimônio Histórico e Artístico Nacional, atual IPHAN, e o projeto da
Enciclopédia Brasileira, quando era funcionário do Instituto Nacional do
Livro, vinculado ao MEC);

\item a famosa correspondência de Mário de Andrade, especialmente com os
poetas Carlos Drummond de Andrade e Manuel Bandeira;
\end{enumerate}

\SideImage{Mário de Andrade (Hotsite Missões de Pesquisas Folclóricas, fotógrafo: Luís Saia; CC BY-NC-ND 2.0)}{PNLD0005-21.png}

Além de elementos da vida de Mário de Andrade, alguns grupos de alunos
podem dedicar-se à apresentação de painéis de contextualização da época,
especialmente no que se refere ao contexto das artes europeias da época,
as chamadas Vanguardas. Elas podem ser estudadas separadamente. Da mesma
maneira, os alunos mais interessados nas artes visuais podem apresentar
à sala as obras de Tarsila do Amaral e Anita Malfatti; os que gostam de
música podem debruçar-se sobre Heitor Villa-Lobos; os que preferem a
literatura podem pesquisar Oswald de Andrade e Manuel Bandeira, entre
outros autores.

De forma geral, o que interessa é que esta atividade sirva para a
contextualização da leitura dos poemas de nossa antologia. Os temas
acima propostos são sugestões gerais, às quais você pode incorporar
outras, de acordo com o interesse dos alunos e com o seu percurso de
ensino.

\vspace{3mm}

2. Cada um dos grupos deve pesquisar os temas apresentados em arquivos
da internet. Cabe ao professor orientar esse processo de pesquisa. Quais
são as fontes mais confiáveis? Quais são as menos confiáveis? Por quê?
\BNCC{EM13LP30}

\vspace{3mm}

3. Depois de completo o processo de pesquisa, os estudantes podem
apresentar os resultados obtidos de uma das seguintes maneiras:

\begin{enumerate}

\item \textit{podcast} de até 15 minutos, com entrega de roteiro na forma de texto;

\item vídeo publicado no YouTube, no formato de documentário, em até 15
minutos, com entrega de roteiro na forma de texto;

\item peça teatral de até 15 minutos, com entrega de roteiro na forma de
texto;

\item leitura em voz alta de folheto de cordel, de até 15 minutos, com
entrega desse folheto ao professor;

\item apresentação formal tradicional, de até 15 minutos, com entrega de
relatório de pesquisa;

\item declamação de poema no formato de slam poesia, de até 15 minutos, com
entrega do poema escrito;

\item canção de até 15 minutos, que pode ser de composição original dos
estudantes, ou paródia de canção já existente, com entrega da letra e da
cifra.
\end{enumerate}

4. É necessário destacar que cada uma das formas de apresentação dos
resultados requer dos estudantes a pesquisa sobre o gênero específico
dos textos;

\vspace{3mm}

5. Em qualquer um dos formatos de relatório, é fundamental que, além de
apresentar fatos objetivos a respeito da vida e da obra de Mário de
Andrade, os estudantes argumentem em defesa de um ponto de vista;

\vspace{3mm}

6. Depois das apresentações, os alunos deverão apresentar aos colegas as
dificuldades enfrentadas nos processos de pesquisa e preparação dos
relatórios específicos. Deverão também apresentar o percurso por meio do
qual chegaram às conclusões e pontos de vista propostos nos relatórios a
respeito das polêmicas da vida e da obra de Mário de Andrade.

\subsubsection{Leitura}


\paragraph{Tema} Panorama da poesia de Mário de Andrade

\paragraph{Conteúdo} 1. Leitura, compreensão e análise de poemas da
antologia; 2. Características da lírica moderna.

\paragraph{Objetivos} 1. Estimular a leitura dos poemas da antologia, em
grupo ou individualmente; 2. Promover a reflexão sobre os elementos que
compõem a lírica moderna; 3. Ler, compreender, analisar e comparar os
textos da antologia; 4. Estabelecer relações entre a poesia de Mário de
Andrade e tendências contemporâneas da poesia, especialmente o
\emph{slam} e os saraus das periferias.

\paragraph{Justificativa} As edições de 2011, 2015 e 2019 da pesquisa
Retratos do Brasil revelam um dado inegável: a influência do professor
na formação dos leitores. Para despertar o gosto pela leitura entre
nossos alunos, precisamos ser \emph{mediadores} entre a obra, sua
linguagem, suas estruturas e o estudante, de preferência estabelecendo
uma relação fundamentada no prazer, na identificação e na liberdade de
interpretação. Eis o nosso desafio: \emph{ler com os alunos},
apresentando-lhes as passagens decisivas de um texto -- porque são
engraçadas, assustadoras, emocionantes etc. --, explicando-lhes
\emph{por que} elas chamam a atenção, estabelecendo \emph{conexões}
entre elas e nossa vida presente, revelando as intuições do autor quanto
às práticas sociais atuais, seus conflitos, dilemas, conquistas, ouvindo
as impressões dos estudantes a respeito de tudo isso e propondo-lhes a
análise do que eles gostam -- canções, séries de TV ou de plataformas de
\emph{streaming}, novelas, filmes e outros produtos culturais -- a
partir dessa comparação.
\BNCC{EM13LGG602}

Dizendo de maneira simples, o processo de
formação de leitor deve estar baseado no repertório de nossos próprios
estudantes, a partir do qual podemos estabelecer os pontos de contato
com as obras que lhes apresentamos.

\paragraph{Metodologia}

1. A primeira atividade imprescindível é a leitura, compreensão e
análise de poemas da antologia. O ideal é que alguns textos sejam lidos
\emph{integralmente} na sala de aula. Conscientes das dificuldades dessa
atividade no contexto escolar, sugerimos que esse exercício de leitura
seja preparado previamente: os alunos podem preparar-se para uma
\emph{leitura interpretativa}, que não seja monótona, com a orientação
prévia do professor;

\vspace{3mm}

2. O exercício de compreensão do texto se baseia na análise dialogada de
elementos fundamentais de alguns textos da antologia -- que permitirão,
por sua vez, reconhecer as características da poesia, como gênero. É
fundamental sugerir aqui que essa análise não seja feita de forma
teórica, anterior à leitura, mas que \emph{seja depreendida a partir da
leitura do texto}. Sabemos que esse é um exercício desafiador, mas nos
parece ainda o melhor caminho para habilitar os estudantes à compreensão
e à análise de textos literários;

\vspace{3mm}

3. Os elementos fundamentais dos textos -- as figuras de linguagem,
especialmente as sonoras, as regularidades ou irregularidades, e o
ritmo, do ponto de vista sonoro; as imagens e os temas típicos da lírica
moderna, no plano do conteúdo -- revelarão características gerais do
poema e as especificidades de cada um deles. Na teoria tradicional, a
poesia lírica se caracteriza pela expressão dos sentimentos do eu lírico
em formas fixas, mas a lírica moderna é muito mais livre, do ponto de
vista formal, além de contemplar temas que a tradição considerava
menores, como os acontecimentos cotidianos;
\BNCC{EM13LP52}

\vspace{3mm}

4. A comparação entre poemas da antologia é fundamental para a
compreensão, não apenas dos elementos internos de cada um deles, mas
também do gênero a que pertencem, de maneira geral;

\vspace{3mm}

5. Lembre-se de que a formação de leitores se dá pela
\emph{identificação afetiva:} pais, mães e professores devem \emph{ler
com} crianças e jovens, para mediar essa experiência. Por isso, nossa
proposta é que você leia com seus alunos, ria com eles, mergulhe no
texto e mostre o que é mais interessante. Desvende com eles os detalhes
e os elementos analítico-interpretativos que eles deixariam passar sem a
sua orientação. Mostre a seus estudantes que um dos prazeres do leitor é
ser uma espécie de detetive que vai encontrando pistas para desvendar os
sentidos do poema;

\vspace{3mm}

6. Como já foi sugerido acima, incentive a \emph{leitura interpretativa}
em sala de aula, de forma lúdica, em que você e os alunos interpretem o
texto como se fossem atores ou apresentadores de tevê, analisando os
modos de entoar: esse exercício dialogado e coletivo servirá de ponto de
partida para conhecer os meandros da lírica. Teste os ritmos possíveis
para cada verso, estrofe e poema completo, antes de buscar um
significado do poema; mostre, sobretudo, que a identificação do ritmo do
poema é a chave para encontrar-lhe o sentido profundo. Reconheça, com os
alunos, as imagens predominantes e lembre sempre a eles que a poesia não
precisa, e muitas vezes \emph{não quer} contar uma história com começo,
meio e fim, pretensão mais associada aos gêneros da prosa, como o
romance ou o conto;

\vspace{3mm}

7. A poesia lírica, de forma geral, como destacamos acima, se
caracteriza pela expressão de sentimentos, emoções e aflições do eu
lírico, por isso a tendência é que esses poemas sejam curtos e intensos;
na lírica moderna, essa subjetividade se manifesta em formas livres,
abertas, com temas e imagens da sociedade urbana industrial. Poemas de
\emph{Pauliceia Desvairada} são bastante úteis para reconhecer essas
características, mas não nos esqueçamos de que Mário de Andrade era
grande pesquisador das manifestações populares de nossa cultura, que se
afastam desse ambiente citadino. Como sempre, o caso brasileiro destoa
das definições mais ou menos fechadas dos modernismos europeus: aqui, as
propostas vanguardistas se combinam de forma vária com aquelas
manifestações. O extenso poema ``Carnaval Carioca'', do \emph{Clã do
Jabuti}, é bem útil para reconhecer esse amálgama;

\vspace{3mm}

8. O que interessa na atividade de leitura é, sobretudo, ler os poemas
\emph{com} os alunos, analisá-los com eles e sugerir interpretações
criativas, que serão exploradas e aprofundadas na atividade de
pós-leitura.

\subsubsection{Pós-leitura}


\paragraph{Temas} Reescrita dos poemas de Mário de Andrade em outras
linguagens artísticas
%BNCC-POS-LEITURA
\BNCCC{EM13LP50; EM13LP50; EM13LGG601; EM13LP53; EM13LGG604.} %Básicas

\paragraph{Conteúdo} Criação coletiva de uma peça de teatro, um sarau ou
uma batalha de poesia

\paragraph{Objetivos} 1. Produzir, revisar e reescrever textos de Mário de
Andrade e dos próprios alunos, de modo a aprofundar-se na obra poética
desse escritor e na escrita de autoria própria; 2. Produzir textos
orais, considerando sua adequação ao contexto em que serão enunciados.

\paragraph{Justificativa} Enquanto os exercícios de leitura, compreensão e
análise caracterizam-se, de forma geral, pelas primeiras aproximações do
texto, seguidas de atividades de descrição de suas características, as
práticas interpretativas abrem espaço para que os estudantes avaliem,
opinem e expressem juízos de valor quanto às leituras. Os professores
devemos estimular essa dinâmica, na exata medida em que ela serve ao
exercício do espírito crítico e abre canais para que possamos fazer
mediações entre a realidade dos textos analisados e a de nossos
estudantes.

\BNCC{EM13LP47}

Acreditamos que a criação de peças de teatro, saraus ou batalhas de
poesia permite a expressão desses juízos de valor de forma criativa,
para além da mera redação de textos argumentativos, que também é útil,
mas que não conta, como será o caso aqui, com a força da expressão
corporal e do trabalho coletivo.

\paragraph{Metodologia}

Primeiramente, peça aos estudantes que se dividam em grupos e que
escolham pelo menos um poema significativo da antologia, explicando os
critérios utilizados para selecioná-lo. Por que eles gostaram desse
poema? Qual a relação dele com a realidade atual? Cada grupo deve
apresentar essas justificativas bem argumentadas.

Na sequência, o grupo deve escolher uma forma de apresentação do poema:
peça de teatro, sarau de poesia ou batalha de poesia, chamada de
\emph{slam}. O mais importante é que o grupo se dedique à adaptação do
texto escolhido, recriando-o, atualizando-o, nos termos da forma de
apresentação que escolheu e do público que assistirá ao espetáculo. A
sala, como um todo, deve decidir a quem a apresentação será feita:
apenas aos colegas da turma, a outras salas, aos pais e/ou ao público em
geral.

A seguir, apresentamos aspectos que devem ser levados em consideração
pelos alunos, de acordo com a forma de apresentação que escolheram.

\subparagraph{\textsc{Peça de teatro}}

Para criar uma peça de teatro, o professor deverá dividir os
estudantes em grupos cujos objetivos correspondam às habilidades com as
quais se identifiquem e/ou a respeito das quais eles pretendam conhecer.
Sugerimos a seguinte divisão:

\begin{enumerate}
 
\item grupo dos alunos de redação da peça de teatro propriamente dita. O
primeiro desafio deste grupo é a redação de um enredo coerente a partir
dos poemas de Mário de Andrade. Evidentemente, a compreensão aprofundada
de um ou mais poemas do livro é pressuposto fundamental dessa atividade.
Note-se, ainda, que o processo de redação, aqui, é coletivo;

\item grupo dos alunos da \emph{produção} da peça: as equipes de
sonoplastia, cenário, figurino e contrarregra. Caso os alunos não tenham
conhecimento de cada uma dessas funções, o professor deve apresentá-las
de maneira geral e orientá-los a respeito de suas funções. Note-se aqui
que cada uma delas pode ser desdobrada em outras. Apenas a título de
exemplo, imagine-se que a equipe de figurino pode, antes de procurar as
roupas adequadas às personagens, pesquisar o vestuário típico da época
de Mário de Andrade e criar esboços, na forma de desenhos;

\item grupo dos alunos da \emph{direção} da peça: a equipe dos alunos que
concebem o espetáculo como um todo e têm capacidade de liderança
articulada e respeitosa. Outra possibilidade: o professor pode fazer o
papel de diretor da peça e contar com uma equipe de apoio;

\item grupo dos alunos \emph{atores}.

Evidentemente, a proposta apresentada acima pode ser adaptada a
quaisquer poemas da antologia, de acordo com a abordagem do professor.
\end{enumerate}

\subparagraph{\textsc{Sarau de poesia}}

Para criar um sarau de poesia, o professor deverá dividir os
estudantes em grupos cujos objetivos correspondam às habilidades com as
quais se identifiquem e/ou a respeito das quais eles pretendam conhecer.
Sugerimos a seguinte divisão:


\begin{enumerate}

\item grupo do cenário e da sonoplastia: quanto mais elaborado for o espaço
em que ocorre o sarau, mais impactante será a experiência de
apresentar-se nele e de assistir às declamações. Assim, sugerimos que
haja um grupo de alunos que se dedique à criação de um ambiente
adequadamente iluminado e sonorizado para o evento;

\item grupo de apresentação do evento: alguns alunos podem preferir
apresentar os colegas, em formatos diversos, que eles podem criar;

\item grupo de divulgação: caso os alunos e o professor decidam que as
apresentações serão abertas ao público, um grupo deve dedicar-se à
divulgação, especialmente pelas redes sociais. Reúna aqui os alunos que
têm talento em \emph{design} para que eles desenvolvam suas
potencialidades em \emph{banners} físicos e virtuais. Este grupo deverá
ser o responsável pela criação da identidade visual do evento;

\item grupo de transmissão: caso os alunos e o professor decidam que as
apresentações serão transmitidas e/ou gravadas, um grupo deve dedicar-se
à escolha das plataformas que serão utilizadas;

\item grupo dos alunos que vão se apresentar: trata-se dos alunos que vão
ler poemas de Mário de Andrade, ou seus próprios poemas, criados a
partir da leitura de nosso autor.

\end{enumerate}


\subparagraph{\textsc{Batalha de poesia}}

Para criar uma batalha de poesia, o professor deverá dividir os
estudantes em grupos cujos objetivos correspondam às habilidades com as
quais se identifiquem e/ou a respeito das quais eles pretendam conhecer.
Sugerimos a seguinte divisão:

\begin{enumerate}

\item grupo do cenário e da sonoplastia: quanto mais elaborado for o espaço
em que ocorre a batalha, mais impactante será a experiência de
apresentar-se nele e de assistir às declamações. Assim, sugerimos que
haja um grupo de alunos que se dedique à criação de um ambiente
adequadamente iluminado e sonorizado para o evento;

\item grupo de apresentação do evento: alguns alunos podem preferir
apresentar os colegas, em formatos diversos, que eles podem criar;

\item grupo de divulgação: caso os alunos e o professor decidam que as
apresentações serão abertas ao público, um grupo deve dedicar-se à
divulgação, especialmente pelas redes sociais. Reúna aqui os alunos que
têm talento em \emph{design} para que eles desenvolvam suas
potencialidades em \emph{banners} físicos e virtuais. Este grupo deverá
ser o responsável pela criação da identidade visual do evento;

\item grupo de transmissão: caso os alunos e o professor decidam que as
apresentações serão transmitidas e/ou gravadas, um grupo deve dedicar-se
à escolha das plataformas que serão utilizadas;

\item grupo dos alunos que vão se apresentar: trata-se dos alunos que vão
declamar poemas de Mário de Andrade, ou seus próprios poemas, criados a
partir da leitura de nosso autor.
\SideImage{Foto do autor em sua casa (Divulgação)}{PNLD0005-20.png}

\item grupo dos alunos que vão formular as regras da batalha: a batalha de
poesia se distingue do sarau pelo caráter competitivo do evento, por
isso é preciso criar um regulamento: Quanto tempo os participantes terão
para apresentar-se? Quais recursos de figurino são ou não permitidos? É
permitido usar acompanhamento musical? Aqui, sugerimos que os alunos
pesquisem as regras dos \emph{slams} mais populares e se baseiem nelas
para criar as suas próprias;

\item grupo de alunos jurados, que devem apresentar os critérios que
utilizaram para votar, a partir do regulamento criado pelo grupo
descrito no item anterior.



\end{enumerate}

\section{Proposta de Atividades 2}
%BNCC-GENERICO

\subsection{\emph{Carnaval Carioca} e as políticas culturais brasileiras}

\subsubsection{Pré-leitura}

\paragraph{Tema} Contexto histórico da produção poética de Mário de
Andrade

\paragraph{Conteúdo} A Modernidade e o Modernismo

\paragraph{Objetivo} 1. Analisar visões de mundo, conflitos de interesse e
ideologias presentes nos discursos históricos e ficcionais; 2. Analisar
semelhanças e diferenças entre esses discursos; 3. Analisar diferentes
graus de parcialidade e imparcialidade nesses discursos, levando em
consideração os recursos de linguagem utilizados para obter os efeitos
pretendidos.

\paragraph{Justificativa} A interdisciplinaridade entre Literatura e
História é especialmente interessante para a análise das primeiras
décadas do século \textsc{xx}. Trata-se de período em que a relação humana com as
categorias de tempo e espaço foi radicalmente alterada pela
generalização do trabalho assalariado industrial, nas nações centrais da
Europa e nos Estados Unidos, e pelas inovações tecnológicas, como a luz
elétrica e os meios de transporte. Era natural que a percepção dos
artistas fosse tocada pelas experiências associadas a essas mudanças.


\paragraph{Metodologia}

1. Para começar, sugerimos aos professores de História que promovam um
debate a respeito da Modernidade. Evidentemente, o ponto de partida deve
ser a Revolução Industrial e seus aspectos econômicos, sociais e
políticos. Sugerimos não se restringir aos elementos genéricos e
conceituais; reflitam também sobre a interferência de inovações como a
luz elétrica e os meios de transporte na vida das pessoas comuns. Como
era a vida \emph{antes} dessas inovações? Quais eram as restrições aos
ciclos de trabalho antes do desenvolvimento da luz artificial? Quanto
tempo levava para cruzar determinadas distâncias antes das linhas de
ferro?
\BNCC{EM13LP01} %Contexto (História)

\vspace{3mm}

2. Depois desse debate, sugerimos criar um grande painel de informações
sobre a sociedade, a economia e a política, utilizando a linguagem do
\emph{infográfico};
\BNCC{EM13LP54} % Mural


3. Depois da confecção do painel, divida os alunos em grupos. Cada um
deles escolherá um tema sugerido no painel para trabalhar. A seguir,
damos alguns exemplos de temas a discutir, aos quais você pode adicionar
outros que lhe pareçam pertinentes ou que surjam no debate em sala de
aula:

\begin{enumerate}

\item Aspectos sociais da Revolução Industrial, no mundo;

\item Aspectos econômicos da Revolução Industrial, no mundo;

\item Aspectos políticos da Revolução Industrial, no mundo;

\item Impactos sociais da Revolução Industrial europeia no Brasil;

\item Impactos econômicos da Revolução Industrial europeia no Brasil;

\item Impactos políticos da Revolução Industrial europeia no Brasil.

\end{enumerate} 
\BNCC{EM13CHS102}

Evidentemente, as sugestões acima são genéricas. Os grupos podem
escolher aspectos específicos para trabalhar, como a incorporação dos
ex-escravizados ao conjunto da sociedade brasileira e a República Velha
e suas conexões com a cultura do café. Quanto mais os temas escolhidos
envolverem nossos alunos, melhor para o trabalho, daí a importância do
painel inicial, que pode mostrar a eles a multiplicidade dos trabalhos e
discussões a desenvolver a respeito do período.

Outra sugestão é o desenvolvimento de análises a respeito da cultura do
período. Imagine-se, por exemplo, que uma pesquisa sobre as matrizes da
atual canção popular brasileira pode ser muito instigante. Sabemos que,
paralelamente à Semana de Arte Moderna, se consolidavam também grandes
nomes da nossa canção. A propósito das pesquisas de Mário de Andrade,
podemos sugerir aos alunos que investiguem as \emph{matrizes populares}
de nossa canção -- as Casas das Tias, no Rio de Janeiro, como a Casa da
Tia Ciata -- associadas à sua \emph{matriz industrial} -- os primórdios
da nossa indústria fonográfica, como a Casa Edison -- culminando no
mercado brasileiro de canção contemporânea. Uma informação importante:
somos um dos poucos mercados mundiais em que se consome mais a produção
local do que a norte-americana. O que essa informação diz sobre nossas
culturas?

Para apresentar os resultados, voltamos a sugerir formatos diversos:
\BNCC{EM13LP17}

\begin{enumerate}

\item podcast de até 15 minutos, com entrega de roteiro na forma de texto;

\item vídeo publicado no YouTube, no formato de documentário, em até 15
minutos, com entrega de roteiro na forma de texto;

\item peça teatral de até 15 minutos, com entrega de roteiro na forma de
texto;

\item leitura em voz alta de folheto de cordel, de até 15 minutos, com
entrega desse folheto ao professor;

\item apresentação formal tradicional, de até 15 minutos, com entrega de
relatório de pesquisa;

\item declamação de poema no formato de slam poesia, de até 15 minutos, com
entrega do poema escrito;

\item canção de até 15 minutos, que pode ser de composição original dos
estudantes, ou paródia de canção já existente, com entrega da letra e da
cifra.

\end{enumerate} 

4. É necessário destacar que cada uma das formas de apresentação dos
resultados requer dos estudantes a pesquisa sobre o gênero específico
dos textos;

\vspace{3mm}

5. Em qualquer um dos formatos de relatório, é fundamental que os
estudantes argumentem em defesa de um ponto de vista;

\vspace{3mm}

6. Depois das apresentações, os alunos deverão apresentar aos colegas as
dificuldades enfrentadas nos processos de pesquisa e preparação dos
relatórios específicos. Deverão também apresentar o percurso por meio do
qual chegaram às conclusões e pontos de vista propostos nas
apresentações de resultado.

\subsubsection{Leitura}

\paragraph{Tema} O carnaval brasileiro

\paragraph{Conteúdo} Análise do poema ``Carnaval Carioca'', que servirá de
ponto de partida para pesquisa sobre o carnaval como expressão das
culturas brasileiras

\paragraph{Objetivos} 1. Conhecer a história do carnaval, para além das
manifestações presentes dessa festa, no contexto brasileiro; 2.
Estabelecer relações entre a história do carnaval e suas manifestações
presentes, no Brasil; 3. Identificar diferenças entre as diferentes
expressões do carnaval; 4. Analisar as expressões contemporâneas do
carnaval; 5. Utilizar a linguagem artística, especialmente a expressão
corporal, em um processo de produção colaborativa para veicular
conteúdos sobre o carnaval e posicionar-se criticamente a respeito dele.

\paragraph{Justificativa} O carnaval é a festa popular mais conhecida do
Brasil. De forma geral, essa celebração é entendida como a mais perfeita
tradução de nossas culturas. Os modernistas, e Mário de Andrade em
especial, retrataram o carnaval em poemas, contos e romances, porque
propunham que as matrizes populares de nossa cultura não só se
manifestassem nas expressões chamadas de eruditas, como também as
\emph{enformassem}, em boa medida. Nossa proposta é, portanto, que
professores e alunos pesquisem o carnaval de modo a avaliar-lhe a
história e o alcance.
\BNCC{EM13LP01}

\paragraph{Metodologia}

\Image{Mário de Andrade, em pé e de chapéu e outros modernistas de 1922, em São Paulo, Brasil, 1922. (Autoria desconhecida/ ``A Imagem de Mário'', Editora Alumbramento.; Domínio Público)}{PNLD0005-07.png}

1. Primeiramente, sugerimos que os professores de História e Geografia
leiam com os alunos poemas de Mário de Andrade em que lugares
específicos sejam descritos. Para ilustrar nossa proposta, propomos aqui
a leitura e a compreensão do poema ``Carnaval Carioca'', do \emph{Clã do
Jabuti};

\vspace{3mm}

2. ``Carnaval Carioca'' é um poema extenso, bastante útil para debater a
grande festa, que talvez seja a grande expressão das culturas
brasileiras: o carnaval;

\vspace{3mm}

3. Depois da leitura e compreensão do poema, sugerimos ao professor de
História que proponha um debate guiado pelas seguintes perguntas:

\begin{enumerate}

\item Quais são as origens históricas do carnaval brasileiro?

\item Por que é que nosso carnaval acabou sendo tão diferente de outros
carnavais ao redor do mundo?

\item Quais são as características do carnaval carioca descritas no poema
de Mário de Andrade?

\item Quais são as características de outros carnavais conhecidos
nacionalmente?

\item Quais são as características do carnaval de sua cidade?

\item O carnaval mudou muito do tempo de Mário de Andrade para cá?

\end{enumerate} 

Essas perguntas devem ser o ponto de partida para que os alunos escolham
os temas pelos quais mais se interessam e se dividam em grupos, para
pesquisá-los e apresentá-los.

Novamente, apenas a título de sugestão, propomos alguns dos temas que
podem ser pesquisados:

\begin{enumerate}

\item História do carnaval na Europa e suas ligações com o carnaval
brasileiro;

\item O carnaval carioca, o mercado da ``maior festa do planeta'' e a
imagem do Brasil no exterior;

\item A História no Carnaval: os sambas-enredo e as interpretações da
História do Brasil;

\item As escolas de samba, os desfiles e o samba-enredo;

\item Grandes temas das marchinhas de carnaval;

\item O carnaval e o mercado de música no Brasil.

\end{enumerate} 


Mais uma vez, sugerimos que as pesquisas sejam apresentadas em diversos
formatos:

\BNCC{EM13LP17}

\begin{enumerate}

\item podcast de até 15 minutos, com entrega de roteiro na forma de texto;

\item vídeo publicado no YouTube, no formato de documentário, em até 15
minutos, com entrega de roteiro na forma de texto;

\item peça teatral de até 15 minutos, com entrega de roteiro na forma de
texto;

\item leitura em voz alta de folheto de cordel, de até 15 minutos, com
entrega desse folheto ao professor;

\item apresentação formal tradicional, de até 15 minutos, com entrega de
relatório de pesquisa;

\item declamação de poema no formato de slam poesia, de até 15 minutos, com
entrega do poema escrito;

\item canção de até 15 minutos, que pode ser de composição original dos
estudantes, ou paródia de canção já existente, com entrega da letra e da
cifra.

\end{enumerate} 


Talvez seja o caso, também, de propor aos estudantes que privilegiem
apresentações nas quais se destaque a \emph{expressão corporal}. Se o
carnaval é a festa do \emph{corpo}, sugira a eles que criem formas de
relatório que possam comunicar-se pela linguagem do corpo. Esse desafio
pode estimular a inventividade dos estudantes, especialmente aqueles que
preferem a linguagem corporal. Não se trata apenas de apresentar uma
forma de dançar, mas de transmitir os conteúdos por meio dela.

Evidentemente, nem todos os grupos e alunos se sentirão à vontade para
essa apresentação, mas podemos abrir essa possibilidade para aqueles que
se sentem aptos à expressão corporal, que é uma forma de comunicação.

\subsubsection{Pós-Leitura}


\paragraph{Tema} O direito à cultura

\paragraph{Conteúdo} Políticas culturais e formas institucionais e não
institucionais de engajamento na política da cultura

\paragraph{Objetivos} 1. Refletir sobre as diferentes formas de
engajamento na vida cultural; 2. Refletir sobre o direito à cultura; 3.
Analisar legislações, discursos políticos e formas não
institucionalizadas de participação social; 4. Engajar os estudantes na
busca de solução de problemas referentes ao direito à cultura.

\paragraph{Justificativa} É urgente o debate a respeito do direito à
cultura e das políticas a ele relacionadas. De um lado, as populações
mais carentes não têm acesso à cultura; de outro, populações
privilegiadas relegam bibliotecas e museus ao abandono, devido à
desvalorização desses espaços. O direito à cultura está previsto na
Constituição Federal de 1988 e deve ser estudado para que possa ser
garantido.
\BNCC{EM13LGG302}
\BNCC{EM13LGG303}

\paragraph{Metodologia}

1. \textit{Os artistas devem participar da administração pública do Brasil?} Essa
é a pergunta que deve servir de ponto de partida para um debate fértil e
necessário na sala de aula, especialmente agora, em que a educação e a
arte têm se tornado arenas de grandes discussões de natureza política;

\vspace{3mm}

2. A atuação de Mário de Andrade no serviço público serve de exemplo
para análise. Utilizando obras sugeridas na \textbf{Bibliografia
comentada}, explique o contexto histórico da atuação de Mário de Andrade
na vida pública, especialmente as iniciativas desse autor no cargo de
Diretor do Departamento de Cultura da Cidade de São Paulo, como a Missão
de Pesquisas Folclóricas e Discoteca Pública, e no papel de idealizador
do antigo SPHAN (Serviço do Patrimônio Histórico e Artístico Nacional),
atual IPHAN (Instituto do Patrimônio Histórico e Artístico Nacional).
Também vale lembrar o projeto da Enciclopédia Brasileira, quando Mário
estava ligado ao Instituto Nacional do Livro, como forma de
democratização da cultura e como elemento importante para o
desenvolvimento social e econômico;
\BNCC{EM13CHS104}

\vspace{3mm}

3. Depois de analisar o alcance e os limites dessas e outras
experiências que julgar adequadas (como a atuação do compositor Gilberto
Gil como Ministro da Cultura, de 2003 a 2008), proponha aos alunos as
seguintes perguntas:

\begin{enumerate}

\item Quais são as instituições públicas da área de cultura que eles usam,
ou que gostariam de usar, se estivessem disponíveis?

\item Quais são os museus e bibliotecas mais utilizados pelos alunos?

\item Quais são os equipamentos culturais que fazem falta na cidade de
vocês? Por quê? Quem são os responsáveis por essa falta? O que é
possível fazer para conquistar esses aparelhos?

\end{enumerate} 


Lembramos aqui que as habilidades de atuação na vida pública estão
previstas na Base Nacional Comum Curricular, no Campo de Atuação na Vida
Pública, descritos nos códigos de EM13LP22 a EM13LP26. Por isso, devemos
ensinar nossos estudantes a engajarem-se na participação social, aqui
proposta por meio da luta pelo direito à cultura.

Certamente, a juventude pode contribuir para a valorização dos espaços
culturais, conscientizando-se da importância deles e, sobretudo,
utilizando-os. Sugerimos, aqui, que os alunos façam uma pesquisa
profunda a respeito dos documentos legais que garantem o acesso à
cultura, começando pelo Artigo 215 da Constituição Federal de 1988, a
respeito dos direitos culturais, do acesso às fontes da cultura nacional
e da valorização e difusão das manifestações culturais. Também vale
pesquisar a legislação estadual e municipal da cultura.

\SideImage{Anotação de Mário de Andrade, 1926 (Mário de Andrade; Domínio Público)}{PNLD0005-05.png}

Além dessas pesquisas, sugira também que eles façam levantamento e
avaliação dos equipamentos culturais da cidade em que moram,
distinguindo os que pertencem à federação, ao estado e ao município. Se
concluírem que os equipamentos deixam a desejar, ou que faltam
equipamentos na região, sugira a eles que formulem possibilidades de
solução dos problemas identificados, como campanhas e textos
reivindicatórios.

\section{Aprofundamento}

No difícil e turbulento ano de 2020, foi lançado o filme \emph{Amarelo
-- É tudo pra ontem}, dirigido por Fred Ouro Preto. Esse documentário
foi veiculado pela maior plataforma de \emph{streaming} do mundo.
Trata-se, inicialmente, de um registro de uma apresentação do rapper
paulistano Emicida no Theatro Municipal de São Paulo, no ano anterior.

A escolha desse espaço, tradicionalmente associado à música erudita,
para servir de palco ao espetáculo de Emicida não foi casual. O rapper
estava decidido a ressignificar o Theatro Municipal, celebrando ali não
apenas o conjunto de sua própria obra e carreira, mas também o legado de
muitos artistas. Entre eles, Mário de Andrade é homenageado de forma
especial: é o ``nosso modernista preferido''.

Essa exaltação da obra de Mário de Andrade diz muito sobre o alcance do
escritor que estamos estudando aqui. Primeiramente, ele não se
restringiu à pesquisa e à análise das chamadas expressões eruditas e
escritas da cultura, típicas das classes dominantes. Ao contrário:
dedicou-se extensamente às manifestações populares, propondo que elas é
que constituíam a espinha dorsal das culturas brasileiras. Basta ler os
títulos das obras de Mário de Andrade para começar a entender seu
projeto: \emph{Ensaio sobre a Música Brasileira}, \emph{Música de
Feitiçaria no Brasil}, \emph{As Melodias do Boi e Outras Peças},
\emph{Modinhas Imperiais}, \emph{Os Cocos} e \emph{Danças Dramáticas do
Brasil} -- sobre nossa canção e nossas danças; \emph{Padre Jesuíno do
Monte Carmelo} e \emph{Aspectos das Artes Plásticas no Brasil} -- a
respeito das artes plásticas no Brasil; e \emph{Aspectos da Literatura
Brasileira} -- evidentemente sobre nossa literatura. Esses e outros
títulos ensinam que nosso autor pretendia fazer um inventário tão
completo quanto possível das nossas manifestações culturais. Não era
apenas o Brasil das elites que interessava a Mário de Andrade, mas o
Brasil como um todo.

Esse projeto ambicioso de pesquisa, atuação política e produção
artística foi reconhecido, quase cem anos depois, pelo rapper Emicida e
sua equipe. A música e a canção eram importantíssimas para Mário de
Andrade, pela sua formação e pela capacidade que essas expressões
artísticas têm de manifestar as identidades do Brasil. Da mesma forma, a
participação de Mário de Andrade na Semana de Arte Moderna de 1922 o
associa diretamente à cadeia de artistas que dedicam a vida e a obra à
demolição das barreiras do privilégio e da exclusividade de espaços
físicos e simbólicos. Dizendo de forma simples, há um ponto forte de
contato entre a apresentação de Emicida no Theatro Municipal e a atuação
de Mário de Andrade na Semana de 22: ambos propuseram a abertura daquele
espaço a manifestações artísticas não prestigiadas pelas elites.
Poderíamos dizer, ainda, que Mário de Andrade, se pudesse ter assistido
à apresentação de Emicida, vibraria de felicidade. O rapper tem plena
consciência do papel que cumpre na canção e na arte brasileiras
contemporâneas: ele é uma espécie de \emph{vértice}, para o qual
converge um conjunto de tradições e ancestralidades que, em sua maioria,
não guardam matrizes nos setores privilegiados de nossa sociedade, e
\emph{vórtice}, a partir do qual esse passado riquíssimo se renova e se
expande, atribuindo novos sentidos à experiência musical\footnote{Tomamos
  emprestada essa metáfora da obra como vértice e vórtice da
  pesquisadora Jerusa Pires Ferreira, especialista em literatura de
  cordel. Cf. FERREIRA, Jerusa Pires. \textbf{Cavalaria em Cordel: o
  passo das águas mortas}. São Paulo: Hucitec, 1993.}.

Mais uma vez, foi Mário de Andrade quem estabeleceu as bases para a
análise dessas relações entre tradição e contemporaneidade. Em uma
famosa conferência proferida em 1942, chamada de ``O Movimento
Modernista'', nosso autor teve a oportunidade de avaliar a Semana de 22
e a geração que dela participou, além do papel que cumpriu no
Modernismo. Para ele, esse movimento deixou fundamentalmente três
legados, três ``princípios fundamentais'': ``o direito permanente à
pesquisa estética'', ``a atualização da inteligência brasileira'' e ``a
estabilização de uma consciência criadora nacional''.

Sem a pretensão de analisar cada um desses princípios a fundo, propomos
aqui uma afirmação simples: Mário acreditava que o Modernismo de 22
chamara a atenção do artista brasileiro para a consciência que ele
deveria ter a respeito do seu papel e do diálogo que estabelecia com a
tradição e com o momento de sua produção. O ``direito à pesquisa''
pressupõe que o artista produz não apenas em termos intuitivos, mas que
faz parte de sua produção uma agência crítica no que se refere à
tradição de que participa; a ``atualização da inteligência artística''
assevera essa consciência crítica no que se refere ao presente local e
global; a ``estabilização da consciência criadora nacional'' se refere à
faculdade de criar sempre nesses termos, de modo a contribuir para a
\emph{formação} e para o \emph{delineamento} de um caráter específico,
de uma \emph{cara nossa} na arte, a partir da experiência coletiva e
individual de \emph{ser brasileiro}.

Emicida é herdeiro dessa tradição. No filme, ao mesmo tempo em que
apresenta o extenso conjunto de compositores e artistas que o
influenciaram e contribuíram em sua formação, o rapper explica seu
projeto artístico, em diálogo com a tradição e com o presente. Não é à
toa, portanto, que Mário de Andrade é chamado, no documentário, de
``nosso modernista preferido'': o legado construído por ele é retomado e
ressignificado por Emicida.

Esse exemplo interessante e construtivo dá a medida dos impactos
presentes que a obra de Mário de Andrade tem e ainda pode ter. O
Modernismo de 22 renovou a arte brasileira e abriu portas para diversas
inovações que viriam nos anos seguintes. Como já sabemos, Mário de
Andrade ocupa espaço de destaque nessa geração, seja pela liderança
intelectual, seja pela produção artística e ensaística. Desde jovem,
nosso autor era considerado mentor e líder de sua geração. O conjunto de
sua obra reflete uma personalidade radicalmente \emph{erudita}, que
pesquisou de maneira incansável as culturas brasileiras, \emph{em todas
as suas expressões}, como já observamos acima. Interessado nas artes em
geral, Mário escreveu sobre música, canção, dança, literatura, artes
plásticas e arquitetura, sempre no sentido de perscrutar nossas culturas
e valorizá-las.

As inúmeras cartas que escreveu para centenas de amigos e colegas de
todo o Brasil fazem parte desse conjunto. Trata-se de documentos em que
a afabilidade da escrita de Mário se combina com seu espírito
investigador e polemista. Ao mesmo tempo em que usa a correspondência
para as confissões pessoais e as descrições de miudezas cotidianas,
Mário chama os interlocutores ao debate que mais lhe interessava: o
Brasil e sua arte. É famoso e emblemático o chamamento que ele faz ao
jovem Carlos Drummond de Andrade: ``Nós temos que dar ao Brasil o que
ele não tem e que por isso até agora não viveu, nós temos que dar uma
alma ao Brasil e para isso todo sacrifício é grandioso, é sublime''. E a
seguir, na mesma carta: ``A língua que escrevo, as ilusões que prezo, os
modernismos que faço são pro Brasil''\footnote{ANDRADE, Carlos Drummond
  de. \textbf{Carlos e Mário: correspondência entre Carlos Drummond de
  Andrade e Mário de Andrade}. Lélia Coelho Frota (org.). Prefácio de
  Silviano Santiago. Rio de Janeiro: Bem-Te-Vi Produções Literárias,
  2002. p. 51.}. É esse o projeto de Mário de Andrade: dar ao Brasil uma
alma que ainda não fora organizada, devido ao desprestígio das
manifestações populares de nossa arte entre os especialistas -- a que
nosso autor se opôs, na prática, dedicando a elas uma vida de pesquisa e
difusão.

É também desse projeto que nasce a Missão de Pesquisas Folclóricas,
concebida por Mário de Andrade quando era Diretor do Departamento de
Cultura do Município de São Paulo, na gestão do Prefeito Fábio Prado.
Nessa Missão, Mário enviou pesquisadores para estados do Norte e do
Nordeste do Brasil para registrar danças, canções, roupas, festas e
costumes populares, em fotografia, áudio e vídeo. Essa iniciativa
ambiciosa demonstra que, embora preferisse a vida intelectual, Mário
aceitou a Diretoria do Departamento de Cultura para levar adiante o
ideal de preservar as diversas manifestações da cultura. Foi também
nessa gestão, por exemplo, que nosso autor criou a Discoteca Pública,
uma novidade para a época e que é referência de pesquisa de música e
canção até hoje, em que conta com mais de 70 mil discos, preservados
pelo Centro Cultural São Paulo (CCSP). Outra ideia de Mário, que acabou
não saindo do papel na época, mas que é muito utilizada hoje, é a das
bibliotecas populares ambulantes.

Uma observação mais ampla do período revela que Mário pertence a uma
geração de artistas que estavam comprometidos em mudar o Brasil por meio
da valorização de nossa diversidade cultural. Outros exemplos marcantes
são os de Carlos Drummond de Andrade, Heitor Villa-Lobos e Cândido
Portinari, mais ou menos próximos do Governo Vargas, que estimulava
iniciativas importantes, por meio do Instituto Nacional do Livro, o
braço cultural do Ministério da Educação e da Saúde (que daria lugar ao
Ministério da Educação e da Cultura, em 1953), capitaneado por Gustavo
Capanema. Para esses e muitos outros intelectuais, o momento era de
tomada de consciência do Brasil por meio da arte, valorizando as
culturas tradicionais e combinando-as com a modernização que se
avizinhava. É claro que podemos hoje, à distância de cerca de oitenta
anos, avaliar com mais isenção as ações do governo e dos artistas e
perceber diferentes interesses e suas contradições, mas essa análise não
invalida um sentido geral da época: o de conhecer e valorizar as nossas
culturas.

Talvez \emph{Macunaíma} seja a obra-prima de Mário de Andrade
precisamente por sintetizar o projeto grandioso de nosso autor. Em pouco
mais de cem páginas, uma miríade de narrativas, tradições, lendas,
canções, danças, brincadeiras e piadas de todos os cantos do Brasil
desfilam diante dos olhos do leitor, compondo uma espécie de
\emph{utopia literária}, numa linguagem vária e múltipla, de todos os
nossos povos e tempos.

Da mesma forma, o conjunto dos poemas desta antologia é uma amostra
desse projeto e dessa personalidade erudita e popular, tradicional e
moderna. A liberdade formal da obra poética de Mário de Andrade e de
seus parceiros modernistas abriu portas para todos os poetas que viriam
depois e desfrutariam dessa conquista. A expansão do eu na
irregularidade livre das formas se combina aos temas característicos do
Modernismo: a dinâmica da cidade e do trabalho, sua celeridade
vertiginosa, e a capacidade do eu lírico de flagrar os pequenos
episódios cotidianos que quase se apagam nessa paisagem fascinante e
brutal. A poesia de Mário de Andrade é uma janela para a São Paulo que
se modernizava e para o Brasil que se expandia e se descobria para além
do litoral e das capitais. Como em toda grande obra, a poesia de Mário
se equilibra entre esses dois polos: de um lado, a urbanidade de São
Paulo e Rio de Janeiro, cujas potencialidades nosso autor celebra; de
outro, a riqueza das matrizes sertanejas, cujas festas, danças e canções
preservavam um legado que Mário amava e valorizava.

Na obra poética de nosso escritor, esse par não é rigorosamente
opositivo, mas complementar, e se manifesta mais plenamente numa das
maiores conquistas de Mário: a linguagem, na qual erudição e oralidade
se combinam de maneira inovadora, por meio de neologismos e ritmos
surpreendentes, formulados a partir da fala e da musicalidade popular.
Insistimos aqui na ideia de que Mário de Andrade estaria fascinado com
os rumos da Literatura Brasileira Contemporânea: a forma de entoação do
rap, entre a declamação e o canto, por meio da qual os intérpretes
protestam; os saraus de poesia da periferia de São Paulo, que têm
formado inúmeros leitores e autores; as batalhas de \emph{slam} no
Brasil inteiro, que têm trazido a juventude de volta para a literatura.
Todas essas expressões literárias dialogam, em boa medida, com o projeto
estético-político de Mário de Andrade.

Finalmente: da mesma maneira que a obra de Emicida, a de Mário de
Andrade também é vértice para o qual convergem saberes, vozes e
tradições populares e orais de nossas culturas e vórtice pelo qual tudo
isso se ressignifica e compõe referência pra novas gerações. É por todos
esses motivos que vale a pena ler os poemas desta antologia.

\section{Sugestões de referências complementares}\label{sugestoes}

\begin{itemize}
\item \textbf{Missão de Pesquisas Folclóricas}

Os registros da Missão de Pesquisas Folclóricas estão disponíveis na
internet. Esse material pode ser ouvido em diversos canais do YouTube e
também pode ser adquirido em formato de CD nas lojas virtuais do SESC
São Paulo.

\SideImage{Mário de Andrade sentado, no chão, em frente à Anita Malfatti e Zina Aita (à esquerda de Anita), São Paulo, Brasil, 1922. (Prefeitura de São Paulo; Domínio Público)}{PNLD0005-04.png}

\item \textbf{Os artistas e o espaço da Semana de Arte Moderna}

Para conhecer a pintura de Anita Malfatti, Cândido Portinari, Di
Cavalcanti e Tarsila do Amaral, consulte os sites do \href{https://masp.org.br/}{Museu de Arte de
São Paulo}, da \href{http://pinacoteca.org.br/}{Pinacoteca do Estado de São
Paulo} e do \href{https://mnba.gov.br/}{Museu Nacional de Belas Artes do Rio de Janeiro}. 
Consulte também o \href{https://artsandculture.google.com/}{Google Arts \& Culture} e
procure pelos artistas e museus. O acervo é incrível e as possibilidades
de aproximação permitem a visualização de detalhes de muitas obras.

Para conhecer a música de Heitor Villa-Lobos, você pode começar sua
pesquisa pelo site do \href{https://museuvillalobos.museus.gov.br/}{Museu Villa-Lobos}, 
mas o mais interessante a fazer é escutar a obra desse grande compositor. Para começar,
sugerimos a versão do ``Trenzinho do Caipira'', que faz parte das
\emph{Bachianas Brasileiras}, numa \href{https://youtu.be/KTKVgaY56NI}{gravação emocionante da Orquestra
Sinfônica do Estado de São Paulo}, produzida durante a pandemia em 2020.

Conheça também o site do \href{https://theatromunicipal.org.br/}
{Theatro Municipal de São Paulo}. E, quando for possível, faça uma visita 
guiada ao espaço e assista a um espetáculo por lá.

\item \textbf{\emph{Amarelo -- É tudo pra ontem}}, dirigido por Fred Ouro Preto (2020)

O documentário contém a filmagem do espetáculo do rapper Emicida no
Theatro Municipal de São Paulo, mas é muito mais do que isso. No
conjunto do filme, Emicida apresenta as matrizes de nossa arte com as
quais suas composições dialogam e propõe sua ressignificação a partir
dos desafios do momento presente. Mário de Andrade tem lugar especial
nessa narrativa: é o modernista preferido de Emicida.
\end{itemize}


\section{Bibliografia comentada}

\begin{itemize}
\item  \textsc{alvarenga}, Oneyda. \emph{Mário de Andrade, um pouco}. Rio de Janeiro:
Livraria José Olympio Editora, 1974.

Oneyda Alvarenga foi aluna de Mário de Andrade, sua colega no
Departamento de Cultura do Município de São Paulo e sobretudo sua amiga
próxima. Neste livro, estão reunidos textos dela a respeito da obra e da
personalidade de nosso escritor.

\item \textsc{andrade}, Mário de. \emph{O Movimento Modernista}. In: \_\_\_\_\_\_.
\emph{Aspectos da Literatura Brasileira}. 6ª edição. Belo Horizonte:
Editora Itatiaia, 2002. p.253-280.

Essa conferência de Mário de Andrade foi proferida em 1942, em homenagem
aos vinte anos da Semana de Arte Moderna de 1922. Trata-se de uma
avaliação de nosso autor a respeito do evento do qual ele foi um dos
maiores protagonistas.

\item \textsc{andrade}, Carlos Drummond de. \emph{Carlos e Mário: correspondência
entre Carlos Drummond de Andrade e Mário de Andrade}. Lélia Coelho Frota
(Org.). Prefácio de Silviano Santiago. Rio de Janeiro: Bem-Te-Vi
Produções Literárias, 2002.

A correspondência entre Mário de Andrade e Carlos Drummond de Andrade é
do maior interesse para compreender o projeto literário de Mário de
Andrade.

\item \textsc{bomeny}, Helena. \emph{Um poeta na política: Mário de Andrade, paixão e
compromisso}. Rio de Janeiro: Casa da Palavra, 2012.

Obra de fácil leitura, sem perder a densidade, ideal para conhecer de
maneira geral a atuação de Mário de Andrade na vida pública, o contexto
em que ela ocorreu, suas tensões com os organismos estatais e as cartas
que ele trocou com o Ministro Gustavo Capanema.

\item \textsc{bosi}, Alfredo. \emph{Mário de Andrade}. In: \_\_\_\_\_\_. \emph{História
Concisa da Literatura Brasileira}. São Paulo: Cultrix, 2015.
p.370-379.

Apresentação geral da obra de Mário de Andrade por Alfredo Bosi, um dos
maiores críticos literários do Brasil.

\item \textsc{castro}, Moacir Werneck de. \emph{Mário de Andrade: exílio no Rio}.
Belo Horizonte: Autêntica, 2016.

Depoimento pessoal de Moacir Werneck de Castro sobre sua relação afetiva
e intelectual com Mário de Andrade durante o período em que este residiu
no Rio de Janeiro, entre 1938 e 1941. A amizade de Mário com os chamados
``Rapazes da \emph{Acadêmica}'', em referência à \emph{Revista
Acadêmica}, exerceu grande influência recíproca entre eles. Além de
Castro, o grupo era formado também por Murilo Miranda, Lúcio Rangel e
Carlos Lacerda. O livro traz ainda as cartas de Mário para o autor.

\item \textsc{lopez}, Telê Porto Ancona. \emph{Mário de Andrade: ramais e caminho}.
São Paulo: Duas Cidades, 1972.

Telê Porto Ancona Lopez é provavelmente a maior especialista na obra de
Mário de Andrade. Nessa obra de fôlego, ela analisa o sentido de
compromisso e a ligação da obra de nosso autor com a produção literária
popular.

\item \textsc{moraes}, Marcos Antonio de (org.). \emph{Correspondência de Mário de
Andrade e Manuel Bandeira.} São Paulo: Edusp, 2001.

Reunião das cartas trocadas entre os dois poetas do modernismo
brasileiro, de 1922 a 1945, revelando os bastidores da criação
literária, a amizade e os rumos do movimento modernista. Baseado em
exaustiva pesquisa documental, o livro traz ainda dossiê de fotografias
e fac-símiles de textos relacionados aos dois escritores.

\item \textsc{souza}, Cristiane Rodrigues de. \emph{Clã do Jabuti: uma
partitura de palavras}. São Paulo: Annablume, 2006.

Este livro contém a dissertação de mestrado defendida pela autora em
2004, na Unesp de Araraquara. A autora discute conceitos musicais na
escrita dos versos. Para ela, o projeto literário de Mário de Andrade
encontra na musicalidade sua síntese maior.

\item \textsc{souza}, Eneida Maria de; CARDOSO, Marília Rothier. \emph{Modernidade
toda prosa.} Rio de Janeiro: PUC-Rio; Casa da Palavra, 2014.

As autoras são duas das mais importantes pesquisadoras sobre o
modernismo brasileiro. O livro inova ao ampliar o conceito de prosa e
suas ressonâncias a partir do modernismo, abordando não apenas os
gêneros literários em prosa, como o romance e o conto, mas também
expressões como o cinema e as artes plásticas.

\item \textsc{souza}, Gilda de Mello e. \emph{O tupi e o alaúde: uma interpretação de
\emph{Macunaíma}}. São Paulo: Duas Cidades; Editora 34, 2003.

Apesar de conter uma análise de \emph{Macunaíma}, obra que não estudamos
neste manual, o livro de Gilda de Mello e Souza é útil para a análise
dos poemas de Mário de Andrade pelas hipóteses a respeito dos pontos de
contato da obra de nosso autor com as matrizes populares da cultura
brasileira.

\item \textsc{tércio}, Jason. \emph{Em busca da alma brasileira: biografia de Mário
de Andrade}. Rio de Janeiro: Estação Brasil, 2019.

Trata-se da mais recente biografia de Mário de Andrade, na qual o autor,
que é jornalista, realizou uma pesquisa exaustiva e meticulosa em
jornais, revistas, manuscritos e outros documentos mantidos nos arquivos
relacionados ao escritor paulista.

\end{itemize}

\end{document}

