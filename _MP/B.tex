\subsection{Pós"-leitura}

\paragraph{Tema} Ciências humanas -- geografia: a região de Samara e o rio Volga.

\BNCC{EM13CHS102}
\BNCC{EM13CHS206}
\BNCC{EM13CHS106}

\paragraph{Conteúdo}
Localização, pontos de interesse e importância da região de Samara e do
rio Volga para a Rússia.

\paragraph{Objetivos}
Familiarização do educando com o espaço geográfico em que ocorre a obra,
caracterizando e analisando sua importância histórico"-econômica.

\paragraph{Justificativa}
As relações do homem com a natureza têm enorme importância no livro
\emph{A infância de Nikita,} que destaca vários elementos, como o rio
Volga (Samara, onde fica a aldeia em que se passa o livro, encontra"-se na
região do Volga). Valiosíssimo para a Rússia, o Volga, com seus 3688~km,
é o mais longo rio da Europa e também o maior do
continente em caudal (volume/unidade de tempo) e na área de bacia
hidrográfica. O educando é capaz de apontar a localização da região de
Samara e do rio Volga no mapa? E ir mais além, discorrendo sobre a
história e importância econômica e cultural do rio para a Rússia?

\Image{``No Volga'', quadro de A. Arkhípov (1862-1883) (Wikipédia; Domínio público).}{PNLD0049-07}

\paragraph{Metodologia}
Leia com os educandos as páginas 92 a 95 da obra. Neste trecho, Nikita
desenha o mapa da América do Sul a pedido da mãe --- que queria que ele
deixasse de fazer ``figuras bobas'' --- e conduz ``o rio Amazonas para o
lado errado, passando pelo Paraguai e pelo Uruguai para desembocar na
ilha da Terra do Fogo''. (\textsc{tolstói}, 2021, p. 92) Acontece que o protagonista mora
próximo a um afluente, na região de Samara, do Volga, rio
importante para o país por suas dimensões e localização. Nikita,
como vemos desde o início da história, é um menino sonhador e desligado,
e não é dos estudantes mais aplicados, para dizer o mínimo. Aqui,
iniciaremos a atividade pedindo aos alunos que apontem no mapa da Rússia
esses dois pontos: o rio Volga e a região de Samara. É possível que não
consigam realizar a atividade de primeira. Pode"-se, então, aproveitar
para trabalhar com eles o mapa: tanto o russo --- apontando e deixando
que eles identifiquem a capital, Moscou, assim como a ex"-capital, São
Petersburgo, e outras cidades importantes, como Ekaterimburgo, ou os
montes Urais, que convencionalmente definem a fronteira entre Ásia e
Europa ---, como o europeu. Para complementar a atividade, sugerimos que
os educandos se dividam em grupos para pesquisar sobre Samara e o rio
Volga e discorrer sobre eles em uma aula posterior. Quais as atividades
desenvolvidas na região? Qual sua importância econômica?

\paragraph{Tempo estimado} Duas a três aulas de 50 minutos.