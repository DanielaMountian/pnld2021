\documentclass[12pt]{extarticle}
\usepackage{manualdoprofessor}
\usepackage{fichatecnica}
\usepackage{lipsum,media9,graficos}
\usepackage[justification=raggedright]{caption}
\usepackage{bncc}
\usepackage[lunna]{../edlab}


\begin{document}


\newcommand{\AutorLivro}{Akutagawa}
\newcommand{\TituloLivro}{Rashômone e outros contos}
\newcommand{\Tema}{Ficção, mistério e fantasia}
\newcommand{\Genero}{Conto, crônica e novela}
\newcommand{\imagemCapa}{./images/PNLD0009-01.png}
\newcommand{\issnppub}{---}
\newcommand{\issnepub}{---}
% \newcommand{\fichacatalografica}{PNLD0009-00.png}
\newcommand{\colaborador}{\textbf{Cesar Augusto Araujo Oyakawa, Bruno Gradella e Vicente Castro} é uma pessoa incrível e vai fazer um bom serviço.}


\title{\TituloLivro}
\author{\AutorLivro}
\def\authornotes{\colaborador}

\date{}
\maketitle

\baselineskip=1.20\baselineskip\par

\begin{abstract}

\textbf{Ryûnosuke Akutagawa} (Tóquio, 1892--\textit{id.} 1927) é o grande
  expoente do moderno conto japonês. Nasceu no bairro Kyôbashi, na ``cidade
  baixa'', filho de um pai extremamente rígido e de uma mãe louca, sob a égide
  de ``filho do Dragão''. Adotado pelos tios maternos, mais cultos, deixou de
  utilizar o sobrenome do pai, Niihara. Ainda criança, entrou em contato com
  traduções de Ibsen e Anatole France. Na primeira juventude, traduziu Yeats, e
  especializou"-se em Literatura Inglesa na Universidade Imperial de Tóquio,
  período em que se tornou discípulo do escritor japonês Sôseki Natsume
  (1867--1916) e passou a escrever os primeiros de seus cento e cinquenta
  textos curtos em prosa. Aos 26 anos, casou"-se com Fumiko Tsukamoto, com quem
  teve três filhos.  Na década de 1920, sua obra passa a revelar fortes traços
  autobiográficos: a loucura, o suicídio, a ética cristã, os antigos costumes
  japoneses e a modernização do período Meiji (1868--1912), num profundo
  conflito em busca de uma solução moral definitiva.  Suicidou"-se aos 35 anos
  com uma dose de Veronal. 
        
\textbf{Rashômon e outros contos} reúne dez contos de diversos períodos da
  breve existência do autor: \textit{Rashômon} (1915) e \textit{Dentro do
  bosque} (1922) retratam a cultura de Heian (atual Quioto). Em
  \textit{Memorando ``Ryôsai Ogata''} (1917), \textit{Ogin} (1923) e \textit{O
  mártir} (1918), a temática cristã é o fio condutor.  \textit{Devoção à
  literatura popular} (1917) e \textit{Terra morta} (1918) têm como pano de
  fundo a cultura de Edo (atual Tóquio). A abertura do Japão para o Ocidente no
  período Meiji compõe o enredo de \textit{O baile} (1912). Por fim, dois
  contos de caráter autobiográfico, do final da vida de Akutagawa:
  \textit{Passagens do caderno de notas de Yasukichi} (1923) e \textit{A vida
de um idiota} (1927).      \end{abstract}

\tableofcontents

\section{Introdução}

\textit{Rashômon e outros contos} reúne dez contos de diversos períodos da
breve existência do autor. Akutagawa, grande expoente do moderno conto japonês,
nasceu no bairro Kyobashi, na cidade baixa de Tóquio, filho de um pai rígido e
de uma mãe com problemas de saúde mental, sob a égide de “filho do dragão''.
Especializou-se em literatura inglesa na universidade imperial de Tóquio,
período em que se tornou discípulo do escritor japonês Natsume Sôseki e passou
a escrever os primeiros de seus cento e cinquenta textos curtos em prosa.
 
Na década de 1920, sua obra passa a revelar fortes traços autobiográficos: a
loucura, a morte, a ética cristã, os antigos costumes japoneses e a
modernização do período Meiji. 

“Rashômon” e “Dentro do bosque” retratam a cultura da atual Kioto. Em
“Memorando”, “Ryosai Ogata'' e “O mártir”, a temática cristã é o fio condutor.
“Devoção à literatura popular” e “Terra morta” têm como pano de fundo a cultura
da cidade de Edo (atual Tóquio). A abertura do japão para o ocidente no período
meiji compõe o enredo de “O baile”.
 
Por fim, dois contos de caráter autobiográfico, do final da vida de Akutagawa:
“Passagens do caderno de notas de yasukichi” e “A vida de um idiota”.

O leitor de hoje pode observar o extremo cuidado do autor com anacronismos
históricos. Sua pesquisa é minuciosa em termos de imagem, vestuário, objetos de
cena, cargos, fatos verídicos, mas principalmente para utilizar a língua de
acordo com cada gênero e época.
\Image{Ryūnosuke Akutagawa, depois de graduado em Literatura Inglesa, torna-se
discípulo do escritor japonês Natsume Soseki (Ogawa Kazumasa; Domínio
Público)}{PNLD0009-07.png} 

A feiura do egoísmo humano e o valor da arte enquanto redentora da miséria da
vida cotidiana são considerados temas recorrentes em seus escritos.  Essa
leitura nos ajuda a olhar e compreender os problemas da existência humana.
 
Akutagawa é considerado parte do grupo de intelectuais e estetas contrários ao
naturalismo, ao humanismo de cunho social de shirakaba, cujos membros em geral
provinham de elite econômica e à literatura proletária, que se manteve bem
restrita quanto à forma.
 
A obra de Akutagawa pode ser dividida em narrativas relacionadas à sua própria
vivência, ainda que sublimadas e elaboradas artisticamente, e narrativas
inspiradas na história literária e em personagens históricos.
 
Quanto ao gênero literário, a obra apresenta uma série de contos.
 
O conto é uma narrativa breve, com uma unidade formada por um só drama, um
conflito e que conduz diretamente ao final.
 
Enquanto movimento literário, o modernismo buscou romper ou reinterpretar com
as estéticas tradicionais, produzindo uma arte condizente com as realidades
impostas pelo cotidiano da virada do século XIX para o XX.  No caso japonês, a
resposta foi dada a elementos da cultura tradicional.


Apresentamos em seguida as atividades um e dois, as referências complementares
e a bibliografia propostas no manual do professor de rashomon e outros contos,
de akutagawa.

\section{Atividades 1}

A seguir apresentamos uma sequência didática de atividades que podem ser
apresentadas nessa em conjunto ou separadamente. 

\subsection{Pré"-leitura}

\bnccativividadespreleitura


\begin{enumerate} 
\item Em primeiro lugar, sugerimos tratar do gênero literário
do conto,  das narrativas breves, e da noção de períodos curtos e/ou
médios na literatura: conto, novela e romance.  Procure definir, bem como
discutir com os educandos a estrutura do conto que Akutagawa atribui: um
único drama, conflito e história; bem como preliminarmente observar a
familiaridade dos mesmos com a proposta literária. Peça para guardarem
bem essas categorias: (a) drama, (b) conflito, (c) história ou enredo.

\item Sugira uma breve pesquisa em casa ou durante uma aula em que os alunos se
dividam em grupos de até quatro pessoas. Peça para buscarem o que conseguirem
sobre textos japoneses, budistas e seculares. Qualquer coisa que estiver a
mão sobre a religião budista e o Japão já serão muito importantes para o
jovem leitor que pode desconhecer por completo a realidade de um país tão
distante. Uma busca na biblioteca ou uma consulta na internet com as
palavras chave ``meiji'' e ``taisho'', período no qual o Japão se abriu
completamente para a cultura ocidental, é fundamental. Induza então que
procurem no meio da sua pesquisa bastante genérica esses dois termos e peça
para pesquisarem algo sobre o Japão antigo e o contemporâneo. O objetivo é
fazer com que os alunos entendam que houve um período, na história recente
japonosa, de rompimento com a cultura tradicional. E a obra de Akutagawa é
fruto disso.  

\item Por fim, sugerimos apresentar o filme
\href{https://www.youtube.com/watch?v=xCZ9TguVOIA}{\textit{Rashômon}} (1950),
do grande cineasta japonês, Akira Kurosawa (1910--1998), ou algum material
sobre ele, tais como imagens, trailers publicitários,
comentários.\footnote{Caso não se tenha acesso a uma cópia do filme, não
desanime. Você pode achar alguns bons trailers sobre o filme na internet e
caso não tenha como passar em sala de aula, simplesmente peça para os
alunos assistirem em grupo no celular ou em casa.  Nós indicaremos algumas
dessas referências ao longo da atividade.} Este filme, baseado na obra de
Akutagawa, relada estupro de uma noiva e o assassinato de seu marido
samurai, do ponto de vista de quatro personagens: um bandido, a noiva, o
fantasma do samurai e um lenhador.  A história é baseada no conto ``Dentro do
bosque''.  

\end{enumerate}

\SideImage{Cartaz original do filme de Akira Kurosawa}{PNLD0009-20.png}
 

\SideImage{Retrato de Ryūnosuke Akutagawa (Yokohama,
CC-BY-SA-4.0)}{PNLD0009-03.png}

\subsection{Leitura}

\bnccativividadesleitura

\begin{enumerate} 

\item Comece lendo com os alunos em sala de aula mesmo o
conto ``Dentro do bosque''.  Procure realizar uma lista de personagens.
Tal qual um detetive, é interessante apontar e criar um perfil, contendo
o nome de cada personagem, bem como resumir o relato (a ``versão''),
trechos interessantes, estabelecer uma teia de relações, etc.  O conto
narra quatro depoimentos sobre um crime. Tanto em Rashômon quanto em
Dentro do bosque elabore um jogo de palavras com os quais vão ambientando
o espaço da narrativa. Por ser estático, o autor se vale de diferentes
termos para a descrição da paisagem e das personagens.  Aqui também seria
interessante a criação de imagens conforme as palavras se enquadram.

\Image{Samurai aprisionado por bandido. <<Rashômon>> (1050), de A.
Kurosawa.}{PNLD0009-21.png} 


\item Retome o filme. Lembramos que embora o filme de Kurosawa tenha o mesmo
nome do conto, o diretor japonês mesclou as narrativas de Akutagawa e
utilizou somente o cenário do conto ``Rashômon' e a trama do conto ``Dentro
do bosque''. A falta de correspondência direta com o livro pode ser uma
provocação importante, pois os alunos geralmente procuram a correspondência
ou a ilustração de uma história que leram ou conhecem em adaptações de
filmes. E provavelmente não entenderão a rearticulação das narrativas em
uma nova obra literária, como propõem Kurosawa. Isso certamente os
instigará a uma crítica sobre a falta de elementos ou confusões na
adaptação, após a atividade de leitura. Por fim, faça com eles uma
discussão detalhada sobre o que há de diferente entre o conto e o filme. 


\end{enumerate}
 
%\BNCC{EM13LGG103} \BNCC{EM13LP02} \BNCC{EM13LP48}



\subsection{Pós"-leitura}

\bnccativividadesposleitura

%\BNCC{EM13LGG102} \BNCC{EM13LGG303} \BNCC{EM13LGG402} \BNCC{EM13LGG703}
%\BNCC{EM13LP13} \BNCC{EM13LP14} \BNCC{EM13LP28} \BNCC{EM13LP29}
%\BNCC{EM13LP52}

Após a leitura é interessante pesquisar e discutir a articulação entre
literatura e cinema a partir dos dois contos de Akutagawa e o filme
''Rashomon'' (1950) de Akira Kurosawa. Promover o debate entre as narrativas
paralelas sobre um mesmo evento, bem como as contradições entre personagens.
Aqui cabe também informar um aspecto estético importante do Japão denominado
honkadori, isso é, fazer alusão a um texto original de autor reconhecido. É
possível estabelecer, então, uma relação dos contos budistas do séc.
\textsc{xii} com os contos e com o filme de Kurosawa.

\Image{Filme de Akira Kurosawa baseado em dois contos de Ryūnosuke Akutagawa:
"Rashomon" e "Dentro do bosque" (Daiei; Domínio Público)}{PNLD0009-06.png}

A fantasmagoria, os elementos do terror e do suspense, investigação policial,
está bastante presente nos dois contos. Também podemos notar no decorrer da
trama a distorção entre ficção e realidade. Elabore com os educandos uma
oficina, criando um acervo de informações (filmes e séries), que se assemelham
com os contos.

\Image{Noiva do Samurai em perigo. <<Rashômon>>
(1050), de A. Kurosawa.}{PNLD0009-22.png} 



\section{Atividades 2}
\bnccreferenciasgerais


A obra \emph{Rashômon e outros contos} possibilita trabalhos interdisciplinares
e integradores de diferentes campos do saber e áreas de conhecimento. A seguir,
propomos algumas atividades que podem ser desenvolvidas conjuntamente com
professores de outras áreas. Além das habilidades de Linguagens e suas
Tecnologias e de Língua Portuguesa, indicadas nas etapas da seção anterior e
válidas também para esta, listamos a seguir as habilidades de outras áreas,
presentes na abordagem interdisciplinar:

%\BNCC{EM13CNT201} \BNCC{EM13CNT303} \BNCC{EM13CHS101} \BNCC{EM13CHS102}
%\BNCC{EM13CHS106} \BNCC{EM13CHS401}


\subsection{Pré"-leitura}

Tanto em ``Rashômon'' quanto em ``Dentro do bosque'' elabore um jogo de palavras com os
quais vão ambientando o espaço da narrativa. Por ser estático, o autor se vale
de diferentes termos para a descrição da paisagem e das personagens. Aqui
também seria interessante a criação de imagens conforme as palavras se
enquadram.

Estas narrativas se enquadram no grupo de temas cristãos. Aborde o
contexto da Companhia de Jesus tanto no Japão como no restante do mundo,
incluindo no contexto brasileiro/português. É importante retratar o contexto
social e histórico do século \textsc{xvi}, como apresentado na seção dos
aspectos gerais da obra.

Uma forma interativa de abordar o assunto é trazer os elementos simbólicos que
marcaram o choque cultural religioso nas diferentes regiões do mundo. Seja
entre os indígenas, japoneses, com os europeus, estabeleça conexões e
intersecções entre os diferentes agentes.
\Image{Noiva ameaçada. <<Rashômon>> (1050), de A.
Kurosawa.}{PNLD0009-23.png}

Outros elementos interessantes são os conceitos de sincretismo religioso e
hibridismo religioso. A partir de diferentes localidades e tempos (tanto nos
passado como no presente), estabeleça diferenciações entre esses fenômenos
culturais. Selecione vídeos, imagens e símbolos, montando uma exposição,
galeria de arte sobre o assunto.

\subsection{Leitura}

É muito interessante no decorrer dos contos, formular e elencar as diferentes
palavras que marcaram o contato dos missionários cristãos com os japoneses.
Palavras como \textit{kirishitan} (``cristão''), \textit{bateren} (``padre''), 
\textit{kurusu} (``cruz''),
surgem no texto e pode despertar a curiosidade em pesquisar se outras sofreram
essa mesma influência.  Estimule os educandos para que busquem mais, e notem
inclusive na nossa língua portuguesa processos semelhantes de empréstimo
linguístico. Para além disso, note também com os educandos as tensões dos
valores confucionistas/\,budistas com os valores medievais/\,cristãos.

Principalmente nos contos Ogin e O Mártir, observe com os educandos a relação
com elementos cristãos como Deus, Maria, os santos, o diabo, em paralelo com os
elementos budistas como Amida, Zen, Shakyamuni, Buda.  Estabeleçam tópicos que
relacionem as ideias religiosas, as diferentes correntes e as personagens.


\subsection{Pós"-leitura}

Na pós"-leitura é interessante produzir com os educandos materiais lúdicos e
artísticos abordando o sincretismo/hibridismo religioso, bem como os jogos de
palavras entre os diferentes empréstimos linguísticos.

Uma outra possibilidade de reflexão é a relação do choque religioso
estabelecido no Japão com a Europa medieval, em relação a bruxaria e a
Inquisição. Com isso deve"-se pesquisar textos, iconografias e elaborar uma
rede de conexões e intersecções sobre o assunto.

Como exemplo, sugere"-se o anime, se refere à ``animação'' ou desenho animado
da pronúncia japonesa. Com o fim da Segunda Guerra Mundial, e a vinda da
cultura pop dos Estados Unidos, esses elementos contribuíram para um gênero
artístico único no Japão e atualmente é uma marca da cultura japonesa. Pesquise
juntamente com os educandos os diferentes elementos religiosos presentes nos
animes e crie uma galeria de arte sobre o assunto do universo religioso
presente neste estilo.

\Image{"A viagem de Chihiro" (2001), Studio Ghibli.  (Srchaos; CC BY-SA
4.0)}{PNLD0009-10.png}

\section{Aprofundamento}


Nesta seção, desenvolvemos um trabalho de aprofundamento que, em diálogo com a
formação continuada de professores, oferece subsídios para a abordagem do texto
literário.

\subsection{A Obra}

A obra \emph{Rashômon e outros contos}, de Ryunosuke Akutagawa (1892-1927) nos
convida a este processo, apresentando elementos únicos da literatura japonesa,
retratando a vida cotidiana de diferentes momentos da sua História, bem como
seus costumes e a cultura, além dos elementos narrativos intimistas e
particulares do autor. Partindo de uma perspectiva mais global, podemos
identificar três distintos períodos: o Heian (794-1185), Kamakura"-Muromachi
(1185-1573), e o período ``contemporâneo'' que compreende Edo (1603-1868),
Meiji (1868-1912) e Taishô (1912-1926). Feita esta linha do tempo, a seguir
elencaremos os aspectos principais destes períodos para que nos familiarizemos
com os contextos dos quais Akutagawa se apropria. Vale lembrar que não
necessariamente a ordem dos contos corresponde com a publicação dos mesmos,
apenas para fins didáticos assim estabeleceremos.

\Image{Morador de vilarejo
relembrando a trágica história. <<Rashômon>> (1050), de A.
Kurosawa.}{PNLD0009-24.png}

% Imagem muito estranha!
%\Image{Painel da Era Meiji  (Khalili Collections; CC BY-SA
%4.0)}{PNLD0009-05.png}

\subsection{Um mergulho no universo histórico e cultural do Japão}

Nos contos \emph{Rashômon} e \emph{Dentro do bosque}, encontra"-se o primeiro
recorte histórico, marcado pelo fim do ``período clássico'' da História
japonesa, onde ocorre uma intensa transformação na cultura e nos costumes
japoneses. Cabe aqui compreender a estrutura social japonesa até então: havia a
figura do imperador, tido como figura simbólica do Estado e ao mesmo tempo
descendente de Amaterasu (a deusa do sol na religião xintoísta); seguida pelos
\emph{xoguns}, o chefe supremo militar que exercia o poder de governo concedido
pelo próprio imperador; os \emph{daimiôs}, que exerciam propriedades de terras
hereditárias (apropriando um conceito ocidental seriam semelhantes aos senhores
feudais da Europa); os \emph{samurais}, servidores com funções militares e
administrativas dos daimiôs e que durante este período conseguiram bastante
ascensão social; além da nobreza que compunha a corte de cada um deles e o
próprio campesinato, como na Idade Média ocidental.

As transformações do período Heian se deram pelo ápice do intercâmbio cultural
com a China, no qual se destacam alguns pensamentos filosófico"-religiosos: o
budismo, confucionismo e o taoísmo. Muito mais do que meramente religiões, elas
influenciaram os diferentes estamentos sociais japoneses e o próprio
funcionamento do Estado, além de mudanças nos seus costumes e comportamentos.
Por exemplo, o ordenamento da burocracia e o respeito aos ritos
político"-sociais (confucionismo); a questão contemplativa diante ao espaço do
campo e das vilas, reverência ao imperador e a nação (budismo e taoísmo).

A seguir, encontramos três contos com temáticas voltadas ao sincretismo
religioso: \emph{Memorando ``Ryôsai Ogata''}, \emph{Ogin} e \emph{O mártir.} O
século \textsc{xvi} do Japão é identificado pela descentralização e pela
instabilidade política, porém de uma intensa afluência e intercâmbio cultural e
econômico. Conheceram os mosquetes e os monges. O primeiro no quesito da
tecnologia, a produção de armas de fogo e o intenso comércio com os continentes
asiático e europeu; os segundos, na questão religiosa, já estavam presentes as
chinesas (citadas anteriormente) e agora um adicional: o cristianismo. Assim
como nas Américas, os religiosos da Companhia de Jesus aportaram em terras
nipônicas em prol da evangelização cristã, sobretudo na região de Nagasaki
(ilha de Kyushu), local onde se passam as três narrativas de Akutagawa. No ano
de 1587, com a ascensão de Toyotomi Hideyoshi, intensificou"-se a perseguição a
missionários cristãos decretando a proibição de seu culto.


\Image{Mulher japonesa do período Meiji montada em um cavalo de carga,
1892-1895 (T. Enami ; Domínio Público)}{PNLD0009-09.png}
A segunda metade da obra, que vai de \emph{Terra morta} a' \emph{Vida de um
idiota}, retratam temas contemporâneos, bem como aspetos que influenciaram a
vida do autor e sua forma de escrita. Do período Edo (1603-1868), encontra"-se
\emph{Terra morta} e \emph{Devoção à literatura popular}, onde temos como plano
de fundo a mudança da antiga capital Quioto para Edo (atual Tóquio), não de
forma oficial, mas pela questão política e econômica com a qual ela ascendeu. O
período é marcado pela intensa transformação interna do Japão, muito por conta
do isolamento com o qual o clã Tokugawa impôs ao território nacional, bem como
a intensa urbanização e florescimento de uma arte e cultura únicas (uma espécie
de ``iluminismo'' japonês) que reuniu elementos estéticos xintoístas,
neoconfucianas e budistas. Destaque para dois personagens retratados nos dois
contos, respectivamente: Bashô Matsuo (1644-1694) e Bakin Takizawa (1767-1848).

O período Meiji (1868-1912) põe fim a um período de isolamento ao mundo
exterior, e inúmeras foram as causas que ocasionaram esse movimento.
Destacam"-se a queda do clã Tokugawa, as pressões que o Ocidente impunha e as
pressões internas -- tanto políticas e econômicas, visando a ``modernização''.
Com a abolição do até então ``sistema feudal'' japonês, o Estado"-nação
centralizou"-se na figura do Imperador e a economia industrial capitalista
revolucionou todo o território. Em termos culturais, quase que como um
hibridismo, houve uma fusão e síntese dos elementos tradicionais japoneses e a
modernidade global do século \textsc{xix}. Nesse plano de fundo, insere"-se o
conto \emph{O baile}.

Por fim, os contos \emph{Passagens do caderno de notas de Yasukichi} e \emph{A
vida de um idiota,} temos um quadro mais ``contemporâneo'', não necessariamente
definidos em um período histórico, dado seu caráter mais intimista e de
devaneios que marcou o fim da vida de Akutagawa, culminado com seu suicídio em
1927. É possível estabelecer uma relação dialógica entre autor e escritor, as
influências ocidentais e locais que marcaram seu estilo e arte. Encerram"-se o
livro com estes dois, quase que como ``testamento'' ou ``carta suicida''
(principalmente o último) que demonstram a sua complexidade bem como seus
dramas pessoais, assim como a difícil tarefa de enquadrá"-lo nos movimentos
estéticos"-literários: Romântico na essência, Modernista no seu tempo, com
sopros de Realismo, Naturalismo e Humanismo.

\Image{"Leg of the horse , manuscrito do autor, 1925, Museu Edo-Tokyo – Sumida,
Tóquio, Japão. (Edo-Tokyo Museum; Domínio Público)}{PNLD0009-08.png}

\subsection{Por que ler \textit{Rashômon e outros contos}?}

A obra Rashomôn e outros contos nos permite adentrar no universo e da estrutura
do conto como gênero literário, do qual Akutagawa é um expoente no seu período
e do movimento literário o qual está inserido, mesmo diante uma vida e carreira
curtas (produziu pouco mais de 150).  Nos 10 presentes no livro, publicados em
2013, apresentam elementos íntimos e universais, orientais e ocidentais. Tais
características marcam o modernismo do qual faz parte, que rompeu barreiras e
reinterpretações estéticas que, até então, dois séculos antes no seu país, era
proibido alcançar. Mais do que contos, também são verdadeiros documentos
históricos que retratam diferentes períodos históricos bem como seus
personagens da cultura tradicional japonesa.



Os contos exigem uma leitura ativa, uma vez que nos deparamos com costumes,
tradições, personagens tanto comuns socialmente quanto reais, que devemos nos
desfazer das amarras ocidentais e entrar no universo japonês, que amplia a
nossa compreensão de mundo. Mesmo sendo narrativas curtas ou médias, o grau de
profundidade e adensamento é inesgotável.  Suas releituras históricas,
devaneios, nos permite também conhecer aspectos universais e atemporais.

Sua obra é de difícil enquadramento estético"-literário, pois além de elementos
modernistas, claramente notados em Passagens do caderno e notas de Yasukichi e
O baile, também podem ser notados aspectos ultra"-românticos com em A vida de
um idiota, através dos seus devaneios, morbidez, também presentes em Rashomôn e
Dentro do bosque. Porém estes dois últimos apresentam muitos elementos
humanistas e naturalistas.  Temas que escapam da estética comum com seus
``temas cristãos'' como Ogin, O mártir e Memorando ``Ryôsai Ogata''. Os contos
Terra morta e Devoção a literatura popular, que carregam elementos históricos,
metalinguísticos e com sutilezas deterministas.

Mas tudo isso é apenas um exemplo, uma forma de interpretação da obra. É
possível estabelecer outras características e elementos de sua narrativa,
podendo assim estimular uma leitura crítica e buscar sempre novas recriações,
intersecções, debates e ferramentas para estudo. Em tempos de descolonização,
concomitante à busca de um universal mesmo que distante geograficamente, a obra
de Akutagawa permite ser uma excelente fonte para metodologias ativas,
itinerários de aprendizagem e uma saída curricular interessante e atraente
tanto para professor quanto para o educando.

\section{Sugestões de atividades complementares: relações dialógicas e
intertextuais}

%\BNCC{EM13LP03} \BNCC{EM13LP04} \BNCC{EM13LP49} \BNCC{EM13LP51}

No Ensino Médio, da mesma forma que no Ensino Fundamental, a \textsc{bncc}
organiza o trabalho com as práticas de linguagem em cinco \textbf{campos de
atuação social}. São eles: campo da vida pessoal, campo da vida pública, campo
jornalístico"-midiático, campo artístico"-literário e campo das práticas de
estudo e pesquisa.

De acordo com essa divisão, propomos na sequência um trabalho interdiscursivo e
intertextual com a obra \emph{Rashomôn e outros contos}

\subsection{Campo da vida pessoal}

Utilize o recurso de ``ilhas do conhecimento'', aonde o educando percorra nas
diferentes escolas literárias com citações, poesias, trechos de outros autores.
Sem que eles saibam, preenchem uma ficha constando as características que
justifiquem a escolha. Promova um debate uma vez concluído o circuito, e
compare respostas e as justificativas elencadas.

Uma vez concluída a leitura dos contos, faça com que os educandos enquadrem os
elementos específicos de Ryunosuke Akutagawa com os diferentes movimentos
estéticos"-literários abordados na pré"-leitura, numa espécie de
``quebra"-cabeças'', ou a criação de um quiz.

Por fim, identifique com os educandos os aspectos da vida pessoal de Akutagawa
presentes no conto A vida de um idiota. Discuta e estabeleça uma lógica entre
biografia e delírio do autor.

\subsection{Campo de atuação na vida pública}

\begin{quote} No cerne do campo de atuação na vida pública estão a ampliação da
  participação em diferentes instâncias da vida pública, a defesa dos direitos,
  o domínio básico de textos legais e a discussão e o debate de ideias,
  propostas e projetos. {[}\ldots{}{]}

Ainda no domínio das ênfases, indica"-se um conjunto de habilidades que se
relacionam com a análise, discussão, elaboração e desenvolvimento de propostas
de ação e de projetos culturais e de intervenção social.  (\textsc{bncc}, p.
494) \end{quote}

Uma das questões que exigem muita atenção sobre a obra e o livro em si é o ano
de publicação dos contos. Por mais que eles se insiram nas décadas de 10 e 20,
assim como visto anteriormente, não é uma tarefa fácil inserir em um único
movimento artístico"-estético a seleção dos contos: Romantismo?
Naturalismo/Realismo? Modernismo? Na realidade, são todos. O que é interessante
é pontuar, por exemplo:

\begin{enumerate} \item as características; \item suas influências; \item os
recursos estilísticos adotados; \end{enumerate}

Os contos elencados neste conjunto de atividades servem de um bom itinerário
para este propósito, de como uma obra é ``aberta'' e ampla de sentidos e
significados. Primeiramente, vale elencar os elementos essenciais das
diferentes escolas literárias previamente. O uso de quadros simbólicos,
post"-its, mosaicos, são interessantes para tornar a atividade lúdica e
interativa.

É importante neste momento inserir a biografia de Akutagawa, seus elementos
pessoais e de formação são essenciais para a elaboração de um quadro histórico
e estilístico. Por mais que se enquadre no modernismo, Ryunosuke absorveu
textos religiosos, românticos e naturalistas/realistas e compôs elementos
únicos em seus contos. Cabe aqui elencar autores nacionais, da língua
portuguesa nesse rol de autores, como Machado de Assis, Castro Alves, Guimarães
Rosa, Fernando Pessoa, e tantos outros autores. É interessante elencar esse
quadro juntamente com os educandos.

\Image{Foto em grupo com Kan Kikuchi, Ryūnosuke Akutagawa, Mutō Chōzō, Nagami
Tokutarō (da esquerda para direita), 1919. (Sem Autoria; Domínio
Público)}{PNLD0009-04.png}

\subsection{Campo jornalístico"-midiático}


Em Rashomôn, é possível estabelecer relações com outras obras como Os Sertões
(1902) de Euclides da Cunha, em torno do determinismo, no qual mostra a
influência do meio sobre as ações humanas. Cabe aqui trazer as diferenciações,
uma vez que o conto denota também um caráter místico em torno do portal
Rashomôn: a separação entre o sagrado e o profano.

\subsection{Campo artístico"-literário}

Num primeiro momento é possível que tanto professor quanto educando estranhem
as referências de personagens reais da arte e literatura tanto japonesa quanto
ocidental. É interessante montar uma galeria deles, criando sinopses que os
identifiquem. Aproveite também para montar um glossário, com os elementos
artísticos e culturais específicos do Japão, que também estã--1912) a Taishô
(1912-1926). Atentar para o cuidado do autor em não cometer anacronismos e
exatidão de costumes e comportamentos da época, marcado pela ocidentalização e
ao mesmo tempo ``marcar'' os elementos identitários da cultura japonesa,
destaque principalmente para os contos Devoção a literatura popular e o O
baile. É interessante elencar estes elementos comparando Ocidente e Oriente.

\subsection{Campo das práticas de estudo e pesquisa}


Faça uma breve pesquisa sobre o período Heian (794--1185). Aqui é interessante
situar as características tanto específicas do Japão quanto globalmente.
Estabelecer relações com a Idade Média, por exemplo, criando um quadro
comparativo Ocidente e Oriente.

O objetivo do quadro é poder ampliar o horizonte histórico, pois se trata de
uma cultura muito distinta, que ultrapassa o eixo ocidental, ao mesmo tempo em
que se revisa temas centrais da História. É interessante elaborá"-lo de forma
colaborativa, através de tópicos, esquemas e quadros sinóticos.

Aqui também cabe uma breve pesquisa sobre o Budismo, Confucionismo e o
Xintoísmo. Religiões e tradições filosóficas que podem ser de caráter
interdisciplinar, pois abarca Ciência da religião, Antropologia e Filosofia.

Nestes dois contos, Akutagawa baseou"-se na obra Konjaku monogatarishu
(coletânea de narrativas de ontem e de hoje, do século \textsc{xii}),
narrativas budistas e seculares que podem ser usadas como fonte de busca de
informações acerca dos contos.

\section{Referências complementares}

\subsection{Livros}

\begin{itemize} \item\textsc{dazai}, Osamu. \textit{Declínio de um Homem}. São
      Paulo: Estação Liberdade, 2015.

O livro conta a história de Yozo, um jovem provinciano que tenta sobreviver em
    Tóquio, enfrentando diversas hostilidades. O protagonista é um alter ego do
    autor e sintetiza muitas de suas angústias.

\item\textsc{soseki}, Natsume. \textit{O portal}. São Paulo: Estação Liberdade,
  2014.

Publicado em 1910, nos anos finais da Era Meiji, o livro gira em torno do casal
    de meia"-idade Sosuke e Oyone, que mantêm um casamento afetuoso, mas sem a
    paixão da juventude. A obra fecha uma trilogia composta pelos romances
    Sanshiro e E depois.

\item\textsc{wakisaka}, Geny. \textit{Contos da Era Meiji}. São Paulo:
  \textsc{cjusp}, 1993.

Este compilado de histórias da Era Meiji reflete o encontro do moderno e
inovador com o antigo e o tradicional, característico dos anos em que o Japão
se abriu para o Ocidente, na virada do século \textsc{xix} para o \textsc{xx}.
\end{itemize}

\subsection{Filme}

\begin{itemize} 

\item\textbf{Rashomon}. Direção: Akira Kurosawa (Japão, 1951).

Dentro de uma estrutura de flashback, é mostrado o julgamento de Tajomaru,
homem acusado de estuprar Masako e de assassinar seu marido, Takehiro, um
samurai. A história se passa no Japão do século \textsc{xi}. 

Veja o trailer do filme no youtube. 
\href{https://www.youtube.com/watch?v=WKd4tIHB064}{Rashomon, de Akira Kurosawa}  

\item\textbf{Robert Altman fala sobre <<Rashomon>>} 
Assista também o belíssimo relato do diretor e roterista Robert Altman 
sobre o impacto e a influência do filme de Kurosawa sobre ele. 
O vídeo está disponínvel no youtube. 
\href{https://www.youtube.com/watch?v=oYWQa0GExt8}{Robert Altman on 
<<Rashomon>> by Kurosawa.} Lembramos que pode acrescentar legendas em 
português.\footnote{Para isso acesse o vídeo que você quer assistir.
O ícone (CC) será exibido no canto inferior direito do player de vídeo.
Para ativar as legendas, clique em (CC) e escolha a língua.}

\item\textbf{Como Akira Kurosawa filmou <<Rashômon>>} Assista 
o detalhado documentário
sobre as filmagens e escolhas do diretor japonês, também 
disponível no youtube. 
\href{https://www.youtube.com/watch?v=-tkDU0_r8dU}{How Akira Kurosawa Framed Rashômon}
Nas palavras do documentarista ``esteja Kurosawa 
filmando seus personagens para parecerem primitivos, 
ou simplesmente desobedecendo a regra para efeito adicional, 
a visão e direção magistral de Akira Kurosawa é o que torna <<Rashômon>> 
o filme perfeito. Embora o assunto seja intrigante, ele desmoronaria 
sem os vários estilos de enquadramento que Kurosawa emprega ao longo 
do filme. Neste ensaio em vídeo, veja como e por que ele filmou 
as cenas daquela maneira.''

\end{itemize}

\section{Bibliografia comentada}

\begin{itemize} \item\textsc{blitzer}, Charles. \textit{Japão Antigo}. Rio de
      Janeiro: José Olympio, 1973. (Coleção Biblioteca de História Universal
      Life.)

  A obra constrói um panorama histórico da cultura japonesa e permite
    compreender valores e tradições presentes no cotidiano e nas artes do país.

\item\textsc{ortiz}, Renato. \textit{O próximo e o distante: Japão e
  modernidade"-mundo}. São Paulo: Brasiliense, 2000.

  O autor traça coordenadas espaço"-temporais que compõem um retrato singular
    do Japão, tanto em termos de identidade nacional quanto de relações
    internacionais.

\item\textsc{said}, Edward. \textit{Orientalismo: o oriente como invenção do
  ocidente}. São Paulo: Companhia de Bolso, 2008.

  O autor, na perspectiva dos estudos culturais, revela as concepções sociais e
    políticas por trás da imagem de Oriente construída por autores ocidentais.

\item\textsc{tanizaki}, Junichiro. \textit{Em louvor da sombra}. São Paulo:
  Penguin/ Companhia das Letras, 2002.

  O ensaio sobre estética japonesa trata de arquitetura, teatro, gastronomia e
    moda para abordar o contraste entre sombra e claridade na cultura japonesa.

\item\textsc{todorov}, Tzvetan. \textit{Introdução à literatura fantástica}.
  São Paulo: Perspectiva, 1975.

  O autor analisa as atrações insólitas exercidas pelo gênero fantástico na
prosa de ficção, a partir da articulação entre o estranho, o sobrenatural e o
maravilhoso.  \end{itemize}


\end{document}

