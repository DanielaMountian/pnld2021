olá,
é com muita alegria que apresentamos o
livro: 
"tratado descritivo do brasil”, de gabriel soares de souza.

o tratado descritivo do brasil em 1587 está dividido em dois livros: roteiro geral com largas informações de toda a costa do brasil e memorial e declaração das grandezas da bahia de todos os santos, de sua fertilidade e das notáveis partes que tem. o primeiro livro contém um proêmio e 74 capítulos. a segunda parte, por sua vez, abrange 196 capítulos. tanto a edição de 1851 quanto a de 1879 são finalizadas com 270 comentários e observações de francisco adolpho de varnhagen:

o termo “roteiro” refere-se a um gênero recorrente, sobretudo na época dos descobrimentos, utilizado para descrever em detalhes uma viagem ou estabelecer uma rota ou guia para os navegantes, apontando precisamente os cabos, baixios, ilhas, portos ou rios navegáveis; enfim, tudo o que poderia servir para orientar um navegante conquistador. o memorial da bahia faz parte de um gênero também corrente à época, o registro de lembranças.

as informações prestadas por gabriel em seus livros empregam diversas áreas do conhecimento, tais como a náutica, a bot nica, a zoologia e a memorialística, narrando as vivências, memórias e impressões do autor sobre os lugares, animais, plantas e gentes do brasil. gabriel tinha o poder da escrita, capaz de agradar qualquer tipo de público leitor.

“sem dúvida alguma, é possível reconhecer atrativos literários em seu discurso: uma capacidade descritiva e persuasiva capaz de conquistar leitores especialistas ou não” (azevedo, 2015, p 28).

a pesquisadora afirma ainda que gabriel, mesmo não tendo conhecido pessoalmente todos os cantos da costa brasileira, não deixou de descrevê-los em seu tratado, muito provavelmente recorrendo a outras fontes para isso:

por certo percorreu toda a circunvizinhança da cidade de salvador, por terra e por mar. mas, provavelmente, bebeu em outras fontes para se referir com tanta minudência ao restante do território.

boa parte do que soares escreveu deve ter sido compilado de informações encontradas na própria península durante o longo período em que lá esteve possivelmente recebendo letras remetidas de suas fazendas. ao que tudo indica não percorreu toda costa brasileira, do “rio de vicente pinzon”, acima do “rio das amazonas”, até algumas léguas depois da baía de são matias, na atual argentina

na primeira parte do livro, o roteiro geral, com largas informações de toda a costa do brasil, sousa descreve, de maneira bem detalhada, tudo o que se poderia encontrar na região costeira do brasil, do rio amazonas, passando pelo rio são francisco e a baía do rio de janeiro, até o rio da prata.

conta quem foram os primeiros povoadores das capitanias e os heróis colonizadores, fala quais são as atividades e potencialidades econômicas e como é a navegabilidade de cada região costeira, onde é possível adentrar com caravelões, como são os portos, o relevo, o clima, a vegetação, os frutos, a fauna, os rios, a fertilidade do solo, a localização e condições dos povoados, engenhos, plantações e áreas com pau-brasil, onde é possível haver fluxo de pessoas e de mercadorias e, por fim, onde é possível encontrar os tão cobiçados metais e pedras preciosas

para ganhar a confiança do rei espanhol e mostrar credibilidade, o colono português fornece todo o tipo de informações estratégicas, coordenadas exatas de como chegar nos lugares, quais as vantagens econômicas de tudo que pode ser encontrado na colônia, bem como quais são as necessidades e perigos encontrados na mesma.

em tom de crítica ou de conselho, dá sugestões de administração ao rei, pede mais recursos e investimentos, dizendo quais lugares necessitam de fortificação contra corsários estrangeiros, ou contra “maus selvagens”, e quais localidades são boas para se conquistar, povoar e explorar, implantar engenhos, criar animais ou estabelecer plantações, assim como fez ao descrever o rio são francisco:

além disso, pelo viés econômico e exploratório, gabriel mostra o quanto a colônia não deixa nada a desejar, ou é superior, em comparação com o que pode ser encontrado em portugal, espanha, índias ou nos demais territórios sob o domínio das coroas ibéricas, como ao falar das vacas criadas nas capitanias de são vicente e santo amaro
desde o proêmio (prefácio) do livro, o colono português já deixou claro um de seus maiores objetivos com a redação de seu tratado.

soares de sousa pede ao rei filipe ii que a colônia seja fortificada, frisando as perdas materiais e de vidas que poderiam ser geradas por uma iminente invasão de corsários estrangeiros, além dos enormes gastos que o monarca teria para conseguir expulsar os invasores.
o senhor de engenho ainda alerta o rei espanhol, há poucos anos no cargo, sobre a existência de pedras e metais preciosos a serem explorados e chama a atenção para todas as riquezas e potencialidades do brasil, que teriam sido alvo de descuido por parte dos reis anteriores

para mais informações sobre as
habilidades trabalhadas nesta obra, consulte o material digital do professor.


==== pnld0030 ====================================================================
é com muita alegria que apresentamos a edição de “poemas escolhidos”, de orides fontela.


o livro reúne alguns dos versos publicados em “transposição”, “helianto”, “alba”, rosácea”, “teia” e inéditos que saíram apenas na sua “poesia completa”.


mas antes de comentar sobre os textos da antologia, vamos falar um pouco sobre a import ncia de ler "orides fontela" na escola e justificar a sua escolha.


em primeiro lugar, é preciso resgatar as grandes escritoras da literatura brasileira, que foram deixadas de lado na formação do nosso canone.


as razões são inúmeras, e nenhuma delas se refere à qualidade da obra das autoras e sua importancia.

"orides fontela" é uma poeta única.

poucas escritoras e escritores conseguiram encontrar uma voz tão própria na contramão das tendencias de seu próprio tempo.


por isso é difícil definir a poesia de orides em uma palavra ou expressão,  mas talvez possamos arriscar uma definição usando o título de seu último livro: "teia".

a poesia de "orides fontela" é como uma teia:  cada fio e cada ponto em que eles se encontram abrem uma infinidade de outros caminhos, onde seguimos sem direção, mas orientados por sua forma bem desenhada e construída,uma forma nascida ..de lá de dentro.

e na sua aparente fragilidade e leveza, nos deparamos com uma força que agarra tudo aquilo que tenta passar por ela e se ilumina com os primeiros raios de sol.

ela é simples e profunda
áspera e bela 
concreta e transcendente
clássica e moderna
silenciosa e circular.

"orides fontela" é uma artista e tanto, capaz de nos inquietar com a serenidade de quem domina a sua arte.


ela aprofunda e renova as principais conquistas do modernismo na poesia, mas também se une a um período marcado pelo fim das vanguardas e por novos diálogos com o passado.


o movimento modernista nos trouxe a possibilidade de jogar com as palavras no que elas tem de mais comum e cotidiano, mas sem desconsiderar o sentido elevado que elas podem alcançar.

e depois do clima de novidade e invenção das vanguardas dos anos cinquenta, a tradição voltou a influenciar o trabalho dos poetas, mas com outras maneiras de lidar com ela.

manuel bandeira, carlos drummond de andrade, joão cabral de melo neto  e "henriqueta lisboa" ,    são algumas das principais referências de "orides fontela", além de "fernando pessoa" e "charles baudelaire".


"orides" se junta a outros grandes poetas de sua geração, como "adélia prado", "hilda hilst", "roberto piva" e "paulo leminski".
mas não alcançou a mesma popularidade deles na época, e mesmo nos dias atuais ainda se fala muito  pouco de sua poesia.

alguns atribuem isso ao seu temperamento difícil, outros dizem que foi por conta  das dificuldades financeiras que ela atravessou na vida.


…..ser poeta realmente não é fácil….
mas sem poesia, a vida seria bem mais difícil,... não é verdade?

e por falar em poesia, vamos agora comentar um pouco sobre os textos de "orides fontela".

seus poemas são marcados pelos limites entre a experiencia e a revelação.


nos textos de orides, a palavra poética assume múltiplos significados a partir de um trabalho consciente com a linguagem.


eles são curtos e concisos, mas são carregados de símbolos que, ao mesmo tempo que nos colocam diante do etéreo e do místico, nos prendem ao corpo e ao chão.


“transposição” é seu livro de estreia, publicado em …
mil novecentos e sessenta e nove,    e 
neles já estão presentes algumas das principais características de sua obra.


“helianto” é de mil novecentos e setenta e três.


“alba”, de mil novecentos e oitenta e três.


“rosácea” veio em mil novecentos e oitenta e seis.


e “teia”, seu último livro publicado em vida, surgiu nas prateleiras em  mil novecentos e noventa e seis.


recentemente, sua poesia completa foi editada e vem despertando o interesse de novos críticos e leitores.


e não faltam motivos para ler a obra poética dessa grande escritora paulista.


a linguagem simples e economica, dizendo  tanto com tão pouco:
os objetos e temas do cotidiano,

o jogo com as palavras na construção de cada verso.

as reflexões sobre a existência,


a relação entre o mundo material e o transcendente.

e... a consciência dos limites da vida real.


agora vamos falar um pouco sobre o genero literário dos textos.



a poesia é uma composição geralmente em versos, rimados ou não, ...em que palavras, ritmos e imagens se articulam.

ela pode ser lírica, épica ou dramática.


no caso dos versos de "orides fontela", temos exemplos de poemas líricos, em que o texto expressa o mundo interior da poeta ou até mesmo de um nós que se mistura ao eu.

sua poesia é também antilírica.

a subjetividade dá lugar aos objetos e à forma como a poeta lida com eles.

é um lirismo da matéria, mas também do pensamento.


ela é uma poeta do período pós-vanguardista, no qual as rupturas promovidas pelas vanguardas dos anos cinquenta se esgotam e dão lugar a uma geração com diferentes projetos,
que vão desde o resgate do verso e da transgressão de valores,   até a circulação dos textos à margem do mercado editorial e da mistura da cultura pop com a erudita.



já orides fontela não nega o seu tempo,
mas segue uma tendência de retorno ao passado,
não em busca de reproduzir modelos, e sim de dialogar com eles e encontrar outros jeitos de dizer.

por isso, a poesia de orides está muito ligada à poética modernista, pelo seu trabalho consciente com a linguagem,
pelo uso de ritmos variados,
por um certo humor irônico,
pela combinação do sublime com o vulgar
e pelos temas retirados do cotidiano.

mesmo assim, ela não se fixou em nenhum deles, mas teceu seus versos na teia serena, lúcida e feroz da poesia.


==== pnld0051 ====================================================================

olá 

é com muita alegria  que apresentamos o livro  rio de sonhos, escrito por maria valéria rezende.

# sobre a obra

trata-se de uma obra envolvente  que circula e mergulha em um brasil imenso  e diverso,  de olho no futuro  sem perder a memória do passado. 

a leitura desta obra  pode  contribuir para a formação de um repertório múltiplo   em contato com conhecimentos de história, geografia, e artes,  além dos saberes dos povos tradicionais de nosso país.

através desse texto  você vai refletir sobre a diversidade,  o meio ambiente,  a preservação do patrimonio histórico e cultural brasileiro  e a pluralidade do brasil 

além disso  rio de sonhos é uma história de amadurecimento  de afeto,  de companheirismo  e de descobertas   

o livro aborda as escolhas pessoais  que fazemos para sair da inf ncia,  e reflete sobre os impactos dessas escolhas na nossa vida coletiva.  

# sobre o autor

maria valéria rezende  é uma das mais importantes autoras da literatura brasileira  em atividade 

ela nasceu em santos,  no ano de mil novecentos e quarenta e dois 

o interesse pela população carente  começou aos sete anos de idade  nas andanças com o pai  que era um médico conhecido na santa casa de misericórdia de santos 

enquanto o pai examinava os doentes do morro do saboó  a pequena maria valéria lia livros para as crianças dessa região 

essa experiencia marcou para sempre a vida da escritora 

sua vida de milit ncia e ativismo  começou na juventude estudantil católica 

depois  ela entrou para a congregação de nossa senhora,  cônegas de santo agostinho  em mil novecentos e sessenta e cinco,   e ainda hoje faz parte dessa instituição 

maria valéria  passou grande parte de sua vida  como missionária e educadora popular 

nos anos sessenta e setenta  escreveu livros em prosa e verso  a respeito da igreja e das classes populares brasileiras  

mas a estreia na literatura aconteceu mais tarde  em dois mil e um,  com quase sessenta anos de idade  

poucos anos depois  maria valéria se consagrou  ganhando a mais importante e tradicional distinção literária do país:  o prêmio jabuti  

ela levou o troféu  pela primeira vez em dois mil e nove,  com a obra infantil  "no risco do caracol" . 

em dois mil e treze  tornou a ser premiada,  dessa vez pela obra juvenil  "ouro dentro da cabeça" 

em dois mil e quinze  ela ganhou pela terceira vez,  na categoria  livro do ano de ficção  e romance  por sua obra quarenta dias  
maria valéria rezende  ainda faz parte da congregação de nossa senhora  e mora em joão pessoa, na paraíba . 

ela já publicou inúmeras obras  e foi traduzida para diversos idiomas 

seus livros são conhecidos  e apreciados pelo público  e pela crítica  

agora  vamos conhecer a obra  rio de sonhos 

no começo desse livro  paulo salva iara  à beira do rio são francisco

os dois decidem sair em busca do avô da menina  um vaqueiro de origem indígena  que estava desaparecido na caatinga 

ao longo dessa busca  os dois jovens cruzam com os povos tradicionais do brasil  e encontram pinturas rupestres  

eles também enfrentam  juntos  a urgência de preservar um território  prestes a ser alagado para a construção de uma hidrelétrica  e contemplam a beleza natural do rio são francisco e do interior do brasil  

# motivação à leitura

rio de sonhos  é um livro leve e divertido  com linguagem acessível e descrições são claras.

em rio de sonhos,  o leitor tem a sensação  de que está caminhando ao lado de paulo e iara  


# relação entre obra, tema e gênero literário

a obra pode ser classificada como uma novela que se utiliza de características do conto. 

a novela  se destaca pela pluralidade dramática,  com diversos conflitos e ações que se articulam,  formando um conjunto amplo de personagens em diversos espaços e em tempo mais extenso.

já o conto  contém um só drama,  um só conflito,  uma ação,  com poucas personagens e restrição do tempo e do espaço essa unidade flui para um único efeito

em rio de sonhos  por mais que a ação seja centrada na jornada de paulo e iara,  várias características do texto  permitem que nós o classifiquemos como uma novela  

essas características incluem o amplo espaço no qual a ação de desenvolve  como também a união de várias situações dramáticas  que levam à resolução final do texto 

rio de sonhos é um livro delicioso,  esperamos que aproveitem a leitura! 

==== pnld0018 ====================================================================

olá! é com muita alegria que apresentamos a você o rabi de bacherach, de heinrich heine.

seja por uma fatalidade do destino – possível perda de capítulos incendiados, ou pela dificuldade em finalizar o pretensioso projeto – uma obra imortal sobre a grande dor judaica, heinrich heine nos instiga com seu “fragmento de romance”.

desejoso de secularizar a cultura judaica, o jovem escritor propôs-se a criar uma epopeia sobre o povo judeu diante da escalada antissemita na alemanha do século xix.


publicado em 1840, dezesseis anos após sua concepção inicial, o texto reflete as diversas facetas do autor:
ascendência judaica, lirismo poético, ironia, crítica à religião e índole revolucionária.


heinrich heine nasceu numa família judia assimilada. seu pai era um comerciante que, durante a ocupação francesa, se beneficiou dos novos ideais de igualdade cívica para todos os cidadãos, em particular importante para os judeus, uma minoria discriminada nos territórios da atual alemanha. 

quando o negócio do pai faliu, heine foi enviado para hamburgo, onde um tio salomon, um rico banqueiro, financiou os estudos e encorajou-o a iniciar uma carreira comercial.

em breve se tornou evidente que heine não tinha um interesse na carreira comercial e assim, voltou-se para o estudo de direito. 

descobriu também que estava menos interessado no direito do que na literatura, apesar de se ter licenciado em direito em 1825, ao mesmo tempo que decidiu converter-se do judaísmo para o cristianismo luterano,nomeando-se a si próprio oficialmente pelo nome de heinrich heine.

decidiu-se pela conversão considerando as várias proibições e restrições aos judeus, então vigentes em muitos estados alemães. o exercício de várias profissões e cargos em determinadas instituições, assim como o acesso a certas universidades eram proibidos aos judeus. 

assim proclamou sua conversão como o "bilhete de admissão na cultura européia", apesar de a realidade ter sido bem diferente. outra grande razão para a conversão de heine foi o possível acesso que o teria ao mundos dos escritores rom nticos, em que a religião luterana e católica desempenhavam importante papel.

como poeta, heine fez a sua estreia em 1821. heine trocou a alemanha por paris em 1831, onde sofreu influência dos socialistas utópicos, seguidores do conde saint-simon, cujo partido político intitula-se, em português, são simonistas, um grupo que pregava um paraíso igualitário baseado na meritocracia.

a opção por paris, a principio, foi voluntária, pois heine acreditava que encontraria na capital francesa maior liberdade de expressão e maior compreensão de suas ideias por parte da sociedade francesa, o que de fato aconteceu.

seus escritos geraram desconforto nas autoridades alemãs e heine foi tido como um subversivo e sofreu com a censura. suas obras foram banidas da alemanha, assim como outros escritos associados ao movimento da jovem alemanha de 1835, liderado por heine. 

o escritor foi então proibido de voltar a viver em sua terra natal e permaneceu exilado na frança.

embora no exílio, heine sempre manteve uma profunda ligação com a alemanha, que se exercia através da crítica constante da situação política de seu país. a influência dos ideais franceses sobre seu espírito libertaram afinal em uma renovação da literatura alemã.

heine foi um grande mediador dos aspectos culturais entre frança e alemanha e acreditava também em uma união entre a filosofia alemã e espírito revolucionário francês que culminariam na emancipação política e cultural da europa.

foi um crítico mordaz das instituições religiosas. a famosa expressão que qualifica a religião como "ópio do povo" - havia sido adiantada por heine. 

de fato, entre os livros queimados pelos nazistas, em 1933, na praça da ópera de berlim estavam as obras de heine. como ele próprio dissera, sociedades que queimam livros não demoram a queimar seres humanos.

heine teve uma influência muito maior ao redor do mundo que na própria alemanha. na frança, sua obra foi aclamada e o escritor chegou a receber uma pensão do governo francês. também no japão e china foi admirado e na europa oriental foi tido com uma das grandes influências na formação de uma literatura nacional, assim como goethe.

por que ler esta obra? seria convincente responder a essa questão relatando que heinrich heine foi um autor admirado por machado de assis, castro alves e álvares de azevedo?

tanto o prosador heine quanto o poeta rom ntico mundialmente reconhecido.

caso os olhos do leitor não tenham ainda brilhado, o fragmento de romance permite viajar pela saga judaica com matizes jornalísticas, poéticas, irônicas e paródicas. a prosa do autor é refinada, rica em imagens e precisa.


o romance histórico, gênero da obra em questão, é uma narrativa ficcional ambientada no passado e marcado pela influência de personagens e eventos históricos.

heine traz estas referências, sob a forma de guarnições jornalísticas, que antecedem o bloco de ações das personagens e contextualizam os acontecimentos da narrativa. assim, o panorama geral é em seguida particularizado.  observam-se tais aspectos no relato sobre a cidade medieval de bacherach e sobre as perseguições aos judeus.

aproxima-se do romance épico na medida em que os protagonistas personificam o caráter heroico dos judeus. abraão e sara remetem ao casal homônimo bíblico, cujos inúmeros descendentes herdariam a terra prometida. o próprio abraão das escrituras é considerado o fundador do judaísmo. o rabino e sua esposa encarnam traços universais de sua cultura e destinam-se à perseguição.

o êxodo hebreu do egito é oportunamente recontado quando da celebração da páscoa judaica pelo rabi.


==== pnld0040 ====================================================================
olá! é com muita alegria que apresentamos a você “sagas”, de strindberg.

você terá oportunidade de conhecer alguns contos de um dos principais autores da suécia, escritos com base no folclore escandinavo.


o autor sueco, johan august strindberg, além de escritor, foi dramaturgo, pintor e fotógrafo.

com muita ironia, strindberg jamais foi simplista, nunca deixou de duvidar de tudo e principalmente de si mesmo.

como ele próprio se descrevia, numa de suas citações mais conhecidas: “eu sou um homem maldito, o que sei são muitas artes”.

o autor sueco, johan august strindberg, além de escritor, foi dramaturgo, pintor e fotógrafo.

após concluir seus estudos, dedicou-se à carreira de professor, ao mesmo tempo em que estudou medicina.

mais tarde tenta lançar-se como ator, mas em mil oitocentos e setenta vai estudar na universidade de uppsala, onde começa a escrever. dois anos mais tarde interrompe os estudos por razões financeiras

em seguida, o autor passa a trabalhar em um jornal local e depois na biblioteca nacional da suécia.

strindberg teve uma vida turbulenta — eis um consenso. a questão mais polêmica é a história de sua “loucura”. jávier blánquez diz que “geralmente se fala do dramaturgo como um ‘demente’, como alguém que, em determinado momento de sua vida, perdeu o controle de si”, tornando-se alheio à realidade. 

“mas os psiquiatras com os quais tenho conversado asseguram que quatro meses, o tempo que durou a crise que o levou a escrever ‘inferno’, não são suficientes para determinar que era louco. suas crises foram recorrentes, o desestabilizaram, mas não era loucura”, afirma jordi guinart. por certo, era melancolia ou, quem sabe, depressão circunstancial.

em relação ao seu país de origem, a suécia, foi tão meticuloso quanto pôde na exposição das suas contradições.

provocava sem piedade os seus compatriotas, tornando-se problemático aos olhos de muitos e incômodo para o gosto convencional.

ele foi também pioneiro em trazer para o teatro as coisas que o faziam sofrer. poderíamos chamar esse teatro de “sacrificial” para acentuar o valor da mortificação e do sacrifício no cristianismo? 

o fato é que no teatro também dito “onírico”, por onde é visível a infiltração do inconsciente, encontramos “continuadamente uma longa sequência de vícios e obsessões. 

não há dúvidas de que estávamos diante de um novo tempo e um novo espaço. rimbaud costumava dizer que o amor, o tempo e o espaço, estavam para ser reinventados. strindberg se esforçava por agir nesta direção. 

o tempo como personagem, tempo estéril, cujo força corrosiva atinge homens e idéias, já fora tomado como tema pela literatura nórdica, mais precisamente pela dinamarquesa. seu inferno é sua crise religiosa. 

a trilogia dramática “o caminho para damasco” é considerada marco inicial do teatro expressionista do século xx. a partir de então escreve alguns dramas tendo como pano de fundo a história sueca.

a última fase revela um strindberg mais anti-dogmático e dando asas à fantasia. é conhecida como fase onírica, nome dado também ao seu teatro íntimo. várias peças de c mara são deste momento: por exemplo “a tempestade”, “pelicano”, “sonata dos espectros” e “o sonho” (1902).


embora tenha escrito para sua filha que tinha apenas um ano, strindberg não deixou de se preocupar com seu aspecto literário, o que faz dessa obra uma leitura fascinante para todas as idades.

há um lugar para o herói mínimo, o herói obscuro, o herói negativo e até para o anti-herói.

portanto, é desconcertante que esse mesmo autor tenha criado algumas das imagens mais persistentes sobre a natureza física e humana de seu país natal.

por meio de sua obra, podemos mergulhar no rico universo de uma cultura estrangeira.


seu estilo é influenciado, e passa do realismo ao expressionismo. rechaçado pela academia sueca, que até hoje concede o prêmio nobel

para estreitar o convívio com a filha, strindberg elabora uma série de contos, não só para divertir a pequena filha, mas também mostrar ensinamentos da ética, da história, da política e do convívio social.

produzida nesse período, sua obra sagas reúne contos que vão de breves peças morais a narrativas oníricas.

o título abrange esse complexo de intenções. sagas, neste caso, não são apenas os feitos épicos dos grandes heróis, nem as longas narrativas do folclore escandinavo.

passando por temas históricos, humorísticos e heróicos, trazendo um tom fabulesco, ou um tom direto, grande parte dessas histórias foram inspiradas por acontecimentos na vida do autor.

além disso, são igualmente fábulas por terem seu conteúdo moral, exemplificativo, no personagem que serve de modelo para a formação do caráter.



para mais informações sobre as habilidades trabalhadas nesta obra, consulte o material digital do professor. obrigado e até a próxima!


==== pnld0041 ====================================================================
# abertura resumo da obra
olá, 
é com muita alegria que apresentamos a obra “feitiço de amor e outros contos”, de johann ludwig tieck

ele foi um dos mais importantes fundadores do movimento rom ntico alemão. 


# a obra e o autor
estudou história, filologia e literatura moderna.

conheceu o filósofo schlegel e se envolveu com as discussões do primeiro romantismo. 

conviveu também com escritores como novalis, brentano e os filósofos fichte e schelling. 

ludwig tieck foi um dos pioneiros do romantismo alemão e uma das figuras mais conhecidas de seu tempo. 

seus textos foram populares na alemanha e tiveram forte impacto na grã-bretanha, na espanha e nos estados unidos.

depois de sua morte, ficou esquecido, mas teve sua obra revalorizada a partir de meados do século vinte. 

hoje está sendo recuperado como um dos autores  da literatura rom ntica que associa elementos do maravilhoso a traços góticos na alemanha.

além de poeta e escritor, tieck se destacou como editor e tradutor.

traduziu o “dom quixote” de cervantes e boa parte da obra de shakespeare, dando continuidade às traduções iniciadas por schlegel.

consagrou a camões uma novela histórica sobre a morte do poeta português, na qual reconta o destino do poeta maior da língua portuguesa 

estabelecido em berlim, participou da sociedade literária de então 

“feitiço de amor e outros contos” reúne seis das narrativas macabras ou fantásticas de “phantasus”, colet nea concebida em três partes, cada uma com sete textos. 

os textos mais antigos e a ideia de integrá-los numa composição intermediada por diálogos remete à série de contos italianos “decameron”, do escritor italiano giovanni boccaccio.. 

a complementação, através de intervenções e de novos textos, se deu em um poema chamado “f ntaso”, deus grego dos sonhos com seres inanimados. 

ele é o guia que inicia o poeta pelas manifestações assustadoras da natureza no universo: o medo, a tolice, o gracejo, o amor.

a tradução dos contos macabros do poeta rom ntico ludwig tieck apresenta o espírito rom ntico e poético que ele nos legou. 

a pesquisa sobre as raízes e heranças dessa literatura é complexa e remete a canções de trovadores medievais e à literatura inglesa, que o escritor admirava. 



# motivação de leitura (tema: ficção, mistério e fantasia)
ludwig tieck foi um dos pioneiros do romantismo alemão e uma das figuras mais conhecidas de seu tempo. 

a seleção dos contos desta publicação privilegia o horror nas narrativas de “phantasus”. 

num poema apresentado ainda no começo do livro, “f ntaso”, deus grego dos sonhos com seres inanimados, empresta nome à série de contos e peças teatrais de tieck. 

esse personagem é o guia que inicia o poeta pelas manifestações assustadoras da natureza no universo: o medo, a tolice, o gracejo, o amor. 

ao despertar febril e inspirado pela vidência, o poeta passou a contar aos amigos a primeira história: “o loiro eckbert”. 

a narrativa acolhe o fantástico de maneira tão natural que é possível se deparar com o maravilhoso como se fosse o mero cotidiano. 

esse recorte se encontra representado principalmente no conto “feitiço de amor”. 

em narrativas como o conto “o cálice”, o macabro predomina e permeia toda a história

assim como nos “contos populares”, a trama nos “contos artísticos” de tieck nem sempre conduz a um final macabro. 

as traduções desta obra representam o primeiro esforço para apresentar um autor chave do romantismo alemão ao público brasileiro.

sobre tieck, o crítico otto maria carpeaux escreveu: “nos vinte e dois volumes das obras completas de tieck muita coisa boa e bela está enterrada e esquecida. durante trinta anos, tieck rivalizou com goethe: muitos o consideravam o verdadeiro centro da literatura alemã.”

você tem em mãos uma obra inteiramente dedicada à ficção de ludwig tieck.


# gênero e movimento literário
a tradução dos contos macabros do poeta rom ntico ludwig tieck apresenta o espírito rom ntico e poético que ele nos legou. 

a pesquisa sobre as raízes e heranças dessa literatura é complexa e remete a canções de trovadores medievais e à literatura inglesa, que o escritor admirava. 

o escritor recorria aos modelos da literatura inglesa e dos trovadores alemães medievais na composição de sua obra. 

é da afinidade com o maravilhoso e o horror provenientes dessas fontes que nascem seus contos.

a influência de ossian, criações do escocês james macpherson baseadas em baladas gálicas, é evidente no conteúdo.

percebemos semelhante fecundidade da fantasia e o mundo de heróis em infindáveis batalhas.

por outro lado, atribuindo grande valor às características da poesia medieval dos trovadores, tieck resgatou a forma predileta desses poetas, os versos de sete sílabas 

além disso, a recepção de shakespeare nos territórios de língua alemã é inaugurada concretamente durante o iluminismo.

no contexto do movimento literário “tempestade e ímpeto”, que prenuncia o romantismo alemão, a pesquisa de johann    gottfried herder se baseava em canções, baladas e contos populares. 

foi nesse mesmo ensaio que herder, pela primeira vez, empregou o conceito de “canções populares”. 

o filósofo herder buscava reabilitar o espírito literário vital e vigoroso, mas também o “nacional”, o que gerou interpretações controversas 

a polêmica em torno do conceito “contos populares” desencadeia uma cisão dentro do romantismo alemão. 

as pesquisas do folclore alemão nortearam as pesquisas dos irmãos grimm, célebres pelos “contos maravilhosos para o lar e as crianças”, histórias extraídas da tradição oral. 

outros rom nticos, entre eles ludwig tieck e novalis, transformaram o conteúdo popular num significado individual e subjetivo. 

tieck admirava o maravilhoso em shakespeare, presente no ambiente mágico de peças como “a tempestade”.

o autor alemão admirava a “prodigiosa fecundidade da fantasia” do dramaturgo inglês.

tieck também amava a poesia medieval, que, segundo ele, dava sensibilidade a palavras e sons. 

preservando o caráter das canções de trovadores medievais, tieck desenvolveu a habilidade de empregar o verso em redondilha maior em passagens de alguns contos como “a montanha das runas” e “eckart”. 

o escritor alemão, portanto, adaptou as trovas medievais, colocando-as nos contos de inspiração popular.

o conto é o gênero literário praticado por ludwig tieck e nele o autor valoriza a fantasia e a imaginação.

com a narrativa breve que segue diretamente para o final, tieck introduz o “maravilhoso”, efeito que admirou e elogiou em peças como “a tempestade” e “sonho de uma noite de verão”. 

nos desfechos dos contos maravilhosos de tieck, prevalece o tom do gênero maravilhoso aliado ao sombrio.

a literatura dedicada a temas sinistros e sobrenaturais representa uma ruptura com o racionalismo clássico. 

entre os vários gêneros literários, o romance gótico foi o mais popular e causou o maior impacto sobre a imaginação popular.

muitas características desse gênero estão presentes nas histórias macabras de autores como wilkie collins, h. p. lovecraft e stephen king, assim como nos romances policiais. 


para mais informações sobre as habilidades trabalhadas nesta obra, consulte o material digital do professor. obrigado e até a próxima!

==== pnld0043 ====================================================================

# abertura resumo da obra
olá, é com muita alegria que apresentamos a obra “diários de adão e eva e outras sátiras bíblicas”, de mark twain.

a obra reúne os principais textos da sátira de twain a alguns livros e personagens bíblicos.


# a obra e o autor
o escritor atuou como tipógrafo, repórter, colunista, palestrante, minerador e piloto de barco.

a diversidade de atividades praticadas levaram a conhecer diferentes realidades pelos estados unidos, inspiração para grande parte de seus personagens e romances

usando linguagem irônica e cheia de humor, criticou a sociedade escravocrata da época, onde fanatismo e hipocrisia eram alvos prediletos de sua sátira.

porém, curiosamente, o autor não demonstrava tamanho interesse pela escola, pela leitura e a escrita.

entre outras muitas experiências, vivenciou a de soldado durante a guerra civil, antes de se convencer de que seu verdadeiro talento era o de inventar histórias, onde realizou muitas palestras.

suas narrativas eram contadas com sotaque sulista , com acento carregado, lento, anasalado, sem pressa, recheado de vocábulos e expressões típicas do sul.

um regionalista que aclamava sua origem

nunca tentou ser quem não era. nunca quis passar por pessoa culta, erudita

mark twain tentou promover leituras públicas de seus livros, mas os primeiros resultados foram desanimadores. “eu achava que bastava imitar dickens”, ele disse em sua autobiografia, “subir na plataforma e ler o livro. mas tentei fazer isso e foi um fracasso. coisas escritas não servem para discursos”. ele então desenvolveu seu estilo próprio e divertido de dar palestras como se estivesse conversando com o público.
uma curiosidade, é que o autor era obcecado pela figura do personagem adão.

entre suas cartas trocadas com a sua esposa, havia uma em que estava escrito:“ adáo, o pai da raça humana, certamente merece ser honrado”

motivado por tal obsessáo veio a redigir uma petição ao congresso americano para conceder à cidade de elmira permissão para construir um monumento a adão, sem sucesso.

seguindo sua jornada, enfim escreve os diários de adão e eva, em  mil oitocentos e noventa e trës e mil novecentos e cinco, respectivamente, não consegue que sejam imediatamente publicados

para conseguir sua publicação, buscou apoio de um amigo antigo, então secretário de turismo da região de buffalo,

o político sugere que ele adapte o manuscrito de forma a ambientar o enredo do mítico paraíso de adão e eva nas cataratas do niágara, pois assim poderia ser vendido como souvenir durante a exposição pan-americana de chicago,

uma das modificações mais óbvias é uma em que eva começa a organizar o paraíso colocando placas que indicavam os diversos locais de visitação do lugar, como se o jardim do éden fosse um parque público.

no relato adão diz que algumas dessas placas indicam: “por aqui para a piscina de hidro”, “por aqui para a ilha das cabras”, “caverna dos ventos por aqui”
a segunda parte do diário, “escrita” por eva, é primeiramente publicada como uma edição especial, natalina


# motivação de leitura (tema: ficção, mistério e fantasia)
a plateia ria do momento em que ele abria a boca até o final

suas famosas tiradas deixam visível a intenção de dizer a verdade, de fazer as pessoas perderem a pose rindo até se darem conta de que deveriam rir, ou ter vergonha de si mesmas.

mark twain permanece como um dos mais amados autores ligados ao individualismo americano. ao contrário de tantos dos seus contempor neos, ele não via os estados unidos como uma extensão da europa. 

o autor exaltou o empreendedorismo individual e se posicionou contra as injustiças onde quer que as encontrasse. enquanto morou em vienna, desafiou a imprensa antissemita e defendeu o capitão francês alfred dreyfus, que as cortes militares tinham condenado por traição por ser judeu.

quando estava dando palestras na inglaterra em 1894, sua filha morreu de meningite. livy, sua esposa por 34 anos, sucumbiu por conta de um problema cardíaco em 1904. “durante os anos que se seguiram à morte da minha esposa”, escreveu em sua autobiografia, “eu mergulhei em uma triste maré de banquetes e discursos sobre causas elevadas e santas; mas enquanto essas coisas me animavam e alimentavam intelectualmente, elas tocavam apenas momentaneamente meu coração, que então ficava seco e empoeirado”.

mark twain celebrava o país como uma civilização distinta e defendia a liberdade e a justiça, e também promovia a paz. 

retratava espíritos livres, determinados, rústicos, que superavam obstáculos intimidadores para realizar seus destinos. seu charme pessoal e seu humor afiado ainda fazem as pessoas sorrir. suas obras voltadas ao público juvenil apresentam protagonistas fortes e inspiradores.


# gênero e movimento literário
curiosamente, a parte dois, contém, no subtítulo, a observação: “traduzido do original”.

essa observação insinua é que mark twain não assume a autoria da obra, mas sim a tradução.

ou seja, o autor conduz o leitor a pensar que ele foi “mero” tradutor, e não quem ficcionalizou os acontecimentos narrados no jardim do éden.

se por um acaso do destino a obra visse a ser acusada de blasfêmia, ironia ou deboche, não seria mark twain quem teria de se sentar no banco dos réus.

a última subdivisão, escrita por eva quarenta anos depois, já longe do paraíso, tem um estilo extremamente “meloso”.
o objetivo de twain era relacionar as narrativas do livro do gênesis ao presente e ao mesmo tempo demonstrar sua teoria cíclica da história

o gênero é o diário, com suas entradas e datas, que remetem a uma visão ciclica da história.


# (quadro de habilidades referentes à obra)

# encerramento ( vinheta com logo )

para mais informações sobre as habilidades trabalhadas nesta obra, consulte o material digital do professor. obrigado e até a próxima!


==== pnld0045 ====================================================================

olá! é com muita alegria que apresentamos a você “tipógrafo, poeta e editor”, de joão martins de athayde.

esta é uma obra produzida por aquele que é considerado um dos maiores cordelistas brasileiros.

o autor nasceu em mil oitocentos e setenta e sete, na paraíba,  e é considerado  o príncipe dos poetas populares do nordeste do brasil  

mesmo sem frequentar a escola,  seu maior sonho sempre foi criar histórias e compor versos

depois de cada dia de trabalho, que às vezes se prolongava até a madrugada, ele tinha uma grande paixão, que era o cinema 

ninguém sabe, com certeza, quantos folhetos foram escritos e publicados por joão martins de athayde

sua gráfica, trabalhando a todo vapor, quase que semanalmente lançava um título novo ou mais uma edição de um folheto que estivesse fazendo sucesso

as encomendas recebidas de seus agentes espalhados por todo o nordeste chegavam quase todos os dias, e ele procurava entregar, imprimindo durante as madrugadas

é com athayde que se realizam mudanças na relação entre os poetas e o proprietário da gráfica, e na apresentação dos folhetos

ele fez surgir os contratos de edição com o pagamento de direitos de propriedade intelectual, o uso de subtítulos e pre mbulos em prosa e o padrão fixo dos folhetos pelo número de páginas em múltiplos de quatro

joão martins de athayde contribuiu para o desenvolvimento da arte e da comercialização do folheto popular no recife, para onde se mudou

foi o desbravador da indústria do folheto de cordel no país, industrializando e comercializando sua produção e a de outros artistas

criou uma grande rede de atividades lucrativas no nordeste, que se espalhou para outras regiões brasileiras, possibilitando a diversos poetas populares se dedicarem exclusivamente à poesia como atividade profissional

joão martins de athayde, no ano de mil novecentos e quarenta e nove, após haver passado por um acidente vascular cerebral, se afastou da atividade de editor e vendeu a sua tipografia 

athayde foi acusado de comprar originais de dezenas de poetas populares e publicá-los sem mencionar os nomes dos autores

esse fato dificultou a identificação da autoria de histórias rimadas da literatura de cordel

no entanto, esse fato não diminui a import ncia da sua obra, tampouco sua contribuição para a poesia popular no brasil

suas obras até hoje são reimpressas, quando seu estilo irônico e jornalístico se revela nos versos que faziam crítica aos costumes modernos

nesta obra,  você vai encontrar uma seleção de oito cordéis  que abordam o universo do sertão nordestino .

o imaginário da literatura de cordel  é criado a partir de relações entre mundos culturais distintos  

nesses poemas,  a imagem poética não representa a realidade 

ela faz parte de um imaginário construído,  partilhado pelo  poeta  com seus ouvintes e leitores 

a poesia popular do nordeste do brasil  tem muita influência da linguagem oral  e  aproveita muito da língua coloquial,  praticada nas ruas e na comunicação cotidiana  

por isso,  a poesia narrativa do cordel  é compartilhada e desfrutada de forma coletiva  e atinge uma grande resson ncia social  

no cordel,  a métrica e a rima são recursos que favorecem a memorização 

todo grande cordelista dialoga com uma longa tradição de poetas do passado,  recuperando e recriando  temas,  formas  e imagens 

assim,  a sabedoria de matriz oral e popular,  acumulada nessa tradição,  é transmitida pelos cordelistas 

o contato com essa literatura  é de extrema import ncia para compreender  a diversidade cultural do nosso país,  além de favorecer a quebra de estereótipos  e ampliar a discussão sobre a produção artística brasileira. 

muitos temas do cordel  são originários das tradições populares e eruditas da europa medieval e moderna .

nos folhetos  encontramos temas das novelas de cavalaria da idade média  e das narrativas bíblicas 

a variedade linguística utilizada pelo poeta é caracterizada pelo registro, na escrita, de formas típicas da linguagem oral

o cordel é uma forma poética popular de origem europeia., mas foi incorporado e desenvolvido na cultura brasileira, sobretudo do nordeste do país

os poemas em cordel costumam ser longos e narram histórias repletas de reviravoltas e aventuras

a linguagem é marcada pelo humor e pelo suspense, com o objetivo de captar a atenção do público

desde a sua origem, para facilitar a memorização e a recitação falada, esses textos apresentam um esquema próprio de rimas e uma métrica específica

antigamente, os cordéis eram vendidos nas feiras, em barracas e tendas que os expunham pendurados em varais de barbante, presos por pregadores de roupas ou exibidos em balcões e esteiras

muitas vezes, performances eram realizadas ali mesmo, pelos próprios autores que declamavam oralmente as narrativas em versos

havia uma roda de ouvintes, formada por passantes que paravam para ouvir as histórias

hoje, em vários lugares, essa configuração mudou um pouco e os cordéis são expostos dentro de plásticos transparentes, em bancas de revistas, e acabam sendo mais lidos silenciosamente

algumas dessas obras podem também ser musicadas e interpretadas por cantores

a matriz oral dos cordéis remonta a tempos ancestrais, quando histórias eram compostas e contadas por rapsodos e aedos para entreter um grupo de ouvintes

muitas narrativas célebres que conhecemos hoje, como os poemas épicos e as antigas sagas, nasceram nesse contexto de oralidade e, somente depois de muito tempo, ganharam versões escritas

no caso do cordel, temos uma forma poética cujas origens podem ser encontradas nas tradições poéticas orais da antiguidade clássica

além disso, havia as práticas dos trovadores medievais da europa que, em portugal, na espanha e no sul da frança, compunham e recitavam poemas narrativos

durante o século dezenove,  no brasil e, em especial, a partir do estado da paraíba, os poemas narrativos começaram a circular em forma de livretos expostos nas feiras populares

no caso de athayde, havia poemas abrangendo vários ciclos, dentre eles o do sertanejo, o heroico, o circunstancial, o da crítica de costumes, o das pelejas e desafios, o das fábulas, o dos repentes

a obra do poeta athayde deve ser observada em conjunto, nos seus vários aspectos

o folclorista luís da c mara cascudo afirmou que athayde foi "o maior poeta, mais tradicionalista do nordeste brasileiro"


==== pnld0046 ====================================================================

olá,
é com muita alegria que apresentamos a obra “viagem em volta do meu quarto”, de xavier de maistre.


xavier de maistre, um oficial francês de 27 anos que, em 1790, ficou preso num quarto, por seis semanas.

e foi assim que ele ficou famoso por criar uma nova modalidade de turismo: a viagem pelo quarto.

“viagem ao redor do meu quarto” e sua continuação, o livro “expedição noturna ao redor do meu quarto”, narram as andanças de uma pessoa presa num único ambiente. xavier faz uma paródia das grandes jornadas que seus contempor neos escreviam – como cook, banks e solander -, mostrando que mesmo com pouco espaço, a imaginação pode viajar livremente.

logo no capítulo de abertura, de maistre convida: “todo homem sensato, tenho certeza, irá adotar meu sistema. 

quaisquer que sejam suas características peculiares ou temperamento. seja ele miserável ou pródigo, rico ou pobre, jovem ou velho, nascido numa zona tórrida ou próximo aos polos, ele poderá viajar comigo. entre a imensa família de homens que lotam a terra, não há um, não, nenhum (quero dizer, daqueles que habitam salas), que, depois de ler este livro, pode recusar sua aprovação do novo modo de viajar que apresento ao mundo”.

ao longo dos 42 curtos capítulos, o francês descreve sua cama, seus livros e suas gavetas. também fala de seus quadros, suas relações com o criado e sua cachorra, faz reflexões filosóficas sobre a vida, relembra histórias do passado. tudo com o bom humor e a ironia de quem não pode sair de casa. 


“o mais preguiçoso dos seres não vai ter mais nenhuma razão para hesitar antes de partir em busca de prazeres que não vão lhes custar dinheiro ou esforço”.

por que sua jornada dura exatamente 42 dias? ele conta aos leitores que nem ele mesmo sabe responder a essa pergunta. a verdade é que xavier de maistre encontra-se em prisão domiciliar em turim, no norte da itália, por causa de sua participação num duelo. 

o livro acabou tornando-se um grande sucesso, que influenciou até mesmo machado de assis, que o menciona na abertura de memórias póstumas de brás cubas.

quer viajar pelo quarto ou pela casa como xavier? ele deixa dicas sobre os rumos de sua jornada: “quando viajo pelo meu quarto, raramente fico em linha reta. da minha mesa, vou em direção a uma foto colocada em um canto; daí parto numa direção oblíqua para a porta; 

e então, embora, ao iniciar, pretendesse retornar à minha mesa, ainda assim, se por acaso caísse com minha poltrona no caminho, imediatamente, e sem a menor cerimônia, assento-me nela. a propósito, que artigo de mobília importante uma poltrona é e, acima de tudo, quão conveniente para um homem pensativo.”

ele também reflete sobre os pequenos prazeres em seu espaço fechado:

as reflexões são sobre como o prazer que sentimos ao viajar depende mais da mentalidade do que o destino para o qual vamos. 

um aspecto interessante a destacar neste trecho corresponde à tematização da noção de tédio. de acordo com o capítulo i de viagem à volta o meu quarto, o relato que o autor oferece da viagem no quarto proporciona um meio seguro contra o tédio, o que parece apontar para uma diferença relativamente ao pensamento pascaliano. enquanto que, para pascal, ficar no quarto em repouso se constitui como uma fonte de tédio, para xavier de maistre ficar confinado num quarto constitui-se como fonte de divertimento, podendo, dessa forma, configurar-se como um modo de combater o tédio. 

porém, para se compreender de que forma o confinamento num quarto se constitui como fonte de divertimento, é necessário ter em consideração a caracterização que xavier de maistre apresenta da natureza humana enquanto uma natureza dupla no capítulo vi de viagem à volta do meu quarto. de acordo com este capítulo, o «homem é duplo», correspondendo essa duplicidade do homem àquilo que xavier de maistre denomina de «sistema da alma e do animal» (maistre, 2015, p. 25). 

tendo em conta todas estas considerações, viajar à volta do quarto, isto é, ser capaz de produzir divertimento no confinamento do quarto, significa libertar a «alma» dos
automatismos que prendem o «animal» à vida quotidiana, fazendo com que a alma se
possa elevar acima da quotidianidade, ao mesmo tempo que o animal caminha autonomamente no campo das solicitações que o ocupam.

a distinção entre alma e animal apresentada no capítulo vi de viagem à volta do meu quarto não deve ser confundida com a distinção moderna entre corpo e alma. 

o termo sátia menipeia é geralmente utilizado por gramáticos clássicos e filólogos para diferenciar as sátiras em prosa (por oposição às sátiras em verso de juvenal e imitadores). as atitudes mentais típicas ridicularizadas pelas sátiras menipeias são os "pedantes, os sectários, os excêntricos, os arrivistas. o termo se diferencia da sátira praticada anteriormente por aristófanes, por exemplo, que era baseada em ataques pessoais. 

o gênero é importante para bakhtin, que o considera uma das origens do romance polifônico.[3] a sátira menipeia teria influenciado vários autores posteriores, como rabelais, erasmo de roterdã, voltaire, entre outros.

"na sátira menipéia, desaparecem todos os resquícios das barreiras hierárquica, social, etária, sexual, religiosa, ideológica, nacional, linguística etc.; (...) tudo é alvo de rebaixamento grosseiro e inversões ousadas, nas quais os momentos elevados do mundo aparecem às avessas, com uma faceta oposta àquela em que antes se manifestavam"

é muito fácil deixar de apreciar o que é comum e familiar. segundo o autor da arte de viajar, após nos habituarmos a um lugar, nos tornamos cegos aos seus atrativos. caímos no hábito de considerar chato um universo perfeitamente alinhado às nossas expectativas.

quantas vezes, lhe pergunto, você já não admirou durante uma viagem algo que na sua própria cidade considera irrelevante? quantas vezes tentou passear pelo seu próprio bairro com o mesmo olhar inocente e curioso que tem como turista numa cidade estrangeira? e alguma vez já tentou olhar para a sua própria casa com o mesmo olhar de admiração e prazer que sente em um hotel num local desconhecido, rever os cantinhos e momentos prazerosos que o lar te traz?


==== pnld0059 ====================================================================
# abertura resumo da obra
olá! é com muita alegria que apresentamos a você popol vuh.

publicado em mil quinhentos e cinquenta, é um poema épico que narra o mito maia de criação do mundo e das formas de vida

seus temas principais envolvem a diversidade cultural, a cosmogonia, a mitologia, as lendas de formação e criação


este livro consiste no registro escrito do mito de criação maia, até então expresso por meio de poesia oral

sendo uma das poucas obras da cultura maia sobreviventes, o livro é de suma import ncia para preservar a memória desse povo

a obra traz a versão escrita de tradições orais do povo maia 

com a invasão espanhola, houve a perda de muitos elementos das culturas mesoamericanas

terminada a conquista do méxico pelos espanhóis, hernán cortés, recebendo a notícia da existência de terras habitadas por inúmeras tribos da guatemala, decidiu enviar o mais intrépido dos seus capitães, pedro de alvarado, para submete-las

várias nações indígenas, descendentes dos antigos maias, ocupavam o território da guatemala no século dezesseis entre elas, uma das mais importantes era a quiche essa nação era a mais poderosa e culta entre as que ocupavam o território da américa central

quando alvarado cruzou as fronteiras de quiché, os índios resistiram vigorosamente porém, após sangrentas batalhas, acabaram se rendendo aos espanhóis

como recurso desesperado, os reis quiché propuseram a alvarado recebe-lo em paz na sua capital mas, uma vez dentro dos seus muros, o capitão espanhol foi para os campos vizinhos e se apoderou dos reis, a quem condenou à morte como traidores e executou diante da população apavorada

logo depois, mandou arrasar a cidade, cujos habitantes se dispersaram por todas as direções

por isso, este livro é um precioso registro da cultura de um povo da américa pré-colombiana

considerado mais importante documento poético-político da antiguidade das américas, o popol vuh, livro do conselho ou livro da comunidade apresenta a cosmogonia, o amanhecer da natureza e da humanidade, a mitologia heroica, a história e a genealogia dos maias-quiché da guatemala 

seu legado milenar estava vivo na tradição oral e em livros hieroglíficos até o final do século quinze e início do dezesseis

foi quando um anônimo mestre da palavra, na tentativa de preservá-lo da ameaça dos invasores espanhóis, o registrou em um manuscrito em língua quiché, mas com alfabeto latino

esse original  manuscrito foi copiado e traduzido pelo frei dominicano francisco ximénez

para identificar esse livro original, só temos as informações registradas pelo desconhecido autor do livro quiche 

no entanto, pelo conhecimento que se tem do sistema de escrita dos índios americanos de antes da conquista, não se sabe se o livro quiché foi um documento de forma fixa 

é provável supor que foi um livro de pinturas que os sacerdotes interpretavam para o povo para preservar a memória das origens da raça e os mistérios da sua religião

no centro do cenário de popol vuh, um escritório repleto de livros, está o tradutor  cercado pelos demais personagens: o paleógrafo, o dicionário, o etimologista, o epigrafista, o literalista e o nativo


o popol vuh que hoje conhecemos pelo manuscrito de ximénez sugere uma mistura de funções e registros, diferentes dos ocidentais

escrito por um nobre quiché consciente das tradições mitológicas e poéticas mesoamericanas, o texto revela uma amplitude entre história, mito e filosofia essa união o aproxima dos grandes livros sagrados e de obras clássicas como a ilíada e a odisseia

popol vuh une o interesse e a beleza do romance com a austeridade da história pinta com as mais vivas cores a vida e a mentalidade de um grande povo


o gênero literário da poesia épica está ligado a narrativas amplas, geralmente subdivididas em cantos menores, cujo conteúdo gira em torno de grandes acontecimentos ou ações heroicas, com elementos do fantástico

a obra “popol vuh” foi organizada nos primeiros anos após a conquista hisp nica é um registro de elementos da cultura maia, quando ainda havia pouco contato com a cultura espanhola 

no entanto, alguns especialistas já encontram elementos culturais europeus nos escritos, o que revelaria um possível sincretismo cultural

a influencia da bíblia é evidente na descrição da criação, mas essa circunstancia não é suficiente para apagar o sabor indígena do livro quiche 

suas primeiras frases podem parecer uma transcrição do livro do genesis

na época em que o popol vuh foi escrito, os índios da guatemala já eram influenciados por pinturas, livros e c nticos dos missionários espanhóis

ainda assim, o popol vuh parece ter sido escrito em parte de memória, segundo originais antigos, e em parte copiado dos livros sagrados dos quiché, aos quais se dá o nome de popol vuh, ou livro dos príncipes

os povos mesoamericanos ficaram famosos por sua riqueza cultural e construções monumentais. 

piramides e templos chamam a atenção até hoje mas esses são apenas alguns elementos das culturas desses povos 

as narrativas de criação do mundo foram imortalizadas em suportes como os códices ou ainda em uma variedade de objetos de cer mica e placas de pedra

eles são fundamentais para compreender a vida dos povos mesoamericanos no passado e no presente

os povos indígenas da atual guatemala descendem do tronco comum dos maias, que desenvolveram sua maravilhosa civilização na parte norte do país e no atual território de yucatán

as características físicas da população e a semelhança que existe entre suas línguas demonstram o parentesco entre eles

os índios do méxico e da guatemala também conservavam suas histórias e outros escritos em pinturas feitas em pano alguns deles se salvaram da destruição geral de seus livros e documentos

além disso, os mitos e lendas se inscrevem em outros tempos e lugares o popol vuh é exemplo disso

é provável que elementos transmitidos pela tradição ao longo de séculos fossem apreendidos pelos escritores que criaram o popol vuh entre eles, temos a existencia de dois gemeos, heróis culturais, presentes em outras narrativas mesoamericanas

alguns desses heróis tem a característica de tricksters, personagens espertas e trapaceiras que transgridem ordens preestabelecidas, com comportamentos que vão contra as regras sociais convencionais

heróis com essa característica são comuns em todo o mundo ameríndio, como demonstraram o antropólogo claude lévi-strauss e mário de andrade, autor brasileiro do clássico “macunaíma”, baseado em narrativas amazônicas

a peça também sugere que traduzir e entender o popol vuh é uma aventura que exige um esforço multidisciplinar e é uma tarefa interminável

nesta edição, o tradutor intercalou à narrativa em prosa trechos dispostos em versos, principalmente quando há invocações, cantos, conjuros, exortações e marcas de oralidade


para mais informações sobre as habilidades trabalhadas nesta obra, consulte o material digital do professor. obrigado e até a próxima!


==== pnld0060 ====================================================================

olá! é com muita alegria que apresentamos a você a origem das espécies, de charles darwin.

teriam os seres vivos surgido com a complexidade que apresentam hoje, ou teriam eles se transformado no decorrer do tempo?

durante milenios, filósofos e naturalistas debateram esse tema.


esse livro é considerado a base da biologia evolutiva que redefiniu para sempre a ciência moderna!

publicado em mil oitocentos e cinquenta e nove, a obra apresenta a teoria de darwin com base em dados coletados ao longo de sua jornada na expedição beagle.

desde muito jovem, já demonstrava seu amor pela ciencia,
dedicando-se às suas coleções e realizando experimentos  com seu irmão,  em um laboratório de química.

aos 16 anos, darwin iniciou o curso de medicina,  seguindo uma tradição de família, mas acabou abandonando o curso.

com muito custo, formou-se em teologia, para se tornar um clérigo.
mas foi uma viagem de cinco anos que definiu o futuro do jovem.
em 1831, darwin embarcou no beagle, um navio enviado pela coroa britanica para atualizar os mapas das costas da américa do sul,  áfrica e austrália.

a função do jovem era: observar e coletar amostras de seres vivos desses lugares.

darwin se baseou  muito no que ele observou durante a viagem no beagle, que durou cinco anos, 
mas também aplicou parte do conhecimento que tinha sobre a criação de animais.

ele observou que, dadas condições ideais, todos os animais em cativeiro sobrevivem.

mas o criador pode selecionar indivíduos com as características que mais lhe interessam para se reproduzir.

foi assim que se criaram as diferentes raças (subespécies) de galinhas, pombos e porcos.

se o homem é capaz de fazer essa seleção artificial,
então a natureza deve fazer a própria seleção natural.

em resumo, pelo darwinismo, os seres vivos se desenvolvem com base na variação adaptação e seleção natural das espécies.

segundo o darwinismo, a seleção natural é um processo de longo prazo.


os indivíduos nascem com pequenas diferenças e algumas delas facilitam sua sobrevivencia.


ao se reproduzirem, esses indivíduos transmitem a característica favorável a seus descendentes.


o meio ambiente não induz a nenhuma variação, apenas funciona como filtro que seleciona os organismos mais adaptados ou mais aptos a sobreviver.


um dos mais interessantes conceitos de darwin é o de deriva genética. 

é o mecanismo pelo qual um acontecimento aleatório altera a frequencia de determinado alélo numa população.


essa alteração ocorre ao acaso, ou seja, não é provocada por seleção natural.


cientista e pensador, darwin foi também, à sua maneira, um escritor, que produziu uma obra volumosa e interessante.


de linguagem acessível, a obra oferece as premissas dos mecanismos de seleção natural e consequente evolução das espécies.

ainda que trabalhos posteriores tenham complementado a teoria de darwin, a precisão com a qual explica e preve os rumos da evolução mantém sua obra como extremamente relevante.


apesar das acirradas polemicas, as teses do evolucionismo permanecem até hoje,   e o nome de darwin ficou gravado na história!


o livro representa um grande salto científico,
consolidando conhecimentos passados e contemporaneos acerca da biologia.

também propos a explicação das razões da diversidade das espécies,    ao dissertar sobre a seleção natural.

à época da morte de darwin, sua teoria da evolução já era universalmente aceita. 

em homenagem ao conjunto de seu trabalho científico, ele foi enterrado na abadia de westminster ao lado de reis, rainhas e outras ilustres figuras da história brit nica. 

desenvolvimentos na genética e na biologia molecular levaram a algumas mudanças no entendimento da teoria evolucionista, porém as idéias de darwin permanecem até hoje essenciais no campo da biologia.

a obra é basilar por oferecer de maneira didática os conceitos de seleção natural e evolução das espécies, sendo muito útil como primeiro contato a esses conceitos e um primeiro contato à ciência da biologia evolutiva.

darwin foi ao mesmo tempo geólogo, bot nico, zoologista e homem de ciência. 

sua famosa viagem foi uma preparação fundamental para a sua carreira de pesquisador e escritor. 

na introdução de seu livro, ele assim se refere: 
... >>>   "as relações geológicas que existem entre a fauna extinta da américa meridional me impressionaram profundamente quando da minha viagem a bordo do beagle, na condição de naturalista. 
estes fatos parecem lançar alguma luz sobre a origem das espécies". <<<<

em todo o lugar aonde ia, darwin reunia grandes coleções de rochas, plantas e animais.

eram seres fósseis e vivos enviados à sua pátria. 

imediatamente, após seu regresso à inglaterra, darwin iniciou um caderno de notas sobre a evolução, reunindo dados sobre a variação das espécies. 

deu assim os primeiros passos para a “origem das espécies”. 

no começo, o grande enigma era explicar o aparecimento e o desaparecimento das espécies.

assim surgiram, em sua cabeça, várias questões:  

""por que se originavam as espécies?"”
"" por que se modificavam com o passar dos tempos, diferenciavam-se em numerosos tipos e frequentemente desapareciam do mundo por completo""?

darwin encontrou a chave do mistério  casualmente na leitura: 
"ensaio sobre a população", de malthus.

depois disso, nasceu à famosa doutrina darwinista da seleção natural, da luta pela sobrevivência ou da sobrevivência do mais apto : pedra fundamental da origem das espécies.

as pesquisas e estudos feitas pelo naturalista durante a viagem a bordo do beagle é que fundamentaram sua teoria da evolução.

este é um livro base para o entendimento de uma área fundamental da ciência: a biologia evolutiva.

oferece ao leitor uma explicação clara e didática a respeito da evolução por seleção natural das espécies.

além disso,  por ser um marco da história da ciencia, 
é uma oportunidade do leitor vivenciar esse ponto de inflexão no desenvolver científico.


o genero textual  é: "tratado científico",   que é  é uma produção de caráter técnico,  organizada a partir da coleta de dados e escrutínio de premissas, de linguajar específico a determinado campo do conhecimento,  mantendo, contudo, o caráter público.




para mais informações sobre as habilidades trabalhadas nesta obra, consulte o material digital do professor. obrigado e até a próxima!
