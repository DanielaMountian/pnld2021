

\section[Introdução, por Bernadete Limongi]{introdução}

\subsection{joseph conrad,\break um estranho no ninho}

``De repente, sem sequer nos dar tempo para arrumar os
pensamentos ou preparar nossas frases, nosso hóspede nos deixou; e sua
saída, sem adeus ou cerimônia, está de acordo com sua chegada
misteriosa, há muitos anos, para se estabelecer neste país. Pois houve
sempre um ar de mistério em torno dele. Isto aconteceu, em parte, pelo
seu nascimento na Polônia, em parte pela sua aparência memorável, em
parte ainda por preferir viver nas profundezas do campo, longe dos
mexericos e dos rumores, fora do alcance das anfitriãs, de tal forma
que para se ter notícias dele tinha{}-se que confiar no relato de
simples visitantes com o hábito de tocar à porta das pessoas sem se
anunciarem, que contavam, a respeito de seu anfitrião desconhecido,
tratar{}-se de um homem com bons modos, olhos muito brilhantes e que
falava inglês com um forte sotaque estrangeiro.''\footnote{ Virginia
Woolf. ``Joseph Conrad''. In \textit{Collected Essays}, vol.~1, London: The
Hogarth Press, 1980, p.~302.} Assim Virginia Woolf inicia sua homenagem a Conrad, em agosto
de 1924, por ocasião da  morte do escritor. 

Numa declaração anterior, semelhante à de Virginia Woolf, mas
menos jocosa, E.~M.~Forster refere{}-se a nosso autor ao escrever sobre
a coletânea de ensaios que Conrad reuniu em \textit{Notes on
Life and Letters} e ao volume de reflexões biográficas que
havia publicado em 1912, \textit{A Personal Record}. 
Forster refere{}-se ao personagem Conrad como elusivo, por uma de duas
possíveis razões: ``A primeira razão já foi indicada: o pavor que o
autor tem da intimidade. Ele tem uma concepção rígida sobre os direitos
do público e determinou que nós não ficássemos debruçados sobre ele;
ofereceu{}-nos  no máximo o vestíbulo e ironicamente nos convidou a
tomá{}-lo, se o quiséssemos, por seu apartamento privado. Não podemos
vê{}-lo claramente porque ele não quer ser visto. Mas também pode ser
que não o vejamos claramente porque ele seja essencialmente
nebuloso\ldots\ Isto não é uma crítica estética ou moral. É apenas uma
sugestão de que nossas dificuldades em relação ao sr.~Conrad procedam
em parte de suas dificuldades internas''.\footnote{ E.~M.~Forster. ``Joseph Conrad: 
A Note''. In Joseph Conrad, \textit{Heart of Darkness}. Edited by Paul B.~Armstrong. 
New York and London: a Norton Critical Edition, fourth edition, 2005, p.~315.}

Nascido em 1857, na região ucraniana da Polônia, então dominada pela
Rússia, Józef Teodor Konrad Korzeniowski foi o filho único de Apollo e
Evelina Korzeniowski. Apesar de ter estudado Direito e línguas na
Universidade de São Petersburgo, ter traduzido Victor Hugo e escrito
poemas e peças de teatro, os principais interesses de Apollo eram
políticos.  Suas atividades políticas e revolucionárias atraíram a ira
das autoridades russas e fizeram com que a família fosse banida para o
nordeste da Rússia, onde o clima rigoroso e as dificuldades do exílio
contribuíram para a morte prematura de Evelina, em 1865, quando Conrad
tinha apenas  sete anos. Nos três ou quatro anos seguintes o garoto foi
cuidado alternadamente pelo pai ou pelo tio materno, Thaddeus
Bobrowski, até que em 1869 Apollo recebeu permissão para mudar{}-se
para Cracóvia, onde veio a falecer em maio do mesmo ano. Thaddeus
assumiu a responsabilidade pelo menino, tendo sido, até o fim da  vida,
seu  amigo, mentor,  provedor e conselheiro financeiro. Com a idade de
quinze anos Conrad já \mbox{manifestava} o desejo de tornar{}-se um homem do
mar, embora nascido numa região distante da costa. Em 1874, aos 17
anos, conseguiu integrar a tripulação de um navio mercante francês,
fazendo sua primeira viagem, de Marselha à Martinica. Voltando a
Marselha um ano mais tarde, foi aceito como aprendiz em outro navio
francês, viajando para as Antilhas. Passou a integrar a tripulação do
navio britânico Mavis, chegando na Inglaterra pela primeira vez em
1878, portanto, aos 21 anos. Na ocasião sabia apenas algumas palavras
em inglês, embora fosse fluente em francês, além de sua língua materna.
Fez então carreira no Serviço Mercante Britânico (British Merchant
Service),  chegando  a comandante depois de aprovado nos exames
necessários. Em 1886 tornou{}-se cidadão britânico, adotando o nome de
Joseph Conrad. Viajou para Singapura, para a Península Malaia, Bangcoc, Sidney, Ilhas
Maurício, Melbourne e Adelaide. Em 1890 chegou ao Congo, pois há muito
alimentava o desejo de se aventurar na África. Fora contratado para
comandar um navio a vapor de uma companhia belga cujo capitão havia
sido assassinado. Para atingir o navio precisou subir o rio e fazer um
percurso de cerca de 200 milhas a pé, que durou 36 dias, ao final dos
quais descobriu que seu navio estava encalhado e precisando de reparos.
Depois de alguns meses pediu para voltar à Europa, com a saúde para sempre abalada.

Conrad ainda viajou para a Austrália, em 1891, mas tentativas
para conseguir outras viagens falharam e ele estabeleceu{}-se
definitivamente na Inglaterra, visando dedicar{}-se à literatura, em
1895, ano em que foi publicado seu primeiro romance, \textit{Almayer's Folly} 
(A Loucura de Almayer). No ano seguinte  desposou Jessie George, com quem 
teve dois filhos. O mar foi o maior inspirador e o principal cenário de seus romances 
e contos, embora Conrad demonstrasse sentimentos antagônicos em relação a ele.
Desejava ser reconhecido como ficcionista, e não como o marinheiro que
se tornou ficcionista. Pelo menos é o que se depreende da carta que
escreveu a seu amigo Richard Curle, em julho de 1923: ``Eu esperava que
numa visão panorâmica eu tivesse a oportunidade de me livrar da
história infernal dos navios, daquela obsessão da minha vida no mar que
tem tanta importância na minha existência literária, na minha qualidade
de escritor, quanto a enumeração  das  salas de visita que Thackeray
frequentou pode ter tido em seu talento de grande romancista. Afinal de
contas, posso ter sido um homem do mar, mas sou um autor de prosa. Na
realidade, a natureza do meu ato de escrever corre o risco de ser
obscurecida pela natureza do meu material. Admito que seja natural; mas
apenas a apreciação de uma inteligência especial pode contrabalançar a
apreciação superficial da inteligência inferior da massa de leitores e
críticos''.\footnote{ Richard Curle (1928), ``Conrad  to a Friend''. 
In Joseph Conrad, \textit{Heart of Darkness}. New York: W.~W.~Norton \& Company, 1971.}


Conrad permanece um caso único no panorama literário inglês.
Como bem diz Muriel C.~Bradbrook, ``Na história da literatura inglesa
nunca houve nada como a história de Joseph Conrad; nem, ao que me
consta, houve alguém como ele em qualquer outra literatura
europeia''.\footnote{ Muriel C.~Bradbrook. \textit{Joseph Conrad: Poland's English Genius}.
New York: Cambridge University Press, 1941, pp.~5{}-6. Reprinted by
permission of the publisher in Joseph Conrad. \textit{Heart of
Darkness}. Edited by Robert Kimbrough, op.~cit., p.~98.}
O inglês foi sua terceira língua e ele a dominou como um
mestre.  Virginia Woolf assim se refere à beleza de seu estilo: ``Quando
abrimos suas páginas sentimo{}-nos como Helena deve ter se sentido ao
olhar{}-se no espelho e verificar que, fizesse ela o que fizesse,
jamais seria uma mulher comum. Assim também Conrad recebeu o talento e
esforçou{}-se para desenvolvê{}-lo, e sentia tamanha  obrigação para
com uma língua estranha, caracteristicamente apreciada por
suas qualidades latinas em vez das saxônicas, que lhe parecia
impossível fazer um movimento feio ou insignificante com sua pena. Sua
amante, seu estilo,  às vezes parece sonolenta, em repouso. Mas ai de
quem dirigir{}-se a ela, pois de forma magnífica ela se imporá, e com
que cores, triunfo e majestade!''.\footnote{ Virginia Woolf. ``Joseph Conrad''. 
In \textit{Collected Essays}, op.~cit., p.~302.}

\subsection{a experiência no congo}

No século \textsc{xix} as nações europeias disputavam entre si a exploração das
riquezas da África. Britânicos, holandeses, portugueses e franceses
eram os principais rivais. Grandes exploradores, como o galês Stanley e
o inglês David Livingstone, entre outros, alimentavam a imaginação dos
jovens com o relato de suas descobertas e experiências. Por volta dos
10 anos Conrad leu um livro muito popular na época, de autoria de
Leopold  McClintoc, \textit{The Voyage of the `Fox' in the Arctic
Seas}. O livro despertou seu interesse por mapas, e do Ártico para a
África foi apenas um passo. As aventuras de Livingstone, que se recusou
a deixar o Congo, lá morrendo ainda à procura da nascente do Nilo, o
levaram a sonhar com o distante continente. Ele realizaria seu sonho
muitos anos depois, sonho este que se revelaria um pesadelo.

A situação do Congo na época era bastante peculiar. Não era
disputado por  nenhuma nação e, provavelmente por isso, tornou{}-se
objeto da cobiça do rei Leopoldo \textsc{ii} da Bélgica. Em 1876, Leopoldo
realizou uma conferência em Bruxelas para examinar a situação da
África, nela expondo o seu interesse em ``abrir para a civilização a
única parte do nosso globo onde o cristianismo ainda não penetrou e
eliminar a escuridão que cobre a população \mbox{inteira}''.\footnote{ Maurice 
N.~Hennessy. \textit{Congo}. London: Pall Mall, 1961. Reprinted by
permission in Joseph Conrad. \textit{Heart of Darkness}. Edited by
Robert Kimbrough, op.~cit., p.~87.} 
Como resultado da conferência, foi criada a Associação Africana
Internacional, que se tornou uma organização pessoal do rei belga e o
levou a criar a Associação Internacional do Congo.


Em 1884, Bismarck organizou uma conferência em Berlim justamente para
aparar as arestas entre as nações europeias rivais na África e
estabelecer regras de exploração que evitassem conflitos. A conferência
revelou{}-se hipócrita e ineficiente e a única resolução aparentemente
acatada por todos foi a de que cada nação deveria notificar as outras
sobre seus planos de colonização, inclusive com a descrição dos
territórios em que pretendia se instalar.  O mais surpreendente, no
entanto, nessa conferência foi o fato de terem todos concordado com a
propriedade pessoal do Congo pelo rei da Bélgica. Três meses depois o
parlamento belga ratificou essa decisão. Leopoldo \textsc{ii} da Bélgica foi
confirmado chefe do Estado fundado na África pela Associação Internacional do Congo, 
e a união entre o novo estado e a Bélgica se
daria em termos exclusivamente pessoais. O Congo permaneceu propriedade
particular de Leopoldo \textsc{ii} até sua morte, em 1908. Ele legou o Estado
africano, em  testamento, à Bélgica, em troca de um polpudo empréstimo
concedido pela Câmara Legislativa belga.


O Congo era gerido da Bélgica por um Administrador Geral. Havendo
muitas disputas e dificuldades, o país foi finalmente dividido em 15
distritos, cada qual dirigido por um Comissário que representava o
Administrador Geral. Diferentemente dos ingleses, que administravam
suas possessões por meio de chefes locais, através dos quais mantinham
sua autoridade, os administradores belgas optaram por reduzir o poder
dos chefes locais. Desta forma, as tribos nativas ficaram à mercê dos
funcionários belgas, ou mesmo de um nativo submisso, 
não lhes sendo garantido nenhum direito. Os brancos tinham sempre 
razão e sua palavra era  lei. Várias companhias
receberam concessão de Leopoldo para explorar o marfim do Congo, em
troca  do pagamento de taxas. O trabalho era de semi{}-escravidão, pois
o salário dos nativos era irrisório. A tirania e a brutalidade dos
exploradores fomentaram uma amargura e um ódio generalizados no povo
congolês. De acordo com Maurice N.~Hennessy, ``somente em uma concessão,
142 africanos foram mortos''.\footnote{ Maurice Hannessy. \textit{Congo}. 
London: Pall Mall, 1961. In Joseph Conrad.
\textit{Heart of Darkness}. Ed. by  Robert Kimbrough. New York:
W.W.~Norton \& Company, 1971, p.~90.}

Foi para esse Congo, cuja situação Conrad desconhecia completamente, que
o autor se deslocou. Para conseguir o comando do barco a vapor precisou
da interferência de uma prima, a quem tratava por tia, viúva de um
primo de sua mãe, que morava na Bélgica e frequentava a alta sociedade.
Foi esta mesma tia que depois intercedeu a seu favor, para que fosse
levado de volta à Europa, tendo em vista sua saúde debilitada pela
febre e disenteria. Já na África, mas ainda não no Congo, Conrad 
começa a perceber a situação difícil que viria a enfrentar, como se
deduz das cartas que escreve a amigos e à tia. Seu Diário, em dois
volumes, é também bastante revelador.

Apesar da sra.~Conrad negar ter o autor consultado o Diário ao
escrever \textit{No coração das trevas}, fato
reforçado por seu amigo Richard Curle que ainda afirma que Conrad não
guardava anotações, há muitas semelhanças entre o que é relatado no
Diário e o que aparece na novela, como a presença infernal dos
mosquitos, os gritos  e o rufar dos tambores ao longe, o barco sem
condições de navegar, os esqueletos amarrados em postes, assim como 
certos personagens, se é que assim podem ser chamados, como o Arlequim, o
homem branco doente que precisa ser levado na rede pelos nativos. Em
uma carta à tia, Conrad diz: ``Tudo me é repelente aqui. Os homens e as
coisas, mas especialmente os homens. E eu sou repelente para eles
também. Do gerente na África --- que se deu ao trabalho de dizer  a um
bom número de pessoas que eu o desagrado profundamente --- até o mecânico
mais humilde, todos têm o dom de atacar meus nervos; e consequentemente
eu não devo ser para eles tão agradável como poderia
ser''.\footnote{ ``Letters of Joseph Conrad to Marguerite Poradowska''. Translated and edited by
John A.~Gee and Paul J.~Sturm., 1940. In Joseph Conrad. \textit{Heart of Darkness}. 
Edited by Robert Kimbrough, op.~cit., p.~119.}

À constatação de que cometeu um grande erro indo para o Congo,
onde sequer cumpriu a missão para a qual tinha sido
contratado,\footnote{ ``Esta história [\textit{No coração das trevas}] e uma outra que não
está neste volume [\textit{An Outpost of Progress}] são todo o espólio
que eu trouxe do centro da África onde, realmente, eu não tinha nada a
fazer.'' Joseph Conrad. ``From `Author's Note'\,'' (1917), \textit{Youth}.
London: J.~M.~Dent \& Sons, 1921. In Joseph Conrad. \textit{Heart of
Darkness}. Ed. by Robert Kimbrough, op.~cit., p.~160.} 
justapõe{}-se uma outra, bem mais reveladora: ``Antes do Congo
eu era apenas um animal''.\footnote{ From Edward Garnett.``Introduction'', 
\textit{Letters from Conrad 1895---1924}, London, The Nonesuch Press, 1928. In Joseph Conrad.
\textit{Heart of Darkness}. Edited by Robert Kimbrough, op.~cit., p.~122.}
Para Albert J.~Guerard, referindo{}-se a \textit{No coração das trevas}, 
``a base autobiográfica da narrativa é bem conhecida e seu viés introspectivo, óbvio; esta é a mais longa jornada
de Conrad em direção a seu ser mais profundo. Mas é bom lembrar que
\textit{No coração das trevas} também trata de coisas mais superficiais: 
é uma narrativa de viagem vívida e sensível
e um comentário sobre a luta por pilhagem mais vil que desfigurou a
história da consciência humana e da exploração
geográfica''.\footnote{ Albert Guerard. \textit{Conrad the Novelist}, pp.~33---38, Mass., 1958,
The Fellows of Harvard College. In Joseph Conrad. \textit{Heart of
Darkness}. Edited by Robert Kimbrough, op.~cit., p.~122.}

G.~Jean{}-Aubry chega mesmo a dizer que ``A doença que ele
trouxe do Congo, ao limitar sua atividade física e confiná{}-lo ao
quarto durante meses, obrigou{}-o a refletir sobre si mesmo, a
relembrar sua vida que, embora ele tivesse apenas 35 anos, já era
extraordinariamente plena, e tentar avaliar o valor de suas lembranças
sob o ponto de vista humano e literário\ldots\ Pode{}-se dizer que a
África matou Conrad, o marinheiro, e fortaleceu Conrad, o
ficcionista''.\footnote{ G.~Jean{}-Aubry. \textit{Life and Letters}. London: William Heinemann,
1927. In Joseph Conrad. \textit{Heart of Darkness}. Ed.~by Robert Kimbrough, op.~cit., p.~125.}

\subsection{no coração das trevas}

A primeira obra de Conrad, \textit{Almayer's Folly}, foi terminada em
1894 e publicada em 1896. Em 1895, \textit{An Outcast of the Islands}
foi aceito para publicação.  Em 1897 ele completou \textit{The Nigger
of the `Narcissus'} e um volume de cinco contos a que deu o título de
\textit{Tales of Unrest}. Em  1898, escreveu e publicou \textit{Youth},
o primeiro conto em que aparece Marlow. No outono daquele ano iniciou
\textit{No coração das trevas}, publicado primeiro em capítulos, pela
Blackwood's Magazine, entre fevereiro e abril de 1899, e na forma de livro em 1902.

Para Albert J.~Guerard, não resta dúvida de que ``a estória não
é principalmente sobre Kurtz ou sobre a brutalidade dos funcionários
belgas e sim, sobre Marlow, seu narrador''.\footnote{ Albert J.~Guerard, op.~cit., p.~124.}
Já A.~C.~Ward refere{}-se a Marlow como o artifício mais
engenhoso inventado por Conrad, utilizado não apenas em \textit{Heart of Darkness}, 
mas também em \textit{Lord Jim} (1900), \textit{Youth}, e \textit{Chance}.\footnote{ A.C.~Ward.
\textit{Longman Companion to Twentieth Century Literature}. Essex: Longman, 3\textsuperscript{rd} ed.~1981.}
De acordo com Ward,  Marlow ``funciona como receptor e filtro
das evidências fornecidas por várias fontes, desta forma permitindo que
personagens e eventos sejam vistos de vários ângulos, permitindo ao leitor uma compreensão
maior e mais profunda da obra''.  O próprio Conrad refere{}-se à sua
criatura como se fosse um ser real, que tivesse entrado em sua vida por
acaso: ``Não houve plano nenhum. O homem Marlow e eu nos encontramos da
mesma forma casual com que se estabelecem relações em 
\textit{resorts}. Às vezes essas relações tornam{}-se autênticas
amizades, e foi isto que aconteceu conosco. Apesar de suas fortes
convicções e opiniões, ele não invade minha privacidade\ldots\ De todos os
meus personagens [\textit{people}, no original], ele é o único que
nunca me aborreceu. Trata{}-se de um homem muito discreto e compreensivo\ldots''.

\textit{No coração das trevas} permite várias
interpretações. O próprio autor, ao responder a uma carta de Barrett H.~Clark, 
declara: ``Primeiro gostaria de deixar bem clara uma proposição:
a de que raramente um trabalho artístico é limitado a um significado
único e exclusivo, e não necessariamente tende a uma conclusão
definitiva. E isto pela simples razão de que, quanto mais próximo da
arte, mais simbólico se torna\ldots\ Todas as grandes criações
literárias são simbólicas, e com isso ganham em complexidade, poder,
profundidade e beleza''.\footnote{ G.~Jean{}-Aubry. \textit{Life and Letters}, op.~cit., p.~154.}

Robert Kimbrough, no Prefácio de \textit{Heart of Darkness}, por ele
editado e já citado, assim se refere ao conto: ``\textit{Heart of
Darkness} tende a ser interior, sugestivamente analítico, e altamente
psicológico. Em suma, ele introduz um novo modo na ficção de Conrad: o
simbólico''. E lembra que o próprio autor tinha dúvidas sobre o
simbolismo com que havia envolvido Kurtz: ``O que eu claramente admito é
o erro de ter feito Kurtz simbólico demais ou mesmo apenas simbólico.
Sendo a estória principalmente um veículo para transmitir impressões
pessoais, dei rédea à minha preguiça mental e segui o caminho do menor esforço''.

A fortuna crítica de \textit{Heart of Darkness} evidencia o interesse
que o conto até hoje provoca, e as múltiplas leituras que sugere. A
nova edição da Norton, de 2005, sob a responsabilidade de Paul B.~Armstrong, 
em substituição a Robert Kimbrough, traz uma série de novos
artigos sobre a obra, abordando Impressionismo e Simbolismo, racismo, 
a jornada interior, as mulheres no conto, ou a mulher africana,
metafísica, canibalismo, modernismo e Vietnã, cultura popular, a visão
do novo historicismo, do pós{}-colonialismo. Traz ainda alguns artigos
que já constavam na edição de Robert Kimbrough, como  os de Virginia
Woolf, Albert Guerard, Ford Madox Ford e Chinua Achebe.

Não há dúvida de que o autor atingiu o que queria: ``Aquele tema
sombrio precisava de uma ressonância sinistra, uma tonalidade própria,
uma vibração contínua que, eu esperava, ficaria suspensa no ar e
permaneceria nos ouvidos depois do toque da última
nota''.\footnote{ Joseph Conrad. ``Author's Note'' (1917), \textit{Youth}, 
1921. In Joseph Conrad. \textit{Heart of Darkness}. Ed.~by Robert Kimbrough, op.~cit., p.~160.}
Ao terminar sua leitura, ressoam em nossos ouvidos as últimas
palavras de Kurtz: ``O horror! O  horror!''. Referia{}-se ele à África? A
si mesmo? Ao homem europeu? Perguntamo{}-nos também  se, ao contrário
de Rousseau, não estaria Hobbes com a razão, e  se não é a civilização,
tão criticada nos tempos atuais, que atuaria como um fator de controle
sobre o bicho que parece se esconder no interior do ser humano.
Indagamo{}-nos ainda se a selva não deprava o homem ``civilizado'', tema 
tão bem abordado por \mbox{William} Golding em \textit{Senhor das moscas} 
(\textit{Lord of the Flies}, no original), e interrogamo{}-nos sobre o significado do post{}-scriptum
de Kurtz ao seu texto sobre a ``Supressão dos Costumes Selvagens'':
``Exterminem  todos os brutos!''.  Enfim, Conrad leva{}-nos a refletir e a
constatar que ao continente negro contrapõem{}-se as trevas inerentes
ao ser humano, sua complexidade, sua imensa capacidade para o bem e para o mal.




