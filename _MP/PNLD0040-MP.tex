\documentclass[12pt]{extarticle}
\usepackage{manualdoprofessor}
\usepackage{fichatecnica}
\usepackage{lipsum,media9,graficos}
\usepackage[justification=raggedright]{caption}
\usepackage[one]{bncc}
\usepackage[blooks]{../edlab}



\begin{document}

\baselineskip=1.20\baselineskip\par


\newcommand{\AutorLivro}{August Strindberg}
\newcommand{\TituloLivro}{Sagas}
\newcommand{\Tema}{Ficção, mistério e fantasia}
\newcommand{\Genero}{Conto, crônica e novela}
\newcommand{\imagemCapa}{./images/PNLD0040-01.png}
\newcommand{\issnppub}{---}
\newcommand{\issnepub}{---}
% \newcommand{\fichacatalografica}{PNLD0040-00.png}
\newcommand{\colaborador}{{Bruno Gradella e Vicente Castro}}


\title{\TituloLivro}
\author{\AutorLivro}
\def\authornotes{\colaborador}

\date{}
\maketitle

\begin{abstract}\addcontentsline{toc}{section}{Carta ao professor}
Este Manual tem como objetivo fornecer subsídios para o trabalho com a
obra literária \emph{Sagas}, obra de autoria de August Strindberg.

Johan August Strindberg (Estocolmo, 1849--\textit{id.}, 1912) 
foi um escritor, dramaturgo, pintor e fotógrafo sueco. Já com o devido
reconhecimento do público e da crítica, a publicação, em 1882, de 
\textit{Det Nya Riket} (O novo reino), obra repleta de críticas às 
instituições, rende"-lhe tantas críticas que o autor vê"-se obrigado
a deixar seu país natal. No exterior, escreve uma parcela significativa 
de sua obra, como o romance \textit{Inferno}, ao mesmo tempo em que 
luta contra graves problemas psicológicos.

Escritas para sua filha recém"-nascida, as \textit{Sagas} reúnem contos
inspirados por acontecimentos na vida do autor sempre com uma preocupação 
literária. Encontraremos narrativas abordando a história da Suécia, fábulas 
regionais, acontecimentos contemporâneos, criaturas mitológicas e lições de 
moral destinadas à filha. 

Em ``Nos dias de verão'' uma pomba canta as alegrias do reino de Deus, 
pinheiros e rouxinóis vão à ajuda de uma desamparada mãe com sua filha. 
Já em ``A grande peneira'', as principais personagens são enguias, percas, 
bacalhaus e outros animais marinhos. Já em ``Os segredos do celeiro de fumo''
e ``Jubal sem identidade'', de caráter mais moral, veremos histórias sobre 
artistas desgraçados pela sua arrogância, quando deixam o talento subir à cabeça.

Esperamos que as indicações propostas aqui sejam muito úteis no trabalho em
sala de aula! 


\end{abstract}


\tableofcontents


\section{Introdução}

O autor sueco Johan August Strindberg, além de escritor, foi dramaturgo, pintor e fotógrafo.
O escritor, ironicamente, nunca deixou de duvidar de tudo e principalmente de si mesmo.
Como ele próprio se descrevia, numa de suas citações mais conhecidas: ``eu sou um homem maldito, o que sei são muitas artes''.
\SideImage{Foto de Strindberg após seu aniversário de 50 anos. (Autor desconhecido; Domínio Público)}{PNLD0040-03.png}


Para estreitar o convívio com a filha, Strindberg elabora uma série de contos, não só para divertir a pequena filha, mas também mostrar ensinamentos da ética, da história, da política e do convívio social. Produzida nesse período, \textit{Sagas} reúne contos que vão de breves peças morais a narrativas oníricas.

Embora tenha escrito o livro para sua filha que tinha apenas um ano, Strindberg não deixou de se preocupar com seu aspecto literário, o que faz dessa obra uma leitura fascinante para todas as idades.
Há um lugar para o herói mínimo, o herói obscuro, o herói negativo e até para o anti"-herói.
Seu estilo passa do realismo ao expressionismo. 
Sagas, neste caso, não são apenas os feitos épicos dos grandes heróis, nem as longas narrativas do folclore escandinavo.
Passando por temas históricos, humorísticos e heroicos, trazendo um tom fabuloso, ou um tom direto, grande parte dessas histórias foram inspiradas por acontecimentos na vida do autor.
Além disso, são igualmente fábulas por terem seu conteúdo moral, exemplificativo, na personagem que serve de modelo para a formação do caráter.


Após concluir seus estudos, dedicou"-se à carreira de professor, ao mesmo tempo em que estudou medicina.
Mais tarde tenta lançar"-se como ator, mas em 1870 vai estudar na Universidade de Uppsala, 
onde começa a escrever. Dois anos mais tarde, interrompe os estudos por razões financeiras.
Em seguida, o autor passa a trabalhar em um jornal local e depois na Biblioteca Nacional da Suécia.

Strindberg teve uma vida turbulenta. A questão mais polêmica é a história de sua ``loucura''. Jávier 
Blánquez diz que ``geralmente 'se fala do dramaturgo como um ‘demente’, como alguém que, em determinado momento de sua vida, perdeu o controle de si'', tornando"-se alheio à realidade. 
Durante uma crise aguda foi levado a escrever \textit{Inferno}.

Suas crises eram recorrentes e o desestabilizavam, mas isso não era o suficiente para classificá"-lo 
segundo alguma doença psiquiátrica, de acordo com alguns especialistas. 
Por certo, o autor era melancólico ou padecia de depressão.

Em relação ao seu país de origem, a Suécia, Strindberg foi tão 
crítico quanto pôde ao expor suas contradições.
Provocava sem piedade os seus compatriotas, tornando"-se problemático aos olhos de muitos e incômodo para o gosto convencional.

O autor foi também pioneiro em trazer para o teatro as coisas que o faziam sofrer. Poderíamos chamar esse teatro de ``sacrificial'', por acentuar o valor da mortificação e do sacrifício no cristianismo. 
O fato é que no seu teatro também dito ``onírico'', encontramos continuadamente uma longa sequência 
de vícios e obsessões. Esta foi a última fase do autor, que revelou Strindberg mais anti"-dogmático e dando asas à fantasia. Várias peças de câmara são deste momento: por exemplo ``A tempestade'', ``Pelicano'', ``Sonata dos espectros'' e ``O sonho''.


Vale lembrar por fim que ainda que o autor foi rechaçado pela Academia sueca, responsável pelo Prêmio Nobel.



\section{Propostas de Atividades I}


\subsection{Pré"-leitura}

\paragraph{Tema} Contos de fadas.

\paragraph{Conteúdo} Pesquisa sobre os contos de fadas de tradição europeia e sobre a mitologia de matriz nórdico"-escandinava. Criação de uma história em quadrinhos inspirada em heróis e divindades escandinavos.
\BNCC{EM13LP12}

\paragraph{Objetivo} Introduzir aos estudantes alguns aspectos da mitologia de matriz nórdico"-escandinava para possibilitar uma maior compreensão das origens de Strindberg.

\paragraph{Justificativa} Essa atividade possibilita uma introdução ao gênero conto e também incentiva os alunos a criarem suas próprias histórias.

\paragraph{Metodologia}

\begin{enumerate}

\item Introduza o tema dos contos de fadas com
os alunos. Após uma breve explicação, sugira que os estudantes pesquisem
contos de fadas de tradição europeia. Cabe ao professor orientar o aluno a estar atento a elementos
como tramas comuns, itens e elementos do ambiente que se repetem.

\item Feito esse levantamento, observe se encontraram um material acerca da mitologia germânica, em especial os de tradição de matriz
nórdico"-escandinava. Caso tenham tido dificuldade para encontrar contos
dessa região, o professor pode sugerir alguns títulos. Importante frisar, mesmo contos
que não sejam utilizados nesse exercício poderão servir em atividades
futuras, de modo que não devem ser descartados. 

\item Peça aos alunos que escolham um dos contos e que estejam atentos quanto aos
principais heróis e divindades presentes. 

\SideImage{Quadro ``Thor luta com os gigantes'', pintado por Mårten Eskil Winge em 1872. (Mårten Eskil Winge; Domínio Público)}{PNLD0040-06.png}


\SideImage{Representação de Freir, divindade nórdica da paz e da fertilidade, das colheitas e dos casamentos (Jacques Reich; Domínio Público)}{PNLD0040-10.png}


\item  A partir desse material, cada aluno pode produzir uma breve história em
quadrinhos, utilizando as paisagens, os heróis, os monstros, os itens
mágicos, entre outros, para forjar sua própria narrativa.
\BNCC{EM13LP54}


\end{enumerate}


\Image{Ilustração de 1850 representa Odim, uma das mais importantes divindades nórdicas. (Otto Henrik Wallgren; Domínio Público)}{PNLD0040-08.png}


\Image{Desenho a lápis representa Odim, o deus da guerra segundo a mitologia escandinava. (Victor Villalobos; CC-BY-SA 4.0)}{PNLD0040-09.png}

\paragraph{Tempo estimado} Três aulas de 50 minutos.


\subsection{Leitura}

\paragraph{Tema} \emph{Sagas} e os movimentos artísticos do século \textsc{xix}.

\paragraph{Conteúdo} Debate a partir da leitura do conto ``Nos dias de verão''.

\paragraph{Objetivo} Incentivar o paralelo com os contos pesquisados na atividade anterior
e estimular a relação entre os textos e os principais movimentos artísticos do século \textsc{xix}.
\BNCC{EM13LP50}

\paragraph{Justificativa} É interessante promover uma associação entre a leitura de \emph{Sagas} 
e a pesquisa feita pelos alunos na atividade de pré"-leitura, com o intuito de aprimorar
o conhecimento dos estudantes acerca de múltiplos movimentos literários.

\paragraph{Metodologia}

\begin{enumerate}

\item  Retome os contos levantados pelos alunos na atividade de pré"-leitura e peça que
façam um quadro comparativo entre esses contos e o primeiro conto da obra,
intitulado ``Nos dias de verão''. Incentive o
paralelo entre esses contos, observando as
similaridades e diferenças nas narrativas entre eles.
\BNCC{EM13LP02}

\item A partir dessa comparação, é interessante que o professor apresente um conteúdo
acerca dos principais movimentos artísticos"-literários do século \textsc{xix}.

\item Para auxiliar no desenvolvimento da
atividade, é aconselhado que a sala de aula seja dividida em grupos e
que os alunos busquem informações gerais acerca desses movimentos literários.

\item Feito
isso, cada grupo produzirá um material de consulta, como um site ou publicação
enciclopédica, que relacione os movimentos artístico"-literários
com os contos pesquisados. Esse material poderá ser publicado no site do colégio ou 
em uma rede social para artigos como o \url{medium.com}.
\BNCC{EM13LP54}
\end{enumerate}

\paragraph{Tempo estimado} Quatro aulas de 50 minutos

\subsection{Pós"-leitura}

\paragraph{Tema} Sagas brasileiras

\paragraph{Conteúdo} Exercício de escrita pautada no folclore brasileiro. 

\paragraph{Objetivo} Incentivar que os estudantes se debrucem acerca do imaginário
do folclore brasileiro e exercitem a escrita criativa.
\BNCC{EM13LP52}

\paragraph{Justificativa} Após a leitura completa das \emph{Sagas}, somadas ao
conhecimento obtido nas atividades anteriormente desenvolvidas, pode ser interessante 
que os estudantes possam explorar um lado criativo e criar as suas próprias sagas.

\paragraph{Metodologia}

\begin{enumerate}

\item Proponha
um exercício de escrita criativa para a elaboração de um outro livro de
sagas, mas dessa vez, pautado no folclore brasileiro.
\BNCC{EM13LGG103}

\item Para isso, ofereça
trechos de obras que trabalhem o folclórico, o imaginário, o fantástico
dentro do universo cultural brasileiro. Também é interessante trazer
músicas, filmes, vídeos, desenhos, quadrinhos etc. que tratem do tema.

\item Instrua aos alunos a olharem para o livro de Strindberg como modelo para
a estrutura do texto que será escrito, mas que se valham dos outros
elementos da cultura nacional, trazidos para essa atividade, para o
conteúdo do texto. Ao final, os contos podem ser compilados em um livro,
que pode ser impresso ou disponibilizado online para os alunos.

\end{enumerate}

\paragraph{Tempo estimado} Duas aulas de 50 minutos.



\section{Propostas de Atividades II}

\subsection{Pré"-leitura}

\paragraph{Tema} A Suécia do século \textsc{xix}.

\paragraph{Conteúdo} Pesquisa em grupo e apresentação de seminário acerca do contexto histórico e cultural em que Strindberg estava inserido. 

\paragraph{Objetivo} Ampliar o conhecimento dos estudantes sobre o universo da obra de Strindberg.
\BNCC{EM13LP01}

\paragraph{Justificativa} Essa atividade tem como pano de fundo preparar os alunos para a leitura de \emph{Sagas}
a partir de uma abordagem interdisciplinar.

\paragraph{Metodologia}

\begin{enumerate}

\item Para essa atividade, recomenda"-se que, junto aos professores de Humanidades, seja
realizada uma pesquisa aprofundada sobre o contexto histórico"-cultural
que o autor estava inserido.

\item Recomenda"-se a divisão da
sala em grupos, que deverão pesquisar diferentes aspectos dos elementos que
possivelmente compuseram as referências de Strindberg. Para tanto,
recomenda"-se que um grupo pesquise aspectos geográficos da Suécia.
Outro, fie"-se às questões históricas envolvendo o país. Um outro, busque
referências culturais e folclóricas e, por fim, um último grupo que busque
informações acerca do contexto cultural e literário europeu da época de
Strindberg.

\item Feito isso, convida"-se à realização de um seminário onde os
alunos exporão suas descobertas. Durante as exposições, os alunos
deverão, também, anotar os detalhes do divulgado pelos outros grupos
para, ao fim, a sala montar um texto informativo que aglutine todos os
elementos obtidos.
\BNCC{EM13LP16}

\end{enumerate}

\paragraph{Tempo estimado} Duas aulas de 50 minutos.


\subsection{Leitura}

\paragraph{Tema} A referências da mitologia escandinava em \emph{Sagas}.

 \paragraph{Conteúdo} Estudo interdisciplinar do conto ``Os elmos dourados de Ålleberg'',
 com o auxílio de professores da área de Ciências Humanas.

\paragraph{Objetivo} Possibilitar os estudantes a aprofundar a leitura da obra através do
estudo de textos mitológicos e documentos históricos europeus.

\paragraph{Justificativa} Nessa atividade, ficará evidente que o termo saga é muito
próprio da cultura teuto"-escandinava. Entretanto, outros povos da Europa
também detinham narrativas com estruturas análogas e mesmo histórias e
personagens similares. É
interessante a comparação entre as origens de distintas narrativas, com 
o intuito de apontar as especificidades das referências mitológicas de \emph{Sagas}.
\BNCC{EM13LP50}

\paragraph{Metodologia}

\begin{enumerate}

\item A partir do conto ``Os elmos dourados de Ålleberg'',
os professores de Humanidades devem propor um debate com a sala
acerca das referências da mitologia nórdico"-escandinava presentes
na obra.

\item É interessante explorar o conceito de saga a
partir de documentos históricos provenientes de diferentes nações
europeias. Proponha aos alunos um estudo
comparativo de textos folclóricos e mitológicos dos povos europeus. 
Podem ser comparadas narrativas de origem teuto"-escandinavas,
celtas, greco"-latinas, eslavas, indo"-arianas, entre outras.

\item Por fim, os alunos devem redigir um pequeno texto
sobre as semelhanças e diferenças que encontraram no conto ``Os elmos dourados de Ålleberg''
e nos textos mitológicos estudados.
\BNCC{EM13LP53}

\end{enumerate}


\paragraph{Tempo estimado} Duas aulas de 50 minutos.

 
\subsection{Pós"-leitura}

\paragraph{Tema} A Península da Escandinávia.

\paragraph{Conteúdo} Pesquisa sobre as características naturais da Península da
Escandinávia.

\paragraph{Objetivo} Propor uma abordagem interdisciplinar da ambientação de \emph{Sagas},
com conteúdo da área de Geografia e Biologia.

\paragraph{Justificativa} Dentro das histórias do livro, são apresentados uma série de elementos da
natureza. É interessante aproveitar essas descrições para realizar um estudo 
interdisciplinar da obra, junto dos professores de Ciências da Natureza. 

\paragraph{Metodologia}

\begin{enumerate}

\item Com o auxílio dos professores de Ciências da Natureza,
deve ser feita uma pesquisa sobre os aspectos geológicos, biológicos e morfoclimáticos
da Península da Escandinávia. Os aspectos estudados devem ser listados
e separados em temas.
\BNCC{EM13LP30}

\item Em seguida, a sala deve ser dividida em grupos e cada um deve ficar
responsável por um dos temas estudados. Com o material pesquisado,
devem montar uma apresentação para os demais colegas.

\item O resultado dessa investigação pode ser trabalhado por meio de 
uma apresentação para o restante da turma.
\BNCC{EM13LP16}

\end{enumerate}

\paragraph{Tempo estimado} Uma aula de 50 minutos.


\section{Aprofundamento}

Nesta seção, desenvolvemos um trabalho de aprofundamento que, em diálogo
com a formação continuada de professores, oferece subsídios para a
abordagem do texto literário. A leitura em sala de aula de \emph{Sagas}
pode ser enriquecida pelo aprofundamento no universo literário em que a
obra está inserida.

\subsection{A obra}

\emph{Sagas} é uma produção que reúne diversos contos do sueco August
Strindberg. A obra foi concebida como uma espécie de presente e guia
para a filha do autor que, à época, tinha um ano de idade. Desse modo, a
obra é composta pelo elemento do fantástico, do mágico, entre outros
fundamentos que compõem a temática infantil, no caso de \emph{Sagas}, muito
inspirados pelo folclore nórdico.

Entretanto, o aspecto ``guia'' da obra se deve ao fato de que muitos
contos possuem caráter moralizantes, heroicos, fabulosos, carregando
consigo um tom didático sobre ocasiões comuns à vida.

\subsection{Da vida do autor para o livro que prepara para a vida}

Quando o livro foi produzido, Strindberg já se encontrava separado de
sua esposa, de modo que as Sagas foram uma forma de o autor se fazer
presente e, de certo modo, contar suas histórias, posto que vários
contos são inspirados em suas experiências, e ao mesmo tempo, poder
exercer o papel de educar a filha, mesmo à distância, por meio das
morais advindas no final de cada uma das histórias.


\Image{Retrato de Strindberg com 25 anos. (Studio de Math. S. Hansen; Domínio Público)}{PNLD0040-04.png}


A ideia era de se estreitar o convívio, ao mesmo tempo em que divertia a
filha e a introduzia em ensinamentos primários da ética, da história, da
política e do convívio social, de sorte que os heróis e personagens
assumem traços arquetípicos, sendo o livro um bom guia para o
enfrentamento do cotidiano para aqueles que têm olhares atentos.

Desse modo, vê"-se na obra a clara assinatura de Strindberg, que escreve
de maneira simples, porém sem ser simplista. Muito pelo contrário, é no
tom direto e até simplório que se encontra a grande profundidade de sua
mensagem.

Com isso, nota"-se que, embora tenha escrito para sua filha de apenas um
ano de idade, Strindberg não deixou de se preocupar com o aspecto
literário, o que faz dessa obra uma leitura fascinante para todas as
idades.


\Image{Retrato de Strindberg pintado por Richard Bergh em 1905 (Richard Bergh; Domínio Público)}{PNLD0040-05.png}


O autor da trilogia dramática \emph{O Caminho para Damasco}, peça
considerada marco inicial do teatro expressionista do século \textsc{xx},
imprimiu os seus dotes também na presente obra, dotando"-a de rico
conteúdo simbólico e matizes oníricas.

Essa obra que tem roupagem de literatura infantil, mas revela"-se muito
além disto, como uma lâmpada maravilhosa literária, é composta de contos
que vão de breves peças morais a narrativas oníricas.

A influência do folclore nórdico é patente, a começar pelo título ``Sagas''.
Entretanto, neste caso, não são apenas os feitos épicos dos grandes
heróis, nem as longas narrativas escandinavas. Há temas históricos,
humorísticos, heroicos, em tom fabuloso, ou direto, grande parte dessas
histórias foram inspiradas por acontecimentos na vida do autor.

Fabuloso por terem seu conteúdo moral, exemplificativo, no personagem
que serve de modelo para a formação do caráter.

E Strindberg guardava uma estreita relação com a natureza de seu país
natal. O que é visível em grande parte de suas obras. Em certa medida, é
possível intuir que o autor sentia que o povo sueco era quase como um
produto de sua natureza. Então, retomar temas antigos, das épocas dos
vikings, fazia sentido no caráter formador de sua obra. Afinal seria
recuperar as origens do povo e do meio que sua filha encontraria nos
anos vindouros.


\Image{Escultura Viking localizada em Largs, Reino Unido. (Dave Hitchborne; CC-BY-SA 2.0)}{PNLD0040-07.png}


Strindberg dá voz própria à natureza e escuta seus clamores: o mar e a
terra, o sol e a neve, todos são personagens em sua história,
convivendo, observando e descrevendo seu constante trajeto de
dificuldades e vitórias, como se compusessem um coro elementar para a
narrativa.

No mesmo sentido, não há personagens menores, há personagens que
aparecem menos, mas cada qual traz uma função narrativa importante, para
amalgamar ou provocar a narrativa, como uma folha de uma especiaria
rara, colocada ao final do caldo para juntar e tornar pujantes todos os
outros temperos e ingredientes já misturados.

\subsection{Obra polifônica}

É notória a polifonia nas \emph{Sagas} de Strindberg. É claro que Strindberg
tinha uma opinião formada sobre a vida e todos seus elementos, isso se
torna claro ao se observar sua trajetória biográfica. Entretanto, no
microcosmo que retrata, nunca se coloca de maneira direta.

Restava na pantomima imaginária, observando a atuação dos entes de sua
criação no tecido narrativo que produzira. Entendia que, ainda que
detivesse um \emph{telos}, ele não seria atingido por uma imposição
abrupta. Ao invés disso, convida o leitor a percorrer o caminho para
encontrar a moral por ele entendida e desejada, quase como a narrativa
do herói que, após agruras, encontra o tesouro.

\subsection{Por que ler as \emph{Sagas}?}

Embora aborde cenários infantis, o autor não deixa passar a oportunidade
de invocar questões sociais e políticas, provocando a reflexão em seus
leitores acerca dos mesmos, ainda que não de forma escancarada.

Desafia a glória de governantes, e a glória dos cientistas, pondo em
xeque se o conhecimento realmente conduz a uma vida mais plena e
agradável.

Aborda questões de convivência e cria um conteúdo exemplificativo para
narrar a boa moral e o bom caráter.

É curioso que, apesar da linguagem infantil de seus contos, a
complexidade dos temas tratados e o decorrer da história indicam que, a
parte a roupagem inicial, são textos voltados a todas as idades e que,
apesar da roupagem do fantástico, narra, em verdade, situações
cotidianas.

\subsection{Atividades para o aprofundamento da pesquisa}

No Ensino Médio, da mesma forma que no Ensino Fundamental, a \textsc{bncc}
organiza o trabalho com as práticas de linguagem em cinco \textbf{campos
de atuação social}. São eles: campo da vida pessoal, campo da vida
pública, campo jornalístico"-midiático, campo artístico"-literário e campo
das práticas de estudo e pesquisa.

De acordo com essa divisão, propomos na sequência um trabalho
interdiscursivo e intertextual com a obra \emph{Sagas.}



\subsection{Entre o livro e um filme}


Como visto, Strindberg produziu \emph{Sagas} como uma espécie de
presente"-guia para sua filha, ainda em tenra idade quando da escrita
da obra. Outro caso similar é o do helenista Jean"-Pierre Vernant, que
escreveu o trabalho \emph{O Universo, Os Deuses e os Homens}, obra que
abarca narrativas da mitologia grega em um linguajar simplificado,
após passar finais de semana cuidando de sua neta.

Agora, convide o aluno a se imaginar em uma situação como essa. Peça
para que ele selecione um filme, um livro, uma peça que seja de seu
agrado e que, feito isso, transcreva"-a de uma forma que seja palatável
para uma criança muito jovem. Para isso, sugere"-se a seleção de trechos
da obra citada e comparação com trechos das obras de onde Vernant buscou
a inspiração, como em Hesíodo ou Homero.
\BNCC{EM13LP20}
    
\subsection{O tom moralizante em narrativas}

Como visto, Strindberg se valeu de sua obra, uma coletânea de
narrativas infantis, para propagar ideais morais. Desse modo, é
possível se valer de histórias escritas de maneira simples para
divulgar ideias úteis e saudáveis à sociedade. Um local onde vemos
isso muito comumente é nas fábulas. Entretanto, por vezes adotando um
tom crítico, encontramos uma iniciativa de perspectiva similar nas
tirinhas.

Portanto, nessa atividade, o aluno deverá valer"-se de um dos contos de
Strindberg, onde consiga enxergar o tom moralizante, e produzir uma
tirinha a respeito.

Sugere"-se, então, na etapa inicial, o planejamento da produções, divida
os alunos em grupos. Cada grupo irá selecionar os materiais artísticos a
serem utilizados, uma vez que os gêneros textuais em questão fazem uso
da linguagem não verbal.
\BNCC{EM13LP11}

Recomenda"-se utilizar o roteiro abaixo para planejar a atividade:

\begin{itemize}
\item
Construam pequenos parágrafos narrando a sequência dos principais
acontecimentos.
\item
Quem serão as personagens do texto? Caracterizem cada uma delas e, se
necessário, façam pequenos esboços dos diálogos, empregando,
inicialmente, dois"-pontos e travessões.
\item
Em que cenário ocorrerão os fatos? Descrevam os principais elementos
do espaço e verifiquem se ele contribui para a ambientação e para os
efeitos de humor ou reflexão propostos pelo texto.
\end{itemize}

Ao final, as produções podem ser compiladas e divulgadas no site do
colégio e também compor o acervo da biblioteca da sua escola.
Sugere-se ainda publicar o material na rede social \url{medium.com}.


\subsubsection{Os limites do marketing}


Ao longo da Era Industrial, a propaganda se tornou peça chave na
atividade de qualquer indústria. Afinal, por meio dela, há a
divulgação e, em certa medida, uma pretensa ratificação da qualidade
do produto. Outrossim, não é raro vermos propagandas abusivas, as
quais são postas em xeque pelo Ministério Público e de órgãos
governamentais. 
\BNCC{EM13LGG202}

As propagandas também foram relativizadas ao longo da
história. Hoje, não se tem mais anúncios de cigarro e bebida na
televisão, como ocorria outrora. No mesmo sentido, não se vê mais os
famigerados ``cigarros de chocolate'', uma sobremesa infantil que, de
certo modo, induzia ao fumo. Entretanto, há hoje um bombardeamento das
pessoas com propagandas em sites e aplicativos. Um local que gera
muita preocupação para a veiculação dos anúncios é nos jogos de
celular. Isso posto, abra uma roda de discussão com a turma a respeito
da temática. Procure abordar as questões dos limites, como até onde se
tem um marketing agressivo e quando ele passa a ser abusivo? Deve ser
intensificada a censura nos meios de comunicação? Como as pessoas
devem se preparar para não serem vítimas de propaganda enganosa e
quais os meios legais para se proteger caso o sejam.

\subsubsection{A diferença entre fábulas, contos de fadas, e
textos moralizantes.}


Na obra \emph{Sagas}, Strindberg vale"-se de diversos formatos
literários para escrever suas histórias. Bem verdade, há uma mescla de
elementos de gêneros e estilos diversos. Entretanto, um estudioso
atento, perceberia elementos das fábulas, de contos de fadas, e mesmo
textos moralizantes em forma de parábola ao longo das \emph{Sagas}.
Nesta atividade sugere"-se que os alunos façam uma pesquisa sobre esses
gêneros, ficando a critério de cada um a escolha do que pesquisar. Com
a informação obtida, cada aluno deverá escrever um conto, fábula, ou
parábola, no formato de história em quadrinhos. Para recursos
narrativos é possível disponibilizar um material de apoio, mas,
aconselha"-se que os alunos sejam direcionados para tentar desenvolver
seu texto a partir dos moldes do autor estudado.
\BNCC{EM13LP49}

Dentro das histórias, são apresentados uma série de elementos da
natureza, os quais podem até causar certa impressão ao leitor num
primeiro momento. Diante disso, a atividade sugere, juntamente com
auxílio do professor de humanidades e de ciências da natureza, uma
pesquisa sobre as características naturais da Península da
Escandinávia, seus aspectos geológicos, biológicos e morfoclimáticos.
O resultado dessa investigação pode ser trabalhado por meio de um
artigo, ou por meio de uma apresentação para os demais colegas,
devendo, nesse caso, haver uma separação prévia da sala em grupos,
onde cada qual atuará com um tema.

\section{Sugestões de referências complementares}\label{sugestoes}

\subsection{Filmes e séries}

\begin{itemize}

\item\textit{Beowulf \& Grendel}. Dirigido por Sturla Gunnarsson. (Canadá, Islândia e Reino Unido, 2005).

Filmado nas encantadoras paisagens da Islândia, o filme oferece uma viagem ao universo dos 
antigos povos escandinavos a partir desta adaptação do poema nórdico ``Beowulf''.

\item\textit{A Fonte da Donzela}. Dirigido por Ingmar Bergman. (Suécia, 1960).

Ambientado no século \textit{xiv} do norte da Europa, o filme é uma amostra da sociedade
que vivia entre os valores tradicionais encontrados nas sagas nórdicas e o cristianismo.

\item\textit{Os Nibelungos -- A Morte de Siegfried}. Dirigido por Fritz Lang. (Alemanha, 1924).

Filme expressionista inspirado em lendas escandinavas medievais sobre criaturas que habitam o 
nevoeiro e o subsolo, chamadas de Nibelungos. O filme pode ser encontrado na íntegra 
e remasterizado no \href{https://www.youtube.com/watch?v=HH1BjSZYJ-w}{YouTube}.

\item\textit{Vikings}. Escrito e criado por Michael Hirst. (Canadá e Irlanda, 2013"-2020).

Série televisiva de ficção histórica baseada nas sagas de Ragnar Lodbrok. A série retrata 
Ragnar como um fazendeiro que alcança a fama por seus ataques bem"-sucedidos na Inglaterra e, 
mais tarde, se torna um rei escandinavo, com o apoio de sua família e colegas guerreiros. 
É uma boa sugestão aos que sejam afeitos ao formato das séries e queiram adentrar 
o universo dos povos nórdicos antigos.

\end{itemize}

\subsection{Sites}

\textbf{Museu Nórdico}. Estocolmo, Suécia.

Por meio da plataforma \href{https://artsandculture.google.com/partner/nordiska-museet}{Google Arts and Culture} é possível visitar algumas obras deste 
museu que fica na capital sueca e que reúne referências fundamentais da cultura nórdica.

\textbf{Museu Strindberg} 

Localizado em Estocolmo, na antiga casa do autor. O site conta com fotos e arquivos da vida e obra de Strindberg.
\url{https://www.strindbergsmuseet.se/english/}

\subsection{Vídeos}

\begin{itemize}
\item \href{https://www.youtube.com/channel/UCyi7TQJOK1kyIU6e4pRJrUQ}{Canal do Youtube do NEVE: Núcleo de Estudos Vikings e Escandinavos}

O grupo interinstitucional NEVE tem por principal objetivo o estudo e a divulgação da História e cultura da Escandinávia Medieval e em especial da Era Viking.

\item \href{https://www.youtube.com/watch?v=dxv7rPx8A5Q}{Vídeo ``Vida de August Strindberg''}

Filmagens da antiga casa de Strindberg, em Estocolmo, atual Museu Strindberg, e também ao Teatro Íntimo, fundado em 1907 por Strindberg.

\end{itemize}


\section{Bibliografia comentada}

\begin{itemize}
\item\textsc{wallin}, Claudia. \textbf{Suécia: um país sem excelências e mordomias.}
São Paulo: Geração Editorial, 2014.

A autora prova que existem políticos que não têm mordomias, que não
aumentam seu próprio salário, que usam transporte público e não estão na
vida pública para fazer fortuna. Um sistema apoiado em três pilares:
transparência, escolaridade e igualdade.

\item\textsc{carpeaux}, Otto Maria\textbf{. A história Concisa da Literatura Alemã.}
Alphaville: Faro Editorial, 2013.

O autor faz uma síntese dos grandes momentos, livros e autores da
literatura alemã, e conta com uma avaliação crítica de sua importância
para a cultura e o desenvolvimento do país e sua influência nos
principais movimentos culturais do mundo contemporâneo.

\item\textsc{carpeaux}, Otto Maria\textbf{. A história Literatura Ocidental.} São
Paulo: Leya, 2019.

Da literatura grega à contemporânea, Carpeaux analisa e critica as obras
com seu arcabouço teórico. Seu conhecimento nos leva ao encontro dos
mais importantes autores da literatura ocidental.

\item\textsc{larsson}, Stieg. \textbf{Os homens que não amavam as mulheres.} São
Paulo: Companhia das Letras, 2010.

Os homens que não amavam as mulheres é uma fascinante e assustadora
aventura vivida por um veterano jornalista e uma jovem e genial hacker
cujo comportamento social beira o autismo.

\item\textsc{lindrgen}, Astrid. \textbf{Píppi Meialonga}. São Paulo: Companhia das
Letras, 2001.

Píppi é uma menina de nove anos incrivelmente forte. Não tem pai nem mãe
e mora sozinha. Seus companheiros são um cavalo e um macaquinho e a
garota faz suas roupas, enfrenta valentões e sempre possui uma resposta
na ponta da língua, com grande confiança em si.


\end{itemize}

\end{document}

