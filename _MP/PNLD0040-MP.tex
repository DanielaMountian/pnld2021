\documentclass[12pt]{extarticle}
\usepackage{manualdoprofessor}
\usepackage{fichatecnica}
\usepackage{lipsum,media9,graficos}
\usepackage[justification=raggedright]{caption}
\usepackage{bncc}
\usepackage[blooks]{../edlab}



\begin{document}


\newcommand{\AutorLivro}{Strindberg}
\newcommand{\TituloLivro}{Sagas}
\newcommand{\Tema}{Ficção, mistério e fantasia}
\newcommand{\Genero}{Conto, crônica e novela}
\newcommand{\imagemCapa}{./images/PNLD0040-01.png}
\newcommand{\issnppub}{---}
\newcommand{\issnepub}{---}
% \newcommand{\fichacatalografica}{PNLD0040-00.png}
\newcommand{\colaborador}{\textbf{Bruno Gradella e Vicente Castro} é uma pessoa incrível e vai fazer um bom serviço.}


\title{\TituloLivro}
\author{\AutorLivro}
\def\authornotes{\colaborador}

\date{}
\maketitle

\begin{abstract}
Este Manual tem como objetivo fornecer subsídios para o trabalho com a
obra literária \emph{Sagas}, obra de autoria de August Strindberg.
\end{abstract}

\tableofcontents




\section{Atividades 1}

%\BNCC{EM13LP26}

\subsection{Pré-leitura}

%\BNCC{EM13LGG302}
%\BNCC{EM13LGG704}
%\BNCC{EM13LP10}
%\BNCC{EM13LP19}

Antes da leitura, introduza o tema dos contos de fadas com
os alunos. Após uma breve explicação, sugira que os estudantes pesquisem
contos de fadas de tradição europeia. Importante frisar, mesmo contos
que não sejam utilizados nesse exercício poderão servir em atividades
futuras, de modo que não devem ser descartados.

Feito esse levantamento, observe se encontraram um material robusto
acerca da mitologia germânica, em especial os de tradição de matriz
nórdico-escandinava. Caso tenham tido dificuldade para encontrar contos
dessa região, o professor pode sugerir alguns títulos. Com esses já em
mãos, cabe ao professor orientar o aluno a quedar atento a elementos
como tramas comuns, itens e elementos do ambiente que se repetem. Também
é importante alertar os alunos para estarem atentos quanto aos
principais heróis e divindades presentes nas sagas escandinavas. Tendo
todo esse material, cada aluno pode produzir uma breve história em
quadrinhos, utilizando os as paisagens, os heróis, os monstros, os itens
mágicos, entre outros, para forjar sua própria narrativa.


\Image{Quadro "Thor luta com os gigantes", pintado por Mårten Eskil Winge em 1872. (Mårten Eskil Winge; Domínio Público)}{PNLD0040-06.png}


\Image{Representação de Freir, divindade nórdica da paz e da fertilidade, das colheitas e dos casamentos (Jacques Reich; Domínio Público)}{PNLD0040-10.png}


\Image{Ilustração de 1850 representa Odim, uma das mais importantes divindades nórdicas. (Otto Henrik Wallgren; Domínio Público)}{PNLD0040-08.png}


\Image{Desenho a lápis representa Odim, o deus da guerra segundo a mitologia escandinava (Victor Villalobos; CC BY-SA 4.0)}{PNLD0040-09.png}



\subsection{Leitura}

%\BNCC{EM13LGG103}
%\BNCC{EM13LP02}
%\BNCC{EM13LP48}

Após esse mergulho na cultura teuto-escandinava, retome os
outros contos levantados pelos alunos na atividade 1. Incentive o
paralelo entre esses contos de fadas pesquisando, observando as
similaridades nas narrativas entre eles. Feito isso, estimule a relação
entre os contos e os movimentos artísticos do século XIX, em especial o
romantismo que valorizava e se inspirava em tradições folclóricas em
nome da identidade das nações. Para auxiliar no desenvolvimento da
atividade, é aconselhado que a sala de aula seja dividida em grupos e
que os alunos busquem informações críticas acerca da literatura, como
por exemplo, quais são os gêneros literários? E quais são as escolas
literárias mais importantes que existiram na história? O domínio desse
conteúdo é importante, pois fomenta uma maior absorção do conteúdo
direto e simbólico que um autor desejou imprimir em uma obra. Feito
isso, cada grupo produzirá um material de consulta, como um site
enciclopédico, que poderá ser linkado no site do colégio, ou ficar
disponível no sistema da biblioteca.

\subsection{Pós-leitura}

%\BNCC{EM13LGG102}
%\BNCC{EM13LGG303}
%\BNCC{EM13LGG402}
%\BNCC{EM13LGG703}
%\BNCC{EM13LP13}
%\BNCC{EM13LP14}
%\BNCC{EM13LP28}
%\BNCC{EM13LP29}
%\BNCC{EM13LP52}

Após a leitura completa das \emph{Sagas}, somadas ao
conhecimento obtido nas atividades anteriormente desenvolvidas, proponha
um exercício de escrita criativa para a elaboração de um outro livro de
sagas, mas dessa vez, pautado no folclore brasileiro. Para isso, ofereça
trechos de obras que trabalhem o folclórico, o imaginário, o fantástico
dentro do universo cultural brasileiro. Também é interessante trazer
músicas, filmes, vídeos, desenhos, quadrinhos, etc. que tratem do tema.
Instrua aos alunos a olharem para o livro de Strindberg como modelo para
a estrutura do texto que será escrito, mas que se valham dos outros
elementos da cultura nacional, trazidos para essa atividade, para o
conteúdo do texto. Ao final, os contos podem ser compilados em um livro,
que pode ser impresso ou disponibilizado on-line para os alunos.


\section{Atividades 2}

%\BNCC{EM13CNT201}
%\BNCC{EM13CNT303}
%\BNCC{EM13CHS101}
%\BNCC{EM13CHS102}
%\BNCC{EM13CHS106}
%\BNCC{EM13CHS401}

A obra \emph{Sagas} possibilita trabalhos interdisciplinares e
integradores de diferentes campos do saber e áreas de conhecimento. A
seguir, propomos algumas atividades que podem ser desenvolvidas
conjuntamente com professores de outras áreas. Além das habilidades de
Linguagens e suas Tecnologias e de Língua Portuguesa, indicadas nas
etapas da seção anterior e válidas também para esta, listamos a seguir
as habilidades de outras áreas, presentes na abordagem interdisciplinar:

\subsection{Pré-leitura}

Para conhecer melhor o universo da obra de Strindberg,
recomenda-se que, juntamente com os professores de humanidades, seja
realizada uma pesquisa aprofundada sobre o contexto histórico-cultural
que o autor estava inserido. Nesta atividade, recomenda-se a divisão da
sala em grupos, cada qual deverá pesquisar um aspecto dos elementos que
possivelmente compuseram as referências de Strindberg. Para tanto,
recomenda-se que um grupo pesquise aspectos geográficos da Suécia.
Outro, fie-se nas questões históricas envolvendo o país. Um outro busque
referências culturais e folclóricas e, por fim, um último grupo busque
informações acerca do contexto cultural e literário europeu da época de
Strindberg. Feito isso, convida-se à realização de um seminário onde os
alunos exporão suas descobertas. Durante as exposições, os alunos
deverão também, anotar os detalhes do divulgado pelos outros grupos
para, ao fim, a sala montar um texto informativo que aglutine todos os
elementos obtidos.


\Image{Foto de Strindberg após seu aniversário de 50 anos. (Autor desconhecido; Domínio Público)}{PNLD0040-03.png}


\subsection{Leitura}

Durante a leitura, é interessante explorar o conceito de saga a
partir de documentos históricos provenientes de diferentes nações
europeias. Nessa atividade, ficará evidente que o termo saga é muito
próprio da cultura teuto-escandinava. Entretanto, outros povos da Europa
também detinham narrativas com estruturas análogas e, mesmo histórias e
personagens similares. Dessa maneira, proponha aos alunos um estudo
comparativo de textos folclóricos e mitológicos dos povos europeus. É
interessante a comparação entre as narrativas teuto-escandinavas,
celtas, greco-latinas, eslavas, indo-arianas, entre outras.

\subsection{Pós-leitura}

Após a leitura, recomenda-se a organização de uma oficina de
escrita criativa. Nela os alunos devem observar elementos da
contemporaneidade, enxergando problemas de conduta e possíveis soluções.
O desafio, então, reside na elaboração de uma fábula, um conto
moralizante, ou uma parábola a respeito do tema. Deve ser destacado a
necessidade de alguns elementos na história, como o ponto de vista de um
narrador-onisciente, elementos fantasiosos, uma lição moralizante. A
partir daí, a turma pode organizar uma coletânea de autoria conjunta dos
contos produzidos. o resultado final poderá compor o acervo da
biblioteca do colégio.

\section{Aprofundamento}

Ao chegar ao Ensino Médio, é necessário que os estudantes se aprofundem
na compreensão das múltiplas linguagens e, sobretudo, da linguagem
literária. Em relação à literatura, a BNCC traz as seguintes
considerações:

\begin{quote}
{[}...{]} a leitura do texto literário, que ocupa o centro do trabalho
no Ensino Fundamental, deve permanecer nuclear também no Ensino Médio.
Por força de certa simplificação didática, as biografias de autores, as
características de épocas, os resumos e outros gêneros artísticos
substitutivos, como o cinema e as HQs, têm relegado o texto literário a
um plano secundário do ensino. Assim, é importante não só (re)colocá-lo
como ponto de partida para o trabalho com a literatura, como
intensificar seu convívio com os estudantes. Como linguagem
artisticamente organizada, a literatura enriquece nossa percepção e
nossa visão de mundo. Mediante arranjos especiais das palavras, ela cria
um universo que nos permite aumentar nossa capacidade de ver e sentir.
Nesse sentido, a literatura possibilita uma ampliação da nossa visão do
mundo, ajuda-nos não só a ver mais, mas a colocar em questão muito do
que estamos vendo/vivenciando. (Brasil, 2018, p. 491)
\end{quote}

Nesta seção, desenvolvemos um trabalho de aprofundamento que, em diálogo
com a formação continuada de professores, oferece subsídios para a
abordagem do texto literário. A leitura em sala de aula de \emph{Sagas}
pode ser enriquecida pelo aprofundamento no universo literário em que a
obra está inserida.

\subsection{A obra}

Sagas é uma produção que reúne diversos contos o sueco August
Strindberg. A obra foi concebida como uma espécie de presente e guia
para a filha do autor que, à época, tinha um ano de idade. Desse modo, a
obra é composta pelo elemento do fantástico, do mágico, entre outros
elementos que compõem a temática infantil, no caso de Sagas, muito
inspirados pelo folclore nórdico.

Entretanto, o aspecto ``guia'' da obra se deve ao fato de que muitos
contos possuem caráter moralizantes, heroicos, fabulescos, carregando
consigo um tom didático sobre ocasiões comuns à vida.

\subsection{Da vida do autor para o livro que prepara para a vida}

Quando o livro foi produzido, Strindberg já se encontrava separado de
sua esposa, de modo que as Sagas foram uma forma de o autor se fazer
presente e, de certo modo, contar suas histórias, posto que vários
contos são inspirados em suas experiências, e ao mesmo tempo, poder
exercer o papel de educar a filha, mesmo à distância, por meio das
morais advindas no final de cada uma das histórias.


\Image{Retrato de Strindberg com 25 anos (Studio de Math. S. Hansen; Domínio Público)}{PNLD0040-04.png}


A ideia era de se estreitar o convívio, ao mesmo tempo em que divertia a
filha e a introduzia em ensinamentos primários da ética, da história, da
política e do convívio social, de sorte que os heróis e personagens
assumem traços arquetípicos, sendo o livro um bom guia para o
enfrentamento do cotidiano para aqueles que tem olhares atentos.

Desse modo, vê-se na obra a clara assinatura de Strindberg, que escreve
de maneira simples, porém sem ser simplista. Muito pelo contrário, é no
tom direto e até simplório que se encontra a grande profundidade de sua
mensagem.

Com isso, nota-se que, embora tenha escrito para sua filha de apenas um
ano de idade, Strindberg não deixou de se preocupar com o aspecto
literário, o que faz dessa obra uma leitura fascinante para todas as
idades.


\Image{Retrato de Strindberg pintado por Richard Bergh em 1905 (Richard Bergh; Domínio Público)}{PNLD0040-05.png}


O autor da trilogia dramática \emph{O Caminho para Damasco}, peça
considerada marco inicial do teatro expressionista do século XX,
imprimiu os seus dotes também na presente obra, dotando-a de rico
conteúdo simbólico e matizes oníricas.

Essa obra que tem roupagem de literatura infantil, mas revela-se muito
além disto, como uma lâmpada maravilhosa literária, é composta de contos
que vão de breves peças morais a narrativas oníricas.

A influência do folclore nórdico é patente, a começar pelo título Sagas.
Entretanto, neste caso, não são apenas os feitos épicos dos grandes
heróis, nem as longas narrativas escandinavas. Tem-se temas históricos,
humorísticos, heroicos, em tom fabulesco, ou direto, grande parte dessas
histórias foram inspiradas por acontecimentos na vida do autor.

Fabulesco por terem seu conteúdo moral, exemplificativo, no personagem
que serve de modelo para a formação do caráter.

E Strindberg guardava uma estreita relação com a natureza de seu país
natal. O que é visível em grande parte de suas obras. Em certa medida, é
possível intuir que o autor sentia que o povo sueco era quase como um
produto de sua natureza. Então, retomar temas antigos, das épocas dos
vikings, fazia sentido no caráter formador de sua obra. Afinal seria
recuperar as origens do povo e do meio que sua filha encontraria nos
anos vindouros.


\Image{Escultura Viking localizada em Largs, Reino Unido (Dave Hitchborne; CC BY-SA 2.0)}{PNLD0040-07.png}


Strindberg dá voz própria a natureza e escuta seus clamores: o mar e a
terra, o sol e a neve, todos são personagens em sua história,
convivendo, observando e descrevendo seu constante trajeto de
dificuldades e vitórias, como se compusessem um coro elementar para a
narrativa.

No mesmo sentido, não há personagens menores, há personagens que
aparecem menos, mas cada qual traz uma função narrativa importante, para
amalgamar ou provocar a narrativa, como uma folha de uma especiaria
rara, colocada ao final do caldo para juntar e tornar pujantes todos os
outros temperos e ingredientes já misturados.

\subsection{Obra polifônica}

É notória a polifonia nas Sagas de Strindberg. É claro que Strindberg
tinha uma opinião formada sobre a vida e todos seus elementos, isso se
torna claro ao se observar sua trajetória biográfica. Entretanto, no
microcosmo que retrata, nunca se coloca de maneira direta.

Restava na pantomima imaginária, observando a atuação dos entes de sua
criação no tecido narrativo que produzira. Entendia que, ainda que
detivesse um \emph{telos}, ele não seria atingido por uma imposição
abrupta. Ao invés disso, convida o leitor a percorrer o caminho para
encontrar a moral por ele entendida e desejada, quase como a narrativa
do herói que, após agruras, encontra o tesouro.

\subsection{Por que ler Sagas?}

Embora aborde cenários infantis o autor não deixa passar a oportunidade
de invocar questões sociais e políticas, provocando a reflexão em seus
leitores acerca dos mesmos, ainda que não de forma escancarada.

Desafia a glória de governantes, e a glória dos cientistas, pondo em
xeque se o conhecimento realmente conduz a uma vida mais plena e
agradável.

Aborda questões de convivência e cria um conteúdo exemplificativo para
narrar a boa moral e o bom caráter.

É curioso que, apesar da linguagem infantil de seus contos, a
complexidade dos temas tratados e o decorrer da história indicam que, a
parte a roupagem inicial, são textos voltados a todas as idades e que,
apesar da roupagem do fantástico, narra, em verdade, situações
cotidianas.

\section{Sugestões de atividades complementares: relações dialógicas e
intertextuais}

%\BNCC{EM13LP03}
%\BNCC{EM13LP04}
%\BNCC{EM13LP49}
%\BNCC{EM13LP51}

No Ensino Médio, da mesma forma que no Ensino Fundamental, a \textsc{bncc}
organiza o trabalho com as práticas de linguagem em cinco \textbf{campos
de atuação social}. São eles: campo da vida pessoal, campo da vida
pública, campo jornalístico"-midiático, campo artístico"-literário e campo
das práticas de estudo e pesquisa.

De acordo com essa divisão, propomos na sequência um trabalho
interdiscursivo e intertextual com a obra \emph{Sagas.}



\subsection{Campo da vida pessoal}

\begin{quote}
O campo da vida pessoal pretende funcionar como espaço de articulações
e sínteses das aprendizagens de outros campos postas a serviço dos
projetos de vida dos estudantes. As práticas de linguagem privilegiadas
nesse campo relacionam"-se com a ampliação do saber sobre si, tendo em
vista as condições que cercam a vida contemporânea e as condições
juvenis no Brasil e no mundo.

Está em questão também possibilitar vivências significativas de práticas
colaborativas em situações de interação presenciais ou em ambientes
digitais e aprender, na articulação com outras áreas, campos e com os
projetos e escolhas pessoais dos jovens, procedimentos de levantamento,
tratamento e divulgação de dados e informações e o uso desses dados em
produções diversas e na proposição de ações e projetos de natureza
variada, para fomentar o protagonismo juvenil de forma
contextualizada. (\textsc{bncc}, p. 494)
\end{quote}

Como visto, Strindberg produziu \emph{Sagas} como uma espécie de
presente-guia para sua filha, ainda em tenra idade quando da escrita
da obra. Outro caso similar é o do helenista Jean-Pierre Vernant, que
escreveu o trabalho \emph{O Universo, Os Deuses e os Homens}, obra que
abarca narrativas da mitologia grega em um linguajar simplificado,
após passar finais de semana cuidando de sua neta.

Agora, convide o aluno a se imaginar em uma situação como essa. Peça
para que ele selecione um filme, um livro, uma peça que seja de seu
agrado e que, feito isso, transcreva-a de uma forma que seja palatável
para uma criança muito jovem. Para isso, sugere-se a seleção de trechos
da obra citada e comparação com trechos das obras de onde Vernant buscou
a inspiração, como em Hesíodo ou Homero.


\subsection{Campo de atuação na vida pública}

\begin{quote}
No cerne do campo de atuação na vida pública estão a ampliação da
participação em diferentes instâncias da vida pública, a defesa dos
direitos, o domínio básico de textos legais e a discussão e o debate de
ideias, propostas e projetos. {[}\ldots{}{]}

Ainda no domínio das ênfases, indica"-se um conjunto de habilidades que
se relacionam com a análise, discussão, elaboração e desenvolvimento de
propostas de ação e de projetos culturais e de intervenção social.
(\textsc{bncc}, p. 494)
\end{quote}


Como visto, Strindberg se valeu de sua obra, uma coletânea de
narrativas infantis, para propagar ideais morais. Desse modo, é
possível se valer de histórias escritas de maneira simples para
divulgar ideias úteis e saudáveis à sociedade. Um local onde vemos
isso muito comumente é nas fábulas. Entretanto, por vezes adotando um
tom crítico, encontramos uma iniciativa de perspectiva similar nas
tirinhas.


Portanto, nessa atividade, o aluno deverá valer-se de um dos contos de
Strindberg, onde consiga enxergar o tom moralizante, e produzir uma
tirinha a respeito.

Sugere-se, então, na etapa inicial, o planejamento da produções, divida
os alunos em grupos. Cada grupo irá selecionar os materiais artísticos a
serem utilizados, uma vez que os gêneros textuais em questão fazem uso
da linguagem não verbal.

Recomenda-se utilizar o roteiro abaixo para planejar a atividade:

\begin{itemize}
\item
Construam pequenos parágrafos narrando a sequência dos principais
acontecimentos.
\item
Quem serão as personagens do texto? Caracterizem cada uma delas e, se
necessário, façam pequenos esboços dos diálogos, empregando,
inicialmente, dois-pontos e travessões.
\item
Em que cenário ocorrerão os fatos? Descrevam os principais elementos
do espaço e verifiquem se ele contribui para a ambientação e para os
efeitos de humor ou reflexão propostos pelo texto.
\end{itemize}

Ao final, as produções podem ser compiladas e divulgadas no site do
colégio e também compor o acervo da biblioteca da escola.

\subsection{Campo jornalístico"-midiático}

\begin{quote}
Em relação ao campo jornalístico"-midiático, espera"-se que os jovens
que chegam ao Ensino Médio sejam capazes de: compreender os fatos e
circunstâncias principais relatados; perceber a impossibilidade de
neutralidade absoluta no relato de fatos; adotar procedimentos básicos
de checagem de veracidade de informação; identificar diferentes pontos
de vista diante de questões polêmicas de relevância social; avaliar
argumentos utilizados e posicionar"-se em relação a eles de forma ética;
identificar e denunciar discursos de ódio e que envolvam desrespeito aos
Direitos Humanos; e produzir textos jornalísticos variados, tendo em
vista seus contextos de produção e características dos gêneros. Eles
também devem ter condições de analisar estratégias
linguístico"-discursivas utilizadas pelos textos publicitários e de
refletir sobre necessidades e condições de consumo.

No Ensino Médio, os jovens precisam aprofundar a análise dos interesses
que movem o campo jornalístico midiático, da relação entre informação e
opinião, com destaque para o fenômeno da pós"-verdade, consolidar o
desenvolvimento de habilidades, apropriar"-se de mais procedimentos
envolvidos na curadoria de informações, ampliar o contato com projetos
editoriais independentes e tomar consciência de que uma mídia
independente e plural é condição indispensável para a democracia.

Como já destacado, as práticas que têm lugar nas redes sociais têm
tratamento ampliado. (\textsc{bncc}, p. 494-495)
\end{quote}

Ao longo da Era Industrial, a propaganda se tornou peça chave na
atividade de qualquer indústria. Afinal, por meio dela, há a
divulgação e, em certa medida, uma pretensa ratificação da qualidade
do produto. Outrossim, não é raro vermos propagandas abusivas, as
quais são postas em xeque pelo Ministério Público e de órgãos
governamentais. As propagandas também foram relativizadas ao longo da
história. Hoje, não se tem mais anúncios de cigarro e bebida na
televisão, como ocorria outrora. No mesmo sentido, não se vê mais os
famigerados ``cigarros de chocolate'', uma sobremesa infantil que, de
certo modo, induzia ao fumo. Entretanto, há hoje um bombardeamento das
pessoas com propagandas em sites e aplicativos. Um local que gera
muita preocupação para a veiculação dos anúncios é nos jogos de
celular. Isso posto, abra uma roda de discussão com a turma a respeito
da temática. Procure abordar as questões dos limites, como até onde se
tem um marketing agressivo e quando ele passa a ser abusivo? Deve ser
intensificada a censura nos meios de comunicação? Como as pessoas
devem se preparar para não serem vítimas de propaganda enganosa e
quais os meios legais para se proteger caso o forem.

\subsection{Campo artístico"-literário}

\begin{quote}
No campo artístico"-literário busca"-se a ampliação do contato e a
análise mais fundamentada de manifestações culturais e artísticas em
geral. Está em jogo a continuidade da formação do leitor literário e do
desenvolvimento da fruição. A análise contextualizada de produções
artísticas e dos textos literários, com destaque para os clássicos,
intensifica"-se no Ensino Médio. Gêneros e formas diversas de produções
vinculadas à apreciação de obras artísticas e produções culturais
(resenhas, vlogs e podcasts literários, culturais etc.) ou a formas de
apropriação do texto literário, de produções cinematográficas e teatrais
e de outras manifestações artísticas (remidiações, paródias,
estilizações, videominutos, fanfics etc.) continuam a ser considerados
associados a habilidades técnicas e estéticas mais refinadas.

A escrita literária, por sua vez, ainda que não seja o foco central do
componente de Língua Portuguesa, também se mostra rica em possibilidades
expressivas. (\textsc{bncc}, p. 495-496).
\end{quote}

Na obra \emph{Sagas}, Strindberg vale-se de diversos formatos
literários para escrever suas histórias. Bem verdade, há uma mescla de
elementos de gêneros e estilos diversos. Entretanto, um estudioso
atento, perceberia elementos das fábulas, de contos de fadas, e mesmo
textos moralizantes em forma de parábola ao longo das \emph{Sagas}.
Nesta atividade sugere-se que os alunos façam uma pesquisa sobre esses
gêneros, ficando a critério de cada um a escolha do que pesquisar. Com
a informação obtida, cada aluno deverá escrever um conto, fábula, ou
parábola, no formato de história em quadrinhos. Para recursos
narrativos é possível disponibilizar um material de apoio, mas,
aconselha-se que os alunos sejam direcionados para tentar desenvolver
seu texto a partir dos moldes do autor estudado.

\subsection{Campo das práticas de estudo e pesquisa}

\begin{quote}
O campo das práticas de estudo e pesquisa mantém destaque para os
gêneros e habilidades envolvidos na leitura/escuta e produção de textos
de diferentes áreas do conhecimento e para as habilidades e
procedimentos envolvidos no estudo. Ganham realce também as habilidades
relacionadas à análise, síntese, reflexão, problematização e pesquisa:
estabelecimento de recorte da questão ou problema; seleção de
informações; estabelecimento das condições de coleta de dados para a
realização de levantamentos; realização de pesquisas de diferentes
tipos; tratamento dos dados e informações; e formas de uso e
socialização dos resultados e análises.

Além de fazer uso competente da língua e das outras semioses, os
estudantes devem ter uma atitude investigativa e criativa em relação a
elas e compreender princípios e procedimentos metodológicos que orientam
a produção do conhecimento sobre a língua e as linguagens e a formulação
de regras. (\textsc{bncc}, p. 495-496)
\end{quote}

Dentro das histórias, são apresentados uma série de elementos da
natureza, os quais podem até causar certa impressão ao leitor num
primeiro momento. Diante disso, a atividade sugere, juntamente com
auxílio do professor de humanidades e de ciências da natureza, uma
pesquisa sobre as características naturais da Península da
Escandinávia, seus aspectos geológicos, biológicos e morfoclimáticos.
O resultado dessa investigação pode ser trabalhado por meio de um
artigo, ou por meio de uma apresentação para os demais colegas,
devendo, nesse caso, haver uma separação prévia da sala em grupos,
cada qual atuará com um tema.

\section{Referências complementares}

\begin{itemize}
\item\textsc{larsson}, Stieg. \textbf{Os homens que não amavam as mulheres.} São
Paulo: Companhia das Letras, 2010.

Os homens que não amavam as mulheres é uma fascinante e assustadora
aventura vivida por um veterano jornalista e uma jovem e genial hacker
cujo comportamento social beira o autismo.

\item\textsc{lindrgen}, Astrid. \textbf{Píppi Meialonga}. São Paulo: Companhia das
Letras, 2001.

Píppi é uma menina de nove anos incrivelmente forte. Não tem pai nem mãe
e mora sozinha. Seus companheiros são um cavalo e um macaquinho e a
garota faz suas roupas, enfrenta valentões e sempre possui uma resposta
na ponta da língua, com grande confiança em si.
\end{itemize}

\section{Bibliografia comentada}

\begin{itemize}
\item\textsc{wallin}, Claudia. \textbf{Suécia: um país sem excelências e mordomias.}
São Paulo: Geração Editorial, 2014.

A autora prova que existem políticos que não têm mordomias, que não
aumentam seu próprio salário, que usam transporte público e não estão na
vida pública para fazer fortuna. Um sistema apoiado em três pilares:
transparência, escolaridade e igualdade.

\item\textsc{carpeaux}, Otto Maria\textbf{. A história Concisa da Literatura Alemã.}
Alphaville: Faro Editorial, 2013.

O autor faz uma síntese dos grandes momentos, livros e autores da
literatura alemã, e conta com uma avaliação crítica de sua importância
para a cultura e o desenvolvimento do país e sua influência nos
principais movimentos culturais do mundo contemporâneo.

\item\textsc{carpeaux}, Otto Maria\textbf{. A história Literatura Ocidental.} São
Paulo: Leya, 2019.

Da literatura grega à contemporânea, Carpeaux analisa e critica as obras
com seu arcabouço teórico. Seu conhecimento nos leva ao encontro dos
mais importantes autores da literatura ocidental.
\end{itemize}

\end{document}

