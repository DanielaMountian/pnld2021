\documentclass[12pt]{extarticle}
\usepackage{manualdoprofessor}
\usepackage{fichatecnica}
\usepackage{lipsum,media9,graficos}
\usepackage[justification=raggedright]{caption}
\usepackage{bncc}
\usepackage[edlab]{../edlab}

\begin{document}


\newcommand{\AutorLivro}{Mark Twain}
\newcommand{\TituloLivro}{Diários de Adão e Eva}
\newcommand{\Tema}{Ficção, mistério e fantasia}
\newcommand{\Genero}{Conto, crônica e novela}
\newcommand{\imagemCapa}{./images/PNLD0043-01.png}
\newcommand{\issnppub}{---}
\newcommand{\issnepub}{---}
% \newcommand{\fichacatalografica}{PNLD0043-00.png}
\newcommand{\colaborador}{\textbf{Michelle Etienne Florence, Bruno Gradella e Vicente Castro} é uma pessoa incrível e vai fazer um bom serviço.}


\title{\TituloLivro}
\author{\AutorLivro}
\def\authornotes{\colaborador}

\date{}
\maketitle

\baselineskip=1.15\baselineskip\par


\begin{abstract}
Este Manual tem como objetivo fornecer subsídios para o trabalho com a
obra literária \emph{Diários de Adão e Eva}, de Mark Twain.

\textbf{Mark Twain}, pseudônimo de Samuel Langhorne Clemens, nasceu em 1835,
foi um dos mais importantes e prolíficos escritores dos Estados Unidos.
Atuou como tipógrafo, repórter, colunista, palestrante, minerador e piloto de
barco a vapor no rio Mississipi, ocupação que o levou a conhecer a realidade 
do sul e do meio-oeste americanos, inspiração e pano de fundo para
grande parte de seus personagens e romances, entre eles os famosos \textit{The
Adventures of Tom Sawyer} (1876) e \textit{Adventures of Huckleberry Finn}
(1885). Dono de uma linguagem irônica e cheia de humor, criticou a sociedade
escravocrata da época, cujos valores, fanatismo e hipocrisia foram alguns dos
alvos prediletos de sua sátira.

\textbf{\textit{Diários de Adão e Eva e outras sátiras bíblicas}} reúne os 
principais textos, inéditos, da sátira de Twain a alguns livros e personagens 
bíblicos. Encontraremos aqui a história bíblica do Gênesis recontada pela 
perspectiva divertidíssima de seus mais icônicos participantes, Adão e Eva. 
Como se cada um deles tivesse escrito um diário, lemos suas percepções da vida 
e do mundo, no paraíso e no fora dele, e da forma como cada um enxerga o outro, 
conforme vão se conhecendo.

Conheceremos um Adão de poucas palavras, mas que, depois de comer o fruto,
torna-se mais falante, com falas mais complexas e vocabulário mais rico, 
preparado para a jornada de autoconhecimento que a expulsão do Paraíso
lhe provoca. Já Eva, que era mais solta e animada, torna-se mais madura, e,
tal qual Adão, passa a buscar um aprofundamento das relações depois do sonho
distante do Paraíso. 

Além desses \textit{Diários}, nesta edição também se encontram outros com a 
mesma temática, como o “Solilóquio de Adão'', a “Autobiografia de Eva” e 
“Passagens do diário de Satã”, que só pelos títulos já aguçam a imaginação 
do leitor sobre o que ele há de encontrar!

Esperamos que as indicações propostas aqui sejam muito úteis no trabalho em
sala de aula!

\end{abstract}

\tableofcontents

\section{Introdução}

% abertura resumo da obra
olá, é com muita alegria que apresentamos a obra “diários de adão e eva e outras sátiras bíblicas”, de mark twain.

A obra reúne os principais textos da sátira de twain a alguns livros e personagens bíblicos.


% a obra e o autor
o escritor atuou como tipógrafo, repórter, colunista, palestrante, minerador e piloto de barco.

A diversidade de atividades praticadas levaram a conhecer diferentes realidades pelos estados unidos, inspiração para grande parte de seus personagens e romances

Usando linguagem irônica e cheia de humor, criticou a sociedade escravocrata da época, onde fanatismo e hipocrisia eram alvos prediletos de sua sátira.

Porém, curiosamente, o autor não demonstrava tamanho interesse pela escola, pela leitura e a escrita.

Entre outras muitas experiências, vivenciou a de soldado durante a guerra civil, antes de se convencer de que seu verdadeiro talento era o de inventar histórias, onde realizou muitas palestras.

Suas narrativas eram contadas com sotaque sulista , com acento carregado, lento, anasalado, sem pressa, recheado de vocábulos e expressões típicas do sul.

Um regionalista que aclamava sua origem

Nunca tentou ser quem não era. Nunca quis passar por pessoa culta, erudita

Mark twain tentou promover leituras públicas de seus livros, mas os primeiros resultados foram desanimadores. “eu achava que bastava imitar dickens”, ele disse em sua autobiografia, “subir na plataforma e ler o livro. Mas tentei fazer isso e foi um fracasso. Coisas escritas não servem para discursos”. Ele então desenvolveu seu estilo próprio e divertido de dar palestras como se estivesse conversando com o público.
uma curiosidade, é que o autor era obcecado pela figura do personagem adão.

Entre suas cartas trocadas com a sua esposa, havia uma em que estava escrito:“ adáo, o pai da raça humana, certamente merece ser honrado”

Motivado por tal obsessáo veio a redigir uma petição ao congresso americano para conceder à cidade de elmira permissão para construir um monumento a adão, sem sucesso.

Seguindo sua jornada, enfim escreve os diários de adão e eva, em  mil oitocentos e noventa e trës e mil novecentos e cinco, respectivamente, não consegue que sejam imediatamente publicados

Para conseguir sua publicação, buscou apoio de um amigo antigo, então secretário de turismo da região de buffalo,

O político sugere que ele adapte o manuscrito de forma a ambientar o enredo do mítico paraíso de adão e eva nas cataratas do niágara, pois assim poderia ser vendido como souvenir durante a exposição pan-americana de chicago,

Uma das modificações mais óbvias é uma em que eva começa a organizar o paraíso colocando placas que indicavam os diversos locais de visitação do lugar, como se o jardim do éden fosse um parque público.

No relato adão diz que algumas dessas placas indicam: “por aqui para a piscina de hidro”, “por aqui para a ilha das cabras”, “caverna dos ventos por aqui”
a segunda parte do diário, “escrita” por eva, é primeiramente publicada como uma edição especial, natalina


% motivação de leitura (tema: ficção, mistério e fantasia)
a plateia ria do momento em que ele abria a boca até o final

Suas famosas tiradas deixam visível a intenção de dizer a verdade, de fazer as pessoas perderem a pose rindo até se darem conta de que deveriam rir, ou ter vergonha de si mesmas.

Mark twain permanece como um dos mais amados autores ligados ao individualismo americano. Ao contrário de tantos dos seus contempor neos, ele não via os estados unidos como uma extensão da europa. 

O autor exaltou o empreendedorismo individual e se posicionou contra as injustiças onde quer que as encontrasse. Enquanto morou em vienna, desafiou a imprensa antissemita e defendeu o capitão francês alfred dreyfus, que as cortes militares tinham condenado por traição por ser judeu.

Quando estava dando palestras na inglaterra em 1894, sua filha morreu de meningite. Livy, sua esposa por 34 anos, sucumbiu por conta de um problema cardíaco em 1904. “durante os anos que se seguiram à morte da minha esposa”, escreveu em sua autobiografia, “eu mergulhei em uma triste maré de banquetes e discursos sobre causas elevadas e santas; mas enquanto essas coisas me animavam e alimentavam intelectualmente, elas tocavam apenas momentaneamente meu coração, que então ficava seco e empoeirado”.

Mark twain celebrava o país como uma civilização distinta e defendia a liberdade e a justiça, e também promovia a paz. 

Retratava espíritos livres, determinados, rústicos, que superavam obstáculos intimidadores para realizar seus destinos. Seu charme pessoal e seu humor afiado ainda fazem as pessoas sorrir. Suas obras voltadas ao público juvenil apresentam protagonistas fortes e inspiradores.


% gênero e movimento literário
curiosamente, a parte dois, contém, no subtítulo, a observação: “traduzido do original”.

Essa observação insinua é que mark twain não assume a autoria da obra, mas sim a tradução.

Ou seja, o autor conduz o leitor a pensar que ele foi “mero” tradutor, e não quem ficcionalizou os acontecimentos narrados no jardim do éden.

Se por um acaso do destino a obra visse a ser acusada de blasfêmia, ironia ou deboche, não seria mark twain quem teria de se sentar no banco dos réus.

A última subdivisão, escrita por eva quarenta anos depois, já longe do paraíso, tem um estilo extremamente “meloso”.
o objetivo de twain era relacionar as narrativas do livro do gênesis ao presente e ao mesmo tempo demonstrar sua teoria cíclica da história

O gênero é o diário, com suas entradas e datas, que remetem a uma visão ciclica da história.


% (quadro de habilidades referentes à obra)

% encerramento ( vinheta com logo )

Para mais informações sobre as habilidades trabalhadas nesta obra, consulte o material digital do professor. Obrigado e até a próxima!


\section{Atividades 1}

%\BNCC{EM13LP26}

\subsection{Pré-leitura}

%\BNCC{EM13LGG302}
%\BNCC{EM13LGG704}
%\BNCC{EM13LP10}
%\BNCC{EM13LP19}

Antes de adentrar na leitura da obra, é interessante que os
alunos estejam familiarizados com o gênero diário. Na contemporaneidade,
a prática de produção de diários pessoais se inscreve na fronteira entre
os formatos analógico e digital. Convivem simultaneamente as formas de
registro manuscritas, que adotam tradicionais cadernos e agendas como
suporte, e as novas tecnologias digitais de comunicação e informação,
baseadas no uso de editores de texto e compartilhamento de dados em
blogs e sites.

O gênero textual diário pertence ao grupo das experiências que envolvem
a escrita de si, no âmbito das quais se inserem também, por exemplo, a
autobiografia e o memorial.

Na presente atividade, os alunos irão redigir um diário digital, que
poderá ser veiculado junto ao portal do colégio.
\Image{Retrato de Mark Twain, 1907 (A.F. Bradley; Domínio Público)}{PNLD0043-03.png}

Para esta atividade, sugere-se a realização de um exercício do
\emph{journaling} -- como é conhecida, em inglês, a prática de produção
frequente de diários pessoais. É uma oportunidade para conhecer a
própria personalidade e, no caso de ter acesso a diários de outras
pessoas, descobrir os gostos, interesses e vivências de colegas e
familiares. Nesta atividade, oriente os alunos a produzirem textos sobre
seus cotidianos. Os textos serão escritos individualmente, mas a versão
digital será compartilhada com a turma. Desse modo, indique que os
registros manuscritos poderão incluir todas as experiências que se
desejar anotar, mas as entradas do blog deverão ser filtradas e conter
apenas informações que possam ser lidas e acessadas abertamente por
outros.

\Image{Biblioteca do primeiro andar da casa de Mark Twain (Jack E. Boucher; Domínio Público)}{PNLD0043-05.png}

\subsection{Leitura}

%\BNCC{EM13LGG103}
%\BNCC{EM13LP02}
%\BNCC{EM13LP48}

Outra forma de registro muito comum na atualidade são os
fotoblogs, ou aplicativos cujo objetivo principal é divulgar fotos e
vídeos próprios, ou de assuntos que o detentor do perfil considera
interessantes. Nessa atividade, sugere-se o trabalho com fotografia ou
desenho, a depender das possibilidades e disponibilidades da turma. A
ideia é que os alunos se desloquem a algum lugar com verde, seja um
jardim, uma praça, um parque, uma horta, etc. Nesse ambiente, sugira que
os alunos produzam fotografias ou desenhos que ilustrem as paisagens
descritas na obra de Mark Twain, nas quais Adão e Eva conviveram e
produziram seus diários. O material produzido poderá ser exibido no
pátio ou mesmo no site do colégio.

\SideImage{Casa onde Mark Twain passou sua infância, Hannibal, Missouri, EUA. (Andrew Balet/Wikimedia Commons; CC BY 2.5)}{PNLD0043-06.png}
\SideImage{Casa de Mark Twain em Hartford, Connecticut, EUA. (Srett/Flickr; CC BY-SA 2.0)}{PNLD0043-04.png}

\subsection{Pós-leitura}

%\BNCC{EM13LGG102}
%\BNCC{EM13LGG303}
%\BNCC{EM13LGG402}
%\BNCC{EM13LGG703}
%\BNCC{EM13LP13}
%\BNCC{EM13LP14}
%\BNCC{EM13LP28}
%\BNCC{EM13LP29}
%\BNCC{EM13LP52}

Nesta atividade sugere-se que sejam separadas previamente
cenas de O Paraíso Perdido, de John Milton. Ofereça os trechos
escolhidos para a leitura dos alunos. A ideia aqui é promover a
comparação entre dois textos literários distintos que tratam da mesma
temática. Para essa atividade, é interessante comparar tanto a questão
da estrutura das obras, que se organizam na forma de diário e de poesia
épica, bem como trabalhar a significação e interpretação que cada um dos
autores oferece acerca da história bíblica do gênesis. Feito os
levantamentos iniciais, solicite que os alunos redijam um pequeno texto,
onde apontarão seus próprios pareceres. Ao final, sugere-se formar uma
roda com a sala de aula para debater os pontos levantados.

\Image{Capa da primeira edição de ``O Paraíso Perdido'' de John Milton, 1667. (Houghton Library; Domínio Público)}{PNLD0043-07.png}

\section{Atividades 2}

%\BNCC{EM13CNT201}
%\BNCC{EM13CNT303}
%\BNCC{EM13CHS101}
%\BNCC{EM13CHS102}
%\BNCC{EM13CHS106}
%\BNCC{EM13CHS401}

A obra \emph{Diários de Adão e Eva} possibilita trabalhos
interdisciplinares e integradores de diferentes campos do saber e áreas
de conhecimento. A seguir, propomos algumas atividades que podem ser
desenvolvidas conjuntamente com professores de outras áreas. Além das
habilidades de Linguagens e suas Tecnologias e de Língua Portuguesa,
indicadas nas etapas da seção anterior e válidas também para esta,
listamos a seguir as habilidades de outras áreas, presentes na abordagem
interdisciplinar:

\subsection{Pré-leitura}

Antes de ler, sugere-se a exposição de matérias gráficos
sobre a história do Gênesis. Através dos tempos, várias foram as
representações feitas da história bíblica. Filmes, desenhos,
videoclipes, gravuras, murais, entre outros, sempre essa história foi
retratada. Entretanto, é no campo da pintura onde se encontram as
representações gráficas mais prolíficas. A fim de adentrar já no
universo da obra que será lida, sugere-se que essas imagens sejam
expostas aos alunos. É possível, também, solicitar que os estudantes
realizem uma pesquisa sobre o tema, trazendo eles próprios essas
imagens, juntamente com informações sobre os artistas que as produziram
e o contexto em que estavam inseridos na época.

\subsection{Leitura}

O Livro Diários de Adão e Eva, pode ser entendido como um
movimento dialético, que, em suma, significa o confronto de duas ideias,
por meio do qual, suas essências são depuradas. Aristóteles entendia
que, por meio da dialética, chegar-se-ia à verdade. Hegel, por sua vez,
definiu que, por meio desse processo, obter-se-ia o conhecimento
absoluto. A ideia por trás da dialética está intimamente ligada com a
ideia de falseabilidade, desenvolvido por Popper, que se tornou um pilar
da epistemologia, e quase como um princípio da ciência. Isto, por sua
vez, significa que, ao ser posta a prova, qualquer ideia pode ser
verificada como falsa, parcial ou integralmente, ou então comprovada, na
exata proporção oposta. Com isso, confrontando-se várias vezes uma ideia
em específico, caso ela não tenha sido imediatamente descartada em um
primeiro confronto, ela será paulatinamente depurada, ao ser
desconectada de seus pontos equivocados, chegando, então, mais próxima
de um conhecimento ou lei científica.

Durante o século XIX, houve um grande furor com as descobertas de
Heinrich Schliemann, que comprovou a existência de locais descritos nas
obras homéricas, nas primeiras iniciativas próximas ao que viria a se
tornar a arqueologia. Em vista do exposto, peça aos alunos que realizem
uma pesquisa acerca de descobertas arqueológicas tangentes a locais,
povos e eventos descritos no Antigo Testamento. Feito isso, solicite que
os alunos escrevam um breve texto, indicando o que das sagradas
escrituras pode ser comprovado e o que não foi capaz de ser confirmado.

\Image{Capa de ``Eve's Diary'', 1906 (Wikimedia Commons; Domínio Público)}{PNLD0043-08.png}

\subsection{Pós-leitura}

Na obra lida, somos expostos às visões de Adão e Eva sobre
acontecimentos análogos e simultâneos e, quando não, os mesmos. Desde o
início da organização da sociedade, foi realizada uma divisão sexual das
tarefas, em geral, restando as mulheres mais próximas ao núcleo da
ocupação, enquanto os homens saíam para busca de alimentos, em uma
circunscrição mais ampla. Outrossim, isso acabou por acarretar
distinções sociais que se desenvolvera e, de certa forma, se perpetuaram
no decorrer da história. A consequência nefasta dessa situação é que, em
muitas sociedades, a mulher foi relegada a um segundo plano, obtendo
recentemente, e apenas em alguns lugares, a igualdade de direitos e
oportunidades devidas. Acerca disso, proponha uma discussão com os
alunos a respeito do tema, levantando elementos capazes de compor um
arcabouço adequado para o entendimento. Após, peça para que cada aluno
realize uma pesquisa a produção de um texto crítico, abordando um caso
concreto, e contextualizando no cenário geral da temática.

\section{Aprofundamento}

Ao chegar ao Ensino Médio, é necessário que os estudantes se aprofundem
na compreensão das múltiplas linguagens e, sobretudo, da linguagem
literária. Em relação à literatura, a BNCC traz as seguintes
considerações:

\begin{quote}
{[}...{]} a leitura do texto literário, que ocupa o centro do trabalho
no Ensino Fundamental, deve permanecer nuclear também no Ensino Médio.
Por força de certa simplificação didática, as biografias de autores, as
características de épocas, os resumos e outros gêneros artísticos
substitutivos, como o cinema e as HQs, têm relegado o texto literário a
um plano secundário do ensino. Assim, é importante não só (re)colocá-lo
como ponto de partida para o trabalho com a literatura, como
intensificar seu convívio com os estudantes. Como linguagem
artisticamente organizada, a literatura enriquece nossa percepção e
nossa visão de mundo. Mediante arranjos especiais das palavras, ela cria
um universo que nos permite aumentar nossa capacidade de ver e sentir.
Nesse sentido, a literatura possibilita uma ampliação da nossa visão do
mundo, ajuda-nos não só a ver mais, mas a colocar em questão muito do
que estamos vendo/vivenciando. (Brasil, 2018, p. 491)
\end{quote}

Nesta seção, desenvolvemos um trabalho de aprofundamento que, em diálogo
com a formação continuada de professores, oferece subsídios para a
abordagem do texto literário.

\subsection{A Obra}

Em uma narrativa leve e divertida, Mark Twain reconta a história bíblica
do Gênesis, da perspectiva de seus mais icônicos participantes, Adão e
Eva. Como se cada um deles tivesse escrito um diário, lemos suas
percepções da vida e do mundo -- tanto no paraíso, quanto após a perda
deste --, e da forma como cada um enxerga o outro, conforme vão se
conhecendo.

\subsection{A representação da vida moderna em uma história antiga}

Os diários são uma representação da humanidade, nos seus aspectos
masculino e feminino, interpretados pelo primeiro casal da história,
segundo a crença ocidental.

No casal primordial, enxergamos traços comuns das personalidades, de
modo que, ainda que retratados em tempos imemoriais, os sentimentos de
Adão e Eva são exatamente o que muitos ainda sentem.

Traço comum das convivências modernas, à ausência de compreensão da
perspectiva do outro muitas vezes é causa de atritos entre as pessoas em
um relacionamento. Mark Twain, portanto, em poucas páginas oferece uma
profunda reflexão, nos colocando um anteparo refletor nas pessoas do
casal primordial, para que nós percebamos que, às vezes estando
envolvidos na relação, deixamos de refletir e de procurar entender como
aquela situação se apresenta diante do entendimento do outro. Ainda na
vida moderna, que muitas vezes parece estar tão distante do paraíso.

\subsection{O desenvolvimento da relação}

No começo, Adão achava Eva uma pessoa chata. Ele entende que a chegada
dela foi um percalço, algo que atrapalhou seu bem estar. Eva, por outro
lado, enxerga em Adão um ser interessante e uma possível companhia,
afinal, ficar sozinha e não ter com quem falar é a coisa que ela menos
gosta.

Adão muito simples, muito prático. Eva, romântica por excelência e
curiosa, sonhava em decorar o cabelo com as estrelas. Só depois de
muitas tentativas, frustrada, desistiu, ao perceber o quão distantes
elas estavam.

Aos poucos, a presença de Eva se torna impossível de ser ignorada, de
modo que Adão não consegue mais nem pensar sem utilizar as palavras
inventadas por Eva. Eva, aliás, é uma grande inventora, gosta de dar
nome às coisas, de investigar e descobrir. E se frustra por Adão não
demonstrar interesse em coisas que não vê um senso prático.

Na verdade, muitas das angústias de Eva, em razão do laconismo de Adão,
são fruto de uma certa incompreensão. Ele nascera no mundo sozinho,
socializar-se é uma novidade ao primeiro homem. Entretanto, ele pensa
que deve ser paciente e se esforçar por ela. Que ela é mais nova e mais
empolgada que ele, não podendo ser julgada por sua ótica.

Há momentos em que Adão apenas diz que ``aguentou firme''. Numa primeira
leitura, isso dá a impressão que, para ele, a presença de Eva seria
insuportável. Entretanto, ao fim de uma leitura completa da obra, parece
muito mais como um ato de resignação, de paciência diante de outra
pessoa, que sim, o frustra em alguns aspectos, mas que se torna
fundamental à sua vida e que ele passa a compreender, tentando tomar o
ponto de vista de sua companheira para as coisas.

Eva, por sua vez, entende que Adão é um ser desprovido de inteligência,
mas mesmo assim, sua presença a agrada e ela sublima a aparente
limitação de seu par.

\Image{Capa de ``Extracts From Adam's Diary'', 1904 (Wikimedia Commons; Domínio Público)}{PNLD0043-09.png}

\subsection{O fruto do pecado e do conhecimento}

Como já indicado anteriormente, Adão é um ser de poucas palavras. Isso
se reflete na escrita em seu diário, como ela aparece no início da obra.
Entretanto, após comer o fruto, imediatamente, os textos de Adão
aparecem alongados, sendo dotados de um vocabulário mais complexo e um
conteúdo mais reflexivo e profundo. A expulsão do paraíso é um momento
de temor, em que o ambiente de segurança cai em uma ruína essencial,
entretanto, é ao mesmo tempo o momento em que há o verdadeiro
autoconhecimento de Adão e o reconhecimento da beleza da vida e o
bem-estar que sua companheira lhe causa.

Mesmo Eva, que sempre foi mais falante, apresenta um tom mais maduro em
suas linhas. O mundo deixa de ser um local de aventura e brincadeira e
os laços emocionais se tornam mais intensos, sendo quase como o fio
condutor de sua vida. O paraíso é como se fosse um sonho, uma lembrança
bonita. Mas, assim como para Adão, o bem se encontra na pessoa do outro
e a felicidade emana da convivência com ela... por mais que às vezes
seja deveras estressante.

\subsection{Por que ler os Diários de Adão e Eva?}

Os diários trazem a visão de cada um dos acontecimentos cotidianos, mas
nem sempre se sobrepõem. Há momentos que sim, ambos narram os mesmos
eventos de perspectiva distintas, mas não é o que ocorre sempre. De toda
sorte, a visão distinta de ambos dos acontecimentos diuturnos, compõe,
na verdade, uma grande unidade.

O livro é belo na simplicidade com que descreve situações mais básicas,
da perspectiva de alguém que desconhece palavras, como a descrição que
Adão faz do choro de Eva. No caso, é o contar de uma história da
perspectivas de dois seres que já nasceram adultos, não foram crianças,
então comportamentos infantis e maduros se misturam. E, por mais que
haja situações cômicas, como o nomear das coisas, em muitos casos, os
comportamentos mais infantilizados retratados não são tão estranhos a
pessoas mais adultas da contemporaneidade. Especialmente quando tangem à
frustração, ao temor da solidão, às dificuldades em dividir o espaço e
conviver.

Ao fim, a história desses diários é a narrativa de como o amor se
constrói entre esses dois seres tão diferentes. A beleza e a delicadeza
com que ambos se enxergam, mesmo após a perda do paraíso é tocante.
Mesmo os motivos que os levaram à expulsão e até mesmo o pecado de comer
o fruto proibido se tornam relativos.

\Image{Ilustração do livro ``Eve's Diary'', 1907 (Project Gutenberg; Domínio Público)}{PNLD0043-10.png}

\section{Sugestões de atividades complementares: relações dialógicas e
intertextuais}

%\BNCC{EM13LP03}
%\BNCC{EM13LP04}
%\BNCC{EM13LP49}
%\BNCC{EM13LP51}

No Ensino Médio, da mesma forma que no Ensino Fundamental, a \textsc{bncc}
organiza o trabalho com as práticas de linguagem em cinco \textbf{campos
de atuação social}. São eles: campo da vida pessoal, campo da vida
pública, campo jornalístico"-midiático, campo artístico"-literário e campo
das práticas de estudo e pesquisa.

De acordo com essa divisão, propomos na sequência um trabalho
interdiscursivo e intertextual com a obra \emph{Diários de Adão e Eva}.

\subsection{Campo da vida pessoal}

\begin{quote}
O campo da vida pessoal pretende funcionar como espaço de articulações
e sínteses das aprendizagens de outros campos postas a serviço dos
projetos de vida dos estudantes. As práticas de linguagem privilegiadas
nesse campo relacionam"-se com a ampliação do saber sobre si, tendo em
vista as condições que cercam a vida contemporânea e as condições
juvenis no Brasil e no mundo.

Está em questão também possibilitar vivências significativas de práticas
colaborativas em situações de interação presenciais ou em ambientes
digitais e aprender, na articulação com outras áreas, campos e com os
projetos e escolhas pessoais dos jovens, procedimentos de levantamento,
tratamento e divulgação de dados e informações e o uso desses dados em
produções diversas e na proposição de ações e projetos de natureza
variada, para fomentar o protagonismo juvenil de forma
contextualizada. (\textsc{bncc}, p. 494)
\end{quote}


No texto lido, resta claro que uma das chaves para a boa convivência
do casal primordial é a tolerância. Ambos são distintos, enxergam os
defeitos um do outro, mas, apesar disso, detêm amor mútuo. Na verdade,
o que ambas as personagens desenvolveram foi o sentimento de
aceitação, empatia e boa convivência. Entretanto, para que esses se
formem, é necessário que haja, inicialmente, uma predisposição à
tolerância.

Forme uma roda com os alunos, discuta a questão da tolerância. Procure
debater questões como meios para a desconstrução de preconceitos e para
um bem viver em conjunto, tanto no seio da família, quanto no colégio,
ou em sua comunidade. Levante as questões com os jovens e, procure,
coletivamente, construir propostas para a melhoria de pontos que não
estejam ideais.

\subsection{Campo de atuação na vida pública}

\begin{quote}
No cerne do campo de atuação na vida pública estão a ampliação da
participação em diferentes instâncias da vida pública, a defesa dos
direitos, o domínio básico de textos legais e a discussão e o debate de
ideias, propostas e projetos. {[}\ldots{}{]}

Ainda no domínio das ênfases, indica"-se um conjunto de habilidades que
se relacionam com a análise, discussão, elaboração e desenvolvimento de
propostas de ação e de projetos culturais e de intervenção social.
(\textsc{bncc}, p. 494)
\end{quote}

Uma das implicações das distinções sociais fomentadas a partir dos
gêneros, uma das mais comumente vistas é a divisão sexual do trabalho.
Inicialmente, à mulher era reservado um papel doméstico, recebendo o
homem o papel de ``ganha-pão'', sem o qual uma família não teria
condições de receber um bom sustento, ou mesmo uma boa orientação.
Essa visão arcaica há muito caiu por terra. Entretanto, mesmo hoje,
mesmo nos países mais igualitários, as mulheres quase sempre recebem
um salário menor que os homens. Com isso em mente, solicite aos alunos
que produzam um texto investigativo-jornalístico, onde deverão
abordar, o histórico da divisão sexual do trabalho, dando ênfase ao
momento em que as mulheres passam a atuar contundentemente no mercado
de trabalho; a circunstância atual e casos concretos onde haja grande
discrepância salarial; o planejamento e implementação de politicas de
equiparação; e, por fim, narrar como é a realidade da questão na
comunidade em que habitam. O material produzido pode ser veiculado no
jornal da escola, ou ser postado no site do colégio.


\subsection{Campo jornalístico"-midiático}

\begin{quote}
Em relação ao campo jornalístico"-midiático, espera"-se que os jovens
que chegam ao Ensino Médio sejam capazes de: compreender os fatos e
circunstâncias principais relatados; perceber a impossibilidade de
neutralidade absoluta no relato de fatos; adotar procedimentos básicos
de checagem de veracidade de informação; identificar diferentes pontos
de vista diante de questões polêmicas de relevância social; avaliar
argumentos utilizados e posicionar"-se em relação a eles de forma ética;
identificar e denunciar discursos de ódio e que envolvam desrespeito aos
Direitos Humanos; e produzir textos jornalísticos variados, tendo em
vista seus contextos de produção e características dos gêneros. Eles
também devem ter condições de analisar estratégias
linguístico"-discursivas utilizadas pelos textos publicitários e de
refletir sobre necessidades e condições de consumo.

No Ensino Médio, os jovens precisam aprofundar a análise dos interesses
que movem o campo jornalístico midiático, da relação entre informação e
opinião, com destaque para o fenômeno da pós"-verdade, consolidar o
desenvolvimento de habilidades, apropriar"-se de mais procedimentos
envolvidos na curadoria de informações, ampliar o contato com projetos
editoriais independentes e tomar consciência de que uma mídia
independente e plural é condição indispensável para a democracia.

Como já destacado, as práticas que têm lugar nas redes sociais têm
tratamento ampliado. (\textsc{bncc}, p. 494-495)
\end{quote}

Na contemporaneidade, muito se discute a respeito da veracidade das
informações e, também, das qualidades das fontes que a embasaram. A
propagação de notícias falsas não é exclusiva dos tempos modernos.
Desde muito cedo na história da humanidade, percebeu-se o efeito
social e político que uma notícia poderia ter. Podia fazer com que as
pessoas desconfiassem de seu rei, ou então podia inflamar a população
para guerra.

É verdade também que veículos de mídia podem ter preferências políticas,
de acordo com a pessoa ou o grupo aos quais eles pertencem. Assim sendo,
um mesmo fato pode ser abordado de distintas maneiras.

Como exemplo histórico, sugira a leitura dos trechos a seguir, um
extraído de um texto produzido pela cote do rei assírio Senaqueribe, e,
o outro, um trecho do velho testamento:

\begin{itemize}
\item
O rei Ezequias recusou-se a se submeter, eu cerquei as quarenta e seis
cidades de seu reino e as tomei. duzentas pessoas, entre homens,
mulheres e crianças, camelos, mulas, asno, cavalos e um incontável
número de gado bovino e ovino eu tomei como espólio. O próprio
Ezequias foi engaiolado como um pássaro. Mandei colocar sua jaula na
porta de Jerusalém, onde nele eu atirei vasos e cerâmicas. Ezequias
contemplou minha majestade enquanto eu lhe impunha o tributo de trinta
talentos de ouro e oitocentos talentos de prata e, além disso,
entregava as cidades de seu reino para o domínio de Padi e levava à
Nínive sua filha, seu harém, seus músicos, ouro, prata, marfim e
joias.''\footnote{Prisma de Senaqueribe (c. 689 a.C.). Material: Argila. 6 faces -- 38 cm de altura por 14 cm de largura. Idioma: Acadiano -- escrito em
cuneiforme. Localização atual -- Instituto Oriental em Chicago --
EUA.}

\item
  Assim diz o Senhor acerca do rei da Assíria: Não entrará nesta cidade,
  nem lançará nela flecha alguma; tampouco virá perante ela com escudo,
  nem levantará contra ela trincheira alguma. Pelo caminho por onde
  vier, por ele voltará; porém nesta cidade não entrará, diz o Senhor.
  Porque eu ampararei a esta cidade, para a livrar, por amor de mim e
  por amor do meu servo Davi. Sucedeu, pois, que naquela mesma noite
  saiu o anjo do Senhor, e feriu no arraial dos assírios a cento e
  oitenta e cinco mil deles; e, levantando-se pela manhã cedo, eis que
  todos eram cadáveres. Então Senaqueribe, rei da Assíria, partiu, e se
  foi, e voltou e ficou em Nínive.\footnote{2 Reis 19:1-37.}
\end{itemize}

Perceba como ambas narram o mesmo evento, mas cada uma conta um desfecho
completamente distinto.

Dito isso, para a atividade, observe com os alunos como uma mesma
notícia pode ter matizes distintas conforme o veículo que a divulga.
Sugere-se a procura de um mesmo tema em diversas publicações brasileiras
e estrangeiras. Veja que elas podem ter vieses diferentes.

A partir disso, sugira que os alunos produzam um texto dissertativo, no
qual deve ser discutido a questão da importância da informação e suas
implicações. E os perigos da manipulação dos fatos. Para enriquecer a
argumentação, indique outras publicações e casos famosos onde a atuação
de um veículo de informação gerou grande consequências políticas e
sociais. Também recomenda-se a exibição do TED o \emph{Perigo de uma
Única História}, exposto por Chimamanda Adichie.

\subsection{Campo artístico"-literário}

\begin{quote}
No campo artístico"-literário busca"-se a ampliação do contato e a
análise mais fundamentada de manifestações culturais e artísticas em
geral. Está em jogo a continuidade da formação do leitor literário e do
desenvolvimento da fruição. A análise contextualizada de produções
artísticas e dos textos literários, com destaque para os clássicos,
intensifica"-se no Ensino Médio. Gêneros e formas diversas de produções
vinculadas à apreciação de obras artísticas e produções culturais
(resenhas, vlogs e podcasts literários, culturais etc.) ou a formas de
apropriação do texto literário, de produções cinematográficas e teatrais
e de outras manifestações artísticas (remidiações, paródias,
estilizações, videominutos, fanfics etc.) continuam a ser considerados
associados a habilidades técnicas e estéticas mais refinadas.

A escrita literária, por sua vez, ainda que não seja o foco central do
componente de Língua Portuguesa, também se mostra rica em possibilidades
expressivas. (\textsc{bncc}, p. 495-496).
\end{quote}

A partir das atividades 2 e 4, em que os alunos entraram em contato
com obras artísticas que retratam o Éden, bem como montaram um cenário
de acordo com o descrito por Twain, solicite que cada aluno faça uma
lista, de coisas que considera fundamentais na sua própria visão de
paraíso. Com essa lista em mãos, peça para que cada aluno realize uma
representação visual de seu paraíso individual. Isso pode ser feito
por meio de fotografia, desenho, maquete, computação gráfica,
escultura, etc. Apenas indique para que o estudante esteja atento à
lista que previamente formulou. O produto final poderá ser exposto no
colégio, ou então pode ser desenvolvido um acervo virtual a ser
veiculado na página da escola.

\subsection{Campo das práticas de estudo e pesquisa}

\begin{quote}
O campo das práticas de estudo e pesquisa mantém destaque para os
gêneros e habilidades envolvidos na leitura/escuta e produção de textos
de diferentes áreas do conhecimento e para as habilidades e
procedimentos envolvidos no estudo. Ganham realce também as habilidades
relacionadas à análise, síntese, reflexão, problematização e pesquisa:
estabelecimento de recorte da questão ou problema; seleção de
informações; estabelecimento das condições de coleta de dados para a
realização de levantamentos; realização de pesquisas de diferentes
tipos; tratamento dos dados e informações; e formas de uso e
socialização dos resultados e análises.

Além de fazer uso competente da língua e das outras semioses, os
estudantes devem ter uma atitude investigativa e criativa em relação a
elas e compreender princípios e procedimentos metodológicos que orientam
a produção do conhecimento sobre a língua e as linguagens e a formulação
de regras. (\textsc{bncc}, p. 495-496)
\end{quote}

Além de aspectos sociais e psicológicos, a questão fisiológica é outro
fator que deve ser levado em conta quando questões de gênero são
debatidas. Esse aspecto, inclusive, e por sua própria natureza, tem
fator preponderante na definição do gênero biológico. Nesse sentido, é
interessante que os alunos tenham maior conhecimento do aparato
genético que define o sexo, bem como as influências hormonais que
cromossomos os X e Y direcionam no organismo. Para essa atividade,
sugira uma investigação, contando com o auxílio do professor de
ciências da natureza, sobre genética e fisiologia, a partir da qual os
alunos coletarão informações sobre as recentes descobertas científicas
a respeito, bem como entenderão melhor o funcionamento de nosso
organismo, além de conseguirem, com isso, ter um maior conhecimento da
evolução de nossa espécie, tendo contato com informações, como por
exemplo, a possibilidade que temos de reconstruir a história das
migrações da humanidade a partir do DNA mitocondrial.

\section{Referências complementares}

\begin{itemize}
\item\textsc{andrade}, Carlos Drummond de. \textit{Amar se aprende amando: Poesia de
convívio e de humor}. São Paulo: Companhia das Letras, 2018.

O que encontramos nos 68 poemas que compõem este volume é o fruto do
esforço do poeta em conciliar sentimento e experiência, em um percurso
pelas diferentes formas que o sentimento amoroso assume em sua poética.

\item\textsc{bauman}, Zygmunt. \textit{Amor líquido: Sobre a fragilidade dos laços
humanos}. Rio de Janeiro: Zahar, 2004. 

A modernidade líquida, "um mundo repleto de sinais confusos, propenso a
mudar com rapidez e de forma imprevisível" em que vivemos, traz consigo
uma misteriosa fragilidade dos laços humanos, um amor líquido.

\item\textsc{milton}, John. \textit{Paraíso perdido}. São Paulo: Editora 34, 2019.

O célebre poema épico de Milton, publicado originalmente em inglês,
apresenta uma das mais famosas releituras literárias da história de Adão
e Eva.

\item\textsc{twain}, Mark.\textit{Autobiografia de Mark Twain}. Belo Horizonte:
Itatiaia, 1961.

O registro da vida de um dos maiores autores norte-americanos, escrito
por ele mesmo, permite contextualizar sua vida e sua obra.
\end{itemize}

\section{Bibliografia comentada}

\begin{itemize}
\item\textsc{rougemont}, Denis De. \textit{O Amor e o Ocidente}. Rio de Janeiro:
Guanabara, 1988.

Denis de Rougemont acreditava que o casamento vivia uma crise no início
do século XX e em seu livro pretende apontar os culpados por ela. Sua
tese é sobre a antítese amor e paixão.

\item\textsc{campbell}, Joseph. \textit{O Poder do Mito}. São Paulo: Palas Athena,
2014.

O livro é o fruto de uma série de conversas mantidas entre Joseph
Campbell e o jornalista Bill Moyers, numa combinação de sabedoria e
humor, sobre o casamento, os nascimentos virginais, a trajetória do
herói e o sacrifício ritual.

\item\textsc{greenblatt}, Stephen. \textit{Ascensão e queda de Adão e Eva}. São Paulo: Companhia das Letras, 2018.

O autor apresenta a trajetória do mito de Adão e Eva ao longo do tempo,
na religião, na literatura e nas artes visuais.

\item\textsc{paz}, Octavio. \textit{A Dupla Chama}. São Paulo: Mandarim Editora, 1999.

O livro fornece uma história social e literária do amor e do erotismo,
comparando as manifestações modernas com as épocas anteriores, embora
observando a relação especial entre erotismo e poesia.
\end{itemize}

\end{document}


