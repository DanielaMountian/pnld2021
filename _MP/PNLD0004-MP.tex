\documentclass[12pt]{extarticle}
\usepackage{manualdoprofessor}
\usepackage{fichatecnica}
\usepackage{lipsum,media9,graficos}
\usepackage[justification=raggedright]{caption}
\usepackage[one]{bncc}
\usepackage[madalena]{../edlab}


\begin{document}

%Precisa diminuir o título do livro no final de todas as páginas.

\newcommand{\AutorLivro}{Mário de Andrade}
\newcommand{\TituloLivro}{Zeppelin: algumas crônicas de Mário de Andrade}
\newcommand{\Tema}{Ficção, mistério e fantasia}
\newcommand{\Genero}{Conto, crônica e novela}
\newcommand{\imagemCapa}{./images/PNLD0004-01.png}
\newcommand{\issnppub}{---}
\newcommand{\issnepub}{---}
% \newcommand{\fichacatalografica}{PNLD0004-00.png}
\newcommand{\colaborador}{Rodrigo Jorge Ribeiro Neves}


\title{\TituloLivro}
\author{\AutorLivro}
\def\authornotes{\colaborador}

\date{}
\maketitle

\begin{abstract}\addcontentsline{toc}{section}{Carta ao professor}
Este manual tem o objetivo de auxiliá-lo no desenvolvimento de práticas
pedagógicas que estabeleçam o diálogo entre a obra literária aqui
apresentada e os estudantes, de modo a ampliar não apenas a leitura do
texto em si, mas também a sua relação com o mundo.

O livro \emph{Algumas crônicas} consiste em uma antologia de crônicas de
Mário de Andrade, publicadas em diversos jornais e revistas, como
\emph{Diário Nacional} e \emph{O Estado de S. Paulo}. A maioria dos
textos selecionados foram extraídos do livro \emph{Os filhos da
Candinha} (1943); os demais, de obras organizadas por especialistas no
autor, como \emph{Táxi e crônicas no Diário Nacional} (1976) e
\emph{Será o Benedito!} (1992). Nas crônicas, Mário apresenta alguns dos
principais elementos que caracterizam sua literatura, como a relação
entre as linguagens escrita e falada, a cultura popular, o conflito de
classes, os contrastes sociais da modernização e a busca por uma
identidade nacional. Em alguns textos, atua também como crítico,
colocando em cena nomes importantes da literatura, das artes plásticas e
da música.

Mário de Andrade é conhecido por sua atuação em diversos gêneros e
áreas do saber, e a como cronista foi uma das mais
exercitadas. Assim, você tem a chance de enriquecer as suas aulas
apresentando também uma das facetas instigantes desse grande escritor
brasileiro, tão importante para a valorização da nossa cultura e para
que possamos, desta maneira, nos (re)conhecer cada vez mais.




Para isso, apresentamos aqui propostas de atividades, aprofundamento,
referências complementares e uma bibliografia comentada, a fim de que o
material possa ser útil nas suas aulas para estimular os estudantes a
desbravar um universo de possibilidades através de um dos escritores
mais múltiplos da nossa literatura. Além disso, é ótimo trabalhar com
crônicas em sala de aula, pois é um gênero literário bastante fecundo e
que, pela sua forma curta, possibilita a dinamização das atividades e a
exploração de uma variedade maior de temas para discussão com os
estudantes.

Aproveite bastante este material. Ele foi feito com muita dedicação e
carinho para você! Boa aula!
\end{abstract}

\tableofcontents


\section{Propostas de atividades I}

\BNCC{EM13LP30}
\BNCC{EM13LGG202}


\subsection{Pré-leitura}

\paragraph{Tema} A crônica e sua relação com as narrativas produzidas nas
  redes sociais.

\paragraph{Conteúdo} Discussão sobre a atualidade do gênero crônica a
partir de experiências com as narrativas construídas nas redes sociais
pelos usuários, como, por exemplo, Twitter, Instagram, TikTok e YouTube.

\paragraph{Objetivo} Estimular e habilitar os estudantes a reconhecer e
compreender os principais elementos e as estratégias estéticas e
narrativas da crônica como gênero literário, tendo em vista os relatos
do cotidiano que eles acompanham e/ou constroem nas principais redes
sociais.

\paragraph{Justificativa} Não é nenhuma novidade que as redes sociais e
outras plataformas do mundo virtual vêm ocupando cada vez mais o
cotidiano de todos nós, especialmente dos estudantes, que vêm crescendo
e se formando na chamada Web 2.0. Tendências, estilos e pontos de vista
são profundamente influenciados pelo contato dos jovens com essa
realidade (ou seria multi-realidade?), conferindo novas maneiras de
perceber, sentir, se expressar e se posicionar no mundo.
\SideImage{Mário de Andrade com 35 anos, em 1928. (Michelle Rizzo; Domínio Público)}{PNLD0004-03.png}

Mas nem tudo é apenas novidade. Embora os meios de comunicação e
as experiências dos indivíduos tenham sido transformados nas últimas
décadas, há elementos e estruturas que são mais antigos que os nossos
próprios avós. Mário de Andrade, em resposta aos críticos do suposto fim
dos ``assuntos poéticos'' no modernismo, disse que o amor ainda existia,
só que passou a andar de automóvel. Parafraseando o escritor paulista,
diríamos, então, que o amor ainda existe, mas agora chega pela WhatsApp.
Ou seja, há situações que nunca mudam, apenas são atualizadas as formas
como são apresentadas. As redes sociais, ainda que sejam uma realidade
completamente diferente daquela que vivíamos há poucas décadas atrás,
conservam práticas, pensamentos e sensibilidades desde épocas
imemoriais, refletindo a sociedade em suas principais dimensões.
\SideImage{Casa em que morou Mário de Andrade no Largo do Paissandu. (Arquivo Nacional; Domínio Público)}{PNLD0004-04.png}




A crônica é um gênero moderno e cotidiano, sem contas a prestar com
qualquer tentativa de sistematização. Suas fronteiras não têm limites,
todos são bem-vindos! É como uma conversa despretensiosa com um
desconhecido no ônibus lotado, o papo entre colegas de trabalho em uma
mesa de bar ou uma confissão, ao telefone, com o melhor amigo. Assim, as
suas estratégias se aproximam das principais redes sociais, em suas
tentativas de narrar o dia a dia dos seus usuários ou desabafar sobre um
problema qualquer. Por isso, é fundamental que o estudante, com a
proposição e adequada mediação do professor, discuta a aproximação do
gênero crônica com os usos dessas plataformas.

\paragraph{Metodologia} Como ponto de partida, o professor pode começar
perguntando aos estudantes o que é uma crônica, sem se preocupar com uma
definição teórica ou determinada nesse momento. Anotar na lousa as
definições mais recorrentes dada pelos participantes da aula ou pedir
para que eles as registrem e mantenham para a etapa posterior da
discussão são boas opções. O mais importante, nesse momento, é elaborar,
coletivamente, uma definição do gênero literário tendo em vista as
experiências de cada um. Ainda que ninguém tenha lido uma crônica na
vida (algo bem difícil de acontecer), em sentido estrito, é fundamental
que o educador os encoraje a dizer o que eles entendem como um texto do
gênero, pois, muitas vezes, os contatos se deram por meio de gêneros
fronteiriços ou híbridos. E essa percepção é bastante útil na presente
atividade, pois estimula os estudantes a pensar no texto levando em
conta não a sua forma acabada e fechada, mas o seu processo de
construção e os elementos estruturais que o compõem, ou seja, da parte
para o todo.
\SideImage{Fotografia de desenho de Mário de Andrade de 1928. (Arquivo Nacional; Domínio Público)}{PNLD0004-08.png}

É muito importante que, durante essa discussão, o educador procure
sempre exemplificar com textos, de preferência outros que não sejam
desta coletânea, a fim de que o processo de aprendizagem seja ampliado.
Claro que é imprescindível que a conversa se desenvolva com textos
conhecidos dos estudantes, principalmente levando em conta as sugestões
dadas por eles, mas não se deve perder, de maneira alguma, a
oportunidade de sugerir novas obras, incutindo, sempre, a curiosidade e
o gosto pela leitura.

Após essa etapa introdutória, o professor pode sugerir aos estudantes
que tentem, a partir dos resultados anotados na conversa sobre a
definição do gênero crônica, relacionar cada item com os recursos das
redes sociais que eles conhecem e/ou utilizam. Naturalmente as
definições vão ser reformuladas de acordo com a plataforma comparada, de
maneira que elas se adequem as suas respectivas funcionalidades. Por
isso, o educador precisa ficar atento para que não se perca de vista o
objetivo da atividade, que é reconhecer e compreender como os elementos
compositivos da crônica, enquanto gênero literário, pode estar presente
nos recursos das redes sociais, ou seja, elas também podem ser espaços
de criação de crônicas pelos seus relatos do cotidiano.

Peça para que os estudantes formem grupos ou duplas e proponha que
utilizem os recursos de qualquer rede social para adaptar uma crônica de
Mário de Andrade. Nesse caso, quem sabe o grupo ou dupla proponha o
piloto de uma série? Todas as propostas são bem-vindas. Na aula em que
os trabalhos vão ser apresentados, incentive que todos discutam as
crônicas adaptadas para as plataformas virtuais, apontando os elementos
que se aproximam do gênero literário e os que se afastam e/ou colocam 
o gênero em xeque, testando os limites não apenas da crônica, mas também das
redes sociais como espaços de produção e compartilhamento de relatos do
cotidiano. Mostre que a crônica não tem limites, não apenas em relação à
temática, mas também quanto à forma. Elas podem tanto narrar uma pequena
história sobre um episódio quanto vir em forma de carta a alguém, ou se
apresentar como prosa poética, ou até mesmo em forma de um conto. As
possibilidades são inúmeras.

\paragraph{Tempo estimado} Duas aulas de 50 minutos.

\subsection{Leitura I}

\BNCC{EM13LP15}
\BNCC{EM13LP09}

\paragraph{Tema} Pequenas biografias do cotidiano através da crônica.
 
\paragraph{Conteúdo} Produção de pequenas biografias ou perfis biográficos
de personagens do cotidiano descritos em crônicas de Mário de Andrade.

\paragraph{Objetivo} Habilitar os estudantes, por meio da leitura dos
textos de Mário de Andrade, a compreender a importância dos tipos
humanos nas relações cotidianas e estimular a criatividade na maneira de
ler o mundo.

\paragraph{Justificativa} Escrever uma vida é um desafio. Como abarcar
todas as dimensões da vida de um indivíduo em algumas palavras escritas,
que contenham uma sintaxe e uma série de verbos, substantivos e adjetivos
que, de certa maneira, são uma tentativa de delimitá-la em algumas
linhas? Mas escrever uma vida é diferente de escrever sobre uma vida.
Quando submetemos o indivíduo ao texto, estamos, sim, definindo limites
que necessariamente estão ali. No entanto, quando escrevemos a vida que
buscamos retratar, estamos também conferindo outras possibilidades de
compreendê-la, enxergá-la, inseri-la no mundo. Escrever uma vida não é
apenas reinventá-la. Escrever uma vida é conferir uma dimensão à própria
escrita.

Por ser um gênero do cotidiano, e não há cotidiano sem o elemento
humano, a crônica mobiliza as vidas comuns do dia a dia, seja a do
cronista, seja a dos personagens colocados em cena por ele. Por isso,
esta atividade é importante na formação crítica e criativa dos alunos,
não apenas porque busca estimular a imaginação e exercitar, assim, suas
capacidades de contar histórias a partir de episódios simples de suas
rotinas, mas também porque intenta promover sua inserção na comunidade
em que vivem e despertar a percepção do papel que todos possuem nas
relações sociais.



\paragraph{Metodologia} 

\begin{enumerate}

\item
Inicialmente, você pode abrir uma roda de
conversa sobre o que constitui uma biografia, como escrever sobre a vida
de alguém. É possível apreender todos os aspectos da vida de um
indivíduo a partir de palavras determinadas que tentem descrevê-lo?
Quais são as dimensões mais importantes da vida de alguém que devem ser
consideradas na hora de escrever sobre elas? Peça para que os estudantes
tentem descrever a vida de alguém próximo, de preferência com quem eles
tenham contato no dia a dia. Pode ser um parente, amigo ou funcionário
de um estabelecimento que eles frequentem, como a própria escola, por
exemplo.

\item
Organize a turma em duplas ou grupos. Eles devem selecionar um ou
mais personagens citados nas crônicas de Mário de Andrade, mas não podem ser
nomes famosos, como os artistas e intelectuais citados pelo escritor em
alguns textos. A seguir, sugerimos algumas crônicas que podem ser
utilizados nesta etapa, mas você pode também realizar essa seleção
depois de sua leitura com os educandos:

\begin{itemize}
\item
  ``Educai vossos pais'';
\item
  ``Conversa à beira do cais'';
\item
  ``A pesca do dourado'';
\item
  ``Romances de aventura'';
\item
  ``Sociologia do botão;
\item
  ``Meu engraxate'';
\item
  ``A sra. Stevens'';
\item
  ``Fábulas'';
\item
  ``Idílio novo''.
\end{itemize}

\item
Os estudantes podem escolher até mesmo figuras diluídas em uma
multidão indeterminada. Eles devem relacionar o perfil biográfico do
indivíduo (ou indivíduos) com o episódio relatado na crônica escolhida.
Exemplifique com os textos escritos para os livros informando os dados
de seu autor. A forma como eles irão apresentar o texto biográfico é
livre, desde que relacione a vida escrita com a crônica selecionada.
Pode ser atribuída uma imagem retirada de alguma publicação ou, a
depender da vontade do aluno, ele mesmo pode fazer um retrato do
biografado.
\end{enumerate}

\paragraph{Tempo estimado} Duas aulas de 50 minutos.

\subsection{Leitura II}

\BNCC{EM13LGG401}
\BNCC{EM13CHS601}

\paragraph{Tema} Aproximações e diferenças da crônica com outros gêneros em prosa.


\paragraph{Conteúdo} Compreensão das relações estruturais entre a crônica
e os demais gêneros literários em prosa, como o conto, a novela e o
romance, evidenciando a inserção de cada um na realidade
representada pela obra literária.

\paragraph{Objetivo} Estimular e habilitar os estudantes a perceber os
vínculos estruturais entre os principais gêneros literários em prosa com
a crônica, a fim de que compreendam as relações entre a realidade e a
forma pela qual ela demanda que seja representada.

\paragraph{Justificativa} A literatura, assim como toda obra de arte, é a 
representação da realidade social por meio da expressão subjetiva, a fim
de que possamos compreender essa realidade e, de alguma maneira,
transformá-la. Para isso, existe uma forma que busca corresponder às
questões suscitadas por ela. Às vezes, é um poema, uma canção, um
quadro; outras, uma performance, um vídeo, um romance, uma série de TV.
Claro que a formação e as preferências individuais do artista, moldadas
pelas suas origens e história de vida, vão determinar os meios que ele
irá escolher para expressar sua inquietação diante do mundo. No entanto,
o próprio mundo ao qual ele se debruça exigirá um olhar particular. 
Dentro de cada obra de arte, há uma forma de expressão com um conjunto de elementos,
que indicam o modo de funcionamento daquela linguagem para expressar o
mundo a ser representado.

A escolha de um gênero literário nunca é arbitrária. Para Mário de
Andrade, em uma ``Advertência'' em \emph{Os filhos da Candinha},
escrever crônica era sua ``aventura intelectual'', sua ``válvula
verdade'' por onde ele podia se ``desfatigar'' de si mesmo, por onde ele
``brincava de escrever''. Isso não o impediu de escrever sobre temas
sérios, como exposições de literatura, de artes plásticas e da
importância do patrimônio artístico e cultural. No entanto, serviu para
que ele encontrasse no pequeno gesto da inquietação do dia a revelação
ou aceitação do mistério que os grandes feitos não conseguem devassar.

A discussão sobre essas relações pode ajudar o estudante a compreender o
projeto envolvido na concepção das crônicas de Mário de Andrade.

\paragraph{Metodologia}

\begin{enumerate}

\item
A partir das crônicas selecionadas desta
coletânea, peça para que os estudantes sugiram qual (ou quais) poderia
ser um bom material para um romance e por quê. Quais elementos contidos no
artigo de jornal escolhido podem ser ampliados para que se desenvolvam
em uma narrativa mais longa? São necessários mais personagens? Quais e
como eles caracterizariam esses personagens adicionais? De que modo eles
se relacionariam com os personagens da crônica e em que medida estes
seriam modificados para que componham o romance?

\item
Eles podem também investigar as relações entre a crônica e outros
gêneros em prosa de extensão mais curta, como a novela ou o conto. Há
textos de \emph{Algumas crônicas} com elementos de narrativa curta
ficcional. De certa maneira, a literatura de Mário de Andrade é
atravessada pela influência de diversos gêneros, então, é comum que as
fronteiras entre eles sejam, às vezes, difíceis de delimitar. Certas
histórias acabam servindo também como espécie de laboratório para o
desenvolvimento de questões estéticas e culturais mais amplas. Na
crônica ``O diabo'', por exemplo, temos o personagem ``Belazarte'',
criado por Mário de Andrade em suas ``Crônicas de Malazarte'', série de
pequenas narrativas ficcionais publicada na revista \emph{América
Brasileira}, entre 1923 e 1924. O personagem reaparece depois em
\emph{Os contos de Belazarte} (1934). Os estudantes podem comparar a
crônica com alguma dessas narrativas e discutir as relações na
construção da personagem em cada uma delas.


\SideImage{Paulicéia Desvairada é um dos livros que originaram os contos presentes nessa antologia. (Biblioteca Brasiliana Guita e José Mindlin; Domínio Público)}{PNLD0004-05.png}


\item
Proponha uma discussão sobre a intertextualidade entre a crônica e
outros gêneros literários. Você pode começar com esta crônica que
acabamos de comentar. Inicialmente, os grupos podem fazer uma roda de
conversa sobre a crônica escolhida, como em um clube de leitura, expondo
seus principais elementos. Depois cada um pode apresentar uma proposta
de outro gênero a partir dessa crônica. A linguagem escolhida para a
apresentação fica a critério dos educandos. Não há limites para a
imaginação! O importante é convencer os demais que o texto escolhido
pode ser transformado em um outro gênero literário.
\end{enumerate}

Depois da atividade, você, professor, pode retomar as perguntas
iniciais. Os alunos podem responder em aula ou entregar por escrito, em
uma ou duas páginas no máximo.

\paragraph{Tempo estimado} Duas aulas de 50 minutos.

\subsection{Pós-leitura}

\BNCC{EM13LGG601}
\BNCC{EM13LP51}

\paragraph{Tema} A crônica como espaço de pesquisa e criação.

\paragraph{Conteúdo} Realização de uma oficina de criação literária para a
produção de crônicas, tomando como referência os textos selecionados de
Mário de Andrade para \emph{Algumas crônicas}. A observação do pormenor,
as referências culturais e os interesses individuais devem ser
exercitados.

\paragraph{Objetivo} Habilitar os estudantes a empreender uma leitura
ativa da crônica e estimular a sua criatividade a partir da observação
de eventos cotidianos, valorizando a sua relação com os elementos mais
comuns de sua rotina.

\paragraph{Justificativa} Um dos nossos maiores cronistas, Rubem Braga,
afirmou, certa vez, que, se é aguda, não é crônica. A frase cômica é
reveladora de um dos traços mais marcantes do gênero, que é não se levar
tão a sério. Isso não significa que todas as crônicas devem,
necessariamente, flertar com a comédia, mas sim que ela
tem uma leveza do cotidiano simples que nos acompanha desde a hora que
preparamos o nosso café pela manhã até a escolha do jantar numa visita
inevitável ao supermercado. Por sua natureza, portanto, a crônica pode
comportar todos os gêneros, até mesmo os mais sérios. E por seu olhar
simples sobre as pequenas coisas, a sua prática pode ser um excelente
exercício não apenas de escrita, mas de vivência.

A dinâmica de uma oficina literária permite não apenas refletir e
discutir sobre esses registros de inquietudes do cotidiano por meio de
um texto curto, mas também aguçar a criatividade, a percepção, a
sensibilidade e a visão de mundo dos educandos. A atividade não é para
ser uma espécie de ``agência de caça-talentos'' literária, claro. Não que
não seja possível que exista entre os alunos uma sementinha de uma
grande escritora ou de um grande escritor. Se houver, ótimo! Contudo, o
mais importante é que todos possam exercitar ativamente sua capacidade
de criar e recriar visões de mundo a partir de uma história, aprendendo
que escrever é também uma forma crítica de leitura.

\paragraph{Metodologia} Esta atividade pode ser o projeto de algo mais
amplo e, quem sabe, permanente dentro da escola, abrangendo outros
gêneros literários e outras turmas. Evidentemente nada impede que você
já exercite a prática de outros gêneros com essa proposta de oficina. No
entanto, como primeiro passo, seria interessante dedicarmos a atenção à
crônica, que está dentro do escopo deste manual e de suas propostas de
atividades.

Para a criação da oficina literária de crônicas, você pode seguir os
passos a seguir, adaptando de acordo com o lugar, o tempo e as condições
disponíveis para que a atividade seja desenvolvida:

\begin{enumerate}
\item
  Organize os alunos em uma roda, a fim de que a atividade de construção
  de relatos do cotidiano seja, sobretudo, uma atividade de
  compartilhamento de experiências, que é, afinal, o fundamento de toda
  concepção de um texto literário;
\item
  Embora a escrita seja considerada, em princípio, como uma empreitada
  individual, nada impede que também seja coletiva, portanto, permita
  que grupos de dois ou mais estudantes sejam formados. É uma boa
  oportunidade, aliás, para discutirem a questão da autoria na criação
  literária. Apenas limite o número de integrantes, a fim de que a
  atividade seja proveitosa para todos. Um grupo formado por um número
  muito grande de alunos pode acabar dispersando-os e frustrando o
  objetivo da oficina;
\item
  Inicialmente, discuta com os alunos os vários tipos de crônicas,
  utilizando os exemplos da antologia de Mário de Andrade e de outras
  recortadas do jornal, em blogs na internet ou publicadas em livro. As
  crônicas podem ser jornalísticas, históricas ou literárias, ou até
  mesmo atravessar esses aspectos. Os alunos devem definir os temas que
  pretendem desenvolver, a fim de que recolham os elementos necessários
  para começar a trabalhar em seus textos;
\item
  Os jovens cronistas devem discutir estratégias para que seu texto
  desperte o interesse de seu leitor. Portanto, quais elementos seriam
  indispensáveis, além de uma boa escrita? O surpreendente, o inusitado
  ou até mesmo o sobrenatural podem funcionar como ótimos recursos para
  construir uma boa crônica;
\item
  Se o cronista optar pelo caráter informativo, recorte exemplos do
  noticiário, mas é preciso definir um tema, para que o texto não se
  perca em digressões;
\item
  É importante também discutir construção de personagens, seja para os
  textos ficcionais ou referenciais. Recolha exemplos das crônicas de
  Mário de Andrade ou de outros cronistas, identificando de que maneira
  o personagem se articula com o relato construído e as características
  de cada um deles são desenvolvidas;
\item
  A crônica ``Zeppelin'', de Mário de Andrade, propõe um exercício de
  criação no fim. Utilize como atividade para que os estudantes possam
  desenvolver a criatividade;
\item
  O gênero crônica, como nos ensina Mário de Andrade, também pode ser um
  espaço de pesquisa, ou, como ele nos advertiu, ``uma aventura
  intelectual''. Portanto, incentive também que eles realizem um dedicado
  trabalho de pesquisa sobre o tema escolhido;
\item
  À medida que os encontros da oficina avançarem, incentive sempre que
  eles apresentem parte do desenvolvimento de seus textos, ainda que
  estejam incompletos ou empacados em algum ponto determinado. A
  discussão em sala de aula pode contribuir para que eles solucionem
  alguns desses problemas;
\item
  Quando finalizadas, as histórias podem ser apresentadas da forma como
  cada aluno ou grupo preferir. Os textos podem ser reunidos de diversas
  maneiras, a depender dos recursos que estiverem disponíveis. Se houver
  a possibilidade de publicação do material impresso, não hesite, claro,
  mas não deixe de estimular a divulgação do trabalho através dos
  incontáveis meios que o universo digital oferece, como blogs,
  Instagram e Twitter, por exemplo.
\end{enumerate}

\paragraph{Tempo estimado} dois bimestres.

\section{Propostas de atividades II}

\subsection{Pré-leitura}

\BNCC{EM13LGG202}
\BNCC{EM13CHS102}
\BNCC{EM13CHS104}

\paragraph{Tema} A cultura popular na literatura e em outras artes.

\paragraph{Conteúdo} Discussão e reflexão sobre as relações entre cultura
popular e produção artística em geral, bem como as formas pelas quais esses
dois campos se influenciam reciprocamente, tendo em vista as
transformações que a categoria do ``popular'' atravessou com o passar do
tempo.

\paragraph{Objetivo} Habilitar os estudantes a analisar, discutir e
compreender a importância da cultura popular na construção da nossa
identidade por meio da literatura e das outras artes.

\paragraph{Justificativa} A cultura popular está presente em diversas
expressões artísticas. Ela não está amarrada a uma determinada escola ou
estilo e é atemporal, pois é de todos os tempos. Em geral, associamos
o popular a algo sem sofisticação, pitoresco, rudimentar e carregado de
certo exotismo, sempre de forma pejorativa. É comum também separar, de
maneira quase delimitada, as manifestações produzidas pela cultura
popular e pela cultura erudita. Nada mais enganoso. Embora sejam
distintos em vários aspectos, o popular e o erudito estão imbricados de
maneira dinâmica e, assim, se contagiam mutuamente em fronteiras
tênues.

Na arte brasileira, o modernismo, especialmente através da produção
literária e ensaística de Mário de Andrade, resgatou a importância da
cultura popular na formação da nossa sociedade, tornando-se elemento
fundamental na busca pela identidade nacional, um dos pressupostos defendidos pelo movimento. Por isso, esta atividade pode contribuir não apenas para a área de língua portuguesa, mas também para outras áreas das ciências humanas e sociais aplicadas.


\Image{Bloco de Maracatu em Olinda (PE), uma forma de expressão da cultura popular brasileira. (Tetraktys; CC-BY-SA-3.0)}{PNLD0004-12.png}


\paragraph{Metodologia} 

\begin{enumerate}
\item
Estabeleçam uma conversa inicial sobre o que
seria cultura popular e o que a diferencia da cultura erudita, além de
tentar apontar as relações entre elas. Dentro dessa discussão, pode ser
colocada também a questão entre arte e artesanato, artista e artesão, o
que eles possuem de semelhante e no que se distanciam. Você pode propor
esse debate a partir dos filmes relacionados a seguir, ou qualquer um de
sua escolha, desde que esteja dentro da temática de nossa atividade:

\begin{itemize}
\item
  \emph{Moacir Arte Bruta} (2005), de Walter Carvalho;
\item
  \emph{O Auto da Compadecida} (2000), de Guel Arraes;
\end{itemize}

\item
Peça para que eles redijam um texto curto, de no máximo 2 páginas,
apontando os principais elementos que caracterizam o que chamamos de
cultura popular, arte e artesanato, no que eles dialogam e se
distinguem. O primeiro filme trata de um artista pobre do interior do
Brasil com deficiência auditiva, que desenha desde os sete anos de
idade. As figuras carregam certo grau de primitivismo, mas possuem
traços de extrema originalidade. O segundo é mais conhecido do público
geral. Baseado na peça homônima de Ariano Suassuna, ele apresenta as
desventuras de uma dupla de sertanejos pobres e espertalhões para
sobreviver, em um vilarejo tomado por uma galeria de personagens que
mobilizam questões sobre tradição, banditismo do cangaço, pobreza e
religiosidade.

\item
Depois de cumprida essa etapa, organize a turma em grupos e
oriente-os a apresentar um trabalho sobre qualquer manifestação de
cultura popular. A forma de apresentação dos trabalhos também é livre.
Pode ser através de texto, música, fotografia, dança, teatro e vídeo. A
utilização de recursos das plataformas digitais e redes sociais deve ser
incentivada, o que pode promover um debate interessante sobre a
circulação, preservação e transformações da própria categoria do que é
popular na era da cultura de massas.
\end{enumerate}

\SideImage{Livro \textit{O Movimento Modernista}, publicado em 1942. (Biblioteca Brasiliana Guita e José Mindlin; Domínio Público)}{PNLD0004-07.png}
\paragraph{Tempo estimado} três aulas de 50 minutos.

\subsection{Leitura I}

\BNCC{EM13LP52}
\BNCC{EM13CHS605}

\paragraph{Tema} Modernismo, modernidade e modernização.

\paragraph{Conteúdo} Discussão e reflexão sobre as relações entre
modernismo, modernidade e modernização, categorias que, embora se
assemelhem na grafia, se referem a processos diferentes tanto em
perspectiva teórica quanto histórica.

\paragraph{Objetivo} Habilitar os estudantes a compreender as diferenças e
relações entre modernismo, modernidade e modernização, a partir dos
contos de Mário de Andrade e do seu projeto estético.

\paragraph{Justificativa} Modernismo, modernidade e modernização se encontram, de
alguma maneira, não apenas em termos fonéticos e
gráficos. São palavras semelhantes, mas que se referem a categorias bem
distintas. Evidentemente elas se referem a condições que dialogam entre
si, no entanto, é fundamental conhecer a especificidade de cada uma para
compreender as maneiras como elas se articulam em uma perspectiva não
apenas histórica, mas também social, política e cultural.




Para identificar e discutir essas relações, é fundamental que os
estudantes compreendam os mecanismos que mobilizam cada um dos conceitos
e os localizem dentro do panorama de formação da nossa sociedade. As
crônicas de Mário de Andrade podem ser um instrumento de apreensão das
fronteiras entre essas categorias.


\SideImage{O modernismo brasileiro na arquitetura: a Catedral de Brasília projetada por Oscar Niemeyer. (Mario Roberto Durán Ortiz; CC-BY-SA 4.0)}{PNLD0004-10.png}


\Image{"O Abaporu": obra de Tarsila do Amaral é ícone do Modernismo Brasileiro. (Cesar Cardoso; CC-BY 2.0)}{PNLD0004-11.png}



\paragraph{Metodologia} 

\begin{enumerate}

\item
Proponha uma discussão bem geral sobre os
conceitos de modernidade, modernismo e modernização. Historicamente, a
modernidade é uma categoria que vem se desenvolvendo desde o
Renascimento e se consolidou na Era Industrial, abrangendo as várias
dimensões da vida: política, social, cultural e econômica. Discute-se,
ainda, a sua permanência ou não nos dias atuais. O modernismo se refere
a um conjunto de tendências estéticas e movimentos artísticos, com
origem na Europa, que vicejaram no início do século XX, mas que ainda
apresenta ressonâncias na arte contemporânea. E a modernização trata de
um processo de expansão e desenvolvimento urbano da sociedade,
especialmente no final do século XIX e no início do século XX.


\Image{Exposição com obras do Museu d’Orsay, de Paris, trazem quadros com estilo Modernista europeu. (Danilo Verpa; CC-BY 2.0)}{PNLD0004-13.png}


\Image{A influência do movimento modernista europeu no trabalho dos modernistas brasileiros pode ser vista na experimentação de formas e técnicas. (Danilo Verpa; CC-BY 2.0)}{PNLD0004-14.png}


\item
A partir das crônicas de Mário de Andrade, proponha aos alunos que
desenvolvam trabalhos que discutam essas categorias. Eles devem
identificar e analisar não apenas os eventos históricos envolvidos no
relato cronístico, mas também desenvolver uma reflexão sobre as críticas
do cronista em que estejam evidentes as diferenças e aproximações entre
as categorias propostas nesta atividade. Relacionamos, a seguir, algumas
recomendações de crônicas que podem ser utilizadas com os alunos para a
discussão. No entanto, você pode fazer uma seleção dos textos a serem
discutidos, incluindo, se considerar oportuno, a produção de outros
cronistas que tratem da mesma questão, realizando, assim, uma análise
comparada. Eis as crônicas:

\begin{itemize}
\item
  ``Zeppelin'';
\item
  ``O diabo'';
\item
  ``O culto das estátuas'';
\item
  ``Anjos do Senhor'';
\item
  ``Cai, cai, balão!'';
\item
  ``Voto secreto'';
\item
  ``A sra. Stevens'';
\item
  ``Esquina'';
\end{itemize}
\end{enumerate}

Como orientação para os trabalhos, peça para levem em consideração os
seguintes tópicos na discussão dos textos selecionados:

\begin{enumerate}
\item
  Quais são as consequências do processo de modernização do Brasil, em
  termos sociais, culturais e econômicos?
\item
  Como as mudanças na sociedade transformam a configuração das classes
  sociais no país e de que maneira isso está vinculado aos processos de
  modernização que ele atravessou?
\end{enumerate}

A forma como cada trabalho será apresentado é livre, mas é importante
que os educandos se mantenham atentos ao texto de Mário de Andrade,
propondo, inclusive, novas questões a partir dos tópicos sugeridos
acima.

\paragraph{Tempo estimado} três aulas de 50 minutos.

\subsection{Leitura II}

\BNCC{EM13CHS106}
\BNCC{EM13CHS206}

\paragraph{Tema} Cartografia e diversidade cultural

\paragraph{Conteúdo} Criação de um mapa interativo a partir dos dados
fornecidos pelas crônicas de Mário de Andrade sobre os lugares onde
ocorrem os fatos e os que são referenciados, apresentando informações
sobre os elementos culturais da região.

\paragraph{Objetivo} Habilitar os estudantes a compreender as dinâmicas
sociais, econômicas e culturais na configuração territorial de um
espaço.

\paragraph{Justificativa} Um espaço é mais do que a circunscrição
delimitada geometricamente de um lugar. É uma resultante das dinâmicas
sociais, culturais, econômicas e políticas que mobilizam grupos de
indivíduos na construção de suas identidades e os modos como eles se
relacionam. As pessoas se definem e redefinem pelos espaços que ocupam e
pelos caminhos por onde se deslocam.

Com o advento do mundo virtual e dos aplicativos de localização, é quase
impossível não termos o nosso itinerário registrado pelas principais
plataformas de busca. Se, por um lado, isso pode soar assustador, por
outro, pode servir de ferramenta para novas formas de relação com o
espaço, de aplicação de políticas voltadas para o desenvolvimento de cada
lugar, de transformação das redes de sociabilidade e de descoberta de
identidades que esses espaços produzem.

Mário de Andrade não foi apenas um viajante, foi também um turista aprendiz. Na
própria forma como escolheu se definir, temos um indivíduo em permanente
deslocamento: por um lado, aberto ao novo e a ser incorporado nele, e
por outro, sem nunca o alcançar em sua totalidade. Portanto, desenvolver
uma análise e reflexão sobre espaços e trajetórias pode ser fundamental
para compreender esses mecanismos de busca e construção de identidades
e culturas.

\paragraph{Metodologia} 

\begin{enumerate}
\item
Organize as turmas em grupos e selecionem
crônicas de Mário de Andrade em que há referências a lugares que podem
ser identificados em um mapa. Pode ser estado, cidade, bairro, um
estabelecimento, como livraria e restaurante, ou ainda construções e
monumentos. Considerando que a obra de Mário de Andrade está calcada em
pesquisa e reflexão sobre a cultura brasileira, deve-se considerar
apenas as menções ligadas a este país. Se o lugar for identificado
através de um gentílico, a referência também pode ser utilizada;

\item
Algumas sugestões de crônicas:

\begin{itemize}
\item
  ``Zeppelin'';
\item
  ``O culto das estátuas'';
\item
  ``Conversa à beira do cais'';
\item
  ``Abril'';
\item
  ``Cai, cai, balão!'';
\item
  ``Tacacá com tucupi'';
\item
  ``Rei Momo'';
\item
  ``Calor'';
\item
  ``Tempo de dantes'';
\item
  ``Esquina'';
\item
  ``Será o Benedito!''
\item
  ``Moringas de barro'';
\item
  ``SPHAN''
\end{itemize}

\item
Os estudantes devem anotar as referências dos locais e marcá-las no
mapa, que pode ser tanto virtual quanto físico. A partir da crônica, cada
estudante deve anotar informações culturais referentes ao local, sempre
em diálogo com o texto de origem. O registro pode ser feito por meio de
notas, imagens, músicas, filmes, entre outros elementos gráficos.
\end{enumerate}

\paragraph{Tempo estimado} quatro a seis aulas de 50 minutos.

\subsection{Pós-leitura}

\BNCC{EM13LGG104}
\BNCC{EM13LGG604}

\paragraph{Tema} O cronista como testemunha da história.

\paragraph{Conteúdo} Produção de relatos de eventos históricos,
independentemente do período, por meio da crônica, ressaltando o olhar
sobre situações cotidianas e a valorização do pormenor.

\paragraph{Objetivo} Habilitar os estudantes a identificar e analisar as
relações entre literatura e história por meio da crônica, a fim de
perceber os diversos usos possíveis do gênero.

\paragraph{Justificativa} As relações entre literatura e história ensejam
debates intermináveis sobre as recíprocas influências entre os dois
campos. Por um lado, questiona-se a legitimidade de se considerar o
texto literário como documento histórico, considerando que, além de ser
um tipo de registro inscrito em um determinado momento da história, ele
reflete a atuação e o pensamento de um indivíduo ou grupo de indivíduos
em seu respectivo tempo. A história, por outro lado, se utiliza de
recursos próprios da literatura para relatar o evento que quer
reconstituir, de modo a não apenas comunicá-lo ao seu público, mas
também de expor criticamente as relações e tensões das fontes a partir
dos quais o discurso histórico foi construído.

A crônica encontrou na modernidade as condições para se desenvolver como
artefato literário e alcançar um público mais amplo, mas sua relação com
a literatura não é tão recente. A chamada crônica histórica remonta a
períodos bem mais antigos, em que se fazia uso da narrativa para
testemunhar a ocorrência de um determinado evento. A crônica moderna não
perdeu esse viés, combinando história, literatura e jornalismo.

\paragraph{Metodologia} 

\begin{enumerate}

\item
Estabeleça uma discussão com os alunos sobre as
diferenças e aproximações entre o discurso histórico e o discurso
literário, mas sem incursões longas em conceitos teóricos. A distinção
feita por Aristóteles entre o historiador e o poeta, em sua
\emph{Poética}, pode ser um ótimo ponto de partida para essa conversa.
Na obra, o filósofo grego afirma que o historiador relata os
acontecimentos como sucederam, enquanto o poeta, por sua vez, relata os
acontecimentos como poderiam ter sucedido;

\item
Documentários são importantes instrumentos de debate dessa relação
entre literatura e história. Uma sugestão é o filme de Eduardo Coutinho
chamado \emph{Jogo de cena}, em que atrizes, famosas e não
famosas, tentam reconstituir os depoimentos de mulheres comuns. Os
impasses diante do esforço em reconstruir um outro indivíduo, com sua
trajetória, sua dicção, sua fisionomia e seu estilo, esbarram nos
limites dos recursos que as próprias artistas possuem para a
reconstrução daquelas vidas. De uma certa maneira, a história e a
literatura também se veem em face desse conflito;

\item
Utilize exemplos de publicações. No jornalismo brasileiro, há muitos
casos de textos que saíram das páginas da crônica política, esportiva,
cultural ou econômica para os livros. Um dos exemplos é o premiado livro
de Mário Magalhães, \emph{Sobre lutas e lágrimas: uma biografia de 2018,
o ano em que o Brasil flertou com o apocalipse}. Na obra, o jornalista e
biógrafo reuniu artigos publicados em jornais on-line, acrescentando
novos capítulos na organização do livro. Ele explica que os textos foram
escritos ``a quente'', ``com o olho no torvelinho'', ou seja, nasceram
no instante da ocorrência dos eventos, o que pode influenciar na própria
formulação das ideias. Uma das crônicas de Mário de Andrade que pode
servir na discussão é ``Voto secreto'', em que o escritor debate a
situação eleitoral e política do país nos anos 1930.

\item
A forma como cada estudante ou grupo pode apresentar é livre. Claro
que eles podem tratar de qualquer episódio histórico, mas seria
interessante que fossem discutidos eventos mais recentes, de modo a
poder discutir a relação entre história e literatura pela perspectiva do
cronista como testemunha de seu tempo.
\end{enumerate}

\paragraph{Tempo estimado} duas a três aulas de 50 minutos.

\section{Aprofundamento}


``Na rua Aurora eu nasci / Na aurora de minha vida / E numa aurora
cresci. // No largo do Paiçandu / Sonhei, foi luta renhida, / Fiquei
pobre e me vi nu. // Nesta Rua Lopes Chaves / Envelheço, e envergonhado.
/ Nem sei quem foi Lopes Chaves. // Mamãe! me dá essa lua, / Ser
esquecido e ignorado / Como esses nomes de rua''. Publicados em
\emph{Lira paulistana}, em 1945, estes versos traçam uma espécie de
síntese da vida de Mário de Andrade, um dos maiores escritores da
literatura brasileira e um dos principais representantes do modernismo.
Não é uma tarefa simples descrever em poucas linhas alguém tão múltiplo e
diverso como ele. 

Mário teve a sua ``aurora'' no dia 9 de outubro de 1893, na cidade de
São Paulo. Desde cedo, demonstrou talento para a música, se destacando
como pianista. Também foi autodidata. Ele se dedicou ao estudo da
literatura, da pintura e de outras artes, mas foi a poesia que o pegou
em cheio. Quase uma paixão à primeira vista, ou melhor, à primeira
leitura, talvez.

Seu primeiro livro foi \emph{Há uma gota de sangue em cada poema}
(1917), que ele assinou com o pseudônimo de Mário Sobral. Nele, ainda
estão presentes as influências do simbolismo e do parnasianismo, mas já
aparece nesse livro um pouco da sua ``cara'' como escritor modernista.
Poucos anos depois, ele se engajou no modernismo, movimento que, inicialmente, 
se opôs de forma radical a essas estéticas anteriores, influenciado
pelas vanguardas artísticas europeias.

Em 1922, Mário publicou o livro de poemas \emph{Pauliceia desvairada},
sua obra-manifesto, no mesmo ano em que trabalhava na organização de um
dos eventos mais importantes da vida intelectual e cultural brasileira
no século XX, a Semana de Arte Moderna. Ocorrida entre os dias 11 e 18
de fevereiro de 1922, no Theatro Municipal de São Paulo, contou
com a participação de artistas como Oswald de Andrade, Anita
Malfatti e Heitor Villa-Lobos.


\Image{O Theatro Municipal de São Paulo foi palco da Semana de Arte Moderna de 1922. (Wilfredor; CC0)}{PNLD0004-09.png}


Ainda nos anos 1920, Mário pega o caminho da prosa de ficção. 
\emph{Macunaíma}, lançado em 1928, e \emph{Amar, verbo instransitivo}, 
que saiu um ano antes, são duas obras-primas, das mais importantes da literatura
brasileira. Mário também foi um excelente contista. Muitos dos
temas e inquietações que ele tinha como artista e intelectual estão presentes
em suas narrativas curtas, pequenas histórias que são também
grandes na sua importância para a literatura do nosso país.


\SideImage{Obra mais importante do autor, em publicação de 1928. (Biblioteca Nacional; Domínio Público)}{PNLD0004-06.png}


A edição de \emph{Algumas crônicas} reúne textos publicados nos livros
\emph{Os filhos da Candinha} (1943), \emph{Táxi e crônicas do Diário
Nacional} (1976) e \emph{Será o Benedito!} (1992). Destes, como se pode
notar, apenas o primeiro foi publicado em vida pelo autor. Os demais
coletam a farta produção jornalística de Mário de Andrade mantida em seu
arquivo, no Instituto de Estudos Brasileiros da Universidade de São
Paulo. Como confessa na ``Advertência'' do livro de crônicas que
publicou em vida, era a sua ``aventura intelectual'', em que o cronista
``brincava de escrever''. O múltiplo Mário de Andrade tem, na crônica,
um espaço descomprometido para a escrita, mas também de confissão, fuga,
encenação, memória e de afetos que buscou salvar do esvanecimento do
tempo.

A edição de \emph{Os filhos da Candinha} foi pensada por ocasião do
projeto das \emph{Obras completas}, pela Livraria Martins Editora. O
livro reúne crônicas publicadas nos jornais paulistanos \emph{Diário
Nacional}, \emph{Diário de S. Paulo} e \emph{O Estado de S. Paulo}, e
nos magazines cariocas \emph{Movimento Brasileiro} e \emph{Revista
Acadêmica}, além da soteropolitana \emph{Letras}. Mário colaborou com
diversas publicações, com seus textos curtos ao sabor das
inquietações cotidianas. Embora presente o caráter lúdico de quem queria
``brincar de escrever'', ele não deixou de expor as tensões de uma
sociedade ricamente diversa, social e culturalmente, mas com problemas
profundos, que se refletem nos conflitos de classe e nas
relações de poder.

As outras duas edições são póstumas, a partir do trabalho exaustivo de
pesquisadores e estudiosos da obra de Mário de Andrade em seu arquivo de
documentos pessoais. O livro \emph{Táxi e crônicas no Diário Nacional},
organizado por uma das maiores especialistas no escritor paulista, Telê
Ancona Lopez, e publicado em 1976, juntou todas as crônicas publicadas
em sua coluna ``Táxi'' no periódico paulistano, entre 1927 e 1933.
Algumas foram selecionadas pelo escritor para figurar em \emph{Os filhos
da Candinha}.

Quanto ao livro \emph{Será o Benedito!}, de 1992, além da crônica que o
intitula, foram resgatados os textos em que Mário de Andrade atua
também como crítico, falando sobre pintura, literatura, artes plásticas
e sobre patrimônio artístico e cultural. São textos publicados no ``Suplemento
em Rotogravura'', de \emph{O Estado de S. Paulo}, entre setembro de 1937
e novembro de 1941.

As crônicas de Mário de Andrade merecem ser lidas porque retratam, de
forma simples e divertida, a diversidade da sociedade brasileira a
partir de situações cotidianas, seja por meio da ficção ou pelo registro
documental de um acontecimento. São textos bastante atuais, porque
expõem as inquietações do indivíduo comum diante das transformações
observadas no dia a dia, mas que também valoriza as tradições culturais
da sociedade em que vive e sua diversidade.

Como cronista, Mário dialoga com grandes escritores de seu tempo que
também se dedicaram ao gênero, como Manuel Bandeira, Carlos Drummond de
Andrade e Rubem Braga. Entre os cronistas contemporâneos, podemos
observar relações entre os textos de Mário e os de João Ubaldo Ribeiro,
Aldir Blanc e Luiz Antonio Simas. Este último, aliás, merece destaque, não
apenas pela simplicidade e qualidade de texto, mas pelo diálogo com os
estudos de cultura popular e pela valorização das miudezas do cotidiano,
temas tão caros ao universo de Mário de Andrade, que ainda exerce
profunda influência sobre muitos autores brasileiros contemporâneos.


\Image{A Biblioteca Mário de Andrade, localizada no centro de São Paulo, cujo nome traz memória ao legado do autor. (Gabriel Fernandes; CC-BY-SA-4.0)}{PNLD0004-15.png}


\section{Referências complementares}

\subsection{Audiovisual}

\begin{enumerate}
\item
  \emph{Imagens do Estado Novo} (2016), direção de Eduardo Escorel;
\item
  \emph{A marvada carne} (1985), direção de André Klotzel;
\item
  \emph{Macunaíma} (1969), direção de Joaquim Pedro de Andrade;
\item
  \emph{Tempos modernos} (1936), direção de Charles Chaplin;
\item
  \emph{Mário de Andrade -- Reinventando o Brasil} (2001), da série
  ``Mestres da Literatura'', produzida pelo Ministério da Educação;
\item
  \emph{AmarElo -- É tudo pra ontem} (2020), direção de Fred Ouro Preto.
\end{enumerate}

\subsection{Musical}

\begin{enumerate}
\item
  \emph{Sobrevivendo no Inferno} (1997), Racionais MC's;
\item
  \emph{Perfil de São Paulo} (2000), Inezita Barroso;
\item
  \emph{Macunaíma Ópera Tupi} (2008), Iara Rennó.
\end{enumerate}

\subsection{Artes visuais}

\begin{enumerate}
\item
  \emph{Coleção Mário de Andrade -- Artes plásticas} (1998), Marta
  Rosseti Batista e Yone Soares de Lima;
\item
  \emph{Coleção Mário de Andrade -- Religião e magia, música e dança,
  cotidiano} (2004), Marta Rosseti Batista
\end{enumerate}

\subsection{Website}

\begin{enumerate}
\item
  \emph{Portal da crônica brasileira} (2018), Instituto Moreira Salles.
  Endereço: https://cronicabrasileira.org.br
\end{enumerate}

\section{Bibliografia comentada}

\begin{enumerate}
\item
  Candido, Antonio. ``A vida ao rés-do-chão''. In: \emph{Para gostar de
  ler}: \emph{crônicas}. São Paulo. Ática, 1981. Volume 5. Apresentação
  de alguns dos principais cronistas brasileiros por um de nossos
  maiores críticos literários, Antonio Candido. A caracterização da
  crônica como ``um gênero menor'' feita por Candido nos coloca diante
  das condições de produção de um gênero que, por essa razão, possui
  grande versatilidade.
\item
  Castro, Moacir Werneck de. \emph{Mário de Andrade}: exílio no Rio.
  Belo Horizonte: Autêntica, 2016. Depoimento pessoal de Moacir Werneck
  de Castro sobre sua relação afetiva e intelectual com Mário de Andrade
  durante o período em que este residiu no Rio de Janeiro, entre 1938 e
  1941. A amizade de Mário com os chamados ``Rapazes da
  \emph{Acadêmica}'', em referência à \emph{Revista Acadêmica}, exerceu
  grande influência recíproca entre eles. Além de Castro, o grupo era
  formado também por Murilo Miranda, Lúcio Rangel e Carlos Lacerda. O
  livro traz ainda as cartas de Mário para o autor.
\item
  Moraes, Marcos Antonio de (org.). \emph{Correspondência de Mário de
  Andrade e Manuel Bandeira}. São Paulo: Edusp, 2001. Reunião das cartas
  trocadas entre os dois poetas do modernismo brasileiro, de 1922 a
  1945, revelando os bastidores da criação literária, a amizade e os
  rumos do movimento modernista. Baseado em exaustiva pesquisa
  documental, o livro traz ainda dossiê de fotografias e fac-símiles de
  textos relacionados aos dois escritores.
\item
  Simas, Luiz Antonio. \emph{Almanaque brasilidades: um inventário do
  Brasil popular}. Rio de Janeiro: Bazar do Tempo, 2018. O livro é uma
  coleção de temas que caracterizam a diversidade da cultura brasileira,
  apresentados na forma como eram conhecidos os almanaques populares. A
  escrita concisa, leve e dinâmica é envolvente como as páginas das
  melhores crônicas já feitas, e, ao mesmo tempo, nos traz uma galeria
  de expressões, personagens e tradições das culturas populares.
\item
  Souza, Eneida Maria de; Cardoso, Marília Rothier. \emph{Modernidade
  toda prosa}. Rio de Janeiro: PUC-Rio; Casa da Palavra, 2014. As
  autoras são duas das mais importantes pesquisadoras sobre o modernismo
  brasileiro. O livro inova ao ampliar o conceito de prosa e suas
  ressonâncias a partir do modernismo, abordando não apenas os gêneros
  literários em prosa, como o romance e o conto, mas também expressões
  como o cinema e as artes plásticas.
\item
  Tércio, Jason. \emph{Em busca da alma brasileira}: biografia de Mário
  de Andrade. Rio de Janeiro: Estação Brasil, 2019. A premiada biografia
  de Mário de Andrade é mais do que a reconstituição de episódios
  decisivos de sua vida, mas uma reflexão sobre aspectos importantes da
  vida cultural e política brasileira da primeira metade do século XX. O
  jornalista Jason Tércio se debruçou em farta documentação dos
  arquivos, dando consistência ao seu trabalho.
\end{enumerate}


\end{document}

