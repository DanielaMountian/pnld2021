\documentclass[12pt]{extarticle}
\usepackage{manualdoprofessor}
\usepackage{fichatecnica}
\usepackage{lipsum,media9,graficos}
\usepackage[justification=raggedright]{caption}
\usepackage[one]{bncc}
\usepackage[papagaio]{../edlab}

\begin{document}

% Dicas para a redação das atividades %%%%%%%%%%%%%%%%%%%%%%%%%%%%%%%%%%%%%%
% Gente de H. é inspiração para o filme Dogville do Lars von Trier. 
% O grupo Dogma tem grande ligação com Stringberg. 
% Gente de H. conta a história de um homem que chega em uma cidadezinha
% e bela bondado dos outros vai se sendo enredado numa trama de loucura
% até se matar. 

\newcommand{\AutorLivro}{August Strindberg}
\newcommand{\TituloLivro}{Gente de Hemsö}
\newcommand{\Tema}{Ficção, mistério e fantasia}
\newcommand{\Genero}{Romance}
\newcommand{\imagemCapa}{./images/PNLD0039-01.png}
\newcommand{\issnppub}{978-65-89810-08-7}
\newcommand{\issnepub}{978-65-89810-04-9}
% \newcommand{\fichacatalografica}{PNLD0039-00.png}
\newcommand{\colaborador}{Bruno Gradella e Vicente Castro} 
%Sofia Boldrini (edição)}


\title{\TituloLivro}
\author{\AutorLivro}
\def\authornotes{\colaborador}

\date{}
\maketitle

\baselineskip=1.15\baselineskip\par

\begin{abstract}\addcontentsline{toc}{section}{Carta ao professor}
Este manual tem como objetivo fornecer subsídios para o trabalho com a
obra literária \emph{Gente de Hemsö}, obra de autoria de August
Strindberg.

Johan August Strindberg (Estocolmo, 1849--\textit{id.}, 1912) 
foi escritor, dramaturgo, pintor e fotógrafo sueco. Grande provocador
de seu tempo, recebe muitas críticas e acaba se exilando fora de 
seu país. No exterior, escreve uma parcela significativa de sua obra,
como o romance \textit{Inferno}, ao mesmo tempo em que luta contra graves 
problemas psicológicos.

\textit{Gente de Hemsö} é considerado uma das obras-primas de August 
Strindberg. Escrito em sua maior parte quando o autor se encontrava no 
exílio, foi publicado pela primeira vez em 1887. Encontraremos aqui, 
numa obra muito bem aceita pelo grande público, um retrato do povo e das 
paisagens dos arquipélagos da Suécia. Veremos como um administrador com 
algumas noções de agricultura vai administrar uma propriedade decadente 
na ilha de Hemsö, onde o povo vivia da pesca. Sua chegada causa 
conflitos entre habitantes e o representante dos ``novos tempos''. A Igreja 
e seu pastor representam um elemento de equilíbrio entre os interesses 
opostos.

Em um dado momento esse equilíbrio é desfeito, quando o administrador 
declara para o povo que é ``cada um por si e Deus por todos'', lema do 
egoísmo predador do ``novo mundo''. Acabará, no entanto, vítima do que 
defende quando, precisando da ajuda da população para se salvar de uma 
enchente do mar, ouve como resposta um não.

Esperamos que as indicações propostas aqui sejam muito úteis no trabalho 
em sala de aula! 
\end{abstract}
\pagebreak

\tableofcontents

\pagebreak
\section{Propostas de Atividades I}

\subsection{Pré-leitura}
\SideImage{Foto de Strindberg após seu aniversário de 50 anos. (Autor desconhecido; Domínio Público)}{PNLD0039-03.png}

\paragraph{Tema} O naturalismo e outras escolas literárias.

\paragraph{Conteúdo} Conhecimento e articulação dos diferentes gêneros 
literários e suas características, posicionando junto a eles o naturalismo. 

\paragraph{Objetivo} Estimular e habilitar os estudantes a perceber a 
correlação entre as diferentes escolas literárias, neste caso, entre o 
naturalismo e outros estilos do século \textsc{xix}.

\paragraph{Justificativa} O romance de Strindberg aproxima"-se do 
naturalismo francês, ligação evidente na maneira escancarada em que os 
elementos populares são apresentados, e não nos parece exagero
especular se Strindberg aqui não antecipa um tom quase joyceano. 
Não tanto quanto à técnica modernista, mas na forma despudorada e saborosa 
em que ele retrata os aspectos vulgares da existência humana.
Vemos como em \textit{Gente de Hemsö} um autor profundamente intelectual 
e urbano, como era Strindberg, se deixa levar para longe das convenções 
da boa sociedade e se apaixonar pela escatologia e a rudeza da vida,
tentando plasmá"--la em seu texto. Tendo depois \textit{Gente de Hemsö} 
atravessado o longo século \textsc{xx}, é impressionante como essas
passagens funcionam ainda hoje e dão força à obra. 

Essa aproximação com o naturalismo pode ser observada nas muitas passagens 
em que Strindberg faz as vezes de um naturalista ou botânico, colecionando 
com avidez e prazer um infindável número de espécies animais e vegetais 
dos arquipélagos e da natureza escandinava. Peixes, pássaros, árvores,
flores vêm em profusão sobre nós. Em outras horas, Strindberg assume o 
papel de um antropólogo e elenca meticulosamente todo um arsenal de
ferramentas de navegação, pesca, caça e trabalho na terra. Nota"-se que
ele reuniu com grande afinco todos esses elementos para a composição de
seu retrato, dando"-lhe, além da qualidade literária, um rico valor 
documental --- o que torna, sem dúvida, uma empreitada com muitas 
dificuldades traduzir \textit{Gente de Hemsö} para o português de nossos dias.

O domínio desse conteúdo é importante, pois fomenta uma maior absorção do 
conteúdo direto e simbólico que os autores desejaram imprimir em uma obra 
e, principalmente, para a obra em questão, \textit{Gente de Hemsö}.

\paragraph{Metodologia}

\begin{enumerate}

\item
Para introduzir a discussão sobre escolas literárias, proponha aos alunos
uma atividade de debate em grupo nos quais eles levantem os nomes dos 
estilos que eles conhecem. Perguntas como ``quais são os gêneros literários?'' 
e ``quais são as escolas literárias mais importantes que existiram na 
história?'' podem ser um bom começo para a discussão.

\item
Em seguida, apresente à turma o Naturalismo e suas características. Assim 
eles terão a base comum para fazer comparações com as outras escolas que 
eles irão pesquisar. 

\item
Desse modo, é aconselhado que a sala de aula seja dividida em grupos, cada 
qual adotando uma escola literária para pesquisar, e que os alunos realizem 
buscas que considerem os elementos que compõem cada um desses gêneros 
literários. 
\BNCC{EM13LP12}

\item
Como produto dessa atividade de pré"-leitura espera"--se a produção de um
material de consulta, como um caderno coletivo que todos possam consultar 
ou ainda, se possível, um site/\,blog enciclopédico, que poderá estar 
vinculado ao site do colégio.
\BNCC{EM13LP34}

\end{enumerate}

\paragraph{Tempo estimado} Duas aulas de 50 minutos. 


\subsection{Leitura}

\paragraph{Tema} As personagens em \textit{Gente de Hemsö}.

\paragraph{Conteúdo} Compreensão das múltiplas personagens e a forma 
como Strindberg as descreve.

\paragraph{Objetivo} Estimular e habilitar os estudantes a identificar características 
de cada uma das personagem ao longo da leitura e como se desenvolvem ao 
longo da trama, atentando"--se para os seus comportamentos. 

\paragraph{Justificativa} Nas próprias palavras de Strindberg para o editor 
Pehr Staaf: ``Trata"--se de um retrato em forma de romance sobre a vida 
rural nos arquipélagos suecos, o primeiro romance, genuíno, que escrevi.
Repleto de paisagens suecas, camponeses, homens ilustrados,
pastores, capelães etc. Coisas belas, coisas feias, coisas tristes,
alegres, cômicas, vívidas, todas elas publicáveis.''

Quanto aos aspectos humanos do universo de sua obra, quanto
aos seus personagens e à tal “ferramenta da psicologia moderna”
com que Strindberg os descreverá, é evidente que
não há aqui nada de esquemático. Apenas, talvez, quanto
aos encadeamentos de certos aspectos na narrativa, notamos
um certo artificialismo quando Strindberg procura conferir
ao enredo um tom exemplar. Mas os personagens de
Strindberg, mesmo os menores, não estão ali como fantoches para quaisquer 
teses. Ou melhor: embora Strindberg certamente tivesse uma opinião formada 
sobre o microcosmo que ele retratava, encenando"--a em seus personagens,
ele nunca a impõe de maneira direta ou grosseira. Como
autor diferenciado que era, Strindberg escondia muito bem
as costuras de sua criação e nos dá criaturas que, mais do
que expressão de uma visão particular, são dotados de vida
autônoma.

Em \textit{Gente de Hemsö}, a rusticidade da vida rural é pintada em tintas
fortes e é evidente como o autor a apresenta como uma
alternativa à moralidade burguesa da cidade. 

\paragraph{Metodologia}


Como ponto de partida para essa atividade de leitura, proponha aos alunos 
o registro de personagens em uma espécie de lista, posteriormente, colocando-a 
no formato de um mapa conceitual. Em seguida, após ler a BNCC utilizada nessa atividade
sugerimos os seguintes itens:

\BNCC{EM13LP28}
\begin{enumerate}

\item
Nossa sugestão é, nesse primeiro momento, analisar coletivamente a maneira 
com que Strindberg retrata a figura do pastor Nordström. Esse homem, 
oriundo da cidade, de uma cultura refinada, e desde a juventude sabedor
de latim e dogmática, vai se transformando ao longo dos muitos anos vividos 
no arquipélago, tornando"--se finalmente quase indistinguível dos rudes 
pescadores locais. Plenamente imerso nos costumes e no linguajar destes, 
ele já não tem papas na língua e fala para sua congregação nos próprios 
termos dela.

\item
Após essa leitura em conjunto, atente a turma para como o ambiente está 
diretamente vinculado às mudanças que acontecem com essa personagem de 
Nordström e como o lugar passa a moldar seus comportamentos. Na sequência, 
sugira aos alunos que indiquem as personalidades e características de cada 
uma das personagens. Além de como as personagens se desenvolvem ao longo 
da trama, os alunos podem, conforme a leitura for avançando, anotar e 
fixar, ao lado de cada uma das personagens, adjetivos que possibilitem 
caracterizar comportamentos, trechos de falas ou de passagens emblemáticas. 

\item
Enquanto representação gráfica do mapa conceitual, os aluno podem se valer 
também de setas coloridas para reforçar visualmente os vínculos que vão 
se formando entre os dados coletados. Esse mapa poderá ajudar em etapas 
posteriores da análise e da recuperação e síntese de informações pós-leitura. 
Poderá também ser enriquecido por outros elementos verbais e não verbais, 
decorrentes das atividades interdisciplinares com a obra.

\end{enumerate}

\paragraph{Tempo estimado} Duas aulas de 50 minutos.

\subsection{Pós"-leitura}

\paragraph{Tema} Seria o indivíduo determinado pelo ambiente? 

\paragraph{Conteúdo} Conhecimento e articulação das formas que se dão as 
relações sociais, as relações com a cidade, com a natureza e com o tempo.

\paragraph{Objetivo} Orientar os alunos a perceber a 
correlação entre indivíduo e ambiente e como ele influencia o modo de vida 
das pessoas.

\paragraph{Justificativa} Em muitas passagens, Strindberg
faz as vezes de um naturalista ou botânico, colecionando
com avidez e prazer um infindável número de espécies
animais e vegetais dos arquipélagos e da natureza escandinava.
Peixes, pássaros, árvores, flores vêm em profusão
sobre nós. Em outras horas, Strindberg assume o papel
de um antropólogo e elenca meticulosamente todo um arsenal
de ferramentas de navegação, pesca, caça e trabalho na
terra. Nota"-se que ele reuniu com grande afinco todos esses
elementos para a composição dos retratos de sua obra, dando"-lhe,
além da qualidade literária, um rico valor documental.

A importância de entender as técnicas que o autor faz uso e como a natureza 
apresenta"-se diretamente vinculada à construção dessas personagens, 
possibilita entendê"--las dentro do contexto do pensamento naturalista. 
No entanto, é necessário olhá"-las também com lentes que questionem esses determinismos.

\paragraph{Metodologia}

\begin{enumerate}
\item
Como atividade de pós"-leitura, proponha uma roda com os estudantes para 
debater sobre os elementos da realidade rural retratados na obra. Pergunte 
à turma se é possível perceber reverberações com a realidade brasileira? 
O que as aproxima e o que as separa? 

\item
Depois desse primeiro momento de conversa, encaminhe o diálogo para o fato 
de que mesmo se passando em um arquipélago, em um país distante, é possível, 
sim, encontrar semelhanças com elementos de nossa cultura. 
Sugerimos que se parta da análise da personagem Rundqvist, que, com suas 
artimanhas e mandingas, pode ser comparada a entidades da nossa cultura 
tradicional como, por exemplo, o Saci. 
\BNCC{EM13CHS101}

\item
Proponha aos alunos que se debrucem em materiais que tratam do
sertanismo e/\,ou que se ambientem no universo rural. 
O professor pode ficar livre nesta escolha, mas sugerimos 
\emph{Seara Vermelho} de Jorge Amado como material 
para complementar essa atividade. 
\BNCC{EM13LP46}

\item
Por fim, munidos da leitura de Strindberg e com os elementos fornecidos 
pelas obras da literatura nacional, convide os alunos a produzirem um 
conto, de linguagem simples e narrativa direta nos moldes do autor sueco, 
porém ambientado em território brasileiro. 
Saliente a importância da descrição dos ambientes e das pessoas, de seus 
costumes, dos animais locais, etc. O material poderá, se possível, ficar disponibilizado no site do colégio posteriormente.
\BNCC{EM13LP54}
\end{enumerate}

\paragraph{Tempo estimado} Duas aulas de 50 minutos.


\section{Propostas de Atividades II}

% A obra \emph{Gente de Hemsö} possibilita trabalhos interdisciplinares e
% integradores de diferentes campos do saber e áreas de conhecimento. A
% seguir, propomos algumas atividades que podem ser desenvolvidas
% conjuntamente com professores de outras áreas. Além das habilidades de
% Linguagens e suas Tecnologias e de Língua Portuguesa, indicadas nas
% etapas da seção anterior e válidas também para esta, listamos a seguir
% as habilidades de outras áreas, presentes na abordagem interdisciplinar:

\subsection{Pré"-leitura}

\paragraph{Tema} Viagem à Suécia a partir de August Strindberg.

\paragraph{Conteúdo} Introdução e contextualização do período histórico"-
cultural em que estava inserida a produção da obra \emph{Gente de Hemsö}.

\paragraph{Objetivo} Ambientar os estudantes na realidade vivida pelo 
autor no contexto de produção de suas obras, em especial o livro 
\emph{Gente de Hemsö}. Propondo paralelos e/\,ou diferenças ao período 
atual e ao contexto brasileiro. 

\paragraph{Justificativa} Strindberg, em sua obra e passagens, propõe"-se 
a causar um estranhamento no leitor em relação ao mundo que ele está
apresentando, até mesmo no leitor sueco de sua época, habituado à
cidade e desacostumado com os arquipélagos.

A personagem principal da trama, Carlsson, tem a impressão
de ter “chegado a uma terra estranha”, quando conhece Hemsö. E essa 
sensação é transposta aos leitores, sendo esse impacto multiplicado ainda 
mais pela distância geográfica e cultural, enquanto brasileiros e 
estrangeiros, mas também pelas mudanças que aconteceram ao longo do tempo 
entre o mundo da ilha e o nosso.

Faz"-se, assim, necessário contextualizar a obra e aprofundar questões 
particulares ao universo sueco daquela época para que se tenha uma leitura 
situada histórica"-geográfica"-culturalmente, permitindo uma maior 
compreensão da obra. 

\paragraph{Metodologia}

\begin{enumerate}

\item
Para essa atividade de pré"-leitura, recomenda"-se que, juntamente com 
os professores das Ciências Humanas, seja realizada uma pesquisa 
aprofundada sobre o contexto histórico"-cultural
em que o autor estava inserido. 
\BNCC{EM13LP01}

\item
Proponha a divisão da sala em grupos e cada grupo deverá pesquisar um 
aspecto dos elementos que possivelmente compuseram as referências de 
Strindberg. Para tanto, recomenda"-se que um grupo pesquise aspectos 
geográficos da Suécia. Outro, fie"-se nas questões históricas envolvendo 
o país. Um outro busque referências culturais e folclóricas e, por fim, 
um último grupo busque informações acerca do contexto cultural e literário 
europeu da época de Strindberg. 
\BNCC{EM13CHS101}

\item
Feito isso, sugira à turma a realização de um seminário onde os
alunos poderão compartilhar suas descobertas. Durante as exposições, os 
alunos deverão, também, anotar os detalhes do que foi exposto pelos 
outros grupos.
\BNCC{EM13LP16}

\item
Como produto final para essa atividade, após o seminário, proponha 
a elaboração coletiva de um texto informativo que aglutine os elementos 
obtidos, a fim de que se torne mais um guia norteador para a leitura do 
livro.

\end{enumerate}

\paragraph{Tempo estimado} Duas aulas de 50 minutos.


\subsection{Leitura}

\paragraph{Tema} Nacionalismo, naturalismo e \emph{Gente de Hemsö}.

\paragraph{Conteúdo} Compreensão e aprofundamento de conceitos 
característicos da período em que a obra está inserida, a partir outras representações na literatura e no cinema. 

\paragraph{Objetivo} Familiarizar os estudantes com conceitos como 
nacionalismo e naturalismo, a partir da leitura de alguns trechos de 
\emph{Gente de Hemsö} e outros autores, além de reverberações da obra de 
Strindberg nos filmes de Ingmar Bergman. 
Os alunos devem conseguir identificar as principais semelhanças entre 
essas produções, além de se atentarem na maneira como esses conceitos são apresentados. 

\Image{O diretor Ingmar Bergman, conterrâneo de Strindberg, no set do 
filme \textit{Morangos Silvestres}. (Louis Huch; Domínio Público)}{PNLD0039-09.png}

\paragraph{Justificativa} O período em que a obra \emph{Gente de Hemsö} 
está inscrita, foi, ao mesmo tempo, de otimismo e de contestação. Havia a contraposição das identidades nacionais, 
coletivas e sociais, frente às angústias do indivíduo. Os debates sociais 
eram muito frequentes nesse momento, havendo a busca por direitos das 
mulheres, questões de identidade nacional e recuperação elementos típicos 
da cultura e da história escandinavas.

A sofisticação nos traços psicológicos, o esmero técnico, nunca retroceder 
diante de uma contradição, a relação difícil com o entorno cultural e 
pessoal, a crítica implacável aos moralismos e às convenções, a 
universalidade de seus questionamentos quanto à condição humana, são 
característicos de Strindberg enquanto autor e figura pública. 

Seu humor momentâneo, a leveza inesperada, o lirismo e a esperança final 
de redenção, são encontrados em outro ícone sueco, o diretor Ingmar Bergman (1918"--2007). Com menos de meio século de diferença entre as vidas 
artísticas dos dois, o cineasta, em seus filmes, revela a produtiva 
relação de influência. Bergman dirigiu várias das peças teatrais de
Strindberg e foi marcado, sobretudo, por suas fases criativas
posteriores, onde ele entraria em densos e mais sombrios
territórios da psiquê humana e das aporias dos relacionamentos
afetivos.

Ambos apresentam o mesmo retrato da natureza sueca, uma natureza que 
mantém com os personagens uma relação a um só tempo silenciosa e 
devastadora. Em seus filmes, bem como em muitos outros aspectos, 
certamente Bergman deve muito a Strindberg. Nas palavras de Ingmar Bergman: 

``Strindberg tem me perseguido a vida inteira: já o
amei, já o odiei, joguei seus livros contra a
parede; mas livrar--me dele, jamais.''

Dessa maneira, entender o contexto cultural, que influenciou  e outros 
interlocutores que dialogam com a obra e com a produção de Strindberg 
revela"--se de grande valia para maior compreensão de \emph{Gente de Hemsö}.

\paragraph{Metodologia}

\begin{enumerate}

\item
Para essa atividade de leitura, proponha aos alunos a realização de uma 
pesquisa sobre a biografia do autor e sua produção literária. 

\item
Com essas informações em mãos, e com o auxílio dos professores de
Humanidades, proponha um debate acerca do nacionalismo e naturalismo, da 
ideia de pertencimento ou exclusão a um determinado grupo.

Fomente a discussão desses temas com os
alunos, valendo"-se desse diálogo com os professores de Ciências Humanas 
para melhor contextualização. 
\BNCC{EM13CHS204}

\item
Para aprofundarem o exercício, é interessante, neste momento, abordar 
obras de outros escritores escandinavos, como Søren Kierkegaard e Hans 
Christian Andersen, apresentando aos alunos pequenos trechos de suas obras.
\BNCC{EM13LP50}

Também sugerimos a exibição de trechos dos filmes de Ingmar Bergman 
ou seus \textit{trailers}, 
como \emph{Monika e o desejo}, \emph{O sorriso de uma noite de amor}, 
\emph{Sétimo selo} ou \emph{Persona}. Procure evidenciar como Bergman 
apresenta o mesmo retrato da 
natureza sueca que Strindberg em sua obra \emph{Gente de Hemsö}. 
\BNCC{EM13LP14}

\item
Por fim, peça aos alunos que busquem elementos em comum, afinal, todos 
viveram em um período próximo, que foi uma época em que questões de 
identidade pessoal, nacional e social eram muito discutidas.

\end{enumerate}

\paragraph{Tempo estimado} Duas aulas de 50 minutos.

\Image{Retrato de Strindberg com 25 anos (Studio de Math. S. Hansen; Domínio Público)}{PNLD0039-04.png}

\subsection{Pós"-leitura}

\paragraph{Tema} Um passeio pela Escandinávia. 

\paragraph{Conteúdo} Exercício de compilação sobre o que aprenderam 
ao longo da leitura da obra e das outras 
atividades correlacionadas. 

\paragraph{Objetivo} Retomar junto com os alunos os pontos importantes da 
cultura escandinava e identificar 
nos espaços das cidades, das paisagens, das produções artísticas e dos 
costumes as reverberações do que nos 
é apresentado por Strindberg em seu livro. 

\paragraph{Justificativa} O universo da ilha de Hemsö, ainda 
que fictícia, poderia representar um microcosmos 
das visões e culturas escandinavas da época. O livro estrutura"-se a 
partir da relação originária das pessoas 
com o seu entorno natural, com as condições climáticas e com as 
tradições milenares. E Strindberg retrata 
as nuances sem reduzir a complexidade de sua história cultural 
a um punhado de elementos.

A obra faz pensar sobre o que significa pertencer a um determinado povo e 
quais seriam seus traços fundamentais. 
Respostas imediatas a esse tipo de pergunta têm contraindicações óbvias, 
e mesmo assim as damos mais frequentemente 
do que gostaríamos de admitir. Seja por hábito inconsciente, seja por 
preguiça intelectual, acabamos recorrendo a um 
repertório estabelecido de referências que se impõe -- ainda que 
momentaneamente —- sobre as controvérsias.

Assim, revela"-se pertinente, como recapitulação do que foi lido e 
estudado, revisitar ``presencialmente'' os espaços 
criados pelo autor, aprofundando os estudos escandinavos. 

\paragraph{Metodologia}

\begin{enumerate}

\item
Munidos do conhecimento do universo de Hemsö, os alunos
poderão, com o auxílio dos professores de outras disciplinas, 
como Humanidades e Ciências da Natureza, 
promover a realização de uma feira de cultura escandinava no colégio. 
\BNCC{EM13LP47}

\item
Esta pode ser feita no local, com o aproveitamento das salas para exposição 
de réplicas de elementos de cultura material. As salas podem estar, por 
exemplo, divididas de acordo com períodos, recriando as atmosferas 
climáticas e estéticas nórdicas.

\item
Podem ser encenadas danças típicas e reproduções das músicas tradicionais 
podem ser apresentadas. Os alunos podem pesquisar e cozinhar receitas 
típicas da região. Caso seja feita a opção virtual, os alunos podem criar 
um acervo de sites que permitam passear pelo universo escandinavo, 
adentrando um pouco na história e cultura da Suécia e de outros países 
nórdicos.
\BNCC{EM13LGG201}

\end{enumerate}

\paragraph{Tempo estimado} Duas aulas de 50 minutos.


\section{Aprofundamento}

Nesta seção, desenvolvemos um trabalho de aprofundamento que, em diálogo
com a formação continuada de professores, oferece subsídios para a
abordagem do texto literário. A leitura em sala de aula de \emph{Gente
de Hemsö} pode ser enriquecida pelo aprofundamento no universo literário
em que a obra está inserida.

\subsection{A obra}

\emph{Gente de Hemsö} é um romance que descreve a vida dos habitantes da 
ilha de Hemsö -- local fictício que, teoricamente, estaria localizado nas
ilhotas do Báltico, próximo à capital sueca. É uma obra que dialoga com
o próprio autor, pois, em sua infância, frequentemente visitava ilhas
nesses arquipélagos e, quando escreveu, estava em um auto-exílio, sendo
a obra, também, uma forma de se reconectar com seu país e sua história.


\Image{Vista da ilha Öland, na Suécia, no mar Báltico, localização 
aproximada da fictícia ilha de Hemsö. (Arnold Paul; CC-BY-SA 2.5)}{PNLD0039-06.png}


\subsection{A sabedoria da simplicidade}

Uma vez, em tom de lamento, August Strindberg afirmou:

\begin{quote}
A maioria acredita ser profundo aquilo que apenas é rebuscado. Isso é
errado. O rebuscado é apenas o mal"-realizado; o obscuro costuma ser
falso. A sabedoria mais alta é simples, clara e atravessa o cérebro até
o coração.
\end{quote}

Essa frase de Strindberg define o tecido que envolve sua obra. É um
romance simples. Uma história sobre a vida rural nos arquipélagos
suecos. Trata"-se de uma história sem estampidos, mas mesmo assim, muito
tocante.

Strindberg foge da afetação burguesa, comumente retratada na literatura
de seu tempo, para narrar a simplesmente a vida como ela é. Ela ocorre,
é certa como os ventos do Mar Báltico, oferta seus percalços, mas mesmo
assim segue. E, como restará claro no desenrolar da história, muitas
vezes os problemas surgem quando tentamos atribuir predicados àquilo que
não precisa. Uma vida boa seria, então, uma vida tranquila. No caso, uma
vida desprovida de preocupações. E as preocupações praticamente vão
embora quando nós apenas aceitamos as coisas como elas são. Na sua
organicidade natural, elas se ajeitam.


\Image{Retrato de Strindberg pintado por Richard Bergh em 1905 (Richard Bergh; Domínio Público)}{PNLD0039-05.png}


\subsection{A simplicidade na escrita}

Strindberg escreve de uma maneira muito clara. Embora em um primeiro
momento até assuste a quantidade de descrições oferecidas pelo autor,
isso é algo que, paulatinamente vai se perdendo ao longo do texto.

É provável que o autor, com isso, desejou oferecer ao leitor a
perspectiva da personagem principal, Carlsson, que, seguramente, vindo
da cidade para a região de Hemsö se deslumbrava com os elementos
diversos do mundo que conhecia.

Entretanto, com o tempo, ele próprio é cooptado para o universo de
Hemsö, de modo que não lhe mais chamam tanto a atenção esses detalhes. E
curiosamente, as descrições diminuem conforme Carlsson se integra ao
universo.

Vemos também que Strindberg usa uma forma de escrever muito direta. As
personagens sabem de antemão o que acontecera em outra cena, eliminando
assim longos diálogos. Ao próprio leitor também são dadas as informações
sem explicações demoradas.

Os capítulos também são breves e objetivos. No caso, recuperando seu
próprio pensamento, Strindberg aplica a sabedoria do necessário.
Eliminando de seu texto tudo que é supérfluo.

Chega a ser engraçada a parte que Carlsson pula trechos da Bíblia a fim
de terminar a reunião religiosa mais rapidamente. Quase como se
Strindberg brincasse com os leitores que faziam isso em algumas de suas
descrições um pouco mais longas.

\subsection{Relação originária com a natureza}

Nesse livro observamos as constantes menções aos ciclos, rotação de
cultura, rotação nos defesos e temporadas de pesca. Enfim, ao ritmo da
natureza. E, em certo sentido, a natureza em Hemsö domina o homem. Para
ilustrar isso, há de se rememorar a passagem em que Strindberg escreve
sobre o pastor Nordström:

\begin{quote}
Aquele que não conhecesse o pastor Nordström jamais adivinharia que este
ilhéu exercia uma função espiritual (...) de tanto que os trinta anos
tomando conta das almas do arquipélago haviam transformado o antes
refinado predicante, quando este veio ordenado de Uppsala.
\end{quote}

E toda vez que nos é apresentado o pastor, ele é tratado como um
grosseirão. Sem paciência, beberrão, que gosta de passar mais tempo
pescando do que presidindo os cultos.

Assim, é muito mais próximo dos habitantes de Hemsö, que não se
incomodam nas longas caçadas ou pescarias que varam dias. Mas consideram
inconveniente pegar o barco para ir até a Igreja.

Curiosamente, Carlsson é constantemente maltratado pelo pastor. É só
após seu fim, que o primeiro recebe a redenção do segundo. Claramente já
o enxergando como um par. Desse modo, é outro que veio da cidade mas foi
cooptado pelo ambiente de Hemsö.

Mas de forma alguma, essa ``aculturação'' é transmitida de uma forma
ruim. Ela apenas acontece, sem juízo de valor. É apenas algo natural
àquele ambiente.

A obra, assim, guarda linda semelhança com o filme 
\textit{Morangos Silvestres}, de Ingmar Bergman, conterrâneo de Strindberg, 
que, como este, foi capaz de contar uma bonita história pautada na boa 
relação com o ambiente e na simplicidade.


\Image{O diretor Ingmar Bergman e o ator principal do filme \textit{Morangos Silvestres}, Victor Sjöström. (Åke Blomquist; Domínio Público)}{PNLD0039-10.png}


É curioso que no início do livro, Strindberg indica que o lápis e o
caderno de Carlsson passaram a ser respeitados, ou seja, os ensinamentos
da cidade, que surgiram em Hemsö nas mãos daquele forasteiro passaram a
ser respeitados, pois era clara a produtividade que eles traziam à
propriedade. Porém, a dicotomia que se faz na história é que, a cidade
otimiza o campo, influenciando. Mais o campo influencia o homem. A
tecnologia extrai aquilo que há na terra, mas a natureza extrai o que há
no homem.


\Image{Ilha de Utö, localizada no arquipélago de Estocolmo. Ao fundo, 
vemos o mar Báltico. (Arild Vågen; CC-BY-SA 2.0)}{PNLD0039-07.png}


\subsection{Atividades para o aprofundamento da pesquisa}

% No Ensino Médio, da mesma forma que no Ensino Fundamental, a \textsc{bncc}
% organiza o trabalho com as práticas de linguagem em cinco \textbf{campos
% de atuação social}. São eles: campo da vida pessoal, campo da vida
% pública, campo jornalístico"-midiático, campo artístico"-literário e campo
% das práticas de estudo e pesquisa.

% De acordo com essa divisão, propomos na sequência um trabalho
% interdiscursivo e intertextual com a obra \emph{Gente Hemsö}.

\subsubsection{A trajetória de Carlsson}

Na história de Strindberg, acompanhamos a trajetória de Carlsson,
jovem habilidoso que vai até Hemsö para tentar a vida como capataz da
fazenda. Sendo estranho àquela comunidade, Carlsson passa por uns
percalços no início de sua aventura, entre eles, a desconfiança dos
outros habitantes da fazenda e da comunidade ao redor dela. Tal
situação não é incomum no mercado de trabalho. Nesse sentido, é
possível até dizer que Carlsson teve sorte, pois conseguira o emprego.
Muitos jovens, no entanto, são barrados até mesmo antes disso. Uma
forma clássica, mas ainda muito eficaz de se adentrar no mercado de
trabalho é por meio de um bom currículo. E um bom currículo não
significa um currículo robusto, mas sim, organizado, claro e bem
apresentado. 

Nessa atividade, proponha um exercício de redação para os
alunos, cujo produto final será a formulação de um currículo para cada
um deles. Para aumentar o acervo de referências, pode--se pesquisar
modelos de currículos para áreas diferentes, bem como se levar em
conta estudos universitários estatísticos que indicam quais as
informações mais buscadas por recrutadores.
\BNCC{EM13LP54}

\subsubsection{A vida no campo}

Em \emph{Gente de Hemsö}, vimos que a propriedade da família Flod está
decadente à época da chegada de Carlsson no local. Carlsson, no
início, se mostra um rapaz estudado, conhecedor dos métodos capazes de
tornarem as terras mais produtivas. Neste momento, o livro alerta para
um ponto muito importante, que é o fato de que não basta ter as
terras, é preciso saber utilizá--las. Nesse sentido, é interessante a
proposição de uma atividade com os alunos, valendo--se do auxílio de
professores das Ciências da Natureza, para debater a boa utilização
das terras para o plantio e pecuária. 

Os alunos, divididos em grupos,
devem formular apresentações em que apresentem boas estratégias de
produção agropecuária nas diversas regiões brasileiras, observando
suas peculiaridades e necessidades, pensando também em como
desenvolver essa atividade de forma que não agrida a natureza.
Para o desenvolvimento desta atividade, sugere--se a pesquisa em sites
relacionados à produção agropecuária, bem como em vídeos explicativos
nas plataformas online.
\BNCC{EM13CHS302}

\subsubsection{Progresso ou vida rústica?}

Em razão do exposto, o exercício sugere uma maior atenção ao trecho da
obra em que a indústria mineradora se instalou em Hemsö para a
extração de Feldspato. A partir desse trecho, os alunos devem ser
divididos em grupos, submetendo a questão de se a indústria deve ou
não ser instalada no local.

Para a realização do debate, devem buscar auxílio de professores de
diversas áreas do colégio, para trazer elementos do potencial econômico
do Feldspato para a economia do país e para a economia da região. Qual o
impacto ambiental dessa instalação? E o impacto sócio--cultural. Também
aconselha-se a indicação de desastres ambientais recentes como
precedentes, bem como indicar as potencialidades do desenvolvimento
sustentável. De modo que todos os grupos tenham argumentos válidos e
robustos para um bom e construtivo debate.
\BNCC{EM13CHS304}

\subsubsection{Regionalismos}

Conforme os alunos poderão perceber ao longo da leitura, especialmente
na metade inicial da Obra, Strindberg se vale de regionalismos em
muitos dos diálogos escritos, mas principalmente nas descrições de
locais e situações. Esse recurso, que seguramente dificulta o trabalho
de tradutores, em geral cumpre a função de imergir o leitor na obra,
ou então, como no caso de \emph{Gente de Hemsö}, tem o objetivo de incutir
estranhamento nas pessoas de grandes centros que toma contato com a
história, posto que esta se desenrola no meio rural.

A utilização de regionalismos é característica marcante do grande
escritor brasileiro Guimarães Rosa que, ao invés da vida no arquipélago
báltico, descreve a vida no sertão brasileiro. Dito isso, convém indicar
trechos de obras de Guimarães em que descrições com termos regionais são
contundentes.

Peça para os alunos selecionarem os termos que desconhecem. Num primeiro
momento, é interessante que os estudantes busquem no dicionário o
significado das palavras que desconhecem, tanto no texto de Guimarães,
quanto no de Strindberg. Feito isso, cada estudante pode escrever um
pequeno texto, no qual discorrerá sobre esse recuso, observando a
maneira, e as semelhanças e diferenças na sua utilização por cada um dos
autores.
\BNCC{EM13LP28}

\subsubsection{Campo das práticas de estudo e pesquisa}

Dentro da história, são apresentados uma série de elementos da
natureza, os quais podem até causar certa impressão ao leitor num
primeiro momento. Por isso, antes da leitura dessa obra, é bom que o
aluno esteja ciente do universo em que a mesmo se ambientará. Hemsö é
uma localidade fictícia, porém, é baseada no Arquipélago de Estocolmo.
Diante disso, a atividade sugere, juntamente com auxílio do professor de
humanidades e de ciências da natureza, uma pesquisa sobre essa formação.
Isto é, o que é um arquipélago? Quais são suas principais
características? Como ele se forma? Que espécies são favorecidas por sua
morfologia. Quais características especiais um arquipélago na região
Báltica pode deter? 

O resultado dessa investigação pode ser trabalhado
por meio de um artigo, ou por meio de uma apresentação para os demais
colegas, devendo, nesse caso, haver uma separação prévia da sala em
grupos, cada qual atuará com um tema. Proponha que se localizem arquipélagos
importantes no Brasil, como o do 
\href{https://pt.wikipedia.org/wiki/Parque_Nacional_de_Anavilhanas}{Parque Nacional de Anavilhanas}, no Rio Negro.
\BNCC{EM13CHS206}


\Image{Mapa do arquipélago de Estocolmo (Demis map server; Domínio Público)}{PNLD0039-08.png}


\section{Sugestões de referências complementares}\label{sugestoes}

\subsection{Filmes}

\begin{itemize}
\item \textit{Dogville}. Direção: Lars von Trier (Dinamarca/Suécia/Noruega/Finlândia/Reino Unido/França/Alemanha/Países Baixos, 2003).

Buscando refúgio de um grupo de \emph{gangsters}, uma desconhecida encontra um 
vilarejo e aceita trabalhar em troca de asilo na pequena cidade, por duas 
semanas. Esse seria o período de teste para depois, em uma votação, os 
cidadãos decidirem se ela pode ficar. Porém, assim que concedem permissão 
para ela ficar, ela passa a entender os verdadeiros termos e condições de 
morar em Dogville.

\item \textit{O mal não espera a noite --- Midsommar}. Direção: Ari Aster (Estados Unidos/Suécia, 2019).

Em uma remota vila sueca, durante o festival de verão, um jovem casal 
americano decide passar suas férias. Logo o paraíso interiorano, em uma 
terra sempre solar e iluminada, começa a preocupar os jovens, uma vez que 
os moradores do vilarejo participam de atividades estranhas e perturbadoras.

\item \textit{Morangos Silvestres}. Direção: Ingmar Bergman (Suécia, 1957).

Ao longo do trajeto para receber um prêmio, um professor de medicina 
revisita diversos momentos de sua vida, suscitando sentimentos de culpa 
e nostalgia misturados às lembranças de decepções e frustrações. 
No entanto, quando encontra seu filho, os sentimentos se intensificam.

\item \textit{Persona}. Direção: Ingmar Bergman (Suécia, 1966).

Uma atriz em crise e sua cuidadora se isolam em uma praia, 
para um retiro. Com a comunicação restrita por parte da atriz, 
que se recusava a falar, a cuidadora cria o hábito de conversar 
pelas duas. Porém, à medida em que os dias passam, a interação entre as 
duas mulheres se intensifica, ganhando outras proporções. 

\item \textit{O sétimo selo}. Direção: Ingmar Bergman (Suécia, 1957).

Um cavaleiro que retorna das Cruzadas encontra seu país devastado 
pela peste negra. Refletindo sobre o significado da vida e com sua fé 
em Deus abalada, a Morte aparece para ele, dizendo que ele precisa ir 
com ela. A fim de ganhar mais tempo, ele propõe um jogo de xadrez 
e a Morte aceita, uma vez que nunca perde o jogo.
\end{itemize}

\subsection{\emph{Sites}}

\begin{itemize}
\item Diálogos Nórdicos

No \href{https://www.dialogosnordicos.com/?fbclid=IwAR0XdcQuRZ3HH0mRrjc-z5BpiSHjWi9uj9mjTTrJ0GWvHTSsVQdf4tA4po8}{\emph{site} da página Diálogos Nórdicos}, em um 
projeto criado pelas embaixadas dos países escandinavos, propõe"--se um 
intercâmbio cultural, de perspectivas, desafios e experiências entre 
o Brasil e esses outros países.

\item Fundação Hilma af Klint

A artista sueca Hilma af Klint, pioneira na arte abstrata e que passou despercebida durante grande parte do século \textsc{xx}, teve uma retrospectiva de sua obra na \href{https://pinacoteca.org.br/en/programacao/hilma-af-klint/}{Pinacoteca} do Estado de São Paulo. Suas obras podem ser acessadas também, além do site da \href{https://www.hilmaafklint.se/en/}{Fundação Hilma af Klint}, a partir desse vídeo sobre a exposição que contou com mais de 130 obras suas em São Paulo, disponível no \href{https://www.youtube.com/watch?v=jBMQSHFEUaw&ab_channel=TVCRECI}{Youtube}.

\item Grupo Dogma 95

O movimento cinematográfico conhecido como \href{https://www.institutodecinema.com.br/mais/conteudo/movimentos-do-cinema-o-que-foi-o-dogma-95}{Dogma 95}, do qual participou o controverso diretor de cinema Lars Von Trier, tem grande 
ligação com a obra de Strindberg, sua escrita e seus componentes.
\end{itemize}

\section{Bibliografia comentada}

\begin{itemize}
\item\textsc{wallin}, Claudia. \textit{Suécia: um país sem excelências e mordomias.}
São Paulo: Geração Editorial, 2014.

A autora prova que existem políticos que não têm mordomias, que não
aumentam seu próprio salário, que usam transporte público e não estão na
vida pública para fazer fortuna. Um sistema apoiado em três pilares:
transparência, escolaridade e igualdade.

\item\textsc{carpeaux}, Otto Maria. \textit{A história Concisa da Literatura Alemã.}
Alphaville: Faro Editorial, 2013.

O autor faz uma síntese dos grandes momentos, livros e autores da
literatura alemã, e conta com uma avaliação crítica de sua importância
para a cultura e o desenvolvimento do país e sua influência nos
principais movimentos culturais do mundo contemporâneo.

\item \_\_\_\_\_\_. \textit{A história da Literatura Ocidental.} São
Paulo: Leya, 2019.

Da literatura grega à contemporânea, Carpeaux analisa e critica as obras
com seu arcabouço teórico. Seu conhecimento nos leva ao encontro dos
mais importantes autores da literatura ocidental.

\item\textsc{larsson}, Stieg. \textit{Os homens que não amavam as mulheres.} São
Paulo: Companhia das Letras, 2010.

Os homens que não amavam as mulheres é uma fascinante e assustadora
aventura vivida por um veterano jornalista e uma jovem e genial hacker
cujo comportamento social beira o autismo.

\item\textsc{lindrgen}, Astrid. \textit{Píppi Meialonga}. São Paulo: Companhia das
Letras, 2001.

Píppi é uma menina de nove anos incrivelmente forte. Não tem pai nem mãe
e mora sozinha. Seus companheiros são um cavalo e um macaquinho e a
garota faz suas roupas, enfrenta valentões e sempre possui uma resposta
na ponta da língua, com grande confiança em si.
\end{itemize}

\end{document}

