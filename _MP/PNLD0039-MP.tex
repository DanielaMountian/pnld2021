\documentclass[12pt]{extarticle}
\usepackage{manualdoprofessor}
\usepackage{fichatecnica}
\usepackage{lipsum,media9,graficos}
\usepackage[justification=raggedright]{caption}
\usepackage{bncc}
\usepackage[papagaio]{../edlab}



\begin{document}


\newcommand{\AutorLivro}{Strindberg}
\newcommand{\TituloLivro}{Gente de Hemsö}
\newcommand{\Tema}{Ficção, mistério e fantasia}
\newcommand{\Genero}{Romance}
\newcommand{\imagemCapa}{./images/PNLD0039-01.png}
\newcommand{\issnppub}{---}
\newcommand{\issnepub}{---}
% \newcommand{\fichacatalografica}{PNLD0039-00.png}
\newcommand{\colaborador}{\textbf{Bruno Gradella e Vicente Castro} é uma pessoa incrível e vai fazer um bom serviço.}


\title{\TituloLivro}
\author{\AutorLivro}
\def\authornotes{\colaborador}

\date{}
\maketitle

\baselineskip=1.15\baselineskip\par

\begin{abstract}
Este Manual tem como objetivo fornecer subsídios para o trabalho com a
obra literária \emph{Gente de Hemsö}, obra de autoria de August
Strindberg.

\end{abstract}


\tableofcontents




\section{Atividades 1}

%\BNCC{EM13LP26}

\subsection{Pré-leitura}

%\BNCC{EM13LGG302}
%\BNCC{EM13LGG704}
%\BNCC{EM13LP10}
%\BNCC{EM13LP19}

Antes de iniciar a leitura, é conveniente que os alunos
busquem informações críticas acerca da literatura, como por exemplo,
quais são os gêneros literários? E quais são as escolas literárias mais
importantes que existiram na história? O domínio desse conteúdo é
importante, pois fomenta uma maior absorção do conteúdo direto e
simbólico que um autor desejou imprimir em uma obra.

Desse modo, é aconselhado que a sala de aula seja dividida em grupos e
que os alunos pesquisem, de acordo com a divisão proposta pelo
professor, essas características que compõem as obras literárias.

Para isso, a sala pode ser dividida em grupos, cada qual adotando uma
escola literária para pesquisar, tendo como objetivo a produção de um
material de consulta, como um site enciclopédico, que poderá ser linkado
no site do colégio.

\subsection{Leitura}

%\BNCC{EM13LGG103}
%\BNCC{EM13LP02}
%\BNCC{EM13LP48}

Com o material da atividade anterior em mãos, e conforme a
leitura se desenvolve, é interessante propor o registro de personagens
em uma espécie de lista, posteriormente, colocando-a no formato de um
mapa conceitual. É possível, então, indicar características de
personalidades de cada uma das personagens. E como as personagens se
desenvolvem ao longo da trama, os alunos podem, conforme a leitura for
avançando, anotem e fixem, ao lado de cada uma das personagens,
adjetivos que ajudem a caracterizar comportamentos, trechos de falas ou
de passagens emblemáticas. Podem se valer também de setas coloridas para
reforçar visualmente os vínculos que vão se formando entre os
caracteres. Esse mapa poderá ajudar em etapas posteriores da análise e
da recuperação e síntese de informações pós-leitura. Poderá também ser
enriquecido por outros elementos verbais e não verbais, decorrentes das
atividades interdisciplinares com a obra.

\subsection{Pós-leitura}

%\BNCC{EM13LGG102}
%\BNCC{EM13LGG303}
%\BNCC{EM13LGG402}
%\BNCC{EM13LGG703}
%\BNCC{EM13LP13}
%\BNCC{EM13LP14}
%\BNCC{EM13LP28}
%\BNCC{EM13LP29}
%\BNCC{EM13LP52}

Após a leitura, abra uma roda com os estudantes, feito isso
proponha um debate sobre elementos da realidade rural. Como eles
influenciam o modo de vida entre das pessoas. As relações sociais, as
relações com a cidade, com a natureza e com o tempo. Mesmo se passando
em um arquipélago, em um país distanate, podemos encontrar similitudes
com elementos de nossa cultura. Uma curiosidade é analisar a personagem
Rundqvist, que com suas artimanhas e mandingas pode ser comparado aos
sacis de nosso folclore. Indique aos alunos materiais que tratam do
sertanismo, ou que se ambientem no universo rural. Munidos da leitura de
Strindberg, com os elementos fornecidos pelas obras da literatura
nacional, convide os alunos a produzirem um conto, de linguagem simples
e narrativa direta nos moldes do autor sueco, porém ambientado em
território brasileiro. Saliente a importância da descrição dos ambientes
e das pessoas, de seus costumes, dos animais locais, etc. O material
poderá ser disponibilizado no site do colégio posteriormente.

\section{Atividades 2}

%\BNCC{EM13CNT201}
%\BNCC{EM13CNT303}
%\BNCC{EM13CHS101}
%\BNCC{EM13CHS102}
%\BNCC{EM13CHS106}
%\BNCC{EM13CHS401}

A obra \emph{Gente de Hemsö} possibilita trabalhos interdisciplinares e
integradores de diferentes campos do saber e áreas de conhecimento. A
seguir, propomos algumas atividades que podem ser desenvolvidas
conjuntamente com professores de outras áreas. Além das habilidades de
Linguagens e suas Tecnologias e de Língua Portuguesa, indicadas nas
etapas da seção anterior e válidas também para esta, listamos a seguir
as habilidades de outras áreas, presentes na abordagem interdisciplinar:

\subsection{Pré-leitura}

Para conhecer melhor o universo da obra de Strindberg,
recomenda-se que, juntamente com os professores de humanidades, seja
realizada uma pesquisa aprofundada sobre o contexto histórico-cultural
que o autor estava inserido. Nesta atividade, recomenda-se a divisão da
sala em grupos, cada qual deverá pesquisar um aspecto dos elementos que
possivelmente compuseram as referências de Strindberg. Para tanto,
recomenda-se que um grupo pesquise aspectos geográficos da Suécia.
Outro, fie-se nas questões históricas envolvendo o país. Um outro busque
referências culturais e folclóricas e, por fim, um último grupo busque
informações acerca do contexto cultural e literário europeu da época de
Strindberg. Feito isso, convida-se à realização de um seminário onde os
alunos exporão suas descobertas. Durante as exposições, os alunos
deverão também, anotar os detalhes do divulgado pelos outros grupos
para, ao fim, a sala montar um texto informativo que aglutine todos os
elementos obtidos.

\subsection{Leitura}

Com o material produzido na atividade anterior, é válido
explorar como a obra de Strindberg está inserida no contexto de sua
época. Foi um período, ao mesmo tempo de otimismo, e de contestação.
Havia a contraposição das identidades nacionais, coletivas e sociais,
frente às angústias do indivíduo. Pesquisando um pouco a biografia do
autor e sua produção literária fomente a discussão desses temas com os
alunos, podendo se valer do auxílio dos professores de humanidades para
melhor contextualização. Entretanto, ao fim, peça para que os alunos se
debrucem sobre o conteúdo do material que está sendo lido. Peça para que
esses escrevam um pequeno parágrafo acerca da solução proposta por
Strindberg para todas essas angústias.


\Image{Foto de Strindberg após seu aniversário de 50 anos. (Autor desconhecido; Domínio Público)}{PNLD0039-03.png}


Debates sociais eram muito frequentes nesse momento, havendo a busca por
direitos das mulheres, questões de identidade nacional e
recupera elementos típicos da cultura e da história escandinavas. É
interessante, neste momento, abordar obras de outros escritores
escandinavos, como Søren Kierkegaard e Hans Christian Andersen,
apresentando aos alunos pequenos trechos de suas obras.

Feito isso, aconselha-se que os alunos busquem elementos em comum,
afinal, todos viveram em um período próximo, que foi uma época em que
questões de identidade pessoal, nacional e social eram muito discutidas.
Com essas informações em mãos, e com o auxílio dos professores de
humanidades, proponha um debate acerca do nacionalismo, da ideia de
pertencimento ou exclusão a um determinado grupo.

\subsection{Pós-leitura}

Munidos do conhecimento do universo de Hemsö, os alunos
poderão, com o auxílio dos professores, promover a realização de uma
feira de cultura escandinava no colégio. Esta pode ser feita no local,
com o aproveitamento das salas para exposição de réplicas de elementos
de cultura material. As salas podem estar, por exemplo, divididas de
acordo com períodos. Podem ser encenadas danças típicas, e os alunos
podem pesquisar e cozinhar receitas típicas da região. Caso seja feita a
opção virtual, os alunos podem criar um museu virtual, com auxílio de
aplicativos próprios para isso, onde o visitante será convidado a
singrar por salas onde encontrará um pouco da história e cultura da
Suécia e de outros países nórdicos.

\section{Aprofundamento}

Ao chegar ao Ensino Médio, é necessário que os estudantes se aprofundem
na compreensão das múltiplas linguagens e, sobretudo, da linguagem
literária. Em relação à literatura, a BNCC traz as seguintes
considerações:

\begin{quote}
{[}...{]} a leitura do texto literário, que ocupa o centro do trabalho
no Ensino Fundamental, deve permanecer nuclear também no Ensino Médio.
Por força de certa simplificação didática, as biografias de autores, as
características de épocas, os resumos e outros gêneros artísticos
substitutivos, como o cinema e as HQs, têm relegado o texto literário a
um plano secundário do ensino. Assim, é importante não só (re)colocá-lo
como ponto de partida para o trabalho com a literatura, como
intensificar seu convívio com os estudantes. Como linguagem
artisticamente organizada, a literatura enriquece nossa percepção e
nossa visão de mundo. Mediante arranjos especiais das palavras, ela cria
um universo que nos permite aumentar nossa capacidade de ver e sentir.
Nesse sentido, a literatura possibilita uma ampliação da nossa visão do
mundo, ajuda-nos não só a ver mais, mas a colocar em questão muito do
que estamos vendo/vivenciando. (Brasil, 2018, p. 491)
\end{quote}

Nesta seção, desenvolvemos um trabalho de aprofundamento que, em diálogo
com a formação continuada de professores, oferece subsídios para a
abordagem do texto literário. A leitura em sala de aula de \emph{Gente
de Hemsö} pode ser enriquecida pelo aprofundamento no universo literário
em que a obra está inserida.

\subsection{A obra}

Gente de Hemsö é um romance que descreve a vida dos habitantes da ilha
de Hemsö -- local fictício que, teoricamente, estaria localizado nas
ilhotas do Báltico, próximo à capital sueca. É uma obra que dialoga com
o próprio autor, pois, em sua infância, frequentemente visitava ilhas
nesses arquipélagos e, quando escreveu, estava em um auto-exílio, sendo
a obra, também, uma forma de se reconectar com seu país e sua história.


\Image{Vista da ilha Öland, na Suécia, no mar Báltico, localização aproximada da fictícia ilha de Hëmso. (Arnold Paul; CC BY-SA 2.5)}{PNLD0039-06.png}


\subsection{A sabedoria da simplicidade}

Uma vez, em tom de lamento, August Srindberg afirmou:

\begin{quote}
A maioria acredita ser profundo aquilo que apenas é rebuscado. Isso é
errado. O rebuscado é apenas o mal-realizado; o obscuro costuma ser
falso. A sabedoria mais alta é simples, clara e atravessa o cérebro até
o coração.
\end{quote}

Essa frase de Strindberg define o tecido que envolve sua obra. É um
romance simples. Uma história sobre a vida rural nos arquipélagos
suecos. Trata-se de uma história sem estampidos, mas mesmo assim, muito
tocante.


\Image{Retrato de Strindberg com 25 anos (Studio de Math. S. Hansen; Domínio Público)}{PNLD0039-04.png}


Strindberg foge da afetação burguesa, comumente retratada na literatura
de seu tempo, para narrar a simplesmente a vida como ela é. Ela ocorre,
é certa como os ventos do Mar Báltico, oferta seus percalços, mas mesmo
assim segue. E, como restará claro no desenrolar da história, muitas
vezes os problemas surgem quando tentamos atribuir predicados àquilo que
não precisa. Uma vida boa seria, então, uma vida tranquila. No caso, uma
vida desprovida de preocupações. E as preocupações praticamente vão
embora quando nós apenas aceitamos as coisas como elas são. Na sua
organicidade natural, elas se ajeitam.


\Image{Retrato de Strindberg pintado por Richard Bergh em 1905 (Richard Bergh; Domínio Público)}{PNLD0039-05.png}


\subsection{A simplicidade na escrita}

Strindberg escreve de uma maneira muito clara. Embora em um primeiro
momento até assuste a quantidade de descrições oferecidas pelo autor,
isso é algo que, paulatinamente vai se perdendo ao longo do texto.

É provável que o autor, com isso, desejou oferecer ao leitor a
perspectiva da personagem principal, Carlsson, que, seguramente, vindo
da cidade para a região de Hemsö se deslumbrava com os elementos
diversos do mundo que conhecia.

Entretanto, com o tempo, ele próprio é cooptado para o universo de
Hemsö, de modo que não lhe mais chamam tanto a atenção esses detalhes. E
curiosamente, as descrições diminuem conforme Carlsson se integra ao
universo.

Vemos também que Strindberg usa uma forma de escrever muito direta. As
personagens sabem de antemão o que acontecera em outra cena, eliminando
assim longos diálogos. Ao próprio leitor também são dadas as informações
sem explicações demoradas.

Os capítulos também são breves e objetivos. No caso, recuperando seu
próprio pensamento, Strindberg aplica a sabedoria do necessário.
Eliminando de seu texto tudo que é supérfluo.

Chega a ser engraçada a parte que Carlsson pula trechos da Bíblia a fim
de terminar a reunião religiosa mais rapidamente. Quase como se
Strindberg brincasse com os leitores que faziam isso em algumas de suas
descrições um pouco mais longas.

\subsection{Relação originária com a natureza}

\Image{O diretor Ingmar Bergman, coterrâneo de Strindberg, no set do filme Morangos Silvestres. (Louis Huch; Domínio Público)}{PNLD0039-09.png}
Nesse livro observamos as constantes menções aos ciclos, rotação de
cultura, rotação nos defesos e temporadas de pesca. Enfim, ao ritmo da
natureza. E, em certo sentido, a natureza em Hemsö domina o homem. Para
ilustrar isso, há de se rememorar a passagem em que Strindberg escreve
sobre o pastor Nordström:

\begin{quote}
Aquele que não conhecesse o pastor Nordström jamais adivinharia que este
ilhéu exercia uma função espiritual (...) de tanto que os trinta anos
tomando conta das almas do arquipélago haviam transformado o antes
refinado predicante, quando este veio ordenado de Uppsala.
\end{quote}

E toda vez que nos é apresentado o pastor, ele é tratado como um
grosseirão. Sem paciência, beberrão, que gosta de passar mais tempo
pescando do que presidindo os cultos.

Assim, é muito mais próximo dos habitantes de Hemsö, que não se
incomodam nas longas caçadas ou pescarias que varam dias. Mas consideram
inconveniente pegar o barco para ir até a Igreja.

Curiosamente, Carlsson é constantemente maltratado pelo pastor. É só
após seu fim, que o primeiro recebe a redenção do segundo. Claramente já
o enxergando como um par. Desse modo, é outro que veio da cidade mas foi
cooptado pelo ambiente de Hemsö.

Mas de forma alguma, essa ``aculturação'' é transmitida de uma forma
ruim. Ela apenas acontece, sem juízo de valor. É apenas algo natural
àquele ambiente.

A obra, assim, guarda linda semelhança com o filme Morangos Silvestres,
de Ingmar Bergman, conterrâneo de Strindberg, que, como este, foi capaz
de contar uma bonita história pautada na boa relação com o ambiente e na
simplicidade.




\Image{O diretor Ingmar Bergman e o ator principal do filme Morangos Silvestres, Victor Sjöström. (Åke Blomquist; Domínio Público)}{PNLD0039-10.png}


É curioso que no início do livro, Strindberg indica que o lápis e o
caderno de Carlsson passaram a ser respeitados, ou seja, os ensinamentos
da cidade, que surgiram em Hemsö nas mãos daquele forasteiro passaram a
ser respeitados, pois era clara a produtividade que eles traziam à
propriedade. Porém, a dicotomia que se faz na história é que, a cidade
otimiza o campo, influenciando. Mais o campo influencia o homem. A
tecnologia extrai aquilo que há na terra, mas a natureza extrai o que há
no homem.


\Image{Ilha de Utö, localizada no arquipélago de Estocolmo. Ao fundo, vemos o mar Báltico. (Arild Vågen ; CC BY-SA 2.0)}{PNLD0039-07.png}


\section{Sugestões de atividades complementares: relações dialógicas e
intertextuais}

%\BNCC{EM13LP03}
%\BNCC{EM13LP04}
%\BNCC{EM13LP49}
%\BNCC{EM13LP51}

No Ensino Médio, da mesma forma que no Ensino Fundamental, a \textsc{bncc}
organiza o trabalho com as práticas de linguagem em cinco \textbf{campos
de atuação social}. São eles: campo da vida pessoal, campo da vida
pública, campo jornalístico"-midiático, campo artístico"-literário e campo
das práticas de estudo e pesquisa.

De acordo com essa divisão, propomos na sequência um trabalho
interdiscursivo e intertextual com a obra \emph{Gente Hemsö}

\subsection{Campo da vida pessoal}

\begin{quote}
O campo da vida pessoal pretende funcionar como espaço de articulações
e sínteses das aprendizagens de outros campos postas a serviço dos
projetos de vida dos estudantes. As práticas de linguagem privilegiadas
nesse campo relacionam"-se com a ampliação do saber sobre si, tendo em
vista as condições que cercam a vida contemporânea e as condições
juvenis no Brasil e no mundo.

Está em questão também possibilitar vivências significativas de práticas
colaborativas em situações de interação presenciais ou em ambientes
digitais e aprender, na articulação com outras áreas, campos e com os
projetos e escolhas pessoais dos jovens, procedimentos de levantamento,
tratamento e divulgação de dados e informações e o uso desses dados em
produções diversas e na proposição de ações e projetos de natureza
variada, para fomentar o protagonismo juvenil de forma
contextualizada. (\textsc{bncc}, p. 494)
\end{quote}

Na história de Strindberg, acompanhamos a trajetória de Carlsson,
jovem habilidoso que vai até Hemsö para tentar a vida como capataz da
fazenda. Sendo estranho àquela comunidade, Carlsson passa por uns
percalços no início de sua aventura, entre eles, a desconfiança dos
outros habitantes da fazenda e da comunidade ao redor dela. Tal
situação não é incomum no mercado de trabalho. Nesse sentido, é
possível até dizer que Carlsson teve sorte, pois conseguira o emprego.
Muitos jovens, no entanto, são barrados até mesmo antes disso. Uma
forma clássica, mas ainda muito eficaz de se adentrar no mercado de
trabalho é por meio de um bom currículo. E um bom currículo não
significa um currículo robusto, mas sim, organizado, claro e bem
apresentado. Nessa atividade, proponha um exercício de redação para os
alunos, cujo produto final será a formulação de um currículo para cada
um deles. Para aumentar o acervo de referências, pode-se pesquisar
modelos de currículos para áreas diferentes, bem como se levar em
conta estudos universitários estatísticos que indicam quais as
informações mais buscadas por recrutadores.

\subsection{Campo de atuação na vida pública}

\begin{quote}
No cerne do campo de atuação na vida pública estão a ampliação da
participação em diferentes instâncias da vida pública, a defesa dos
direitos, o domínio básico de textos legais e a discussão e o debate de
ideias, propostas e projetos. {[}\ldots{}{]}

Ainda no domínio das ênfases, indica"-se um conjunto de habilidades que
se relacionam com a análise, discussão, elaboração e desenvolvimento de
propostas de ação e de projetos culturais e de intervenção social.
(\textsc{bncc}, p. 494)
\end{quote}

Em \emph{Gente de Hemsö}, vimos que a propriedade da família Flod está
decadente à época da chegada de Carlsson no local. Carlsson, no
início, se mostra um rapaz estudado, conhecedor dos métodos capazes de
tornarem as terras mais produtivas. Neste momento, o livro alerta para
um ponto muito importante, que é o fato de que não basta ter as
terras, é preciso saber utilizá-las. Nesse sentido, é interessante a
proposição de uma atividade com os alunos, valendo-se do auxíio de
professores das ciências da natureza, para debater a boa utilização
das terras para o plantio e pecuária. Os alunos, divididos em grupos,
devem formular apresentações em que apresentem boas estratégias de
produção agropecuária nas diversas regiões brasileiras, observando
suas peculiaridades e necessidades, pensando também, em como
desenvolver essa atividade de uma forma que não agrida a natureza.
Para o desenvolvimento desta atividade, sugere-se a pesquisa em sites
relacionados à produção agropecuária, bem como em vídeos explicativos
nas plataformas online.

\subsection{Campo jornalístico"-midiático}

\begin{quote}
Em relação ao campo jornalístico"-midiático, espera"-se que os jovens
que chegam ao Ensino Médio sejam capazes de: compreender os fatos e
circunstâncias principais relatados; perceber a impossibilidade de
neutralidade absoluta no relato de fatos; adotar procedimentos básicos
de checagem de veracidade de informação; identificar diferentes pontos
de vista diante de questões polêmicas de relevância social; avaliar
argumentos utilizados e posicionar"-se em relação a eles de forma ética;
identificar e denunciar discursos de ódio e que envolvam desrespeito aos
Direitos Humanos; e produzir textos jornalísticos variados, tendo em
vista seus contextos de produção e características dos gêneros. Eles
também devem ter condições de analisar estratégias
linguístico"-discursivas utilizadas pelos textos publicitários e de
refletir sobre necessidades e condições de consumo.

No Ensino Médio, os jovens precisam aprofundar a análise dos interesses
que movem o campo jornalístico midiático, da relação entre informação e
opinião, com destaque para o fenômeno da pós"-verdade, consolidar o
desenvolvimento de habilidades, apropriar"-se de mais procedimentos
envolvidos na curadoria de informações, ampliar o contato com projetos
editoriais independentes e tomar consciência de que uma mídia
independente e plural é condição indispensável para a democracia.

Como já destacado, as práticas que têm lugar nas redes sociais têm
tratamento ampliado. (\textsc{bncc}, p. 494-495)
\end{quote}


Em razão do exposto, o exercício sugere uma maior atenção ao trecho da
obra em que a indústria mineradora se instalou em Hemsö para a
extração de Feldspato. A partir desse trecho, os alunos devem ser
divididos em grupos, submetendo a questão de se a indústria deve ou
não ser instalada no local.

Para a realização do debate, devem buscar auxílio de professores de
diversas áreas do colégio, para trazer elementos do potencial econômico
do Feldspato para a economia do país e para a economia da região. Qual o
impacto ambiental dessa instalação? E o impacto sócio-cultural. Também
aconselha-se a indicação de desastres ambientais recentes como
precedentes, bem como indicar as potencialidades do desenvolvimento
sustentável. De modo que todos os grupos tenham argumentos válidos e
robustos para um bom e construtivo debate.

\subsection{Campo artístico"-literário}

\begin{quote}
No campo artístico"-literário busca"-se a ampliação do contato e a
análise mais fundamentada de manifestações culturais e artísticas em
geral. Está em jogo a continuidade da formação do leitor literário e do
desenvolvimento da fruição. A análise contextualizada de produções
artísticas e dos textos literários, com destaque para os clássicos,
intensifica"-se no Ensino Médio. Gêneros e formas diversas de produções
vinculadas à apreciação de obras artísticas e produções culturais
(resenhas, vlogs e podcasts literários, culturais etc.) ou a formas de
apropriação do texto literário, de produções cinematográficas e teatrais
e de outras manifestações artísticas (remidiações, paródias,
estilizações, videominutos, fanfics etc.) continuam a ser considerados
associados a habilidades técnicas e estéticas mais refinadas.

A escrita literária, por sua vez, ainda que não seja o foco central do
componente de Língua Portuguesa, também se mostra rica em possibilidades
expressivas. (\textsc{bncc}, p. 495-496).
\end{quote}

Conforme os alunos poderão perceber ao longo da leitura, especialmente
na metade inicial da Obra, Strindberg se vale de regionalismos em
muitos dos diálogos escritos, mas principalmente nas descrições de
locais e situações. Esse recurso, que seguramente dificulta o trabalho
de tradutores, em geral cumpre a função de imergir o leitor na obra,
ou então, como no caso de Gente de Hemsö, tem o objetivo de incutir
estranhamento nas pessoas de grandes centros que toma contato com a
história, posto que esta se desenrola no meio rural.

A utilização de regionalismos é característica marcante do grande
escritor brasileiro Guimarães Rosa que, ao invés da vida no arquipélago
báltico, descreve a vida no sertão brasileiro. Dito isso, convém indicar
trechos de obras de Guimarães em que descrições com termos regionais são
contundentes.

Peça para os alunos selecionarem os termos que desconhecem. Num primeiro
momento, é interessante que os estudantes busquem no dicionário o
significado das palavras que desconhecem, tanto no texto de Guimarães,
quanto no de Strindberg. Feito isso, cada estudante pode escrever um
pequeno texto, no qual discorrerá sobre esse recuso, observando a
maneira, e as semelhanças e diferenças na sua utilização por cada um dos
autores.

\subsection{Campo das práticas de estudo e pesquisa}

\begin{quote}
O campo das práticas de estudo e pesquisa mantém destaque para os
gêneros e habilidades envolvidos na leitura/escuta e produção de textos
de diferentes áreas do conhecimento e para as habilidades e
procedimentos envolvidos no estudo. Ganham realce também as habilidades
relacionadas à análise, síntese, reflexão, problematização e pesquisa:
estabelecimento de recorte da questão ou problema; seleção de
informações; estabelecimento das condições de coleta de dados para a
realização de levantamentos; realização de pesquisas de diferentes
tipos; tratamento dos dados e informações; e formas de uso e
socialização dos resultados e análises.

Além de fazer uso competente da língua e das outras semioses, os
estudantes devem ter uma atitude investigativa e criativa em relação a
elas e compreender princípios e procedimentos metodológicos que orientam
a produção do conhecimento sobre a língua e as linguagens e a formulação
de regras. (\textsc{bncc}, p. 495-496)
\end{quote}

Dentro da história, são apresentados uma série de elementos da
natureza, os quais podem até causar certa impressão ao leitor num
primeiro momento. Por isso, antes da leitura dessa obra, é bom que o
aluno esteja ciente do universo em que a mesmo se ambientará. Hemsö é
uma localidade fictícia, porém, é baseada no Arquipélago de Estocolmo.
Diante disso, a atividade sugere, juntamente com auxílio do professor de
humanidades e de ciências da natureza, uma pesquisa sobre essa formação.
Isto é, o que é um arquipélago? Quais são suas principais
características? Como ele se forma? Que espécies são favorecidas por sua
morfologia. Quais características especiais um arquipélago na região
Báltica pode deter? O resultado dessa investigação pode ser trabalhado
por meio de um artigo, ou por meio de uma apresentação para os demais
colegas, devendo, nesse caso, haver uma separação prévia da sala em
grupos, cada qual atuará com um tema.


\Image{Mapa do arquipélago de Estocolmo (Demis map server; Domínio Público)}{PNLD0039-08.png}


\section{Referências complementares}

\begin{itemize}
\item\textsc{larsson}, Stieg. \textbf{Os homens que não amavam as mulheres.} São
Paulo: Companhia das Letras, 2010.

Os homens que não amavam as mulheres é uma fascinante e assustadora
aventura vivida por um veterano jornalista e uma jovem e genial hacker
cujo comportamento social beira o autismo.

\item\textsc{lindrgen}, Astrid. \textbf{Píppi Meialonga}. São Paulo: Companhia das
Letras, 2001.

Píppi é uma menina de nove anos incrivelmente forte. Não tem pai nem mãe
e mora sozinha. Seus companheiros são um cavalo e um macaquinho e a
garota faz suas roupas, enfrenta valentões e sempre possui uma resposta
na ponta da língua, com grande confiança em si.
\end{itemize}

\section{Bibliografia comentada}

\begin{itemize}
\item\textsc{wallin}, Claudia. \textbf{Suécia: um país sem excelências e mordomias.}
São Paulo: Geração Editorial, 2014.

A autora prova que existem políticos que não têm mordomias, que não
aumentam seu próprio salário, que usam transporte público e não estão na
vida pública para fazer fortuna. Um sistema apoiado em três pilares:
transparência, escolaridade e igualdade.

\item\textsc{carpeaux}, Otto Maria\textbf{. A história Concisa da Literatura Alemã.}
Alphaville: Faro Editorial, 2013.

O autor faz uma síntese dos grandes momentos, livros e autores da
literatura alemã, e conta com uma avaliação crítica de sua importância
para a cultura e o desenvolvimento do país e sua influência nos
principais movimentos culturais do mundo contemporâneo.

\item\textsc{carpeaux}, Otto Maria\textbf{. A história Literatura Ocidental.} São
Paulo: Leya, 2019.

Da literatura grega à contemporânea, Carpeaux analisa e critica as obras
com seu arcabouço teórico. Seu conhecimento nos leva ao encontro dos
mais importantes autores da literatura ocidental.
\end{itemize}

\end{document}


