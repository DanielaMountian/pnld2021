\documentclass{extarticle}
\usepackage{manualdoprofessor}
\usepackage{fichatecnica}
\usepackage{lipsum,media9,graficos}
\usepackage[justification=raggedright]{caption}
\usepackage[one]{bncc}
\usepackage[piseagrama]{../edlab}
\begin{document}


\newcommand{\AutorLivro}{Machado de Assis}
\newcommand{\TituloLivro}{O Alienista — O Imortal — A Cartomante}
\newcommand{\Tema}{Ficção, mistério e fantasia}
\newcommand{\Genero}{Conto, crônica e novela}
\newcommand{\imagemCapa}{./images/PNLD0048-01.png}
\newcommand{\issnppub}{---}
\newcommand{\issnepub}{---}
% \newcommand{\fichacatalografica}{PNLD0048-00.png}
\newcommand{\colaborador}{\textbf{Carlos Rogério Duarte Barreiros}}


\title{\TituloLivro}
\author{\AutorLivro}
\def\authornotes{\colaborador}

\date{}
\maketitle

\begin{abstract}\addcontentsline{toc}{section}{Carta ao professor}

É com muita alegria que lhe apresentamos esta edição de uma novela e
dois contos de Machado de Assis: ``O Alienista'', ``O Imortal'' e ``A
Cartomante''. Cada um desses textos tem um interesse específico e,
reunidos, eles compõem um conjunto bastante fértil para análise em sala
de aula. Mas, antes de falar de cada um deles especificamente, é
necessário avaliar a importância de ler Machado de Assis no Ensino Médio
e justificar, de forma geral, nossa escolha pelos textos desse autor
consagrado.

Primeiramente, acreditamos que nosso maior clássico pode ser um ótimo
ponto de partida para os estudantes conhecerem outros autores.
Acreditamos também que tudo depende da escolha dos textos: para ser
porta de entrada para outras obras, o Machado que vamos ler tem de ser
do interesse de nossos alunos -- por isso é que selecionamos os três
contos desta antologia: em todos eles podem ser identificados temas
atuais, que merecem discussão. Esse é um dos maiores prazeres da
leitura, que queremos despertar em nossos estudantes: um texto puxa o
outro, um tema contido aqui se repete ali e alimenta a curiosidade, o
interesse, a pesquisa. Antes de existirem os mecanismos de busca da
internet, os melhores livros já continham em si, potencialmente, a
dinâmica do \textit{link}, que leva a outro, e a mais outro, quase infinitamente.

Mas não é só isso. Machado de Assis não se tornou nosso maior clássico
por acaso. Entre as muitas habilidades que tinha como escritor, uma se
destaca para nós, professores: a de conseguir transformar em forma
literária concreta, que podemos saborear na leitura, a dinâmica
específica de nosso país. Dizendo de forma simples, Machado sabia
transformar a experiência de ser brasileiro em texto literário, não
apenas por meio de descrições de lugares, pessoas e práticas sociais,
mas também por criar uma \emph{forma literária} que nos caracteriza e
com a qual nos identificamos. Somente grandes escritores têm essa
habilidade -- e é nossa função mostrá-la aos estudantes, para que eles
possam avaliar melhor os escritores contemporâneos à luz dessa
experiência acumulada. Não precisamos nos restringir à leitura de
Machado de Assis nas nossas salas de aula, mas ele é sempre um bom farol
para iluminar as avaliações que fazemos da produção literária atual,
porque poucos escritores entenderam nossa sociedade de forma tão
aprofundada e precisa.

Um exemplo simples dessa habilidade de Machado de Assis está na
linguagem de seu texto. Se compararmos as obras de Machado com as de
autores da mesma época, como Raul Pompeia, Eça de Queirós ou Júlia Lopes
de Almeida -- todos os três exímios prosadores -- perceberemos como
Machado se destaca, não porque seja necessariamente \emph{melhor} do que
eles (cada um deles é grande escritor à sua maneira), mas por soar, de
certa forma, mais próximo do leitor do presente. A capacidade de manter
uma prosa que tem um pé nos registros mais formais e outro na
informalidade quase coloquial permite que o jovem leitor do século \textsc{xxi}
se aproxime com mais facilidade de Machado de Assis. Nesse sentido,
talvez só Lima Barreto tenha conseguido uma linguagem tão acessível.
Esses dois escritores inscrevem no vocabulário, na sintaxe, na dicção e
no ritmo de seus textos um registro fluente, sem abandonar a linguagem
típica da época, mais estilizada e menos acessível aos nossos
estudantes, mas ultrapassando"-a e renovando"-a. Machado de Assis se
torna, assim, um caso quase único de \emph{clássico} de leitura
relativamente fácil. E esse é também um ótimo motivo para continuar a
lê"-lo na sala de aula.

Com Machado de Assis é sempre assim: cada geração descobre em seus
textos aspectos que não tinham a atenção dos leitores do passado,
renovando a obra de nosso grande escritor. E é por isso que convidamos
você, professora, professor, a mergulhar mais uma vez na obra dele. Nas
próximas páginas, você encontrará:



\begin{itemize}
\item
  a \textbf{Proposta de Atividades I}, voltada aos professores de Língua
  Portuguesa, formulada nos termos da Base Nacional Comum Curricular,
  dividida em atividades pré-leitura, atividades de leitura e atividades
  pós-leitura;
\item
  a \textbf{Proposta de Atividades II,} com os mesmos objetivos e
  estrutura, mas voltada especificamente a professores de outras
  disciplinas, para que eles possam usar também os textos de nossa
  antologia;
\item
  o \textbf{Aprofundamento}, também voltado aos professores de Língua
  Portuguesa, com orientações que lhe permitam compreender melhor a
  obra, seu gênero e sua linguagem, por meio do exercício da leitura
  crítica, criativa e propositiva, articulando-a com outras, literárias
  e não literárias;
\item
  as \textbf{Sugestões de referências complementares}, com diversas
  fontes de análise, para que você possa fazer os contos analisados
  dialogarem com outras obras e a \textbf{bibliografia comentada}, com
  referências bibliográficas para que você possa preparar sua aula com
  profundidade.
\end{itemize}

Bom trabalho!

\end{abstract}

\tableofcontents


\section{Atividades 1}
%\BNCC{EM13LP26}


\subsection{Pré-leitura}

%\BNCC{EM13LGG102}
%\BNCC{EM13LGG202}
%\BNCC{EM13LGG301}
%\BNCC{EM13LGG603}
%\BNCC{EM13LGG703}
%\BNCC{EM13LP01}
%\BNCC{EM13LP13}
%\BNCC{EM13LP14}
%\BNCC{EM13LP15}
%\BNCC{EM13LP47}

\paragraph{Tema} Vida e obra de Machado de Assis.

\paragraph{Conteúdo} Análise da vida e da obra de Machado de Assis.
\SideImage{Retrato de Machado de Assis, 1890 (Marc Ferrez; Domínio público)}{PNLD0048-03.png}

\paragraph{Objetivos}

\begin{enumerate}
\item Introduzir conhecimentos gerais a respeito da
vida e da obra de Machado de Assis; 
\item Contextualizar os textos da
coletânea no conjunto dessa obra e em seu momento histórico; 
\item Debater
os critérios devido aos quais Machado de Assis é considerado o maior
escritor da Literatura Brasileira; 
\item Avaliar a atualidade da obra de
Machado de Assis e seu alcance entre os leitores do século \textsc{xxi}.
\end{enumerate}

\paragraph{Justificativa} Contextualizar a leitura é motivar os alunos a
efetuá-la, daí a proposta de apresentar o autor e o conjunto da sua obra
antes de partir para os textos da antologia. No caso específico de
Machado de Assis, essa atividade talvez seja ainda mais necessária
devido ao caráter canônico da obra desse escritor. De forma simples,
trata-se de responder à pergunta ``por que Machado de Assis é
considerado o maior escritor da Literatura Brasileira?'' e ``por que é
necessário estudá"-lo ainda hoje, cerca de 140 anos depois de sua
atividade literária mais marcante?''.

\paragraph{Metodologia}

\begin{enumerate}
\item Dividir os estudantes em grupos de pesquisa que analisarão diferentes
passagens da vida e da obra machadianas e as apresentarão aos colegas e
ao professor. Sugerimos a divisão a seguir:

\begin{enumerate}
\item A infância de Machado de Assis: nascimento no Morro do Livramento;
classe social da família de Machado de Assis; o contexto econômico,
social e político do Rio de Janeiro na virada da década de 1830 para a
de 1840;

\Image{Morro do Livramento, Rio de Janeiro, onde Machado de Assis nasceu (a seta aponta no alto do canto direito a possível casa de nascimento), sem data. (ABL; Domínio público)}{PNLD0048-04.png}

\item O jovem Machado de Assis: as primeiras amizades literárias do autor e
a ``Sociedade Petalógica''; a importância da personalidade de Paula
Brito para a formação de Machado de Assis; a importância do jornalismo
na vida de Machado de Assis;

\Image{Gravura de Francisco de Paula Brito  (Autor Desconhecido, Século XIX. Pinto, Ana Flávia Magalhães. De Pele Escura e Tinta Preta: a imprensa negra do século XIX (1833-1899). Mestrado. Universidade de Brasília, 2006; Domínio público)}{PNLD0048-05.png}

\item Primeiros romances de Machado de Assis: importância e recepção dos
romances de Machado de Assis (\emph{Ressurreição}, 1872; \emph{A Mão e a
Luva}, 1874; \emph{Helena}, 1876; \emph{Iaiá Garcia}, 1878); por que
esses romances não são considerados os \emph{grandes romances} do autor?

\item Maturidade da obra de Machado de Assis: importância e recepção dos
chamados romances da maturidade (\emph{Memórias Póstumas de Brás Cubas},
1881; \emph{Quincas Borba}, 1892; \emph{Dom Casmurro}, 1900; \emph{Esaú
e Jacó}, 1904; \emph{Memorial de Aires}, 1908); por que esses romances
foram considerados \emph{grandes}?

\item A importância da Academia Brasileira de Letras e as críticas que se
podem fazer a ela: o que é a Academia Brasileira de Letras? Qual a
importância dessa academia na vida de Machado de Assis? O que ela
significa na vida cultural brasileira? Quais são as críticas que podem
ser feitas à Academia Brasileira de Letras? Quais são os episódios
recentes que revelam a importância e os limites dessa instituição?
Lembrar, aqui, a polêmica referente à candidatura da escritora Conceição
Evaristo à cadeira de número 7 da ABL, em 2018;

\item A polêmica a respeito da cor da pele de Machado de Assis: o que é o
chamado ``branqueamento'' de Machado de Assis? Quais são os exemplos
concretos desse processo? Qual é o contexto em que esse processo ocorre?
Por que esse processo ocorre? Existem outras personalidades públicas do
Brasil que também foram branqueadas? Em que medida o chamado racismo
estrutural brasileiro contribuiu para esse processo?

\item Machado de Assis fora do Brasil: recentemente, Machado de Assis foi
publicado com sucesso nos Estados Unidos. Quais são os textos de Machado
de Assis publicados no exterior? Quais são os estudiosos estrangeiros
que analisaram a obra de Machado de Assis fora do Brasil? Por que a obra
de Machado de Assis teve sucesso fora do Brasil?
\end{enumerate}

\item Cada um dos grupos deve pesquisar os temas apresentados em arquivos
da Internet. Cabe ao professor orientar esse processo de pesquisa. Quais
são as fontes mais confiáveis? Quais são as menos confiáveis? Por quê?

\item Depois de completo o processo de pesquisa, os estudantes podem
apresentar os resultados obtidos de uma das seguintes maneiras:

\begin{enumerate}
\item \emph{podcast} de até 15 minutos, com entrega de roteiro na forma de texto;

\item vídeo publicado no YouTube, no formato de documentário, em até 15
minutos, com entrega de roteiro na forma de texto;

\item peça teatral de até 15 minutos, com entrega de roteiro na forma de
texto;

\item leitura em voz alta de folheto de cordel, de até 15 minutos, com
entrega desse folheto ao professor;

\item apresentação formal tradicional, de até 15 minutos, com entrega de
relatório de pesquisa;

\item declamação de poema no formato de \emph{slam} poesia, de até 15 minutos, com
entrega do poema escrito;

\item canção de até 15 minutos, que pode ser de composição original dos
estudantes, ou paródia de canção já existente, com entrega da letra e da
cifra.
\end{enumerate}

\item É necessário destacar que cada uma das formas de apresentação dos
resultados requer dos estudantes a pesquisa sobre o gênero específico
dos textos;

\item Em qualquer um dos formatos de relatório, é fundamental que, além de
apresentar fatos objetivos a respeito da vida e da obra de Machado de
Assis, os estudantes argumentem em defesa de um ponto de vista;

\item Depois das apresentações, os alunos deverão apresentar aos colegas as
dificuldades enfrentadas nos processos de pesquisa e preparação dos
relatórios específicos. Deverão também apresentar o percurso por meio do
qual chegaram às conclusões e pontos de vista propostos nos relatórios a
respeito das polêmicas da vida e da obra de Machado de Assis.
\end{enumerate}

\subsection{Leitura I}

%\BNCC{EM13LGG102}
%\BNCC{EM13LGG104}
%\BNCC{EM13LGG202}
%\BNCC{EM13LP01}
%\BNCC{EM13LP02}
%\BNCC{EM13LP03}
%\BNCC{EM13LP06}
%\BNCC{EM13LP13}
%\BNCC{EM13LP47}
%\BNCC{EM13LP48}

\paragraph{Tema} Conto e novela na obra de Machado de Assis.

\paragraph{Conteúdo} 

\begin{enumerate}
\item Leitura, compreensão e análise das obras da antologia; 
\item Características do conto; 
\item Características da novela; 
\item Elementos gerais da narrativa: foco narrativo, enredo, personagens,
tempo e espaço.
\end{enumerate}

\paragraph{Objetivos} 

\begin{enumerate}
\item Estimular a leitura das obras da antologia, em grupo ou individualmente; 
\item Promover a reflexão sobre os elementos que compõem a narrativa
(foco narrativo, enredo, personagens, tempo e espaço); 
\item Ler, compreender, analisar e comparar os textos da antologia.
\end{enumerate}

\paragraph{Justificativa} As edições de 2011, 2015 e 2019 da pesquisa
Retratos do Brasil revelam um dado inegável: a influência do professor
na formação dos leitores. Para despertar o gosto pela leitura entre
nossos alunos, precisamos ser \emph{mediadores} entre a obra, sua
linguagem, suas estruturas e o estudante, de preferência estabelecendo
uma relação fundamentada no prazer, na identificação e na liberdade de
interpretação. Eis o nosso desafio: \emph{ler com os alunos},
apresentando-lhes as passagens decisivas de um texto -- porque são
engraçadas, assustadoras, emocionantes etc. --, explicando"-lhes
\emph{por que} elas chamam a atenção, estabelecendo \emph{conexões}
entre elas e nossa vida presente, revelando as intuições do autor quanto
às práticas sociais atuais, seus conflitos, dilemas, conquistas, ouvindo
as impressões dos estudantes a respeito de tudo isso e propondo-lhes a
análise do que eles gostam -- canções, séries de TV ou de plataformas de
\emph{streaming}, novelas, filmes e outros produtos culturais -- a
partir dessa comparação. Dizendo de maneira simples, o processo de
formação de leitor deve estar baseado no repertório de nossos próprios
estudantes, a partir do qual podemos estabelecer os pontos de contato
com as obras que lhes apresentamos.

\paragraph{Metodologia}

\begin{enumerate}
\item A primeira atividade imprescindível é a leitura, compreensão e
análise dos textos da antologia. O ideal, se houver tempo, é que pelo
menos um deles seja lido \emph{integralmente} na sala de aula.
Conscientes das dificuldades dessa atividade no contexto escolar,
sugerimos que esse exercício de leitura seja preparado previamente: os
alunos podem preparar-se para uma \emph{leitura interpretativa}, que não
seja monótona, com a orientação prévia do professor a respeito do enredo
e das personagens do texto escolhido. Sugerimos especialmente a leitura
de ``A Cartomante'', pela conclusão surpreendente e pela menor extensão
do texto;

\item É possível, ainda, ler trechos dos três textos da antologia com os
alunos. Da mesma maneira, sugere-se a \emph{leitura interpretativa}
desses trechos, de modo a envolver os estudantes na atividade;

\item O exercício de compreensão do texto se baseia na análise dialogada de
elementos fundamentais de cada um dos textos da antologia -- que
permitirão, por sua vez, distinguir a novela ``O Alienista'' dos contos
``O Imortal'' e ``A Cartomante''. É fundamental sugerir aqui que a
distinção das duas categorias não seja feita de forma teórica, anterior
à leitura, mas que \emph{seja depreendida a partir da leitura do texto}.
Sabemos que esse é um exercício desafiador, mas nos parece ainda o
melhor caminho para habilitar os estudantes à compreensão e à análise de
textos literários;

\item Os elementos fundamentais dos textos -- foco narrativo, enredo,
personagens, espaço e tempo -- revelarão as especificidades de cada um
deles. Apenas a título de exemplo: ``O Alienista'' é texto que se
destaca pela \emph{extensão}, o que já o situa entre o conto, mais
curto, e o romance, mais longo. Essa constatação simples pode ser o
ponto de partida para a análise de outros elementos, orientada por
perguntas simples, como: ``por que é que `O Alienista' acabou tendo uma
extensão maior?''; a leitura do texto dará a resposta: na teoria
tradicional, o conto se caracteriza pela unidade da ação, em torno de um
só conflito, com restrição do espaço, do tempo e evidentemente das
personagens envolvidas. É nesse sentido que ``O Alienista'' se afasta
dessas características gerais, aproximando-se da novela (embora não
completamente): por mais que os experimentos de Simão Bacamarte deem a
unidade geral ao texto, ele ``reserva espaço para narrativas paralelas,
dentre as quais a viagem (e fausta recepção) da comitiva local, que
viajara para o Rio de Janeiro; a disputa do poder entre os barbeiros
Porfírio Caetano das Neves e João Pina; as mudanças de posição do
vereador Sebastião Freitas'', como afirma Jean Pierra Chauvin na
apresentação disponível na edição da antologia. Comparado a ``O
Alienista'', ``A Cartomante'' corresponde, de forma geral, àquela
proposição geral das feições que o conto deve ter. Caso curioso, também,
é o de ``O Imortal'', no qual o narrador nos apresenta a história de Rui
de Leão, contada por seu filho a duas outras personagens, numa estrutura
de conto dentro do conto, assemelhada, superficialmente, à estrutura das
Mil e Uma Noites, em que ao leitor interessam não apenas a história do
pai do médico homeopata, mas também a expectativa do Coronel Bertioga e
do tabelião João Linhares quanto à narrativa;

\item A comparação entre os três textos da antologia é, portanto,
fundamental para a compreensão, não apenas dos elementos internos de
cada um deles, mas também do gênero de que se aproximam: ``O Alienista''
com o traço marcante da pluralidade dramática da novela ao lado da
unidade que lhe é conferida pelos experimentos e pela personalidade de
Simão Bacamarte, desencadeadores e condutores das outras ações; o conto
dentro do conto em ``O Imortal'', com o traço marcante do conto
fantástico, baseado evidentemente na imortalidade de Rui de Leão, e a
multiplicidade de ações cometidas por ele nos mais de duzentos anos de
vida -- características que também o situam numa posição ambígua, entre
novela e conto; finalmente, a unidade do enredo e a restrição do espaço
e do tempo de ``A Cartomante'', que permitem aproximar esse texto da
definição mais estrita de conto;

\item Depois de lidos, compreendidos e analisados, os textos devem ser
interpretados pelos estudantes com a mediação do professor. Esse
processo é mais adequadamente descrito na atividade de pós-leitura.
\end{enumerate}

\subsection{Leitura II}

%\BNCC{EM13LGG103}
%\BNCC{EM13LGG202}
%\BNCC{EM13LP35}
%\BNCC{EM13LP37}
%\BNCC{EM13LP44}

\paragraph{Temas} Quatro gêneros da linguagem jornalística.

\paragraph{Conteúdo} Criação coletiva de um jornal virtual com três textos
e um vídeo: uma reportagem noticiando a internação de Simão Bacamarte na
Casa Verde; uma entrevista com Rui de Leão, protagonista de ``O
Imortal''; um editorial analítico-argumentativo a respeito do
assassinato de Rita e Camilo, relacionando"-o ao feminicídio e ao
machismo e criticando a abordagem de um vídeo, também
feito pelos alunos, reproduzindo a linguagem sensacionalista dos
programas policiais de televisão.

\paragraph{Objetivos}
\begin{enumerate}
\item Aprofundamento da leitura dos textos da antologia
por meio da adaptação a outras linguagens; 
\item Valorização da organização
coletiva do trabalho, de modo a alcançar um resultado final em que as
diferentes partes estejam coerentemente organizadas; 
\item Produção de
conteúdos críticos a respeito de assuntos polêmicos, antecedida de
debates; 
\item Debates a respeito das diferentes linguagens e dos pontos de
contato entre elas; 
\item Criação de página da internet, de modo a revelar
o conhecimento da linguagem e comunicação digital.
\end{enumerate}

\paragraph{Justificativa} Enquanto os exercícios de leitura, compreensão e
análise caracterizam-se, de forma geral, pelas primeiras aproximações do
texto, seguidas de atividades de descrição de suas características, as
práticas interpretativas abrem espaço para que os estudantes avaliem,
opinem e expressem juízos de valor quanto às leituras. Os professores
devemos estimular essa dinâmica, na exata medida em que ela serve ao
exercício do espírito crítico e abre canais para que possamos fazer
mediações entre a realidade dos textos analisados e a de nossos
estudantes.

Acreditamos que a criação de um jornal virtual permita: 

\begin{enumerate}
\item Que os alunos
aprofundem a leitura dos textos da antologia para adaptá-los a outras
linguagens; 

\item Que o professor avalie uma ampla gama de competências e
habilidades dos estudantes, de acordo com as potencialidades de cada um;


\item Que os alunos produzam coletivamente conteúdos críticos, em formatos
diversos; 

\item Que professor e alunos debatam as diferentes linguagens e
os pontos de contato entre elas; 

\item Que a produção dos alunos ganhe
visibilidade no meio digital, de forma a contar com a avaliação não
apenas no contexto escolar, mas também no ambiente das redes, de acordo
com as interações alcançadas; 

\item Que assuntos polêmicos do cotidiano
sejam debatidos pelos alunos, de modo a estimular o debate, a
argumentação, o espírito crítico e a negociação.
\end{enumerate}

\paragraph{Metodologia}

Para criar um jornal virtual, o professor deverá dividir os estudantes
em grupos cujos objetivos correspondam às habilidades com as quais os
alunos se identifiquem e/ou a respeito das quais eles pretendam
conhecer. Sugerimos a seguinte divisão:

\begin{enumerate}
\item Grupo dos alunos de redação da reportagem a respeito da internação
definitiva de Simão Bacamarte na Casa Verde, relatando o enredo de ``O
Alienista'' na forma jornalística, exercitando a paráfrase. O primeiro
desafio deste grupo é a redação de um texto repleto de marcadores
típicos da impessoalidade característica das reportagens, em oposição
aos textos opinativos, adaptando os eventos da novela de Machado de
Assis para o contexto e a linguagem do século \textsc{xxi}. Além disso, esse
grupo também deve ser capaz de produzir uma paráfrase sucinta, mas
completa, dos eventos relatados no conto. Note"-se, ainda, que o processo
de redação, aqui, é coletivo: deve"-se estimular que os alunos se dividam
em grupos de \emph{redatores} (os que efetivamente escrevem o texto),
\emph{revisores} (os que fazem a revisão gramatical do texto) e
\emph{editores} (responsáveis pela coerência do texto como um todo, já
que ele terá sido escrito a muitas mãos);

\item Grupo dos alunos de redação da entrevista com o narrador do conto ``O
Imortal'', em que ele relata, em primeira pessoa, alguns dos acontecidos
com o pai dele e as lições que ele teria aprendido. O grande desafio
deste grupo é transpor a linguagem do conto de Machado de Assis para a
oralidade típica das entrevistas, adaptando os eventos da novela de
Machado de Assis para o contexto e a linguagem do século \textsc{xxi}. Também
deve estar disponível, ao final da página da transcrição da entrevista,
um arquivo virtual com um trecho dessa entrevista, representado por dois
alunos, um no papel de entrevistado, outro no de entrevistador;

\item Grupo dos alunos de redação do editorial crítico ao assassinato de
Rita, enfatizando as questões polêmicas sugeridas no texto de Machado de
Assis: especialmente o feminicídio e o machismo. Do ponto de vista do
gênero textual, o desafio deste grupo é redigir um texto
analítico"-argumentativo, atualizando o conto de Machado de Assis para o
contexto do Brasil atual, partindo do princípio de que o enredo de ``A
Cartomante'' aconteceu no século \textsc{xxi}, fazendo as adaptações necessárias.
Além disso, a equipe de redatores deve pesquisar os temas do feminicídio
e do machismo, analisar"-lhes as implicações um no outro, verificar"-lhes
a atualidade e \emph{tomar partido};

\item Grupo de alunos que redigirão o roteiro do programa de TV
sensacionalista, em que será apresentada uma reportagem a respeito do
assassinato de Rita e Camilo. O desafio desses alunos é reproduzir a
linguagem violenta desses programas no limite da paródia, de modo a
evidenciar a crítica que se pode fazer a eles;

\item Grupo dos alunos que trabalharão com as imagens, seu tratamento e sua
adequação aos textos. Deve"-se estimular a criação de imagens próprias,
de autoria dos alunos, baseadas na iconografia tradicional a respeito de
Machado de Assis. Quanto mais autorais forem as ilustrações da página,
mais valorizada elas serão. Para a reportagem, sugere"-se ilustração com
representações de Simão Bacamarte e da Casa Verde, além de outras que os
estudantes julgarem ilustrativas; para a entrevista, sugere"-se a
representação do narrador de ``O Imortal'', além de imagens que ilustrem
as passagens históricas citadas no conto; finalmente, para o editorial,
sugere"-se a criação de gráficos que ilustrem os casos de feminicídio no
Brasil;

\item Grupo dos alunos que organizarão a página da Internet, suas fontes,
disposição, legibilidade. A tarefa mais desafiadora deste grupo é o
diálogo com os outros grupos. A identidade visual final da página da
Internet deve corresponder tanto à iconografia criada pelo grupo das
imagens quanto ao conteúdo dos textos criados pelos grupos de redação;

\item Grupo dos alunos que formularão estratégias de divulgação da página
da Internet, bem como apresentarão os resultados dessa divulgação e
serão responsáveis pela interação com os leitores. O desafio deste grupo
é conhecer a fundo o conteúdo formulado pelos grupos de redação, de modo
a produzir respostas coerentes na interação com leitores. A qualidade do
trabalho desse grupo também será avaliada de acordo com o número de
leitores alcançado, de preferência extrapolando as ``bolhas'' virtuais
dos próprios alunos da sala;

\item Grupo de alunos envolvidos na produção do vídeo. O desafio desses
alunos está na criação de uma ambiência convincente, baseada em cenário,
figurino, iluminação e sonoplastia que reproduzam a linguagem violenta
dos programas policiais. Da mesma maneira, é preciso escolher os alunos
que farão o papel de atores, de acordo com o texto preparado pelo grupo
de roteiristas, e os que se responsabilizarão pelas atividades técnicas:
direção, edição e publicação do vídeo.
\end{enumerate}

\subsection{Pós"-leitura}

%\BNCC{EM13LGG101}
%\BNCC{EM13LGG102}
%\BNCC{EM13LGG202}
%\BNCC{EM13LGG303}
%\BNCC{EM13LP01}
%\BNCC{EM13LP13}
%\BNCC{EM13LP14}
%\BNCC{EM13LP15}
%\BNCC{EM13LP47}

\paragraph{Tema} A imortalidade na literatura e no cinema de vampiros.

\Image{Pintura ``O vampiro'' (1897) de Philip Burne-Jones (Philip Burne-Jones; Domínio público)}{PNLD0048-07.png}

\paragraph{Conteúdo}
\begin{enumerate}
\item História da literatura e do cinema de vampiros; 

\item Características da literatura e do cinema de vampiros; 

\item Comparação: o
tema da imortalidade no conto ``O Imortal'', de Machado de Assis, e na
literatura de vampiros.
\end{enumerate}

\paragraph{Objetivo} 
\begin{enumerate}
\item Analisar a produção e a circulação da literatura e
do cinema de vampiros; 

\item Analisar temas, imagens e linguagens típicas
da literatura de vampiros; 

\item Comparar os elementos analisados nos itens
1 e 2 com o conto ``O Imortal'', de Machado de Assis; 

\item Promover
debates por meio dos quais os estudantes possam perceber a atualidade da
literatura machadiana, por meio da \emph{imortalidade}, tema tão caro à
literatura de vampiros, gênero muito difundido entre os jovens; 

\item
Propor aos estudantes a adaptação e/ou atualização do conto ``O
Imortal'' para a temática vampiresca, na forma de conto, crônica,
reportagem de jornal, teatro, dança ou vídeo;
\end{enumerate}

\paragraph{Justificativa} Evidentemente, um dos temas do conto ``O
Imortal'', de Machado de Assis, é a imortalidade, comum à
\emph{literatura de vampiros} ou \emph{vampiresca}, de bastante
penetração no público jovem, seja pela literatura, seja pelo cinema. O
destaque desse tema comum permitirá ao estudante observar a atualidade
da literatura machadiana e seu diálogo com obras literárias
contemporâneas. Não importa aqui que a literatura vampiresca seja
considerada, por alguns professores e intelectuais, de menor
importância, por pertencer à categoria da literatura de entretenimento;
trata"-se de mostrar aos estudantes que há pontos de contato entre obras
canônicas e obras mais voltadas ao mercado. É o caso, inclusive, de
promover o debate a respeito dessa distinção: por que obras de grande
aceitação entre o grande público, muitas vezes, são menosprezadas pela
crítica? Ou ainda, a mesma reflexão, por outro caminho: por que obras
consagradas pelos leitores especializados, muitas vezes não encontram
aquela aceitação nos leitores comuns? E, mais importante: pode Machado de Assis ocupar um
lugar intermediário entre essas duas pontas, isto é: pode ele ser
amplamente aceito pelo grande público e pelo público especializado?
Quais são as características de sua obra que lhe conferem esse duplo
caráter?

\paragraph{Metodologia}
\begin{enumerate}
\item Propor aos alunos que façam um levantamento da história e das
características da literatura e do cinema de vampiros. Quando essa
literatura começou a ganhar mais leitores? Quais são os grandes
clássicos da literatura e do cinema de vampiros? Quais são os temas, as
imagens e a linguagem típicas desse gênero?

\item Depois da pesquisa a respeito da história e das características da
literatura e do cinema de vampiros, propor aos alunos que
\emph{escolham} ao menos uma obra, de acordo com seu gosto pessoal, para
comparar com ``O Imortal'', de Machado de Assis;

\item Apenas a título de sugestão, para dar início às atividades,
apresentamos a seguir algumas das obras que podem ser escolhidas pelos
alunos.

\begin{enumerate}
\item Na literatura:

\begin{itemize}
\item \emph{Drácula}, de Bram Stoker, romance de 1897; 

Image{O vampiro Conde Drácula movendo-se como um lagarto ao longo da parede de seu castelo. Ilustração da capa de Holloway para a décima terceira edição do romance de Bram Stoker (Londres, William Rider & Son, 1919). (Holloway; Domínio público)}{PNLD0048-06.png}

\item \emph{Salem}, de Stephen King, romance de 1975; 

\item \emph{Entrevista com o vampiro}, de Anne Rice, romance de 1976, de especial interesse porque foi traduzido para o português por Clarice Lispector; 

\item \emph{Os Sete}, de André Vianco, romance vampiresco brasileiro, publicado em 2002; 

\item A saga \emph{Crepúsculo}, de Stephenie Meyer, publicada a partir de 2005.

\Image{Crepúsculo é um filme norte-americano de 2008, dirigido por Catherine Hardwicke, sendo o primeiro da saga, adaptado do primeiro livro de Stephenie Meyer (Flickr ; Creative Commons CC-BY-SA 2.0)}{PNLD0048-09.png}

\end{itemize}

\item No cinema: 
\begin{itemize}
\item \emph{Nosferatu} (1922), de F.W. Murnau; 

\Image{Placa comemorativa para o clássico do cinema mudo Nosferatu (1922) em Wismar, na Alemanha. Partes do filme foram filmadas em Wismar. (J.H. Janßen; Domínio público)}{PNLD0048-08.png}

\item \emph{Drácula de Bram Stoker} (1992), de Francis Ford Coppola; 

\item \emph{Entrevista com o vampiro} (1994), de Neil Jordan; 

\item A saga \emph{Crepúsculo}, composta de cinco filmes, filmados entre 2008 e 2012.
\end{itemize}
\end{enumerate}
\end{enumerate}



\begin{enumerate}\setcounter{enumi}{3}
\item A reflexão e comparação deve ser orientada pela seguinte pergunta:
``Nas obras analisadas, quais são as reflexões feitas a respeito da
imortalidade? Ela é uma \emph{dádiva}? Ou é uma \emph{maldição}? Por
quê?''

\item Depois do debate, propor aos alunos a criação de uma adaptação e/ou
atualização do conto ``O Imortal'' para a temática vampiresca, nas
seguintes formas

\begin{enumerate}
\item conto: o estudante deve redigir um conto no qual Rui de Leão fique
sabendo da existência de vampiros brasileiros no Rio de Janeiro do
Segundo Império; a seguir, ele os procurará, para partilhar experiências
com outros imortais. Ao encontrá"-los, descobrirá que esses vampiros
pertencem aos setores privilegiados da sociedade brasileira, e é
possível caracterizá-los como vampiros também porque eles ``vampirizam''
brasileiros sem propriedades e, especialmente, os escravos. A proposta,
aqui, é trabalhar com o duplo sentido do verbo \emph{vampirizar},
caprichando na contextualização do Brasil escravista e imperial. Esse
tema está sugerido em \emph{Entrevista com o vampiro}, de Anne Rice;

\item crônica: cansado da vida que levou, dos amores e entes queridos que
perdeu e especialmente esgotado pelo desencanto com a humanidade, o Rui
de Leão escreverá uma extensa crônica de jornal, descrevendo, em um
relato pessoal e íntimo, o que é viver por mais de duzentos anos. O
objetivo é flagrar a condição da imortalidade nos pequenos detalhes do
cotidiano;

\item entrevista para a imprensa escrita: baseando"-se no best"-seller de
Anne Rice, depois adaptado para o cinema, \emph{Entrevista com o
Vampiro}, o estudante deverá escrever uma entrevista feita com Rui de
Leão a respeito de sua vida, dos eventos históricos de que participou,
dos amores que viveu, dos lugares que conheceu;

\item entrevista para \textit{podcast}: imaginando que o pai de Dr. Leão não tomou o
elixir ao final do conto, o estudante, ou grupo de estudantes, deverá
preparar um podcast com uma entrevista com Rui de Leão, a respeito de
sua vida, dos eventos históricos de que participou, dos amores que
viveu, dos lugares que conheceu.
\end{enumerate}

\item Outra possibilidade de atividade é a produção, de forma colaborativa,
de um blog no qual os alunos, na forma de textos escritos, analisem a
diferença de classificação de obras como os contos de Machado de Assis,
de um lado, e romances e contos vampirescos, de outro. A finalidade
dessa proposta é promover a pesquisa dos estudantes a respeito dos
critérios utilizados pela crítica para avaliar as obras literárias e o
exercício do espírito crítico quanto a esses critérios.
\end{enumerate}


\section{Atividades 2}

\subsection{Pré-leitura}

%\BNCC{EM13LGG101}
%\BNCC{EM13LGG102}
%\BNCC{EM13LP03}
%\BNCC{EM13LP04}
%\BNCC{EM13CHS101}
%\BNCC{EM13CHS102}

\paragraph{Tema} Relações entre história e literatura\

\paragraph{Conteúdo} Comparação entre textos e discursos da história e da
literatura\

\paragraph{Objetivo}
\begin{enumerate} 
\item Analisar visões de mundo, conflitos de interesse e
ideologias presentes nos discursos históricos e ficcionais; 

\item Analisar
semelhanças e diferenças entre esses discursos; 

\item Analisar diferentes
graus de parcialidade e imparcialidade nesses discursos, levando em
consideração os recursos de linguagem utilizados para obter os efeitos
pretendidos.
\end{enumerate}

\paragraph{Justificativa} Os limites entre o discurso histórico e o
discurso ficcional têm motivado, de um lado, pesquisas que consideram
que o texto literário é documento histórico precioso e, de outro, obras
literárias que, para além dos romances históricos, questionam a validade
das narrativas históricas oficiais, recontando, de maneira crítica,
irônica ou paródica, eventos do passado e suspeitando de suas versões
consagradas. Entendemos que, em ``O Alienista'' e ``O Imortal'', Machado
de Assis antevia, de certa forma, os limites difusos entre história e
ficção, de modo que propomos aqui a reflexão sobre eles.

\paragraph{Metodologia}
\begin{enumerate}
\item Em diálogo com os alunos, o professor deve apresentar as
especificidades do discurso histórico, seus alcances e limites, sem
preocupar"-se com exposição extensiva de textos teóricos. Um exemplo
bastante interessante, para dar início ao debate, é a distinção
estabelecida por Aristóteles, na \emph{Poética}, proposta a seguir de
forma geral: a famosa afirmação de que o historiador conta \emph{o que
aconteceu}, enquanto o poeta conta \emph{o que poderia ter acontecido};

\Image{``Poética'' de Aristóteles é um conjunto de anotações de suas aulas sobre o tema da poesia e da arte em sua época. (Biblioteca da Universidade de Sidney ; Domínio público)}{PNLD0048-10.png}

\item Depois desse debate inicial, o professor pode dar exemplos da
parcialidade dos discursos históricos a partir de pontos de vista
específicos. É importante fazê"-lo a partir de exemplos concretos que se
inscrevam na linguagem, como as diferentes designações de um mesmo
evento: na perspectiva dos povos indígenas, a \emph{invasão} de suas
terras em 1500; na perspectiva dos portugueses, o \emph{descobrimento}
do Brasil. Da mesma maneira, em 1964, o \emph{Golpe Civil"-Militar}, do
ponto de vista do campo democrático, e a \emph{Revolução de 1964}, dos
militares. Trata"-se, fundamentalmente, de preparar os alunos para o
discurso que encontrarão na leitura de dois dos três textos da
coletânea: no enredo de ``O Alienista'', as referências recorrentes aos
cronistas de Itaguaí; no de ``O Imortal'', as alusões a diversos eventos
históricos do Brasil e do mundo, na perspectiva de Rui de Leão;

\item Finalmente, estimular entre os estudantes o debate contemporâneo a
respeito da disputa de narrativas e da pós"-verdade. De maneira geral,
propor a eles uma reflexão adensada sobre eventos recentes interpretados
de maneiras distintas de acordo com pressupostos anteriores;

\item Selecionar eventos contemporâneos cuja interpretação é polêmica e
promover debates qualificados, entre grupos, por meio dos quais seja
possível reconhecer os pressupostos que amparam o ponto de vista
escolhido pelos alunos. Menos do que reconhecer a prevalência de uma
argumentação sobre outra, o objetivo dessa atividade é explicitar
pressupostos dos discursos defendidos pelos estudantes.
\end{enumerate}

\subsection{Leitura I}

%\BNCC{EM13LGG101}
%\BNCC{EM13LGG102}
%\BNCC{EM13LP03}
%\BNCC{EM13LP04}
%\BNCC{EM13CHS101}
%\BNCC{EM13CHS102}

\paragraph{Tema} Relações de intertextualidade e interdiscursividade entre
ficção e história.

\paragraph{Conteúdo} Alusões à Revolução Francesa em ``O Alienista'', de
Machado de Assis.

\paragraph{Objetivo} Analisar relações de intertextualidade e
interdiscursividade entre ``O Alienista'' e a historiografia da
Revolução Francesa.

\paragraph{Justificativa} A leitura, compreensão e análise da novela ``O
Alienista'' certamente ficará incompleta se não se considerarem as
relações intertextuais e interdiscursivas entre esse texto e a
historiografia da Revolução Francesa. Trata"-se de uma oportunidade
bastante significativa de estabelecer pontos de contato entre literatura
e história, refletir sobre limites e alcances de cada um desses
discursos e estender o debate para a ``disputa de narrativas''
contemporânea, em que são perceptíveis as possibilidades de produção e
reprodução de visões de mundo, conflitos de interesse e ideologias nos
discursos veiculados em diferentes mídias.

\paragraph{Metodologia}
\begin{enumerate}
\item Levantamento, em ``O Alienista'', dos elementos que aludem, de forma
mais ou menos explícita, à Revolução Francesa. Apenas a título de
exemplo, alguns desses elementos mais evidentes são: a rima entre os
nomes Simão Bacamarte e Napoleão Bonaparte; o período histórico em que
se passam os eventos de ``O Alienista'' (os últimos anos do século
\textsc{xviii}); o título do Capítulo \textsc{v}, ``O Terror'';

\item Depois desse levantamento, estimular o debate, entre os alunos, a
respeito das semelhanças e diferenças entre a Revolução Francesa e a
Revolta dos Canjicas (atenção ao artigo ``Machado de Assis e a [sua]
Revolução Francesa'', de André Dutra Boucinhas, indicado na bibliografia
comentada);

\item Debate a respeito de possibilidades analítico"-interpretativas de ``O
Alienista'', orientadas por uma pergunta simples e fértil: ``por que
Machado de Assis criou uma novela com tantas referências à Revolução
Francesa?''.
\end{enumerate}

\subsection{Leitura II}

%\BNCC{EM13LGG101}
%\BNCC{EM13LGG102}
%\BNCC{EM13LP03}
%\BNCC{EM13LP0}
%\BNCC{EM13CHS101}
%\BNCC{EM13CHS102}

\paragraph{Tema} Relações de intertextualidade e interdiscursividade entre
ficção e história.

\paragraph{Conteúdo} Alusões a diversos eventos históricos em ``O
Imortal'', de Machado de Assis.

\paragraph{Objetivo} Analisar relações de intertextualidade e
interdiscursividade entre ``O Imortal'' e a historiografia brasileira e
geral.

\paragraph{Justificativa} A leitura, compreensão e análise do conto ``O
Imortal'' pressupõe a análise das relações intertextuais e
interdiscursivas entre esse texto e a historiografia do Brasil e Geral.
Trata-se de uma oportunidade bastante significativa de estabelecer
pontos de contato entre literatura e história, refletir sobre limites e
alcances de cada um desses discursos e estender o debate para a
``disputa de narrativas'' contemporânea, em que são perceptíveis as
possibilidades de produção e reprodução de visões de mundo, conflitos de
interesse e ideologias nos discursos veiculados em diferentes mídias.

\paragraph{Metodologia}
\begin{enumerate}
\item Levantamento, em ``O Imortal'', dos elementos que aludem, de forma
mais ou menos explícita, a eventos históricos: a Invasão Holandesa; as
relações entre os colonos portugueses e os povos indígenas no século
\textsc{xvii}; as perseguições religiosas na Europa do mesmo século; a filosofia
de Spinoza; a Rebelião de Montmouth e o contexto mais amplo em que ela
se insere na História da Inglaterra, no fim dos seiscentos; a acentuação
do tráfico de escravos no século \textsc{xviii}; o Quilombo dos Palmares; as
perseguições da Inquisição; a Revolução Francesa, em 1789, e seus
desdobramentos; os momentos marcantes da História do Brasil da primeira
metade do século \textsc{xix}: a viagem da corte portuguesa ao Rio de Janeiro, em
1808; a Independência, em 1822; a Maioridade;

\item Depois desse levantamento, tentar analisar o ponto comum a esses
eventos históricos dos séculos \textsc{xvii, xviii e xix}: todos eles convergem,
de forma geral, para as profundas alterações da Modernidade do século
\textsc{xix};

\item Debate a respeito de possibilidades analítico"-interpretativas de ``O
Imortal''.
\end{enumerate}

\subsection{Pós-leitura}

%\BNCC{EM13LGG101}
%\BNCC{EM13LGG102}
%\BNCC{EM13LGG103}
%\BNCC{EM13LGG202}
%\BNCC{EM13LGG301}
%\BNCC{EM13LGG703}
%\BNCC{EM13LP01}
%\BNCC{EM13LP03}
%\BNCC{EM13LP04}
%\BNCC{EM13LP09}
%\BNCC{EM13LP12}
%\BNCC{EM13LP13}
%\BNCC{EM13LP15}
%\BNCC{EM13LP29}
%\BNCC{EM13LP31}
%\BNCC{EM13LP33}
%\BNCC{EM13LP44}
%\BNCC{EM13CHS101}
%\BNCC{EM13CHS102}

\paragraph{Tema} O discurso médico"-científico.

\paragraph{Conteúdo} Análise da hegemonia do discurso médico"-científico no
presente.

\paragraph{Objetivo}
\begin{enumerate} 
\item Reconhecer em ``O Alienista'' a emergência do
discurso médico"-científico; 

\item Pesquisar notícias, textos históricos e
pesquisas acadêmicas, e partilhar experiências pessoais, que revelem a
predominância desse discurso na vida presente; 

\item Analisar o processo
histórico de consolidação desse discurso.
\end{enumerate}

\paragraph{Justificativa} É indiscutível a hegemonia do discurso
médico"-científico na vida presente, especialmente no contexto escolar,
em que se verifica a tendência à \emph{medicalização} de muitos
comportamentos, considerados doenças, transtornos ou problemas médicos.

\paragraph{Metodologia}
\begin{enumerate}
\item Propor aos estudantes a pesquisa a respeito das diferenças de
classificação da loucura, de acordo com o tempo e o espaço em que
ocorrem. A obra fundamental, aqui, é a \emph{História da Loucura}, de
Michel Foucault. As perguntas que devem guiar a proposta do professor
são as seguintes, de maneira geral: ``Os loucos de ontem são os mesmos
do passado?'' e ``O conceito de loucura varia no tempo e no espaço?'';

\item Propor aos estudantes rodas de conversa em que eles apresentem
experiências pessoais com a hipótese de que \emph{sanidade} e
\emph{insanidade} são conceitos que podem variar no tempo e no espaço;

\item Propor aos estudantes que pesquisem a importância do discurso
médico"-científico na construção daqueles conceitos e da sua
argumentação. Uma obra de apoio bastante significativa é
\emph{Holocausto Brasileiro}, de Daniela Arbex, a respeito do Centro
Hospitalar Psiquiátrico de Barbacena;

\item Promover entre os alunos debates que permitam observar a correlação
entre a hegemonia do discurso médico"-científico e a formação da
sociedade industrial. Nesse sentido, retornar a ``O Alienista'' e
avaliar quais os traços desse discurso hegemônico que já se manifestam
na obra de Machado de Assis;

\item Propor aos estudantes a criação colaborativa de um blog informativo a
respeito de saúde mental. O blog deve conter textos, vídeos e podcasts;

\item Sugere"-se a divisão dos alunos nos seguintes grupos:

\begin{enumerate}
\item Grupo dos alunos de \emph{redação, revisão e edição} dos textos e dos
roteiros dos vídeos e \textit{podcasts}. O primeiro desafio deste grupo é a
redação de textos repletos de marcadores típicos da impessoalidade
característica dos textos informativos. Além disso, esse grupo também
deve ser capaz de produzir paráfrases sucintas dos textos que
pesquisaram nos itens 1 e 3. Note"-se, ainda, que o processo de redação,
aqui, é coletivo: deve"-se estimular que os alunos se dividam em grupos
de \emph{redatores} (os que efetivamente escrevem o texto),
\emph{revisores} (os que fazem a revisão gramatical do texto) e
\emph{editores} (responsáveis pela coerência do texto como um todo, já
que ele terá sido escrito a muitas mãos);

\item Grupo dos alunos que trabalharão com as imagens ilustrativas, seu
tratamento e sua adequação aos textos produzidos pelo grupo anterior.
Deve"-se estimular a criação de imagens próprias, de autoria dos alunos.
Quanto mais autorais forem as ilustrações do blog, mais valorizada elas
serão;

\item Grupo dos alunos que organizarão o blog na internet, suas fontes,
disposição, legibilidade. A tarefa mais desafiadora deste grupo é o
diálogo com os outros grupos. A identidade visual final do blog deve
corresponder tanto à iconografia criada pelo grupo das imagens quanto ao
conteúdo dos textos criados pelos grupos de redação;

\item Grupo dos alunos que formularão estratégias de divulgação do blog,
bem como apresentarão os resultados dessa divulgação e serão
responsáveis pela interação com os leitores. O desafio deste grupo é
conhecer a fundo o conteúdo formulado pelo grupo de redação, de modo a
produzir respostas coerentes na interação com leitores. A qualidade do
trabalho desse grupo também será avaliada de acordo com o número de
leitores alcançado, de preferência extrapolando as ``bolhas'' virtuais
dos próprios alunos da sala.
\end{enumerate}
\end{enumerate}

\section{Aprofundamento}

\subsection{``O Alienista'': entre a insanidade e a lucidez, entre o fato e a ficção}

``O Alienista'' -- texto de difícil classificação, entre o conto e a
novela, como vamos observar a seguir -- é das obras mais conhecidas de
Machado de Assis. Numa Itaguaí do século \textsc{xviii}, a construção de uma Casa
de Orates -- um hospício -- causa convulsões sociais que são comparáveis
às da Revolução Francesa, tudo isso contado por um narrador ambivalente.
Desnecessário frisar a atualidade do texto, especialmente se
considerarmos a luta antimanicomial, como se Machado já pudesse prever,
em pleno 1882, ano da publicação de ``O Alienista'' no livro
\emph{Papéis Avulsos}, que as pesquisas e terapias psiquiátricas podem
chegar ao extremo da \emph{medicalização} dos comportamentos, servindo a
interesses de caráter pessoal, muito distantes da suposta objetividade
neutra da ciência.

\Image{Quadro ``A liberdade guiando o povo'' de Eugène Delacroix, de 1830. (Eugène Delacroix; Domínio público)}{PNLD0048-11.png}

Tudo é ambivalente em ``O Alienista'', a começar do gênero a que
pertence esse texto. Jean Pierre Chuavin, ao analisá"-lo na introdução
desta antologia, afirma que durante muito tempo ele foi considerado um
conto, mas que essa classificação está longe de ser pacífica: embora a
pesquisa de Simão Bacamarte na Casa Verde seja o fio condutor do texto,
articulando"-o como todo, há diversas narrativas paralelas que ganham
alguma autonomia no conjunto. Eis aí um dos desafios de trabalhar com
Machado de Assis em sala de aula: a impossibilidade, muitas vezes, de
apresentar uma classificação fechada a respeito de seus textos. Só para
citar mais uma dessas dificuldades, lembremo"-nos do famoso prólogo da
quarta edição das \emph{Memórias Póstumas de Brás Cubas}, em que o autor
transcreve uma pergunta de Capistrano de Abreu: esse livro é um romance?
No caso de ``O Alienista'', se ele tivesse prólogo, Machado poderia
perguntar: ``O Alienista'' é conto ou novela?

Mas poderíamos questionar ainda mais: quais são os efeitos obtidos por
meio dessa ambivalência? Talvez o primeiro deles seja a sugestão de que
o texto, da mesma forma que os eventos nele relatados, é
inclassificável, em boa medida. O narrador, por exemplo, nos conta
eventos ocorridos em Itaguaí no fim do século \textsc{xviii}, mas não nos
apresenta datas específicas, apenas as sugere, por meio de referências
esparsas. Não deixa de ser curioso que o texto comece aludindo às
``crônicas de Itaguaí'', na mesma linha, refira"-se a ``tempos remotos''
e se conclua com alusões a boatos; ora, se a referência são as crônicas,
nada mais natural que elas contenham informações mais ou menos claras
sobre a localização dos fatos no tempo; mas aqui estamos em pleno oceano
machadiano, em que homens, coisas e fatos são elas e seu inverso,
\emph{são e não são simultaneamente} -- e por isso o narrador fala em
tempos distantes, sem especificá"-los. Estamos entre fatos e ficção,
entre história e literatura -- e é nesse espaço intervalar e indefinido
que Machado de Assis transita. Habituar"-se a lê"-lo é, de certa maneira,
aprender a lidar com esse interstício e a habitá"-lo, como leitor. Essa
é, aliás, uma das linhas de força de Machado.

Coloquemo"-nos no lugar de nossos estudantes: o que é que se ganha, eles
nos perguntariam, com tanta indefinição? Para começo de conversa, um
ponto de vista privilegiado para contar a história. Aparentemente ao
lado de Simão Bacamarte, o narrador não se restringe à perspectiva dele,
e transita por muitas outras, de forma a dar ao leitor uma visão mais
abrangente do que aparenta à primeira vista. Além disso, as incertezas
obrigam a novas leituras.

Nosso debate em sala de aula pode concentrar"-se, por exemplo, na
personagem marcante de Simão Bacamarte, espécie de encarnação brasileira
do médico à moda do século \textsc{xix}, mais cientista do que clínico, que usa
as experiências do cotidiano terapêutico para avançar as hipóteses que
vai compondo ao sabor dos resultados. Há nessa personagem uma certa
desumanização, como se ele fosse mais cientista do que gente,
especialmente quando escolhe a esposa pelas ``condições fisiológicas e
anatômicas de primeira ordem'' que ela apresenta, sem apaixonar"-se por
ela. Essa desumanização é também flagrante nos desdobramentos da teoria
da razão, especialmente quando Bacamarte aprisiona os habitantes
supostamente normais de Itaguaí. O leitor já entendeu, nessa passagem do
texto, que o conceito de normalidade se expandiu e se refere, agora,
também a comportamentos de ordem moral ou ética. O projeto final de
Simão Bacamarte é, trocando em miúdos, arruinar qualquer manifestação de
altruísmo, como se o tratamento proposto apresentasse a canalhice e o
egoísmo como remédios para o respeito aos outros. Somente uma pessoa
\emph{desumanizada} como Bacamarte poderia propor essa terapêutica às
avessas.

Outra análise claramente fértil de ``O Alienista'' é a do tempo. Já
citamos acima as diversas alusões à Revolução Francesa que se fazem
presentes no texto de maneira bastante explícita. A questão fundamental
aqui -- e também no conto ``O Imortal'' -- é tentar entender por que
Machado de Assis dá tanta importância à virada do século \textsc{xviii}, tanto no
Brasil quanto fora dele. É evidente que esse momento histórico contém
uma das mais marcantes viradas sociais, econômicas, políticas e
culturais da História humana: a derrocada final das hierarquias e
dependências características dos séculos \textsc{xvi, xvii e xviii}, em que se
formava o capitalismo, que dão lugar às liberdades de inspiração
burguesa, emergentes precisamente naquela virada. Para nós, é importante
perceber que Machado de Assis reconhecia semelhanças entre esse processo
no Brasil e fora dele, como se em Itaguaí, em escala distinta, se
reproduzissem os acontecimentos da Revolução na França, com diferenças
marcantes, que é preciso registrar. Para reconhecer essas diferenças e
semelhanças, leia o artigo de artigo ``Machado de Assis e a (sua)
Revolução Francesa'', de André Dutra Boucinhas, indicado na bibliografia
comentada.

Note"-se, ainda, que, embora os eventos narrados em ``O Alienista'' se
passem no fim do século \textsc{xviii}, a mentalidade científica de Simão
Bacamarte lembra muito o processo de \emph{medicalização} dos
comportamentos, que foi iniciado no século \textsc{xix} e que dura até hoje: a
obsessão pelo ``exame da patologia cerebral'' tem o sabor quase sádico
das pesquisas médicas que se desenvolveram nos oitocentos, cometeram
horrores no século \textsc{xx} contra pessoas e animais e seguem até hoje. O
desejo da classificação e sua consequente discriminação entre os homens,
por meio de pesquisas que ferem limites éticos, é um dos temas marcantes
desse texto, com potencial de exploração bastante grande em aula.

O debate a respeito dos limites entre razão e loucura interessa bastante
a Machado de Assis -- veja a importância desse assunto, por exemplo, nas
\emph{Memórias Póstumas de Brás Cubas} e \emph{Quincas Borba}, no
filósofo que dá título ao segundo romance. Simão Bacamarte e Quincas
Borba são parecidos em alguns aspectos: são personagens cuja mentalidade
e comportamento são guiados por estudos de prestígio -- a medicina, no
primeiro caso; a filosofia, no segundo -- numa indicação clara de que a
especialização crescente dos estudos da ciência, apesar de amparada em
critérios rigorosos, leva à perda da compreensão do real. Ora, há poucos
assuntos tão atuais quanto esse: que dizer dos médicos especialistas que
nos mandam de um consultório a outro, sob a desculpa de que nosso
problema não faz parte da especialidade deles? Qual é o limite entre
causas genéticas e fisiológicas, de um lado, e ambientais e sociais, de
outro, dos transtornos de ordem mental? Afinal: quais de nós podemos
dizer que estamos em plenas condições de apreensão do real, se ele
próprio se apresenta adoecido e insano, muitas vezes?

Costumamos atribuir essas perguntas às circunstâncias de nosso tempo --
as tecnologias digitais, a Internet, as redes sociais e suas
consequências sociais, econômicas, políticas e culturais -- mas Machado
de Assis já fazia perguntas similares às nossas no século \textsc{xix}. ``O
Alienista'' pode ser lido como uma pesquisa de Machado de Assis a
respeito de um evento histórico marcante para a constituição do mundo
contemporâneo -- a Revolução Francesa -- à procura das bases sobre as
quais foi erigida a sociedade moderna em que vivemos, especialmente o
discurso médico"-científico.

\subsection{A fantástica história de ``O Imortal''}

``O Imortal'' é um conto especialmente impressionante, em primeiro lugar
por aproximar"-se do gênero do conto fantástico. Nele, em 1855, Dr. Leão
relata a dois outros homens a vida do pai, nascido em 1600. Esse
despropósito -- a distância de 255 anos entre o nascimento do pai e a
vida adulta do filho -- abre o conto, cujo ponto de partida é uma lenda
indígena da imortalidade, repleta de referências históricas: Rui de Leão
teria participado da Invasão Holandesa e da brutal aniquilação do
Quilombo dos Palmares, no Brasil, além de correr mundo e contribuir para
eventos como a Rebelião de Monmouth, na Inglaterra, e daí de volta ao
Brasil, no tráfico de escravos, na chegada da corte portuguesa em 1808 e
na Independência, em 1822. Da mesma maneira que as \emph{Memórias
Póstumas de Brás Cubas} surpreendem pela tagarelice de um morto que
conta a própria vida, ``O Imortal'' chama a atenção pela fantástica
imortalidade do protagonista; e, em ambos os textos, quase sem que o
leitor perceba, o misterioso e o insólito desviam"-lhe a atenção,
impedindo"-o de observar o desfile de eventos históricos que lhe passam
pelos olhos. Nesse sentido, a leitura de Machado de Assis é formativa --
para professores e alunos -- porque requer de nós que suspeitemos sempre
de quem nos conta a história.

Contos fantásticos são sempre fascinantes para as gerações mais jovens.
Aqui, para preparar"-se para o debate a respeito deles, o professor pode
ler a \emph{Introdução à Literatura Fantástica}, de Tzvetan Todorov, e
partir do pressuposto de que o conto desse gênero se caracteriza pela
incerteza: num mundo como o nosso, sem criaturas maravilhosas, dominado
pelo \emph{discurso científico que se propõe a explicar tudo}, ocorre um
evento inacreditável; ou a ciência ainda não descobriu as explicações
desse evento, ou ele é falso, ilusório, imaginário. É precisamente nesse
interstício que se produz o fantástico, que é, para Todorov, ``a
hesitação experimentada por um ser que só conhece as leis naturais, face
a um acontecimento aparentemente sobrenatural''\footnote{TODOROV,
  Tzvetan. \textbf{Introdução à Literatura Fantástica}. São Paulo:
  Perspectiva, 2012. p.31.}.

De fato, uma vida humana de mais de duzentos anos é posta em xeque desde
as primeiras páginas do conto: os ouvintes de Dr. Leão inicialmente não
acreditam que isso seja possível, mas se deixam envolver pela capacidade
narrativa desse médico homeopata, pelo fascínio da imortalidade e pela
multiplicidade das experiências vividas por Rui de Leão no tempo e no
espaço. Eis aí um dos atrativos de ``O Imortal'': a possibilidade de
participar de eventos históricos muito distintos, em espaços remotos.
Numa perspectiva histórica -- e uma das atividades que propusemos
aprofunda essa interpretação -- os anos que correm de 1600 até à morte
do pai do narrador, em 1855, correspondem ao período de formação do
capitalismo: primeiramente, assistimos aos anos do século \textsc{xviii}, no qual
o mundo observou a circulação do primeiro produto de massas da história
-- o açúcar -- e no qual os ingleses deram os primeiros passos no
sentido da modernização de suas estruturas políticas, que culminariam na
Revolução Industrial do século \textsc{xix}. O pai do narrador participou, de
certa forma, desses dois eventos, por estar no Brasil quando ocorriam as
Invasões Holandesas ao Brasil e por contribuir em alguma medida nos
eventos que levaram à Revolução Gloriosa, na Inglaterra. Da mesma
maneira, ele também traficou pessoas escravizadas no século \textsc{xviii},
marcando presença no comércio que é o pressuposto de todas as atividades
econômicas que, no Brasil, estão ligadas à Independência e, no exterior,
à formação da sociedade industrial. Rui de Leão pode ser entendido,
assim, como um privilegiado que teria assistido a dois dos séculos
supostamente mais férteis da história humana: aqueles que abriram as
portas à Revolução Industrial.

Mas essa euforia não corresponde à avaliação final: cansado das
decepções pessoais, prevendo infinitas perdas de entes queridos, que
inevitavelmente acabaria por enterrar, Rui de Leão está desesperançado.
Considerando tudo que viu, poderíamos concluir, numa primeira análise,
que essa personagem encarna o que a tradição crítica chamou de
\emph{negatividade} em Machado de Assis: a humanidade pode avançar em
termos científicos e tecnológicos, mas é a mesma em termos das
barbaridades que inflige a si mesma; diante dos afetos, o desfile dos
grandes eventos históricos se apequena. O pai de Dr. Leão prefere a
morte à vida perpétua, porque esta se tornara mera repetição
interminável.

Não deixemos de notar que o ano da morte dele é 1855: apenas cinco anos
depois da extinção do tráfico de escravos no Brasil, que finalmente
cedera às pressões inglesas. Também podemos ver aí mais uma virada
histórica: depois de duzentos e cinquenta e cinco anos de tráfico de
africanos, o Brasil se preparava para abolir a ignomínia -- mas a
experiência de nosso protagonista não lhe permitia ter a esperança de
dias melhores, como se nos ensinasse que a humanidade sabe bem repetir e
aprofundar as brutalidades que comete contra si mesma e contra a
natureza.

\subsection{Adivinhação e realidade em ``A Cartomante''}

``A Cartomante'' é, finalmente, um dos mais conhecidos contos de Machado
de Assis -- e, mais uma vez, se nos deixarmos levar pela superfície do
enredo, talvez não percebamos a riqueza de detalhes que costuram a
tragédia de um crime ambientado no Rio de Janeiro do século \textsc{xix}. Podemos
arriscar dizer que as matrizes patriarcais da sociedade brasileira estão
claramente manifestas na passagem final do conto, em que um marido
assassina a própria esposa que o traía com o melhor amigo, num retrato
explícito da violência que recai sobre as mulheres em um país como o
nosso. Mas não se trata apenas disso: também é preciso desvendar, no
conjunto do conto, a importância da alusão imprecisa de uma das
personagens ao \emph{Hamlet}, de Shakespeare, e o debate por trás da
cartomante que dá título ao conto. Nas primeiras páginas dessa história
(que poderia perfeitamente figurar em um dos programas sensacionalistas
de fim de tarde, na televisão), Rita, a esposa que será morta pelo
marido, relata ao amante Camilo a visita que fizera a uma cartomante;
também ele fará a mesma coisa, antes de ser assassinado -- e o leitor se
pergunta: que desejará Machado dizer ao flagrar cartomantes que não
alertam seus clientes da tragédia por vir?

Certamente, ``A Cartomante'' é dos melhores contos da Literatura
Brasileira. A alusão imprecisa a uma passagem de ninguém menos do
William Shakespeare, nas primeiras linhas, revela a distância que vai
entre o que se poderia chamar de \emph{céu} -- a literatura
grandiloquente do passado, da tragédia ocorrida entre nobres da corte do
Reino da Dinamarca -- e a \emph{terra} -- os eventos familiares que se
restringem a três personagens e que também terminarão em desgraça; aqui,
as proporções são menores, certamente sem implicações no xadrez
político da Europa, mas o clímax do conto, condensado em seu último
parágrafo, causa no leitor, especialmente pela estrutura do conjunto,
uma forte impressão. A alusão a Shakespeare, a conversa inicial, o
\emph{flash-back} no qual o leitor toma contato com a história do
adultério, o curtíssimo bilhete de Vilela a Camilo, a hesitação deste em
atender ao chamado e sua consulta à cartomante -- todos esses elementos
convergem para o parágrafo final, cuja violência surpreende até mesmo o
mais atento e experiente dos leitores.

Se nos permitirmos uma divagação, podemos nos perguntar quais serão as
consequências do assassinato de Camilo e Rita para Vilela -- e talvez
elas não fossem das piores. É curioso saber que o Código Penal
Brasileiro que vigorou de 1890 a 1940 previa que crimes cometidos por
uma pessoa privada dos próprios sentidos e da inteligência poderiam não
levar à prisão ou a outras penas. Em resumo: quem cometia crime fora de
si não era responsabilizado por ele, o que nos leva a crer que Vilela
talvez não fosse preso pelos dois assassinatos. ``A Cartomante'' foi
publicado na Gazeta de Notícias em 1884 -- isto é, apenas seis anos
antes de esse código entrar em vigor. Crimes como o cometido por Vilela
eram -- e infelizmente ainda são -- comuns na sociedade brasileira. O
conto de Machado de Assis pode ser lido, assim, como um documento de
época a respeito da violência cotidiana, que nos ensina bastante a
respeito de nosso próprio tempo.

Do ponto de vista da escolha do tema, além de dialogar com as grandes
tragédias familiares, como a de Hamlet, apesar das diferenças evidentes
de contexto, ``A Cartomante'' também dialoga com o jornalismo
sensacionalista: as páginas policiais do ``fait divers'', expressão
francesa que, traduzida ao pé da letra, significa ``fatos diversos'' e
que se refere às notícias cotidianas, que correm mundo; são os crimes
que chocam -- e que guardam pelo menos um laço com as tragédias:
Aristóteles dizia que o efeito catártico causado por elas advinha do
\emph{temor} e da \emph{pena} do público. O temor, especificamente,
derivava da percepção de que \emph{a tragédia ocorrida com as
personagens pode acontecer também comigo}. Talvez esteja aí a força da
impressão causada pelo final de ``A Cartomante'': também nós podemos
acabar cedendo aos desejos e traindo nossos amigos, esposas e maridos;
também nós podemos não querer ver a realidade como ela é, tentando
escapar dela pelos charlatães em geral -- porque não se trata de
analisar os fatos, mas de querer \emph{acreditar} nesses embusteiros;
finalmente, também nós podemos sofrer consequências graves daquilo que
fizemos, depois de tentar tapar o sol com a peneira.

\Image{``Hamlet'' de William Shakespeare edição de 1866 (Internet Archive; Domínio público)}{PNLD0048-12.png}

É muito curioso saber que a Gazeta de Notícias, o jornal em que Machado
de Assis publicou ``A Cartomante'', tinha colunas de ``fait divers''. As
seções de nomes ``Ocorrências da Rua'', iniciada em 1878,
``Acontecimentos Notáveis'', em 1880, e ``Casos Policiais'', em 1890,
dividiam o espaço com notícias, editoriais e contos como o de Machado de
Assis -- e noticiavam crimes chocantes, no que se pode considerar o
nascedouro da imprensa sensacionalista brasileira. ``A Cartomante'' pode
ser lido, assim, como texto em que Machado de Assis aproximou a tragédia
de Shakespeare das tragédias cariocas, com a mediação da imprensa.

A título de conclusão dessa análise, cabem aqui duas reflexões.

A primeira, a respeito desses elementos todos -- a citação a
Shakespeare, a estrutura do conto e a semelhança com as colunas de
ocorrências violentas e policiais: talvez Machado de Assis esteja
fazendo, em ``A Cartomante'', um extenso exercício de \emph{linguagem
literária} produzindo mediações entre o céu e a terra, expressões
tomadas aqui, evidentemente, no sentido figurado. De forma geral, o
autor nos mostra que as tragédias familiares se repetem, seja na corte
da Dinamarca, seja no Rio de Janeiro, apesar das diferenças; que essas
tragédias impressionam porque nos identificamos com elas, isto é, porque
todos sabemos que podemos incorrer naqueles erros e sofrer
consequências; e finalmente: na época de Machado de Assis, e também na
nossa, essas tragédias são mercantilizáveis, seja na forma de notícias
sensacionalistas, seja na forma de contos. Nesse sentido, desponta mais
uma vez a capacidade que Machado tem de antever as coisas: programas de
televisão que só noticiam crimes brutais; canais virtuais com vídeos
sangrentos; jornais sensacionalistas -- todos eles seguem transformando
a tragédia alheia em notícia, raras vezes respeitando o sofrimento das
pessoas envolvidas, expondo-as severamente.

A segunda diz respeito ao nosso desejo de não ver o que está diante de
nossos olhos. É o que costuma acontecer às vítimas de tragédias: elas
não percebem que já estão enredadas pelo destino, à beira da própria
desgraça. Camilo não percebe, pouco antes de ser morto por Vilela, que
será assassinado e prefere acreditar no embuste da cartomante. Não
seremos todos nós, ao menos um pouco, como Camilo? Mais ainda: não somos
nós, leitores, levados pelo narrador a preferir acreditar que nada vai
acontecer? Provavelmente sim -- e por isso tomamos aquele susto ao ler o
último parágrafo de ``A Cartomante'': porque continuamos acreditando que
existe ``muita coisa misteriosa e verdadeira neste mundo'', apesar dos
fatos que são noticiados todos os dias na imprensa.


\section{Sugestões de referências complementares}\label{sugestoes}

\subsection{Filmes}

\begin{enumerate}
\item\textbf{Nise, o coração da loucura (2016)}

Direção de Roberto Berliner

A psiquiatra brasileira Nise da Silveira -- personagem principal desse
filme baseado nas experiências reais vividas por ela -- pode ser
entendida como uma espécie de anti"-Simão Bacamarte: enquanto o alienista
machadiano é a personificação da ciência, que desumaniza médicos e
pacientes, a Dra. Nise da Silveira, por meio da terapia ocupacional e da
arte, humanizou os internos do Centro Psiquiátrico Nacional Pedro II,
revolucionando a psiquiatria brasileira.

\item\textbf{Bicho de Sete cabeças (2000)}

Direção de Laís Bodanzky

Baseado na vida de Austregésilo Carrano Bueno, autor do livro \emph{O
Canto dos Malditos}, o filme conta a história de Neto (representado por
Rodrigo Santoro), adolescente que é internado pelo pai em uma
instituição psiquiátrica por abuso de drogas. A crítica contundente
dessa obra às violências cometidas contra os pacientes de manicômios
estimulou o debate a respeito desse tema.

\item\textbf{Azyllo Muito Louco (1969)}

Direção de Nélson Pereira dos Santos

Adaptação de ``O Alienista'' para o cinema, ambientada na cidade de
Paraty. O filme tem grande interesse porque serve de porta de entrada
dos estudantes para o Cinema Novo brasileiro e para a obra de Nélson
Pereira dos Santos, um dos maiores diretores do Brasil.

\item\textbf{Drácula de Bram Stoker (1994)}

Direção de Francis Ford Coppola

Produção luxuosa do diretor norte-americano Francis Ford Coppola.
Trata-se de uma adaptação, para o cinema, do romance de Bram Stoker, na
qual se pode observar claramente que a pretensão da imortalidade dialoga
diretamente com a aurora da Modernidade e da Revolução Industrial. Vale
a pena assistir, também, ao \emph{Nosferatu}, de F.W. Murnau, para
observar a intertextualidade entre esses filmes.

\item\textbf{Nosferatu (1922)}

Direção de F.W. Murnau

Clássico do cinema expressionista alemão, pode ser uma ótima pedida para
apresentar aos alunos os primórdios da linguagem artística do cinema,
além de mostrar o diálogo entre as imagens marcantes desse filme e as do
\emph{Drácula de Bram Stoker}, de Francis Ford Coppola.

\item\textbf{A Cartomante (2004)}

Direção de Wagner Assis e Pablo Uranga

\item\textbf{A Cartomante (1974)}

Direção de Marcos Faria

Duas adaptações do conto de Machado de Assis.
\end{enumerate}

\section{Bibliografia comentada}

\subsection{Panoramas gerais de vida e obra de Machado de Assis}

\textsc{bosi}, Alfredo et al. \textbf{Machado de Assis}. São Paulo: Ática, 1982.

Embora seja de difícil acesso, porque não foi reeditado, esse livro
contém uma das mais alentadas coletâneas de textos de Machado de Assis,
por isso vale a pena indicá"-la. Desde uma biografia inicial, escrita por
Valentim Facioli, da Universidade de São Paulo, até textos de diversos
pesquisadores e especialistas, passando por uma coletânea significativa
dos textos de Machado de Assis, em todos os gêneros nos quais ele se
destacou: crítica, crônica, conto, romance e poesia. Trata"-se de
exemplar fundamental em qualquer coleção de estudiosos de Machado de
Assis, apesar da aludida dificuldade de encontrá"-lo.

_________________________. ``Machado de Assis''. In: \_\_\_\_\_\_. \textbf{História
Concisa da Literatura Brasileira}. 50.ed. São Paulo: Cultrix, 2015.
p.184-194.

Apresentação geral da obra de Machado de Assis por Alfredo Bosi, um dos
maiores pesquisadores brasileiros da obra desse escritor.

\textsc{candido}, Antonio. ``Esquema de Machado de Assis''. In: \_\_\_\_\_\_.
\textbf{Vários Escritos}. 3.ed.rev.ampl. São Paulo: Duas Cidades, 1995.
p. 17"-40.

Apresentação geral da obra de Machado de Assis por Antonio Candido. Vale
especialmente pelo levantamento dos grandes temas da obra do escritor.

\textsc{magalhães jr.}, R. \textbf{Vida e obra de Machado de Assis}. v.1.
Aprendizado; v.2. Ascensão; v.3. Maturidade; v.4. Apogeu. Rio de
Janeiro: Ed. Record, 2008.

A biografia mais extensa e minuciosa de Machado de Assis.

\textsc{merquior}, José Guilherme. ``Machado de Assis e a prosa impressionista''.
In: \_\_\_\_\_\_. \textbf{De Anchieta a Euclides: breve história da
literatura brasileira}. São Paulo: É Realizações, 2014. p. 243"-321.

Apresentação geral da obra de Machado de Assis por José Guilherme
Merquior, um dos maiores estudiosos desse escritor. Vale especialmente
pelo panorama da prosa realista da segunda metade do século \textsc{xix}.

\textsc{pereira}, Lúcia Miguel. \textbf{Prosa de ficção (de 1870 a 1920):
História da Literatura Brasileira.} 2ª ed. Rio de Janeiro: José Olympio,
1957.

Apresentação geral da obra de Machado de Assis por Lúcia Miguel Pereira,
uma das maiores estudiosas dos contos desse escritor.

\subsection{Sobre os contos desta antologia e sobre assuntos a eles relacionados}

\textsc{arbex}, Daniela. \textbf{O Holocausto Brasileiro}. Rio de Janeiro:
Intrínseca, 2019.

A assustadora história do Centro Hospitalar Psiquiátrico de Barbacena,
onde ocorreu uma das maiores barbáries da história do Brasil: mais de 60
mil internos morreram devido aos maus tratos que lhes eram infligidos.
Homossexuais, prostitutas, epiléticos, mães solteiras, meninas
problemáticas, mulheres engravidadas pelos patrões, moças que haviam
perdido a virgindade antes do casamento, mendigos, alcoólatras,
melancólicos, tímidos e todo tipo de gente considerada fora dos padrões
sociais foram internados ao lado de pacientes psiquiátricos. Certamente
é leitura fundamental para apresentar aos alunos os abusos causados pela
excessiva \emph{medicalização} dos comportamentos e a questão manicomial
no Brasil.

\textsc{boucinhas}, André Dutra. ``Machado de Assis e a (sua) Revolução
Francesa''. In: \textbf{Machado de Assis em linha.} Ano 2, número 4,
dezembro 2009.

Disponível em:

\url{http://machadodeassis.fflch.usp.br/sites/machadodeassis.fflch.usp.br/files/u73/num04artigo06.pdf}

Análise minuciosa das referências à Revolução Francesa em ``O
Alienista''. Trata"-se de artigo indispensável para investigar as
relações entre esse evento histórico e a ``Revolução dos Canjicas''.

\textsc{foucault}, Michel. \textbf{História da Loucura}. São Paulo: Perspectiva,
1995.

Uma das obras monumentais de Foucault, contém extensa e minuciosa
pesquisa histórica e crítica das ideias, práticas, instituições e
produções artísticas relacionadas ao conceito de loucura. Quando lido
conjuntamente com \emph{O Holocausto Brasileiro}, de Daniela Arbex, é
útil para apresentar as consequências da excessiva \emph{medicalização}
dos comportamentos.

\textsc{gama}, Vítor Castelões e \textsc{laurentino}, Werbson Azevedo. ``Machado de Assis:
a literatura fantástica no conto O Imortal''. In: \textbf{Revista
Investigações}, Recife, v. 33, n. 1, p. 1"-11, 2020.

Disponível em:
\url{https://periodicos.ufpe.br/revistas/INV/article/view/241490}

Análise do fantástico no conto ``O Imortal'', a partir da obra de
Tzvetan Todorov, também indicada nesta bibliografia.

\textsc{gerlach}, Carmen Lúcia Cruz Lima. ``O Imortal de Machado de Assis''. In:
\textbf{Travessia}. n. 19 (1989). p. 119"-124.

Disponível em:

\url{https://periodicos.ufsc.br/index.php/travessia/article/view/17343}

Breve mas esclarecedora análise do conto ``O Imortal'' e suas matrizes
no conto fantástico.

\textsc{gomes}, Roberto. ``O Alienista: loucura, poder e ciência''. In:
\textbf{Tempo Social}. {[}online{]}. 1993, vol.5, n.1"-2 pp.145"-160.

Disponível em:

\url{http://www.scielo.br/scielo.php?script=sci_arttext\&pid=S0103-20701993000100145\&lng=en\&nrm=iso}

Análise das pretensões e impasses das concepções científicas do século
XIX em ``O Alienista'', especialmente o vínculo entre ciência e poder.

\textsc{matrangano}, Bruno e \textsc{tavares}, Enéias. \textbf{Fantástico Brasileiro: O
Insólito Literário do Romantismo ao Fantasismo.~}Curitiba: Arte e Letra,
2019.

Obra extensa, atual e competente de dois jovens pesquisadores, na qual
são investigados cerca de 200 anos de obras fantásticas da Literatura
Brasileira, do romantismo ao chamado ``fantasismo'', termo cunhado mais
recentemente.

\textsc{parrine}, Raquel. ``Aspectos de Teoria do Conto em Machado de Assis''.
In: \textbf{Eutomia: Revista Online de Literatura e Linguística.} v.1,
n.3, 2009. Disponível em:

\url{https://periodicos.ufpe.br/revistas/EUTOMIA/article/view/1902/1489}

Análise da consolidação do conto, no Brasil, por meio da obra de Machado
de Assis. Leitura útil para conhecer a teoria do conto de Edgar Allan
Poe, na qual a autora fundamenta sua análise dos textos machadianos, e
para compreender-lhes a arquitetura literária. Atenção à análise da
matriz jornalística do conto do século \textsc{xix}, do hibridismo de muitos
contos de Machado de Assis e do interesse do autor pelo conto
fantástico.

\textsc{sasse}, Pedro. ``A besta dentro de cada um: metamorfoses do vampiro na
Literatura Brasileira''. In: \textbf{Revista Abusões}. n.09. v.09. ano
05. p.11"-45.

Disponível em:

\url{http://dx.doi.org/10.12957/abusoes.2019.40692}

O artigo contém um breve mas precioso histórico da literatura vampiresca
e uma análise das especificidades desse tema na literatura brasileira.

\textsc{todorov}, Tzvetan. \textbf{Introdução à Literatura Fantástica}. Trad.
Maria Clara Correa Castelo. São Paulo: Perspectiva, 2012.

Obra introdutória à Literatura Fantástica, para aprofundamento da
análise do conto ``O Imortal'', de Machado de Assis.


\end{document}

