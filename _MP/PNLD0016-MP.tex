\documentclass[12pt]{extarticle}
\usepackage{manualdoprofessor}
\usepackage{fichatecnica}
\usepackage{lipsum,media9,graficos}
\usepackage[justification=raggedright]{caption}
\usepackage[one]{bncc}
\usepackage[edlab]{../edlab}


\begin{document}


\newcommand{\AutorLivro}{Luiz Gama}
\newcommand{\TituloLivro}{O herói da abolição}
\newcommand{\Tema}{Diálogos com a sociologia e com a antropologia}
\newcommand{\Genero}{Diário, biografia, autobiografia, relatos, memórias}
\newcommand{\imagemCapa}{./images/PNLD0016-01.png}
\newcommand{\issnppub}{---}
\newcommand{\issnepub}{---}
% \newcommand{\fichacatalografica}{PNLD0016-00.png}
\newcommand{\colaborador}{{Eduardo Modesto de Carvalho, Bruno Gradella e Vicente Castro}}


\title{\TituloLivro}
\author{\AutorLivro}
\def\authornotes{\colaborador}

\date{}
\maketitle

\begin{abstract}\addcontentsline{toc}{section}{Carta ao professor}
Este Manual tem como objetivo fornecer subsídios para o trabalho com a
obra literária \emph{O Herói da Abolição: A Luta de Luiz Gama no Império
do Brasil}, de Luiz Gama.

Luiz Gama nasceu em Salvador, Bahia, em 1830. Filho de uma mulher negra chamada 
Luiza Mahin, africana da região da Costa da Mina, e de um homem de aparência branca, 
herdeiro de uma família rica de ascendência portuguesa. Aos dez anos de idade, vendido 
pelo pai afim de pagar dívidas de bebidas e jogatinas, foi levado ao Rio de Janeiro 
onde foi comprado por um contrabandista que, não conseguindo revendê-lo, levou-o
para o interior de São Paulo, mantendo-o em cativeiro por oito anos. Lá, ficou amigo 
de um estudante que lhe ensinou a ler e fazer contas. Com isso, conseguiu secretamente 
provas incontestáveis de sua liberdade e fugiu do cativeiro. Assentou praça no exército, 
onde conseguiu cargo de escrivão e pôde se dedicar, nos intervalos do trabalho, à leitura. 
Mesmo sem uma formação em Direito, frequentando as aulas apenas como ouvinte, 
Luiz Gama conquistou uma licença para advogar. Assim, ao longo de sua vida,
contribuiu de forma notável à defesa jurídica de escravizados. 

A presente obra é um compilado de cartas que o jurista escreve 
destinadas ao abolicionista Ferreira de Menezes, além de outros documentos. Os 
textos aqui presentem evidenciam o significado os escritos de Luiz Gama. 
Em seus diversos artigos publicados, Gama 
sempre defendeu a causa abolicionista, sendo um dos principais bastiões na luta 
contra a escravidão. Seu interlocutor, Ferreira de Menezes, naquele momento, 
estava situado no Rio de Janeiro, de modo que muito do que Gama dizia e fazia 
referência em suas cartas eram situações conhecidas de seu correspondente, 
como por exemplo, as maneiras e estratégias do combate à escravidão. 
Além do mais, notamos que na escrita das cartas o jurista buscava fomentar uma 
nova geração de militantes para a causa abolicionista.

Esperamos que as indicações propostas neste manual sejam muito úteis no trabalho em
sala de aula! 
\end{abstract}

\tableofcontents


\section{Atividades 1}

%\BNCC{EM13LP26}

\subsection{Pré"-leitura}

%\BNCC{EM13LGG302}
%\BNCC{EM13LGG704}
%\BNCC{EM13LP10}
%\BNCC{EM13LP19}

\paragraph{Tema} O abolicionismo ao redor do mundo.

\paragraph{Conteúdo} A pesquisa sobre publicações do movimento abolicionista, brasileiras e estrangeiras.

\paragraph{Objetivo} Estimular e habilitar os estudantes a ter contato
com as diversas formas de abordar a luta pelo fim da escravidão. 
Com essa atividade, pretende"-se também capacitar o aluno a
realizar um cotejo entre diversas fontes e extrair informações a partir delas.

\paragraph{Justificativa} A partir de documentos históricos, como as publicações abolicionistas,
é possível uma compreensão maior sobre o significado de abolicionismo e as diferenças de 
como esse movimento se deu ao redor do mundo.

Abolicionismo é o nome que se dá ao movimento político que
visava ao fim da escravidão. Ainda que tenha variado em metodologias e
políticas de ação, esse movimento, em geral, buscava o fim do tráfico
negreiro, a emancipação dos escravos e a integração da população negra
na sociedade.

\paragraph{Metodologia} A realização de uma ampla pesquisa, por parte dos estudantes,
visando encontrar publicações abolicionistas, brasileiras e estrangeiras, 
oferecidas em diversos gêneros e formatos. A partir dessa investigação, peça para que 
eles debatam sobre a definição de abolicionismo e discutam sobre as semelhanças e 
diferenças no processo de abolição da escravatura no Brasil e nos outros países pesquisados.

\paragraph{Tempo estimado} Duas aulas de 50 minutos.

\subsection{Leitura}

%\BNCC{EM13LGG103}
%\BNCC{EM13LP02}
%\BNCC{EM13LP48}

\paragraph{Tema} A instituição da escravidão e seus efeitos na sociedade atual.

\paragraph{Conteúdo} Compreensão de aspectos históricos e sociológicos
da escravidão e dos fenômenos sociais de hoje em dia que são decorrentes
do passado escravocrata, como o racismo e as desigualdades sociais.

\paragraph{Objetivo} Ampliar o conhecimento dos alunos acerca da instituição da 
escravidão e suas consequências. Habilitar os estudantes a reconhecer a importância 
dos escritos de Luiz Gama como afronta aos documentos que visavam legitimar a escravidão.

\paragraph{Justificativa} Esta segunda atividade visa ampliar o conhecimento do aluno
acerca da instituição da escravidão. Comumente se comenta os aspectos
negativos da mesma, mas nem sempre é feito um estudo aprofundado de suas
mazelas. Além disso, cada vez mais a escravidão se restringe ao campo da
história, distanciando"-se da memória, a chaga que ela representa pode
ficar desbotada a olhares menos atentos. Isso acarretaria na dificuldade
da compreensão de fenômenos sociais presentes, como o racismo e as
desigualdades sociais.

Com a leitura dos escritos de Luiz Gama, é de suma importância
frisar os movimentos de resistência negra, a ação de abolicionistas
e as revoltas escravas ocorridas no Brasil e no
mundo.

\paragraph{Metodologia} A produção de uma monografia, por parte dos alunos,
acerca dos aspectos históricos e sociológicos da instituição da escravidão
e suas heranças na sociedade atual. Os estudantes devem pesquisar sobre
o cotidiano escravista ao longo da história e em outras realidades sociais.
Devem observar as particularidades do tráfico atlântico do período moderno e 
o porquê de as consequências disso serem sentidas até hoje. 

Para a produção da monografia, devem ser considerados os escritos de Luiz Gama
que visavam combater os documentos e discursos políticos que legitimaram a escravidão.

Também é conveniente que apareça no trabalho como se deu o fim do
escravismo e como foi o processo de integração dos antigos cativos na
sociedade.

\paragraph{Tempo estimado} Duas aulas de 50 minutos.

\subsection{Pós"-leitura}

%\BNCC{EM13LGG102}
%\BNCC{EM13LGG303}
%\BNCC{EM13LGG402}
%\BNCC{EM13LGG703}
%\BNCC{EM13LP13}
%\BNCC{EM13LP14}
%\BNCC{EM13LP28}
%\BNCC{EM13LP29}
%\BNCC{EM13LP52}

\paragraph{Tema} Os aspectos legais e econômicos do Império Brasileiro.

\paragraph{Conteúdo} Escrita de um texto jornalístico abordando aspectos legais e
econômicos que eram existentes durante o Império Brasileiro.

\paragraph{Objetivo} Analisar documentos históricos e exercitar a escrita jornalística.

\paragraph{Justificativa} Pretende"-se ampliar a compreensão dos estudantes acerca das razões que
tornaram o país tão inerte à mazela da escravidão.

\paragraph{Metodologia} Valendo"-se da pesquisa anterior, os estudantes
devem escrever um texto jornalístico sobre os aspectos legais e econômicos
existentes durante o Império Brasileiro que buscaram legitimar a escravidão
e a fizeram perdurar por tanto tempo.

No trabalho, devem ser abordados os ciclos econômicos que o país passou,
para que se compreenda porque a mão"-de"-obra escrava foi se tornando
tão intrinsecamente ligada às questões produtivas. 

Ademais, é importante observar a formulação de legislação sobre o tema, tanto
os documentos que visaram legitimar o sistema escravocrata quanto os escritos abolicionistas
que pressionaram o Parlamento para decretar o seu fim. No texto, devem ser debatidas
a eficácia e aplicabilidade dessas leis e de que modo as mesmas geraram, ou não,
mudanças significativas na vida da população escravizada.

\paragraph{Tempo estimado} Duas aulas de 50 minutos.

\section{Atividades 2}

%\BNCC{EM13CNT201}
%\BNCC{EM13CNT303}
%\BNCC{EM13CHS101}
%\BNCC{EM13CHS102}
%\BNCC{EM13CHS106}
%\BNCC{EM13CHS401}


\subsection{Pré"-leitura}


\paragraph{Tema} A escravidão ao longo da história.

\paragraph{Conteúdo} Estudo comparado da escravidão na Idade Antiga e Moderna.

\paragraph{Objetivo} Compreender as mudanças da instituição
escravidão em diferentes momentos históricos.

\paragraph{Justificativa}  Existente desde os primórdios das civilizações, a instituição
da escravidão variou ao longo da história humana, bem como adotou
diferentes matizes, de acordo com o local onde aconteceu. A forma que
uma pessoa era reduzida à condição de escravizado também variou ao longo
do tempo. Na Roma antiga, os escravos, ou \emph{servus}, em latim, eram
aqueles que eram conservados dos povos conquistados, ou seja, os
que não haviam sido mortos no campo de batalha. Entretanto, em Roma, o
escravo era protegido pelo ditame da \emph{ius gentium}, ou direito das
gentes, ou seja, mesmo como escravo, era considerado pessoa. Já nas
Américas, na época moderna, a pessoa na condição de escrava era tratada 
de forma desumanizada, não como um sujeito de direitos. 

\paragraph{Metodologia} Para a realização dessa atividade, deve ser buscado
auxílio de professores de Ciências Humanas para oferecer subsídios aos alunos a
respeito das variáveis que a instituição da escravidão deteve ao longo
de sua história. Nessa atividade, é interessante observar questões 
culturais, legais, o \emph{status} do escravo, suas possibilidades de
emancipação, entre outros. O resultado das pesquisas e discussões deve
ser anotado pois será utilizado na próxima atividade.

\paragraph{Tempo estimado} Duas aulas de 50 minutos.

\subsection{Leitura}

\paragraph{Tema} A integração de ex-escravizados no período pós"-escravidão.

\paragraph{Conteúdo} Debate com professores da Ciências Humanas
sobre problemas atuais de países do continente americano e sua
relação com a forma como se deu o fim da escravidão.

\paragraph{Objetivo} Refletir acerca dos problemas contemporâneos dos países do 
continente americano que podem ser oriundos da falta de planejamento da 
integração da população ex"-escravizada na sociedade.

\paragraph{Justificativa} Durante a atividade de pré"-leitura foram abordados aspectos
envolvendo a instituição do escravismo. Ao se observar a emancipação dos
escravizados negros nas Américas, vê"-se um traço comum em quase todos os
países, que foi a ausência de um planejamento para a integração dessa
população na sociedade, ou de políticas reparadoras que visassem
diminuir as consequências de anos de abuso. 

\paragraph{Metodologia} A atividade propõe uma reflexão 
sobre problemas contemporâneos (como racismo,
segregação espacial, desigualdade), partilhados por muitos países
americanos. A ideia é que os alunos, com auxílio do professor de
Ciências Humanas, tracem paralelos entre as experiências de países como
Brasil, Estados Unidos, Haiti, Colômbia, Cuba, por exemplo, observando
como essas nações lidaram com as questões sociais do pós"-escravismo, e
como isso reflete até os dias de hoje.

Durante o debate, seria interessante que os alunos se inspirassem
na trajetória de Luiz Gama e em suas iniciativas jurídicas 
para tirar milhares de pessoas, ilegalmente escravizadas, desta
condição. Os estudantes podem apontar ideias 
sobre quais seriam as contribuições de Gama no período pós"-escravidão,
enquanto políticas reparadoras para a população ex-escravizada. 
Apenas seis anos separam a morte de Luiz Gama, em 1882, e a Abolição da escravidão, em 1888. 
Portanto, ele não viveu para ver o seu sonho de um Brasil sem escravizados se realizar.
Ainda assim, como intérprete sofisticado que era da política e da história do Brasil, via
como inevitável o fim da sociedade escravocrata e seguramente teria seguido em sua 
luta para garantir direitos para a população negra.

\paragraph{Tempo estimado} Duas aulas de 50 minutos.

\subsection{Pós"-leitura}

\paragraph{Tema} O trabalho análogo à escravidão no presente.

\paragraph{Conteúdo} Debate e reflexão acerca da escravidão
moderna e as possíveis iniciativas para combatê-la.

\paragraph{Objetivo} Ampliar o conhecimento dos estudantes
acerca da constituição brasileira e incentivar a pesquisa
sobre os princípios dos Direitos Humanos na área trabalhista, em países distintos.

\paragraph{Justificativa} Oficialmente, a Mauritânia foi, em 1981, o último país do
mundo a abolir a escravidão. Entretanto, não é raro o país aparecer nos
jornais com denúncias de a escravidão lá persistir até hoje. Questões
econômicas e culturais, bem como negligência governamental com a
questão, fazem com que esse mal persista no país. Ainda que pareça
distante, a escravidão moderna é um problema existente em muitos países,
inclusive, aqui no Brasil. E pode ocorrer tanto em locais afastados,
como em fazendas isoladas, como em grandes centros urbanos, onde
pessoas, muitas das vezes, estrangeiros, são confinadas em seus
ambientes de trabalho, recebendo pouco mais que a
alimentação, sendo impedidos de sair em busca de algo melhor. 

\paragraph{Metodologia} Com o auxílio de professores de Ciências Humanas, 
proponha um debate acerca da escravidão moderna.
Para isso, aconselha"-se que a turma leia o texto constitucional
brasileiro, além de notícias, nacionais e internacionais, acerca do tema.
Sugere"-se também que sejam realizadas pesquisas em sites como
o portal:
\url{https://www.freedomunited.org/}, que tem o objetivo de investigar, 
denunciar e combater iniciativas de reduzir uma pessoa à condição de escravidão.

\paragraph{Tempo estimado} Duas aulas de 50 minutos.

\section{Aprofundamento}

Ao chegar ao Ensino Médio, é necessário que os estudantes se aprofundem
na compreensão das múltiplas linguagens e, sobretudo, da linguagem
literária. Em relação à literatura, a \textsc{bncc} traz as seguintes
considerações:

\begin{quote}
{[}\ldots{}{]} a leitura do texto literário, que ocupa o centro do trabalho
no Ensino Fundamental, deve permanecer nuclear também no Ensino Médio.
Por força de certa simplificação didática, as biografias de autores, as
características de épocas, os resumos e outros gêneros artísticos
substitutivos, como o cinema e as \textsc{hq}s, têm relegado o texto literário a
um plano secundário do ensino. Assim, é importante não só (re)colocá"-lo
como ponto de partida para o trabalho com a literatura, como
intensificar seu convívio com os estudantes. Como linguagem
artisticamente organizada, a literatura enriquece nossa percepção e
nossa visão de mundo. Mediante arranjos especiais das palavras, ela cria
um universo que nos permite aumentar nossa capacidade de ver e sentir.
Nesse sentido, a literatura possibilita uma ampliação da nossa visão do
mundo, ajuda"-nos não só a ver mais, mas a colocar em questão muito do
que estamos vendo/vivenciando. (Brasil, 2018, p. 491)
\end{quote}

Nesta seção, desenvolvemos um trabalho de aprofundamento que, em diálogo
com a formação continuada de professores, oferece subsídios para a
abordagem do texto literário. A leitura em sala de aula \emph{O Herói da
Abolição: A Luta de Luiz Gama no Império do Brasil} pode ser enriquecida
pelo aprofundamento no universo literário em que a obra está inserida.


\Image{Desenho de Luiz Gama, publicado no jornal "O Mequetrefe" do Rio de Janeiro, em 1882, feito pelo escritor e cartunista Raul Pompeia. (Raul Pompeia; Domínio Público)}{PNLD0016-05.png}


\subsection{A motivação da obra}

O texto \emph{O Herói da Abolição: A Luta de Luiz Gama no Império do
Brasil}, de Luiz Gama, é um compilado de cartas que o jurista escreve,
destinadas ao abolicionista Ferreira de Menezes. O conteúdo de tais
cartas mostra claramente o que eram os escritos de Luiz Gama.


\Image{Foto de Luiz Gama (Arquivo Nacional; Domínio Público)}{PNLD0016-03.png}


Em seus diversos artigos publicados, nos anos de 1880 e 1881, entre
outras publicações anteriores, Gama sempre defendera a causa
abolicionista, sendo um dos principais bastiões na luta contra o
escravismo no Brasil.

Seu interlocutor, Ferreira de Menezes, naquele momento, estava situado
no Rio de Janeiro, de modo que muito do que Gama dizia em suas cartas
estava ligado a situações conhecidas de seu correspondente, como por
exemplo, as maneiras e estratégias do combate à escravidão. Além do
mais, notamos que na escrita das cartas, o jurista tencionava fomentar
uma nova geração de militantes para a causa abolicionista.

\subsection{As cartas}

Nestas cartas, além do conteúdo já citado, vemos que Luiz Gama escreve
para os companheiros de luta do Rio de Janeiro acerca de seus
adversários, alertando sobre liberais e republicanos daquele período,
que, segundo ele, passavam"-se por abolicionistas, o que seria apenas uma
fachada, pois o real desejo destes políticos seria o fim da abolição
somente no século \textsc{xx}, mantendo assim o \textit{status quo} vigente naquele
momento.


\Image{Documento original da carta que Luiz Gama escreveu em julho de 1880 a Lúcio de Mendonça - Parte 1. (Biblioteca Nacional; Domínio Público)}{PNLD0016-09.png}


\Image{Parte 2 da carta. (Biblioteca Nacional; Domínio Público)}{PNLD0016-10.png}


\Image{Parte 3 da carta. (Biblioteca Nacional; Domínio Público)}{PNLD0016-11.png}


\Image{Última parte da carta. (Biblioteca Nacional; Domínio Público)}{PNLD0016-12.png}


Nestes escritos, podemos verificar que Luiz Gama discorre e se indigna
acerca de acontecimentos ocorridos entre senhores de escravos contra
seus cativos em algumas cidades da província de São Paulo. No primeiro
deles, o escritor repudia o assassinato de quatro escravos por parte da
população da cidade de Itu, no interior paulista. Toda a sua indignação
acerca deste fato mostra não só um Luiz Gama combativo em relação à
brutalidade do sistema escravista, como também a intrínseca relação que
ele propõe entre racismo e escravidão. E, junto a isto, o escritor pontua
de maneira clara: ``O escravo que mata o senhor e cumpre o direito
natural nunca irá se confundir com o povo que assassina heróis''.

Nesta mesma carta onde é citado o caso da cidade de Itu, nos é apresentada
a história de um escravo que tenta fugir de seu senhor em uma fazenda
na cidade de Limeira, também interior de São Paulo. Ao ser descoberto, o
escravo sofre grandes castigos brutais por parte de seu senhor, vindo a
morrer queimado. Portanto, ao vermos estes dois relatos, notamos como
funcionavam as dinâmicas da escravidão. Ao mesmo tempo em que temos a
brutalidade dos senhores e da população para com as pessoas
escravizadas, percebemos que os escravos também eram agentes sociais
que se posicionavam contra o sistema.

\subsection{Os comentários sobre as normas legais}

Em algumas sessões do Parlamento, Luiz Gama indicara, por exemplo, a ineficácia
da Lei Feijó, de 1831, visto que o tráfico continuava a crescer.
Anos mais tarde, Luiz Gama argumenta que a Lei Eusébio de Queirós, de
1850, foi a que realmente conseguiu cessar com o tráfico negreiro. 
Isso é corroborado pela análise de números absolutos,
que mostram realmente uma grande queda na entrada de negros escravizados
que chegavam do continente africano.

Em outro momento, também em uma carta dirigida a Ferreira de Menezes,
Luiz Gama vai aborda a ``Lei Áurea'', de 1871, que, na posteridade, ficou
conhecida como a Lei do Ventre Livre. Para o jurista, tal lei era
enganadora e retrógrada, posto que entendia ser apenas um paliativo para
a questão escravista.


\Image{Princesa Isabel, Conde D'Eu e Machado de Assis na missa em celebração da Abolição da Escravatura. (Antonio Luiz Ferreira; Domínio Público)}{PNLD0016-08.png}


\Image{Missa campal celebrada em ação de graças pela Abolição da Escravatura no Brasil, 1888. Campo de São Cristóvão, Rio de Janeiro-RJ, Brasil. (Brasiliana Iconográfica; Domínio Público)}{PNLD0016-07.png}


Em nova carta para Ferreira de Menezes datada de janeiro de 1881, Luiz
Gama cita novamente a Lei do Ventre Livre, a qual, segundo ele, não gerou
boas aspirações para a abolição da escravidão. Partindo daí, Gama
discorre acerca da força do povo diante da tirania do Estado -- povo que,
segundo ele, tivera acesso somente a leituras fáceis de jornais, fato que
o impedia de criar um pensamento crítico e se levantar contra o sistema.
Portanto, o escritor pede para que as pessoas releiam todo o projeto da
Lei do Ventre Livre e o combatam, em razão de suas intrínsecas inaptidão
e incoerência. Esta argumentação, inclusive, problematiza muito a
questão do que era a educação da sociedade brasileira na segunda metade
do século \textsc{xix}, acessível apenas para poucos e quando o povo tinha um
certo acesso a ela, era apenas para coisas simples e banais do
cotidiano.


\Image{Quadro "Abolição da Escravatura", de Victor Meirelles (1888) (Brasiliana Iconográfica; Domínio Público)}{PNLD0016-06.png}


\subsection{Trabalho de cunho sociológico}

Desta maneira, é também interessante pontuar que, em pleno século \textsc{xix},
Luiz Gama apresenta um pensamento de vanguarda, porque segundo ele,
mesmo em condição de domínio das feras humanas, os senhores de escravos
não impediam que os seus cativos fossem o fator de grandeza do Brasil
daquele período. Esta visão aparecerá novamente em estudos na
área de Ciências hHumanas somente em meados do século \textsc{xx}.

Um outro ponto que vale ressaltar é a análise que Gama faz sobre
os ricos da cidade de São Paulo em suas cartas. Homens, na maioria
das vezes donos de escravos, que sempre vociferavam contra a liberdade
dos cativos. Na opinião de Gama, isto se dava muito por não entenderem o
que era estar em posição de subordinado e não compreenderem, em verdade,
o que realmente era o movimento abolicionista. Além do mais, o apoio de
políticos era muito evidente e concreto junto a esta classe, e se
acrescenta a isso a proximidade destes homens com a religião católica.

\subsection{Um documento histórico}

Em continuidade aos escritos destinados a Ferreira de Menezes, em um datado
de 18 de janeiro de 1881, Gama comenta acerca de uma reunião entre
fazendeiros, negociantes e capitalistas, chamada Club da Lavoura e do
Comércio. Mesmo criticando este evento, o escritor indica que dois
abolicionistas participariam das conversas e que a principal pauta seria
a aquisição de colonos para a lavoura e a mudança do trabalho servil
para o trabalho assalariado.

Portanto, podemos ver muito bem o contexto histórico em que estamos
situados ao analisarmos estas cartas. É o fim do século \textsc{xix}, momento em
que o sistema escravista brasileiro já está em plena decadência, com a
vinda de colonos europeus para um novo tipo de trabalho nas fazendas de
café.

Assim, notamos que o advogado consegue pontuar claramente seu
descontentamento com o sistema escravista do Império e que isto não se
resumia somente à escravidão, mas a todo o contexto sócio"-político ao qual
esta partencia. Junto disto, com estes pequenos acontecimentos que são
expostos, podemos ver bem como era a militância de Gama na causa abolicionista
à segunda metade do século \textsc{xix}, bem como ele relacionava a
escravidão com o racismo existente no Brasil -- pensamento que, para este
período, seria considerado de vanguarda. Desta maneira, podemos colocar
Gama como um dos principais nomes do abolicionismo brasileiro e do
pensamento progressista do século \textsc{xix}.


\Image{Busto em homenagem a Luiz Gama,  situado no Largo do Arouche, na cidade de São Paulo. Inaugurado em 1884. (Everton Zanella Alvarenga; CC0)}{PNLD0016-04.png}


\subsection{Atividades para o aprofundamento da pesquisa}


\subsubsection{Escrever sobre o direito à liberdade}

A leitura de relatos acerca das vivências de ex"-escravizados nos
Estados Unidos do século \textsc{xix} convida à reflexão sobre temas ainda hoje
atuais, mas também possibilita o trabalho no campo das competências
socioemocionais. O registro de trajetórias marcadas pelo sofrimento e
pela superação das adversidades estimula o desenvolvimento da empatia
e da solidariedade em relação a dores alheias. Por isso, na esfera da
subjetividade e dos projetos de vida, os testemunhos de vida podem
servir como ponto de partida para debates e rodas de conversa sobre
temas relevantes para o universo dos adolescentes. Em parceria com
professores de Ciências Humanas, proponha a elaboração individual de
\textbf{crônicas argumentativas} com base nos temas do respeito às
diferenças e do direito à liberdade. 
\BNCC{EM13LGG103} % Escrita criativa

Verifique as visões da turma
sobre a noção de liberdade e observe que valores estão ligados, do
ponto de vista deles, à existência dessa condição. Em seguida,
relacione o estatuto do sujeito livre à necessidade de respeito e
tolerância das diferenças em todos os níveis da existência, mas,
sobretudo, no que tange a questões culturais e étnico"-raciais. Para
fundamentar a argumentação, estimule a pesquisa virtual a passagens da
Constituição que asseguram as liberdades individuais e coletivas.
Oriente, depois, o registro dos argumentos e contra"-argumentos no
caderno, para que seja possível articulá"-los em uma unidade
argumentativa coesa e coerente. Reserve um momento para a produção das
primeiras versões das crônicas, que devem ser preferencialmente
digitadas. Incentive a inserção de exemplos extraídos da atualidade, a
partir de novas consultas a \emph{sites} confiáveis de internet.
Partindo do cotidiano, é possível que a argumentação seja construída
com base em um olhar poético e sensível para a realidade vivida por
muitas pessoas em todo o mundo. O combate à discriminação deve ser
incentivado e argumentos que fujam do senso comum podem ser
construídos a partir da proposição de formas de intervenção concreta
sobre os problemas apontados. As produções podem ser compartilhadas
com os colegas e professores envolvidos na atividade, para que
comentários críticos e construtivos sejam feitos, de forma respeitosa
e democrática. Ao final, as versões definitivas podem ser publicadas
no \emph{site} do colégio, em redes sociais ou em um blog da turma,
após um processo de revisão e edição das produções textuais.

\subsubsection{O preconceito racial no Brasil}

A luta antirracista é uma das principais pautas dos movimentos negros
em todo o mundo. O combate ao preconceito étnico"-racial deve ser
estimulado no cotidiano da sala de aula, sobretudo porque diversos
discursos veiculados socialmente perpetuam o chamado \emph{racismo
estrutural}, que muitas vezes não é notado por grande parte dos
falantes. 
\BNCC{EM13CHS502}


Por isso, a leitura de relatos autobiográficos, produzidos
no contexto da escravidão, pode fundamentar as discussões em sala de
aula. A partir das experiências de leitura de narrativas originárias
dos Estados Unidos, proponha a elaboração, individual ou em dupla, de
um \textbf{artigo de opinião} sobre o preconceito racial no Brasil.
Para isso, com auxílio de professores da área de Ciências Humanas, é
importante estabelecer paralelos comparativos sobre o regime
escravista nos dois países, bem como sobre as consequências
histórico"-sociais da escravidão. Atitudes segregacionistas, racismo
internalizado e estrutural, discriminação nas esferas pública e
privada, violência e desigualdade são aspectos que podem ser
explicados historicamente. Além disso, professores de Ciências da
Natureza podem apresentar argumentos científicos, sobretudo ligados ao
campo de estudo da Genética, para combater o preconceito e as teorias
eugenistas que ganharam força nos séculos \textsc{xix} e \textsc{xx}. A partir dos
debates interdisciplinares, os estudantes poderão articular os
argumentos sob a forma de artigos que discutam, de forma coesa e
coerente, as causas históricas, os impactos sociais e formas concretas
de atuação social e intervenção quanto ao problema do racismo na
sociedade brasileira. É importante incluir elementos trazidos das
leituras, mas também é recomendável estimular a consulta a livros e
\emph{sites} confiáveis sobre a luta antirracista. As primeiras
versões dos textos podem ser compartilhadas, em sala de aula, com os
colegas e professores envolvidos. Na sequência, as produções podem ser
revisadas, digitadas e editadas com auxílio do computador. Ao final,
os textos podem ser publicados no \emph{site} da escola, em redes
sociais ou no blog da turma destinado ao registro das experiências de
leitura.

\subsubsection{Discussão sobre as novas interpretações 
a respeito de determinadas estátuas e monumentos públicos}

A derrubada de estátuas de figuras históricas associadas, sobretudo,
ao racismo, foi uma expressão de protesto multiplicada em muitos
lugares do mundo, no ano de 2020. O movimento foi disparado pelo
assassinato de George Floyd, cidadão negro asfixiado por um oficial
branco, durante uma abordagem policial na cidade de Minnesota, nos
Estados Unidos. A partir desse fato, proponha a produção individual de
um \textbf{ensaio} escrito, por meio do qual os estudantes analisem
criticamente um aspecto da polêmica intervenção sobre estátuas e
monumentos públicos. 
\BNCC{EM13LGG305}


Para isso, estimule a reflexão sobre aspectos
ligados à questão, que podem servir de base para a formulação da tese
e da argumentação desenvolvida no texto.

Verifique se, na opinião dos alunos, a derrubada das estátuas compõe um
gesto: 

\begin{enumerate}
\item agressivo, irresponsável e de violência gratuita contra o
patrimônio histórico; ou 
\item necessário para a conscientização acerca
das injustiças sofridas pelas populações negras ao longo da História;
\item de vandalismo infundado, disfarçado em postura politicamente correta
ou 
\item aceitável para expressar a revolta e tornar público o racismo
presente na sociedade; 
\item de destruição da memória de uma nação e de
iconoclastia de objetos de arte ou 
\item de reparação junto às vítimas
silenciadas pelo opróbrio da escravidão; 
\item de repúdio à suposta
intocabilidade de figuras históricas, cúmplices e responsáveis pela
escravidão ou 
\item de grito de dor, vindo como resposta à segregação
racial, há tanto tempo legitimada e hoje implícita e explícita nas
relações sociais, marcadas pelo racismo estrutural.
\end{enumerate}

Esses tópicos tratam do tema sob duas perspectivas contrárias entre si.
Ao escolher um item para orientar a sua produção, o estudante tomará
partido de um lado ou de outro, mas não deverá desconsiderar os demais
pontos de vista: é preciso também leva"-los em conta ao formular os
argumentos e contra"-argumentos.

Para produzir um ensaio, é preciso mobilizar o repertório sociocultural
e fundamentar os argumentos com base em livros, filmes, letras de
música, obras de arte e textos históricos ou filosóficos, que podem ser
mencionados ao longo do texto. É interessante estimular a consulta a
\emph{sites} de periódicos confiáveis, ligados à divulgação de
atualidades.

O ensaísta é um autor que analisa diferentes aspectos das ideias,
observando cada um dos elementos como se fossem as faces de um prisma.
Lembre"-se, no entanto, de que é preciso defender uma tese central; ao
fazer um ensaio sobre o tema selecionado, é possível adotar diferentes
posturas para construir a argumentação: o aluno poderá acolher uma das
perspectivas, rejeitando a outra, em uma atitude com inclinação ao
formato \emph{ou\ldots{} ou}; também é possível escolher um lugar enunciativo
com inclinação ao formato \emph{e\ldots{} e}, apresentando, para tanto,
soluções alternativas.

Observe, em sala de aula, que as opiniões possíveis a respeito do tema
refletem as vozes às quais cada aluno responderá: elas são, de um lado,
aquelas vozes que comentaram, analisaram e discutiram esse tema na mídia
contemporânea; além disso, de outro lado, há também as vozes dos atores
sociais que, na vivência pretérita, presente e futura dos fatos ligados
à escravidão negra, constituem a História como arena de conflitos
operada discursivamente. Ao comentar, analisar e discutir o tema no
formato de um ensaio, o texto responderá por adesão a determinadas vozes
sociais e não a outras. Em parceria com professores de Ciências Humanas,
reflita profundamente com os alunos sobre o ponto de vista a ser
defendido, uma vez que ser responsivo no ato de criar um ensaio é ser
responsável socialmente.

Os alunos podem elaborar, no caderno, um projeto de texto para organizar
os principais argumentos sobre o tema. É interessante que elaborem mapas
de ideias, façam um \emph{brainstorming} a partir das múltiplas faces da
questão, listem tópicos que auxiliem a posterior elaboração dos
parágrafos e da orientação argumentativa coerente do texto. É importante
desenvolver uma reflexão sobre o aspecto selecionado, verificando se se
as intervenções sobre monumentos do patrimônio público constituem um
gesto agressivo, inconsequente e vão, ou se podem constituir uma
participação necessária, legítima e oportuna na História da humanidade.

O ensaio permite, ao longo do texto, marcas de 1ª pessoa que explicitem
a subjetividade do autor, por meio de expressões como ``Eu creio
que\ldots{}'', ``no meu ponto de vista, \ldots{}'', ``eu penso que\ldots{}'' etc. É
possível até mesmo estabelecer paralelos com experiências pessoais e
fazer breves relatos de situações pessoais que tenham ligação com o
tema. Ao mesmo tempo, esse gênero de texto acolhe modalizadores de
dúvida e incerteza, como ``talvez'', ``é provável que\ldots{}'', ``pode ser
que\ldots{}'', entre outros que captam o processo de construção do pensamento
autoral. Esse gênero textual não visa, portanto, como ocorre em outras
modalidades de argumentação, à construção de um efeito de objetividade,
uma vez que o ensaísta conduz o leitor por um passeio pelo universo das
ideias pessoais e, ao expor um ponto de vista, estimula a reflexão
independente e a formação de opiniões do público. O leitor, portanto,
deve desempenhar um papel ativo e acompanhar o raciocínio do autor,
investigando, completando a análise por meio da autorreflexão e
formulando conclusões consistentes.

A primeira versão escrita do ensaio pode ser compartilhada com os
colegas e com os professores envolvidos na atividade. É recomendável que
os alunos leiam as produções em voz alta e anotem os comentários, as
sugestões e as críticas construtivas. Os posicionamentos podem ser
rediscutidos, com respeito e liberdade de expressão, mas sempre
respeitando os direitos humanos. Após a apresentação, cada aluno fará as
alterações e reformulações necessárias. A segunda versão do texto será
digitada, utilizando o computador , de acordo com a norma padrão da
língua portuguesa. As versões finais podem ser publicadas nas redes
sociais, no \emph{site} da escola ou em um blog destinado à divulgação
dos trabalhos.

\subsubsection{Oficina de poesia sobre a tolerância}


O texto literário oferece possibilidades de reflexão e conhecimento de
situações da vida que podem se aproximar das experiências dos
estudantes ou colocá"-los em contato com realidades vividas por outras
pessoas. Com base na leitura de relatos autobiográficos de
ex"-escravizados norte"-americanos, os alunos podem conhecer
experiências de outros lugares e épocas, que permanecem atuais e
dialogam diretamente com o Brasil da contemporaneidade. Apesar de a
matéria básica das produções ser extraída da vida real, a elaboração
por meio da palavra confere uma dimensão literária às narrativas. Além
de documentos de época, revelam"-se olhares de sujeitos sensíveis,
marcados por situações históricas em contextos de crise. Para
aprofundar o trabalho com conflitos ligados à experiência juvenil,
proponha um mergulho poético nas noções de liberdade e tolerância:
desta vez, os estudantes produzirão \textbf{poemas líricos} para
expressar ideias e sentimentos ligados a esses temas. Como ponto de
partida, é possível retomar produções e discussões feitas ao longo da
leitura, bem como letras de música, pertencentes a gêneros variados,
que tratem igualmente desses valores. Utilizando uma \emph{playlist}
construída coletivamente pelos alunos, privilegiando o repertório do
\emph{hip hop}, reserve um momento para uma oficina de criação
poética. 
\BNCC{EM13LP47} %Sarau


Na sequência, proponha a leitura compartilhada das produções,
comentando"-as e incentivando eventuais reformulações. Por fim, os
textos podem ser digitados, de maneira a compor uma antologia poética
acerca da liberdade e da tolerância, que pode ser publicada no
\emph{site} da escola ou no blog da turma, destinado às atividades
feitas com base nas experiências de leitura.

\subsubsection{Reportagem sobre o tráfico negreiro}


As consequências da escravidão, em diferentes países do mundo,
geraram, ao longo do tempo, experiências semelhantes de preconceito e
discriminação. A leitura de relatos de ex"-escravizados estadunidenses
enriquecem as discussões sobre o tema e permitem traçar paralelos com
realidades diversas, incluindo a da História nacional. No Brasil
contemporâneo, as desigualdades sociais se perpetuam também em
decorrência do passado escravista. Em parceria com professores de
Ciências Humanas, retome as discussões sobre as raízes históricas da
escravidão, com ênfase no Brasil, e proponha uma pesquisa sobre o
tráfico negreiro, desta vez concentrada no contexto brasileiro. Por
meio da pesquisa em livros e \emph{sites} de internet, a turma,
dividida em pequenos grupos, buscará informações sobre números de
pessoas trazidas do continente africano para o Brasil, as
características do sistema social, político e econômico que
sustentavam a escravidão, mapas de rotas do tráfico de escravizados,
relatos acerca do cotidiano nos navios negreiros, as leis que
procuraram modificar esse cenário e poemas produzidos sobre essas
experiências, sobretudo pela vertente socialmente engajada do
Romantismo (conhecida como \emph{condoreira} e praticada, no Brasil,
sobretudo por Castro Alves). A partir dos dados coletados, oriente a
produção de uma \textbf{reportagem} escrita sobre o tráfico negreiro
no Brasil, em linguagem de divulgação, com o objetivo de promover
reflexões sobre questões atuais do país, tanto nas relações pessoais
quanto no mundo do trabalho. As versões finais dessas reportagens,
digitadas e editadas pelos grupos com auxílio do computador, podem ser
compartilhadas no \emph{site} da escola ou no blog da turma, destinado
à publicação de produções ligadas às experiências de leitura.
\BNCC{EM13CHS605} % Direitos Humanos


\section{Sugestões de referências complementares}\label{sugestoes}

\subsection{Filmes e vídeos}
\begin{itemize}
\item\textit{Besouro}. Direção: João Daniel Tikhomiroff (Brasil, 2009).

Na Bahia dos anos 20, o pequeno Manoel é apresentado à capoeira pelo
Mestre Alípio. Ao crescer, Besouro, como passa a ser chamado, recebe a
missão de defender seus semelhantes da opressão e do racismo.

\item\textit{Literatura Fundamental 62 - Luiz Gama - Ligia
Fonseca Ferreira.} São Paulo: Univesp. Disponível em:
\href{https://www.youtube.com/watch?v=WqSuNcU2jdA}{Literatura
Fundamental 62, Ligia F.\~Ferreira. YouTube}. Acesso
em: 03 de março de 2021.

A professora conta a história de Luiz Gama e como o autor foi o primeiro
escritor negro a assumir sua negritude na obra poética e a importância
dele para a literatura brasileira.

\item\textit{O sol é para todos}. Direção: Robert Mulligan (\textsc{eua}, 1963).

Baseado no romance homônimo, o filme conta a história de Tom Robinson,
jovem negro injustamente acusado de violentar uma mulher branca, e do
advogado Atticus Finch, que enfrentou a rejeição da cidade para defender
o réu.

\end{itemize}

\subsection{Lugares para visitar}

\begin{itemize}
\item\textit{Museu Afro Brasil}
(\href{http://www.museuafrobrasil.org.br/}{Museu Afro Brasil}).

O museu, localizado na Vila Mariana, zona sul de São Paulo, reúne um
acervo contendo mais de 5 mil obras relacionadas à cultura africana e
afro"-brasileira.
\end{itemize}


\section{Bibliografia comentada}

\begin{itemize}

\item\textsc{adichie}, Chimamanda Ngozi. \textit{Americanah}. São Paulo: Companhia das Letras, 2014.

Em busca de alternativas às universidades nigerianas, a jovem Ifemelu
emigra para os Estados Unidos. Enquanto se destaca no meio acadêmico,
ela depara com a questão racial e com as dificuldades da vida de mulher
negra e estrangeira.

\item\textsc{adichie}, Chimamanda Ngozi. \textit{O perigo de uma história única}. São Paulo: Companhia das Letras, 2019.

A escritora nigeriana defende que nosso conhecimento é construído pelas
histórias que escutamos e que, quanto mais diversas e numerosas forem
essas narrativas, mais completa será nossa compreensão sobre o mundo.

\item\textsc{angelou}, Maya. \textit{Eu sei por que o pássaro canta na gaiola}.
  Bauru: Astral Cultural, 2018.

Neste romance emocionante, a autora conta a história de Marguerite Ann
Johnson, garota negra criada no sul dos \textsc{eua} pela avó, e dá voz a jovens
que, como ela, enfrentam muitos preconceitos.


\item\textsc{azevedo}, Elciene. \textit{Orfeu de Carapinha: a Trajetória de Luiz Gama na Imperial Cidade de São Paulo.} Campinas: Unicamp, 2005.

O livro retrata a vida de Luiz Gama, filho de fidalgo português e
africana livre. Apesar de ter vivido em cativeiro, conquistou a simpatia
de protetores, alfabetizou"-se, provou que tinha direito à liberdade e
até ingressou na Academia de Direito de São Paulo.

\item\textsc{carvalho}, Gilberto. \textit{O Advogado e o Imperador. A História de Um
Herói Brasileiro.} São Paulo: Duna Dueto, 2015.

O autor recria neste romance histórico a vida do abolicionista Luiz Gama
e o coloca no lugar de protagonista como um herói brasileiro. A relação
entre Luiz Gama e o imperador Dom Pedro \textsc{ii} coloca em foco questões como
liberdade e justiça social.

\item\textsc{evaristo}, Conceição. \textit{Olhos d'água}. Rio de Janeiro: Pallas, 2014.

Com uma escrita lírica e sensível, a escritora mineira reúne breves
contos sobre o cotidiano, concentrando o foco de interesse sobre a vida
da população afro"-brasileira.

\item\textsc{gama}, Luiz. \textit{Com A Palavra. Luiz Gama.} São Paulo: Imprensa
Oficial, 2011.

O livro reúne, pela primeira vez, cerca de quarenta textos integrais de
Gama, muitos deles inéditos, e traz um conjunto de poemas, artigos,
cartas e ensaios dedicados ao autor.

________________________. \textit{Luiz Gama (Retratos do Brasil Negro).} São Paulo:
Selo Negro Edições, 2014.

O objetivo da Coleção é abordar a vida e a obra de figuras fundamentais
da cultura, da política e da militância negra.

\item\textsc{horne}, Gerald. \textit{O sul mais distante}. São Paulo: Companhia das
  Letras, 2010.

Com uma escrita lírica e sensível, a escritora mineira reúne breves
contos sobre o cotidiano, concentrando o foco de interesse sobre a vida
da população afro"-brasileira.

\item\textsc{gama}, Luiz. \textit{Luiz Gama (Retratos do Brasil Negro).} São Paulo:
Selo Negro Edições, 2014.

O objetivo da Coleção é abordar a vida e a obra de figuras fundamentais
da cultura, da política e da militância negra.

\item\textsc{gama}, Luiz. \textit{Com A Palavra. Luiz Gama.} São Paulo: Imprensa
Oficial, 2011.

O livro reúne, pela primeira vez, cerca de quarenta textos integrais de
Gama, muitos deles inéditos, e traz um conjunto de poemas, artigos,
cartas e ensaios dedicados ao autor.

\item\textsc{horne}, Gerald. \textit{O sul mais distante}. São Paulo: Companhia das Letras, 2010.

O autor vê a escravidão em termos hemisféricos e defende que o sul
escravista dos \textsc{eua} via, em uma aliança com o Brasil (o ``sul mais
distante''), uma forma de proteção contra um futuro embate com o norte
estadunidense, na Guerra de Secessão.

\item\textsc{lee}, Harper. \textit{O sol é para todos}. São Paulo: José Olympio,
  2006.

No sul dos \textsc{eua} da década de 1930, uma garotinha esperta e observadora
relata a saga do pai, um advogado que arrisca tudo para defender um
homem negro, injustamente acusado de cometer um crime.

\item\textsc{marquese}, Rafael; \textsc{salles}; Ricardo (org.). \textit{Escravidão e
  capitalismo histórico no século \textsc{xix}: Cuba, Brasil, Estados Unidos}.
  Rio de Janeiro: Civilização Brasileira, 2016.

No sul dos \textsc{eua} da década de 1930, uma garotinha esperta e observadora
relata a saga do pai, um advogado que arrisca tudo para defender um
homem negro, injustamente acusado de cometer um crime.

\item\textsc{marquese}, Rafael; \textsc{salles}; Ricardo (org.). \textit{Escravidão e capitalismo histórico no século \textsc{xix}: Cuba, Brasil, Estados Unidos}. Rio de Janeiro: Civilização Brasileira, 2016.

O livro reúne ensaios de historiadores brasileiros e estrangeiros sobre
a escravização de negros nas Américas, ao longo do século \textsc{xix}.
\end{itemize}


\end{document}

