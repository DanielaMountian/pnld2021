\documentclass[12pt]{extarticle}
\usepackage{manualdoprofessor}
\usepackage{fichatecnica}
\usepackage{lipsum,media9,graficos}
\usepackage[justification=raggedright]{caption}
\usepackage[one]{bncc}
\usepackage[acorde]{../edlab}


\begin{document}

\newcommand{\AutorLivro}{Maria Firmina dos Reis}
\newcommand{\TituloLivro}{Antologia de prosa e versos}
\newcommand{\Tema}{Ficção, mistério e fantasia}
\newcommand{\Genero}{Conto; crônica, novela e poema}
\newcommand{\imagemCapa}{./images/PNLD0014-01.png}
\newcommand{\issnppub}{---}
\newcommand{\issnepub}{---}
% \newcommand{\fichacatalografica}{PNLD0014-00.png}
\newcommand{\colaborador}{\textbf{Rodrigo Jorge Ribeiro Neves}}


\title{\TituloLivro}
\author{\AutorLivro}
\def\authornotes{\colaborador}

\date{}
\maketitle


\begin{abstract}\addcontentsline{toc}{section}{Carta ao professor}

Este manual tem o objetivo de auxiliá-lo no desenvolvimento de práticas
pedagógicas que estabeleçam o diálogo entre a obra literária aqui
apresentada e os estudantes, de modo a ampliar não apenas a leitura do
texto em si, mas também a sua relação com o mundo.

O livro \emph{Antologia de prosa e versos} consiste em uma reunião de
textos de diversos gêneros de Maria Firmina dos Reis, como o conto ``A
escrava'', a novela ``Gupeva'' e poemas extraídos das obras \emph{Cantos
à beira-mar} e \emph{Parnaso maranhense}, além de outras coletâneas mais
recentes da autora. Nesta antologia, a escritora apresenta alguns dos
principais elementos que caracterizam sua literatura, como a condição
dos escravizados, que passam a ter protagonismo nas narrativas, o papel
da mulher na sociedade, as condições dos povos indígenas, um
sentimentalismo romântico amoroso e a exaltação da terra.

Maria Firmina dos Reis é considerada a primeira romancista negra da
história da literatura brasileira e sua obra foi resgatada apenas na
segunda metade do século XX. Além de escritora, foi compositora,
musicista e professora primária, tendo atuação de destaque para a
educação brasileira. Portanto, esta coletânea de suas narrativas curtas
e poemas vem contribuir para que a escritora ocupe o lugar que merece na
história da nossa literatura e da cultura brasileira, trazendo para a
escola a reflexão e discussão de temas ainda tão atuais e que certamente
irão contribuir na formação dos estudantes como leitores críticos da
realidade.

Para isso, apresentamos aqui propostas de atividades, aprofundamento,
referências complementares e uma bibliografia comentada, a fim de que o
material possa ser útil nas suas aulas para estimular os estudantes a
desbravar um universo de possibilidades através de uma das escritoras
mais importantes da nossa literatura. Além disso, é ótimo trabalhar com
contos, novelas e poesia em sala de aula, pois são gêneros literários
bastante fecundos, que possibilitam a dinamização das atividades e a
exploração de uma variedade maior de temas para discussão com os
estudantes.

Aproveite bastante este material. Ele foi feito com muita dedicação e
carinho para você! Boa aula!
\end{abstract}

\tableofcontents

\section{Atividades 1}
%\BNCC{EM13LP26}

\subsection{Pré-leitura}

\paragraph{Tema} Um breve panorama da literatura afro-brasileira.

%(Habilidades BNCC EM13LGG101; EM13LGG102; EM13LP04; EM13LP53)}

\paragraph{Conteúdo} Criação de um panorama dos nomes fundamentais da
literatura afro-brasileira, relacionando suas obras, seus principais
temas e seus vínculos com o seu tempo. A atividade consiste na
elaboração coletiva de uma enciclopédia virtual.

\paragraph{Objetivo} Habilitar os estudantes a discutir e compreender o
conceito de literatura afro-brasileira e a importância das escritoras e
dos escritores que o compõem.


\SideImage{Busto de Maria Firmina dos Reis, localizado na Praça do Pantheon, em São Luís (MA) (Ramsessantos; CC-BY-SA-4.0)}{PNLD0014-03.png}


\paragraph{Justificativa} Nosso maior escritor e um dos mais importantes
da literatura do Ocidente é negro. Machado de Assis, por muito tempo,
foi considerado uma exceção, e até mesmo negligente em relação a temas
raciais. Nada mais injusto. Pesquisadores, como o professor da UFMG
Eduardo de Assis Duarte, demonstraram que Machado de Assis não deixou de
abordar em sua produção ficcional e jornalística a barbárie da
escravidão e as condições torpes em que vivia a população negra de seu
tempo e ao longo da história.




O cânone literário também foi revisto, resgatando escritoras e
escritores afrodescendentes, que denunciaram e combateram o preconceito
e a exploração de seus iguais. Maria Firmina dos Reis, Luiz Gama, Cruz e
Sousa, Lima Barreto, Ruth Guimarães, Nascimento Moraes, Abdias do
Nascimento, Conceição Evaristo, Carolina Maria de Jesus, Mãe Beata de
Iemanjá, entre tantas outras autoras e autores, são uma pequena amostra
do que constitui a nossa literatura afro-brasileira, composta por
escritores que não apenas representam a riqueza dessa categoria na
história da literatura brasileira, mas também a resistência às condições
impostas pelo racismo estrutural na nossa sociedade e que se refletem na
constituição do cânone literário. Com esta atividade, os estudantes
poderão refletir sobre essas condições e contribuir para um novo olhar
sobre a nossa história.

\paragraph{Metodologia}

\begin{enumerate}
\item
A atividade pode ser feita em grupos ou
individualmente. Caso os estudantes optem pelo grupo, seria adequado que
este não ultrapassasse três membros. De qualquer modo, esta é uma
atividade coletiva, em que todos irão atuar em conjunto para a
construção de uma enciclopédia virtual de literatura afro-brasileira.
Portanto, quanto mais verbetes, melhor.
\SideImage{Um dos mais importantes escritores na literatura do Ocidente é negro: Machado de Assis. (Autor desconhecido; Creative Commons)}{PNLD0014-04.png}

\item
Os estudantes podem escolher dentre os nomes relacionados
anteriormente ou pesquisar outros disponíveis, do século XIX até a
atualidade. Eles devem considerar em suas investigações os seguintes
aspectos: vida, obra, contexto histórico e movimento literário. Os
textos produzidos a partir desse levantamento não deve se limitar apenas
à descrição dos fatos. Estimule-os a formular uma análise crítica da
autora ou do autor pesquisado, elegendo uma das obras para comentar com
mais detalhes, enriquecendo, assim, o verbete enciclopédico.

\item
Imagens de fotografias e de fac-símiles de algumas obras podem
enriquecer ainda mais o verbete. Eles podem buscar nas páginas de busca
ou nos acervos virtuais que tenham material sobre as autoras e autores,
como a Biblioteca Nacional, o Arquivo Nacional e a Biblioteca Brasiliana
Guita e José Mindlin, da Universidade de São Paulo.

\item
As obras em domínio público e disponíveis na internet podem ter os
seus respectivos links informados no verbete, o que contribui para o
acesso dos usuários ao material apresentado.

\item
A publicação dos verbetes pode ser através de blog ou da produção de
um e-book com os estudantes. Os verbetes podem conter entre duas a cinco
páginas, contando com as imagens.

\item
Depois de concluída a atividade, faça uma roda de conversa com os
estudantes e proponha uma discussão sobre o que aprenderam com a
pesquisa feita, suas dificuldades, bem como os pontos positivos
trabalho.
\end{enumerate}

\paragraph{Tempo estimado} Um bimestre de aulas.

\subsection{Leitura I}

\paragraph{Tema} A relação do conto com as narrativas produzidas nas redes sociais. 

%(Habilidades BNCC EM13LGG101; EM13LGG703; EM13LGG301; EM13LP50; EM13LP53)}

\paragraph{Conteúdo} Discussão sobre a atualidade do gênero conto a partir
de experiências com as narrativas construídas nas redes sociais pelos
usuários, como, por exemplo, Twitter, Instagram, TikTok e YouTube. O
conto ``A escrava'', de Maria Firmina dos Reis, é o principal objeto
desta atividade.

\paragraph{Objetivo} Habilitar os estudantes a reconhecer e compreender os
principais elementos e as estratégias estéticas e narrativas do conto
como gênero literário, tendo em vista o funcionamento das histórias que
eles acompanham e/ou constroem nas principais redes sociais.

\paragraph{Justificativa} Não é nenhuma novidade que as redes sociais e
outras plataformas do mundo virtual vêm ocupando cada vez mais o
cotidiano de todos nós, especialmente dos estudantes, que vêm crescendo
e se formando na chamada Web 2.0. Tendências, estilos e pontos de vista
são profundamente influenciados pelo contato dos jovens com essa
realidade (ou, seria multi-realidade?), conferindo novas maneiras de
perceber, sentir, se expressar e se posicionar no mundo.

Mas nem tudo é apenas novidade. Embora os meios em que as informações e
as experiências dos indivíduos tenham sido transformados nas últimas
décadas, há elementos e estruturas que são mais antigos que os nossos
próprios avós. As redes sociais, ainda que sejam uma realidade
completamente diferente daquela que vivíamos há poucas décadas atrás,
conservam práticas, pensamentos e sensibilidades desde épocas
imemoriais, refletindo a sociedade em suas principais dimensões. A nossa
necessidade contar histórias é uma delas.

O conto é um gênero literário moderno e, pela sua extensão curta, pode
ser facilmente incorporado, por meio de seus componentes estruturais
narrativos, ao funcionamento da comunicação pelas redes sociais, não
apenas pela sua instantaneidade, mas também por se concentrar em
determinados indivíduos ou situações. Por isso, é fundamental que o
estudante, com a proposição e adequada mediação do professor, os
identifique e analise a partir da sua experiência como usuário das
principais redes sociais e outras plataformas do mundo virtual, para que
se familiarize e compreenda os principais aspectos do conto e suas
diversas possibilidades.

\paragraph{Metodologia}
\begin{enumerate}
\item
Como ponto de partida, o professor pode começar
perguntando aos estudantes o que é um conto, sem se preocupar com uma
definição teórica ou determinada nesse momento. Anotar na lousa as
definições mais recorrentes dada pelos participantes da aula ou pedir
para que eles as registrem e mantenham para a etapa posterior da
discussão são boas opções. O mais importante, nesse momento, é elaborar,
coletivamente, uma definição do gênero literário tendo em vista as
experiências de cada um. Ainda que ninguém tenha lido um conto na vida
(algo bem difícil de acontecer), em sentido estrito, é fundamental que o
educador os encoraje a dizer o que eles entendem como um texto do
gênero, pois, muitas vezes, os contatos se deram por meio de gêneros
fronteiriços ou híbridos, que incorporam os procedimentos narrativos do
conto em suas estruturas. E essa percepção é bastante útil na presente
atividade, pois estimula os estudantes a pensar no texto levando em
conta não a sua forma acabada e fechada, mas o seu processo de
construção e os elementos estruturais que o compõem, ou seja, da parte
para o todo.

\item
É muito importante que, durante essa discussão, o educador procure
sempre exemplificar com textos, de preferência outros que não sejam
desta coletânea, a fim de que o processo de aprendizagem seja ampliado.
Claro que é imprescindível que a conversa se desenvolva com textos
conhecidos dos estudantes, principalmente levando em contas as sugestões
dadas por eles, mas não se deve perder, de maneira alguma, a
oportunidade de sugerir novas obras, incutindo, sempre, a curiosidade e
o gosto pela leitura.

\item
Após essa etapa introdutória, o professor pode sugerir aos estudantes
que tentem, a partir dos resultados anotados na conversa sobre a
definição do gênero conto, relacionar cada item com os recursos das
redes sociais que eles conhecem e/ou utilizam. Naturalmente as
definições vão ser reformuladas de acordo com a plataforma comparada, de
maneira que elas se adequem as suas respectivas funcionalidades, por
isso, o educador precisa ficar atento para que não se perca de vista o
objetivo da atividade, que é reconhecer e compreender como os elementos
compositivos do conto, enquanto gênero literário, pode estar presente
nos recursos das redes sociais, ou seja, elas também podem ser espaços
de narração de histórias curtas.

\item
Peça para que os estudantes formem grupos ou duplas e proponha que
utilizem os recursos de qualquer rede social para narrar o conto ``A
escrava'', de Maria Firmina dos Reis, ou relacionar as personagens e a
ação deste conto com outras narrativas curtas, em perspectiva comparada.
Nesse caso, quem sabe o grupo ou dupla proponha o piloto de uma série?
Todas as propostas são bem-vindas. Na aula em que os trabalhos vão ser
apresentados, incentive que todos discutam o conto adaptado para as
plataformas virtuais, apontando os elementos que os aproximam do gênero
literário e os que os afastam e/ou os colocam em xeque, testando os
limites não apenas da narrativa curta, mas também das redes sociais como
espaços de produção e compartilhamento de histórias. Comece pelo mais
elementar em análise literária, como enredo, personagens, tempo,
ambiente e outros componentes importantes de uma narrativa ficcional,
então, deixe que os estudantes ampliem a partir da recepção do conto e
do processo de criação do grupo para a adaptação da história.
\end{enumerate}

\paragraph{Tempo estimado} Duas aulas de 50 minutos.

\subsection{Leitura II}

\paragraph{Tema} Diálogos entre poesia e música.

%(Habilidades BNCC   EM13LGG301; EM13LP48; EM13LP49)}

\paragraph{Conteúdo} Discussão sobre as relações entre poesia e música,
suas aproximações e diferenças.

\paragraph{Objetivo} Habilitar os estudantes a compreender as
especificidades da poesia e da música em suas relações.

\paragraph{Justificativa} A poesia nasceu com música. Na Grécia Antiga,
como descreve Aristóteles em sua \emph{Poética}, a poesia lírica era
acompanhada de instrumentos musicais e, às vezes, também com dança. Na
Idade Média, os trovadores também faziam uso de instrumentos para compor
as suas cantigas de amor e de amigo. Com o tempo, a poesia e a música
foram tomando caminhos distintos, mas nunca se separaram completamente.
Embora a poesia seja um gênero que não dependa mais da execução de uma
música para ser apreciada há muito tempo, ela ainda traz desta arte o
ritmo dos versos e a sonoridade das palavras na composição de suas
imagens, não importa se possuem métrica ou se são livres.


\Image{O quadro "A música e a poesia", do italiano Adolfo Wildt, de 1920. (Adolfo Wildt; CC-BY-SA-4.0)}{PNLD0014-05.png}


Não são poucos os exemplos de poetas que se dedicaram também à música,
como Mário de Andrade, Vinícius de Moraes e Antonio Cicero, ou de
músicos que fazem de suas canções um belo poema, como Aldir Blanc, Chico
Buarque, José Miguel Wisnik, Baco Exu do Blues e Emicida. As fronteiras
são, às vezes, bem delimitadas, em outras, tênues e quase se misturam, o
que acaba enriquecendo mais essas duas expressões essenciais para a vida
humana. Maria Firmina dos Reis, além de escritora, foi também musicista
e compositora. A partir da discussão de sua produção poética, esta
atividade consiste em analisar a relação entre poesia e música.

\paragraph{Metodologia}
\begin{enumerate}
\item
Comece a atividade com uma conversa com os
alunos sobre como eles entendem a relação entre poesia e música. Quais
as diferenças e semelhanças? Toda poesia pode ser musicada? Toda música
com uma letra pode ser considerada um poema? Quais são as
especificidades de cada uma? Não se prenda, nesse início, a conceitos
teóricos pré-determinados, deixe que as impressões e o conhecimento
prévio dos alunos com essas duas expressões artísticas possam ser o
material para a ponte que vocês irão construir juntos durante a
realização desta atividade.

\item
A cantora e compositora paraibana Socorro Lira lançou, em 2019, o
álbum \emph{Cantos à beira-mar}, com músicas feitas a partir dos poemas
do livro homônimo de Maria Firmina dos Reis. Peça para que os estudantes
escutem em casa e, se puder, escutem juntos em sala de aula algumas das
composições. O álbum está disponível nas principais plataformas de
streaming e alguns vídeos podem ser encontrados no YouTube. Você pode
selecionar os poemas e debater com os alunos as relações entre o texto
escrito e a adaptação feita pela musicista.


\Image{A cantora e compositora paraibana Socorro Lira lançou, em 2019, o álbum Cantos à beira-mar, com músicas feitas a partir dos poemas do livro. (Daniel Kersys; CC-BY-SA-4.1)}{PNLD0014-06.png}


\item
Proponha aos educandos que façam também suas adaptações musicais de
poemas escolhidos de Maria Firmina, independentemente do gênero. A
composição pode ser apenas instrumental também, afinal, o som constrói
imagens, elemento fundamental na concepção da poesia. As possibilidades
são inúmeras, desde apresentação acompanhada de instrumentos musicais e
\emph{Slams} até produção videoclipes. Neste caso, é fundamental ter um
roteiro bem definido.

\item
Após a apresentação dos trabalhos, retome a discussão inicial da
atividade em uma aula, a fim de observar se as noções sobre poesia e
música foram modificadas. Peça para que contem sobre o processo de
criação das músicas, seus desafios e aprendizados.
\end{enumerate}

\paragraph{Tempo estimado} Quatro aulas de 50 minutos.

\subsection{Pós-leitura}

\paragraph{Tema} Criação e crítica em prosa e verso.

%EM13LGG101; EM13LGG301; EM13LP13; EM13LP46; EM13LP53)}

\paragraph{Conteúdo} Realização de uma oficina literária de textos em
prosa e verso, exercitando a criatividade e olhar crítico diante dos
problemas da sociedade.

\paragraph{Objetivo} Habilitar os estudantes a criar contos e poemas em
uma oficina de criação literária e discutir os principais elementos de
seu processo criativo, a fim de compreender as relações e conflitos do
gênero em suas diversas possibilidades.

\paragraph{Justificativa} Escrever é também ler, assim como a leitura
também produz um texto, ainda que não necessariamente escrito. Portanto,
estimular e desenvolver a escrita nos estudantes é fundamental para que
eles possam exercitar não apenas os elementos básicos na construção de
um texto, mas também a sua relação com a realidade que os cercam, em
suas dimensões estética, social, ética e política.

A dinâmica de uma oficina literária permite não apenas refletir e
discutir sobre a concepção de gêneros em prosa e em verso, mas também
aguçar a criatividade, a percepção, a sensibilidade e a visão de mundo
dos educandos. A atividade não é para ser uma espécie de ``agência de
caça-talentos'' literária, claro. Não que não seja possível que exista
entre os alunos uma sementinha de uma grande escritora ou de um grande
escritor. Se houver, ótimo! Contudo, o mais importante é que todos
possam exercitar ativamente sua capacidade de criar e recriar visões de
mundo a partir de uma história, aprendendo que escrever é também uma
forma crítica de leitura.

\paragraph{Metodologia} Esta atividade pode ser o projeto de algo mais
amplo e, quem sabe, permanente dentro da escola, abrangendo outros
gêneros literários e outras turmas.

Para a criação da oficina literária de prosa e poesia, você pode seguir
os passos a seguir, que podem, evidentemente, ser adaptados de acordo
com o lugar, o tempo e as condições disponíveis para que a atividade
seja desenvolvida:

\begin{enumerate}
\item
  Organize os alunos em uma roda, a fim de que a atividade de construção
  de histórias seja, sobretudo, uma atividade de compartilhamento de
  experiências;
\item
  Embora a escrita seja considerada, em princípio, como uma empreitada
  individual, nada impede que também seja coletiva, portanto, permita
  que grupos de dois ou mais estudantes sejam formados. É uma boa
  oportunidade, aliás, para discutirem a questão da autoria na criação
  literária. Apenas limite o número de integrantes, a fim de que a
  atividade seja proveitosa para todos. Um grupo formado por um número
  muito grande de alunos pode acabar dispersando-os e frustrando o
  objetivo da oficina;
\item
  Para os estudantes dedicados à prosa, proponha um exercício de
  construção de personagem. O ponto de partida pode ser alguém do
  convívio do aluno, mas que não seja tão próximo, de uma notícia de
  jornal, de um quadro, de uma fotografia ou de um jogo de videogame. As
  celebridades e as pessoas bem conhecidas devem ser evitadas, pois isso
  pode interferir no processo criativo. O aluno pode criar um personagem
  do nada? Pode, claro. Mas a ideia de utilizar pessoa já existente,
  seja real ou ficcional, é utilizar os elementos humanos percebidos
  pelos alunos em sua primeira leitura da realidade que o cerca e,
  assim, por meio do processo de construção de personagem, estabelecer
  uma leitura mais ativa sobre o seu objeto;
\item
  Eles podem fazer o mesmo em relação ao lugar e à época. As opções são
  ilimitadas. Os nomes podem ser os mesmos do mundo real. Quem sabe a
  rua onde o aluno more ou um lugar que ele adore visitar nas férias, ou
  talvez que gostaria de conhecer? São bem-vindas, sem dúvida, as
  possibilidades ficcionais. Lembra de \emph{O senhor dos anéis}? Se
  você leu o livro de J. R. R. Tolkien ou viu o filme adaptado e
  dirigido por Peter Jackson, sabe que o escritor britânico criou um
  mundo completamente novo por meio da literatura. Os alunos devem se
  sentir livres para criar, no entanto, é preciso dar um passo por vez,
  de modo a compreender os principais elementos que tornam uma história
  verossímil ou não. E é por meio da consciência na utilização desses
  elementos que isso vai ficando mais claro;
\item
  Para os estudantes dedicados à poesia, proponha a descrição de uma
  paisagem a partir da emoção que ela provocou. Pode ser também um
  objeto, uma construção, uma rua ou qualquer outro elemento material.
  Ele também pode partir de uma indignação coletiva, como o problema da
  desigualdade, mas é preciso partir de um exemplo concreto. Mostre o
  ``Poema retirado de uma notícia de jornal'', de Manuel Bandeira, que
  se encontra em seu livro \emph{Libertinagem} (1930); e ``Procura da
  poesia'', de Carlos Drummond de Andrade, que está em seu livro \emph{A
  rosa do povo} (1945). Discutam o processo de criação dos dois poemas e
  os recursos utilizados por cada poeta na construção dos versos;
\item
  Em alguns encontros da oficina, proponha discussões sobre filmes, a
  fim de que possam analisar em conjunto os principais componentes da
  história que está sendo contada. A escolha da obra fica a seu
  critério, procurando sempre ouvir as sugestões dos estudantes;
\item
  Escrever o primeiro parágrafo de uma história, no caso da prosa, pode
  ser decisivo para o seu desenvolvimento. Analisem os inícios dos
  contos desta coletânea e traga outros exemplos clássicos da
  literatura, como, por exemplo, a novela \emph{A metamorfose}, de Franz
  Kafka. Os estudantes podem praticar a escrita do primeiro parágrafo,
  apresentando-o aos demais;
\item
  À medida que os encontros da oficina avançarem, incentive sempre que
  eles apresentem parte do desenvolvimento de seus textos, ainda que
  estejam incompletos e empacados em algum ponto determinado. A
  discussão em sala de aula pode contribuir para que eles solucionem
  alguns desses problemas;
\item
  Quando finalizadas, as histórias e os poemas podem ser apresentados da
  forma como cada aluno ou grupo preferir. Os textos podem ser reunidos
  de diversas maneiras, a depender dos recursos que estiverem
  disponíveis. Se houver a possibilidade de publicação do material
  impresso, não hesite, claro, mas não deixe de estimular a divulgação
  do trabalho através dos meios que o universo digital oferece.
\end{enumerate}

\textbf{Tempo estimado} Dois bimestres.

\section{Atividades 2}

\subsection{Pré-leitura}

\paragraph{Tema} Relatos da escravidão, da resistência e da liberdade.
  
%EM13LGG101; EM13LGG201; EM13LP01; EM13LP33; EM13CHS101; EM13CHS102; EM13CHS601

\paragraph{Conteúdo} Pesquisa e estudo do período da escravidão no Brasil,
a partir dos principais personagens históricos que lutaram pela
liberdade. A atividade consiste na apresentação dos trabalhos por meio
de uma exposição física ou virtual.

\paragraph{Objetivo} Habilitar os estudantes a compreender as dinâmicas
envolvidas no processo que culminou no fim da escravidão e de outras
opressões históricas.

\paragraph{Justificativa} A Abolição da Escravatura, em 13 de maio de
1888, é a data de assinatura de uma lei. Mas, até chegar a ela, os
movimentos de luta e resistência contra a escravidão tiveram importância
decisiva nesse processo. Por isso, é fundamental conhecer os personagens
envolvidos nessas disputas para que eles possam assumir o protagonismo
que tiveram nos respectivos eventos históricos.


\Image{Quadro "Abolição da Escravatura" de Victor Meirelles, de 1888 (Victor Meirelles; Brasiliana Iconográfica)}{PNLD0014-07.png}


Mesmo após essa conquista, não faltaram, e ainda não faltam, episódios
em que grupos e indivíduos tiveram que lutar para que direitos básicos
fossem garantidos e não fossem vitimados pela violência promovida pelo
racismo estrutural. De Zumbi dos Palmares, Luiz Gama e Manuel Congo até
Lélia Gonzalez, Abdias do Nascimento e Sueli Carneiro, são inúmeras as
personagens importantes que lutaram e ainda lutam pela liberdade em
todos os aspectos. Portanto, esta atividade se torna fundamental para
que os estudantes reflitam e compreendam as dinâmicas dos processos
históricos a partir dos seus principais atores.


\SideImage{Estátua de Zumbi dos Palmares, líder do quilombo dos Palmares, um exemplo de resistência dos escravizados. (Elza Fiúza/ABr - Agência Brasil; CC-BY-3.0-BR)}{PNLD0014-08.png}


\Image{Busto em homenagem a Luiz Gama, figura muito importante na luta da abolição da escravatura. (Everton Zanella Alvarenga; CC-0)}{PNLD0014-09.png}


\paragraph{Metodologia}
\begin{enumerate}
\item
Organize as turmas em grupos e proponha que
eles escolham um importante personagem da história do Brasil que tenha
lutado, ou ainda lute, pela liberdade do povo negro em todos os
aspectos. Uma proposta interessante para iniciar essa etapa é a
discussão do samba-enredo da escola de samba Estação Primeira de
Mangueira, de 2019, intitulado ``História para ninar gente grande''. O
desfile também está disponível no YouTube.


\Image{Foto do desfile da Escola Primeira da Mangueira, em 2019 (Vanessa Garcia; CC-BY 2.0)}{PNLD0014-10.png}


\item
Cada grupo deve montar uma exposição física ou virtual sobre a
personagem escolhida. A metodologia precisa ser bem definida. Vai ser
apenas de imagens ou de vídeos? Ou ainda uma combinação dos dois? Vai
haver apresentação de alguns objetos, páginas de jornais, revistas ou
ilustrações? Exemplares de livros ou fac-símile de manuscritos?

\item
Depois de delimitar a metodologia de pesquisa e apresentação dos
resultados, eles devem dividir a exposição em seções, de maneira que as
partes se articulem como em uma história, afinal, uma exposição não é
apenas a presença pura e simples do objeto relacionado ao tema, mas uma
narrativa sobre ele. Então, eles precisam contar uma história!

\item
Os estudantes podem pesquisar tanto nos arquivos físicos quanto nos
virtuais. E são inúmeros! Na rede, eles podem consultar a Hemeroteca
Digital da Biblioteca Nacional, com praticamente todo o acervo de
periódicos mantidos pela instituição, que também disponibiliza imagens
em sua Brasiliana Fotográfica. As páginas de importantes arquivos no
país também prestam esse serviço, como o Arquivo Nacional, assim como
instituições privadas, como Instituto Moreira Salles. A Biblioteca
Brasiliana Guita e José Mindlin, da Universidade de São Paulo, também
disponibiliza diversos periódicos e livros raros em seu acervo digital.
É preciso estar atento, pois algumas cobram uma taxa para a reprodução
de imagens. E não se esqueça de avisá-los de sempre citar a fonte!

\item
Cada grupo deve produzir também um texto de curadoria da exposição, a
ser distribuído entre os colegas, informando a personagem histórica, a
metodologia, os itens expostos, os acervos consultados e uma análise
sobre o conjunto em diálogo com o tema escolhido.
\end{enumerate}

\paragraph{Tempo estimado} Quatro aulas de 50 minutos.

\subsection{Leitura I}

\paragraph{Tema} As condições do negro na sociedade escravista
  
%EM13LGG101; EM13LGG201; EM13LP01; EM13LP33; EM13CHS101; EM13CHS102; EM13CHS601

\paragraph{Conteúdo} Pesquisa e estudo das condições do negro durante o
período da escravidão no Brasil, a partir do conto ``A escrava'' e de
poemas escolhidos de Maria Firmina dos Reis. A atividade consiste na
apresentação dos trabalhos por meio de representações cênicas, como
teatro, cinema, dança ou performance.

\paragraph{Objetivo} Habilitar os estudantes a refletir e discutir sobre
as condições vividas pelo negro em um período de escravidão, a fim de
compreender suas ressonâncias posteriores.

\paragraph{Justificativa} O conselheiro Aires, personagem do último
romance de Machado de Assis, anota no dia 13 de 1888: ``Ainda bem que
acabamos com isto. Era tempo. Embora queimemos todas as leis, decretos e
avisos, não poderemos acabar com os atos particulares, escrituras e
inventários, nem apagar a instituição da história, ou até da poesia''.
Ironicamente, consta que Rui Barbosa, de fato, mandou incendiar arquivos
relacionados à escravidão que eram mantidos por órgãos públicos
vinculados ao Ministério da Fazenda, pouco depois de ser proclamada a
República. Existe, no entanto, um consenso de que eram documentos
fiscais, e que a medida do jurista e abolicionista visava evitar que os
antigos escravizadores pleiteassem indenização junto ao governo.

As marcas estão aí, até hoje, estão aqui. A inexistência de documentos,
fiscais ou não, apenas poderiam atrasar um pouco as tentativas de
reparação histórica, mas a forma como a sociedade atual lida com as
questões raciais, reforçando, cada vez mais, a existência do racismo
estrutural, colocam em evidência essas marcas que não foram e não serão
apagadas. A história, a literatura, os arquivos e os estudos em diversos
outros campos do saber expõem, cada um à sua maneira, a permanência
dessa barbárie de séculos. Assim, a articulação dessas áreas a partir da
leitura dos textos de Maria Firmina dos Reis pode contribuir para a
compreensão dos mecanismos que propiciaram a escravidão no Brasil e sua
permanência na estrutura da sociedade até os dias atuais.

\paragraph{Metodologia}
\begin{enumerate}
\item
Além dos textos de Maria Firmina dos Reis,
comente outras representações literárias e artísticas sobre o período,
como, por exemplo, os quadros de Johann Moritz Rugendas e Jean-Baptiste
Debret. Mostre algumas delas em sala de aula, apontando as relações
entre a linguagem pictórica e o discurso histórico, problematizando as
possibilidades e os limites de cada um.


\Image{Quadro "Navio Negreiro" de Johann Moritz Rugendas (1830) (Johann Moritz Rugendas; Domínio Público)}{PNLD0014-11.png}


\Image{Quadro "O Jantar" de Jean Baptiste Debret (1839) (; Domínio Público)}{PNLD0014-12.png}


\item
Organize a turma em grupos e proponha uma adaptação cênica inspirada
no conto ``A escrava'' ou em poemas escolhidos da escritora que discutam
e exponham a condição do negro na sociedade.

\item
Os alunos podem adaptar o conto integralmente, claro, mas não se pode
perder de vista a intertextualidade e interdisciplinaridade que as
atividades desta seção propõem. Uma outra sugestão é dialogar
comparativamente com obras e questões contemporâneas, demonstrando as
ressonâncias da sociedade escravocrata nos dias atuais.

\item
Eles podem apresentar o trabalho por meio de qualquer linguagem
cênica: teatro, dança, cinema, performance, \emph{Slams} ou até mesmo
animação. Um ótimo exemplo de uso de várias linguagens cênicas para
discutir o tema é o documentário do rapper Emicida, \emph{AmarElo: É
tudo pra ontem}, dirigido por Fred Ouro Preto. É possível que os
estudantes já tenham assistido, mas, se você achar oportuno, leve para a
escola e vejam o filme juntos. Pode ser uma boa fonte de inspiração para
os estudantes.

\item
Não esqueça de pedir a eles o roteiro do trabalho antes de
apresentá-lo, a fim de que possam comentar o texto, o conteúdo e
metodologia utilizada. Depois de apresentarem os trabalhos e realizarem
os devidos ajustes no texto, eles devem entregar o roteiro final, com
todas as alterações e um roteiro de pesquisa, relacionando as
referências bibliográficas, as fontes e os meios utilizados para chegar
ao trabalho final.
\end{enumerate}

\paragraph{Tempo estimado} Entre um a dois bimestres.

\subsection{Leitura II}

\paragraph{Tema} As condições do indígena na história do Brasil e suas
  ressonâncias na contemporaneidade.

%EM13LGG101; EM13LGG201; EM13LP01; EM13LP33; EM13CHS101; EM13CHS102; EM13CHS601

\paragraph{Conteúdo} Pesquisa e estudo das condições do indígena na
história do Brasil e suas ressonâncias nos dias atuais, a partir das
narrativas curtas e poemas de Maria Firmina dos Reis. A atividade
consiste na apresentação dos trabalhos por meio de representações
cênicas, como teatro, cinema, dança, performance ou outras expressões.

\paragraph{Objetivo} Habilitar os estudantes a refletir e discutir sobre a
situação dos indígenas ao longo da história do Brasil até a atualidade.

\paragraph{Justificar} O indianismo foi uma das características do
romantismo brasileiro, que representava o indígena, em geral, de forma
heroicizada, em uma tentativa de criar um mito nacional. Os romances
indianistas de José de Alencar buscaram estabelecer essa relação,
inspirado nas virtudes das personagens dos romances de cavalaria e de
aventura, como em \emph{O guarani}. Outros escritores, como Gonçalves
Dias e Maria Firmina dos Reis, questionaram essa noção de mito de origem
e problematizaram, cada uma a sua maneira, a relação dos ideais da
civilização europeia com os indígenas, o que não representou uma
aproximação harmoniosa.

Quando falamos de povos indígenas, evidentemente, estamos nos referindo
a um número grande de diferentes grupos étnicos, espalhados em todo o
território nacional. Infelizmente, contar a história desses grupos é
também contar o fim da história de muitos deles. A questão em torno da
demarcação de suas terras, ameaçadas desde a era dos descobrimentos, é
não apenas o debate sobre habitação e sustento, mas, sobretudo, de
sobrevivência. Portanto, esta atividade é fundamental para compreender
os lugares dos indígenas na sociedade brasileira e as dinâmicas das
relações com os indivíduos não pertencentes a esses grupos.

\paragraph{Metodologia} 
\begin{enumerate}
\item
Apresente e discuta com os alunos em sala de
aula os aspectos históricos, sociais, econômicos e políticos
relacionados à condição dos povos indígenas no Brasil. Além dos textos
de Maria Firmina dos Reis, comente outras representações literárias e
artísticas sobre o período, como, por exemplo, os quadros de Johann
Moritz Rugendas, Jean-Baptiste Debret e Rodolfo Amoedo. Mostre algumas
delas em sala de aula, apontando as relações entre a linguagem pictórica
e o discurso histórico, problematizando as possibilidades e os limites
de cada um. Compare com as fotografias de Claudia Andujar de sua
exposição ``A luta Yanomami''.


\Image{Território indígena Yanomami (Zeljko; Domínio Público)}{PNLD0014-14.png}

\Image{Localização do território Yanomami (Javierfv1212; CC-BY-SA 3.0)}{PNLD0014-15.png}


\item
Você pode também apresentar e discutir algum filme relacionado ao
tema com eles. Como sugestão, há o filme \emph{Xingu} (2011), dirigido
Cao Hamburguer, ou o documentário \emph{Ex-Pajé} (2018), dirigido por
Luiz Bolegnesi. As questões suscitadas podem contribuir para o
desenvolvimento do trabalho proposto, em diálogo com o texto literário.

\item
Organize a turma em grupos e proponha a adaptação cênica da novela
``Gupeva'' ou do poema ``Por ocasião da tomada de Villeta e ocupação de
Assunção'', de Maria Firmina dos Reis, ou ainda uma combinação dos
textos em prosa e verso. Como na novela há mais de um conflito, eles
podem dividi-la em capítulos ou em episódios de uma série.

\item
Cada grupo pode partir de um personagem específico vinculado ao tema
desta atividade e desenvolver uma narrativa paralela, desde que esteja
relacionada ao texto de origem. Se preferirem, os alunos podem adaptar a
novela integralmente, claro, mas não se pode perder de vista a
intertextualidade e a interdisciplinaridade que as atividades desta
seção propõem. Uma outra sugestão é dialogar comparativamente com obras
e questões contemporâneas, demonstrando as ressonâncias da sociedade
escravocrata nos dias atuais.

\item
Eles podem apresentar o trabalho por meio de qualquer linguagem
cênica: teatro, dança, cinema, performance, \emph{Slams} ou até mesmo
animação.

\item
Não esqueça de pedir a eles o roteiro do trabalho antes de
apresentá-lo, a fim de que possam discutir o texto, o conteúdo e
metodologia utilizada. Depois de apresentarem os trabalhos e realizarem
os devidos ajustes no texto, eles devem entregar o roteiro final, com
todas as alterações e um roteiro de pesquisa, relacionando as
referências bibliográficas, as fontes e os meios utilizados para chegar
ao trabalho final.
\end{enumerate}

\paragraph{Tempo estimado} Entre um a dois bimestres.


\subsection{Pós-leitura}

\paragraph{Tema} Precursoras da educação.

%EM13LGG101; EM13LGG201; EM13LP01; EM13LP33; EM13CHS101; EM13CHS102)}

\paragraph{Conteúdo} Pesquisa e estudo da atuação das mulheres em favor da
educação no Brasil, em especial para a formação de mulheres e de outras
minorias. A atividade consiste na apresentação dos trabalhos no formato
de texto jornalístico, na versão impressa ou digital.

\paragraph{Objetivo} Habilitar os estudantes a refletir e discutir a
importância das mulheres para o acesso universal à educação e o seu
desenvolvimento.

\paragraph{Justificativa} A história da educação no Brasil é marcada por
uma trajetória de avanços fundamentais, mas também de manutenção de
desigualdades profundas, especialmente nas categorias de cor, gênero e
classe. A exclusão das mulheres do sistema educacional durou séculos e
foi revertida a muito custo. Mesmo assim, ainda no século XXI, há muito
a ser discutido e mudado para que a igualdade entre os gêneros seja
plena, não apenas na educação evidentemente. No entanto, é através dela
que as grandes conquistas da sociedade podem ser alcançadas.

Mesmo escritores que dedicaram páginas dos jornais na defesa dos
direitos da mulher e de outras minorias, como Olavo Bilac, reagiam à
tentativa delas em se tornar também escritoras e intelectuais. Em carta
a sua noiva, Amélia de Oliveira, Bilac demonstra desagrado ao saber que
ela havia publicado versos, e avisa que não queria que ela parasse, ao
contrário, ela deveria continuar a escrevê-los, mas para os irmãos,
paras as amigas e, principalmente, para ele. Assim também outras
mulheres foram desencorajadas, bem como os indivíduos pertencentes a
outras minorias.

\paragraph{Metodologia} 
\begin{enumerate}
\item
Apresente um breve panorama da história da
educação no Brasil, citando seus avanços, problemas e principais
personagens. Aproveite a oportunidade para ouvir dos alunos suas
impressões sobre a educação pública brasileira e quais medidas eles
consideram importantes para que ela melhore. Não deixe de perguntar a
eles sobre os avanços que eles perceberam. Se puder, leve alguns dados
de forma bem sintética para que eles possam acompanhar, junto com você,
os principais números que refletem a situação do ensino no país. Compare
com outros sistemas educacionais, mas sempre com cuidado e rigor
metodológico.

\item
Organize a turma em grupos, em que cada um formará uma pequena
redação de jornal. Proponha que as tarefas sejam divididas por função,
mas todos devem estar envolvidos na pesquisa e na divulgação dos
resultados.

\item
O grupo deve pesquisar os nomes de precursoras da educação no Brasil,
principalmente de figuras pouco conhecidas e citadas nesse campo. Uma
delas, por exemplo, pode ser a escritora Júlia Lopes de Almeida, que,
assim como outras escritoras e intelectuais de seu tempo, se dedicaram a
promover a educação para todos, principalmente das mulheres. Maria
Firmina dos Reis, a nossa primeira romancista negra, também teve
importante atuação na área educacional. A ela é atribuída a primeira
escola para meninas e meninos no Brasil. O período histórico pode se
estender até os dias atuais, em perspectiva comparada ou não, citando
exemplos de figuras anônimas que atuam, a contrapelo dos limites
impostos diariamente, para o desenvolvimento da nossa educação.

\item
Depois de definir a educadora, os estudantes jornalistas devem fazer
um levantamento de informações sobre ela e coletar imagens. A primeira
parte do grupo fica responsável pela escrita, revisão e edição do texto,
a segunda cuida da produção, coleta e tratamento das imagens e uma outra
se encarrega de diagramar o material e publicar o jornal. Dependendo do
número de alunos, evidentemente essas tarefas podem se misturar. O
importante é que todos participem da execução do trabalho.

\item
Durante a pesquisa, estimule-os a consultar os arquivos na rede, como
os periódicos mantidos pela Biblioteca Nacional em sua Hemeroteca
Digital, assim como os acervos disponibilizados on-line pelo Arquivo
Nacional. É fundamental que eles não deixem de citar as fontes de
pesquisa.
\end{enumerate}

\paragraph{Tempo estimado} um bimestre de aulas.

\section{Aprofundamento}

É importante ressaltar que as características do Romantismo permeiam a
obra de Maria Firmina. \emph{Úrsula} trata de um amor conturbado entre
dois jovens brancos, no entanto da protagonismo a certos personagens
escravizados, expondo suas reflexões acerca das injustiças presentes em
uma sociedade escravocrata e patriarcal. Questões sociais e
abolicionistas fazem parte da terceira fase do Romantismo, também
conhecida como Condoreirismo. ``Gupeva'', por outro lado, integra
particularidades do indianismo, uma das tendências mais marcantes da
fase romântica. Por fim, grande parte de seus poemas publicados em
\emph{Cantos à beira-mar} exprimem uma inquietação diante do
autoritarismo vigente, fruto do patriarcado escravocrata, e o eu lírico
feminino manifesta a agonia e a melancolia tão presentes na produção do
período romântico.


\SideImage{Folha de rosto da primeira edição do romance Úrsula publicado em 1859. (San' Luiz; Domínio Público)}{PNLD0014-13.png}


No que diz respeito ao contexto histórico, Maria Firmina nasceu no ano
em que o Brasil se tornava independente de Portugal, em meio a uma
sociedade elitista, escravocrata e patriarcal. O estado do Maranhão era
mais um a expressar seu elitismo por meio do acesso limitado ao ensino.
Na época, só havia os cursos de Medicina e Direito, portanto os que
publicavam livros faziam parte de um grupo extremamente restrito. A
maioria dos escritores eram homens brancos, economicamente
privilegiados, com acesso ao estudo das letras e recursos para a
publicação de seus trabalhos. Maria Firmina dos Reis viveu a
Independência do Brasil em 1822, a promulgação da 1ª Constituição em
1824, a~Lei Eusébio de Queiroz de 1850, a abolição da escravidão em
1888, a~Proclamação da República em 1889, assim como todas as mudanças
que surgiram no país e no mundo com a virada do século.

Esta edição é uma miscelânea de gêneros literários praticados por Maria
Firmina dos Reis, entre a prosa e a poesia. Ela reúne os textos, em
prosa, ``A escrava'' e ``Gupeva''; e, em verso, extraídos do livro
\emph{Cantos à beira-mar} (1871) e da coletânea \emph{Parnaso
maranhense} (1861), além do ``Hino à liberdade dos escravos'' (1888).
Apresentamos os textos nessa mesma ordem, sem nenhuma hierarquização
cronológica ou de importância, apenas uma divisão por gênero para que
seja conferida uma coesão estrutural na organização da antologia.

O conto ``A escrava'' foi publicado, pela primeira vez, na edição nº 3
da \emph{Revista Maranhense}, em novembro de 1887. No século XX, vem
sendo editado nas mais diversas antologias sobre a autora, denotando seu
lugar de destaque no conjunto até então conhecido de sua obra e na
história da literatura brasileira. A narrativa tem caráter abolicionista
e marca uma fase mais amadurecida da autora. Ao colocar como
protagonista uma mulher negra escravizada, que fugiu de seu algoz e
relata sua própria história para a narradora, descrita apenas como ``uma
senhora'', Firmina dos Reis constrói um relato não apenas de escravidão,
mas, sobretudo, de resistência e liberdade.

Quanto a ``Gupeva'', trata-se de uma novela publicada, pela primeira vez
e de forma incompleta, em capítulos no semanário \emph{O Jardim
Maranhense}, entre outubro de 1861 e janeiro de 1862. O texto foi
publicado em versão completa e revista em mais dois periódicos pela
escritora, no jornal \emph{Porto Livre}, em 1863, e em \emph{Eco da
Juventude}, em 1865. Ambientada na Bahia, a novela narra a história do
indígena Gupeva e da filha de sua esposa, Épica. Com a morte da mulher
logo após o parto, Gupeva batiza a criança com o mesmo nome da mãe e
passa a cuidar dela como pai. A paixão de Épica, a filha, por um
marinheiro francês traz à tona um passado repleto de conflitos.

O livro de poesia \emph{Cantos à beira-mar} foi publicado, pela primeira
vez, em 1871, pela Typografia do Paiz. Dos 56 poemas, foram selecionados
29 para esta antologia. Nos versos líricos, estão presentes muitos dos
temas relacionados ao romantismo brasileiro, como a exaltação da terra,
o nacionalismo, a idealização e a impossibilidade do sentimento amoroso.
Além disso, há poemas com crítica à sociedade patriarcal e ao papel
destinado às mulheres na sociedade. Há ainda um poema indianista, ``Por
ocasião da tomada de Villeta e ocupação de Assunção'', dialogando com a
novela ``Gupeva''.

Os poemas ``Por ver-te'' e ``Minha vida'' foram transcritos de uma
coletânea com textos de outros poetas da geração de Firmina dos Reis,
como Gonçalves Dias e Sotero dos Reis. O livro \emph{Parnaso maranhense}
foi publicado em 1861 pela Tipografia do Progresso, em São Luís, com
organização de Gentil Homem de Almeida Braga, Antônio Marques Rodrigues,
Raimundo de Brito Gomes de Sousa, Luís Antônio Vieira da Silva, Joaquim
Serra e Joaquim da Costa Barradas. Por fim, temos um dos textos poéticos
mais famosos da autora, o ``Hino à liberdade dos escravos'', gênero
bastante presente em sua obra, composto por ocasião da Abolição da
Escravatura, em 1888.

\section{Referências Complementares}

\subsection{Audiovisual}

\begin{enumerate}
\item
  \emph{Quanto Vale ou é por Quilo?} (2005), direção de Sérgio Bianchi.
\item
  \emph{Emicida: AmarElo -- É Tudo Pra Ontem} (2020), direção de Fred
  Ouro Preto.
\item
  \emph{A Última Abolição} (2018), direção de Alice Gomez.
\item
  \emph{Quilombo} (1984), direção de Cacá Diegues.
\item
  \emph{Xingu} (2011), direção de Cao Hamburguer.
\item
  \emph{Ex-pajé} (2018), direção de Luiz Bolognesi.
\end{enumerate}

\subsection{Literária}

\begin{enumerate}
\item
  ALVES, Castro. \textbf{O navio negreiro e poemas abolicionistas}. São
  Paulo: Companhia Editora Nacional, 2008. (Coleção clássicos da nossa
  língua). (143p.).
\item
  ASSIS, Machado de. Pai contra mãe. In: \textbf{Relíquias de Casa
  Velha}. Rio de Janeiro, H. Garnier Livreiro Editor, 1906.
\item
  EVARISTO, Conceição.~\textbf{Becos da Memória}. Florianópolis: Ed.
  Mulheres, 2013.
\item
  JESUS, Carolina Maria de. \textbf{Quarto de despejo}. 9.ed. São Paulo:
  Ática, 2007.
\end{enumerate}

\subsection{Musical}

\begin{enumerate}
\item
  \emph{Cantos à Beira-mar} (2019), Socorro Lira.
\item
  \emph{Sobrevivendo no Inferno} (1997), Racionais MC's.
\end{enumerate}

\subsection{Outros}

\begin{enumerate}
\item
  MEMORIAL de Maria Firmina dos Reis. Disponível em:
  \url{https://mariafirmina.org.br/}
\item
  Publicações de Maria Firmina dos Reis em \emph{A Pacotilha, Echo da
  Juventude} e \emph{Semanário Maranhense} disponíveis no site da
  Biblioteca Nacional:

  \begin{itemize}
  \item
    \emph{A Pacotilha:}
    \url{http://memoria.bn.gov.br/docreader/DocReader.aspx?bib=168319_01\&pagfis=9008}
  \item
    \emph{Echo da Juventude:}
    \url{http://memoria.bn.gov.br/docreader/DocReader.aspx?bib=738271\&pagfis=48}
  \item
    \emph{Semanário Maranhense:}
    \url{http://memoria.bn.gov.br/docreader/DocReader.aspx?bib=720097\&pagfis=214}
  \end{itemize}
\end{enumerate}

\section{Bibliografia Comentada}

Duarte, Constância Lima; Tolentino, Luana; Barbosa, Maria Lúcia; Coelho,
Maria do Socorro Vieira (Org.).\emph{~Maria Firmina dos Reis}: faces de
uma precursora. Rio de Janeiro: Editora Malê, 2018. Coletânea com
trabalhos reunidos possibilitando um regate da primeira escritora negra
do Brasil. Na obra são utilizados desde estudos mais conhecidos a
pesquisas pouco divulgadas. Uma das produções destacadas é a de Dilercy
Adler que traz a público a correção do equívoco acerca da data de
nascimento de Maria Firmina: 11 de março de 1822.

DUARTE, Eduardo de Assis (Org.). \emph{Literatura e afrodescendência no
Brasil}. Belo Horizonte: UFMG, 2011. Antologia em quatro volumes sobre a
produção de escritoras e escritores afrodescendentes brasileiros; Maria
Firmina dos Reis está presente no primeiro volume ao lado dos autores
nascidos antes de 1930, deslocados, em termos de valores socioculturais,
das elites brancas dominantes.

Gotlib, Nádia Battella. \emph{Teoria do conto}. São Paulo: Ática, 1985.
Excelente introdução aos principais conceitos e discussões teóricas
sobre o gênero conto, abordando as concepções de Vladimir Propp, Edgar
Allan Poe, Júlia Cortázar e outros textos fundamentais.

Montello, Josué. A primeira romancista brasileira, \emph{Jornal do
Brasil}, 11 nov. 1975. Republicado em Madrid, Espanha, com o título
\emph{La primera novelista brasileña}, \emph{Revista Cultura Brasileña},
n. 41, junho 1976, pp. 11-114. Artigo histórico nos quais são citados os
responsáveis pela ``ressurreição'' de Maria Firmina: o historiador
Horácio de Almeida, conhecido por localizar o original \emph{Úrsula} em
um sebo carioca, e o biógrafo Nascimento Morais Filho, autor do resgate
da obra completa de Maria Firmina e da publicação em 1975 de \emph{Maria
Firmina -- fragmentos de uma vida}.

Mott, Maria Lúcia de Barros. Escritoras negras resgatando nossa
história. \emph{Papéis Avulsos}, n. 13. Rio de Janeiro: CIEC/UFRJ, 1989.
Levantamento do século XVII à 1960 da produção de escritoras negras
brasileiras, tais como Rosa Maria Egipiciaca de Vera Cruz, Tereza
Margarida da Silva e Orta e Maria Firmina dos Reis. O objetivo do estudo
é tirar do esquecimento da memória literária no Brasil tantas autoras
ignoradas, apagadas e até hoje, por muitos, desconhecidas.

Muzart, Zahidé Lupinacci (org.). \emph{Escritoras brasileiras do século
XIX}. 2. ed. Florianópolis; Santa Cruz do Sul: Mulheres; EDINISC, 2000.
Coleção em três volumes \emph{que demonstra} integralmente o caráter
masculino e branco do cânone literário brasileiro. Escritoras
esquecidas, ou ignoradas, dos séculos XIX e XX, são organizadas e
classificadas, dentre elas Maria Firmina dos Reis.

Schwarcz, Lilia Moritz; Gomes, Flávio dos Santos. \emph{Dicionário da
escravidão e liberdade}: 50 textos críticos. São Paulo: Companhia das
Letras, 2018. 50 textos críticos escritos por 45 pesquisadores ligados a
diferentes instituições de pesquisa e ensino reunidos em um extenso
panorama de como se cravou, impiedosamente, na sociedade brasileira, a
escravidão. A obra foi publicada a fim de comemorar os 130 anos da
abolição no Brasil.


\end{document}

