\documentclass[12pt]{extarticle}
\usepackage{manualdoprofessor}
\usepackage{fichatecnica}
\usepackage{lipsum,media9,graficos}
\usepackage[justification=raggedright]{caption}
\usepackage[one]{bncc}
\usepackage[araucaria]{../edlab}

% Puxar para o Machado. Lembrar o Sterne e do Luciano. Talvez algo sobre sátira
% http://www.scielo.br/scielo.php?script=sci_arttext&pid=S1983-68212014000100006
% Quando for falar do gênero lembrar das sátiras latinas (e do sátiro!)
% Não tem sátiro na grécia! 
% Todos ficam dando viagens em volta do seu quarto durante uma pandemia. 
% Quarto é o lugar de reflexão (filosofia do quarto = Descartes)
% Puxar para um "cogito ergo sum"

\begin{document}


\newcommand{\AutorLivro}{Xavier de Maistre}
\newcommand{\TituloLivro}{Viagem em volta do meu quarto}
\newcommand{\Tema}{Ficção, mistério e fantasia}
\newcommand{\Genero}{Romance}
\newcommand{\imagemCapa}{./images/PNLD0046-01.png}
\newcommand{\issnppub}{---}
\newcommand{\issnepub}{---}
% \newcommand{\fichacatalografica}{PNLD0046-00.png}
\newcommand{\colaborador}{\textbf{Alessandra Cabral, Bruno Gradella e Vicente Castro} é uma pessoa incrível e vai fazer um bom serviço.}


\title{\TituloLivro}
\author{\AutorLivro}
\def\authornotes{\colaborador}

\date{}
\maketitle

\baselineskip=1.20\baselineskip\par


\begin{abstract}
Este Manual tem como objetivo fornecer subsídios para o trabalho com a
obra literária \emph{Viagem à Volta do Meu Quarto \& Expedição Noturna
em Volta do Meu Quarto e outros contos}, de Xavier De Maistre.

\textbf{Xavier de Maistre} (Chambéry, 1763--São Petersburgo, 1852)
nasceu em família nobre, na região da Savóia que hoje faz parte da 
França, e teve uma educação refinada. A condição de nobreza influenciou 
negativamente sua vida por conta dos ideais então em voga da Revolução Francesa
Mas isso não o impediu de produzir. Embora a pintura tenha sido sua grande paixão, 
foi a literatura, e especialmente a \textit{Viagem em volta do meu quarto} 
e sua continuação, a \textit{Expedição noturna em volta do meu quarto}, 
que fez com que alcançasse fama e reconhecimento. Sua \textit{Viagem\ldots} é 
lembrada por Machado de Assis na dedicatória do seu \textit{Memórias póstumas de Brás Cubas}.   

\textbf{Viagem em volta do meu quarto} foi publicado pela primeira vez
no final do século \textsc{xviii}. O autor, em situação de prisão domiciliar,
resolve fazer da escrita do texto uma oportunidade de se locomover. Tudo vira 
assunto para o narrador: desde as questões políticas da Revolução Francesa, 
mulheres, e a própria exploração de sua individualidade. O tom é bastante irônico, 
mas a exploração psicológica da consciência, juntamente com um estilo que cria 
digressões intermináveis aparentemente sem objetivo, aproximam-no do \textit{\mbox{A vida} 
e as opiniões do cavalheiro Tristram Shandy} (1767) de Sterne, a grande influência desse texto. 

Ja a \textbf{Expedição noturna em volta do meu quarto} foi publicado no
começo do século \textsc{xix}, e, por isso mesmo, ter mudanças significativas
em relação ao primeiro. Já sob a influência do romantismo, o tom é bem mais leve
e menos entrucado do que a \textit{Viagem}. O enredo, porém, segue o mesmo: 
confinado no próprio quarto, o narrador faz expedições pelo seu próprio interior. 

Além de serem obras deslumbrantes por si só, a influência delas sobre um dos 
maiores escritores brasileiros também estudado no Ensino Médio, Machado de Assis, 
é motivo de sobra para que essa leitura seja muito aproveitada em sala de aula!

Esperamos, assim, que as indicações propostas aqui sejam muito úteis nesse trabalho!

\end{abstract}

\tableofcontents


\section{Introdução}

E foi assim que ele ficou famoso por criar uma nova modalidade de turismo: a viagem pelo quarto.
Xavier de maistre, um oficial francês de 27 anos que, em 1790, ficou preso num quarto, por seis semanas.

\textit{Viagem ao redor do meu quarto} e sua continuação, o livro 
\textit{Expedição noturna ao redor do meu quarto}, narram as andanças de uma pessoa presa num único ambiente. 
Xavier faz uma paródia das grandes jornadas que seus contemporâneos escreviam, 
mostrando que mesmo com pouco espaço, a imaginação pode viajar livremente.

\begin{quote}
Logo no capítulo de abertura, de Maistre convida: “todo homem sensato, tenho certeza, irá adotar meu sistema. 
Quaisquer que sejam suas características peculiares ou temperamento. Seja ele miserável ou pródigo, rico ou pobre, jovem ou velho, nascido numa zona tórrida ou próximo aos polos, ele poderá viajar comigo. Entre a imensa família de homens que lotam a terra, não há um, não, nenhum (quero dizer, daqueles que habitam salas), que, depois de ler este livro, pode recusar sua aprovação do novo modo de viajar que apresento ao mundo.
\end{quote}

Ao longo dos 42 curtos capítulos, o escritor francês descreve sua cama, 
seus livros e suas gavetas. Também fala de seus quadros, suas relações com o criado e 
sua cachorra, faz reflexões filosóficas sobre a vida, relembra histórias do passado. 
Tudo com o bom humor e a ironia de quem não pode sair de casa: “O mais preguiçoso dos 
seres não vai ter mais nenhuma razão para hesitar antes de partir em busca de prazeres que não vão lhes custar dinheiro ou esforço.”

E por que sua jornada dura exatamente 42 dias? Ele conta aos leitores que nem ele sabe responder a essa pergunta. A verdade é que Xavier de Maistre encontra-se em prisão domiciliar em Turim, no norte da Itália, por causa de sua participação em um duelo. 

O livro acabou tornando-se um grande sucesso, que influenciou até mesmo Machado de Assis, 
que o menciona na abertura de memórias póstumas de \textit{Brás Cubas}.

Ha de se perguntar aos alunos com seria possível viajar pelo quarto ou pela casa como Xavier. 
Ele deixa dicas sobre os rumos de sua jornada: “quando viajo pelo meu quarto, 
raramente fico em linha reta. Da minha mesa, vou em direção a uma foto colocada 
em um canto; daí parto numa direção oblíqua para a porta.
E então, embora, ao iniciar, pretendesse retornar à minha mesa, ainda assim, se por acaso caísse com minha poltrona no caminho, imediatamente, e sem a menor cerimônia, assento-me nela. A propósito, que artigo de mobília importante uma poltrona é, e, acima de tudo, quão conveniente para um homem pensativo.”

Ele também reflete sobre os pequenos prazeres em seu espaço fechado.

Um aspecto interessante a destacar corresponde à tematização da noção de tédio. De acordo com o primeiro capítulo de a \textit{Viagem à volta o meu quarto}, o relato que o autor oferece da viagem no quarto proporciona um meio seguro contra o tédio, o que parece apontar para uma diferença relativamente ao pensamento pascaliano. Enquanto que, para o filósofo francês Blaise Pascal (1623-1662), ficar no quarto em repouso se constitui como uma fonte de tédio, para Xavier de Maistre ficar confinado num quarto constitui-se como fonte de divertimento, 
podendo, dessa forma, configurar-se como um modo de combater o tédio. 

Porém, para se compreender de que forma o confinamento num quarto se constitui como fonte de divertimento, é necessário ter em consideração a caracterização que Xavier de Maistre apresenta da natureza humana enquanto uma natureza dupla no capítulo 6 da \textit{viagem}. De acordo com este capítulo, o ``homem é duplo'', correspondendo essa duplicidade do homem àquilo que xavier de maistre denomina de ``sistema da alma e do animal''. 

Tendo em conta todas estas considerações, viajar à volta do quarto, isto é, ser capaz de produzir divertimento no confinamento do quarto, significa libertar a alma dos
automatismos que prendem o animal à vida quotidiana, fazendo com que a alma se
possa elevar acima da quotidianidade, ao mesmo tempo que o animal caminha autonomamente no campo das solicitações que o ocupam.

A distinção entre alma e animal apresentada no capítulo VI de \textit{Viagem à volta do meu quarto} 
não deve ser confundida com a distinção moderna entre corpo e alma. 

O termo \textit{sátira menipeica} é geralmente utilizado por gramáticos clássicos e filólogos 
para diferenciar as sátiras em prosa, por oposição às sátiras em verso feitos pelo poeta 
latino Juvenal (55-127) e imitadores. As atitudes mentais típicas ridicularizadas pelas sátiras menipeias são os "pedantes, os sectários, os excêntricos, os arrivistas. O termo se diferencia da sátira praticada anteriormente por aristófanes, por exemplo, que era baseada em ataques pessoais. 

O gênero é importante para escritores como o russo Mikhail Bakhtin, que o considera uma das origens do romance polifônico.  A sátira menipeica teria influenciado vários autores posteriores, como 
François Rabelais (1494--1553), 
Erasmo de Roterdã (1466--1536), 
Voltaire (1694--1778), entre outros.
Na sátira menipéia, desaparecem todos os resquícios das barreiras hierárquica, social, etária, sexual, religiosa, ideológica, nacional, linguística etc. Tudo é alvo de rebaixamento grosseiro e inversões ousadas, nas quais os momentos elevados do mundo aparecem às avessas, com uma faceta oposta àquela em que antes se manifestavam.

É muito fácil deixar de apreciar o que é comum e familiar. Segundo o autor da arte de viajar, após nos habituarmos a um lugar, nos tornamos cegos aos seus atrativos. Caímos no hábito de considerar chato um universo perfeitamente alinhado às nossas expectativas.

Quantas vezes já não admiramos durante uma viagem algo que na sua própria cidade considera irrelevante? Quantas vezes tentou passear pelo seu próprio bairro com o mesmo olhar inocente e curioso que tem como turista numa cidade estrangeira? e alguma vez já tentou olhar para a sua própria casa com o mesmo olhar de admiração e prazer que sente em um hotel num local desconhecido, rever os cantinhos e momentos prazerosos que o lar te traz? Quanto a reclusão se faz obrigatória, como e tempos de pandemia, nada mais útil do que isso. 



\section{Atividades 1}

%\BNCC{EM13LP26}

\subsection{Pré-leitura}

%\BNCC{EM13LGG302}
%\BNCC{EM13LGG704}
%\BNCC{EM13LP10}
%\BNCC{EM13LP19}

% \paragraph{Tema}
% \paragraph{Conteúdo}
% \paragraph{Objetivo}
% \paragraph{Justificativa}
% \paragraph{Metodologia}
% 	\begin{enumerate}
% 	\end{enumerate}
% \paragraph{Tempo estimado}

\paragraph{Tema} Viagem dentro do confinamento. 

\paragraph{Conteúdo} Escrita criativa. 

\paragraph{Objetivo} Inspirados pelos objetos, animados ou não, 
presentes nos quartos dos e das estudantes, propor um exercício 
de escrita criativa que emule a experiência do autor do livro
\textit{Viagem à volta do meu quarto}.

\paragraph{Justificativa}
O filósofo francês René Descartes, no livro \textit{O Discurso sobre o Método}, 
elenca o quarto como o lugar onde paramos para testar a solidez de nossas certezas. 
Este teste deve ser o exercício reflexivo de nossa razão sobre si mesma. É assim
que tomamos contato conosco e construimos aquilo que queremos ser. 

Atualmente boa parte do mundo ainda vive, sem escolha, dentro de suas
casas e de seus quartos. A pandemia do COVID"-19 obrigou a humanidade
a fechar"-se num quarto, no mesmo movimento de Descartes, e olhar para  
si mesma, e organizar o caminho que deseja seguir a partir de então. 

Também Xavier de Maistre estava numa situação similar quando da 
escrita de seu romance \textit{Viagem à volta de meu quarto}. 
Num momento em que estava impossibilitado de se locomover pelos
lugares, o autor encontrou na viagem introspectiva um refúgio. 

\paragraph{Metodologia}
\begin{enumerate}
	\item
	Antes de iniciar a leitura, proponha uma reflexão sobre
experiências de isolamento e confinamento em pequenos espaços. A obra em
questão trata do percurso de um narrador pelo espaço reduzido do
interior do próprio quarto. Com base nas experiências dos alunos com
estudo em casa e, provavelmente, longos períodos recentes de isolamento
social, sugira a produção individual de uma crônica escrita
ambientada no interior do quarto. Esse espaço, considerado simbólico do
ponto de vista da intimidade e da formação da subjetividade, abriga
geralmente objetos muito particulares, característicos da identidade de
quem o habita.
	\item
	A partir do exercício clássico da écfrase, que se caracteriza por uma
	produção escrita a partir de uma imagem, os e as estudantes devem, primeiro,
	criar pequenos textos sobre cada objeto. Espera"-se que deste próprio
	exercício saiam as reflexões e relatos que irão compor o texto final.
	Um objeto pode acionar uma lembraça ou uma vontade que traz para a escrita
	uma narrativa. 
	\item 
	O exercício deve ser feito durante todo o bimestre, ou enquanto
	o romance estiver sendo trabalhado em sala de aula.

\end{enumerate}

\paragraph{Tempo estimado} Um bimestre ou o tempo que durar a leitura do livro.



\subsection{Leitura}

%\BNCC{EM13LGG103}
%\BNCC{EM13LP02}
%\BNCC{EM13LP48}

\paragraph{Tema} Brás Cubas à volta de meu quarto.

\paragraph{Conteúdo} Comparação entre alguns traços das obras de Machado de Assis e 
Xavier de Maistre.

\paragraph{Objetivo} Habilitar os alunos a perceber relações entre as obras dos 
dois escritores, sendo a de Xavier de Maistre assumidamente inspiração
para a do autor brasileiro.

\paragraph{Justificativa} Nas primeiras linhas do romance \emph{Memórias póstumas de Brás Cubas},
o narrador de Machado de Assis cita \emph{Viagens à volta de meu quarto} de Xavier de Maistre.
Se neste caso o narrador faz suas viagens sem sair de seu quarte, apesar fazendo uso das
abstrações da mente, a personagem de Machado de Assis fará o mesmo, num contexto
algo diferente: ele não realiza viagens físicas pois está morto e 
rememora sua vida. 
Levando em conta que a relação com este que é um dos mais famosos
romances brasileiros é que fez circular o nome e a obra de Xavier de Maistre, 
ao menos no Brasil, a articulação as duas no trabalho em sala de aula 
mostra"-se um caminho seguro e profícuo.

\paragraph{Metodologia}
\begin{enumerate}
	\item
	O professor deve iniciar a aula com a leitura dos primeiros capítulos das obras
	\emph{Viagens à volta de meu quarto} e \emph{Memórias póstumas de Brás Cubas}.

	\item
	Após a leitura, o professor deve fazer perguntar norteadoras como:
	quais semelhanças existem entre as duas obras, no que se refere à forma
	(tamanho, estilo, registro da língua) e ao conteúdo. O narrador
	é em primeira ou terceira pessoa?

	\item
	Em seguida, o professor deve orientar a leitura de mais três capítulos de cada
	obra, que deverá ser acompanhada de uma ficha de leitura onde os alunos 
	indicarão as principais semelhanças e diferenças entre cada uma. 
	Além disso, devem fazer um comentário crítico sobre a leitura.	
	Os resultados serão compartilhados na aula seguinte. 

\end{enumerate}

\paragraph{Tempo estimado} Quatro aulas de 50 minutos.



\subsection{Pós-leitura}

%\BNCC{EM13LGG102}
%\BNCC{EM13LGG303}
%\BNCC{EM13LGG402}
%\BNCC{EM13LGG703}
%\BNCC{EM13LP13}
%\BNCC{EM13LP14}
%\BNCC{EM13LP28}
%\BNCC{EM13LP29}
%\BNCC{EM13LP52}


\paragraph{Tema} Sátira menipeia.

\paragraph{Conteúdo} Compreensão do gênero sátira menipeia e as figuras de linguagem
que o compõem.

\paragraph{Objetivo} Apresentar aos alunos as principais características
da sátira menipeia a partir de exemplos do presente livro e de outras obras
da literatura universal.

\paragraph{Justificativa} Uma das semelhanças entre as obras de Xavier de Maistre 
e Machado de Assis que podem ter sido encontradas na atividade
anterior é justamente a presença da ridicularização por meio de figuras de 
linguagem como a ironia, por exemplo, em relação a certos comportamentos e tipos humanos. 
Geralmente associadas aos tipos pedantes, excêntricos,
alpinistas sociais, enfim, todos aqueles que demonstravam um desvio de alguma forma
das normas aceitas, esta prática, na literatura, é conhecida como 
\textbf{sátira menipeia}. Este recurso influenciou e foi usado por
escritores como Erasmo de Roterdã e Voltaire, além dos já citados.

\paragraph{Metodologia}
\begin{enumerate}
	\item
	Os alunos devem realizar uma pesquisa acerca da sátira menipeia.
	Esta pesquisa será dividida em duas partes: primeiro, buscarão
	definições acerca do termo ``sátira'' e sua relação com a figrua
	dos sátiros das mitologias gregas e romanas. 
	Depois, especificamente sobre o recurso da sátira menipeia,
	os principais autores que fizeram uso dela e as obras. 
	É importante ser ressaltado que a pesquisa deve ir mesmo a obras
	da Antiguidade clássica greco"-romana, onde ela, inclusive, se originado.
\end{enumerate}

\paragraph{Tempo estimado} Duas aulas de 50 minutos.

\section{Atividades 2}

%\BNCC{EM13CNT201}
%\BNCC{EM13CNT303}
%\BNCC{EM13CHS101}
%\BNCC{EM13CHS102}
%\BNCC{EM13CHS106}
%\BNCC{EM13CHS401}


\subsection{Pré-leitura}


\paragraph{Tema} Viagens virtuais.

\paragraph{Conteúdo} 

\paragraph{Objetivo} Proporcionar, por meio da relação entre literatura e cinema, 
uma reflexão acerca das linguagens artísticas como formas de expandir as realidades,
levando em consideração o contexto de isolamento social provocado pela pandemia da covid-19.

\paragraph{Justificativa} A solidão e o isolamento, resultantes de uma intensificação
da chamada vida virtual, já vinham sendo classificados como o mal do
século XXI. Não obstante, o quadro da pandemia agravou a situação,
colocando as pessoas em home office e aulas online, o intuito de
preservação da vida, por meio de um controle de disseminação da
covid-19, ocasionou o efeito colateral da solidão, da tristeza, da
angústia, da ansiedade, entre outros, em muitas pessoas.

Isolamentos não são novidades para a humanidade, no entando. Xavier de 
Maistre, impossibilitado de sair de sua casa, inventou um estilo
literário que lhe possibilitou viajar sem sair de casa no romance \emph{Viagens à volta do meu quarto.}
Já no filme \emph{Ela}, de 2013, um escritor solitário se apaixona pelo 
sistema operacional de seu computador.
Em ambos os casos, necessidades humanas (viajar e amar) são
supridas, de alguma forma, por um recurso virtual: a literatura, no primeiro caso,
e a tecnologia informática, no segundo. 

\paragraph{Metodologia}
\begin{enumerate}
	\item
	Antes da exibição do filme, o professor deve fazer uma roda de conversa acerca 
	da realidade vivida em tempos de isolamento social decorrida da pandemia 
	de covid-19. Incluindo na discussão os recursos utilizados pelas
	pessoas, dentre as quais os próprios alunos que devem compartilhar suas
	experiências, para satisfazer as necessidades humanas nesta situação, 
	deve se introduzir o contexto do romance \emph{Viagens à volta do meu quarto},
	onde o narrador, impossibilitado de sair de casa, faz uma viagem 
	mental dentro de seu próprio quarto. 
	\item
	Em seguida à discussão e apresentação geral do presente livro, a turma deve assistir 
	ao filme \emph{Ela}, de 2013, do diretor Spike Jonze, preferencialmente no horário de aula. 
	\item
	Por fim, os alunos, alimentados pela discussão e pelas referências, devem produzir 
	um ensaio acerca do tema. O texto poderá ser compartilhado num jornal ou revista
	coletivo criado pelos próprios alunos afim de divulgar as produções 
	com o resto da escola.
\end{enumerate}

\paragraph{Tempo estimado} Quatro aulas de 50 minutos.


\subsection{Leitura}

\paragraph{Tema} Mapeando as viagens.

\paragraph{Conteúdo} Oficina de leitura e criação de mapa virtual.

\paragraph{Objetivo} Proporcionar aos alunos a criação de um mapa virtual
a partir da leitura o livro \emph{Viagens à volta do meu quarto}.

\paragraph{Justificativa} Em tempos de pandemia global, onde o isolamento
social é uma realidade, a Internet e seus recursos mostraram"-se 
muito eficientes para garantir a realização de necessidades básicas
como a comunicação humana. Mas além dos aplicativos e redes sociais,
outras ferramentas possibilitam atividades impossíveis neste contexto
como a viagem a outras cidades, regiões e mesmo países. 

Com o uso de ferramentas gratuitas como o \emph{Google Maps},
ou o \emph{Google Arts and Culture}, é possível se fazer
visitas a quaisquer lugares do mundo, e mesmo entrar em museus e 
acessar todo seu acervo artístico.

\paragraph{Metodologia}
\begin{enumerate}
	\item
	Com o auxílio do professor de Geografia, os alunos devem realizar uma
	atividade de mapeamento a partir dos lugares citados pelo 
	narrador de \emph{Viagens à volta do meu quarto} e \emph{Expedição noturna em volta do meu quarto}.

	\item
	A turma deve dividir"-se em grupos que ficarão responsáveis por um número
	equivalente de capítulos do livro. Cada grupo deve ler estes capítulos
	atendo"-se aos lugares citados pelo narrador. 

	\item
	Após a leitura, com o auxílio do professor e do recurso do Google \emph{My Maps}, 
	devem criar um único mapa para a turma inteira onde estejam presentes
	todas as indicações de lugares que aparecem nos romances. É importante que haja
	uma organização interna em cada grupo que indique quem ficará responsável
	pela articulação com os outros grupos e o professor na hora prática de 
	criar o mapa.

	Um manual \emph{online} sobre a criação de mapas no recurso \emph{My Maps}
	pode ser acessado em: \href{https://support.google.com/mymaps/topic/3024924?hl=pt-BR}{Google My Maps}

\end{enumerate}

\paragraph{Tempo estimado} Quatro aulas de 50 minutos.


\subsection{Pós-leitura}


\paragraph{Tema} O ``quarto'' ao longo do tempo e nas diferentes culturas.

\paragraph{Conteúdo} 

\paragraph{Objetivo} Incentivar uma pesquisa sobre os espaços para dormir 
no decorrer da história, levando em conta as diferenças entre as culturas.

\paragraph{Justificativa} O conceito que nós, leitores brasileiros,
temos de ``quarto'' é o mesmo que o escritor francês tinha 
quando escreveu \emph{Viagens à volta do meu quarto.} Numa definição genérica,
podemos dizer que o quarto é um espaço dentro de uma casa onde as pessoas dormem
e geralmente ficam sozinhas. 
Dentro das culturas onde temos 

\paragraph{Metodologia}
\begin{enumerate}
	\item
	\item
\end{enumerate}

\paragraph{Tempo estimado}


\section{Aprofundamento}

Ao chegar ao Ensino Médio, é necessário que os estudantes se aprofundem
na compreensão das múltiplas linguagens e, sobretudo, da linguagem
literária. Em relação à literatura, a BNCC traz as seguintes
considerações:

\Image{Gravura de Xavier de Maistre feita por Cyprien Jacquemin no século XIX (Cyprien Jacquemin; Domínio Público)}{PNLD0046-03.png}
\begin{quote}
`{[}...{]} a leitura do texto literário, que ocupa o centro do trabalho
no Ensino Fundamental, deve permanecer nuclear também no Ensino Médio.
Por força de certa simplificação didática, as biografias de autores, as
características de épocas, os resumos e outros gêneros artísticos
substitutivos, como o cinema e as HQs, têm relegado o texto literário a
um plano secundário do ensino. Assim, é importante não só (re)colocá-lo
como ponto de partida para o trabalho com a literatura, como
intensificar seu convívio com os estudantes. Como linguagem
artisticamente organizada, a literatura enriquece nossa percepção e
nossa visão de mundo. Mediante arranjos especiais das palavras, ela cria
um universo que nos permite aumentar nossa capacidade de ver e sentir.
Nesse sentido, a literatura possibilita uma ampliação da nossa visão do
mundo, ajuda-nos não só a ver mais, mas a colocar em questão muito do
que estamos vendo/vivenciando. (Brasil, 2018, p. 491)
\end{quote}

Nesta seção, desenvolvemos um trabalho de aprofundamento que, em diálogo
com a formação continuada de professores, oferece subsídios para a
abordagem do texto literário.

\subsection{A Obra}


\Image{Folha de rosto da obra, original da edição de Dufart, Paris, 1796. (Gallica; Domínio Público)}{PNLD0046-05.png}


\emph{Viagem à Volta do Meu Quarto} \& \emph{Expedição Noturna em Volta
do Meu Quarto} são obras escritas em momentos diferentes, mas se
completam. A narrativa é a viagem realizada pelo narrador dentro de seu
próprio quarto durante 42 dias. Viagem é utilizado como uma metáfora
para definir o mergulho intelectual que ocorre e para criar a paródia
entre as literaturas de aventura que quase sempre contemplam uma viagem.

Xavier De Maistre, o autor citado no início da obra \emph{Memórias
Póstumas de Brás Cubas,} é o criador e narrador dessas aventuras
psicológicas. O narrador que nos conduz pela narrativa em
\emph{Viagem...} se apresenta de forma extrovertida e assim como os
autores contemporâneos do século XVIII, nos envolve em reflexões
filosóficas sobre a sociedade e sobre os homens.




Quando chegamos no século XIX, temos a continuação dessa jornada em
\emph{Expedição Noturna...,} mas agora com um narrador introspectivo e
com um foco maior no debate sobre a individualidade do homem e com
alguns novos elementos acrescentados à sua viagem.

Mais importante do que o destino é a jornada, essa é a ideia expressa na
obra de De Maistre. Preso em seu próprio quarto, o narrador busca
maneiras de realizar suas viagens intelectuais através dos objetos ali
presentes e sempre comparando suas descobertas com as grandes
explorações. De Maistre utiliza a metáfora do encontro com o novo e o
desconhecido para indicar o encontro com a própria consciência.


\Image{A temática do mergulho em seu próprio quarto também foi explorada por outros artistas, como Van Gogh em seu famoso quadro "Quarto em Arles", de 1888. (Museu do Van Gogh; CC-BY-SA 2.0)}{PNLD0046-07.png}


Em \emph{Expedição Noturna...} as novas perspectivas, envolvendo os
novos elementos inseridos na paisagem, acarreta novos debates e a
mudança do estilo literário entre a primeira publicação e a segunda é
muito bem definida durante a narrativa, uma vez que \emph{Expedição
Noturna}...é muito mais objetiva do que a \emph{Viagem...} e expressa
muito bem as características do estilo romântico presente no século XIX.

\subsection{O homem como protagonista do século xix}

Autores como Dostoievski e Liev Tolstói apresentam em suas narrativas o
homem como o objeto de análise. Inserido em uma sociedade burguesa e
pós-revolução Francesa, os personagens são homens que comentem crimes
por se sentirem extraordinários como é o caso de Raskólnikov em
\emph{Crime e Castigo.} E os homens vivem em uma sociedade de aparências
como é a narrativa de \emph{Ana Karenina} e seu marido, por Tostói.


\Image{Pintura de Nikolay Karazin baseada no livro "Crime e Castigo"  de Dostoievski. O homem como objeto de análise é um tema comum da narrativa dessa obra e de "Viagem ao redor do meu quarto".  (Nikolay Karazin; Domínio Público)}{PNLD0046-08.png}


\Image{Folha de rosto da primeira edição do livro "Anna Karenina", de Tolstói, no original russo. O homem como objeto de análise e sua vivência em uma sociedade de aparências é um dos temas presentes.  (Manhattan Rare Books; Domínio Público)}{PNLD0046-09.png}


Na obra de De Maistre o homem sozinho dentro de seu próprio quarto é o
objeto de análise principal de um mergulho dentro de si. Um narrador que
embarca em uma discussão filosófica junto ao leitor e possui um humor
irônico durante a sua jornada. Se em um primeiro momento, temos um
narrador detido em um quarto por razão policiais, quando iniciamos a
leitura de Expedição Noturna...o homem se encontrar no quarto por livre
e espontânea vontade e agora na companhia de uma janela.

Durante sua jornada pelo quarto, o narrador divaga sobre a importância e
o sentido de uma poltrona ou até mesmo de uma cama. Conforme segue o
diálogo com o leitor, ele apresenta a teoria que os homens são compostos
de \emph{uma alma e uma besta} e ele faz um alerta, para entender sua
narrativa é necessário entender essa teoria. Uma teoria criada por
Platão que acredita que o homem tem dentro de si dois seres e um domina
o outro em determinadas situações.

\subsection{O leitor amigo}

Machado de Assis se inspirou em De Maistre para travar o seu diálogo com
o leitor. O narrador em \emph{Viagem...} inicia a narrativa explicado ao
leitor sobre o que será essa viagem e não sabe explicar muito bem a
razão pelo qual essa jornada durou 42 dias.


\Image{Folha de rosto de "Memórias Póstumas de Brás Cubas" dedicada por Machado de Assis para a Biblioteca Nacional. Logo no início do livro, Machado faz referência à Xavier de Maistre, em quem se inspirou na maneira de dialogar com o leitor. (Biblioteca Nacional; Domínio Público)}{PNLD0046-10.png}


Ele compartilha suas teorias para que o leitor possa ter uma breve ideia
do que esperar pela frente e inclusive de como pensar. O narrador não
somente compartilha a teoria como também expõe em exemplos pessoais como
funciona o domínio da besta e da alma.

A narrativa se desenvolve como bate papo, estilo que também podemos
encontrar na obra \textit{Grande Sertão: Veredas} de Guimaraes Rosa. De
Maistre inicia o relato sobre sua teoria, após interromper sua divagação
sobre os aspectos positivos de sua poltrona e sua cama. Antes de todos
esses relatos, ele realiza um convite a todos os leitores, sejam eles
preguiçosos, sem dinheiro ou presidiários.


\Image{O estilo da narrativa que se desenvolve como um bate-papo também é encontrada em Grande Sertão: Veredas de Guimarães Rosa. Nessa foto, é possível ver o autor em uma das viagens que fez ao sertão em 1952. (Eugênio Silva; CC0)}{PNLD0046-11.png}


\subsection{Por que ler Viagem à volta do meu quarto \& Expedição noturna em
volto do meu quarto?}

\emph{Viagem à Volta do Meu Quarto} é uma obra que nos convida a
mergulhar nos detalhes daquilo que nos rodeia. A metáfora da viagem para
conseguir mergulhar de forma filosófica através dos objetos nos
sentimentos e sensações humanas é muito bem trabalhada.

Em primeiro momento temos apenas teorias e objetos para serem
contemplados, mas quando mergulhamos na \emph{Expedição Noturna}...nos
deparamos com uma janela e através dela a figura de uma mulher, a
vizinha de De Maistre. Um cenário que poderia ser o quarto de qualquer
leitor.

A sensação de se deparar com Machado ou Dostoievski não é mera
coincidência, através das teorias sobre os homens e o diálogo com o
leitor. Dessa forma, convido a todos vocês a mergulharem nesse quarto e
descobrirem essa viagem.


\Image{Autorretrato do autor, de aprox.1825 (Xavier de Maistre; Domínio Público)}{PNLD0046-04.png}


\Image{Estátua em homenagem as irmãos Joseph e Xavier de Maistre, localizada no castelo de Chambéry, na França. (Florian Pépellin; CC-BY-SA 4.0)}{PNLD0046-06.png}


\section{Sugestões de atividades complementares: relações dialógicas e
intertextuais}

%\BNCC{EM13LP03}
%\BNCC{EM13LP04}
%\BNCC{EM13LP49}
%\BNCC{EM13LP51}

No Ensino Médio, da mesma forma que no Ensino Fundamental, a \textsc{bncc}
organiza o trabalho com as práticas de linguagem em cinco \textbf{campos
de atuação social}. São eles: campo da vida pessoal, campo da vida
pública, campo jornalístico"-midiático, campo artístico"-literário e campo
das práticas de estudo e pesquisa.

De acordo com essa divisão, propomos na sequência um trabalho
interdiscursivo e intertextual com a obra \emph{Viagem à Volta do Meu
Quarto \& Expedição Noturna em Volta do Meu Quarto e outros contos}.

\subsection{Campo da vida pessoal}

\begin{quote}
O campo da vida pessoal pretende funcionar como espaço de articulações
e sínteses das aprendizagens de outros campos postas a serviço dos
projetos de vida dos estudantes. As práticas de linguagem privilegiadas
nesse campo relacionam"-se com a ampliação do saber sobre si, tendo em
vista as condições que cercam a vida contemporânea e as condições
juvenis no Brasil e no mundo.

Está em questão também possibilitar vivências significativas de práticas
colaborativas em situações de interação presenciais ou em ambientes
digitais e aprender, na articulação com outras áreas, campos e com os
projetos e escolhas pessoais dos jovens, procedimentos de levantamento,
tratamento e divulgação de dados e informações e o uso desses dados em
produções diversas e na proposição de ações e projetos de natureza
variada, para fomentar o protagonismo juvenil de forma
contextualizada. (\textsc{bncc}, p. 494)
\end{quote}

A experiencia de De Maistre durou 42 dias e a primeira parte da
narrativa foi escrita enquanto ele se encontrava detido em Turim,
Itália. Conforme ele se locomove pelo quarto e observa os objetos, ele
se aprofunda no significado das sensações e do conforto que cada
objeto nos oferece. Proponha para aos alunos a criação de um pequeno
diário que eles possam escrever a partir das sensações que eles tenham
em relação a própria juventude. A ideia é estimular o pensamento
crítico sobre si mesmo e sobre a sociedade ao redor.


\subsection{Campo de atuação na vida pública}

\begin{quote}
No cerne do campo de atuação na vida pública estão a ampliação da
participação em diferentes instâncias da vida pública, a defesa dos
direitos, o domínio básico de textos legais e a discussão e o debate de
ideias, propostas e projetos. {[}\ldots{}{]}

Ainda no domínio das ênfases, indica"-se um conjunto de habilidades que
se relacionam com a análise, discussão, elaboração e desenvolvimento de
propostas de ação e de projetos culturais e de intervenção social.
(\textsc{bncc}, p. 494)
\end{quote}

Machado de Assis se inspirou em De Maistre para escrever
\emph{Memórias Póstumas de Brás Cubas.}Brás Cubas trava um diálogo com
o leitor, assim como em \emph{Viagem à Volta do Meu Quarto}, o
narrador está constantemente conversando e convencendo o leitor sobre
alguma teoria. O exercício de literatura comparada é um estímulo para
que os alunos consigam encontrar semelhanças entre os textos, não
apenas absorvendo conteúdo, mas assimilado. Proponha a leitura da obra
\textbf{Memórias Póstumas de Brás Cubas}, Machado de Assis, e a
elaboração de uma resenha com as semelhanças entre as narrativas.

\subsection{Campo jornalístico"-midiático}

\begin{quote}
Em relação ao campo jornalístico"-midiático, espera"-se que os jovens
que chegam ao Ensino Médio sejam capazes de: compreender os fatos e
circunstâncias principais relatados; perceber a impossibilidade de
neutralidade absoluta no relato de fatos; adotar procedimentos básicos
de checagem de veracidade de informação; identificar diferentes pontos
de vista diante de questões polêmicas de relevância social; avaliar
argumentos utilizados e posicionar"-se em relação a eles de forma ética;
identificar e denunciar discursos de ódio e que envolvam desrespeito aos
Direitos Humanos; e produzir textos jornalísticos variados, tendo em
vista seus contextos de produção e características dos gêneros. Eles
também devem ter condições de analisar estratégias
linguístico"-discursivas utilizadas pelos textos publicitários e de
refletir sobre necessidades e condições de consumo.

No Ensino Médio, os jovens precisam aprofundar a análise dos interesses
que movem o campo jornalístico midiático, da relação entre informação e
opinião, com destaque para o fenômeno da pós"-verdade, consolidar o
desenvolvimento de habilidades, apropriar"-se de mais procedimentos
envolvidos na curadoria de informações, ampliar o contato com projetos
editoriais independentes e tomar consciência de que uma mídia
independente e plural é condição indispensável para a democracia.

Como já destacado, as práticas que têm lugar nas redes sociais têm
tratamento ampliado. (\textsc{bncc}, p. 494-495)
\end{quote}

De Maistre transforma sua jornada ao redor do quarto em uma grande
viagem em busca do desconhecido, sendo ele a consciência humana.
Podemos ter momentos de epifania nos pequenos detalhes e quando menos
esperamos. Partindo do princípio da possibilidade de epifanias, o
professor poderá propor uma atividade para ser realizada em grupo. Os
alunos deverão montar um painel de fotografia ou a exibição de
pequenos vídeos, mostrando detalhes da vida escolar e a importância do
espaço para família e comunidade. O painel e os vídeos poderão ser
apresentados aos pais através do portal da escola. Os vídeos poderão
apresentar a interação entre professores e alunos, alunos e alunos,
alunos e o espaço físico escolar.

\subsection{Campo artístico"-literário}

\begin{quote}
No campo artístico"-literário busca"-se a ampliação do contato e a
análise mais fundamentada de manifestações culturais e artísticas em
geral. Está em jogo a continuidade da formação do leitor literário e do
desenvolvimento da fruição. A análise contextualizada de produções
artísticas e dos textos literários, com destaque para os clássicos,
intensifica"-se no Ensino Médio. Gêneros e formas diversas de produções
vinculadas à apreciação de obras artísticas e produções culturais
(resenhas, vlogs e podcasts literários, culturais etc.) ou a formas de
apropriação do texto literário, de produções cinematográficas e teatrais
e de outras manifestações artísticas (remidiações, paródias,
estilizações, videominutos, fanfics etc.) continuam a ser considerados
associados a habilidades técnicas e estéticas mais refinadas.

A escrita literária, por sua vez, ainda que não seja o foco central do
componente de Língua Portuguesa, também se mostra rica em possibilidades
expressivas. (\textsc{bncc}, p. 495-496).
\end{quote}

De Maistre utiliza a texto em prosa para narrar sua jornada e suas
reflexões. A riqueza em detalhes e a presença de metáfora são pontos
fortes dentro da narrativa. As figuras de linguagem são muito
utilizadas na composição de poemas, além da métrica ou a possibilidade
de utilizar versos livres. Proponha aos alunos a escolha de um dos
capítulos da obra \emph{Viagem à Volta do Meu Quarto} e a releitura do
capítulo em forma de poema. O objetivo é conseguir identificar e
trabalhar as figuras de linguagem durante a composição literária.

\subsection{Campo das práticas de estudo e pesquisa}

\begin{quote}
O campo das práticas de estudo e pesquisa mantém destaque para os
gêneros e habilidades envolvidos na leitura/escuta e produção de textos
de diferentes áreas do conhecimento e para as habilidades e
procedimentos envolvidos no estudo. Ganham realce também as habilidades
relacionadas à análise, síntese, reflexão, problematização e pesquisa:
estabelecimento de recorte da questão ou problema; seleção de
informações; estabelecimento das condições de coleta de dados para a
realização de levantamentos; realização de pesquisas de diferentes
tipos; tratamento dos dados e informações; e formas de uso e
socialização dos resultados e análises.

Além de fazer uso competente da língua e das outras semioses, os
estudantes devem ter uma atitude investigativa e criativa em relação a
elas e compreender princípios e procedimentos metodológicos que orientam
a produção do conhecimento sobre a língua e as linguagens e a formulação
de regras. (\textsc{bncc}, p. 495-496)
\end{quote}

De Maistre cita Platão como o autor da teoria sobre \emph{a besta e a
alma}. Platão denomina a besta como \emph{o outro} e entende que há
uma guerra de domínios internos em cada ser humano. Platão é um grande
filósofo grego 
e compartilhou as mais diversas teorias, uma delas é o mito da caverna.
O professor de Língua Portuguesa em parceria com o Professor de
Filosofia poderá propor a escrita de um texto dissertativo-argumentativo
sobre a teoria do mito da caverna. Sugira a leitura da obra \textbf{O
Mundo de Sofia}, Jostein Gaarder, e o filme \textbf{Matrix}, Lana
Wachowski.

\section{Referências complementares}

\begin{itemize}
\item\textsc{assis}, Machado de. \textit{Memórias póstumas de Brás Cubas}. Domínio
público.

Obra inaugural do Realismo brasileiro, o livro de Machado de Assis
apresenta as reflexões de um narrador defunto, contadas segundo o estilo
da sátira menipeia.

\item\textsc{garrett}, Almeida. \textit{Viagens na minha terra}. Domínio público.

O interior português que se descortina em uma viagem a Santarém é pano
de fundo deste clássico do século XIX, centrado no romance entre Carlos
e Joaninha e nos desdobramentos da guerra civil portuguesa.

\item\textsc{perrot}, Michelle. \textit{História dos quartos}. Rio de Janeiro: Paz e
Terra, 2012.

A autora trata da constituição do espaço doméstico privativo e da
presença de espaços reservados à intimidade, tanto na literatura quanto
nas artes visuais.

\item\textsc{rabelais}, François. \textit{Gargântua e Pantagruel}. Belo Horizonte:
Itatiaia, 2009.

Esta obra que revolucionou a estética literária, critica a estagnação
medieval, atacando a igreja, a cavalaria e a política. À sua época, foi
chamada indecorosa, por exprimir os instintos mais básicos dos
indivíduos.
\end{itemize}

\section{Bibliografia comentada}

\begin{itemize}
\item\textsc{griffin}, Dustin. \textit{Satire: A Critical Reintroduction}. Lexington:
University Press of Kentucky, 1994.

A obra é uma boa introdução à sátira e um convite a repensar seus usos,
problemas e prazeres, à luz da teoria contemporânea. Para o autor, a
sátira está mais preocupada em fazer perguntas do que em fornecer
certezas.

\item\textsc{hobsbawm}, Eric John. \textit{A Era dos Impérios}. São Paulo: Paz e Terra, 2012.

A obra destaca fatos que marcaram um período de paz, mas que
desencadearam um período de guerra e crise, construindo uma
interpretação estimulante e inovadora dos anos que definiram o século
XIX.

\item\textsc{oliver}, Élide Valarini. \textit{Rabelais e Joyce: Três Leituras
Menipeias}. São Paulo: Ateliê Editorial, 2008.

Este livro convida o leitor a refletir sobre um gênero pouco conhecido:
a sátira menipeia. De modo interdisciplinar, o volume busca pontos de
convergência entre literatura, filosofia, história da arte e estética.
\end{itemize}

\end{document}


