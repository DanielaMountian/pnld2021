\paragraph{Tema} Cineclube Catarina: o contexto histórico do
surgimento de uma literatura infantojuvenil russa.

\BNCC{EM13LP01}
\BNCC{EM13CHS203}
\BNCC{EM13CHS603}

\paragraph{Conteúdo}
O tempo"-espaço em que começou a florescer a literatura infantojuvenil na
Rússia, ou seja, em meio ao ápice do chamado ``despotismo esclarecido''
russo: o reinado de Catarina \textsc{ii} (Catarina, a Grande).

\paragraph{Objetivos}
Situar o jovem leitor no contexto histórico russo e mundial em que se
inseriu o reinado de Catarina \textsc{ii}, que ajudou a dar os primeiros passos
para a consolidação da literatura infantojuvenil russa.

\paragraph{Justificativa}
A literatura infantojuvenil passou a ser valorizada na Rússia no
romantismo, a partir da segunda década do século \textsc{xix}. Mas, ainda no
final do século anterior, houve eventos importantes para a consolidação
das letras russas infantis (como a criação da revista \emph{Leitura
infantil para o coração e a razão}) e um aumento de publicações voltadas
para esse público. (\textsc{mountian}, 2021, p. 12) Além disso, foi no fim do
século \textsc{xviii} que surgiu este que é considerado o La Fontaine russo: Ivan
Krylóv (1769--1844). A própria imperatriz Catarina, a Grande (que reinou
de 1762 a 1796), pregando os ideais de Rousseau, achava que conhecimento
do mundo deveria ser adquirido desde a infância pela leitura. Prussiana
de nascimento, ela manteve permanente correspondência com filósofos
franceses, como Voltaire, que chegou a visitar a corte de São
Petersburgo, então capital russa. As ``Luzes'' da imperatriz foram, porém,
mais lenda que realidade em alguns âmbitos: embora ela tenha realizado
reformas administrativas importantes, incentivando as universidades de
Moscou e São Petersburgo, estimulando as ciências naturais, apoiando a
agricultura na Ucrânia e nas margens do Volga, foi implacável com
antagonistas políticos. Durante seu reinado, foi criada a Zona de
Assentamento para Judeus, que restringia a circulação deles no restante
do império.

\SideImage{Retrato feito por Fido Nesti de Catarina, a Grande (1729-1796) (Kalinka; Direitos cedidos à Kalinka).}{PNLD0050-11}

\paragraph{Metodologia}
Após a expansão do Grão"-Ducado de Moscou para leste (Urais e Sibéria),
noroeste e sul, a Rússia se tornou um dos reinos que passou por reformas
inspiradas no Iluminismo. Foi no reinado do tsar Pedro, o Grande, que o
país começou a adotar costumes da Europa ocidental, com imposições que
iam desde a obrigatoriedade de todo nobre se barbear e usar trajes
europeus (os nobres russos usavam barbas longas e vestimentas à moda
oriental) e de aprender a ler e a escrever, até a impressão do primeiro
jornal russo, o \emph{Notícias} (\emph{Viédomosti}), em janeiro de 1703.
Tudo isso após o jovem Pedro ter realizado uma inspiradora jornada de
dois anos por países europeus que começara em 1697. Alguns anos depois
de retornar, o tsar determinou a construção de São Petersburgo (capital
do império a partir de 1712) e passou a introduzir costumes europeus na
Rússia, ignorando as queixas dos nobres.

\marginnote{\footnotesize\textbf{PARA LER}\\ ``De calendários a batatas: 5 legados de Pedro, o Grande, para a Rússia'' (\emph{Russia Beyond,} 23/09/17).\\
\url{https://bit.ly/2OxXrYX}}

Mas a época áurea do chamado ``despotismo esclarecido'' ocorreu no reinado
de Catarina, a Grande (1729--1796). Ela escreveu uma cartilha para os netos em
1781 e alguns textos e diálogos edificantes. Duas obras suas são
consideradas pioneiras da prosa infantil russa: ``Conto do Tsarévitche
Cloro'' (1781) e ``Conto dо Tsarévitche Fеvei'' (1783),
que, ``sem viés nacionalista'', ressaltam valores universais, como
justiça e bondade, como observado no prefácio. (\textsc{mountian}, 2021, p.11)

Para realizar esta atividade, sugerimos que os alunos pesquisem,
individualmente ou em grupo, pela internet ou na biblioteca, sobre o
reinado de Catarina, a Grande, e sobre o contexto mundial que permitiu o
desenrolar do chamado ``despotismo esclarecido''.

Em seguida, podem ser exibidos trechos da série \emph{Catarina, a
Grande} (Philip Martin, 2019, disponível no \textsc{hbo go}), principalmente do
primeiro capítulo, que vale ser mostrado integralmente. Após a exibição,
a sala formará uma roda de conversa para debater alguns pontos: Quais
tendências iluministas de Catarina são retratadas pelos autores na
série? O aluno pode fazer menções à relação da soberana com Voltaire e
às concepções dela sobre casamento e divórcio retratadas no episódio; à
sua ideia de abolir a servidão (o que só será realizado, efetivamente,
pelo tsar Alexandre \textsc{ii} em 1861); à presença de Thomas Dimsdale (médico
inglês que desenvolveu um método de prevenção de varíola por inoculação
para proteção de cepas mais virulentas e foi convidado para ir ao
Império Russo em 1768 por Catarina, para inocular a imperatriz, o filho
e 140 membros da corte --- realizando a ``variolação'' antes do ``pai da
imunologia'' Edward Jenner, que testou seu procedimento de vacinação
contra a varíola em 1796); à escrita, em seu quarto, sobre ideias de
igualdade diante da lei. Na via contrária, quais indícios de despotismo
poderiam ser apontados? Quem foi Ivan \textsc{vi}, o ``prisioneiro número 1'' de
Shlisselburg? E por que a relação de Catarina com o filho, Paulo, era
tão turbulenta? Como a política externa russa de Catarina \textsc{ii} deu
continuidade à de Pedro I? Após se aprofundar mais na história da
imperatriz russa, o estudante consegue captar outras nuances na obra
escrita para os netos que aparece em \emph{Contos russos juvenis}?

\paragraph{Tempo estimado} Três a quatro aulas de 50 minutos.
