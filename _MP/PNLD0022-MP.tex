\documentclass{article}
\usepackage{manualdoprofessor}
\usepackage{fichatecnica}
\usepackage{lipsum,media9,graficos}
\usepackage[justification=raggedright]{caption}
\usepackage{bncc}
\usepackage[magrela]{logoedlab}

 \begin{document}


\newcommand{\AutorLivro}{James Joyce}
\newcommand{\TituloLivro}{Stephen Heroi}
\newcommand{\Tema}{Ficção, mistério e fantasia}
\newcommand{\Genero}{Romance}
% \newcommand{\imagemCapa}{PNLD0022-01.png}
\newcommand{\issnppub}{---}
\newcommand{\issnepub}{---}
% \newcommand{\fichacatalografica}{PNLD0022-00.png}
\newcommand{\colaborador}{\textbf{Alessandra Cabral, Bruno Gradella e Vicente Castro} é uma pessoa incrível e vai fazer um bom serviço.}


\title{\TituloLivro}
\author{\AutorLivro}
\def\authornotes{\colaborador}

\date{}
\maketitle
\tableofcontents

\pagebreak

\section{Carta aos professores}

Caro educador / Cara educadora,\\\bigskip

Este Manual tem como objetivo fornecer subsídios para o trabalho com a
obra literária \emph{Stephe Herói}, de James Joyce.

Neste material, são propostas atividades de leitura do texto literário,
em perspectiva interdisciplinar, a partir de propostas de abordagem
envolvendo a integração de diferentes áreas do conhecimento. Todas as
etapas de pré"-leitura, leitura e pós"-leitura atribuem aos professores o
papel de mediadores entre os estudantes e a obra literária. Aos alunos,
por sua vez, é conferido um lugar de protagonismo e autonomia na
construção do conhecimento, a partir do emprego de metodologias ativas e
estratégias de aprendizagem criativa.

Seguindo as competências e habilidades indicadas na nova Base Nacional
Comum Curricular (\textsc{bncc}), o trabalho com o texto literário é desenvolvido
no âmbito dos diferentes campos de atuação social. Para isso, levam"-se
em consideração os campos da vida pessoal, de atuação na vida pública,
das práticas de estudo e pesquisa, jornalístico"-midiático e, sobretudo,
artístico"-literário.

Cada uma das seções do Manual sugere atividades e apresenta informações
complementares para enriquecer a experiência de leitura do texto
literário. A partir de uma proposta dialógica de ensino de literatura,
procura desenvolver habilidades de leitura e produção de textos, visando
à formação do sujeito leitor"-autor competente, capaz de interagir com o
mundo e de atribuir sentido às próprias vivências.

Reforçando o caráter formativo e informativo da literatura, o material
procura articular a formação de leitores à elaboração de projetos de
vida, a partir da ampliação do repertório artístico"-cultural dos
estudantes. A leitura crítica da obra literária é concebida em sentido
amplo e envolve o estabelecimento de relações entre textos pertencentes
a esferas variadas de comunicação e a gêneros discursivos diversos.

Ao mesmo tempo, a releitura de clássicos literários traz, para a
contemporaneidade, novas possibilidades de significação para obras que
permanecem atuais. Para contribuir com essa aproximação entre jovens
leitores e obras consagradas, são propostas atividades que empregam as
novas tecnologias de informação e comunicação, fundamentais para a
formação de leitores inseridos na cultura digital. Valorizam"-se,
portanto, estratégias de leitura e produção textuais no âmbito da
hipertextualidade e do multiletramento.

Para a formação continuada de professores, apresentamos sequências que
estimulam a criatividade e a inovação, com possibilidades de adaptação
às diferentes realidades de ensino. Orientações de caminhos possíveis
são apresentadas como sugestões para as atividades de leitura em sala de
aula, sempre permeáveis aos diferentes perfis de grupos e às
especificidades de gostos e repertórios culturais.

Conforme a \textsc{bncc} assinala, ``no Ensino Médio, os jovens intensificam o
conhecimento sobre seus sentimentos, interesses, capacidades
intelectuais e expressivas; ampliam e aprofundam vínculos sociais e
afetivos; e refletem sobre a vida e o trabalho que gostariam de ter.
Encontram"-se diante de questionamentos sobre si próprios e seus projetos
de vida, vivendo juventudes marcadas por contextos socioculturais
diversos'' (Brasil, 2018, p.481).

Nada mais adequado, portanto, que oferecer textos literários capazes de
estimular reflexões sobre a vida pessoal e rotas possíveis para os
projetos de vida. Para você, professor(a), abrem"-se novas trilhas para o
contato sempre renovado com obras que são, a um só tempo, atuais e
atemporais.

Boa jornada!




\section{Atividades 1}

\BNCC{EM13LP26}

\subsection{Pré"-leitura}

\BNCC{EM13LGG302}
\BNCC{EM13LGG704}
\BNCC{EM13LP10}
\BNCC{EM13LP19}

Nesta primeira atividade, sugere"-se que seja proposta para
os alunos uma atividade de pesquisa a respeito da vida e obra do
escritor James Joyce. Recomenda"-se, para isso, que sejam levantados
aspectos de sua vida pessoal, como infância, juventude, sua relação com
a Igreja, o casamento, a paternidade, o autoexílio, e profissional, como
o início da vida como escritor, a carreira como professor de inglês, as
obras de maturidade (Dublinenses e Ulysses). Para a execução da
atividade, a sala pode ser dividida em grupos e cada um desses grupos
pode apresentar o resultado obtido de uma forma criativa, como, por
exemplo, a produção de um vídeo no YouTube, ou a gravação de um podcast.

\subsection{Leitura}

\BNCC{EM13LGG103}
\BNCC{EM13LP02}
\BNCC{EM13LP48}

Para a atividade de leitura, sugere"-se a pesquisa a respeito
de movimentos artísticos, em específico, os movimentos Naturalismo e
Simbolismo. Sugira aos alunos que anotem aquilo que considerarem
elementos chaves de cada uma das escolas.

Em seguida, peça para que os alunos e alunas façam uma comparação entre
esses movimentos, apontando suas semelhanças e distinções. Por fim, para
finalizar essa atividade, sugere"-se a divisão da sala em grupos, que
podem ser os mesmos da atividade anterior, se for conveniente. A partir
disso, indica"-se que os alunos busquem uma obra artística de cada um
desses movimentos artísticos, de modo que a conceituação se torne mais
lúdica. Por fim, cada grupo deve produzir uma obra de arte. Esta obra
deve articular traços dos dois Movimentos. Para esta atividade propomos
a participação do professor ou professora de artes.

\subsection{Pós"-leitura}

\BNCC{EM13LGG102}
\BNCC{EM13LGG303}
\BNCC{EM13LGG402}
\BNCC{EM13LGG703}
\BNCC{EM13LP13}
\BNCC{EM13LP14}
\BNCC{EM13LP28}
\BNCC{EM13LP29}
\BNCC{EM13LP52}

Ao final, com intuito de estimular o aprofundamento na obra
e universo de James Joyce, recomenda"-se indicar aos alunos a pesquisa e
discussão sobre o conceito de fluxo de consciência do psicólogo William
James e sua utilização na literatura. Para facilitar a compreensão, o
professor pode coletar dúvidas referentes a termos específicos e montar
um dicionário, o qual poderá ser utilizado por toda a turma. Também
recomenda"-se a utilização de outras obras para referenciar o caminho dos
alunos. Ao final, cada aluno individualmente deverá construir um conto,
no qual se explore interioridade dos protagonistas, empregando discurso
indireto livre e praticando o monólogo interior.

\section{Atividades 2}

\BNCC{EM13CNT201}
\BNCC{EM13CNT303}
\BNCC{EM13CHS101}
\BNCC{EM13CHS102}
\BNCC{EM13CHS106}
\BNCC{EM13CHS401}

A obra \emph{Stephen Herói} possibilita trabalhos interdisciplinares e
integradores de diferentes campos do saber e áreas de conhecimento. A
seguir, propomos algumas atividades que podem ser desenvolvidas
conjuntamente com professores de outras áreas. Além das habilidades de
Linguagens e suas Tecnologias e de Língua Portuguesa, indicadas nas
etapas da seção anterior e válidas também para esta, listamos a seguir
as habilidades de outras áreas, presentes na abordagem interdisciplinar:

\subsection{Pré"-leitura}

Para uma boa compreensão da obra, é sempre interessante que
se tenha consciência do lugar e do contexto onde a narrativa se passa.
Nessa primeira atividade de pré"-leitura, sugere"-se a abordagem do
espaço. Para isso, solicite aos alunos que procurem guias de viagem,
mapas e roteiros sobre as cidades irlandesas, em especial Dublin. Peça
para que também busquem fotos e pinturas do local. Solicite, então, que
cada aluno anote suas impressões do espaço. Sugere"-se, inclusive para
tornar a atividade mais lúdica, que utilize mais de uma cor para fazer
as marcações. O aluno deverá criar uma legenda para características que
considera positivas e características da cidade que considera negativas,
montando uma tabela enquanto realiza a classificação. Posteriormente, o
aluno enquanto lê o livro vai anotando na coluna subsequente se suas
impressões prévias se confirmaram ou não.

\subsection{Leitura}

É sabido que toda obra escrita dialoga com eventos do
momento em que ela foi escrita. E com James Joyce não seria diferente.
Então, para uma maior compreensão do universo do autor, recomenda"-se a
realização de uma atividade que explore a história e cultura irlandesa.
Com auxílio dos professores de humanidades, divida os alunos em grupos,
cada grupo deve pesquisar um tema específico, que podem ser divididos em
dois eixos temáticos, como, o primeiro, histórico político, abordando as
razões da presença inglesa na região, as consequências das Revoluções
Inglesas na ilha, o sistema de plantation britânico adotado, a questão
da fome na Irlanda. E, em segundo, a história da cristianização da
Irlanda, a relação do país, agora católico, com sua origem celta, e sua
forte ligação com a igreja católica, o impacto da reforma protestante, o
peso do catolicismo no movimento de independência. Por fim, os grupos
deverão expor, em formato de seminário, o resultado de suas
investigações, podendo este ser virtual ou presencial.

\subsection{Pós"-leitura}

Na leitura de Stephen Herói, o aluno foi apresentado ao
protagonista Stephen Dedalus. Como sói a James Joyce, a escolha do nome
não é um mero acaso, havendo clara inspiração na figura da mitologia
grega. Assim, para uma melhor compreensão da psiquê da personagem
principal do livro lido, sugere"-se a pesquisa sobre o personagem Dédalo,
da Mitologia Grega. A atividade de pré"-leitura também sugere o
desenvolvimento de um projeto em parceria com os professores de ciências
humanas, com o intuito de explorar melhor o universo mitológico
greco"-latino. Para essa atividade, recomenda"-se a apresentação de outros
mitos além da narrativa que envolve Dédalo. Podem ser levantadas
questões históricas tangentes ao período grego. Recomenda"-se, também,
além das fontes primárias, a leitura da obra O Universo, os Deuses e os
Homens, do helenista, Jean"-Pierre Vernant, bem como podem ser utilizadas
par ao trabalho as análises psicológicas feitas acerca da simbologia dos
mitos.

\section{Aprofundamento}

Ao chegar ao Ensino Médio, é necessário que os estudantes se aprofundem
na compreensão das múltiplas linguagens e, sobretudo, da linguagem
literária. Em relação à literatura, a \textsc{bncc} traz as seguintes
considerações:

\begin{quote}
{[}\ldots{}{]} a leitura do texto literário, que ocupa o centro do trabalho
no Ensino Fundamental, deve permanecer nuclear também no Ensino Médio.
Por força de certa simplificação didática, as biografias de autores, as
características de épocas, os resumos e outros gêneros artísticos
substitutivos, como o cinema e as \textsc{hq}s, têm relegado o texto literário a
um plano secundário do ensino. Assim, é importante não só (re)colocá"-lo
como ponto de partida para o trabalho com a literatura, como
intensificar seu convívio com os estudantes. Como linguagem
artisticamente organizada, a literatura enriquece nossa percepção e
nossa visão de mundo. Mediante arranjos especiais das palavras, ela cria
um universo que nos permite aumentar nossa capacidade de ver e sentir.
Nesse sentido, a literatura possibilita uma ampliação da nossa visão do
mundo, ajuda"-nos não só a ver mais, mas a colocar em questão muito do
que estamos vendo/vivenciando. (Brasil, 2018, p. 491)
\end{quote}

Nesta seção, desenvolvemos um trabalho de aprofundamento que, em diálogo
com a formação continuada de professores, oferece subsídios para a
abordagem do texto literário. A leitura em sala de aula de \emph{Stephen
Herói} pode ser enriquecida pelo aprofundamento no universo literário em
que a obra está inserida.

\subsection{A Obra}

Stephen Dedalus está de volta, agora na Universidade. Para aqueles que
já estão familiarizados com as obras de James Joyce, poderão mergulhar
nas reflexões de Stephen. Após o periódico Dana se recusar a publicar Um
Retrato do Artista, Joyce se aprofundou em seu personagem autobiográfico
e nos presentou com Stephen Herói.

Publicado em 1916, Um Retrato do Artista Quando Jovem é a primeira obra
ao qual temos contato com o nosso protagonista Stephen. Envolto por uma
educação católica, em um internato, é dentro desse ambiente que ele terá
seus primeiros questionamentos.

Em Stephen Herói, através do olhar do protagonista, conseguimos
acompanhar a continuação da jornada de seu autoconhecimento. Dentro de
uma Universidade Católica, com regras rígidas, professores e colegas que
pouco, ou quase nada, se aprofundam em questões artísticas, Stephen
continua a quebrar padrões e criar suas teorias.

Visto como uma figura excêntrica pelos seus professores e colegas,
Stephen é um rapaz que mergulha em busca de respostas para os conceitos
de arte, poesia, amor e religião. Enquanto seus colegas buscam apenas
finalizar seus cursos e encontrarem um bom emprego, Stephen transgredia
ao sentir profundamente o significado de versos, pinturas e literatura,
buscando compreender sua essência.

A jornada de Stephen não é solitária, apesar do protagonista nutrir um
certo apreço pela solidão. Ao decorrer da leitura nos deparamos com
personagens que o ajudam em sua reflexão. Ainda que eles não concordem
com as ideias de Stephen, podemos adentrar a mente do protagonista
através dos diálogos entre seus amigos, seus familiares e o corpo
docente da Universidade.

Suas reflexões iniciam"-se no questionamento da concepção do belo,
atravessa o sentimento do amor carnal e coloca em dúvida qual a
importância da religião católica para si e seus pares. O protagonista
busca sempre refletir e tentar colocar em prática tudo aquilo que emerge
de suas conclusões.

\subsection{Um protagonista solitário}

Começamos com um Stephen estudioso, assíduo, apesar de seus recorrentes
atrasos em sala de aula. Um protagonista que se permiti estar em lugares
com ideias que ele pouco, ou nada, concorda, pois estar nessa posição é
o que o ajuda a encontrar sentido, aquilo que ele chama de epifania.

Através da jornada de Stephen, podemos realizar uma autorreflexão sobre
assuntos que dificilmente apresentam conclusões. O debate filosófico
sobre o que é o belo e o sublime, e, o papel do artista dentro de nossa
sociedade. O papel do poeta e como sua produção artística é realizada
também é um dos debates.

O modelo de amar e o amor se apresenta como uma estrada sinuosa para o
debate. Stephen se encontra em um ambiente acadêmico e familiar
altamente religioso. Suas percepções sobre o amor e sobre como lidar com
os desejos os leva a se comportar de maneira inadequada perante a
sociedade e a questionar quão fiel e leal os homens são de fato.

A Igreja e a Irlanda, sua pátria, são uma das duas grandes discussões.
Conforme a narrativa se desenvolve, ele se questiona acerca dos
discursos proferido pelos Jesuítas, seus professores, e, todas as ideias
nacionalistas defendidas por muitos dos seus colegas.

\subsection{Joyce e a Epifania}

Característica principal das obras de James Joyce é a epifania. A
manifestação divina se apresenta com grande recorrência em sua obra.
Joyce descreve a epifania em sua narrativa através do pensamento de
Stephen:


\begin{quote}
Uma manifestação súbita, quer na vulgaridade do discurso ou do gesto,
ou em uma fase memorável da própria mente. Ele acreditava que cabia ao
homem de letras registrar estas epifanias com cuidado extremo, visto que
elas mesmas são os momentos mais delicados e evanescentes.
\end{quote}

Ou seja, esse momento catártico poderia acontecer a qualquer instante e
lugar. O momento de revelação divina poderia estar presente em uma fala,
um gesto ou até em uma memória adormecida. Stephen acreditava também que
cabia ao artista, ``o homem de letras'', registrar essas manifestações
em papel.

Os momentos de epifania do protagonista são maneiras que Joyce encontrou
de nos colocar em contato com sua mente e destrinchar através da
psicologia quais as percepções do personagem. É preciso ir além de
reflexões e questionamentos, é necessário transgredir e ter acesso ao
mais íntimo de autoconhecimento.

\subsection{O processo de criação do artista e o autoconhecimento}

O protagonista Stephen Dedalus é um rapaz na Universidade que busca
entender a si mesmo com a ajuda indireta daqueles que o cerca. A obra
nos mostra a importância de mergulhar em si e buscar compreender quem
somos. Joyce nos apresenta um personagem cercado de convenções sociais
pouco discutidas e conforme o andamento da narrativa vai se
transformando e ajustando o seu mundo.

Ainda que a censura apareça de maneira explícita, quando seu discurso é
vetado pelo professor Jesuíta, ou de maneira implícita, quando Stephen
trava um diálogo com sua mãe sobre a relevância ou não de comungar
durante a Pascoa, Stephen continua sua jornada e busca compreender o que
precisa fazer para se desamarrar.

Quando Stephen reflete sobre o processo de criação de um artista, ele
cita o isolamento como um princípio para criação. Estar sozinho o fazia
mergulhar no passado da humanidade e penetrar nos detalhes. O mesmo
ocorre quando ele se encontra sozinho e refletindo sobre a sua própria
vida. Estar sozinho é como uma condição mínima para olharmos para nosso
interior e penetrarmos em nossos detalhes.

\subsection{Nacionalismo Irlandês: Um debate que atravessa os séculos }

Apesar das reflexões de Stephen não se alongarem ao decorrer da
narrativa em relação aos movimentos nacionalistas, a recorrência de
momentos ao qual ele se encontra cercado por referências e grupos
nacionalistas é muito grande. O cenário temporal da obra se passa no
momento que a Irlanda trava uma revolução pela independência da
Grã"-Bretanha. Diálogos questionando a supremacia da Família Real, a
necessidade excessiva de se aprender o idioma irlandês ou o olhar
preconceituoso em relação à língua inglesa ou qualquer referência
cultural inglesa nos fazem compreender qual o cenário político da
Irlanda do início do século \textsc{xx}.

Por que o debate sobre movimentos nacionalistas é atual? A independência
de países colonizados ou a transformação de países em outros, são
recorrentes em nossa história. Desde a Revolução Americana em 1776 até
Kosovo em 2008, é importante entendermos quais são as novas demandas,
tanto econômicas quanto culturais e quais são os propósitos daquela
população. Stephen fala sobre a liberdade e sobre lugares opressores,
sejam eles lugares físicos ou sociais. Os movimentos de independência
são considerados um caminho em direção a liberdade linguística e
cultural.

Analisar e compreender qual importância da citação desses movimentos
dentro da obra é também entender quais são as provocações sobre o
conceito de liberdade que o protagonista discute em sua narrativa.

\subsection{Stephen Herói e a figura de linguagem}

Ler Stephen Herói é mergulhar em uma dimensão de questionamentos e
reflexões acerca de um mundo religioso e recheado de convenções sociais.
O protagonista cria teorias para definir o que é o artista, o que é o
belo e por muitas vezes para nos pintar um quadro com o seu estado de
espírito ou quais as ações e sentimentos ele se utilizou para finalizar
algum projeto.

O estado de espírito do protagonista é relatado ao decorrer da
narrativa, através de metáforas. É importante termos atenção ao uso de
figuras de linguagem, como a metáfora e a metonímia. Os recursos
estilísticos utilizados por Joyce deixam a narrativa mais rica em
detalhes e cumpre o papel artístico desempenhado pelo protagonista.

\subsection{Por que ler \textit{Stephen Herói}?}

Ler \emph{Stephen Herói} é ser colocado constantemente com ideias e
concepções que nos fazem refletir sobre nosso papel para conosco e nosso
papel perante a sociedade. Os temas abordados não são novos, mas devem
ser sempre revisitados e o debate deve ser sempre estimulado.

Conseguir criar um ambiente que possa acolher as mais diversas
experiências, utilizando como espelho o personagem Stephen é um dos
exercícios mais bem sucedidos. A jornada de autoconhecimento muitas
vezes se inicia na puberdade e adentra a idade adulta. Joyce nos
estimula a criar o hábito de refletirmos sobre quem somos e onde estamos
o tempo todo. A ideia não é buscar respostas concretas, mas saber
questionar e argumentar conforme as situações familiares e cotidianas.

\emph{Stephen Herói} é o que podemos chamar de livro clássico, como
diria Ítalo Calvino, pois a cada releitura conseguimos encontrar uma
nova sensação de descoberta. Ademais, os temas abordados por Joyce são
temas que atravessam gerações e o debate continua vivo, seja em forma de
literatura, cinema ou música. Dessa forma convido a todos vocês a
embarcarem nessa jornada.

\section{Sugestões de atividades complementares: relações dialógicas e
intertextuais}

\BNCC{EM13LP03}
\BNCC{EM13LP04}
\BNCC{EM13LP49}
\BNCC{EM13LP51}

O estudo da Língua Portuguesa, tanto no ensino fundamental quanto no
Ensino Médio, deve contemplar a possibilidade de exercícios
interdisciplinares e exercícios que possam aproximar os conteúdos
programáticos com as vivências dos alunos, proporcionando um ambiente de
inclusão e reflexão.

A ideia da aplicação dos exercícios é conseguir estimular os alunos a
serem autônomos e protagonistas de suas próprias histórias, tornando"-os
responsáveis pelas suas decisões.

\begin{quote}
Essas demandas exigem que as escolas de Ensino Médio ampliem as
situações nas quais os jovens aprendam a tomar e sustentar decisões,
fazer escolhas e assumir posições conscientes e reflexivas, balizados
pelos valores da sociedade democrática e do estado de direito (\textsc{bncc}, p.488)
\end{quote}

Podemos organizar propostas e atividades com base nos campos de atuação
social, sendo eles, campo das práticas de estudos e pesquisa, campo
jornalístico"-midiático, campo de atuação na vida pública, campo
artístico e campo da vida pessoal. Baseando"-se nos conceitos de cada
campo de atuação, podemos propor as seguintes atividades com a obra
\emph{Stephen Herói}:

\subsection{Campo da vida pessoal}

\begin{quote}
O campo da vida pessoal pretende funcionar como espaço de articulações
e sínteses das aprendizagens de outros campos postas a serviço dos
projetos de vida dos estudantes. As práticas de linguagem privilegiadas
nesse campo relacionam"-se com a ampliação do saber sobre si, tendo em
vista as condições que cercam a vida contemporânea e as condições
juvenis no Brasil e no mundo.

Está em questão também possibilitar vivências significativas de práticas
colaborativas em situações de interação presenciais ou em ambientes
digitais e aprender, na articulação com outras áreas, campos e com os
projetos e escolhas pessoais dos jovens, procedimentos de levantamento,
tratamento e divulgação de dados e informações e o uso desses dados em
produções diversas e na proposição de ações e projetos de natureza
variada, para fomentar o protagonismo juvenil de forma
contextualizada. (\textsc{bncc}, p. 494)
\end{quote}

\emph{Stephen Herói} é a narrativa de um jovem em busca de
autoconhecimento. Ele inicia sua jornada em um livro anterior, mas
continua nessa obra. Dentro da Universidade Católica e pertencente à
uma família católica, ele se questiona por diversas vezes sobre sua
religião. Ao mesmo tempo, alguns dos seus colegas fazem parte de
grupos nacionalistas que brigam pela separação da Irlanda e da
Grã"-Bretanha. Levando em conta a dimensão territorial do Brasil e suas
mais diferentes culturas internas, o professor poderá propor aos seus
alunos a \textbf{pesquisa e apresentação} de tradições culturais da
sua comunidade ou de sua família. A ideia é trabalhar a diversidade e
o respeito. Após a coleta de dados, a escola poderá organizar uma
feira de tradições e apresentar em seminários e atividades todo os
trabalhos de pesquisa de seus alunos.

\subsection{Campo de atuação na vida pública}

\begin{quote}
No cerne do campo de atuação na vida pública estão a ampliação da
participação em diferentes instâncias da vida pública, a defesa dos
direitos, o domínio básico de textos legais e a discussão e o debate de
ideias, propostas e projetos. {[}\ldots{}{]}

Ainda no domínio das ênfases, indica"-se um conjunto de habilidades que
se relacionam com a análise, discussão, elaboração e desenvolvimento de
propostas de ação e de projetos culturais e de intervenção social.
(\textsc{bncc}, p. 494)
\end{quote}

Conforme acompanhamos a jornada de autoconhecimento de Stephen, nos
deparamos diversas vezes com episódios de censuras, seja de seus
professores ou de seus amigos e familiares. Em seu primeiro ano de
Universidade, ele se matricula em uma matéria e tem como tarefa a
escrita de um discurso. Acompanhamos a energia que ele coloca em sua
tarefa, mas seus professores não aprovam seu discurso. Proponha aos
alunos o exercício de escrever um \textbf{manifesto} direcionado aos
professores de Stephen e solicitando uma nova chance para que o
protagonista possa ler seu discurso em público na Universidade. A
ideia é conseguir estimular o pensamento crítico, o debate e a prática
do texto dissertativo"-argumentativo.

\subsection{Campo jornalístico"-midiático}

\begin{quote}
Em relação ao campo jornalístico"-midiático, espera"-se que os jovens
que chegam ao Ensino Médio sejam capazes de: compreender os fatos e
circunstâncias principais relatados; perceber a impossibilidade de
neutralidade absoluta no relato de fatos; adotar procedimentos básicos
de checagem de veracidade de informação; identificar diferentes pontos
de vista diante de questões polêmicas de relevância social; avaliar
argumentos utilizados e posicionar"-se em relação a eles de forma ética;
identificar e denunciar discursos de ódio e que envolvam desrespeito aos
Direitos Humanos; e produzir textos jornalísticos variados, tendo em
vista seus contextos de produção e características dos gêneros. Eles
também devem ter condições de analisar estratégias
linguístico"-discursivas utilizadas pelos textos publicitários e de
refletir sobre necessidades e condições de consumo.

No Ensino Médio, os jovens precisam aprofundar a análise dos interesses
que movem o campo jornalístico midiático, da relação entre informação e
opinião, com destaque para o fenômeno da pós"-verdade, consolidar o
desenvolvimento de habilidades, apropriar"-se de mais procedimentos
envolvidos na curadoria de informações, ampliar o contato com projetos
editoriais independentes e tomar consciência de que uma mídia
independente e plural é condição indispensável para a democracia.

Como já destacado, as práticas que têm lugar nas redes sociais têm
tratamento ampliado. (\textsc{bncc}, p. 494-495)
\end{quote}

Em dado momento da narrativa, os colegas da Universidade de Stephen
Dedalus se organizam para elaborar e distribuir um periódico. Stephen
é convidado por McCan para escrever um artigo para esse periódico. O
protagonista acaba declinando da proposta, pois descobre que os textos
acabarão passando por uma espécie de departamento de ``censura'' dos
professores, uma vez que eles estão ajudando a organizar o projeto
também. Organizando a sala em grupos, o professor poderá propor a
\textbf{montagem, escrita e distribuição} de periódicos ou a
publicação no site da escola, se houver. É importante estimular a
organização dos alunos dentro de seus grupos, de forma que eles
consigam trabalhar em grupo e produzir textos reflexivos. É
interessante que o professor apresente algumas mídias alternativas,
como \emph{podcast}, fanzines, \emph{blogs}, canais no \emph{YouTube}
entre outros tipos de mídias.

\subsection{Campo artístico"-literário}

\begin{quote}
No campo artístico"-literário busca"-se a ampliação do contato e a
análise mais fundamentada de manifestações culturais e artísticas em
geral. Está em jogo a continuidade da formação do leitor literário e do
desenvolvimento da fruição. A análise contextualizada de produções
artísticas e dos textos literários, com destaque para os clássicos,
intensifica"-se no Ensino Médio. Gêneros e formas diversas de produções
vinculadas à apreciação de obras artísticas e produções culturais
(resenhas, vlogs e podcasts literários, culturais etc.) ou a formas de
apropriação do texto literário, de produções cinematográficas e teatrais
e de outras manifestações artísticas (remidiações, paródias,
estilizações, videominutos, fanfics etc.) continuam a ser considerados
associados a habilidades técnicas e estéticas mais refinadas.

A escrita literária, por sua vez, ainda que não seja o foco central do
componente de Língua Portuguesa, também se mostra rica em possibilidades
expressivas. (\textsc{bncc}, p. 495-496).
\end{quote}

\emph{Stephen Herói} é uma narrativa que discute e reflete muito sobre
o papel do artista e do poeta na sociedade. O protagonista questiona o
processo de criação da poesia, discute as definições sobre o belo e o
sublime. O artista é um criador, seja de poesia ou literatura. A
atividade proposta pode ser realizar em conjunto com a atividade
sugerida no campo jornalístico"-midiático. O professor poderá sugerir
que os alunos se organizem em grupos para uma \textbf{apresentação
teatral}, que deverá contemplar uma das principais reflexões da obra,
e cada grupo deverá escrever sobre a apresentação em seus periódicos.
Poderá, também, propor a \textbf{escrita de poemas}, que deverão ser
publicados nos periódicos. A ideia é estimular o lado criativo,
crítico e fortalecer o trabalho em grupo.

\subsection{Campo das práticas de estudo e pesquisa}

\begin{quote}
O campo das práticas de estudo e pesquisa abrange a pesquisa,
recepção, apreciação, análise, aplicação e produção de discursos/textos
expositivos, analíticos e argumentativos, que circulam tanto na esfera
escolar como na acadêmica e de pesquisa, assim como no jornalismo de
divulgação científica. O domínio desse campo é fundamental para ampliar
a reflexão sobre as linguagens, contribuir para a construção do
conhecimento científico e para aprender a aprender. (\textsc{bncc}, p. 488-489)
\end{quote}


\emph{Stephen Herói} é uma narrativa autobiográfica e contamos com um
narrador em primeira pessoa. Acompanhamos todo o desenrolar do romance
do ponto de vista de Stephen, o protagonista. Não somente a narrativa
visual, como também temos acesso aos seus pensamentos e suas reflexões.
Estudar os diversos tipos de narradores é muito importante para uma
compreensão aprofundada das mais variadas histórias. Durante, ou após, a
leitura de \emph{Stephen Herói,} o professor poderá realizar um recorte
do texto ``\emph{O Narrador''} de Walter Benjamin, transformado o texto
em uma linguagem mais acessível a fim de propor aos alunos a elaboração
de um pequeno \textbf{artigo} comparando os narradores de \emph{Stephen
Herói} e uma obra de sua livre escolha. Após a entrega do artigo escrito
é interessante que cada aluno apresente para a turma qual a sua obra
escolhida e como o narrador se manifesta.

\section{Referências complementares}

\begin{itemize}
\item\textsc{eagleton}, Terry. \textit{The English Novel. An Introduction}. Hoboken,
New Jersey: Blackwell, 2004.

Este livro fornece uma introdução ampla, acessível e bem"-humorada ao
romance inglês de Daniel Defoe até James Joyce.

\item\textsc{killeen}, Richard. \textit{A Brief History of Ireland}. Philadelphia:
Running Press Book Publishers, 2012.

Killeen caracteriza a Irlanda em um mundo europeu e atlântico,
explorando a notável contribuição da nação internacionalmente: por meio
de sua literatura, sua diáspora e seu gênio para a política popular.

\item\textsc{o'neill}, Bill. \textit{The Great Book of Ireland: Interesting Stories,
Irish History \& Random Facts About Ireland}. Independently published,
2019.

Neste livro de curiosidades, você aprenderá mais sobre a história da
Irlanda, cultura pop, folclore e muito mais.
\end{itemize}

\section{Bibliografia comentada}

\begin{itemize}
\item\textsc{hobsbawm}, Eric John. \textit{A Era dos Impérios}. São Paulo: Paz e
Terra, 2012.

A obra destaca fatos que marcaram um período de paz, mas que
desencadearam um período de guerra e crise, construindo uma
interpretação estimulante e inovadora dos anos que definiram o século
\textsc{xix}.

\item\textsc{moretti}, Franco. \textit{O Romance de Formação}. São Paulo: Todavia,
2020.

Este livro analisa o surgimento, o auge e a decadência do romance de
formação, na qual os representantes nada heroicos da nova classe média
europeia buscam responder a uma questão fundamental: é possível uma vida
feliz e com sentido?

\item\textsc{oliver}, Élide Valarini. \textit{Rabelais e Joyce: Três Leituras
Menipeias}. São Paulo: Ateliê Editorial, 2008.

Este livro convida o leitor a refletir sobre um gênero pouco conhecido:
a sátira menipeia. De modo interdisciplinar, o volume busca pontos de
convergência entre literatura, filosofia, história da arte e estética.
\end{itemize}


\end{document}

