\documentclass{article}
\usepackage{manualdoprofessor}
\usepackage{fichatecnica}
\usepackage{lipsum,media9,graficos}
\usepackage[justification=raggedright]{caption}
\usepackage{bncc}
\usepackage[youkali]{logoedlab}

\begin{document}


\newcommand{\AutorLivro}{Mário de Andrade}
\newcommand{\TituloLivro}{O peru de natal e outros contos}
\newcommand{\Tema}{Ficção, mistério e fantasia}
\newcommand{\Genero}{Conto, crônica e novela}
% \newcommand{\imagemCapa}{PNLD0028-01.png}
\newcommand{\issnppub}{---}
\newcommand{\issnepub}{---}
% \newcommand{\fichacatalografica}{PNLD0028-00.png}
\newcommand{\colaborador}{\textbf{Rodrigo Jorge Ribeiro Neves} é uma pessoa incrível e vai fazer um bom serviço.}


\title{\TituloLivro}
\author{\AutorLivro}
\def\authornotes{\colaborador}

\date{}
\maketitle
\tableofcontents

\pagebreak

\section{Carta aos professores}

Caro educador / Cara educadora,\\\bigskip

Este manual tem o objetivo de auxiliá-lo no desenvolvimento de práticas
pedagógicas que estabeleçam o diálogo entre a obra literária aqui
apresentada e os estudantes, de modo a ampliar não apenas a leitura do
texto em si, mas também a sua relação com o mundo.

O livro \emph{Alguns} \emph{contos} consiste em uma antologia de
narrativas curtas de Mário de Andrade, extraídas de três de suas obras,
\emph{Primeiro andar} (1926), \emph{Os contos de Belazarte} (1934) e
\emph{Contos novos} (1947), em que o escritor apresenta alguns dos
principais elementos que caracterizam sua literatura, como os limites da
linguagem, a incorporação da oralidade na escrita, a cultura popular, a
dimensão psicológica das personagens, o conflito de classes e a busca
por uma identidade nacional.

Embora Mário de Andrade seja conhecido pela sua atuação em diversos
gêneros e áreas do saber, sua atuação como contista é uma das pouco
mencionadas. Assim, você tem a chance de enriquecer as suas aulas
apresentando também uma das facetas menos conhecidas desse grande
escritor brasileiro, tão importante para a valorização da nossa cultura
e para que possamos, desta maneira, nos (re)conhecer cada vez mais.

Para isso, apresentamos aqui propostas de atividades, aprofundamento,
referências complementares e uma bibliografia comentada, a fim de que o
material possa ser útil nas suas aulas para estimular os estudantes a
desbravar um universo de possibilidades através de um dos escritores
mais múltiplos da nossa literatura. Além disso, é ótimo trabalhar com
contos em sala de aula, pois é um gênero literário bastante fecundo e
que, pela sua forma curta, possibilita a dinamização das atividades e a
exploração de uma variedade maior de temas para discussão com os
estudantes.

Aproveite bastante este material. Ele foi feito com muita dedicação e
carinho para você! Boa aula!

\reversemarginpar
\marginparwidth=5cm


\section{Atividades 1}
%\BNCC{EM13LP26}

\subsection{Pré-leitura}


\paragraph{Tema} A atualidade do conto e sua relação com as narrativas
  produzidas nas redes sociais.

%(Habilidades BNCC EM13LGG101; EM13LGG703;   EM13LGG301; EM13LP50; EM13LP53)}

\paragraph{Conteúdo} Discussão sobre a atualidade do gênero conto a partir
de experiências com as narrativas construídas nas redes sociais pelos
usuários, como, por exemplo, Twitter, Instagram, TikTok e YouTube.

\paragraph{Objetivo} Estimular e habilitar os estudantes a reconhecer e
compreender os principais elementos e as estratégias estéticas e
narrativas do conto como gênero literário, tendo em vista o
funcionamento das histórias que eles acompanham e/ou constroem nas
principais redes sociais.

\paragraph{Justificativa} Não é nenhuma novidade que as redes sociais e
outras plataformas do mundo virtual vêm ocupando cada vez mais o
cotidiano de todos nós, especialmente dos estudantes, que vêm crescendo
e se formando na chamada Web 2.0. Tendências, estilos e pontos de vista
são profundamente influenciados pelo contato dos jovens com essa
realidade (ou, seria multi-realidade?), conferindo novas maneiras de
perceber, sentir, se expressar e se posicionar no mundo.

Mas nem tudo é apenas novidade. Embora os meios em que as informações e
as experiências dos indivíduos tenham sido transformados nas últimas
décadas, há elementos e estruturas que são mais antigos que os nossos
próprios avós. Mário de Andrade, em resposta aos críticos do suposto fim
dos ``assuntos poéticos'' no modernismo, disse que o amor ainda existia,
só que passou a andar de automóvel. Parafraseando o escritor paulista,
diríamos, então, que o amor ainda existe, mas agora chega pela WhatsApp.
Ou seja, há situações que nunca mudam, apenas são atualizadas as formas
como são apresentadas. As redes sociais, ainda que sejam uma realidade
completamente diferente daquela que vivíamos há poucas décadas atrás,
conservam práticas, pensamentos e sensibilidades desde épocas
imemoriais, refletindo a sociedade em suas principais dimensões. A nossa
necessidade contar histórias é uma delas.

O conto é um gênero literário moderno e, pela sua extensão curta, pode
ser facilmente incorporado, por meio de seus componentes estruturais
narrativos, ao funcionamento da comunicação pelas redes sociais, não
apenas pela sua instantaneidade, mas também por se concentrar em
determinados indivíduos ou situações. Por isso, é fundamental que o
estudante, com a proposição e adequada mediação do professor, os
identifique e analise a partir da sua experiência como usuário das
principais redes sociais e outras plataformas do mundo virtual, para que
se familiarize e compreenda os principais aspectos do conto e suas
diversas possibilidades.

\paragraph{Metodologia} Como ponto de partida, o professor pode começar
perguntando aos estudantes o que é um conto, sem se preocupar com uma
definição teórica ou determinada nesse momento. Anotar na lousa as
definições mais recorrentes dada pelos participantes da aula ou pedir
para que eles as registrem e mantenham para a etapa posterior da
discussão são boas opções. O mais importante, nesse momento, é elaborar,
coletivamente, uma definição do gênero literário tendo em vista as
experiências de cada um. Ainda que ninguém tenha lido um conto na vida
(algo bem difícil de acontecer), em sentido estrito, é fundamental que o
educador os encoraje a dizer o que eles entendem como um texto do
gênero, pois, muitas vezes, os contatos se deram por meio de gêneros
fronteiriços ou híbridos, que incorporam os procedimentos narrativos do
conto em suas estruturas. E essa percepção é bastante útil na presente
atividade, pois estimula os estudantes a pensar no texto levando em
conta não a sua forma acabada e fechada, mas o seu processo de
construção e os elementos estruturais que o compõem, ou seja, da parte
para o todo.

É muito importante que, durante essa discussão, o educador procure
sempre exemplificar com textos, de preferência outros que não sejam
desta coletânea, a fim de que o processo de aprendizagem seja ampliado.
Claro que é imprescindível que a conversa se desenvolva com textos
conhecidos dos estudantes, principalmente levando em contas as sugestões
dadas por eles, mas não se deve perder, de maneira alguma, a
oportunidade de sugerir novas obras, incutindo, sempre, a curiosidade e
o gosto pela leitura.

Após essa etapa introdutória, o professor pode sugerir aos estudantes
que tentem, a partir dos resultados anotados na conversa sobre a
definição do gênero conto, relacionar cada item com os recursos das
redes sociais que eles conhecem e/ou utilizam. Naturalmente as
definições vão ser reformuladas de acordo com a plataforma comparada, de
maneira que elas se adequem as suas respectivas funcionalidades, por
isso, o educador precisa ficar atento para que não se perca de vista o
objetivo da atividade, que é reconhecer e compreender como os elementos
compositivos do conto, enquanto gênero literário, pode estar presente
nos recursos das redes sociais, ou seja, elas também podem ser espaços
de narração de histórias curtas.

Peça para que os estudantes formem grupos ou duplas e proponha que
utilizem os recursos de qualquer rede social para narrar um conto ou um
fragmento dele, caso o texto escolhido seja um pouco mais complexo.
Nesse caso, quem sabe o grupo ou dupla proponha o piloto de uma série?
Todas as propostas são bem-vindas. Na aula em que os trabalhos vão ser
apresentados, incentive que todos discutam os contos adaptados para as
plataformas virtuais, apontando os elementos que os aproximam do gênero
literário e os que os afastam e/ou os colocam em xeque, testando os
limites não apenas da narrativa curta, mas também das redes sociais como
espaços de produção e compartilhamento de histórias. Comece pelo mais
elementar em análise literária, como enredo, personagens, tempo,
ambiente e outros elementos importantes que compõem uma narrativa
ficcional, então, deixe que os estudantes ampliem a partir da recepção
do conto e do processo de criação do grupo para a adaptação da história.

\paragraph{Tempo estimado} Duas aulas de 50 minutos.

\subsection{Leitura I}

\paragraph{Tema} Quem conta um conto, aumenta um ponto?

%(Habilidades BNCC  EM13LGG101; EM13LGG301; EM13LP45; EM13LP52)}

\paragraph{Conteúdo} Compreensão do conto como gênero literário,
identificando suas especificidades a partir das experiências individual
e coletiva de leitura.

\paragraph{Objetivo} Estimular e habilitar os estudantes, por meio da
leitura dos textos de Mário de Andrade, a identificar o conto como
gênero literário por meio de suas principais características, mudanças e
possibilidades de criação.

\paragraph{Justificativa} Por ser de extensão mais curta que o romance, o
conto costuma ser visto, por alguns, como um gênero mais ``fácil'' de
ser praticado. Não necessariamente. As facilidades que podemos encontrar
em um conto são de outra ordem, como o tempo de dedicação à leitura, a
identificação de todos os personagens ou a utilização em atividades na
sala de aula. Mas o processo criativo que envolve a produção de um
conto, ainda que ele sirva de laboratório para a elaboração de uma obra
mais extensa, também requer domínio dos seus principais elementos
compositivos, aliás, como em toda criação literária.

É comum alguns escritores começarem suas carreiras com os textos de
narrativa curta antes de se dedicarem ao romance, por exemplo. No
entanto, isso não significa que um conto não tenha autonomia e não
apresente complexidade em seu desenvolvimento. Elas são apenas de outra
ordem. Para compreender isso, é fundamental que o educador apresente e
discuta com os estudantes os principais componentes que formam o que
podemos chamar de conto e como a sua estrutura se desenvolve para que
uma história seja narrada dentro de um espaço mais curto. Portanto, a
experiência que cada um tem com a leitura pode ser interessante,
ampliando a percepção dos estudantes acerca da gama de possibilidades
que o gênero conto pode comportar.

A pergunta que intitula esta atividade é, mais do que uma provocação, um
convite para entender o funcionamento do conto por meio de suas
características mais elementares e refletir sobre nossa relação com a
arte de contar histórias, pois é delas, segundo Eduardo Galeano, que
somos feitos.

\paragraph{Metodologia} O professor pode iniciar a aula pedindo que os
estudantes falem sobre sua experiência com a leitura, seja de contos ou
de outro gênero. O próprio educador também deve contribuir com uma
história, talvez iniciando a roda de conversa, a fim de incentivar os
demais. Conte a ``história'' que o levou àquela história, ou seja,
descreva a origem do seu contato com o texto escolhido. Foi o presente
de alguém especial? Era seu aniversário ou algo circunstancial? Fazia
parte da bibliografia de uma determinada matéria da escola? Ou um
professor comentou, rapidamente, em aula e o texto despertou seu
interesse? A capa do livro chamou a sua atenção ou foram os títulos dos
contos? Você se recorda onde estava quando leu a história e o que ela
lhe provocou? Pense nestas e em outras questões relacionadas à
experiência de leitura, que não se resume ao ato de ler propriamente
dito, mas abarca também as condições em que ela se realiza. Dessa
maneira, você está inserindo os educandos no universo do que constitui a
arte de contar histórias, permitindo que a discussão sobre os elementos
específicos do conto, principalmente em sua forma moderna, seja ainda
mais proveitosa.

Organize em grupos com duas ou mais pessoas. Escolham um dos contos da
antologia organizada de Mário de Andrade. Caso os estudantes queiram
propor a inclusão de outras narrativas curtas fora do material
literário, deixe aberta a possibilidade, desde que eles façam uma
comparação ou mesclem com um dos contos de Mário de Andrade.

Cada grupo é livre para escolher a forma como apresentará o conto em
sala de aula. Pode ser por meio de leitura dramatizada, cartazes,
fotografias, música, mímica ou como fazem os contadores de histórias.
Não importa o meio que eles utilizem, mas a atenção aos principais
aspectos do conto, que serão discutidos pela turma ao final de cada
apresentação. Eles devem identificar os elementos que caracterizam a
história adaptada nos diversos formatos de apresentação e o que faz dela
um conto.

Você pode orientá-los a observarem a construção das personagens, as
ações que a definem na história que está sendo narrada, o tempo e o
espaço em que elas ocorrem, os recursos narrativos e a elaboração dos
diálogos. Como cada grupo lidou com a adaptação desses elementos na
forma escolhida para apresentar o conto, quais foram as suas
dificuldades e por quê. A partir desta análise, alguns pontos em comum
devem aparecer, contribuindo para a apreensão dos principais componentes
de um conto.

A atividade também deve ressaltar as transformações que ocorrem no texto
quando ele é submetido ao trabalho coletivo. É possível contar uma
história sem alterá-la? E por que ocorrem as modificações na forma como
cada um constrói o relato? A perspectiva do ficcionista que a compôs é
complementada ou deslocada com a leitura da história? O que essas
alterações revelam sobre o gênero conto e sobre o ato de leitura? É uma
chance de o educador abordar as relações entre a necessidade do ser
humano em contar e ouvir histórias e a dimensão social da atividade
proposta, buscando, por meio dessa experiência, não apenas a mera
distração ou entretenimento, mas a aprendizagem de novas maneiras de
enxergar a realidade e o fortalecimento dos laços que unem um grupo de
pessoas envolvido em algo comum.

\paragraph{Tempo estimado} Duas aulas de 50 minutos.

\subsection{Leitura II}


\paragraph{Tema} Aproximações e diferenças do conto com outros gêneros em prosa.
%(Habilidades BNCC EM13LGG101; EM13LGG301; EM13LP48; EM13LP49)}

\paragraph{Conteúdo} Compreensão das relações estruturais entre o conto e
os demais gêneros literários em prosa, como a crônica, a novela e o
romance, sem deixar de evidenciar a inserção de cada um na realidade
representada pela obra literária.

\paragraph{Objetivo} Estimular e habilitar os estudantes a perceber os
vínculos estruturais entre os principais gêneros literários em prosa, a
partir do conto, a fim de que compreendam as relações entre a realidade
e a forma que ela demanda para que seja representada.

\paragraph{Justificativa} A literatura, assim como toda obra de arte, é
representação da realidade social por meio da expressão subjetiva, a fim
de que possamos compreender essa realidade e, de alguma maneira,
transformá-la. Para isso, existe uma forma que busca corresponder às
questões suscitadas por ela. Às vezes, é um poema, uma canção, um
quadro; outras, uma performance, um vídeo, um romance, uma série de TV.
Claro que a formação e as preferências individuais do artista, moldadas
pelas suas origens e história de vida, vão determinar os meios que ele
irá escolher para expressar sua inquietação diante do mundo. No entanto,
o próprio mundo ao qual ele se debruça exigirá um olhar particular para
que seja alcançado. Dentro de cada obra de arte, há uma forma de
expressão; dentro de cada forma de expressão, um conjunto de elementos
que indicam o modo de funcionamento daquela linguagem para expressar o
mundo a ser representado.

A escolha de um gênero literário nunca é arbitrária. Há escritores que
produziram contos impressionantes e romances razoáveis, outros que
criaram romances brilhantes e contos não tão bons, há ainda aqueles que
foram geniais nos dois gêneros, como, por exemplo, Machado de Assis,
Guimarães Rosa e Clarice Lispector, três dos nossos maiores. As
fronteiras que dividem esses e outros gêneros em prosa vão além das mais
óbvias, como a extensão. Fica fácil saber a diferença entre uma crônica
e um romance tomando esse critério. Quando o aspecto estrutural não é
tomado como referência para estabelecer essa distinção, as fronteiras se
tornam ainda mais tênues. Tudo bem que uma novela costuma ser mais
extensa que um conto, mas há outros elementos que se confundem.

A discussão sobre essas relações pode ajudar o estudante a compreender o
projeto envolvido na concepção dos contos de Mário de Andrade, pois ele,
de certa maneira, também os desenvolve nas suas outras narrativas em
prosa, como nos romances e crônicas.

\paragraph{Metodologia} A partir dos contos selecionados desta coletânea,
peça para que os estudantes sugiram qual (ou quais) poderia ser um bom
material para romance e por quê. Quais elementos contidos na narrativa
curta escolhida podem ser ampliados para que se desenvolvam em uma
narrativa mais longa. São necessários mais personagens? Quais e como
eles caracterizariam esses personagens adicionais? De que modo eles se
relacionariam com os personagens do conto e em que medida estes seriam
modificados para que componham o romance? Eles podem também tentar
investigar as relações entre o conto e outros gêneros em prosa, como a
crônica. De certa maneira, a literatura de Mário de Andrade é
atravessada pela influência de diversos gêneros, então, é comum que as
fronteiras entre eles sejam, às vezes, difíceis de delimitar. Certas
histórias acabam servindo também como laboratório para o desenvolvimento
de questões estéticas e culturais mais amplas. Em ``História com data'',
por exemplo, conto extraído de \emph{Primeiro andar}, o autor inclui uma
série de elementos paratextuais em notas de pé de página, como fontes a
partir de periódicos e livros, estabelecendo um jogo entre as categorias
referencial e ficcional na narrativa curta. Além da evidente
intertextualidade com o livro de contos \emph{Histórias sem data}, de
Machado de Assis, o autor esboça algumas questões que serão exploradas
depois em \emph{Macunaíma}, como a universalidade da cultura e as
tensões que constituem a formação da identidade nacional.

Organize a turma em grupos e proponha uma discussão sobre a
intertextualidade entre os gêneros em prosa, considerando,
especialmente, o conto. Você pode começar com este conto que acabamos de
comentar, ou com outro que considere mais adequado às discussões que
pretenda conduzir em sala de aula. Inicialmente, os grupos podem fazer
uma roda de conversa sobre o conto escolhido, como em um clube de
leitura, expondo os principais elementos da história. Depois cada um
pode apresentar uma proposta de outro gênero em prosa a partir desse
conto. A linguagem escolhida para a apresentação fica a critério dos
educandos. Não há limites para a imaginação! O importante é convencer os
demais que a história em formato de narrativa curta pode ser
transformada em um texto em prosa de outro gênero literário.

Depois da atividade, você, professor, pode retomar as perguntas
iniciais. Os alunos podem responder em aula ou entregar por escrito, em
uma ou duas páginas no máximo. O importante é que todos participem!

\paragraph{Tempo estimado} Duas aulas de 50 minutos.


\subsection{Pós-leitura}

\paragraph{Tema} A escrita e a oralidade no processo de criação literária.

%(Habilidades BNCC EM13LGG101; EM13LGG301; EM13LP13; EM13LP46;  EM13LP53)}

\paragraph{Conteúdo} Realização de uma oficina de criação literária
dedicada, inicialmente, à elaboração de contos, ressaltando as relações
entre os registros oral e escrito no processo criativo. Também será
fundamental, nesse processo, a reflexão em torno da influência da
cultura popular, um dos temas caros ao universo de Mário de Andrade.

\paragraph{Objetivo} Elaborar contos em uma oficina de criação literária e
discutir os principais elementos de seu processo criativo, a fim de
compreender as relações e conflitos do gênero em suas diversas
possibilidades de construção narrativa.

\paragraph{Justificativa} A nossa necessidade de contar e compartilhar
histórias remonta a tempos em que nem mesmo a escrita havia se
desenvolvido. Por isso, a relação entre a oralidade e a escrita na
composição de narrativas, tanto em verso quanto em prosa, é sempre
atravessada por esse lastro. Embora a escrita tenha se estabelecido e
predominado como registro na construção de histórias, a dimensão oral
não deixa ainda de estar presente. Na literatura brasileira, José de
Alencar foi um dos precursores da inclusão de registros da fala oral
cotidiana em seus romances urbanos. Com o modernismo e o projeto
literário de Mário de Andrade, a oralidade passou a ser incorporada na
literatura, abrangendo praticamente todos os gêneros.

A dinâmica de uma oficina literária permite não apenas refletir e
discutir sobre as relações entre os registros oral e escrito na
concepção de narrativas curtas e outros gêneros em prosa relacionados,
mas também aguçar a criatividade, a percepção, a sensibilidade e a visão
de mundo dos educandos. A atividade não é para ser uma espécie ``agência
de caça-talentos'' literária, claro. Não que não seja possível que
exista entre os alunos uma sementinha de uma grande escritora ou de um
grande escritor. Se houver, ótimo! Contudo, o mais importante é que
todos possam exercitar ativamente sua capacidade de criar e recriar
visões de mundo a partir de uma história, aprendendo que escrever é
também uma forma crítica de leitura.

\paragraph{Metodologia} Esta atividade pode ser o projeto de algo mais
amplo e, quem sabe, permanente dentro da escola, abrangendo outros
gêneros literários e outras turmas. Evidentemente nada impede,
professor, que você já exercite a prática de outros gêneros com essa
proposta de oficina. No entanto, como primeiro passo, seria interessante
dedicarmos a atenção ao conto e/ou à novela, que está dentro da
fronteira das chamadas narrativas curtas e do escopo deste manual e suas
propostas de atividades.

Para a criação da oficina literária de contos e/ou outras narrativas
curtas, você pode seguir os passos a seguir, que podem, evidentemente,
ser adaptados de acordo com o lugar, o tempo e as condições disponíveis
para que a atividade seja desenvolvida:

\begin{enumerate}
\item
  Organize os alunos em uma roda, a fim de que a atividade de construção
  de histórias seja, sobretudo, uma atividade de compartilhamento de
  experiências, que é, afinal, o fundamento de toda concepção de uma
  narrativa;
\item
  Embora a escrita seja considerada, em princípio, como uma empreitada
  individual, nada impede que também seja coletiva, portanto, permita
  que grupos de dois ou mais estudantes sejam formados. É uma boa
  oportunidade, aliás, para discutirem a questão da autoria na criação
  literária. Apenas limite o número de integrantes, a fim de que a
  atividade seja proveitosa para todos. Um grupo formado por um número
  muito grande de alunos pode acabar dispersando-os e frustrando o
  objetivo da oficina;
\item
  Inicialmente, proponha um exercício de construção de personagem. O
  ponto de partida pode ser alguém do convívio do aluno, mas que não
  seja tão próximo, de uma notícia de jornal, de um quadro, de uma
  fotografia ou de um jogo de videogame. As celebridades e as pessoas
  bem conhecidas devem ser evitadas, pois isso pode interferir no
  processo criativo. O aluno pode criar um personagem do nada? Pode,
  claro. Mas a ideia de utilizar pessoa já existente, seja real ou
  ficcional, é utilizar os elementos humanos percebidos pelos alunos em
  sua primeira leitura da realidade que o cerca e, assim, por meio do
  processo de construção de personagem, estabelecer uma leitura mais
  ativa sobre o seu objeto;
\item
  Eles podem fazer o mesmo em relação ao lugar e à época. As opções são
  ilimitadas. Os nomes podem ser os mesmos do mundo real. Quem sabe a
  rua onde o aluno more ou um lugar que ele adore visitar nas férias, ou
  talvez que gostaria de conhecer? São bem-vindas, sem dúvida, as
  possibilidades ficcionais. Lembra de \emph{O senhor dos anéis}? Se
  você leu o livro de J. R. R. Tolkien ou viu o filme adaptado e
  dirigido por Peter Jackson, sabe que o escritor britânico criou um
  mundo completamente novo por meio da literatura. Os alunos devem se
  sentir livres para criar, no entanto, é preciso dar um passo por vez,
  de modo a compreender os principais elementos que tornam uma história
  verossímil ou não. E é por meio da consciência na utilização desses
  elementos que isso vai ficando mais claro;
\item
  Em alguns encontros da oficina, proponha discussões sobre filmes, a
  fim de que possam analisar em conjunto os principais componentes da
  história que está sendo contada. A escolha da obra fica a seu
  critério, procurando sempre ouvir as sugestões dos estudantes;
\item
  Escrever o primeiro parágrafo de uma história pode ser decisivo para o
  seu desenvolvimento. Analisem os inícios dos contos desta coletânea e
  traga outros exemplos clássicos da literatura, como, por exemplo, a
  novela \emph{A metamorfose}, de Franz Kafka. Os estudantes podem
  praticar a escrita do primeiro parágrafo, apresentando-o aos demais;
\item
  À medida que os encontros da oficina avançarem, incentive sempre que
  eles apresentem parte do desenvolvimento de suas histórias, ainda que
  estejam incompletas e empacadas em algum ponto determinado. A
  discussão em sala de aula pode contribuir para que eles solucionem
  alguns desses problemas;
\item
  Quando finalizadas, as histórias podem ser apresentadas da forma como
  cada aluno ou grupo preferir. Os textos podem ser reunidos de diversas
  maneiras, a depender dos recursos que estiverem disponíveis. Se houver
  a possibilidade de publicação do material impresso, não hesite, claro,
  mas não deixe de estimular a divulgação do trabalho por meio dos
  incontáveis meios que o universo digital oferece.
\end{enumerate}

\paragraph{Tempo estimado:} Dois bimestres.


\section{Atividades 2}

\subsection{Pré-leitura}

\paragraph{Tema} A cultura popular na literatura e em outras artes.
  %(Habilidades BNCC)}

\paragraph{Conteúdo} Discussão e reflexão sobre as relações entre cultura
popular e produção artística em geral, bem como as formas como esses
dois campos se influenciam reciprocamente, tendo em vista as
transformações que a categoria do ``popular'' atravessou com o passar do
tempo.

\paragraph{Objetivo} Habilitar os estudantes a analisar, discutir e
compreender a importância da cultura popular na construção da nossa
identidade por meio da literatura e das outras artes.

\paragraph{Justificativa} A cultura popular está presente em diversas
expressões artísticas e não está amarrada a uma determinada escola ou
estilo. Ela é atemporal, pois é de todos os tempos. Em geral, associamos
o popular a algo sem sofisticação, pitoresco, rudimentar e carregado de
certo exotismo, sempre de forma pejorativa. É comum também separar, de
maneira quase delimitada, as manifestações produzidas pela cultura
popular e pela cultura erudita. Nada mais enganoso. Embora sejam
distintos em vários aspectos, o popular e o erudito estão imbricados de
maneira dinâmica e, assim, se contagiam mutuamente em fronteiras muito
tênues.

Na arte brasileira, o modernismo, especialmente através da produção
literária e ensaística de Mário de Andrade, resgatou a importância da
cultura popular na formação da nossa sociedade, tornando-se elemento
fundamental na busca pela identidade nacional, um dos pressupostos do
programa defendido pelo movimento. Por isso, esta atividade pode
contribuir não apenas para a área de língua portuguesa, mas para outras
áreas das ciências humanas e sociais aplicadas, como a sociologia, a
geografia, a história e a filosofia.

\paragraph{Metodologia} Estabeleçam uma conversa inicial sobre o que seria
cultura popular e o que a diferencia da cultura erudita, além de tentar
apontar as relações entre elas. Dentro dessa discussão, pode ser
colocada também a questão entre arte e artesanato, artista e artesão, o
que eles possuem de semelhante e no que se distanciam. Você pode propor
esse debate a partir dos filmes relacionados a seguir, ou qualquer um de
sua escolha, desde que esteja dentro da temática de nossa atividade:

\begin{itemize}
\item
  \emph{Moacir Arte Bruta} (2005), de Walter Carvalho;
\item
  \emph{O Auto da Compadecida} (2000), de Guel Arraes;
\end{itemize}

Peça para que eles redijam um texto curto, de no máximo 2 páginas,
apontando os principais elementos que caracterizam o que chamamos de
cultura popular, arte e artesanato, no que eles dialogam e se
distinguem. O primeiro filme trata de um artista pobre do interior do
Brasil com deficiência auditiva, que desenha desde os sete anos de
idade. As figuras carregam certo grau de primitivismo, mas possuem
traços de extrema originalidade. O segundo é mais conhecido do público
geral. Baseado na peça homônima de Ariano Suassuna, ele apresenta as
desventuras de uma dupla de sertanejos pobres e espertalhões para
sobreviver, em um vilarejo tomado por uma galeria de personagens que
mobilizam questões sobre tradição, banditismo do cangaço, pobreza e
religiosidade.

Depois de cumprida essa etapa, organize a turma em grupos e oriente-os a
apresentar um trabalho sobre qualquer manifestação de cultura popular. A
forma de apresentação dos trabalhos também é livre. Pode ser através de
texto, música, fotografia, dança, teatro e vídeo. A utilização de
recursos das plataformas digitais e redes sociais deve ser incentivada,
o que pode promover um debate interessante sobre a circulação,
preservação e transformações da própria categoria do que é popular na
era da cultura de massas.

\textbf{Tempo estimado:} três aulas de 50 minutos.


\subsection{Leitura I}


\paragraph{Tema} Modernismo, modernidade e modernização.

\paragraph{Conteúdo} Discussão e reflexão sobre as relações entre
modernismo, modernidade e modernização, categorias que, embora se
assemelhem na grafia, se referem a processos diferentes tanto em
perspectiva teórica quanto histórica.

\paragraph{Objetivo} Habilitar os estudantes a compreender as diferenças e
relações entre modernismo, modernidade e modernização, a partir dos
contos de Mário de Andrade e do seu projeto estético.

\paragraph{Justificativa} Modernismo, modernidade e modernização, de
alguma maneira, se encontram, e não apenas em termos fonéticos e
gráficos. São palavras semelhantes, mas que se referem a categorias bem
distintas. Evidentemente elas se referem a condições que dialogam entre
si, no entanto, é fundamental conhecer a especificidade de cada uma para
compreender as maneiras como elas se articulam em uma perspectiva não
apenas histórica, mas também social, política e cultural.

Para identificar e discutir essas relações, é fundamental que os
estudantes compreendam os mecanismos que mobilizam cada um dos conceitos
e os localizem dentro do panorama de formação da nossa sociedade. Os
contos de Mário de Andrade podem ser um instrumento de apreensão das
fronteiras entre essas categorias.

\paragraph{Metodologia} Proponha uma discussão bem geral sobre os
conceitos de modernidade, modernismo e modernização. Historicamente, a
modernidade é uma categoria que vem se desenvolvendo desde o
Renascimento e se consolidou na Era Industrial, abrangendo as várias
dimensões da vida, política, social, cultural e econômica. Discute-se,
ainda, a sua permanência ou não nos dias atuais. O modernismo se refere
a um conjunto de tendências estéticas e movimentos artísticos, com
origem na Europa, que vicejaram no início do século XX, mas que ainda
apresenta ressonâncias na arte contemporânea. E a modernização trata de
um processo de expansão e desenvolvimento urbano da sociedade,
especialmente no final do século XIX e no início do século XX.

A partir dos contos de Mário de Andrade, solicite aos alunos que
desenvolvam trabalhos que discutam essas categorias e a forma como o
escritor constrói sua narrativa, atentando não apenas para a forma em
si, mas também para o enredo, os personagens, o espaço e o tempo que
compõe a história. Como recomendação, os contos ``O besouro e a rosa'' e
``Primeiro de maio'' levantam importantes questões para se discutir as
contradições que os processos envolvidos nas categorias de modernidade e
modernização carregam, além de servirem como ótimos exemplos de prosa
modernista. Como orientação para os trabalhos, peça para levem em
consideração os seguintes tópicos na discussão dos textos selecionados:

\begin{enumerate}
\item
  Quais são as consequências do processo de modernização do Brasil, em
  termos sociais e econômicos?
\item
  Qual é a importância do Primeiro de Maio na história do Brasil?
\item
  Como as mudanças na sociedade transformam a configuração das classes
  sociais no país e de que maneira isso está vinculado aos processos de
  modernização que ele atravessou?
\end{enumerate}

A forma como cada trabalho será apresentado é livre, mas é importante
que os educandos se mantenham atentos ao texto de Mário de Andrade,
propondo, inclusive, novas questões a partir dos tópicos sugeridos
acima.

\textbf{Tempo estimado:} três aulas de 50 minutos.

\subsection{Leitura II}

\textbf{Tema} As representações de classe social nos contos de Mário de Andrade.

\paragraph{Conteúdo} Discussão e reflexão sobre as representações de
classe social nos contos de Mário de Andrade, especialmente os segmentos
menos prestigiados socialmente.

\paragraph{Objetivo} Habilitar os estudantes a compreender as dinâmicas na
configuração das classes sociais e dos processos históricos envolvidos,
considerando os fatores sociais, econômicos, políticos e culturais.

\paragraph{Justificativa} Com o processo de urbanização e a construção de
um estado moderno, no início do século XX, o Brasil passou a transitar
para um modelo de sociedade urbana e industrial. Essa nova configuração,
com profundas consequências sociais, políticas e econômicas até hoje,
ensejou também a formação de classes sociais com profundas disparidades
entre si, demandando novas maneiras de relacionamento e organização.

Os contos de Mário de Andrade trazem à tona os indivíduos lançados à
margem dessas transformações, dando corpo e voz a eles através dos
recursos narrativas de que lança mão, colocando em xeque alguns aspectos
do desenvolvimento urbano vivenciado pelo país no período, nas décadas
de 1920 e 1930, visto com entusiasmo por outros intelectuais. Ao mesmo,
Mário promove uma valorização desses indivíduos em sua dimensão
cultural, social e psicológica.

\paragraph{Metodologia} A partir dos textos extraídos do livro
\emph{Contos de Belazarte} e inseridos no material literário, proponha
aos estudantes que descrevam como os personagens são caracterizados
através dos espaços em que residem, do modo como falam e como são
observados pelo narrador. A presença dos imigrantes pobres moradores da
periferia é um dos traços que atravessa essas histórias, como em ``Caim,
Caim e o resto''. Portanto, como exercício de análise e discussão,
proponha que os educandos façam uma pesquisa sobre a imigração no
Brasil, especialmente dos camponeses, operários e outros grupos que
chegaram até aqui, entre outros motivos, para fugir da Segunda Grande
Guerra.

Outra figura presente no livro de Mário de Andrade, como em ``O besouro
e a rosa'', é a da criada, tratada como alguém ``da família'' por
membros da classe mais prestigiada socialmente, mas, ao mesmo tempo,
colocada em seu lugar de subalternidade. Para enriquecer a discussão,
apresente o filme \emph{Que horas ela volta?} (2015), de Anna Muylaert,
e proponha aos alunos uma discussão sobre a precarização das relações de
trabalho e a exploração da atividade doméstica. Eles podem apresentar
uma análise a partir do fenômeno do ``êxodo rural'', entre as décadas de
1960 e 1970, e buscar apresentar as consequências nas configurações da
vida nas cidades e no campo.

Tanto a história da imigração quanto a do êxodo rural podem ser
apresentadas como roteiros de um filme ou peça de teatro. Você pode
discutir com a turma a melhor maneira de apresentação, de acordo com a
disponibilidade de espaço e recursos. Se puder ser das duas maneiras,
ótimo! Quanto mais formas de expressão do conhecimento, mais ainda
ganhamos todos, não é mesmo?

\paragraph{Tempo estimado} quatro aulas de 50 minutos.

\subsection{Pós-leitura}

\paragraph{Tema} A construção e a busca por uma identidade nacional.

\paragraph{Conteúdo} Produção e curadoria de uma exposição fotográfica que
retrate a diversidade étnico-cultural da sociedade brasileira, seja
através de pesquisa em acervos digitais, seja através de registro
próprio feito pelos alunos, a fim de refletir e discutir sobre a
construção e a busca por uma identidade nacional.

\paragraph{Objetivo} Habilitar os estudantes a compreender as dimensões
histórica, social, cultural e política na busca por uma identidade
nacional.

\paragraph{Justificativa} Em uma de suas canções mais famosas, Cazuza
pede: ``Brasil, mostra a tua cara''. Afinal, qual é a nossa cara? Qual é
a fotografia 3x4 no documento de registro geral do nosso país? Em uma
sociedade tão plural como a nossa, formada por diversas matrizes
distintas, como a europeia, a africana e a indígena, não é fácil definir
um rosto, o nosso retrato, uma identidade que reúna todos os elementos
que a compõem. Ou seria a nossa identidade esse mosaico de tendências,
cores e estilos distintos? O movimento modernista, em especial a obra de
Mário de Andrade, se ocupou, através da arte e da cultura, a pensar e
buscar essa identidade, a substância daquilo de que fomos feitos.

Como os conceitos de cultura e identidade são dinâmicos, as
transformações da sociedade brasileira depois do modernismo também
contribuíram para modificar nosso olhar diante do espelho. Tendo isso em
vista, a fotografia pode ser um instrumento poderoso para refletir sobre
essas relações, ainda mais nos dias atuais, em que seu uso é quase tão
natural quanto acordar e levantar pela manhã. E não apenas por meio da
câmera do celular, mas também pelo número quase ilimitado de acervos
fotográficos digitais, seja em páginas especializadas, seja nas redes
sociais. A imagem está presente em tudo, o que, de uma certa maneira,
pode também modificar a nossa percepção e o nosso diálogo com aquilo que
ela busca representar.

\paragraph{Metodologia} 
\begin{enumerate}
\item
Realizar levantamento de elementos que podem
constituir uma identidade, ou identidades, do Brasil, não apenas
evidenciando os tipos humanos, mas também relacionando espaços,
manifestações culturais, gastronomia, religiosidade, entre outros.
Utilize os contos da antologia de Mário de Andrade como mote para o
trabalho, como, por exemplo, ``Briga das pastoras'', ``Jaburu malandro''
e ``Primeiro de maio''. Oriente os estudantes a primeiro estabelecer uma
lista e definirem, em grupo ou individualmente, um tema, a fim de
montarem a exposição de imagens seguindo uma ótica bem delimitada e
poderem aprofundar a discussão;

\item As imagens podem ser produzidas a partir de duas metodologias
diferentes. A escolha vai depender de alguns fatores, como a
disponibilidade de recursos de cada um, o acesso a determinados
materiais ou as preferências do aluno ou grupo envolvido no trabalho;

\item A primeira opção é mais evidente. Os próprios alunos tiram as suas
fotografias, utilizando a câmera do celular mesmo, ou qualquer outra
máquina que possuam. Uma aula pode ser reservada a discutir alguns
aspectos básicos da fotografia, apresentando exemplos de grandes
fotógrafos brasileiros e estrangeiros, especialmente os que realizaram
trabalhos etnográficos com a fotografia, como Sebastião Salgado, Claudia
Andujar e Pierre Verger;

\item Outra opção é selecionar imagens de acervos digitais. Claro que os
estudantes podem selecioná-las da base de dados do Google Imagens, mas
procure incentivá-los a pesquisar as fotografias nos acervos do Arquivo
Nacional, da Biblioteca Nacional, do Instituto Moreira Salles, dos
Arquivos Públicos de alguns estados, como São Paulo e Minas Gerais, de
jornais, como \emph{O Globo} e \emph{Estadão}, ou seja, que explorem os
acervos públicos e privados mantidos por essas instituições. As redes
sociais, como Instagram, também possuem um rico acervo de imagens de
fotógrafos, renomados ou anônimos. É importante destacar: deve-se sempre
citar a fonte e o nome do fotógrafo;

\item A montagem da exposição deve vir acompanhada de um texto, informando
a temática, a metodologia, os acervos pesquisados, se for o caso, e uma
reflexão dos estudantes sobre o conjunto de imagens e de que maneira
elas ilustram a noção de identidade nacional;

\item A exposição pode ser apresentada fisicamente, com a impressão das
imagens e montagem em um espaço na escola, ou virtualmente, em um blog,
e-book ou perfil no Instagram.
\end{enumerate}

\textbf{Tempo estimado:} um bimestre de aulas.

\section{Aprofundamento}

\begin{verse}
Na rua Aurora eu nasci\\
Na aurora de minha vida\\
E numa aurora cresci. 

No largo do Paiçandu\\
Sonhei, foi luta renhida,\\
Fiquei pobre e me vi nu. 

Nesta Rua Lopes Chaves\\
Envelheço, e envergonhado.\\
Nem sei quem foi Lopes Chaves. 

Mamãe! me dá essa lua,\\
Ser esquecido e ignorado\\
Como esses nomes de rua.
\end{verse}

Publicados em
\emph{Lira paulistana}, em 1945, estes versos traçam uma espécie de
síntese da vida de Mário de Andrade, um dos maiores escritores da
literatura brasileira e um dos principais representantes do modernismo.
Não é tarefa simples descrever em poucas linhas alguém tão múltiplo e
diverso como ele, mas vamos tentar.

Mário teve a sua ``aurora'' no dia 9 de outubro de 1893, na cidade de
São Paulo. Desde cedo, demonstrou talento para a música, se destacando
como pianista. Também foi autodidata! Ele se dedicou ao estudo da
literatura, da pintura e de outras artes, mas foi a poesia que o pegou
em cheio. Quase uma paixão à primeira vista, ou melhor, à primeira
leitura, talvez...

Seu primeiro livro foi \emph{Há uma gota de sangue em cada poema}
(1917), que ele assinou com o pseudônimo de Mário Sobral. Nele, ainda
estão presentes as influências do simbolismo e do parnasianismo, mas já
aparecem nesse livro um pouco da sua ``cara'' como escritor modernista.
Poucos anos depois, ele se engajou no modernismo, movimento que, no
começou, se opôs radicalmente a essas estéticas anteriores, influenciado
pelas vanguardas artísticas europeias.

Em 1922, Mário publicou o livro de poemas \emph{Pauliceia desvairada},
sua obra-manifesto, no mesmo ano em que trabalhava na organização de um
dos eventos mais importantes da vida intelectual e cultural brasileira
no século XX, a Semana de Arte Moderna, ocorrida entre os dias 11 e 18
de fevereiro de 1922, no Theatro Municipal de São Paulo, e que contou
ainda com a participação de artistas como Oswald de Andrade, Anita
Malfatti e Heitor Villa-Lobos.

Ainda nos anos 1920, Mário pega o caminho da prosa de ficção. E quando
falamos de prosa na literatura de Mário, do que costumamos nos lembrar?
De \emph{Macunaíma}, claro, que ele lançou em 1928, e também de
\emph{Amar, verbo instransitivo}, que saiu um ano antes. São duas
obras-primas, sem dúvida alguma, duas das mais importantes da literatura
brasileira, mas Mário também foi um excelente contista. E muitos dos
temas e inquietações que ele tinha como artista e intelectual, estão lá
em suas narrativas curtas, suas pequenas histórias que são também
grandes na sua importância para a literatura do nosso país.

Os textos da antologia são, respectivamente, de \emph{Primeiro andar}
(1926), \emph{Os contos de Belazarte} (1934) e \emph{Contos novos}
(1947). Os títulos citados trazem contos que expõem, mobilizam e
discutem aspectos importantes para a concepção artística de Mário de
Andrade.

\emph{Primeiro andar} faz parte dos seus escritos da juventude. O livro
foi publicado pela Casa Editora Antonio Tisi, em 1926, e depois pela
Editora Piratininga em 1932, mas sem nenhuma alteração nos textos.
Naquele ano, no planejamento de suas \emph{Obras completas} para a
Livraria Martins Editora, publicadas apenas em 1960, Mário incluiu
\emph{Primeiro andar} em reunião intitulada \emph{Obra imatura}, em que
também estão o seu livro de estreia, sobre o qual falamos há pouco,
\emph{Há uma gota de sangue em cada poema}, e o ensaio \emph{A escrava
que não é Isaura}. Os contos que escolhemos de \emph{Primeiro andar} são
``Conto de Natal'', ``História com data'', ``Galo que não cantou'' e
``Briga das pastoras''. Neles, o escritor se lança em experimentações de
procedimentos e limites da linguagem e da estrutura do próprio gênero,
percorrendo vários temas, como quem ainda está investigando as
possibilidades da narrativa moderna e buscando se afastar dos modelos
tradicionais. Alguns acabaram servindo, inclusive, como ponto de partida
para outras obras, como ``História com data''. Além da intertextualidade
com um dos clássicos do conto nacional, o livro \emph{Histórias sem
data}, de Machado de Assis, ele também traz questões que são
aprofundadas em \emph{Macunaíma}, como a ideia de que o fragmentário e o
desencaixado pode constituir a identidade.

As histórias de \emph{Os contos de Belazarte} nasceram a partir das
\emph{Crônicas de Malazarte}, publicadas entre outubro de 1923 e julho
1924 na revista \emph{América Brasileira}, em que o narrador e
personagem Belazarte surge pela primeira vez. Para a antologia,
escolhemos as seguintes histórias: ``O besouro e a rosa'', ``Jaburu
malandro'', ``Caim, Caim e o resto'' e ``Piá não sofre? Sofre''.
``Belazarte me contou'' é o recurso narrativo com que o escritor inicia
cada uma das histórias do livro. Mário de Andrade aprofunda as
experiências com a linguagem, incorporando a oralidade na narração e na
fala das personagens e retratando os conflitos de classe que se
aprofundam com a modernização do país.

Por fim, temos \emph{Contos novos}, publicação depois de sua morte.
Quando retornou a São Paulo, em 1941, depois de uma temporada morando no
Rio de Janeiro, Mário começou a trabalhar em contos que ele escreveu
entre 1920 e 1930, intitulados pelo autor de \emph{Contos piores} e
\emph{Contos redivivos.} Infelizmente, o contista faleceu antes de
concluir o trabalho. Reunimos os seguintes contos na antologia:
``Primeiro de maio'', ``O poço'', ``O peru de Natal'' e ``Nelson''.
Neste último livro de narrativas curtas, temos um contista mais maduro,
com a linguagem revitalizada, explorando a dimensão psicológica e a
dramaticidade na constituição das histórias e no conflito das
personagens.

A importância dos contos está na valorização da cultura popular e na
exposição das contradições da modernização. Eles apresentam um retrato
multifacetado do Brasil e de sua formação sociocultural, sem deixar de
destacar seus problemas. Os contos são atuais pela sua linguagem e pela
maneira como expõem a diversidade e as desigualdades que formam o país.
A importância da cultura popular para a compreensão da nossa identidade
também merece destaque.

Ao valorizar o ritmo da oralidade na linguagem escrita e a cultura
popular, os contos de Mário de Andrade dialogam com a produção dos
principais escritores do século XX, como Jorge Amado e Guimarães Rosa. O
tema dos conflitos de classe e dos contrastes regionais do país pode
encontrar ressonância em autores contemporâneos como Marcelino Freire,
Maria Valéria Rezende e Luiz Ruffato.

\section{Referências complementares}

\subsection{Audiovisual}

\begin{enumerate}
\item
  \emph{Imagens do Estado Novo} (2016), direção de Eduardo Escorel;
\item
  \emph{A marvada carne} (1985), direção de André Klotzel;
\item
  \emph{Macunaíma} (1969), direção de Joaquim Pedro de Andrade;
\item
  \emph{Tempos modernos} (1936), direção de Charles Chaplin;
\item
  \emph{Mário de Andrade -- Reinventando o Brasil} (2001), da série
  ``Mestres da Literatura'', produzida pelo Ministério da Educação;
\item
  \emph{AmarElo -- É tudo pra ontem} (2020), direção de Fred Ouro Preto.
\end{enumerate}

\subsection{Musical}

\begin{enumerate}
\item
  \emph{Sobrevivendo no Inferno} (1997), Racionais MC's;
\item
  \emph{Perfil de São Paulo} (2000), Inezita Barroso;
\item
  \emph{Macunaíma Ópera Tupi} (2008), Iara Rennó.
\end{enumerate}

\subsection{Artes visuais}

\begin{enumerate}
\item
  \emph{Coleção Mário de Andrade -- Artes plásticas} (1998), Marta
  Rosseti Batista e Yone Soares de Lima;
\item
  \emph{Coleção Mário de Andrade -- Religião e magia, música e dança,
  cotidiano} (2004), Marta Rosseti Batista
\end{enumerate}

\section{Bibliografia comentada}

\begin{enumerate}
\item
  Andrade, Mário de. ``Contos e contistas''. In: \_\_\_\_\_. \emph{O
  empalhador de passarinho}. Rio de Janeiro: Nova Fronteira, 2012.
  Artigo com uma reflexão de Mário de Andrade sobre o gênero conto, a
  partir de uma enquete encomendada pela \emph{Revista Acadêmica} para
  indicar os dez melhores contos brasileiros.
\item
  Castro, Moacir Werneck de. \emph{Mário de Andrade}: exílio no Rio.
  Belo Horizonte: Autêntica, 2016. Depoimento pessoal de Moacir Werneck
  de Castro sobre sua relação afetiva e intelectual com Mário de Andrade
  durante o período em que este residiu no Rio de Janeiro, entre 1938 e
  1941. A amizade de Mário com os chamados ``Rapazes da
  \emph{Acadêmica}'', em referência à \emph{Revista Acadêmica}, exerceu
  grande influência recíproca entre eles. Além de Castro, o grupo era
  formado também por Murilo Miranda, Lúcio Rangel e Carlos Lacerda. O
  livro traz ainda as cartas de Mário para o autor.
\item
  Moraes, Marcos Antonio de (org.). \emph{Correspondência de Mário de
  Andrade e Manuel Bandeira}. São Paulo: Edusp, 2001. Reunião das cartas
  trocadas entre os dois poetas do modernismo brasileiro, de 1922 a
  1945, revelando os bastidores da criação literária, a amizade e os
  rumos do movimento modernista. Baseado em exaustiva pesquisa
  documental, o livro traz ainda dossiê de fotografias e fac-símiles de
  textos relacionados aos dois escritores.
\item
  Propp, Vladimir. \emph{Morfologia do conto maravilhoso}. Trad. Jasna
  Paravich Sarhan. Rio de Janeiro: Forense Universitária, 2006. A obra
  de Vladimir Propp se dedica a definir o conceito de conto maravilhoso
  a partir de estruturas narrativas observadas na análise de contos
  populares russos.
\item
  Souza, Eneida Maria de; Cardoso, Marília Rothier. \emph{Modernidade
  toda prosa}. Rio de Janeiro: PUC-Rio; Casa da Palavra, 2014. As
  autoras são duas das mais importantes pesquisadoras sobre o modernismo
  brasileiro. O livro inova ao ampliar o conceito de prosa e suas
  ressonâncias a partir do modernismo, abordando não apenas os gêneros
  literários em prosa, como o romance e o conto, mas também expressões
  como o cinema e as artes plásticas.
\item
  Tércio, Jason. \emph{Em busca da alma brasileira}: biografia de Mário
  de Andrade. Rio de Janeiro: Estação Brasil, 2019. A premiada biografia
  de Mário de Andrade é mais do que a reconstituição de episódios
  decisivos de sua vida, mas uma reflexão sobre aspectos importantes da
  vida cultural e política brasileira da primeira metade do século XX. O
  jornalista Jason Tércio se debruçou em farta documentação nos
  arquivos, dando consistência ao seu trabalho.
\end{enumerate}


\end{document}

