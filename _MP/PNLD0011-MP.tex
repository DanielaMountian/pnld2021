\documentclass[12pt]{extarticle}
\usepackage{manualdoprofessor}
\usepackage{fichatecnica}
\usepackage{lipsum,media9,graficos}
\usepackage[justification=raggedright]{caption}
\usepackage{bncc}
\usepackage[relicario]{../edlab}

 

\begin{document}

\newcommand{\AutorLivro}{Bram Stoker}
\newcommand{\TituloLivro}{Sob o por do sol}
\newcommand{\Tema}{Ficção, mistério e fantasia}
\newcommand{\Genero}{Romance}
% \newcommand{\imagemCapa}{PNLD0011-01.png}
\newcommand{\issnppub}{---}
\newcommand{\issnepub}{---}
% \newcommand{\fichacatalografica}{PNLD0011-00.png}
\newcommand{\colaborador}{\textbf{Michelle Etienne Florence, Bruno Gradella e Vicente Castro} é uma pessoa incrível e vai fazer um bom serviço.}


\title{\TituloLivro}
\author{\AutorLivro}
\def\authornotes{\colaborador}

\date{}
\maketitle
\tableofcontents

\pagebreak

\section{Carta aos professores}

Caro educador / Cara educadora,\\\bigskip

Este Manual tem como objetivo fornecer subsídios para o trabalho com a
obra literária \emph{Sob o Pôr do Sol}, de Bram Stoker.

Neste material, são propostas atividades de leitura do texto literário,
em perspectiva interdisciplinar, a partir de propostas de abordagem
envolvendo a integração de diferentes áreas do conhecimento. Todas as
etapas de pré"-leitura, leitura e pós"-leitura atribuem aos professores o
papel de mediadores entre os estudantes e a obra literária. Aos alunos,
por sua vez, é conferido um lugar de protagonismo e autonomia na
construção do conhecimento, a partir do emprego de metodologias ativas e
estratégias de aprendizagem criativa.

Seguindo as competências e habilidades indicadas na nova Base Nacional
Comum Curricular (\textsc{bncc}), o trabalho com o texto literário é desenvolvido
no âmbito dos diferentes campos de atuação social. Para isso, levam"-se
em consideração os campos da vida pessoal, de atuação na vida pública,
das práticas de estudo e pesquisa, jornalístico"-midiático e, sobretudo,
artístico"-literário.

Cada uma das seções do Manual sugere atividades e apresenta informações
complementares para enriquecer a experiência de leitura do texto
literário. A partir de uma proposta dialógica de ensino de literatura,
procura desenvolver habilidades de leitura e produção de textos, visando
à formação do sujeito leitor"-autor competente, capaz de interagir com o
mundo e de atribuir sentido às próprias vivências.

Reforçando o caráter formativo e informativo da literatura, o material
procura articular a formação de leitores à elaboração de projetos de
vida, a partir da ampliação do repertório artístico"-cultural dos
estudantes. A leitura crítica da obra literária é concebida em sentido
amplo e envolve o estabelecimento de relações entre textos pertencentes
a esferas variadas de comunicação e a gêneros discursivos diversos.

Ao mesmo tempo, a releitura de clássicos literários traz, para a
contemporaneidade, novas possibilidades de significação para obras que
permanecem atuais. Para contribuir com essa aproximação entre jovens
leitores e obras consagradas, são propostas atividades que empregam as
novas tecnologias de informação e comunicação, fundamentais para a
formação de leitores inseridos na cultura digital. Valorizam"-se,
portanto, estratégias de leitura e produção textuais no âmbito da
hipertextualidade e do multiletramento.

Para a formação continuada de professores, apresentamos sequências que
estimulam a criatividade e a inovação, com possibilidades de adaptação
às diferentes realidades de ensino. Orientações de caminhos possíveis
são apresentadas como sugestões para as atividades de leitura em sala de
aula, sempre permeáveis aos diferentes perfis de grupos e às
especificidades de gostos e repertórios culturais.

Conforme a \textsc{bncc} assinala, ``no Ensino Médio, os jovens intensificam o
conhecimento sobre seus sentimentos, interesses, capacidades
intelectuais e expressivas; ampliam e aprofundam vínculos sociais e
afetivos; e refletem sobre a vida e o trabalho que gostariam de ter.
Encontram"-se diante de questionamentos sobre si próprios e seus projetos
de vida, vivendo juventudes marcadas por contextos socioculturais
diversos'' (Brasil, 2018, p.481).

Nada mais adequado, portanto, que oferecer textos literários capazes de
estimular reflexões sobre a vida pessoal e rotas possíveis para os
projetos de vida. Para você, professor(a), abrem"-se novas trilhas para o
contato sempre renovado com obras que são, a um só tempo, atuais e
atemporais.

Boa jornada!


\section{Atividades 1}

\subsection{Pré"-leitura}

\BNCC{EM13LGG302}
\BNCC{EM13LGG704}
\BNCC{EM13LP10}
\BNCC{EM13LP19}

Antes da leitura, levante, em uma conversa com os alunos, os
conhecimentos desses acerca do gótico, a polissemia da palavra e qual o
significado da mesma nas artes, bem como suas aplicações nessa esfera.
Feito isso, a atividade sugere que o professor introduza os alunos no
universo da prosa gótica em língua portuguesa, por meio do contato com
autores brasileiros e portugueses, como, por exemplo, Álvares de Azevedo
e Soares de Passos, por exemplo. Sugere"-se, para o desenvolvimento desta
atividade e de atividades futuras que o aluno crie um caderno de
anotações, como uma espécie de diário de viagem, para registrar o
andamento de seu percurso discente no universo gótico.

\subsection{Leitura}

\BNCC{EM13LGG103}
\BNCC{EM13LP02}
\BNCC{EM13LP48}

Durante a leitura, estabeleça paralelos com trechos da obra
lida e fragmentos de outras composições góticas famosas. Os alunos podem
sem valer do magnum opus do autor lido, o romance Drácula, mesclando"-o
com cenas de filmes que o adaptaram, bem como de filmes e séries de
vampirescas. Por fim, proponha aos alunos que criem uma personagem com
as devidas características góticas, a partir da retratação e dos devidos
predicados consolidados por Bram Stoker.


\subsection{Pós"-leitura}

\BNCC{EM13LGG102}
\BNCC{EM13LGG303}
\BNCC{EM13LGG402}
\BNCC{EM13LGG703}
\BNCC{EM13LP13}
\BNCC{EM13LP14}
\BNCC{EM13LP28}
\BNCC{EM13LP29}
\BNCC{EM13LP52}

Após a leitura, com base no desenvolvido nas atividades
anteriores, sugira aos alunos a organização de uma coletânea de contos
de horror. Cada aluno deverá desenvolver um conto, dando o devido
destaque à personagem que criou na atividade 2, desenvolvendo"-a. A
partir disso, é provável que o aluno coloque essa personagem como
antagonista de seu conto (mas, lembre"-se, isso não é obrigatório). De
toda sorte, para permanecer no universo de Sob o Pôr do Sol, deve haver
uma dualidade clara entre protagonista e antagonista. Dessa sorte,
atente os alunos ao fato de que devem proceder com o desenvolvimento
complexo de duas personagens, para então poderem dar seguimento com a
narrativa. Com os textos escritos, é possível a realização de uma
leitura dinâmica diante da turma, apresentando o produto final.


\section{Atividades 2}

A obra \emph{Sob o Pôr do Sol} possibilita trabalhos interdisciplinares
e integradores de diferentes campos do saber e áreas de conhecimento. A
seguir, propomos algumas atividades que podem ser desenvolvidas
conjuntamente com professores de outras áreas. Além das habilidades de
Linguagens e suas Tecnologias e de Língua Portuguesa, indicadas nas
etapas da seção anterior e válidas também para esta, listamos a seguir
as habilidades de outras áreas, presentes na abordagem interdisciplinar:

\BNCC{EM13CNT201}
\BNCC{EM13CNT303}
\BNCC{EM13CHS101}
\BNCC{EM13CHS102}
\BNCC{EM13CHS106}
\BNCC{EM13CHS401}

\subsection{Pré"-leitura}

Antes da leitura, sugere"-se uma exploração do universo de
monstros e seres fantásticos, a luz da perspectiva da ciência. Convide
os professores de ciências humanas e naturais para colaborarem com os
alunos na investigação das possíveis origens de vampiros, ciclopes,
unicórnios, dragões, entre outros. Sugere"-se a observação de como alguns
temas são comuns a diversos povos e, juntamente com os alunos, procurem
investigar as possíveis razões por trás disso.


\subsection{Leitura}

É durante a leitura, professores da área de ciências humanas
podem contextualizar a história e a cultura por trás da vertente gótica
do romantismo.

Isso pode ser abordado de uma perspectiva história e social, pois, à
medida que a sociedade evoluía nos planos econômico, político e
científico"-intelectual, A burguesia encontrou no romance uma forma de
expressão literária. Tal movimento colaborou para o surgimento de novas
vertentes, entre elas a literatura gótica.

Vertente essa que costuma se destacar por cenários medievais,
personagens melodramáticos,/ pelo uso da psicologia do terror (medo,
loucura e etc.) e do imaginário sobrenatural (fantasmas, monstros,
etc.). Assim como outros movimentos literários, é interessante indicar
aos alunos que essa tendência nas artes surgiu como um contraponto ao
tecnicismo que, teve por consequência, o afastamento das pessoas do
contato mais cotidiano com o fantástico. Isso pode ensejar reflexões
sobre o poder, discussões político"-religiosas, concepções estéticas e
filosóficas.

Os autores do gênero gótico combinam o refinamento dos novos costumes
com o barbarismo e a excentricidade. Além disso, possuem a descrição
realista das ações e dos ambientes com sentimentos de desolação e
abandono.

A literatura gótica (ou horror gótico) inicia"-se no século \textsc{xviii}, na
Inglaterra, com a obra o castelo de Otranto (1764), de Horace Walpole.
Costuma"-se destacar, como algumas das principais características desse
tipo de literatura, os cenários medievais (castelos, igrejas,
cemitérios, florestas, ruínas), os personagens melodramáticos (donzelas,
cavaleiros, vilões, os criados), os temas e símbolos recorrentes
(segredos do passado, manuscritos escondidos, profecias, maldições).

Outros lugares comuns da literatura gótica envolvem destacar nos
romances o uso da psicologia do terror (o medo, a loucura, a devassidão
sexual, a deformação do corpo), do imaginário sobrenatural (fantasmas,
demônios, espectros, vampiros, bruxas, monstros).

Uma curiosidade que vale ser ressaltada, é o fato de um grande número de
novelas do gênero gótico ter sido escrito por mulheres, algo até então
incomum.

Na Europa, nomes como Charles Baudelaire e o poeta e Lord Byron ganharam
destaque.

Nos estados unidos ficou a cargo de Edgar Allan Poe, através de contos,
poemas e romances, representar o estilo.

A literatura gótica acabou sendo introduzida e representada no Brasil
por Álvares de Azevedo, com \emph{Noite na Taverna} e \emph{Macário}.

Assim como no estrangeiro, os temas do autor brasileiro eram o obscuro e
a proximidade da morte com o amor.~

Por fim no Brasil, além de Álvares de Azevedo, Augusto dos Anjos e
Alphonsus de Guimarães compreendem a literatura gótica como poucos.

O termo \emph{ficção gótica} é utilizado para incluir os filmes góticos,
quadrinhos e demais mídias que utilizam as convenções da literatura
gótica.

\subsection{Pós"-leitura}

Após trabalhar com o repertório oferecido na atividade
anterior, os alunos podem utilizar as anotações em seu diário de jornada
para, dotados das informações literárias e científicas recebidas,
redigir, conjuntamente, textos de apoio ao livro de contos produzido na
atividade 3. São esses, um prefácio, uma introdução, um glossário, um
epílogo. Atente"-os ao formato e ao conteúdo que se espera sejam
produzidos em cada um desses segmentos. Feito isso, os contos podem ser
compilados em um material a compor o acervo da biblioteca, podendo ser
veiculado, também, digitalmente.

\section{Aprofundamento}

Ao chegar ao Ensino Médio, é necessário que os estudantes se aprofundem
na compreensão das múltiplas linguagens e, sobretudo, da linguagem
literária. Em relação à literatura, a \textsc{bncc} traz as seguintes
considerações:

\begin{quote}
{[}\ldots{}{]} a leitura do texto literário, que ocupa o centro do trabalho
no Ensino Fundamental, deve permanecer nuclear também no Ensino Médio.
Por força de certa simplificação didática, as biografias de autores, as
características de épocas, os resumos e outros gêneros artísticos
substitutivos, como o cinema e as \textsc{hq}s, têm relegado o texto literário a
um plano secundário do ensino. Assim, é importante não só (re)colocá"-lo
como ponto de partida para o trabalho com a literatura, como
intensificar seu convívio com os estudantes. Como linguagem
artisticamente organizada, a literatura enriquece nossa percepção e
nossa visão de mundo. Mediante arranjos especiais das palavras, ela cria
um universo que nos permite aumentar nossa capacidade de ver e sentir.
Nesse sentido, a literatura possibilita uma ampliação da nossa visão do
mundo, ajuda"-nos não só a ver mais, mas a colocar em questão muito do
que estamos vendo/vivenciando. (Brasil, 2018, p. 491)
\end{quote}

Nesta seção, desenvolvemos um trabalho de aprofundamento que, em diálogo
com a formação continuada de professores, oferece subsídios para a
abordagem do texto literário.

\subsection{A obra}

\emph{Sob o Pôr do Sol} foi o primeiro trabalho de certa relevância
concluído por Bram Stoker. E é curioso pensar que o autor de Drácula
tenha produzido um livro cujas histórias sejam tão próximas aos contos
infantis.

A coletânea de histórias, suavemente conectadas, apresentam ao leitor um
universo de fantasia e fascinação. Nelas, somos apresentados a uma
narrativa cativante, em que vemos o surgimento de conflitos e suas
posteriores soluções, com um desfecho moralizador, no sentido que sempre
o justo é recompensado e o mau, punido.

\subsection{Um ferreiro de contos}

A obra, contudo, não deve ser tomada como mais um livro infantil da Era
Vitoriana, ainda que muitas das suas histórias nos aproximem dos contos
de fadas, das obras compiladas por Grimm, ou produzidas por Andersen,
há, em \emph{Sob o Pôr do Sol}, a presença clara da jornada do herói. O
caráter da luta, do enfrentamento, e da vitória pela superação é
marcante em Stoker, já denotando um tom de maior complexidade na
construção das personagens, especialmente nos seus sentimentos mais
íntimos, que caracterizariam as obras futuras do autor.

É possível, fazendo uma analogia, dizer que Stoker tomou a estrutura dos
contos de fada, derreteu e a colocou na forja e, tal como o ferreiro,
produziu um novo item, mais robusto, firme e potente.

Os conflitos, as expressões intensas e toda a concepção dramática que
atingiriam o ápice em Drácula, já despontavam, como primeiros brotos de
uma densa floresta em \emph{Sob o Pôr do Sol}.

\subsection{As inspirações em sua vida pessoal para o \textit{debut} de um grande autor}

Ao se ler a presente obra, o primeiro sucesso de Bram Stoker, temos o
vislumbre do desenvolvimento do autor, com todos os seus valores
literários.

Nessa obra, em verdade, enxergamos Stoker saindo do seu indivíduo
pessoal e construindo seu indivíduo autor.

Stoker fora acometido por uma doença rara quando tinha cerca de sete
anos de idade, sendo obrigado a se isolar dos convívios sociais durante
esse período, sendo obrigado a restar em sua habitação durante esse
tempo, tendo como passatempo apenas a leitura de alguns contos de fadas
e assuntos de seu interesse. Assim, vemos traços da sua biografia nas
histórias contadas, especialmente no ambiente obscuro desenvolvido.
Afinal, tratava"-se de uma criança, ouvindo histórias de aventuras
fantásticas, sem poder sair e ver diretamente a luz do sol. As histórias
a eles contadas eram feixes de luz que adentravam na noite que o jovem
Stoker singrava, entretanto, ao adentrar em seu universo, marcavam
também a existência das sombras que perseguiam o autor.

Nas letras, Stoker age como um fotógrafo, propositalmente alterando o
tempo de captura de luz do obturador, em uma brincadeira focal em um
jogo de luz e sombras.

O vívido de sua narrativa e o jogo de iluminação e obscurantismo que o
autor produz tornam difícil a classificação de sua obra como um conto de
fadas trivial.

Por vezes, o tom moralizador da obra é por demais contundente. E as
punições do mal e do pecaminoso são violentamente severas. Talvez, daí o
forte interesse que Stoker despendeu em Vlad ``Tepes'' Draculesti, sua
inspiração para Drácula.

Ademais, no seu tempo no claustro, tomou contato com a literatura
gótica, que havia surgido recentemente e estava sendo muito consumida. O
ócio de uma longa doença forneceu a oportunidade a muitos pensamentos
que foram frutíferos.

Tanto que esse dado biográfico está refletido em diversos temas tratados
ao longo de toda a sua obra, tais como a impossibilidade da fala e dos
movimentos, como se vivesse incessantemente em um sonho lúcido.

Sendo um homem religioso, os temas cristãos também surgem na obra de
Stoker. O triunfo total do bem sobre o mal, a forte figura da família e
da lealdade entre seus membros são exemplos da característica cristão
presente no microcosmo criado em \emph{Sob o Pôr do Sol}.

\subsection{\textit{Ad astra per aspera}}

\emph{Ad astra per aspera}, é a expressão latina que significa, pelas
dificuldades, as estrelas. Ou seja, a glória após a dificuldade. E é
isso que encontramos na obra de Stoker.

Na musicologia alemã, o brocado latino aparece sob a formulação
\emph{Kampf und Sieg}, a vitória após a luta, ou o triunfo sobre a
adversidade. Assim, encontramos consonância na obra de Stoker com a de
compositores alemães do século \textsc{xix}. A 5ª sinfonia de Ludwig van
Beethoven, com seu duro motivo sombrio e final triunfante, ou mesmo a
ópera Tristão e Isolda, de Richard Wagner, com sua tensão constante,
carregada com ambiguidade tonal, irresolução e suspensão constante até
os estampidos finais, são exemplos dessa forma artística, podendo ser
perfeitas trilhas sonoras para a leitura de \emph{Sob o Pôr do Sol}

\subsection{Por que ler \textit{Sob o pôr do sol}?}

Ao observarmos a condição que Stoker foi acometido na infância, não é
difícil imaginar que seguir de forma resignada a condução da vida tenha
sido um fardo e que, apenas a literatura, a família e a religiosidade
serviram de anteparo para a saúde mental do autor.

Assim, não é de se espantar a intensidade dos conflitos em suas
histórias, posto que ele próprio teve de passar por isso boa parte de
sua vida.

É interessante observar que as narrativas tem em si um tom pedagógico,
sendo outro elemento que as aproximam de histórias infantis. Talvez
sendo muito lúgubres e intensas à primeira vista, as histórias invocam
um dever de retidão moral no leitor.

Em \emph{Sob o Pôr do Sol} vislumbramos um reino do monstruoso e do
imensurável, em que brilhantes feixes reluzem na escuridão desse reino.
Expulsando as sombras que tentam tramar contra os protagonistas.

Em Stoker, não resta claro o cisma entre o universo infantil e o da
maturidade. Colocado em uma reflexão, é como se Stoker esticasse o
caráter infantil até o limite de sua elasticidade, criando pontos te
intersecção com o dito ``mundo adulto'', sem, no entanto, deixar sua
narrativa ser dominada por este. Dessa sorte, o termo infantil que o
classifica, não deve ser entendido como infantilizado ou pueril, ao
contrário, uma vez que a obra é rica e profunda, servindo muito bem,
também, ao público adulto.

\section{Sugestões de atividades complementares: relações dialógicas e
intertextuais}

\BNCC{EM13LP03}
\BNCC{EM13LP04}
\BNCC{EM13LP49}
\BNCC{EM13LP51}

No Ensino Médio, da mesma forma que no Ensino Fundamental, a \textsc{bncc}
organiza o trabalho com as práticas de linguagem em cinco \textbf{campos
de atuação social}. São eles: campo da vida pessoal, campo da vida
pública, campo jornalístico"-midiático, campo artístico"-literário e campo
das práticas de estudo e pesquisa.

De acordo com essa divisão, propomos na sequência um trabalho
interdiscursivo e intertextual com a obra \emph{Sob o Pôr do Sol.}

\subsection{Campo da vida pessoal}

\begin{quote}
O campo da vida pessoal pretende funcionar como espaço de articulações
e sínteses das aprendizagens de outros campos postas a serviço dos
projetos de vida dos estudantes. As práticas de linguagem privilegiadas
nesse campo relacionam"-se com a ampliação do saber sobre si, tendo em
vista as condições que cercam a vida contemporânea e as condições
juvenis no Brasil e no mundo.

Está em questão também possibilitar vivências significativas de práticas
colaborativas em situações de interação presenciais ou em ambientes
digitais e aprender, na articulação com outras áreas, campos e com os
projetos e escolhas pessoais dos jovens, procedimentos de levantamento,
tratamento e divulgação de dados e informações e o uso desses dados em
produções diversas e na proposição de ações e projetos de natureza
variada, para fomentar o protagonismo juvenil de forma
contextualizada. (\textsc{bncc}, p. 494)
\end{quote}

Para esta atividade, solicite que o aluno busque roupas e acessórios
  em seu guarda"-roupa para compor um figurino gótico. Em seguida,
  proponha a realização de um ensaio fotográfico que o ajudará na
  composição e criação de uma personagem. Ao final, conte, num pequeno
  texto, uma vivência desta personagem ao estilo de Bram Stoker.

\subsection{Campo de atuação na vida pública}

\begin{quote}
No cerne do campo de atuação na vida pública estão a ampliação da
participação em diferentes instâncias da vida pública, a defesa dos
direitos, o domínio básico de textos legais e a discussão e o debate de
ideias, propostas e projetos. {[}\ldots{}{]}

Ainda no domínio das ênfases, indica"-se um conjunto de habilidades que
se relacionam com a análise, discussão, elaboração e desenvolvimento de
propostas de ação e de projetos culturais e de intervenção social.
(\textsc{bncc}, p. 494)
\end{quote}

O Halloween é uma festividade popular que reúne componentes como o
horror, a morte, o macabro, o humor negro e o fantástico. A atividade
sugere a pesquisa das origens e influências pagãs e religiosas deste
festival e sua evolução ao longo dos tempos.

\subsection{Campo jornalístico"-midiático}

\begin{quote}
Em relação ao campo jornalístico"-midiático, espera"-se que os jovens
que chegam ao Ensino Médio sejam capazes de: compreender os fatos e
circunstâncias principais relatados; perceber a impossibilidade de
neutralidade absoluta no relato de fatos; adotar procedimentos básicos
de checagem de veracidade de informação; identificar diferentes pontos
de vista diante de questões polêmicas de relevância social; avaliar
argumentos utilizados e posicionar"-se em relação a eles de forma ética;
identificar e denunciar discursos de ódio e que envolvam desrespeito aos
Direitos Humanos; e produzir textos jornalísticos variados, tendo em
vista seus contextos de produção e características dos gêneros. Eles
também devem ter condições de analisar estratégias
linguístico"-discursivas utilizadas pelos textos publicitários e de
refletir sobre necessidades e condições de consumo.

No Ensino Médio, os jovens precisam aprofundar a análise dos interesses
que movem o campo jornalístico midiático, da relação entre informação e
opinião, com destaque para o fenômeno da pós"-verdade, consolidar o
desenvolvimento de habilidades, apropriar"-se de mais procedimentos
envolvidos na curadoria de informações, ampliar o contato com projetos
editoriais independentes e tomar consciência de que uma mídia
independente e plural é condição indispensável para a democracia.

Como já destacado, as práticas que têm lugar nas redes sociais têm
tratamento ampliado. (\textsc{bncc}, p. 494-495)
\end{quote}

As redes sociais não deixam de ser um local em que as pessoas publicam
versões fantasiosas de suas vidas e, portanto, são palco para personas
virtuais. Proponha aos alunos a reflexão sobre os perfis nas redes
sociais, sugerindo que observem se há, nesse universo, demonstração de
todo tipo de vivências inerentes à existência humana; ou se há um tipo
de \emph{romantização}, onde só existe espaço para o belo, o bom, ou o
socialmente aceito. Após essa reflexão, proponha a seguinte
ponderação: Quando um usuário ultrapassa os limites aceitos numa
publicação, ele pode ter agido talvez por dolo, talvez ele não tenha
percebido que foi antiético -- por falta de consciência ou noção sobre
o assunto --, ou por estar cego ou apaixonado defendendo determinada
causa. Nesse caso, há exposição do lado obscuro. Proponha uma
discussão sobre esse tema. Seria a rede social um tipo de narrativa
moderna, em tempo real?


\subsection{Campo artístico"-literário}

\begin{quote}
No campo artístico"-literário busca"-se a ampliação do contato e a
análise mais fundamentada de manifestações culturais e artísticas em
geral. Está em jogo a continuidade da formação do leitor literário e do
desenvolvimento da fruição. A análise contextualizada de produções
artísticas e dos textos literários, com destaque para os clássicos,
intensifica"-se no Ensino Médio. Gêneros e formas diversas de produções
vinculadas à apreciação de obras artísticas e produções culturais
(resenhas, vlogs e podcasts literários, culturais etc.) ou a formas de
apropriação do texto literário, de produções cinematográficas e teatrais
e de outras manifestações artísticas (remidiações, paródias,
estilizações, videominutos, fanfics etc.) continuam a ser considerados
associados a habilidades técnicas e estéticas mais refinadas.

A escrita literária, por sua vez, ainda que não seja o foco central do
componente de Língua Portuguesa, também se mostra rica em possibilidades
expressivas. (\textsc{bncc}, p. 495-496).
\end{quote}

O cinema acolheu o gótico com propriedade. Desde diretores
completamente adeptos ao gênero, como Tim Burton, a Francis Ford
Copolla, mais diversificado, que filmou o próprio Drácula de Bran
Stoker. Divida a sala em grupos e solicite que assistam aos seguintes
filmes: Edward Mãos de Tesoura (Tim Burton), Frankenstein de Mary
Shelley (Kenneth Branagh), O Corvo (Alex Proyas) e O Labirinto do
Fauno (Guillermo del Toro). Em seguida, proponha uma discussão a
respeito dos elementos estruturais e estéticos que os compõem.

\subsection{Campo das práticas de estudo e pesquisa}

\begin{quote}
O campo das práticas de estudo e pesquisa mantém destaque para os
gêneros e habilidades envolvidos na leitura/escuta e produção de textos
de diferentes áreas do conhecimento e para as habilidades e
procedimentos envolvidos no estudo. Ganham realce também as habilidades
relacionadas à análise, síntese, reflexão, problematização e pesquisa:
estabelecimento de recorte da questão ou problema; seleção de
informações; estabelecimento das condições de coleta de dados para a
realização de levantamentos; realização de pesquisas de diferentes
tipos; tratamento dos dados e informações; e formas de uso e
socialização dos resultados e análises.

Além de fazer uso competente da língua e das outras semioses, os
estudantes devem ter uma atitude investigativa e criativa em relação a
elas e compreender princípios e procedimentos metodológicos que orientam
a produção do conhecimento sobre a língua e as linguagens e a formulação
de regras. (\textsc{bncc}, p. 495-496)
\end{quote}

De acordo com Carl Jung, ``ninguém se ilumina imaginando figuras de
luz, mas se conscientizando da escuridão''. Em vista da afirmação do
renomado psicólogo, proponha que os alunos pesquisem sobre o arquétipo
da sombra, e discutam a importância de conhecê"-lo. Com base na
pesquisa, discutam os contos de Bram Stoker, com destaque ao
Construtor de Sombras. Ao final, solicitem que façam uma ilustração
sobre o tema.

\section{Referências complementares}

\begin{itemize}
\item\textsc{doyle} et al. \textit{A causa secreta e outros contos de horror}. São
Paulo: Boa Companhia, 2013.

Seis autores consagrados reúnem neste pequeno livro de histórias de
terror os elementos que aproximam os leitores do obscuro e do indizível.
Há contos de Edgar Allan Poe, do brasileiro Machado de Assis, entre
outros.

\item\textsc{james}, Henry. \textit{A fera na selva}. Rio de Janeiro: Rocco, 2011.

Nesta obra"-prima terrível e fascinante, o escritor inglês envolve em uma
narrativa psicológica um personagem complexo, que carrega em si a
sensação de ``uma fera emboscada na selva'', pronta a saltar sobre ele a
qualquer momento.

\item\textsc{king}, Stephen. \textit{It: a coisa}. São Paulo: Suma, 2014.

Em um verão qualquer, um grupo de sete crianças enfrenta pela primeira
vez um ser sobrenatural e maligno que deixou terríveis marcas na
cidadezinha onde moravam. Cerca de 30 anos depois, eles se reencontram,
diante de uma nova onda de terror.

\item\textsc{lovecraft}, H.\,P. \textit{O horror sobrenatural em literatura}. São Paulo: Iluminuras, 2020.

O grande mestre do terror discursa sobre esse gênero e vai em busca de
suas origens, dos artífices da construção narrativa e investiga suas
próprias obras, que se tornaram referência em todo o mundo.
\end{itemize}

\section{Bibliografia comentada}

\begin{itemize}
\item\textsc{delumeau}, Jean. \textit{História do medo no Ocidente}. São Paulo:
Companhia de Bolso, 2009.

O autor parte da ideia de que não apenas os indivíduos, mas também as
coletividades estão engajadas num diálogo permanente com a menos heroica
das paixões humanas: o medo.

\item\textsc{king}, Stephen. \textit{Sobre a escrita}. São Paulo: Suma, 2015.

Neste álbum de memórias, o mestre do terror relata eventos que marcaram
sua vida, como o batalhado início da carreira, o alcoolismo, o acidente
quase fatal em 1999, pontuando como a vontade de escrever e de viver
foram fundamentais.

\item\textsc{gotlib}, Nádia Battella. \textit{Teoria do conto}. São Paulo: Ática, 2006.

A obra convida o leitor a refletir sobre a natureza do conto com base
nas contribuições teóricas de autores como Vladimir Propp, Edgar Allan
Poe e Anton Tchekhov.

\item\textsc{groom}, Nick. \textit{The gothic: a very short introduction}. Oxford: \textsc{oup} Oxford, 2012.

O autor mostra como o gótico passou a abranger múltiplos significados,
contando sua história desde a antiga tribo que saqueou Roma até a
subcultura dos nossos dias.
\end{itemize}

\end{document}

