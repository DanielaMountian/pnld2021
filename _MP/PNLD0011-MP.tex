\documentclass[12pt]{extarticle}
\usepackage{manualdoprofessor}
\usepackage{fichatecnica}
\usepackage{lipsum,media9,graficos}
\usepackage[justification=raggedright]{caption}
\usepackage[one]{bncc}
\usepackage[relicario]{../edlab}

\begin{document}

\newcommand{\AutorLivro}{Bram Stoker}
\newcommand{\TituloLivro}{Sob o pôr do sol}
\newcommand{\Tema}{Ficção, mistério e fantasia}
\newcommand{\Genero}{Romance}
\newcommand{\imagemCapa}{./images/PNLD0011-01.png}
\newcommand{\issnppub}{---}
\newcommand{\issnepub}{---}
% \newcommand{\fichacatalografica}{PNLD0011-00.png}
\newcommand{\colaborador}{\textbf{Michelle Etienne Florence, Bruno Gradella e Vicente Castro} é uma pessoa incrível e vai fazer um bom serviço.}


\title{\TituloLivro}
\author{\AutorLivro}
\def\authornotes{\colaborador}

\date{}
\maketitle

\baselineskip=1.20\baselineskip\par


\begin{abstract}

É com prazer que lhe apresentamos este manual a fim de auxiliar seu trabalho
com os educandos do Ensino Médio. Contamos com você para mediar o contato 
destes jovens com este clássico da literatura gótica de língua inglesa.

Bram Stoker nasceu em 1847, em Dublin, Irlanda. Logo após seu
nascimento, foi acometido por uma doença desconhecida, que o deixou
acamado e afastado do convívio social até os sete anos de idade. Durante
esse período de convalescença e isolamento, seus familiares liam para
ele histórias, contos de fadas e breves narrativas sobre diversos
assuntos de seu interesse. Sua obra mais conhecida, \emph{Drácula}, 
cujo tema central é o vampirismo, já foi adaptado para teatro, cinema 
e usado como referência ou ponto de partida para outras obras literárias.

O livro \emph{Sob o pôr do sol} inicia com um conto cujo título é
justamente ``Sob o pôr do sol''. Trata"-se de um conto"-moldura, que cria
uma espécie de quadro, com características próprias, ``dentro'' do qual
deve ser feita a leitura dos contos subsequentes. Por meio dessa técnica
literária, Stoker estabelece um fio condutor que alinhava todas~as
narrativas do livro, definindo o lugar e o tema de todas as histórias,
respectivamente: o País sob o pôr do sol e as lutas travadas entre o
``bem'' e o ``mal''.
É no quarto conto da obra, ``O Construtor de Sombras'' que o gótico está melhor
representado. A~personagem que dá nome à narrativa é um ser sombrio, esquivo, 
solitário, que habita um local incerto. Ele constrói sombras das vidas de seres
humanos que passarão por um limiar entre a vida e a morte, viverão e
enfim retornarão a uma procissão de sombras, que permanece a vagar sem
rumo, eternamente. O~terror, o horror e o medo, sentimentos arraigados
no mais íntimo do ser humano, estão explicitados na atitude do
Construtor de Sombras e no lugar sobrenatural em que atua e habita.
Já no último conto do livro, ``A Criança Maravilhosa'', o ambiente noturno 
e sombrio desaparece . A história narra a descoberta de um bebê por
duas crianças, os irmãos Sibold e May, que se aventuram em um mundo
paralelo ao transpassar a fenda de um salgueiro, árvore miticamente
associada à pureza e à imaginação. O~bebê que encontram nessa outra
dimensão, apesar de possuir traços de divindade benfazeja, demonstra
necessitar de cuidados e de atenção, pois sua bondade o expõe a diversos
perigos, sugerindo a associação simbólica com a figura do menino Jesus.

De inquestionável alcance e influência na literatura e nas artes em geral, 
apostamos que essas histórias serão um ótimo instrumento de trabalho para um 
público de leitores já bastante acostumados e afeitos aos temas que elas trazem. 
Nos aproveitaremos, então, dessa familiaridade com as reverberações do gótico 
para tornar a experiência em sala de aula interessante à altura do próprio livro.
Desejamos a todos um ótimo trabalho!

\end{abstract}

\tableofcontents


\section{Atividades 1}

\SideImage{Bram Stoker, 1906 (National Portrait Gallery, Londres; Domínio Público)}{PNLD0011-03.png}

\subsection{Pré"-leitura}

%\BNCC{EM13LGG302}
%\BNCC{EM13LGG704}
%\BNCC{EM13LP10}
%\BNCC{EM13LP19}

\paragraph{Tema} O romantismo e a prosa gótica.

\paragraph{Conteúdo} Compreensão da escola literária romântica e as 
características do estilo de texto da prosa gótica.

\paragraph{Objetivo} Estimular e habilitar os estudantes a identificar as 
elementos desses movimentos, introduzindo"-os ao universo da prosa gótica
em língua portuguesa.  

\paragraph{Justificativa} ''Sob o pôr do sol'' reúne contos que se estruturam 
sobre as oposições ``bem e mal'', ``claro e escuro'', ``verdade e mentira'', 
``razão e desrazão'', antagonismos que marcavam a atmosfera da época.

A prosa gótica brasileira, por sua vez, que teve suas origens com influências de autores do
universo romântico, como Bram Stoker, buscava romper com o racionalismo iluminista 
e materialismo burguês, optando por ambientar sua literatura em áreas escuras, nebulosas,
noturnas, onde o vampirismo, o amor e a morte eram temáticas recorrentes. 

Adentrar o universo conceitual desse estilo que permeia a obra aqui proposta, permitiria
aprofundar os estudos em obras brasileiras dessa mesma escola literária e perceber a importância
desse movimento ao confrontar o maniqueísmo que permeavam as relações e valores da sociedade, 
passando a olhá"-la criticamente. 

\paragraph{Metodologia}

\begin{enumerate}
\item
Como atividade de pré"-leitura propõe"-se um levantamento coletivo dos
conhecimentos prévios dos alunos acerca do gótico, a polissemia da palavra e qual o
significado da mesma nas artes, bem como suas aplicações nessa esfera.

\item
Depois dessa primeira construção feita em conjunto, sugere"-se que o professor introduza 
os alunos no universo da prosa gótica em língua portuguesa, por meio do contato com
autores brasileiros e portugueses, como, por exemplo, Álvares de Azevedo
e Soares de Passos, por exemplo. 

Nossas sugestões de poemas representativos do estilo gótico são: ``Noite na Taverna'' 
de Álvares de Azevedo e ``Macário'', do mesmo autor.

Além de Álvares de Azevedo, também sugerimos Augusto dos Anjos e
Alphonsus de Guimarães que compreendem a literatura gótica como poucos.

\Image{Alvares de Azevedo introduziu a Literatura Gótica no Brasil (Acervo Digital da Biblioteca Nacional; Domínio Público)}{PNLD0011-12.png}

\Image{Augusto dos Anjos, representante da Literatura Gótica (Acervo Digital da Biblioteca Nacional; Domínio Público)}{PNLD0011-13.png}

\Image{Alphonsus Guimarães também representava a Literatura Gótica (Acervo Digital da Biblioteca Nacional; Domínio Público)}{PNLD0011-14.png}

\item
Como produto final para essa atividade de pré"-leitura e para o desenvolvimento de atividades 
futuras, proponha aos alunos a elaboração de um caderno de anotações, como uma espécie de diário de bordo, 
para registrar o andamento de seu percurso discente no universo gótico.

\end{enumerate}

\paragraph{Tempo estimado} Duas aulas de 50 minutos. 

\subsection{Leitura}  

%\BNCC{EM13LGG103}
%\BNCC{EM13LP02}
%\BNCC{EM13LP48}

\paragraph{Tema} Caracterização de personagem nos termos de Bram Stoker.

\paragraph{Conteúdo} Compreensão das características da personagem protagonista 
no conto da obra ``Sob o pôr do sol'' que mais representa o estilo gótico.  

\paragraph{Objetivo} Estimular e habilitar os estudantes a perceber a 
correlação entre as diferentes técnicas utilizadas pelo autor na criação de sua 
personagem, permitindo que criem livremente uma outra, em diálogo com os elementos 
do movimento da prosa gótica. 

\paragraph{Justificativa} O quarto conto de \textit{Sob o pôr do sol}, intitulado 
``O Construtor de Sombras'' melhor representa o gótico nessa obra. Esse trecho do livro 
que leva como título o nome da personagem, trata de um ser sombrio, esquivo, solitário, 
habitante de um lugar incerto. 

Essa personagem constrói sombras das vidas de seres humanos que passarão por um limiar entre 
a vida e  a morte, viverão e enfim retornarão a uma procissão de sombras, que permanece a 
vagar sem rumo, eternamente. O~terror, o horror e o medo, sentimentos arraigados
no mais íntimo do ser humano, estão explicitados na atitude do
Construtor de Sombras e no lugar sobrenatural em que atua e habita e pesadelo se
torna protagonista. 

Tal clima de pesadelo é intensificado ao máximo na
passagem em que uma mãe se esforça por salvar seu bebê, que atravessou o
limiar entre a vida e a morte. Trata"-se de um dos momentos mais belos e poéticos de todo 
o livro. O~tratamento literário desse trecho é perceptível até mesmo na pontuação, manejada
por Stoker com maestria: as frases tornam"-se mais curtas e o andamento é
acelerado, de modo a criar um clima de suspense e preparar o desfecho.

Entender a cadência do texto e como é construída uma personagem complexa e detalhada, em 
termos góticos, é de grande valia para que a/o estudante se aproxime do fazer e pensar poético 
desse movimento. 

\paragraph{Metodologia}

\begin{enumerate}
\item
Como atividade de leitura, proponha aos alunos que estabeleçam paralelos com trechos da obra
lida, principalmente a partir do quarto conto, ``O Construtor de Sombras'', e fragmentos de outras 
composições góticas famosas. 

\item
Para aprofundarem suas pesquisas os alunos podem se valer do magnum opus do autor lido, o romance 
``Drácula'', mesclando"-o com cenas de filmes que o adaptaram, bem como de filmes e séries de
vampirescas, como sugestão ``Nosferatu'' de F.W. Murnau, disponível no \href{https://www.youtube.com/watch?v=FC6jFoYm3xs&ab_channel=TimelessClassicMovies}{Youtube}

\Image{Primeira edição alemã do livro do autor, "Drácula", Max Altmann, Leipzig 1908. (Selfie756; CC-BY-SA-4.1)}{PNLD0011-05.png}

\item
Por fim, proponha aos alunos que criem uma personagem com as devidas características góticas, 
a partir da retratação e dos devidos predicados consolidados por Bram Stoker.
 
\end{enumerate}

\paragraph{Tempo estimado} Duas aulas de 50 minutos. 

\subsection{Pós"-leitura}

%\BNCC{EM13LGG102}
%\BNCC{EM13LGG303}
%\BNCC{EM13LGG402}
%\BNCC{EM13LGG703}
%\BNCC{EM13LP13}
%\BNCC{EM13LP14}
%\BNCC{EM13LP28}
%\BNCC{EM13LP29}
%\BNCC{EM13LP52}

\paragraph{Tema} Da personagem ao conto e coletânea. 

\paragraph{Conteúdo} Exercícios de escrita criativa de contos de horror a partir
da leitura do livro ``Sob o pôr do sol'' e da perspectiva dos alunos hoje em dia.

\paragraph{Objetivo} Orientar a produção de uma coletânea de contos da turma, a partir 
da experiência de leitura da obra, articulando o que foi estudado sobre o estilo da prosa 
gótica e seus elementos característicos. 

\paragraph{Justificativa} Tomando como ponto de partida o próprio livro de Bram Stoker, 
\textit{Sob o pôr do sol}, estruturado a partir de um primeiro conto, que como uma 
espécie de quadro, encaminha a leitura dos contos que seguem, sugere-se como atividade de pós"-leitura 
um exercício análogo. 

Aprofundar a leitura a partir da tradução, enquanto produção escrita coletiva, é de grande valia
para avaliar o quanto se absorveu da obra lida e o quanto foi possível assimilar os conceitos e 
estilos, características e símbolos estudados por meio do livro. 

A escrita criativa soturna pode ser uma saída para tempos de poucas luzes. 

\paragraph{Metodologia}

\begin{enumerate}
\item	
Após a leitura do livro e com base no desenvolvido nas atividades
anteriores, sugira aos alunos, para essa atividade de pós"-leitura, a organização de uma coletânea de contos
de horror. 

\item
Cada aluno deverá desenvolver um conto, dando o devido
destaque à personagem que criou na atividade anterior, de leitura, desenvolvendo"-a. A
partir disso, é provável que o aluno coloque essa personagem como
antagonista de seu conto (mas, lembre"-se, isso não é obrigatório).

\item
De toda sorte, para permanecer no universo de ``Sob o pôr do sol'', deve haver
uma dualidade clara entre protagonista e antagonista. Nesse sentido,
atente os alunos ao fato de que devem proceder com o desenvolvimento
complexo de duas personagens, para então poderem dar seguimento com a
narrativa. 

\item
Como produto final e com os textos escritos, proponha a realização de uma
leitura dinâmica diante da turma para que, coletivamente, seja escolhida a sequência dos contos a serem lidos 
e, assim, constitua"-se a coletânea, 

\end{enumerate}

\paragraph{Tempo estimado} Duas aulas de 50 minutos. 


\section{Atividades 2}

A obra \emph{Sob o Pôr do Sol} possibilita trabalhos interdisciplinares
e integradores de diferentes campos do saber e áreas de conhecimento. A
seguir, propomos algumas atividades que podem ser desenvolvidas
conjuntamente com professores de outras áreas. Além das habilidades de
Linguagens e suas Tecnologias e de Língua Portuguesa, indicadas nas
etapas da seção anterior e válidas também para esta, listamos a seguir
as habilidades de outras áreas, presentes na abordagem interdisciplinar:

%\BNCC{EM13CNT201}
%\BNCC{EM13CNT303}
%\BNCC{EM13CHS101}
%\BNCC{EM13CHS102}
%\BNCC{EM13CHS106}
%\BNCC{EM13CHS401}

\subsection{Pré"-leitura} 

\paragraph{Tema} Universo gótico e universo mítico.

\paragraph{Conteúdo} Introdução e contextualização dos períodos históricos
em que monstros e seres fantásticos eram utilizados para explicar o desconhecido e 
para caracterizar a morte e o sombrio. 

\paragraph{Objetivo} Ambientar os estudantes na realidade vivida pelo autor no contexto 
da produção das obra e os seres míticos mobilizados ao longo dos períodos anteriores. 

\paragraph{Justificativa} A vertente literária gótica costuma se destacar por cenários medievais,
personagens melodramáticos, pelo uso da psicologia do terror -- como o medo,
loucura e etc. -- e do imaginário sobrenatural -- como fantasmas,
demônios, espectros, vampiros, bruxas e monstros.

Assim como outros movimentos literários, é interessante perceber que essa tendência nas artes -- não 
sendo a prosa gótica uma exceção -- surgiu como um contraponto ao tecnicismo e racionalismo iluminista 
que, por consequência, teve o afastamento das pessoas do contato mais cotidiano com o fantástico. 

A partir disso, entendemos que não há como entrar em contato com a obra de Bram Stoker e outros escritores 
do mesmo movimento sem ter conhecimento do contexto cultural e imaginário desta época.

\paragraph{Metodologia}

\begin{enumerate}
\item
Como atividade pré"-leitura, sugere"-se uma exploração do universo de
monstros e seres fantásticos, a luz da perspectiva da ciência. 

\item
Convide os professores de ciências humanas e naturais para colaborarem com os
alunos na investigação das possíveis origens de vampiros, ciclopes,
unicórnios, dragões, entre outros. 

\item
Sugere"-se a observação de como alguns
temas são comuns a diversos povos e, juntamente com os alunos, procurem
investigar as possíveis razões por trás disso.

\end{enumerate}

\paragraph{Tempo estimado} Duas aulas de 50 minutos.

\Image{Em Whitby, Yorkshire,  no Reino Unido, a Abadia de Whitby foi uma inspiração para Bram Stoker em "Drácula". (Michael D Beckwith; CC0)}{PNLD0011-06.png}

\subsection{Leitura} 

\paragraph{Tema} O quanto há de \emph{Drácula} em \emph{Sob o Pôr do Sol}?

\paragraph{Conteúdo} Compreensão das inspirações do autor Bram Stoker a partir
da comparação com outras obras suas, como \emph{Drácula}.

\paragraph{Objetivo} Habilitar os estudantes a perceber reverberações possíveis 
entre as obras do autor e suas origens comuns -- de contexto, de história e de 
cultura --,para cada produção.

\paragraph{Justificativa} É durante a leitura que professores da área de ciências humanas
podem contextualizar a história e a cultura por trás da vertente gótica
do romantismo.

Isso pode ser abordado de uma perspectiva história e social, pois, à
medida que a sociedade evoluía nos planos econômico, político e
científico"-intelectual, A burguesia encontrou no romance uma forma de
expressão literária. Tal movimento colaborou para o surgimento de novas
vertentes, entre elas a literatura gótica.

Os autores do gênero gótico combinam o refinamento dos novos costumes
com o barbarismo e a excentricidade. Além disso, possuem a descrição
realista das ações e dos ambientes com sentimentos de desolação e
abandono.

A literatura gótica ou horror gótico inicia"-se no século \textsc{xviii}, na
Inglaterra, com a obra o castelo de Otranto (1764), de Horace Walpole.
Costuma"-se destacar, como algumas das principais características desse
tipo de literatura, os cenários medievais -- castelos, igrejas,
cemitérios, florestas, ruínas --, os personagens melodramáticos -- donzelas,
cavaleiros, vilões, os criados --, os temas e símbolos recorrentes, como segredos do 
passado, manuscritos escondidos, profecias e maldições.

\Image{O castelo é um importante monumento nacional romeno e serviu como inspiração ao autor para a criação de "Drácula". (Pixabay; Domínio Público)}{PNLD0011-07.png}

\paragraph{Metodologia}

\begin{enumerate}
\item
Comece a aula perguntando aos estudantes o que eles conhecem sobre a personagem Drácula. 
Qual a sua origem? Quais histórias e como é suas aparência física? Faça um apanhado
com as respostas e com os nomes que mais aparecerem.

\item
Depois, apresente à turma trechos do livro original de Bram Stoker e suas representações 
imagéticas. 

Reforce que Bram Stoker partiu de lendas e de referências literárias
para desenvolver sua própria concepção de vampirismo, acrescentando"-lhe
características peculiares e inéditas.

Atente os alunos aos recursos de linguagem empregados em suas descrições e proponha 
um exercício de aproximação das duas obras, \emph{Drácula} e \emph{Sob o Pôr do Sol}.

\item
Por fim, como encerramento dessa atividade de leitura, proponha que a turma identifique 
trechos, em ambas as obras, que revelem a articulação da narrativa com base nas dualidades 
de extremos como ``bem e mal'', ``luz e trevas'', ``amor e ódio''.


\Image{A obra "Carmilla" de Joseph Sheridan Le Fanu, de 1872, possivelmente inspirou a temática e o gênero literário de "Drácula" (David Henry Friston; Domínio Público)}{PNLD0011-08.png}

\Image{A obra "The Vampyre", de 1819, também teria inspirado o autor em seu livro "Drácula" (Houghton Library; Domínio Público)}{PNLD0011-09.png}

\end{enumerate}

\paragraph{Tempo estimado} Duas aulas de 50 minutos.


\subsection{Pós"-leitura}

\paragraph{Tema} Edição da coletânea de terror.

\paragraph{Conteúdo} Discussão sobre os segmentos de uma publicação

\paragraph{Objetivo} Estimular e habilitar os estudantes a produção escrita de preparação de 
texto em diversos formatos. 

\paragraph{Justificativa} Como forma de arrematar todo o estudo e pesquisa feitos ao longo da 
leitura da obra \emph{Sob o Pôr do Sol}, voltar ao exercício de produção de uma coletânea permitiria 
aos alunos uma recapitulação geral do que foi estudado e produzido ao longo das aulas. 

Em diálogo com as outras disciplinas, das ciências da natureza, história e geografia, a coletânea permitiria
articular esses diversos saberes em cada um dos segmentos que são necessários para a formatação de uma 
publicação. 

\paragraph{Metodologia}

\begin{enumerate}
\item
Como atividade de pós'"-leitura, proponha aos alunos retomarem suas anotações de seus diários 
de bordo e compilarem, conjuntamente, frentes para organizar uma simulação do que viria a ser 
a coletânea de terror. 

\item
Divida a turma em 5 grupos, destinando a cada um deles responsabilidades distintas para a edição da 
coletânea. O primeiro grupo seria responsável pelo texto de apoio ao livro de contos, o segundo pelo 
prefácio, o terceiro pela introdução, o quarto pelo glossário e o quinto, por fim, pelo epílogo. 

\item
Atente"-os ao formato e ao conteúdo que se espera sejam
produzidos em cada um desses segmentos. 

\item
Feito isso, os contos podem ser compilados em um material a compor o acervo da biblioteca, podendo, se 
possível, ser veiculado também em formato digital. 

\end{enumerate}

\paragraph{Tempo estimado} Duas aulas de 50 minutos.

\section{Aprofundamento}

Ao chegar ao Ensino Médio, é necessário que os estudantes se aprofundem
na compreensão das múltiplas linguagens e, sobretudo, da linguagem
literária. Em relação à literatura, a \textsc{bncc} traz as seguintes
considerações:

\begin{quote}
{[}\ldots{}{]} a leitura do texto literário, que ocupa o centro do trabalho
no Ensino Fundamental, deve permanecer nuclear também no Ensino Médio.
Por força de certa simplificação didática, as biografias de autores, as
características de épocas, os resumos e outros gêneros artísticos
substitutivos, como o cinema e as \textsc{hq}s, têm relegado o texto literário a
um plano secundário do ensino. Assim, é importante não só (re)colocá"-lo
como ponto de partida para o trabalho com a literatura, como
intensificar seu convívio com os estudantes. Como linguagem
artisticamente organizada, a literatura enriquece nossa percepção e
nossa visão de mundo. Mediante arranjos especiais das palavras, ela cria
um universo que nos permite aumentar nossa capacidade de ver e sentir.
Nesse sentido, a literatura possibilita uma ampliação da nossa visão do
mundo, ajuda"-nos não só a ver mais, mas a colocar em questão muito do
que estamos vendo/vivenciando. (Brasil, 2018, p. 491)
\end{quote}

Nesta seção, desenvolvemos um trabalho de aprofundamento que, em diálogo
com a formação continuada de professores, oferece subsídios para a
abordagem do texto literário.

\subsection{A obra}

\emph{Sob o Pôr do Sol} foi o primeiro trabalho de certa relevância
concluído por Bram Stoker. E é curioso pensar que o autor de Drácula
tenha produzido um livro cujas histórias sejam tão próximas aos contos
infantis.

A coletânea de histórias, suavemente conectadas, apresentam ao leitor um
universo de fantasia e fascinação. Nelas, somos apresentados a uma
narrativa cativante, em que vemos o surgimento de conflitos e suas
posteriores soluções, com um desfecho moralizador, no sentido que sempre
o justo é recompensado e o mau, punido.

\subsection{Um ferreiro de contos}

A obra, contudo, não deve ser tomada como mais um livro infantil da Era
Vitoriana, ainda que muitas das suas histórias nos aproximem dos contos
de fadas, das obras compiladas por Grimm, ou produzidas por Andersen,
há, em \emph{Sob o Pôr do Sol}, a presença clara da jornada do herói. O
caráter da luta, do enfrentamento, e da vitória pela superação é
marcante em Stoker, já denotando um tom de maior complexidade na
construção das personagens, especialmente nos seus sentimentos mais
íntimos, que caracterizariam as obras futuras do autor.

É possível, fazendo uma analogia, dizer que Stoker tomou a estrutura dos
contos de fada, derreteu e a colocou na forja e, tal como o ferreiro,
produziu um novo item, mais robusto, firme e potente.

Os conflitos, as expressões intensas e toda a concepção dramática que
atingiriam o ápice em Drácula, já despontavam, como primeiros brotos de
uma densa floresta em \emph{Sob o Pôr do Sol}.

\subsection{As inspirações em sua vida pessoal para o \textit{debut} de um grande autor}

Ao se ler a presente obra, o primeiro sucesso de Bram Stoker, temos o
vislumbre do desenvolvimento do autor, com todos os seus valores
literários.

Nessa obra, em verdade, enxergamos Stoker saindo do seu indivíduo
pessoal e construindo seu indivíduo autor.

Stoker fora acometido por uma doença rara quando tinha cerca de sete
anos de idade, sendo obrigado a se isolar dos convívios sociais durante
esse período, sendo obrigado a restar em sua habitação durante esse
tempo, tendo como passatempo apenas a leitura de alguns contos de fadas
e assuntos de seu interesse. Assim, vemos traços da sua biografia nas
histórias contadas, especialmente no ambiente obscuro desenvolvido.
Afinal, tratava"-se de uma criança, ouvindo histórias de aventuras
fantásticas, sem poder sair e ver diretamente a luz do sol. As histórias
a eles contadas eram feixes de luz que adentravam na noite que o jovem
Stoker singrava, entretanto, ao adentrar em seu universo, marcavam
também a existência das sombras que perseguiam o autor.

\Image{Local de nascimento de Bram Stoker, 15 Marino Crescent, Dublin. (Smirkybec; CC-BY-SA-4.0)}{PNLD0011-04.png}

Nas letras, Stoker age como um fotógrafo, propositalmente alterando o
tempo de captura de luz do obturador, em uma brincadeira focal em um
jogo de luz e sombras.

O vívido de sua narrativa e o jogo de iluminação e obscurantismo que o
autor produz tornam difícil a classificação de sua obra como um conto de
fadas trivial.

Por vezes, o tom moralizador da obra é por demais contundente. E as
punições do mal e do pecaminoso são violentamente severas. Talvez, daí o
forte interesse que Stoker despendeu em Vlad ``Tepes'' Draculesti, sua
inspiração para Drácula.

\Image{Retrato de Vlad Tepes (governou 1455-1462, 1483-1496) (Utopia, Univeridade do Texas; Domínio Público)}{PNLD0011-11.png}

Ademais, no seu tempo no claustro, tomou contato com a literatura
gótica, que havia surgido recentemente e estava sendo muito consumida. O
ócio de uma longa doença forneceu a oportunidade a muitos pensamentos
que foram frutíferos.

Tanto que esse dado biográfico está refletido em diversos temas tratados
ao longo de toda a sua obra, tais como a impossibilidade da fala e dos
movimentos, como se vivesse incessantemente em um sonho lúcido.

Sendo um homem religioso, os temas cristãos também surgem na obra de
Stoker. O triunfo total do bem sobre o mal, a forte figura da família e
da lealdade entre seus membros são exemplos da característica cristão
presente no microcosmo criado em \emph{Sob o Pôr do Sol}.

\subsection{\textit{Ad astra per aspera}}

\emph{Ad astra per aspera}, é a expressão latina que significa, pelas
dificuldades, as estrelas. Ou seja, a glória após a dificuldade. E é
isso que encontramos na obra de Stoker.

Na musicologia alemã, o brocado latino aparece sob a formulação
\emph{Kampf und Sieg}, a vitória após a luta, ou o triunfo sobre a
adversidade. Assim, encontramos consonância na obra de Stoker com a de
compositores alemães do século \textsc{xix}. A 5ª sinfonia de Ludwig van
Beethoven, com seu duro motivo sombrio e final triunfante, ou mesmo a
ópera Tristão e Isolda, de Richard Wagner, com sua tensão constante,
carregada com ambiguidade tonal, irresolução e suspensão constante até
os estampidos finais, são exemplos dessa forma artística, podendo ser
perfeitas trilhas sonoras para a leitura de \emph{Sob o Pôr do Sol}

\subsection{Por que ler \textit{Sob o pôr do sol}?}

Ao observarmos a condição que Stoker foi acometido na infância, não é
difícil imaginar que seguir de forma resignada a condução da vida tenha
sido um fardo e que, apenas a literatura, a família e a religiosidade
serviram de anteparo para a saúde mental do autor.

Assim, não é de se espantar a intensidade dos conflitos em suas
histórias, posto que ele próprio teve de passar por isso boa parte de
sua vida.

É interessante observar que as narrativas tem em si um tom pedagógico,
sendo outro elemento que as aproximam de histórias infantis. Talvez
sendo muito lúgubres e intensas à primeira vista, as histórias invocam
um dever de retidão moral no leitor.

Em \emph{Sob o Pôr do Sol} vislumbramos um reino do monstruoso e do
imensurável, em que brilhantes feixes reluzem na escuridão desse reino.
Expulsando as sombras que tentam tramar contra os protagonistas.

Em Stoker, não resta claro o cisma entre o universo infantil e o da
maturidade. Colocado em uma reflexão, é como se Stoker esticasse o
caráter infantil até o limite de sua elasticidade, criando pontos te
intersecção com o dito ``mundo adulto'', sem, no entanto, deixar sua
narrativa ser dominada por este. Dessa sorte, o termo infantil que o
classifica, não deve ser entendido como infantilizado ou pueril, ao
contrário, uma vez que a obra é rica e profunda, servindo muito bem,
também, ao público adulto.

\section{Sugestões de atividades complementares: relações dialógicas e
intertextuais}

%\BNCC{EM13LP03}
%\BNCC{EM13LP04}
%\BNCC{EM13LP49}
%\BNCC{EM13LP51}

No Ensino Médio, da mesma forma que no Ensino Fundamental, a \textsc{bncc}
organiza o trabalho com as práticas de linguagem em cinco \textbf{campos
de atuação social}. São eles: campo da vida pessoal, campo da vida
pública, campo jornalístico"-midiático, campo artístico"-literário e campo
das práticas de estudo e pesquisa.

De acordo com essa divisão, propomos na sequência um trabalho
interdiscursivo e intertextual com a obra \emph{Sob o Pôr do Sol.}

\subsection{Campo da vida pessoal}

\begin{quote}
O campo da vida pessoal pretende funcionar como espaço de articulações
e sínteses das aprendizagens de outros campos postas a serviço dos
projetos de vida dos estudantes. As práticas de linguagem privilegiadas
nesse campo relacionam"-se com a ampliação do saber sobre si, tendo em
vista as condições que cercam a vida contemporânea e as condições
juvenis no Brasil e no mundo.

Está em questão também possibilitar vivências significativas de práticas
colaborativas em situações de interação presenciais ou em ambientes
digitais e aprender, na articulação com outras áreas, campos e com os
projetos e escolhas pessoais dos jovens, procedimentos de levantamento,
tratamento e divulgação de dados e informações e o uso desses dados em
produções diversas e na proposição de ações e projetos de natureza
variada, para fomentar o protagonismo juvenil de forma
contextualizada. (\textsc{bncc}, p. 494)
\end{quote}

Para esta atividade, solicite que o aluno busque roupas e acessórios
  em seu guarda"-roupa para compor um figurino gótico. Em seguida,
  proponha a realização de um ensaio fotográfico que o ajudará na
  composição e criação de uma personagem. Ao final, conte, num pequeno
  texto, uma vivência desta personagem ao estilo de Bram Stoker.

\subsection{Campo de atuação na vida pública}

\begin{quote}
No cerne do campo de atuação na vida pública estão a ampliação da
participação em diferentes instâncias da vida pública, a defesa dos
direitos, o domínio básico de textos legais e a discussão e o debate de
ideias, propostas e projetos. {[}\ldots{}{]}

Ainda no domínio das ênfases, indica"-se um conjunto de habilidades que
se relacionam com a análise, discussão, elaboração e desenvolvimento de
propostas de ação e de projetos culturais e de intervenção social.
(\textsc{bncc}, p. 494)
\end{quote}

O Halloween é uma festividade popular que reúne componentes como o
horror, a morte, o macabro, o humor negro e o fantástico. A atividade
sugere a pesquisa das origens e influências pagãs e religiosas deste
festival e sua evolução ao longo dos tempos.

\subsection{Campo jornalístico"-midiático}

\begin{quote}
Em relação ao campo jornalístico"-midiático, espera"-se que os jovens
que chegam ao Ensino Médio sejam capazes de: compreender os fatos e
circunstâncias principais relatados; perceber a impossibilidade de
neutralidade absoluta no relato de fatos; adotar procedimentos básicos
de checagem de veracidade de informação; identificar diferentes pontos
de vista diante de questões polêmicas de relevância social; avaliar
argumentos utilizados e posicionar"-se em relação a eles de forma ética;
identificar e denunciar discursos de ódio e que envolvam desrespeito aos
Direitos Humanos; e produzir textos jornalísticos variados, tendo em
vista seus contextos de produção e características dos gêneros. Eles
também devem ter condições de analisar estratégias
linguístico"-discursivas utilizadas pelos textos publicitários e de
refletir sobre necessidades e condições de consumo.

No Ensino Médio, os jovens precisam aprofundar a análise dos interesses
que movem o campo jornalístico midiático, da relação entre informação e
opinião, com destaque para o fenômeno da pós"-verdade, consolidar o
desenvolvimento de habilidades, apropriar"-se de mais procedimentos
envolvidos na curadoria de informações, ampliar o contato com projetos
editoriais independentes e tomar consciência de que uma mídia
independente e plural é condição indispensável para a democracia.

Como já destacado, as práticas que têm lugar nas redes sociais têm
tratamento ampliado. (\textsc{bncc}, p. 494-495)
\end{quote}

As redes sociais não deixam de ser um local em que as pessoas publicam
versões fantasiosas de suas vidas e, portanto, são palco para personas
virtuais. Proponha aos alunos a reflexão sobre os perfis nas redes
sociais, sugerindo que observem se há, nesse universo, demonstração de
todo tipo de vivências inerentes à existência humana; ou se há um tipo
de \emph{romantização}, onde só existe espaço para o belo, o bom, ou o
socialmente aceito. Após essa reflexão, proponha a seguinte
ponderação: Quando um usuário ultrapassa os limites aceitos numa
publicação, ele pode ter agido talvez por dolo, talvez ele não tenha
percebido que foi antiético -- por falta de consciência ou noção sobre
o assunto --, ou por estar cego ou apaixonado defendendo determinada
causa. Nesse caso, há exposição do lado obscuro. Proponha uma
discussão sobre esse tema. Seria a rede social um tipo de narrativa
moderna, em tempo real?


\subsection{Campo artístico"-literário}

\begin{quote}
No campo artístico"-literário busca"-se a ampliação do contato e a
análise mais fundamentada de manifestações culturais e artísticas em
geral. Está em jogo a continuidade da formação do leitor literário e do
desenvolvimento da fruição. A análise contextualizada de produções
artísticas e dos textos literários, com destaque para os clássicos,
intensifica"-se no Ensino Médio. Gêneros e formas diversas de produções
vinculadas à apreciação de obras artísticas e produções culturais
(resenhas, vlogs e podcasts literários, culturais etc.) ou a formas de
apropriação do texto literário, de produções cinematográficas e teatrais
e de outras manifestações artísticas (remidiações, paródias,
estilizações, videominutos, fanfics etc.) continuam a ser considerados
associados a habilidades técnicas e estéticas mais refinadas.

A escrita literária, por sua vez, ainda que não seja o foco central do
componente de Língua Portuguesa, também se mostra rica em possibilidades
expressivas. (\textsc{bncc}, p. 495-496).
\end{quote}

O cinema acolheu o gótico com propriedade. Desde diretores
completamente adeptos ao gênero, como Tim Burton, a Francis Ford
Copolla, mais diversificado, que filmou o próprio Drácula de Bran
Stoker. Divida a sala em grupos e solicite que assistam aos seguintes
filmes: Edward Mãos de Tesoura (Tim Burton), Frankenstein de Mary
Shelley (Kenneth Branagh), O Corvo (Alex Proyas) e O Labirinto do
Fauno (Guillermo del Toro). Em seguida, proponha uma discussão a
respeito dos elementos estruturais e estéticos que os compõem.

\subsection{Campo das práticas de estudo e pesquisa}

\begin{quote}
O campo das práticas de estudo e pesquisa mantém destaque para os
gêneros e habilidades envolvidos na leitura/escuta e produção de textos
de diferentes áreas do conhecimento e para as habilidades e
procedimentos envolvidos no estudo. Ganham realce também as habilidades
relacionadas à análise, síntese, reflexão, problematização e pesquisa:
estabelecimento de recorte da questão ou problema; seleção de
informações; estabelecimento das condições de coleta de dados para a
realização de levantamentos; realização de pesquisas de diferentes
tipos; tratamento dos dados e informações; e formas de uso e
socialização dos resultados e análises.

Além de fazer uso competente da língua e das outras semioses, os
estudantes devem ter uma atitude investigativa e criativa em relação a
elas e compreender princípios e procedimentos metodológicos que orientam
a produção do conhecimento sobre a língua e as linguagens e a formulação
de regras. (\textsc{bncc}, p. 495-496)
\end{quote}

De acordo com Carl Jung, ``ninguém se ilumina imaginando figuras de
luz, mas se conscientizando da escuridão''. Em vista da afirmação do
renomado psicólogo, proponha que os alunos pesquisem sobre o arquétipo
da sombra, e discutam a importância de conhecê"-lo. Com base na
pesquisa, discutam os contos de Bram Stoker, com destaque ao
Construtor de Sombras. Ao final, solicitem que façam uma ilustração
sobre o tema.

\Image{A obra "The Vampyre", de 1819, também teria inspirado o autor em seu livro "Drácula" (Houghton Library; Domínio Público)}{PNLD0011-09.png}

\section{Referências complementares}

\begin{itemize}
\item\textsc{doyle} et al. \textit{A causa secreta e outros contos de horror}. São
Paulo: Boa Companhia, 2013.

Seis autores consagrados reúnem neste pequeno livro de histórias de
terror os elementos que aproximam os leitores do obscuro e do indizível.
Há contos de Edgar Allan Poe, do brasileiro Machado de Assis, entre
outros.

\item\textsc{james}, Henry. \textit{A fera na selva}. Rio de Janeiro: Rocco, 2011.

Nesta obra"-prima terrível e fascinante, o escritor inglês envolve em uma
narrativa psicológica um personagem complexo, que carrega em si a
sensação de ``uma fera emboscada na selva'', pronta a saltar sobre ele a
qualquer momento.

\item\textsc{king}, Stephen. \textit{It: a coisa}. São Paulo: Suma, 2014.

Em um verão qualquer, um grupo de sete crianças enfrenta pela primeira
vez um ser sobrenatural e maligno que deixou terríveis marcas na
cidadezinha onde moravam. Cerca de 30 anos depois, eles se reencontram,
diante de uma nova onda de terror.

\item\textsc{lovecraft}, H.\,P. \textit{O horror sobrenatural em literatura}. São Paulo: Iluminuras, 2020.

O grande mestre do terror discursa sobre esse gênero e vai em busca de
suas origens, dos artífices da construção narrativa e investiga suas
próprias obras, que se tornaram referência em todo o mundo.
\end{itemize}

\section{Bibliografia comentada}

\begin{itemize}
\item\textsc{delumeau}, Jean. \textit{História do medo no Ocidente}. São Paulo:
Companhia de Bolso, 2009.

O autor parte da ideia de que não apenas os indivíduos, mas também as
coletividades estão engajadas num diálogo permanente com a menos heroica
das paixões humanas: o medo.

\item\textsc{king}, Stephen. \textit{Sobre a escrita}. São Paulo: Suma, 2015.

Neste álbum de memórias, o mestre do terror relata eventos que marcaram
sua vida, como o batalhado início da carreira, o alcoolismo, o acidente
quase fatal em 1999, pontuando como a vontade de escrever e de viver
foram fundamentais.

\item\textsc{gotlib}, Nádia Battella. \textit{Teoria do conto}. São Paulo: Ática, 2006.

A obra convida o leitor a refletir sobre a natureza do conto com base
nas contribuições teóricas de autores como Vladimir Propp, Edgar Allan
Poe e Anton Tchekhov.

\item\textsc{groom}, Nick. \textit{The gothic: a very short introduction}. Oxford: \textsc{oup} Oxford, 2012.

O autor mostra como o gótico passou a abranger múltiplos significados,
contando sua história desde a antiga tribo que saqueou Roma até a
subcultura dos nossos dias.
\end{itemize}

\end{document}

