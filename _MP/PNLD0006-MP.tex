\documentclass[11pt]{extarticle}
\usepackage{manualdoprofessor}
\usepackage{fichatecnica}
\usepackage{lipsum,media9,graficos}
\usepackage[justification=raggedright]{caption}
\usepackage[one]{bncc}
\usepackage[ayllon]{../edlab}


\begin{document}

\newcommand{\AutorLivro}{Marcus Baccega (Org.)}
\newcommand{\TituloLivro}{A demanda do Santo Graal}
\newcommand{\Tema}{Ficção, mistério e fantasia}
\newcommand{\Genero}{Conto, crônica e novela}
\newcommand{\imagemCapa}{./images/PNLD0006-01.png}
\newcommand{\issnppub}{---}
\newcommand{\issnepub}{---}
% \newcommand{\fichacatalografica}{PNLD0006-00.png}
\newcommand{\colaborador}{Clarice Assalim, Bruno Gradella e Vicente Castro}




\title{\TituloLivro}
\author{\AutorLivro}
\def\authornotes{\colaborador}

\date{}
\maketitle


\begin{abstract}\addcontentsline{toc}{section}{Carta ao professor}

Este manual tem por objetivo fornecer apoio pedagógico aos educadores do 
Ensino Médio. Contamos com vocês para mediar este encontro certamente 
inspirador com o universo místico e aventureiro da cavalaria medieval.

Sem autoria conhecida, a presente edição do livro \emph{A Demanda do Santo Graal -- 
O Manuscrito de Heidelberg} foi traduzida da versão alemã do século \textsc{xiii}. 
Os leitores entrarão em contato, aqui, com uma longa tradição de narrativas orais 
produzidas e espalhadas anonimamente na Europa durante a Idade Média. Será a vez 
do percurso dos Cavaleiros da Távola Redonda em busca do Santo Graal, suposto 
cálice usado por Jesus na Última Ceia e, depois, usado para recolher seu sangue na cruz. 
Durante um jantar com o Rei Arthur e seus cavaleiros, surge ao centro da Távola Redonda 
a imagem do cálice sagrado, há muito desaparecido, e todos entendem que devem partir 
em busca dessa relíquia. Símbolo mágico e poderoso do domínio da alma sobre o corpo, 
somente três dos cento e cinquenta cavaleiros que partem em sua busca chegarão perto 
de alcançá-lo: Bohort, Parsifal e Galaat. Dentre estes, alcançará o destino glorioso 
aquele que viver cada aventura como uma oportunidade de purificar-se dos pecados e mais 
aproximar-se da jornada de Cristo. Afinal, é ele quem vai redimir a corte arturiana 
dos seus pecados mortais: incesto, luxúria e adultério.

A partir desta estimulante jornada coletiva em prol de um bem comum, temas 
propícios para os adolescentes como autodescoberta, conquista e superação 
farão parte do ambiente da sala de aula durante esta leitura. Apostamos, 
com as indicações que damos neste manual, que o livro provocará ricas 
discussões a partir de suas questões históricas, culturais e literárias. 
Queremos, com isso, instigar nesses jovens a percepção de que a literatura 
pode, sim, ser uma fonte de prazer e sabedoria em suas vidas, podendo mesmo 
lhes auxiliar na construção dos adultos que querem ser um dia. 

Esperamos que achem este material proveitoso para o trabalho em sala de aula!

\end{abstract}

\tableofcontents


\section{Propostas de Atividades I}

% \BNCC{EM13LGG302}
% \BNCC{EM13LGG704}
% \BNCC{EM13LP10}
% \BNCCC{EM13LP19; EM13LP45.}


\subsection{Pré"-leitura}



Antes de mergulhar no universo da obra \emph{A Demanda do Santo
Graal}, os alunos podem explorar, em outras fontes de informação, os
dois grandes temas presentes na prosa medieval: o \textbf{amor} e a
\textbf{luta}. Para isso, incentive a pesquisa, em livros e em
\emph{sites} de internet, de lendas originárias do universo da cavalaria
e protagonizadas por nomes como Tristão, Persival ou rei Arthur. Algumas
dessas personagens aparecerão no livro \emph{A Demanda do Santo Graal} e os
estudantes observarão que suas aventuras e genealogias podem apresentar
variações conforme as fontes consultadas. \footnote{Verifique 
  a seção \ref{sugestoes}, ``Sugestões de referências complementares'', 
  na página \pageref{sugestoes} desse manual.}

\BNCC{EM13LP30}% Fazer pesquisa

  

Reunida em pequenos grupos, a
turma pode organizar uma sequência de apresentações orais, no formato de
uma roda de conversa ou de histórias, para narrar, com as
próprias palavras, os episódios sobre as personagens pesquisadas. Em
seguida, é interessante propor que cada estudante redija,
individualmente e a partir da memória das narrações escutadas em sala,
uma versão de uma aventura que tenha chamado atenção. As versões finais
dessas produções, compostas em prosa e em formato de pequenos contos,
podem resultar em uma coletânea -- organizada coletivamente -- de mitos
e lendas da Idade Média. Desse modo, os estudantes iniciarão a leitura
com o repertório cultural ampliado e com o imaginário enriquecido pelas
narrativas do universo do livro \emph{A Demanda}.
Leia a seguir sobre a habilidade da BNCC abordada.
  \SideImage{Mais de 200 lugares na Europa reivindicam ter o Cálice Sagrado: essa versão, o Cálice de Antioche, feito em prata, é do século VI e está em exibição no Museu Metropolitan em Nova York. (Metropolitan Museum of Art; Domínio Público)}{PNLD0006-09.png}


\begin{itemize}
\item
  Aproveite para estabelecer um paralelo fundamental: a relação entre as
  novelas de cavalaria e a poesia trovadoresca medieval. Discuta com os
  alunos o conceito de \textbf{amor cortês} e procurem traçar pontos de
  contato com a visão romântica de amor e com os ideais de
  relacionamentos afetivos na contemporaneidade: em que medida amores
  impossíveis e inalcançáveis estão presentes ainda hoje em nossa
  sociedade? Selecione previamente uma breve coletânea de cantigas de
  amor e de amigo, e estimule o contato com a lírica produzida no
  Trovadorismo português. As dificuldades de vocabulário podem ser
  resolvidas por meio de versões modernizadas dos textos poéticos,
  disponíveis na internet. Uma fonte de consulta digital importante é o
  \emph{site} \href{https://cantigas.fcsh.unl.pt/listacantigas.asp}{Cantigas Medievais Galego"-Portuguesas da Universidade
  de Lisboa}. Nesse endereço, podem ser ouvidas as melodias
  originais que acompanhavam as letras de algumas cantigas. O amor
  cortês, reproduzindo as relações estabelecidas no plano
  político"-social entre vassalos e suseranos, envolve valores presentes
  também no imaginário cavaleiresco e na configuração dos heróis
  da obra \emph{A Demanda do Santo Graal}: a hierarquia e a subordinação a um ideal, bem como
  a superação de desafios em nome da honra e do heroísmo. Durante a
  Idade Média, o \textbf{código de cavalaria} determinava que o valor
  pessoal do cavaleiro -- e, por extensão, do amante -- envolvesse
  comportamentos de lealdade, cortesia e demonstração de coragem e de
  virtudes bélicas. Além de trabalhar os elementos literários existentes
  nos bastidores no livro \emph{A Demanda do Santo Graal}, é interessante que
  essa ampliação de repertório seja complementada pela composição de uma
  \emph{playlist} de música medieval, organizada coletivamente pela
  turma -- a partir da divisão em pequenos grupos -- por meio da
  consulta a canais de vídeo e plataformas de \emph{streaming} da
  internet. A \emph{playlist} pode ser utilizada como ambientação de
  fundo para as aulas de leitura do texto literário na obra \emph{A Demanda do Santo Graal}
  ou para momentos de escrita criativa e literária.
\end{itemize}

Sugerimos a leitura conjunta, com a turma sentada
confortavelmente em círculo, da obra \emph{A Demanda do Santo Graal -- O
Manuscrito de Heidelberg}. Ela pode ser realizada na sala de aula, na
biblioteca ou na sala de leitura, e diferentes alunos podem ler
conjuntos de estrofes ou mesmo interpretar diferentes personagens.
Ajuste o cronograma de leitura conforme o perfil do grupo e estimule
diálogos intertextuais e interdiscursivos, comentários e
compartilhamento de impressões, e construções coletivas de análises e
reflexões.

\subsection{Leitura}



A narrativa de aventura de cavaleiros, imbuídos de
sentimentos de nobreza, que saem pelo mundo com o objetivo de vencer
obstáculos e superar desafios -- ganhando, com isso, reconhecimento e
enobrecimento --, é um ponto comum em muitas narrativas medievais e em
obras que derivaram delas.

Filmes como \emph{O Senhor dos Anéis}, \emph{Guerra nas Estrelas}, entre
outros, trazem ao expectador contemporâneo elementos típicos da jornada
do herói, que sai de seu pequeno vilarejo e se aventura pela imensidão
do mundo, aprendendo e se aprimorando ao longo do processo, deixando de
ser um indivíduo comum para tornar"-se um grande herói.

Entretanto, o público comum pode, também, não se limitar ao espaço de
expectador dessas aventuras: a ele foi propiciado, por meio de jogos de
tabuleiros (como \emph{Dungeon and Dragons}), e outros \textsc{rpg}s -- ou
\emph{Role Playing Games} --, jogos de interpretação de papéis, que
convidam o jogador a participar ativamente de narrativas heroicas.

Com o avanço da tecnologia, essa experiência se tornou ainda mais
popular, tendo, entre os títulos mais populares -- para jogos de um
único jogador --, a série \emph{The Elder Scrolls} ou -- quanto a jogos
com múltiplos jogadores simultâneos -- o imensamente popular \emph{World
of Warcraft}.

Graças aos jogos eletrônicos, é muito raro hoje em dia que um jovem
nunca tenha se deparado com um ambiente lúdico ligado à temática medieval.
Aproveitando essa familiaridade do adolescente contemporâneo com esse
tipo de entretenimento, é possível organizar a formulação e posterior
partida de um \textsc{rpg}, tomando como pano de fundo a história e as
personagens d'\emph{A} \emph{Demanda do Santo Graal}. Leia a seguir sobre a habilidade da BNCC abordada. 
% Resenha crítica 

Os estudantes devem observar as estruturas típicas do jogo, a saber:

\begin{itemize}
\item
  Os \emph{Role Playing Games}, em função da própria forma como se
  estruturam os jogos, favorecem a inserção de elementos e conteúdos
  históricos nas narrativas construídas pelo mestre do jogo ou narrador;
  aqui, aconselha"-se que esse papel seja assumido pelo(a) professor(a).
\item
  O narrador assume o papel de árbitro e condutor da sessão de jogo,
  responsável por aplicar ou elaborar um cenário e um roteiro sobre o
  qual os jogadores agirão, de maneira mais ou menos livre, a depender
  do estilo, habilidade de improviso e intenções de cada narrador. De
  certa maneira, o trabalho do narrador assemelha"-se ao do professor:
  traçar uma trajetória pela qual pretende conduzir sua plateia,
  adequando"-a em função das respostas recebidas para melhor contemplar
  os interesses e entendimentos do grupo, bem como preparar materiais
  que o auxiliarão nesse processo. Trata"-se de uma atividade baseada
  igualmente em habilidades intelectuais e sociais, pois um narrador que
  não domine a fundo o sistema de jogo ou cenário não terá como
  responder às situações colocadas pelos jogadores; por outro lado,
  alguém com menor liderança não conseguirá conduzi"-los efetivamente, ou
  o fará apenas em função de si mesmo, não incorporando adequadamente
  aquilo que recebe dos outros jogadores. Essa é uma interessante
  oportunidade para trabalhar as competências socioemocionais e afetivas
  no contexto da educação, estimulando a coesão de grupo e a
  distribuição de papéis adequados ao perfil de cada participante.
\item
  Ainda que cenários realistas possam ser mais atrativos pela
  perspectiva do ensino de História, em função de melhor representarem o
  contexto e as relações sendo estudadas, os cenários fantásticos não
  devem ser desprezados, dado que, com eles, é possível explorar a
  apropriação que se faz do passado no presente e o pensamento
  analógico, caso as origens dos elementos fantásticos estejam ligadas a
  um imaginário ligado historicamente ao período geral que o cenário
  pretende representar. Nesse momento, os estudantes provavelmente já
  terão iniciado, em paralelo, as atividades propostas na próxima seção,
  dedicadas à abordagem interdisciplinar da obra. O trabalho com o
  contexto histórico"-cultural poderá ser transposto com proveito nesta
  etapa de desenvolvimento do jogo.
\item
  É ainda possível trabalhar puramente com elementos isolados, quando
  estes forem bem construídos, de ponto de vista histórico, dentro de um
  cenário fantástico baseado em um período, como a organização política
  de um reino em um cenário de fantasia medieval, considerando que os
  elementos fantásticos não devem causar choques anacrônicos com o
  conteúdo histórico. Optar por cada uma dessas possibilidades é uma
  escolha dos professores envolvidos, tanto em relação a quais elementos
  explorar quanto à tônica que pretende dar ao jogo.
\end{itemize}

Cabe frisar que o(a) professor(a) será o maestro dessa aventura e,
conforme as metodologias ativas, assumirá um papel de mediação e
mentoria, possibilitando o desenvolvimento do protagonismo dos
estudantes. Embora seja aconselhável ater"-se à narrativa da obra, o
princípio da incerteza não pode ser ignorado e seus efeitos tampouco
soterrados. A essência do \textsc{rpg} é a interpretação de personagens e a
liberdade de resposta às situações propostas pelo narrador, o que
constitui uma nova gama de desafios.

Por isso, o(a) professor(a) deve estar preparado para adequar a narração
de acordo com as respostas dadas pelos alunos. Porém, em se tratando de
cenários históricos, pode reservar um tempo para explicar quais seriam
as atitudes esperadas por parte de personagens em determinadas posições
sociais, por exemplo, bem como para apontar a um jogador alguma ação
incoerente ou anacrônica. Desse modo, constrói"-se uma consciência
literária -- do ponto de vista da organização da narrativa -- e
histórica o contexto em que o jogo se insere.

A interpretação de papéis de personagens a partir de formas de
pensamento diferenciadas é um exercício interessante para estimular
reflexão e construir conhecimentos que serão mobilizados para determinar
as motivações e cursos de ação dos personagens, constituindo uma forma
de aprendizagem ativa.

Ao final, o roteiro do jogo pode dar origem a uma produção escrita e
coletiva: um conto de aventuras baseado no universo do jogo criado. 

Após a revisão feita pelos professores envolvidos e as reformulações dos
estudantes, é interessante publicar a versão final no \emph{site} da
escola ou em um blog da sala destinado à experiência de leitura, como o
objetivo de incentivar outras pessoas ao mergulho no universo do jogo.
Leia a seguir sobre a habilidade da BNCC abordada.
\BNCC{EM13LGG103} % Produzir discursos 

Ao longo da leitura, ressalte trechos em que a construção
gramatical evidencia e reforça valores e ideias do período em que a obra
foi composta. Nesse momento, língua e discurso se aproximam e os estudos
literários e linguísticos convergem para a produção de sentido.

Por ter sido escrita no século \textsc{xiii} -- época em
que muitas línguas europeias estavam em processo de formação -- e por
ser resultado da prosificação de narrativas orais, o texto d'\emph{A
Demanda do Santo Graal}, em português, apresenta algumas características
interessantes do ponto de vista linguístico, como flutuação gráfica
(\emph{Meliant} e \emph{Melians}, ou \emph{Natiges} e \emph{Nascius, ou
Camelot e Camlot}), lacunas (como em \emph{{[}adversário{]}}, por
exemplo, em que o editor inseriu a palavra entre colchetes para dar
sentido ao trecho) e inúmeras marcas de oralidade, como a repetição de
conjunção coordenativa, encadeando o
discurso:

\begin{quote}
\emph{\textbf{E} Galaat não respondeu nenhuma palavra \textbf{e}
dirigiu"-se para frente do cavaleiro, \textbf{e} ele veio tão ligeiro
quanto pôde. \textbf{E} por ter vindo muito impetuosamente, perdeu o seu
{[}adversário{]}. Mas Galaat o encontrou tão duro que lhe enfiou a lança
pelos ombros \textbf{e} o trouxe, a ele \textbf{e} ao corcel juntos, por
terra, \textbf{e} a lança quebrou"-se. \textbf{E} Galaat saltou por cima
dele. \textbf{E} com isto, por cuja causa ele deveria volver, então viu
de onde vinha um outro cavaleiro bem armado, que clamou: ``Senhor
cavaleiro, deveis deixar"-me aqui o cavalo''. \textbf{E} Galaat foi a seu
encontro \textbf{e} colou a lança sobre seu escudo \textbf{e} a quebrou,
pois que antes estava partida, \textbf{e} não conseguiu derrubá"-lo da
sela; \textbf{e} Galaat lhe cortou a mão direita com a espada.
\textbf{E} porque ele sentiu que estava ferido, então volveu para a
fuga, quando tinha grande medo de morrer. \textbf{E} Galaat não o caçou
mais que o necessário para que não tivesse como fazer mais mal do que já
tinha feito. \textbf{E} volveu para Meliant \textbf{e} não contemplou
mais o cavaleiro que tinha derrubado.}
\end{quote}

Chama atenção, no fragmento acima, a repetição
da conjunção coordenativa aditiva ``e'' (repetição chamada de
\emph{polissíndeto}). Tal repetição, mais do que um mecanismo sintático
ou estilístico, é um processo responsável pelo encadeamento do discurso,
constituindo recurso de coesão textual.

É de se notar, também, que a conjunção liga
quaisquer elementos de mesmo valor sintático (palavras, orações,
períodos ou parágrafos). Desse modo, as relações estabelecidas por elas
em quaisquer níveis de construção vão formando blocos, que se juntam a
blocos maiores, formando aquilo a que chamamos
\emph{texto}.

Se prestarmos atenção, a opção pela
subordinação ou coordenação sintáticas pode ser associada à retórica
clássica e à retórica cristã, respectivamente. Na primeira, o homem
busca uma explicação para o mundo vinculada ao conhecimento e considera
que a história se desenvolve como uma sucessão de eventos no tempo,
interligados por uma relação de causa, consequência e condição. A lógica
do mundo busca exprimir"-se e sustentar"-se pela lógica do discurso. Essa
visão do tempo desdobra a história em um plano horizontal. É só observar
os textos renascentistas (período em que prevalece o antropocentrismo) e
constatamos o predomínio de uma sintaxe de subordinação. O homem, nesse
período, era o centro de todas as coisas e era responsável por seus
atos. O que ele fazia tinha uma causa e produzia efeitos dentro do tempo
e em determinadas condições. Algo bem diferente acontece no universo
d'\emph{A Demanda}.

A retórica cristã, por sua vez, justapõe os
elementos sem ordená"-los, porque só Deus poderia ordenar o mundo e
compreendê"-lo. A repetição da conjunção aditiva não é recurso exclusivo
d'\emph{A Demanda do Santo Graal}, mas bastante comum em textos
medievais -- como em traduções da \emph{Bíblia}, por exemplo -- e
constitui condição fundamental para que o texto se construa de forma
coesa, revelando que o pensamento de uma época (o teocentrismo medieval)
preside à escolha das estruturas mais adequadas para a sua expressão.
Todos somos iguais (coordenados) perante Deus, que é o centro de tudo. A
escolha pela coordenação revela a concepção de um mundo estreitamente
ligado e fixo: a ideia de um Deus, um universo e um destino sem
horizonte e sem ambiguidades.

Dentro da visão cristã, a história não dispõe de um princípio permanente
de ordenar e compreender os fatos, restando ao homem a sua observação
passiva e resignada. Nessa concepção histórica vertical, desaparece a
temporalidade: os fatos não se condicionam, nem são causa ou
consequência de outros (subordinados), mas desígnios divinos (expressos
de maneira justaposta ou coordenada).


\subsection{Pós"-leitura}


Após a leitura, é interessante propor que os estudantes
retomem a \emph{playlist} formada por músicas ligadas ao Período
Medieval. É possível ser uma interpretação moderna de uma canção do
período ou mesmo uma obra composta olhando para temas comuns no medievo.
Observa"-se que, inclusive, muitas músicas temporalmente mais próximas
dos dias de hoje -- desde as Óperas de Richard Wagner até canções de
bandas de Heavy Metal como Dark Moor e Iron Maiden, ou mesmo a cantora
pop Rosalía -- têm suas inspirações em temáticas ou mesmo em obras
medievais.

A atividade deve ser organizada em três etapas, com auxílio de
professores de Artes. Primeiramente, cada aluno deve escolher uma música
do repertório e, conjuntamente com a escolha, deve redigir um pequeno
parágrafo justificando essa seleção. Recuperando a primeira atividade,
o(a) professor(a), então, a partir das produções dos alunos e
conjuntamente com a música, deve rememorar os temas típicos das artes
medievais, provocando a turma a refletir se eles se encontram em sua
escolha musical. 
Leia a seguir sobre a habilidade da BNCC abordada.
\BNCC{EM13LP21}% Compartilhar playlist

Se, na pré"-leitura, o levantamento do repertório havia
sido mais intuitivo e guiado pelo gosto, aqui é possível justificar as
escolhas e articulá"-las aos valores culturais observados durante a
leitura da obra literária. Ao final, já com esses elementos, cada
estudante pode produzir individualmente um texto ou post para um blog
sobre a música de sua preferência, explicitando por que ela recupera
elementos culturais da Idade Média.

Por fim, a \emph{playlist} pode ser enriquecida com novas músicas
escolhidas e um link ou \textsc{qr}"-Code de acesso pode ser divulgado no
\emph{site} da escola ou no blog da turma.


\section{Propostas de Atividades II}

A obra \emph{A Demanda do Santo Graal} possibilita trabalhos
interdisciplinares e integradores de diferentes campos do saber e áreas
de conhecimento. A seguir, propomos algumas atividades que podem ser
desenvolvidas conjuntamente com professores de outras áreas.

\subsection{Pré"-leitura}

A pré"-leitura sugere o desenvolvimento de um projeto em
parceria com professores da área de Ciências Humanas, com o objetivo de
explorar o contexto histórico"-cultural da Idade Média, a partir das
relações sociais existentes entre as camadas medievais, com ênfase
daquelas que se desenvolviam no seio da nobreza e, também, considerando
a figura dos cavaleiros e dos códigos de cavalaria.

O objetivo é produzir coletivamente um quadro mural ou painel, capaz de
contextualizar e elucidar questões sobre o Período Medieval. Portanto, é
interessante propor uma pesquisa sobre os motivos da estratificação
social, que remontam os tempos do \emph{dominato} romano, somadas às
relações de fidelidade, oriundas tanto dos tratos legionários quanto de
costumes germânicos. Esse é outro aspecto que deve ser explorado, pois é
fundamental para a compreensão do sistema feudal.
Leia a seguir sobre a habilidade da BNCC abordada.
\BNCC{EM13LP54} % Mural

O olhar para o passado romano, em especial o tempo de Diocleciano,
também é uma indicação importante a ser feita, já que muitos postos de
comando militares, políticos e religiosos são oriundos dos tempos desse
imperador.

Nessa atividade, é interessante oferecer subsídios que permitam o
cruzamento dos elementos culturais presentes na obra que será lida, com
temáticas elaboradas junto à historiografia e mesmo itens da cultura
material. Um exercício interessante, por exemplo, é propor aos alunos um
olhar sobre o jogo de xadrez, o qual reflete, no tabuleiro e nas peças,
a organização da sociedade medieval. Também é aconselhável retomar
outras obras medievais, como cantigas e trechos de novelas de cavalaria,
aproveitando o repertório construído pela área de Linguagens na Proposta
de Atividades I. Por fim, recomenda"-se incentivar que cada aluno redija
um breve parágrafo analítico, apresentando semelhanças e diferenças
entre os valores do período medieval e da atualidade, com base no
repertório organizado e nas discussões feitas em sala de aula.

\subsection{Leitura}

Nesta atividade, indica"-se a produção de um relato de viagem
baseado no mapeamento de territórios da Europa medieval. Com auxílio de
professores da área de Ciências Humanas, é possível recuperar traços das
novelas de cavalaria -- apresentados aos alunos pelos professores de
Língua Portuguesa -- para aliar o domínio do imaginário à pesquisa
cartográfica e histórica. Na atividade de produção escrita, o aluno deve
se colocar como alguém que presencia situações da Idade Média, podendo
interagir diretamente com elas -- assumindo a primeira pessoa de um
viajante medieval -- ou apenas narrando o trajeto de uma personagem e
adotando a perspectiva de um narrador distanciado que opera com os
acontecimentos históricos.

Aconselha"-se, nesse trecho, a utilização do material construído em
outras atividades ligadas à \emph{Demanda do Santo Graal}. É importante
o auxílio de professores de diferentes áreas para a produção desse
conteúdo. Professores da área de Linguagens podem indicar caminhos para
o texto adquirir um estilo próximo ao dos relatos medievais, ao passo
que os professores de Ciências Humanas são fundamentais para explicar
tanto o contexto geopolítico da época e a importância de elementos
sociais quanto o folclore ou a religião, bem como a questão da
concentração de centros comerciais em alguns locais e a evasão de moedas
em outros. Para isso, é possível organizar previamente uma antologia de
documentos de época e de mapas para o delineamento da trajetória
percorrida pela personagem. O estatuto social da personagem também é
importante e a construção dos caracteres e a busca de informações pode
ser complementada por pesquisas empreendidas pelos próprios estudantes,
tanto em livros quanto em \emph{sites} de internet.

Professores de Ciências Naturais podem abordar questões como epidemias,
tais como a Peste Negra, e explicar o porquê do rápido contágio na
época. 
\BNCC{EM13CNT105} % CHS ciclo bioquímicos

Em parceria com professores de geografia, é possível observar
como questões de urbanismo e higiene contribuíram para o rápido
alastramento das doenças no passado. Aqui também é interessante fomentar
a consulta a \emph{sites} confiáveis de divulgação científica, para
leitura de artigos e reportagens.

Também é possível a participação de professores das Ciências Exatas,
explicando mecanismos de funcionamento de moinhos de água, por exemplo,
ou mesmo de um trabuco ou balista. Observe que o objetivo da atividade é
a produção de um relato ficcional; apesar disso, o texto deve conter o
maior número de elementos de factualidade para que os estudantes, por
meio da escrita lúdica e criativa, possam compreender e integrar
diferentes campos do saber e áreas do conhecimento.

\subsection{Pós"-leitura}

Fechando a jornada, esta atividade propõe a formação de um
acervo cultural e iconográfico pelos alunos. É possível a divisão de
eixos temáticos entre os grupos. O objetivo é organizar a curadoria de
um minimuseu no colégio, com cópias de artefatos medievais. Esses podem
ser desde imagens -- preferencialmente de domínio público -- obtidas na
internet, como imagens de brinquedos e outros itens medievais, como
pingentes, pinturas etc. Leia a seguir sobre a habilidade da BNCC abordada.
\BNCC{EM13LP53} % Acervo; Resenha; Comentários: minimuseu

É interessante buscar itens de diferentes territórios -- como Península
Ibérica, Ilhas Britânicas, França, Europa Centro"-Oriental, Bizâncio,
Mundo Muçulmano, Escandinávia -- e de diversos períodos, pois isso
tornará claro que o conceito de belo é mutante. Por meio da diversidade
estética, será possível perceber que as noções de beleza são construídas
e reconstruídas ao longo dos tempos.

Um museu tem o benefício social de democratizar a arte. No
entanto, em se tratando de uma atividade ligada à leitura de uma
narrativa lendária, é totalmente possível que os alunos criem um museu
não apenas do real, mas também do imaginário. O cálice sagrado, bem como
outros itens das fantasias medievais, são bem"-vindos nesse acervo. No
mesmo sentido, a \emph{playlist} montada pelos alunos anteriormente pode
ser a trilha sonora a envolver a exposição.

Além disso, o Museu Medieval permite resgatar as obras de arte de seu
lugar e função originais, descontextualizando"-as e metamorfoseando"-as em
quadros, pinturas e retratos, como verdadeiras intervenções artísticas.
É interessante, a partir disso, juntar os alunos em roda, e propor a
reflexão do motivo pelo qual o acervo foi organizado segundo
determinadas escolhas. Verifiquem se seria possível realizar outra
organização. Isso visa mostrar que, mesmo no trabalho com objetos que
remetem ao passado, o manuseio deles no presente influencia a forma como
nós os compreendemos. Como fonte para aprofundamento específico para
esta atividade, recomenda"-se a leitura, pelos professores participantes,
da obra \emph{O Museu Imaginário}, de André Malraux (Lisboa: Edições 70,
2011).

\section{Aprofundamento}

Ao chegar ao Ensino Médio, é necessário que os estudantes se aprofundem
na compreensão das múltiplas linguagens e, sobretudo, da linguagem
literária. \begin{comment}Em relação à literatura, a \textsc{bncc} traz as seguintes
considerações:

\begin{quote}
{[}\ldots{}{]} a leitura do texto literário, que ocupa o centro do trabalho
no Ensino Fundamental, deve permanecer nuclear também no Ensino Médio.
Por força de certa simplificação didática, as biografias de autores, as
características de épocas, os resumos e outros gêneros artísticos
substitutivos, como o cinema e as \textsc{hq}s, têm relegado o texto literário a
um plano secundário do ensino. Assim, é importante não só (re)colocá"-lo
como ponto de partida para o trabalho com a literatura, como
intensificar seu convívio com os estudantes. Como linguagem
artisticamente organizada, a literatura enriquece nossa percepção e
nossa visão de mundo. Mediante arranjos especiais das palavras, ela cria
um universo que nos permite aumentar nossa capacidade de ver e sentir.
Nesse sentido, a literatura possibilita uma ampliação da nossa visão do
mundo, ajuda"-nos não só a ver mais, mas a colocar em questão muito do
que estamos vendo/vivenciando. (Brasil, 2018, p. 491)
\end{quote}
\end{comment}
Nesta seção, desenvolvemos um trabalho de aprofundamento que, em diálogo
com a formação continuada de professores, oferece subsídios para a
abordagem do texto literário. A leitura em sala de aula de \emph{Demanda
do Santo Graal} pode ser enriquecida pelo aprofundamento no universo
literário em que a obra está inserida.

\subsection{A obra}

Narrada em terceira pessoa, \emph{A Demanda do Santo Graal -- O
Manuscrito de Heidelberg} trata das aventuras vividas pelos
\textbf{Cavaleiros da Távola Redonda} durante a busca -- ou demanda --
pelo \textbf{Santo Graal}, o cálice usado por Jesus na última ceia e que
posteriormente foi usado para recolher o sangue do Cristo crucificado.
Por ser um objeto sagrado, o Graal tem poderes mágicos.

Durante um jantar com o Rei Arthur e seus cavaleiros, surge ao centro da
Távola Redonda a imagem do cálice sagrado, que há muito estava
desaparecido, e todos entendem que devem partir em busca dessa relíquia,
que poderia trazer paz e abundância ao reino de Camelot.


\Image{Manuscrito de 1470 representa o Rei Arthur e os Cavaleiros da Távola Redonda, com o Santo Graal ao centro (Evrard d'Espinques; Domínio Público)}{PNLD0006-08.png}


Embora a busca seja realizada pelos 150 cavaleiros do Rei Arthur, só
serão merecedores de encontrar o vaso sagrado aqueles cavaleiros que
forem integralmente bons, puros e éticos.

\subsection{As origens}

\emph{A Demanda do Santo Graal -- O Manuscrito de Heidelberg} não tem
autoria especificada, algo bem comum na Idade Média. Vários manuscritos
circulavam anonimamente na época e foram ganhando edições feitas por
diversas pessoas ao longo do tempo.

É interessante estabelecer relações entre a obra e o contexto em que ela
foi produzida. Para isso, é interessante estabelecer parcerias com
professores da área de Ciências Humanas. Entre as estratégias de leitura
indicadas na \textsc{bncc}, estão aquelas que, ``por um lado, permitam a
compreensão dos modos de produção, circulação e recepção das obras
{[}\ldots{}{]} e o desvelamento dos interesses e dos conflitos que permeiam
suas condições de produção {[}\ldots{}{]}''.

O nome ``Manuscrito de Heidelberg'' tem origem no fato de essa versão
alemã da \emph{Demanda} ter sido dedicada aos condes de Hesse, Frederico
I e sua irmã, Mathilde de Rottemburg, cuja corte se localizava em
Heidelberg.

Posteriormente, o manuscrito alemão foi transcrito e editado pelo
filólogo Hans"-Hugo Steinhoff (1937--2004).

A partir da edição de Steinhoff, a obra \emph{A} \emph{Demanda do Santo
Graal} -- \emph{O Manuscrito de Heidelberg} -- foi organizada e
traduzida por Marcus Baccega, doutor em História Medieval pela
Universidade de São Paulo, professor de História Medieval e Teoria da
História na Universidade Federal do Maranhão (\textsc{ufma}) e autor de diversas
obras referentes à Idade Média.

\subsection{Transmissão do texto}

\emph{A Demanda do Santo Graal} é uma das
principais novelas de cavalaria da Idade Média, pertencente ao chamado
ciclo bretão, um imenso conjunto de narrativas que têm como elementos
fundamentais o Rei Arthur, os Cavaleiros da Távola Redonda e o Santo
Graal. 

\SideImage{Cavaleiros sentados ao redor da Távola Redonda (Biblioteca Nacional da França; Domínio Público)}{PNLD0006-06.png}

\emph{A Demanda do Santo Graal} é fruto das lendas e mitos
que circulavam oralmente pela Europa e que foram registrados em textos
de cunho historiográfico. A partir desses textos, surgiram versos; dos
versos, nasceram, no século \textsc{xiii}, dois ciclos diferentes de textos em
prosa: a primeira prosificação é chamada \emph{Vulgata Arthuriana} (ou
\emph{Ciclo de Lancelot"-Graal,} um conjunto de manuscritos franceses que
compreende os seguintes livros: \emph{Estoire del Saint Graal, Estoire
de Merlin, Le Livre de Lancelot du Lac, Lês aventures ou La Queste del
Saint Graal} e \emph{La Mort le roi Artu} ); a segunda prosificação é a
\emph{Post Vulgata}, que incorpora elementos de um outro livro:
\emph{Tristão em Prosa}.


\Image{Manuscrito Lancelot-Graal. (Biblioteca Nacional da França; Domínio Público)}{PNLD0006-03.png}


Desses dois ciclos (\emph{Vulgata} e \emph{Post
Vulgata}) surgiram várias versões em diferentes partes da Europa e em
diferentes línguas. A obra \emph{A Demanda do Santo Graal -- O
manuscrito de Heidelberg} provém da edição do referido manuscrito
alemão, produzido a partir da \emph{Vulgata}, na segunda metade do
século \textsc{xiii}.


\subsection{História e mito: ler o passado para compreender o presente}

A história e o mito constituem duas formas de conhecimento. A
história trata dos fatos reais e objetivos, enquanto o mito parte da
realidade, mas a transfigura com a imaginação. Os relatos do mito têm
seus componentes essenciais na esfera do sagrado, buscando equacionar
grandes questões espirituais e materiais da sociedade. Sendo uma
manifestação do imaginário, o mito está presente em todos os espaços e
épocas.


\SideImage{Excalibur, a espada da lenda do rei Arthur - ilustração feita em 1902 por Howard Pyle. (Howard Pyle; Domínio Público)}{PNLD0006-07.png}


Em diversos momentos do passado, a diferença
entre a narrativa histórica e a mítica não era considerada importante. É
justamente na expressão das relações entre o real e o imaginário que
reside a essência da criação artística d'\emph{A Demanda do Santo
Graal}.

Por essa razão, não se pode dizer propriamente
que os escritos arturianos sejam simplesmente obras literárias
ficcionais, com intuito de promover entretenimento e prazer. Na Idade
Média, esses escritos eram considerados portadores de acontecimentos
verídicos, dentro de uma convenção retórica de veracidade (inclusive, as
palavras \emph{roman} e \emph{estoire} eram intercambiáveis no período
medieval). Tais narrativas constituíam também discursos moralizantes e
disciplinares, ligados à visão de mundo
religiosa.


\Image{Manuscrito em que há uma representação do Rei Arthur, feito por Geoffrey of Monmout (1095 – 1155?), intitulado History of Kings of Britain - História dos Reis Britânicos. (Geoffrey of Monmouth; CC0)}{PNLD0006-04.png}


\Image{Outro manuscrito medieval em que há uma representação do Rei Arthur. Ele segura sua espada e um brasão com a figura da Virgem Maria e seu filho. (British Library; Domínio Público)}{PNLD0006-05.png}


\subsection{A simbologia do amor e da guerra}

Dois são os temas mais frequentes na Idade Média: amor e luta.

O amor, eixo da poesia lírica, figura, como se vê nas cantigas
de amor trovadorescas, na tentativa de união entre o homem que suplica e
a mulher que o rejeita. Em outras palavras, a poesia lírica enaltece o
amor infeliz. Em conformidade com essa ideia de amor infeliz das
cantigas de amor, figuram as cantigas de amigo, que apresentam temas
como a separação dos amantes e a
saudade.

A luta, por sua vez, é o eixo da matéria épica: das epopeias
nórdicas, das canções de gesta, das façanhas de Carlos Magno, dos
cavaleiros. Vale lembrar que a cavalaria é uma instituição que se firmou
no sistema feudal por volta do ano 1000, sendo cavaleiro, em princípio,
qualquer homem de armas que, além de ter sido ordenado, devia obedecer a
certas regras e seguir um determinado modelo de vida. No entanto, a
partir de meados do século \textsc{xii}, a cavalaria torna"-se uma casta
hereditária, sendo constituída exclusivamente por nobres. Assim, a
divisão social era composta pelo clero, por servos e nobres
(respectivamente os que oravam, os que trabalhavam e os que
guerreavam).

Com o tempo, amor e luta começam a figurar em textos que buscam
unir esses temas e o romance cortês dissemina"-se nos textos bélicos,
introduzindo um componente erótico em que se inserem elementos do mundo
sobrenatural, como filtros mágicos, florestas encantadas, feiticeiras e
fadas.

No entanto, entre os séculos \textsc{xii} e \textsc{xiii}, criou"-se a mais
poderosa ordem religiosa"-militar da Idade Média: a Ordem dos Templários,
os monges"-cavaleiros, cujos principais objetivos eram proteger os
peregrinos cristãos em viagem à Terra Santa, além de colaborar com os
exércitos reais durante as Cruzadas. A partir de então, por influência
da Igreja Católica, algumas modificações temáticas aparecem na
literatura medieval, interferindo no próprio espírito da literatura
cavaleiresca. O elemento sentimental dos textos bélicos direciona"-se
para a aspiração mística de Deus.


\Image{A primeira sede dos cavaleiros templários, a Mesquita de Al-Aqsa, em Jerusalém, o monte do Templo. Os cruzados chamaram-lhe de o Templo de Salomão, como ele foi construído em cima das ruínas do templo original, e foi a partir desse local que os cavaleiros tomaram seu nome de templários. (Berthold Werner; CC-BY-SA 3.0)}{PNLD0006-10.png}


Todas as leituras d'\emph{A Demanda do Santo
Graal} centradas na simbologia mística confirmam que as atividades
cavaleirescas se orientavam para o domínio espiritual. Se as aventuras
vividas pelos cavaleiros da \emph{Demanda} envolvem, antes de mais nada,
a vida dos cavaleiros, a vida no reino, elas também simbolizam a luta
interior pela superação carnal e alcance da paz de
espírito.


\Image{Cavaleiros sentados ao redor da Távola Redonda (Biblioteca Nacional da França; Domínio Público)}{PNLD0006-06.png}


Desse modo, \emph{A Demanda do Santo Graal} é
um romance profundamente religioso. A busca pelo vaso sagrado simboliza
a procura interior de Deus, em um universo em que convivem escolhidos e
renegados. De fato, embora todos os cavaleiros partam em busca da
relíquia, somente três deles terão a honra de vê"-lo, por serem virgens e
castos: Bohort, Parsifal e Galaat. Os demais, presos aos valores
mundanos, viverão várias aventuras, mas não farão parte do seleto grupo
dos escolhidos. Se a morte desses cavaleiros simboliza a justiça divina,
o sucesso obtido pela trindade de cavaleiros demonstra a misericórdia
desse mesmo Deus.


\SideImage{Quadro intitulado ``O Santo Graal'', de Dante Gabriel Rossetti (1860) (Dante Gabriel Rossetti; Domínio Público)}{PNLD0006-11.png}


É de se notar que o privilégio de ver o Santo Graal está
estreitamente ligado à relação dos cavaleiros com o amor erótico: quanto
menor for essa relação, mais próximo o cavaleiro estará do Santo Graal.
Assim, os cavaleiros não poderiam levar mulheres nessa jornada, nem ter
relações amorosas com elas, o que revela um olhar misógino
característico da época.

Dentre os personagens sobre os quais a história se centra,
Lancelot simboliza o pecador arrependido, Galvão representa o homem do
mundo, Boorz {[}Bohort{]} é a santidade laboriosa e exata, enquanto
Persival {[}Parsifal{]} simboliza a ingenuidade e candura infantil.
Galaaz {[}Galaat{]}, por sua vez, é o herói perfeito e a imagem de
Cristo.

\subsection{O cavaleiro escolhido}

Desde o início da novela, sabemos que Galaat é o escolhido para
encontrar o Santo Graal, por ser, dentre todos os cavaleiros, o mais
puro: virgem e de caráter
irrepreensível.

Não por acaso, a história tem início na véspera
do dia de Pentecostes (palavra de origem grega, \emph{pentekosté}, que
significa \emph{quinquagésimo}; para os cristãos, é o dia em que se
comemora a descida do Espírito Santo sobre os apóstolos de Cristo,
cinquenta dias após a Páscoa). Uma donzela chega a Camelot procurando
por Lancelot, a fim de que ele a acompanhe até um convento, para que
possa sagrar cavaleiro o jovem
Galaat.

Uma vez tornado cavaleiro, seguem"-se muitos outros símbolos e
alegorias que comprovam ser ele o ``melhor cavaleiro do mundo'', como
sua procedência da ``alta linhagem do Rei Davi e de José de Arimateia'',
a ocupação do Assento Perigoso na Távola Redonda, a retirada da espada
presa à coluna, o escudo branco com uma cruz vermelha (numa referência
direta aos Templários), dentre outros.

Apesar de ser o escolhido, Galaat deverá passar por várias
provações durante a demanda, a fim de se redimir do pecado de sua
concepção, pois era filho bastardo de Lancelot e Helena, filha do Rei
Pellis. Desse modo, assim como a humanidade carrega a culpa do pecado
original, Galaat também carrega a culpa do erro cometido pelo seu pai,
devendo, por isso, purgar"-se desse pecado. Cada aventura vivida por
Galaat é uma oportunidade de ele purificar"-se e provar ser merecedor do
seu destino glorioso.

Desde o início, as ações de Galaat o aproximam de Cristo:
quando adentra ao salão em que se encontra a Távola Redonda, dizendo ``a
paz esteja convosco'', quando cura leprosos, faz andar um paralítico,
quando expulsa um demônio.

Vai"-se assim, ao longo de toda a narrativa, construindo o mito
messiânico: a vinda do herói salvador que vai redimir a corte arturiana
dos seus pecados mortais: incesto, luxúria e
adultério.

Ao final da jornada, Bohort, Parsifal e Galaat são conduzidos
por um navio (representativo da Igreja), guiado por um forte vento (o
Espírito Santo). Dentro do navio, Galaat encontra, sobre um leito, uma
espada (simbolizando a ordem militar) e uma coroa (ligada à nobreza).
Temos, então, justiça e
salvação.

Devido aos pecados em que viviam os vassalos do Rei Arthur, o
Santo Graal não retorna ao reino de Logres. Ao fim de sua jornada,
Galaat alcança seu objetivo: como dito anteriormente, a busca pelo vaso
sagrado simboliza a procura interior de Deus. Galaat o encontra. E
então, parte da vida terrena e ascende à vida
celestial.

\subsection{Ressignificações na contemporaneidade}

\emph{A Demanda do Santo Graal} tem ressonâncias na literatura, no
cinema e até mesmo nos jogos da atualidade, principalmente no que se
refere à existência de irmandades que agem por um bem comum, tais como
as guildas presentes em jogos (como \emph{Ragnarok}) ou os
monges"-guerreiros (como os cavaleiros da \emph{Ordem dos Jedi} e da
\emph{Ordem dos Sith}, na saga \emph{Star Wars}). Esse é o universo da
\textbf{jornada do herói}, e aqui a lista de exemplos contemporâneos é
infindável: pense em todos os personagens de cinema ou da literatura que
partem de seu mundo comum para viverem aventuras em outros universos,
passando por grandes provações e resistindo a tentações: Percy Jackson,
Luke Skywalker, Harry Potter, entre outros.

O modelo narrativo da jornada heroica é um formato recorrente, baseado
em uma trajetória de autoconhecimento, conquista e superação. É aí que
está a beleza de uma obra como \emph{A Demanda do Santo Graal}: ela
transmite acontecimentos que são eternos e que se repetem na arte e na
vida: o nascimento, a vida e a morte de seres humanos em processo de
autodescoberta. Os leitores podem se identificar com os destinos dos
heróis e, a partir daí, elaborar com maior solidez os projetos de vida e
de identidade.

\subsection{Atividades para o aprofundamento da pesquisa}


% No Ensino Médio, da mesma forma que no Ensino Fundamental, a \textsc{bncc}
% organiza o trabalho com as práticas de linguagem em cinco \textbf{campos
% de atuação social}. São eles: campo da vida pessoal, campo da vida
% pública, campo jornalístico"-midiático, campo artístico"-literário e campo
% das práticas de estudo e pesquisa.

% De acordo com essa divisão, propomos na sequência um trabalho
% interdiscursivo e intertextual com a obra \emph{A Demanda do Santo
% Graal}.

\subsubsection{<<Quem seriam os heróis de nosso tempo?>> Uma pesquisa sobre a coragem na atualidade}

  O universo da cavalaria medieval reforçou a figura de protagonistas
  heroicos, presentes na literatura desde os poemas épicos da
  Antiguidade. Na Idade Média, porém, os valores cristãos e o código
  ético cavaleiresco reforçaram os traços de coragem, bravura e
  perfeição, em um universo imaginário que atravessou séculos e inspirou
  outros clássicos da literatura, como \emph{Dom Quixote de La Mancha},
  de Miguel de Cervantes -- um diálogo paródico com as novelas de
  cavalaria. A imagem do herói continua presente na cultura
  contemporânea e é explorada em diversos filmes, jogos, livros e
  séries. Além disso, o ideal heroico também é valorizado na vida real,
  por meio de pessoas comuns que acabam executando feitos
  extraordinários. Com a turma reunida em pequenos grupos, oriente a
  pesquisa de heróis e heroínas do mundo contemporâneo. Desta vez, o
  universo explorado não será o da ficção, mas o da vida real. Por meio
  da consulta a \emph{sites} de internet, ligados a periódicos da
  atualidade, os estudantes pesquisarão acontecimentos em que um
  indivíduo ou um grupo de indivíduos se destacou pela coragem,
  resiliência ou generosidade. Uma sugestão de reportagem de maior
  fôlego, a respeito do heroísmo contemporâneo, é a obra -- pertencente
  ao gênero \emph{jornalismo literário} -- escrita por Rodrigo Carvalho
  e intitulada \emph{Os meninos da caverna} (Rio de Janeiro: Globo,
  2018), a respeito de um grupo de adolescentes tailandeses que
  precisaram superar grandes obstáculos. Além de sugestão de leitura
  complementar para ampliação de repertório, esse livro -- assim como
  outros textos a serem pesquisados pelos alunos -- permite refletir
  sobre os valores e atitudes que definem o heroísmo na atualidade.

Após o levantamento de informações, os alunos podem
compartilhar oralmente as notícias e reportagens encontradas, e compor
uma coletânea que servirá como banco de dados para a produção de uma
texto sobre heróis e heroínas da atualidade. 
% Produção de coletânea Coletânea


Em primeiro lugar, cada equipe escolherá um ou dois casos reais e recentes
que exemplificam atitudes heroicas e, em seguida, construirão reflexões
sobre o conceito de heroísmo para cada um, os comportamentos socialmente
valorizados e -- o mais importante -- em que medida eles se identificam
com essas figuras. A atividade permite explorar o que há de heroico em
cada indivíduo e possibilita fortalecer a autoimagem de cada aluno.
Projetando ideais para o futuro, verifique em que medida os planos e
sonhos dos estudantes convergem para uma jornada heroica. Ao final, as
versões definitivas das crônicas -- devidamente comentadas pelos
professores envolvidos -- podem ser digitadas e publicadas no blog da
turma, no \emph{site} da escola ou nas redes sociais.

\subsubsection{<<O que é o amor para nós?>> As diferentes noções atuais com relação 
ao apresentado pelos cavaleiros medievais}

  No universo da cavalaria medieval, dois valores eram enaltecidos: o
  amor e a guerra. Esta atividade explora o primeiro desses aspectos.
  Observe em que medida a concepção de amor dos estudantes é herdeira de
  uma visão sobre os sentimentos herdada do Romantismo e, por extensão,
  do amor cortês medieval. Retome as discussões sobre as diferentes
  concepções de amor, feitas na Atividade I, e aprofunde a discussão
  sobre os amores líquidos da contemporaneidade, para trabalhar na
  perspectiva do sociólogo Zygmunt Bauman. Conversem sobre os ideais
  amorosos baseados no amor platônico e inatingível, por um lado, e nas
  relações superficiais e descartáveis da contemporaneidade, por outro.
  Ambas as atitudes estão presentes em nossa sociedade e trazem
  consequências para os hábitos de vida e para a saúde mental. A partir
  das discussões, proponha a pesquisa, em \emph{sites} de internet,
  sobre as definições possíveis para o amor na contemporaneidade. Em
  seguida, organize a produção individual de um \textbf{artigo de
  opinião} sobre os aspectos positivos e negativos da liberdade de
  formas de relacionamento na atualidade, em primeiro lugar, e da
  liquidez e objetificação das relações amorosas, em segundo lugar. 


  Os estudantes deverão construir teses coerentes com o ponto de vista
  defendido e incorporar argumentos filosóficos e sociológicos à
  reflexão. Ao final, é importante incentivar a construção explícita de
  uma proposta de intervenção concreta sobre a situação e propor
  maneiras de, em meio aos desafios dos relacionamentos amorosos atuais,
  construir vínculos sólidos e dinâmicas afetivas saudáveis. Do ponto de
  vista do repertório cultural mobilizado, é interessante estimular o
  uso de referências trazidas dos episódios amorosos e do comportamento
  dos cavaleiros de \emph{A Demanda do Santo Graal}. Uma obra que pode
  servir de base para a discussão é \emph{O amor e o ocidente}, de Denis
  de Rougemont, indicada no final deste Manual; fragmentos dessa obra
  podem ser previamente selecionados por professores das áreas de
  Linguagens e Ciências Humanas, e apresentados aos alunos para ampliar
  o conhecimento sobre a trajetória das visões sobre o amor no mundo
  ocidental. Ao término da atividade, as versões definitivas dos artigos
  -- depois de compartilhados inicialmente com a turma e os professores
  envolvidos -- podem ser digitados e publicados no blog da turma, em
  redes sociais ou no \emph{site} da escola. 

\subsubsection{A recriação de ambientes medievais na cultura pop}


  A cultura medieval exerce até hoje um forte apelo popular. Tanto
  produções voltadas ao entretenimento quanto modismos derivados da
  indústria cultural exploram o universo de cavaleiros, irmandades e
  duelos. Proponha aos alunos uma pesquisa em \emph{sites} de internet a
  respeito da recriação de espaços e eventos com ambientação medieval,
  tanto no Brasil quanto em outros países. Mesmo em lugares que não
  viveram a Idade Média em um formato próximo ao retratado pelas novelas
  de cavalaria, surgem com frequências festivais musicais e
  gastronômicos, espaços temáticos e eventos ligados ao universo
  medieval. Algumas vezes, esses fenômenos procuram ser fiéis a dados
  históricos documentados, enquanto, em outros casos, os eventos e
  espaços se deixam invadir pelo imaginário em torno da Idade Média,
  reforçado durante o período romântico, que conta com uma carga grande
  de aventura e fantasia. 

  Os estudantes, reunidos em pequenos grupos,
  redigirão \textbf{reportagens} sobre os eventos e lugares
  selecionados, e registrarão o histórico e a periodicidade de
  realização dos festivais, o surgimento de lugares com ambientação
  medieval, as características do público frequentador, as principais
  referências que servem de base a esses projetos e algumas curiosidades
  ligadas a esses cenários. 
% Criação de reportagem

  Além da versão escrita da reportagem, cada
  grupo pode apresentar informações básicas sobre os eventos e lugares
  para a turma, em formato de apresentação oral, por meio da projeção de
  \emph{slides} contendo os principais aspectos ligados ao tema
  escolhido. Por fim, as reportagens podem ser revisadas e publicadas no
  formato de uma edição virtual de um \emph{Guia de eventos e atrações
  medievais}, a ser disponibilizado no \emph{site} da escola ou no blog
  da turma.

  Outra possibilidade é a análise de jogos de videogame com a temática 
  medieval. Nesse caso, cada turma pode escolher um jogo e fazer uma descrição 
  dos aspectos ditos medievais, a fim de compará-los com alguns trechos do livro. 
  O objetivo é que eles apontem o que há de anacrônico ou de pouco fiel à 
  temática.  

\subsubsection{Escrita de poemas com características medievais}

  Retome com os alunos as discussões feitas sobre a temática do
  heroísmo, ao longo do processo de leitura da obra, e reserve um
  momento para a pesquisa -- em canais de vídeo da internet ou em
  plataformas de \emph{streaming} -- de músicas nacionais e estrangeiras
  que tratem da figura do herói. Existe um rico repertório musical
  baseado em valores heroicos de coragem, bravura e superação de
  desafios. É enriquecedora a participação de professores de línguas
  estrangeiras durante o trabalho de pesquisa e seleção das letras de
  música. O levantamento de material pode ser realizado individualmente
  ou por duplas, e é interessante a escolha de, pelo menos, duas canções
  a respeito do tema. Em seguida, a turma organizará coletivamente uma
  \emph{playlist} com as obras encontradas. Esse repertório servirá como
  trilha sonora de base para a produção -- individual ou em dupla -- de
  \textbf{poemas} ou \textbf{letras de música} com temática heroica.
  Alunos com algum conhecimento musical -- ou interesse por qualquer
  gênero e estilo -- podem enveredar pela composição de melodias, com
  auxílio do(a) professor(a) de Artes. Em outros casos, é possível
  restringir a atividade a uma oficina de escrita de poemas, com som
  ambiente formado pela \emph{playlist} elaborada conjuntamente.
  Estimule o emprego de rimas interessantes, de jogos de palavras e
  ritmos, bem como de figuras de linguagem que enriqueçam o tema em
  questão. As produções poéticas e musicais podem ser compartilhadas no
  formato de um pequeno sarau, aberto à comunidade escolar, ou no
  formato de vídeos gravados e disponibilizados em canais da internet,
  no \emph{site} da escola ou no blog da turma.
\BNCC{EM13LGG103}


\subsubsection{A canção medieval portuguesa}

  O professor pode tratar ainda das cantigas medievais portuguesas. 
  Propomos analisar o fenômeno da variação linguística no tempo, neste caso, 
  e apresentar as diferenças do português medieval para o atual. Preste atenção 
  nas características do poema também, como rimas e ritmo musical. Escute 
  a interpretação musicada que indicamos. 
\BNCC{EM13LP10}
  Visite o link \href{https://cantigas.fcsh.unl.pt/listacantigas.asp}{Cantigas
  medievais portuguesas} lista uma série de cantigas, incluindo algumas
  com áudio. Veja por exemplo a cantiga 
  \href{https://cantigas.fcsh.unl.pt/versaomusical.asp?cdcant=1307&cdvm=247}{``A do mui bom parecer''}.


\begin{verse}
A do mui bom parecer\\
mandou lo adufe tanger:\\
\quad "Louçana, d'amores moir'eu".\\
A do mui bom semelhar\\
mandou lo adufe sonar;\\
\quad "Louçana, d'amores moir'eu".\\
Mandou lo adufe tanger\\
e nom lhi davam lezer:\\
\quad "Louçana, d'amores moir'eu".\\
Mandou lo adufe sonar\\
$[$e$]$ nom lhi davam vagar:\\
\quad "Louçana, d'amores moir'eu".
\end{verse}


\subsubsection{O cordel brasileiro, herdeiro das tradições ibéricas}

  Proponha a pesquisa, em livros e \emph{sites} de internet, sobre a
  permanência do repertório cultural medieval nas tradições populares
  brasileiras. Estimule, sobretudo, a valorização da diversidade
  regional e oriente a busca de informações sobre \textbf{poemas em
  cordel}, sobretudo do Nordeste do país, que recontam criativamente as
  aventuras de antigos cavaleiros medievais, com ambientação adaptada ao
  espaço sertanejo. Leia a seguir sobre a habilidade da BNCC abordada. 
\BNCC{EM13LP10}

  A turma, dividida em pequenos grupos, ampliará o
  repertório artístico e cultural por meio da leitura desses cordéis. Em
  uma aula reservada à leitura compartilhada, cada equipe pode declamar
  oralmente alguns trechos -- ou mesmo composições inteiras -- de poemas
  em cordel com temática medieval. Além disso, enriqueça a experiência
  de leitura com trechos da obra \emph{Romance da Pedra do Reino}, de
  Ariano Suassuna, escrita a partir do imaginário e do acervo de lendas
  medievais. As leituras dramatizadas podem ser filmadas e
  disponibilizadas no \emph{site} da escola, no blog da turma ou em
  canais de vídeo da internet.

\SideImage{Poeta Zé da Luz (1904-1965), à direita da foto. Fonte: Wikipedia. CC-BY.}{PNLD0006-20.jpg}

  Sugestão: Trabalhe o poema lido pela banda pernambucana ``O cordel do fogo encantado'' 
  de autoria de Zé da Luz, disponível \href{https://youtu.be/8NBauvFV6bo}{aqui}.
  Repare os paralelos entre a cantiga medieval portuguesa da atividade 
  anterior e as aliterações irônicas utilizadas pelo poeta popular paraibano Zé da Luz.



\begin{verse}
    Se um dia nós se gostasse\\
    Se um dia nós se queresse\\
    Se nos dois se empareasse\\
    Se juntin nós dois vivesse\\
    Se juntin nós dois morasse\\
    Se juntin nós dois durmisse\\
    Se juntin nós dois morresse\\
    Se pro céu nos assubisse\\\medskip

    Mas porém acontecesse de São Pedro não abrisse\\
    A porta do céu e fosse te dizer qualquer tolice\\
    E se eu me arriminasse\\
    E tu com eu insistisse pra que eu me aresolvesse\\
    E a minha faca puxasse\\
    E o bucho do céu furasse\\
    Talvez que nos dois ficasse\\
    Talvez que nos dois caísse\\
    E o céu furado arriasse e as virgem todas fugisse\\
\end{verse}
% Territorizalidade region

Incentive os alunos a procurar sobre os tipos de cordel e também de 
repentes, e manifestações populares tais como 
o samba de coco, toré indígena, embolada e o reisado. 

\section{Sugestões de referências complementares}\label{sugestoes}

\subsection{Filmes}

\begin{itemize}

\item\textit{Camelot}. Direção: Joshua Logan (\textsc{eua}, 1967).

Esta longa produção narra o encontro do Rei Arthur com sua futura
esposa, Guinevere, numa floresta encantada próxima a Camelot, bem como
os conflitos que se seguem à aproximação do cavaleiro Lancelot.

\item\textit{Em busca do cálice sagrado}. Direção: Terry Gilliam; ‎Terry
Jones (Reino Unido, 1975).

O filme é uma paródia bem"-humorada do mito arturiano e das aventuras
presentes nas novelas medievais de cavalaria.

\item\textit{Excalibur}. Direção: John Boorman (\textsc{eua}, Reino Unido, 1981).

O filme narra como o rei Arthur conquistou a lendária espada
Excalibur, tornando"-se rei e símbolo de paz em uma Inglaterra marcada
por divisões profundas entre senhores feudais.

\item\textit{O nome da rosa}. Direção: Jean"-Jacques Annaud (Itália/ França/
Alemanha, 1986).

Um monge e um noviço chegam a um mosteiro, no norte da Itália, para
investigar uma série de assassinatos, enquanto enfrentam a 
desconfiança dos moradores do lugar.

\item\textit{O sétimo selo}. Direção: Ingmar Bergman (Suécia, 1959).

Um cavaleiro retorna das Cruzadas e encontra seu país assolado pela
peste. Enquanto reflete sobre o significado da vida, a Morte surge e
eles dão início a um jogo de xadrez.

\item\textit{Rei Arthur}. Direção: Antoine Fuqua (\textsc{eua}, 2004).

Esta produção contemporânea reconta a lenda do Rei Arthur, que deseja
deixar a Bretanha e voltar a Roma para viver em paz. No entanto, ele
encara uma missão junto aos Cavaleiros da Távola Redonda e decide
permanecer no país para liderá"-los.
\end{itemize}

\subsection{\emph{Sites}}

\begin{itemize}
\item Associação Brasileira de Estudos Medievais
(\href{www.abrem.org.br}{abrem.org.br})

O \emph{site} traz artigos sobre a história e a cultura da Idade 
Média, publicados na revista eletrônica \emph{Signum}, além de 
informações sobre eventos relacionados ao período medieval.
\end{itemize}

\section{Bibliografia comentada}

\begin{itemize}
\item\textsc{barthélemy}, Dominique. \textit{A cavalaria: da Germânia Antiga à França do século \textsc{xii}}. Campinas: Editora da Unicamp, 2010.

A obra discute as origens e o significado da cavalaria, instituição 
medieval baseada em um código de ética, cuja realidade documental 
divide espaço com o domínio do imaginário.

\item\textsc{bradley}, Marion Zimmer. \textit{As brumas de Avalon}. São Paulo: Planeta Minotauro, 2018.

Considerado por muitos a versão definitiva do mito do Rei Arthur, este
clássico contemporâneo narra a história do ponto de vista das
personagens femininas, como Morgana e Igraine.

\item\textsc{goff}, Jacques Le. \textit{Heróis e maravilhas da Idade Média}. São
Paulo: Vozes, 2010.

O autor explora o imaginário medieval a partir de dois eixos: os
heróis -- como Rei Arthur e Robin Hood --, e de espaços como a cidade,
o castelo, a catedral e o claustro.

\item\textsc{jung}, Emma; \textsc{von"-franz}, Marie Louise. \textit{A lenda do Graal}. São
Paulo: Cultrix, 1991.

Sob a perspectiva da Psicologia Analítica de Jung, as autoras
apresentam a lenda do Santo Graal como um mito vivo e profundamente
relevante para a vida moderna.

\item\textsc{martins}, Mário. (1975). Alegorias, símbolos e exemplos morais da
literatura medieval portuguesa. Lisboa: Brotéria, 1975.

A obra mergulha no imaginário medieval para traçar um painel dos
principais valores que organizavam a mentalidade do período.

\item\textsc{massardier}, Gilles. \textit{Contos e lendas da Europa medieval}. São
Paulo: Companhia das Letras, 2002

As histórias compiladas revalorizam um período que, por muito tempo, 
foi chamado de Idade das Trevas, revivendo cavaleiros virtuosos,
princesas e magos.

\item\textsc{megale}, Heitor. \textit{O jogo dos anteparos. A Demanda do Santo Graal: a estrutura ideológica e a construção da narrativa}. São Paulo: T. A.
Queiroz, 1992.

A obra trata da transmissão d'\emph{A Demanda do Santo Graal} e
analisa os valores por trás da narrativa, com ênfase na versão
portuguesa.

\item \_\_\_\_\_. \textit{A Demanda do Santo Graal}. São Paulo: Companhia de
Bolso, 2008.

O livro traz a versão portuguesa d'\emph{A Demanda} e, ao acompanhar
os pontos de vista dos cavaleiros que partem em busca do cálice
sagrado, traça um rico panorama da vida e dos costumes medievais.

\item \textsc{mongelli}, Lênia Márcia. \textit{Por quem peregrinam os cavaleiros de
Arthur}. São Paulo: Íbis, 1995.

A autora estuda a lendária novela de cavalaria medieval e acompanha a
busca por redenção dos cavaleiros, ao longo das provações que têm que
suportar.

\item \_\_\_\_\_.; \textsc{macedo}, José Rivair. \textit{A Idade Média no cinema}. São
Paulo: Ateliê Editorial, 2009.

Seis ensaios analisam filmes que reconstroem a Idade Média e buscam
entender por que esse período rendeu tantos roteiros cinematográficos
e apelo popular.

\item \textsc{pastoreau}, Michel. \textit{No tempo dos cavaleiros da Távola Redonda}. São Paulo: Companhia das Letras / Círculo do Livro, 1989.

Em linguagem acessível, o livro traça um rico painel do cotidiano e
dos costumes medievais, integrados ao imaginário do período.

\item \textsc{pyle}, Howard. \textit{Rei Arthur e os cavaleiros da Távola Redonda}. Rio de Janeiro: Zahar, 2015.

Este livro, recheado de imagens, dá vida aos principais eventos da
vida do Rei Arthur e aos personagens da lenda, como a Rainha Morgana,
a Dama do Lago e o mago Merlin.

\item \textsc{rougemont}, Denis de. \textit{História do amor no ocidente}. Rio de
Janeiro: Ediouro, 2003.

A obra tornou"-se um clássico dos estudos sobre as origens do amor na
cultura ocidental, com ênfase na trajetória que articula o mito
medieval de Tristão e Isolda às origens da concepção de amor
romântico.

\item \textsc{spina}, Segismundo. \textit{A cultura literária medieval}. São Paulo:
Ateliê, 1997.

O autor apresenta um painel dos principais autores e obras literárias
que consolidaram o vasto acervo de ideias e crenças da Idade Média.
\end{itemize}

\end{document}

