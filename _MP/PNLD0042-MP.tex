\documentclass{extarticle}
\usepackage{manualdoprofessor}
\usepackage{fichatecnica}
\usepackage{lipsum,media9,graficos}
\usepackage[justification=raggedright]{caption}
\usepackage{bncc}
\usepackage[madalena]{../edlab}

\begin{document}


\newcommand{\AutorLivro}{Liev Tolstói}
\newcommand{\TituloLivro}{A morte de Ivan Ílitch}
\newcommand{\Tema}{Ficção, mistério e fantasia}
\newcommand{\Genero}{Romance}
\newcommand{\imagemCapa}{./images/PNLD0042-01.png}
\newcommand{\issnppub}{---}
\newcommand{\issnepub}{---}
% \newcommand{\fichacatalografica}{PNLD0042-00.png}
\newcommand{\colaborador}{\textbf{Bruno Gradella e Vicente Castro} é uma pessoa incrível e vai fazer um bom serviço.}


\title{\TituloLivro}
\author{\AutorLivro}
\def\authornotes{\colaborador}

\date{}
\maketitle

\begin{abstract}
Este Manual tem como objetivo fornecer subsídios para o trabalho com a
obra literária \emph{A Morte de Ivan Ilitch}, de Lev Tóstoi.

Neste material, são propostas atividades de leitura do texto literário,
em perspectiva interdisciplinar, a partir de propostas de abordagem
envolvendo a integração de diferentes áreas do conhecimento. Todas as
etapas de pré-leitura, leitura e pós-leitura atribuem aos professores o
papel de mediadores entre os estudantes e a obra literária. Aos alunos,
por sua vez, é conferido um lugar de protagonismo e autonomia na
construção do conhecimento, a partir do emprego de metodologias ativas e
estratégias de aprendizagem criativa.

Seguindo as competências e habilidades indicadas na nova Base Nacional
Comum Curricular (BNCC), o trabalho com o texto literário é desenvolvido
no âmbito dos diferentes campos de atuação social. Para isso, levam-se
em consideração os campos da vida pessoal, de atuação na vida pública,
das práticas de estudo e pesquisa, jornalístico-midiático e, sobretudo,
artístico-literário.

Cada uma das seções do Manual sugere atividades e apresenta informações
complementares para enriquecer a experiência de leitura do texto
literário. A partir de uma proposta dialógica de ensino de literatura,
procura desenvolver habilidades de leitura e produção de textos, visando
à formação do sujeito leitor-autor competente, capaz de interagir com o
mundo e de atribuir sentido às próprias vivências.

Reforçando o caráter formativo e informativo da literatura, o material
procura articular a formação de leitores à elaboração de projetos de
vida, a partir da ampliação do repertório artístico-cultural dos
estudantes. A leitura crítica da obra literária é concebida em sentido
amplo e envolve o estabelecimento de relações entre textos pertencentes
a esferas variadas de comunicação e a gêneros discursivos diversos.

Ao mesmo tempo, a releitura de clássicos literários traz, para a
contemporaneidade, novas possibilidades de significação para obras que
permanecem atuais. Para contribuir com essa aproximação entre jovens
leitores e obras consagradas, são propostas atividades que empregam as
novas tecnologias de informação e comunicação, fundamentais para a
formação de leitores inseridos na cultura digital. Valorizam-se,
portanto, estratégias de leitura e produção textuais no âmbito da
hipertextualidade e do multiletramento.

Para a formação continuada de professores, apresentamos sequências que
estimulam a criatividade e a inovação, com possibilidades de adaptação
às diferentes realidades de ensino. Orientações de caminhos possíveis
são apresentadas como sugestões para as atividades de leitura em sala de
aula, sempre permeáveis aos diferentes perfis de grupos e às
especificidades de gostos e repertórios culturais.

Conforme a BNCC assinala, ``no Ensino Médio, os jovens intensificam o
conhecimento sobre seus sentimentos, interesses, capacidades
intelectuais e expressivas; ampliam e aprofundam vínculos sociais e
afetivos; e refletem sobre a vida e o trabalho que gostariam de ter.
Encontram-se diante de questionamentos sobre si próprios e seus projetos
de vida, vivendo juventudes marcadas por contextos socioculturais
diversos'' (Brasil, 2018, p.481).

Nada mais adequado, portanto, que oferecer textos literários capazes de
estimular reflexões sobre a vida pessoal e rotas possíveis para os
projetos de vida. Para você, professor(a), abrem-se novas trilhas para o
contato sempre renovado com obras que são, a um só tempo, atuais e
atemporais.

Boa jornada!
\end{abstract}

\tableofcontents




\section{Atividades 1}

%\BNCC{EM13LP26}

\subsection{Pré-leitura}

%\BNCC{EM13LGG302}
%\BNCC{EM13LGG704}
%\BNCC{EM13LP10}
%\BNCC{EM13LP19}

Em \emph{A Morte de Ivan Ilitch}, como o próprio título já
indica, a temática principal abordada na obra é a morte. Em específico,
como lidar com ela, como ela representa a finitude e como nos esquecemos
de sua inexorável chegada enquanto vivemos, muitas vezes nos prendendo e
despendendo energia em coisas pouco importantes. A temática da morte é
algo comum a todas as culturas, cada qual interpretando essa questão à
sua maneira.

Por isso, a atividade sugere buscar elementos de outras culturas acerca
desse tema. Como cada uma delas lida com esse momento. Também é
interessante abordar perspectivas de diversas religiões sobre o conceito
de morte. A atividade sugere também discriminar como a morte tende a
aparecer em uma obra realista. Comparar com outros estilos, por exemplo,
a morte idealizada dentro da estética romântica. É interessante anotar
as perspectivas dos alunos antes das atividades para a realização de um
comparativo posteriormente.

\subsection{Leitura}

%\BNCC{EM13LGG103}
%\BNCC{EM13LP02}
%\BNCC{EM13LP48}

Durante a leitura, é interessante produzir uma linha do
tempo, onde restarão marcados os eventos da trajetória de Ivan Ilitch.
Procurem, também, registrar uma lista de \textbf{personagens}, no
formato de um \textbf{mapa conceitual}, organizando nomes e suas
relações com a personagem principal. Cada personagem pode ser
identificada por um marcador adesivo contendo o nome e o sobrenome; é
interessante estimular a visualização da história e pedir aos estudantes
que desenhem ou pesquisem ilustrações para as personagens. Conforme a
leitura for avançando, anotem e fixem, tanto os eventos na linha do
tempo, quanto, ao lado de cada uma das personagens no quadro, adjetivos
que ajudem a caracterizar comportamentos, trechos de falas ou de
passagens emblemáticas. Essa atividade poderá ajudar em etapas
posteriores da análise e da recuperação e síntese de informações
pós-leitura. Poderá também ser enriquecido por outros elementos verbais
e não verbais, decorrentes das atividades interdisciplinares com a obra.

\subsection{Pós-leitura}

%\BNCC{EM13LGG102}
%\BNCC{EM13LGG303}
%\BNCC{EM13LGG402}
%\BNCC{EM13LGG703}
%\BNCC{EM13LP13}
%\BNCC{EM13LP14}
%\BNCC{EM13LP28}
%\BNCC{EM13LP29}
%\BNCC{EM13LP52}

Para a pós leitura, cada estudante deverá redigir um texto
argumentativo, ponderando os valores mais importantes da vida e o que é,
na concepção de cada um, o bem viver. Aconselha-se, para isso, oferecer
aos alunos uma bibliografia complementar, que pode incluir obras da
poesia árcade, bem como textos de filosofia, como de Sêneca, Epicuro,
Cícero, etc. Também, podem ser recuperadas informações trazidas na
primeira atividade, uma vez que, sem sombra de dúvidas, aspectos
culturais e religiosos que tratam da morte, também indicam um parecer
sobre a vida.

\section{Atividades 2}

%\BNCC{EM13CNT201}
%\BNCC{EM13CNT303}
%\BNCC{EM13CHS101}
%\BNCC{EM13CHS102}
%\BNCC{EM13CHS106}
%\BNCC{EM13CHS401}

A obra \emph{A Morte de Ivan Ilitch} possibilita trabalhos
interdisciplinares e integradores de diferentes campos do saber e áreas
de conhecimento. A seguir, propomos algumas atividades que podem ser
desenvolvidas conjuntamente com professores de outras áreas. Além das
habilidades de Linguagens e suas Tecnologias e de Língua Portuguesa,
indicadas nas etapas da seção anterior e válidas também para esta,
listamos a seguir as habilidades de outras áreas, presentes na abordagem
interdisciplinar:

\subsection{Pré-leitura}

Na história, ver-se-á que o fato de Ivan Ilitch ter ignorado
a lesão que sofreu com a queda pode ter potencializado o problema que o
levaria a óbito. A própria redação aplicada por Tolstói ao trecho, por
si só, já denota a indiferença com que Ivan Ilitch lidou com a situação.
Isso é algo muito comum entre as pessoas. Sentirem um mal estar, uma dor
e terem a crença de que aquilo irá passar em breve. Muitas das vezes,
sim, passa, porém, há situações em que esses incômodos são indícios de
uma mazela ainda pior por vir. Em vista disso, a presente atividade
sugere debater com os alunos a importância da medicina preventiva como
estratégia de combate às doenças. Devem ser levantadas questões como o
papel do Poder Público no combate às doenças e até onde é possível a
atuação do Estado. Também deve ser levantada a importância da postura do
indivíduo frente às questões de saúde, como a preservação de sua
integridade, o controle da disseminação de doenças e, também,
estratégias que melhoram a qualidade de vida, prolongando o bem estar.

\subsection{Leitura}

Ainda em relação à atividade anterior, esse exercício sugere
a pesquisa sobre a legislação tangente às funções da administração
pública, oferecendo ao aluno subsídios para melhor compreensão do
funcionamento dos órgãos do Estado, por exemplo, de como são preenchidos
os cargos públicos, quais os limites da administração pública, quais são
os deveres da administração pública e quais são os direitos que o
cidadão tem frente a seus administradores e como avaliar se os serviços
são corretamente prestados. Sendo uma atividade complexa, recomenda-se
que a sala seja dividida em grupos. Ao professor cabe uma pequena
explanação inicial sobre a evolução do Estado Democrático de Direito
para que, a partir disso, possa ser feito um exercício em que os alunos
se valham de pesquisa em textos legais para compreender os pontos acima
aventados. Feito isso, os grupos podem apresentar seminários expondo o
conteúdo pesquisado.

Posteriormente, individualmente, os alunos podem fazer uma dissertação
avaliando a real efetividade e qualidade dos serviços públicos prestados
na comunidade local.

\subsection{Pós-leitura}

Na mesma seara, na atividade de pós-leitura, aconselha-se um
debate acerca de mecanismos para a melhoria da saúde pública. Questões
que podem ser propostas para fomentar a discussão podem perpassar a
questão da universalidade de atendimentos; a capacidade financeira do
Estado em prestar o serviço, o direito constitucional à saúde, o dever
moral do Estado em garantir o bem-estar de seus administrados, as
vantagens e desvantagens que parcerias público-privadas podem gerar no
atendimento. Até que ponto é benéfico o papel de convênios, entre
outros. A atividade também sugere que seja ponderada a burocracia
empregada no bom atendimento às pessoas adoentadas, e às pessoas que vão
fazer exames, ou se previnir. Tólstoi, como já visto no livro, é um
crítico do excesso de burocracia estatal na Rússia Imperial. Nesse
sentido, cabe também a reflexão. Até que ponto o Brasil oferece uma
situação melhor a seu povo, ou será que ele está similar ao governo do
czarismo.

\section{Aprofundamento}

Ao chegar ao Ensino Médio, é necessário que os estudantes se aprofundem
na compreensão das múltiplas linguagens e, sobretudo, da linguagem
literária. Em relação à literatura, a BNCC traz as seguintes
considerações:

\begin{quote}
{[}...{]} a leitura do texto literário, que ocupa o centro do trabalho
no Ensino Fundamental, deve permanecer nuclear também no Ensino Médio.
Por força de certa simplificação didática, as biografias de autores, as
características de épocas, os resumos e outros gêneros artísticos
substitutivos, como o cinema e as HQs, têm relegado o texto literário a
um plano secundário do ensino. Assim, é importante não só (re)colocá-lo
como ponto de partida para o trabalho com a literatura, como
intensificar seu convívio com os estudantes. Como linguagem
artisticamente organizada, a literatura enriquece nossa percepção e
nossa visão de mundo. Mediante arranjos especiais das palavras, ela cria
um universo que nos permite aumentar nossa capacidade de ver e sentir.
Nesse sentido, a literatura possibilita uma ampliação da nossa visão do
mundo, ajuda-nos não só a ver mais, mas a colocar em questão muito do
que estamos vendo/vivenciando. (Brasil, 2018, p. 491)
\end{quote}

Nesta seção, desenvolvemos um trabalho de aprofundamento que, em diálogo
com a formação continuada de professores, oferece subsídios para a
abordagem do texto literário. A leitura em sala de aula de \emph{A Morte
de Ivan Ílitch} pode ser enriquecida pelo aprofundamento no universo
literário de Lev Tóstoi e da literatura russa como um todo.

\subsection{A obra}

Publicado em 1886, o conto A Morte de Ivan Ílitch faz parte de uma série
de textos breves que um já consagrado Tolstói produzira por volta dos
seus cinquenta anos de vida. Em uma narrativa leve, mas nem por isso
menos profunda, Tolstói conta a história da vida e da morte do senhor
Ivan Ílitch Golovín, promovendo para o leitor uma reflexão acerca da
vida e da morte, enquanto narra os caminhos da vida da personagem
principal, de sua atividade junto ao judiciário, por meio da qual chega
num ponto alto de sucesso, para então adoecer e sucumbir.

A crítica sutil, salpicada com amaciada ironia, expõe as futilidades e
pequenezas de um universo de classe média na Rússia Imperial.
Curiosamente, alguns apontamentos do autor, ainda que claramente
indiquem a que se remetem, são transcendentais, ultrapassando os limites
físicos e temporais daquela sociedade. Não obstante, os lampejos cada
vez mais humanizados que se presentificam nos pensamentos da personagem
principal a aproximam de qualquer um de nós, na condição de humanos, ao
a acompanharmos pela sua áspera jornada.

\subsection{Uma personagem de muitos convívios sociais, mas essencialmente
sozinha -- crítica às relações humanas}

Logo que o conto se inicia, com o anúncio da morte de Ivan Ilitch
Golovin, funcionário público do Judiciário imperial russo, temos já
muito clara a impressão de como fora a vida desse distinto cidadão.

A notícia saíra em um periódico local, sendo lida por seus colegas de
trabalho. De certo modo consternados, eles debatem acerca da morte do
homem, ao mesmo tempo que refletem sobre as possíveis realocações -- em
razão do óbito -- de nomes nos cargos da repartição, bem como se é de
fato necessário ter de suportar o fardo social do comparecimento ao
velório.

Um desses homens, Piotr Ivanovitch, que havia, inclusive, estado na
convivência de Ivan Ílitch desde os tempos da escola de jurisprudência,
se sente compelido a comparecer ao velório, muito em razão dos fardos
dos costumes sociais, entretanto, isso não o impediu de comentar com a
esposa sobre possíveis benesses que sua família poderia receber pela
realocação dos funcionários nos cargos do tribunal.

Deseja ser breve no velório, afinal marcara um carteado com outros
companheiros, mas, ao chegar lá, é interpelado pela viúva que o chama
para conversar, delongando assim seu passatempo lúdico. Ela deseja
indagar-lhe sobre a possibilidade do recebimento de uma pensão mais
gorda. Ao ouvir o pedido, Piotr Ivánovitch polidamente recriminou o
Estado por sua mesquinhez, indicando não crer ser possível o recebimento
de um benefício mais robusto.

Curioso é que tal diálogo se segue às reclamações da viúva das agruras
que ela vivera nos últimos momentos do defunto, posto que, enquanto este
moribundo, gritara incessantemente por três dias. Disso, observar-se a
falta de empatia que a agora viúva sentira nas últimas horas de
sofrimento de seu marido. Inclusive, Tolstói dá a entender que o assunto
sobre o crepúsculo da vida de Ivan Ilitch se dera unicamente para
cumprir o protocolo social.

É realizada a cerimônia, e, após uma breve conversa com um funcionário
da casa, Piotr Ivánovitch vê que ainda é tempo de participar da
jogatina, de modo que se dirige à casa do amigo onde ela ocorria.

\subsection{A trajetória de Ivan Ilitch -- crítica ao funcionalismo público}

Findado esse trecho introdutório, Tolstói passa a narrar a história de
Ivan Ilitch. Desde o início, Ivan Ilitch é pintado como um sujeitinho
mediano, nascido em uma família de posses medianas, cujo pai atuava como
funcionário público, daqueles que, na definição de Tolstói, estão lá há
muito tempo para serem mandados embora, de modo que tem-lhes que ser
elaborado um cargo, de atribuições fictícias e de salário nada fictício,
o qual preencham.

Aliás, toda vez que o Estado se presentifica na obra, queda claro como é
um Estado, excessivo, sedentário, acomodado. Tolstói escancara a
falência da intenção de Pedro I com sua tabela de patentes. Observa-se
funcionários prevaricadores, preguiçosos, corruptos. As relações se dão
por interesse e o tráfico de influência parece ser o melhor de todos os
planos de carreira.

Ora, as críticas ao funcionalismo público, às burocracias ineficazes e
aos profissionais de qualidade duvidosa são lugar comum na literatura
russa desde os tempos de Gógol, sendo a questão também amplamente
abordada em Dostoiévski. Na literatura nacional, Nelson Rodrigues
explorará também essa temática.

Com tal quadro em mente, e observando o contexto em que a personagem
principal está inserida, vê-se que esta, sem grandes brilhos, Ivan
Ilitch, cresce, estuda, e tal qual o pai ingressa no funcionalismo
público, observando rigorosamente uma cartilha social, a qual segue e
deseja seguir sem imprimir qualquer reflexão acerca desse caminho.
Correspondia ao básico exigido. Em conversas banais, cumpria o protocolo
apresentando um descontentamento leve em relação ao governo, adotando um
tom de liberalismo moderado, adequando-se então aos modismos da época,
mas sem correr o risco de se indispor com o fornecedor de seu pão
cotidiano.

\subsection{A vida familiar e social -- de um incômodo necessário a um fardo
insuportável}

Ivan Ilitch se casou porque era isso que se esperava de um homem
classificado como decente, mas, ao longo da história, ver-se-á que o
matrimônio constituiu pouco mais do que uma fonte de aborrecimento.

Apesar de um início aquecido, o casamento logo esfriara, e o
relacionamento amoroso dera lugar a conflitos cotidianos por motivos
pequenos, o que fazia com que Ivan Ilitch evitasse a companhia dos
familiares, buscando sempre se refugiar no trabalho.

Nota-se a ausência de empatia, de desejo pelo convívio familiar. Todos
ao seu redor, em verdade, parecem estranhos que, por uma determinação
social, é-lhe obrigado o convívio. Não existe envolvimento afetivo, a
família é como se fosse, perante à sociedade, o mesmo que a toga é para
o cargo de juiz, um uniforme, uma formalidade, um elemento tradicional
que confere distinção a seu possuidor.

Trata-se do completo esvaziamento do significado das relações
familiares.

Entretanto, a despeito dessa aparente dedicação ao trabalho, seu
verdadeiro prazer, onde sentia que sobressaíam seus talentos eram nos
jogos de baralho. Mas, também a despeito de sua aparente dedicação ao
trabalho, um remanejamento das cartas no funcionalismo público em sua
província ameaçou seu conforto e possibilidade de promoção.

Frustrado, decide agir. Viaja à capital do Império, buscando realocação
e, por sorte, encontra no trem conhecidos dispostos a lhe auxiliar. Com
isso, consegue a transferência, para um cargo distante daqueles que lhe
opuseram, com um ordenado de cinco mil rublos e mais três mil e
quinhentos de abono pela mudança.

O sucesso do novo posto faz com que até parentes, há muito sem interesse
em um contato, o procurem.

\subsection{O estampido de sucesso e o ocaso da vida de Ivan Ilitch}

Satisfeito e empolgado, muda-se adiante de sua família, com o intuito de
preparar um bom lar, que impressionasse a todos.

Contudo, nos preparativos finais, Ivan Ilitch sofre uma queda, com a
qual acaba recebendo uma pancada no flanco esquerdo. Considerando esse
acidente um infortúnio sem importância, Ivan Ilitch segue sua vida,
recebe a esposa e filhos com felicidade e até brinca, imitando sua
queda. Assume o posto, ganhando bem, um ordenado do qual faltam apenas
quinhentos rublos para uma vida satisfatória. Porém, a euforia advinda
do sucesso profissional e da mudança rapidamente findou-se. Passou a
sentir uma dor constante, que o molestava na hora da alimentação, além
de ter a impressão de possuir um cheiro repugnante saindo de sua boca. A
harmonia conjugal também se deteriorou, retornando ao cotidiano anterior
à nova vida na capital.

O incômodo constante leva Ivan Ilitch a uma constante irritabilidade.
Ora, por que isso acontecia com ele? Por que a dor não cessava? A
perpetuação do incômodo amplificava seu mal estar físico, uma vez que
contaminava seu psicológico a todo instante.

Busca médicos, renomados, famosos, alternativos. Cada qual parece trazer
um diagnóstico, uma cura milagrosa que ocorreria em breve. Ivan Ilitch
vê nos médicos uma empáfia. A arrogância da academia, olhando com
desprezo para seus pacientes, sem o menor interesse em ouvir suas
queixas, em uma postura similar com a qual ele tivera como juiz na
condução dos procedimentos jurídicos e na oitiva dos comuns com quem
cruzara.

O desespero vai lhe consumindo a consciência. Lembra-se dos silogismos
da época de estudo, em que aprendera que se todo homem é mortal, e Caio
é homem, logo caio é mortal. Esse pensamento o entristece. Pouco importa
que Caio é mortal, o que importa é que ele, Ivan Ilitch, também o é.

Não desejando a morte, mas tendo que suportar o incômodo incessante,
Ivan Ilitch passa a utilizar ópio para se aliviar. Enquanto padece,
reflete sobre sua história. E nos caminhos sinuosos de seu pensamento
percebe que tem poucas lembranças de verdadeira felicidade.

Irrita-se ao ver sua família seguindo a vida normalmente, vendo que
tornou-se um estorvo, e suspeita que talvez sempre o tenha sido. Observa
sua mulher interesseira, vê sua filha em um relacionamento com outro
carreirista e seu filho como estudante trilhando passos similares aos
seus.

Chegou a odiar a mulher, atribuindo-lhe a razão de seus malefícios. O
que era comum, sempre atribuía a terceiros as razões de seus
infortúnios. Ela também desejara sua morte, porém se assustou com seu
desejo, afinal, a morte do marido significaria o fim do ordenado.

\subsection{A redenção em um amigo inesperado}

No final da vida, tem um pouco de alívio ao ser cuidado pelo criado
Guerássim. Um mujique, jovem, de vistosa compleição. Ao contrário dos
médicos, e todos do convívio de Ivan Ilitch, tinha ânimo e boa vontade
para auxiliar um moribundo. Levantava-lhe as pernas, única posição em
que Ivan Ilitch sentia um pouco de conforto, quedando-se horas assim,
para o alívio de seu patrão, sem se importar ou emitir qualquer queixa.

Tinha bom espírito para todas as suas atitudes, opondo-se à perpétua
tonalidade \emph{blasé} vivida Ivan Ilitch em praticamente toda a sua
vida. Também fora o único que não cultivou falsas esperanças no patrão
moribundo. Sabia que ele iria morrer e lidava naturalmente com isso, não
vendo perda de tempo em seu cuidado, apesar de que a causa poderia ser
vista como perdida. Fazer bem ao próximo lhe bastava.

Nesse momento, já em seus instantes finais, Ivan Ilitch percebe o qual
insossa foi sua vida. Percebe que a razão dessa colheita insípida foi
fruto de sua própria semeadura medíocre. Lamenta o tempo perdido com
frivolidades. Percebe que toda a pompa fora desimportante, e que tudo em
sua vida se resume, com exceção de algumas recordações de sua meninice,
a uma rotina afetada.

Similarmente a Memórias Póstumas de Brás Cubas, temos em A Morte de Ivan
Ilitch uma história que começa a partir da morte da personagem principal
e, no desenrolar do livro, conhecemos com mais detalhes sua biografia. A
diferença resta no fato de aquele saber que está morto, e sua narrativa
consiste em rememorações de sua história já consciente de seu fim,
enquanto com Ivan Ilitch, embora saibamos de seu fim, percorremos sua
história pari passu com suas angústias e temores.

E é no passar dessas linhas, é-nos apresentada a genialidade de Tolstói,
que tão brandamente conduz a narrativa. No livro, saímos muito
sutilmente, do universo cotidiano, mesquinho, ganancioso e pretensioso
das repartições, para irmos, paulatinamente, quase como em adágio, ao
mais íntimo psicológico de Ivan Ilitch.

Tolstói é tão delicado no conduzir de sua história que mal percebemos
que a causa dos malefícios do protagonista foi sua queda, tal como
quando sentimos uma dor inesperada e não nos lembramos de quando nos
ferimos em um momento anterior.

O livro começa retratando a mediocridade cotidiana, porém, a narrativa
que conduz o caminho longo até a morte vai se esvaindo do relato da
pequenez, até que a personagem principal atinge um algo mais sublime.

E tão delicadamente a personagem sai da sensação de penitência por ter
gastado prodigamente a vida, para um estado mais contemplativo e
benevolente frente à sua finitude.

Já em suas últimas horas, é visitado no cômodo que passara a habitar por
seu filho. Ao ver as olheiras do filho, Ivan Ilitch se compadece do
filho. É um momento de compaixão mútua, talvez um dos primeiros momentos
de sentimentos espontâneos narrados no oceano de afetação que até então
viviam

Nutre compaixão também pela esposa, apesar de dias antes ter nutrido por
ela sentimentos tão negativos.

Também se apieda da filha, de seu noivado com outro funcionário público,
tão medíocre quanto ele fora. Talvez sentisse um incômodo por isso, uma
vez que essa questão, sempre que se apresentava, dominava seu
pensamento, tal qual uma nota aguda em contraponto à melodia.

Essa compaixão é uma espécie de amor. A percepção de que nutria um bem
querer por sua família. Um amor fundamental, remissivo, que permitiu a
paz a Ivan Ilitch, para que ele se permitisse ofuscar diante de uma
realidade maior que a dele próprio. É nessa hora ele se libertou da dor,
do terror da morte e sutilmente, após dias de tormento, fora levado em
seu último suspiro.

A plenitude atingida é o que marca seu último sopro. Ainda tentara dizer
perdão a mulher e aos filhos, porém as palavras não saíram. Mas isso
pouco importava, sua essência e consciência estavam agora harmônicas.
Não à toa que os ``amigos'' de Ivan Ilitch que compareceram ao velório
notaram que seu cadáver guardava uma expressão mais significativa do que
a que tivera em vida.

Assim se encerra o conto. Assim somos confrontados com as reflexões de
um homem moribundo; observamos a penúria de um ser em confronto consigo
mesmo, nos momentos de certeza de que não escapará do destino final e
certo.

Além disso, também somos convidados a olhar para dentro e pensarmos
sobre nós. Desse modo, Ivan Ilitch está em nós. Quando olhamos atônitos
a passagem do tempo e consternados nos debatemos a pensar se aquele
passo fora bem dado, se aquela resposta foi adequada, se há sentido na
escolha feita, se o propósito impresso foi realmente valoroso. E dessa
leitura talvez tenhamos um pequeno susto, um levantar de sobrancelhas,
um menear de cabeça e com isso compreendemos a magnitude desse conto.

\subsection{Por que ler A morte de Ivan Ilitch?}

Ainda que escrito no século XIX, A Morte de Ivan Ilitch é um livro
atual. Tanto em sua crítica ao funcionalismo público, com suas práticas
tão distantes de um ideal cívico, quanto, e principalmente, por nos
ofertar uma reflexão acerca da vida, de sua efemeridade e fragilidade,
da certeza da morte e como com ela lidamos, dos valores empregados e das
opções feitas ao longo do viver. A obra nos faz pensar sobre nossas
ambições e vaidades, sobre aquilo que almejamos, sobre o que nos traz
bem-estar.

Em outros termos, o conto demonstra que o homem, através dos tempos, não
é tão distinto quanto possa parecer a um primeiro olhar. Ao contrário, é
bem similar a nós. Em suas angústias mais íntimas, no imbricado de suas
relações familiares e sociais, nos anseios da vida profissional e no
temor à morte somos todos, tão humanos. Assim, é possível nos vermos
refletidos na personagem, ainda que de outro tempo e outro mundo.

A grandeza da literatura russa está em ser capaz de descrever
visceralmente as situações mais cotidianas. No conto em questão, somos
colocados próximos de uma pessoa que nunca vimos, de um mundo tão
distante, em um tempo longínquo e em uma situação provavelmente distinta
da nossa quando da leitura, ainda assim, nos identificamos e nos
compadecemos, amplificando aquilo que há de mais humano em todos nós.

\section{Sugestões de atividades complementares: relações dialógicas e
intertextuais}

%\BNCC{EM13LP03}
%\BNCC{EM13LP04}
%\BNCC{EM13LP49}
%\BNCC{EM13LP51}

No Ensino Médio, da mesma forma que no Ensino Fundamental, a \textsc{bncc}
organiza o trabalho com as práticas de linguagem em cinco \textbf{campos
de atuação social}. São eles: campo da vida pessoal, campo da vida
pública, campo jornalístico"-midiático, campo artístico"-literário e campo
das práticas de estudo e pesquisa.

De acordo com essa divisão, propomos na sequência um trabalho
interdiscursivo e intertextual com a obra \emph{A Morte de Ivan Ilitch.}

\subsection{Campo da vida pessoal}

\begin{quote}
O campo da vida pessoal pretende funcionar como espaço de articulações
e sínteses das aprendizagens de outros campos postas a serviço dos
projetos de vida dos estudantes. As práticas de linguagem privilegiadas
nesse campo relacionam"-se com a ampliação do saber sobre si, tendo em
vista as condições que cercam a vida contemporânea e as condições
juvenis no Brasil e no mundo.

Está em questão também possibilitar vivências significativas de práticas
colaborativas em situações de interação presenciais ou em ambientes
digitais e aprender, na articulação com outras áreas, campos e com os
projetos e escolhas pessoais dos jovens, procedimentos de levantamento,
tratamento e divulgação de dados e informações e o uso desses dados em
produções diversas e na proposição de ações e projetos de natureza
variada, para fomentar o protagonismo juvenil de forma
contextualizada. (\textsc{bncc}, p. 494)
\end{quote}

Ivan Ilitch, protagonista de \emph{A Morte de Ivan Ilitch}, viveu uma
vida morna, chegando à conclusão disso apenas nos últimos dias de sua
vida, quando percebeu o quanto era infeliz, além de estranho e alheio
às pessoas que o cercavam. Essa situação não é tão incomum, afinal,
muitas pessoas até próximas, desconhecem elementos muito íntimos de
nosso ser. O que é paradoxal, afinal, vivemos em um período de grande
exposição nas redes sociais. A situação torna-se tão \emph{sui
generis} que muitas vezes nós mesmos desconhecemos nossos potenciais.
Diante disso, sugere-se que o aluno monte um quadro pessoal, onde
apontará dez qualidades individuais, cinco no campo das relações
pessoais e cinco no universo escolar e do trabalho. Essa atividade não
precisa ser exposta aos demais, mas se conhecer é um bom ponto de
partida inclusive para o aprendizado.

\subsection{Campo de atuação na vida pública}

\begin{quote}
No cerne do campo de atuação na vida pública estão a ampliação da
participação em diferentes instâncias da vida pública, a defesa dos
direitos, o domínio básico de textos legais e a discussão e o debate de
ideias, propostas e projetos. {[}\ldots{}{]}

Ainda no domínio das ênfases, indica"-se um conjunto de habilidades que
se relacionam com a análise, discussão, elaboração e desenvolvimento de
propostas de ação e de projetos culturais e de intervenção social.
(\textsc{bncc}, p. 494)
\end{quote}

Em \emph{A Morte de Ivan Ilitch}, as alusões a um sistema
jurisdicional caro, burocrático, ineficiente e desumanizado são
constantes. Infelizmente, a impessoalidade do serviço público, que é
um princípio, quando pensada para tratar isonomicamente os
administrados, acaba por se tornar também uma mazela, quando
desumaniza estes, desconsiderando assim, até mesmo, a sua própria
humanidade. Infelizmente, é uma situação muito cotidiana. Entretanto,
reformular o sistema como está posto não é uma tarefa tão simples.
Porém, pode ser pensada. Sugira aos alunos que redijam um projeto de
lei, visando tornar mais fácil o acesso do administrado com seus
administradores, por exemplo, pode-se criar uma sala virtual de
participação em decisões da Câmara Legislativa, entre outras
interseções de atividades presenciais e a distância que podem
aproximar esses dois pólos da relação.

\subsection{Campo jornalístico"-midiático}

\begin{quote}
Em relação ao campo jornalístico"-midiático, espera"-se que os jovens
que chegam ao Ensino Médio sejam capazes de: compreender os fatos e
circunstâncias principais relatados; perceber a impossibilidade de
neutralidade absoluta no relato de fatos; adotar procedimentos básicos
de checagem de veracidade de informação; identificar diferentes pontos
de vista diante de questões polêmicas de relevância social; avaliar
argumentos utilizados e posicionar"-se em relação a eles de forma ética;
identificar e denunciar discursos de ódio e que envolvam desrespeito aos
Direitos Humanos; e produzir textos jornalísticos variados, tendo em
vista seus contextos de produção e características dos gêneros. Eles
também devem ter condições de analisar estratégias
linguístico"-discursivas utilizadas pelos textos publicitários e de
refletir sobre necessidades e condições de consumo.

No Ensino Médio, os jovens precisam aprofundar a análise dos interesses
que movem o campo jornalístico midiático, da relação entre informação e
opinião, com destaque para o fenômeno da pós"-verdade, consolidar o
desenvolvimento de habilidades, apropriar"-se de mais procedimentos
envolvidos na curadoria de informações, ampliar o contato com projetos
editoriais independentes e tomar consciência de que uma mídia
independente e plural é condição indispensável para a democracia.

Como já destacado, as práticas que têm lugar nas redes sociais têm
tratamento ampliado. (\textsc{bncc}, p. 494-495)
\end{quote}

A leitura de todo livro transporta o leitor para outro universo.
Assim, como um expectador contemporâneo, ele observa os eventos
narrados, passando a organizá-los em sua mente, de acordo com suas
memórias e conhecimentos pretéritos. O intelectual Gilberto Freyre uma
vez teria dito que ``O Brasil é a Rússia Americana''. Essa frase se
pautou no fato de ambos os países serem de dimensões continentais,
possuírem clima extremo e terem abolido seus sistemas de trabalho
compulsório tardiamente. Entretanto, pairam ainda elementos muito
parecidos entre os dois países, em suas respectivas realidades. Peça
ao aluno para fazer um quadro comparativo, a partir do livro lido,
entre as similitudes e diferenças de Brasil e Rússia.

\subsection{Campo artístico"-literário}

\begin{quote}
No campo artístico"-literário busca"-se a ampliação do contato e a
análise mais fundamentada de manifestações culturais e artísticas em
geral. Está em jogo a continuidade da formação do leitor literário e do
desenvolvimento da fruição. A análise contextualizada de produções
artísticas e dos textos literários, com destaque para os clássicos,
intensifica"-se no Ensino Médio. Gêneros e formas diversas de produções
vinculadas à apreciação de obras artísticas e produções culturais
(resenhas, vlogs e podcasts literários, culturais etc.) ou a formas de
apropriação do texto literário, de produções cinematográficas e teatrais
e de outras manifestações artísticas (remidiações, paródias,
estilizações, videominutos, fanfics etc.) continuam a ser considerados
associados a habilidades técnicas e estéticas mais refinadas.

A escrita literária, por sua vez, ainda que não seja o foco central do
componente de Língua Portuguesa, também se mostra rica em possibilidades
expressivas. (\textsc{bncc}, p. 495-496).
\end{quote}

Retome com os estudantes o painel de sala, cuja organização foi
proposta na Atividade 2 e cuja montagem foi enriquecida nas sucessivas
etapas de leitura da obra. Por meio das informações colhidas, do
registro de análises e impressões de leitura, e da pesquisa de dados
contextuais, proponha a escrita coletiva de um \textbf{obituário de
Ivan Ilitch}. Com a turma dividida em grupos, transforme a sala na
seção de um periódico do século XIX responsável por divulgar mortes de
cidadãos. Cada equipe redigirá uma nota apresentando as circunstâncias
de morte, a descrição do velório e do enterro do protagonista,
depoimentos imaginários de personagens, elogios irônicos à carreira e
à personalidade do defunto. Estimule a pesquisa, em \emph{sites} de
internet, de obituários de pessoas famosas, com o objetivo de
trabalhar a retórica elegíaca do gênero obituário, e incentive a
recuperação de episódios centrais da vida de Ivan Ilitch que possam
enriquecer a homenagem. Após a revisão da primeira versão do texto
pelo(a) professor(a) responsável, os alunos utilizarão o editor de
textos do computador para redigir a versão final, que poderá ser
publicada no \emph{site} da escola ou em um blog reservado às
experiências de leitura da obra.

\subsection{Campo das práticas de estudo e pesquisa}

\begin{quote}
O campo das práticas de estudo e pesquisa mantém destaque para os
gêneros e habilidades envolvidos na leitura/escuta e produção de textos
de diferentes áreas do conhecimento e para as habilidades e
procedimentos envolvidos no estudo. Ganham realce também as habilidades
relacionadas à análise, síntese, reflexão, problematização e pesquisa:
estabelecimento de recorte da questão ou problema; seleção de
informações; estabelecimento das condições de coleta de dados para a
realização de levantamentos; realização de pesquisas de diferentes
tipos; tratamento dos dados e informações; e formas de uso e
socialização dos resultados e análises.

Além de fazer uso competente da língua e das outras semioses, os
estudantes devem ter uma atitude investigativa e criativa em relação a
elas e compreender princípios e procedimentos metodológicos que orientam
a produção do conhecimento sobre a língua e as linguagens e a formulação
de regras. (\textsc{bncc}, p. 495-496)
\end{quote}

Proponha uma pesquisa para ampliação do repertório cultural e para
compreensão do lugar ocupado pelas personagens na obra lida. Para uma
maior compreensão da sociedade da época de Tolstói, é interessante que
o aluno faça uma pesquisa em que possa compreender o sistema de
ranques de Pedro I, bem como era dos mujiques (camponeses) no período.
Como foram as transformações sócio-econômicas com o fim da servidão e
como as agitações políticas - o próprio Ivan Ilitich se definia como
uma espécie de reformista moderado - do período podem ser vistas como
um prenúncio do que estava por vir. Os alunos podem se dividir em
grupos e apresentar um artigo a respeito das questões acima indicadas,
fazendo uma análise crítica dos acontecimentos históricos que
precederam, envolveram e sucederam os eventos narrados em \emph{A
Morte de Ivan Ilitch.}

\section{Referências complementares}

\subsection{Livros}

\begin{itemize}
\item\textsc{batuman}, Elif. \textit{Os possessos}. São Paulo: Leya, 2012.

O leitor acompanha uma trama policial envolvendo Tolstói, visita a
imperial São Petersburgo, passa pelo Uzbequistão e segue os rastros de
Pushkin no Cáucaso.

\item\textsc{parini}, Jay. \textit{A última estação}. Rio de Janeiro: Record, 2009.

O autor combina verdade e ficção para recriar o último ano da vida de
Tolstói, acompanhando-o até seus últimos momentos, que sucederam sua
dramática fuga de casa.

\item\textsc{salerno}, Silvana (adapt.). \textit{guerra e paz}. São Paulo: Companhia
das Letras, 2010.

O livro é uma adaptação que resume as mais de mil páginas do clássico de
Tolstói, que trata a invasão da Rússia por Napoleão Bonaparte no começo
do século XIX.

\item\textsc{tolstói}, Liev. De quanta terra precisa o homem? São Paulo:
Companhia das Letrinhas, 2009.

O grande escritor russo recria neste conto a dramática história de um
homem obcecado pelo desejo de conseguir mais e mais terras.
\end{itemize}

\subsection{Filme}

\begin{itemize}
\item\textbf{Anna Karenina. Direção: Joe Wright (Reino Unido, 2013).}

Neste filme conceitual, Moscou é um grande palco onde é recriado o drama
de uma das mais poderosas personagens de Tolstói, Anna Karenina, que tem
seu pedido de divórcio negado e é privada do direito de ver seus filhos.

\item\textbf{A última estação. Direção: Michael Hoffman (Reino Unido, Rússia,
Alemanha, 2009).}

Filme dramatiza os últimos momentos da vida de Liev Tolstói, na estação
ferroviária de Astapovo. Também aborda a disputa entre a esposa de
Tólstoi, Sophia, e seu discípulo Vladimir Chertkov pelos direitos de sua
obra.
\end{itemize}

\section{Bibliografia comentada}

\begin{itemize}
\item\textsc{bartlett}, Rosamund. \textit{Tolstói: a biografia}. São Paulo: Biblioteca Azul, 2013.

A autora se debruça sobre a vida e a obra de uma das figuras mais
emblemáticas da Rússia, conhecido não só por seus grandes escritos, mas
também pelo misticismo do homem que passou por uma profunda
transformação ao longo da vida.

\item\textsc{berlin}, I. \textit{Pensadores Russos}. São Paulo: Companhia das Letras, 1989.

Com aguçada capacidade crítica, o autor reúne nessa obra de fôlego
diversos ensaios sobre a Rússia do século XIX, do tempo dos czares.

\item\textsc{billington}, J. H. \textit{The icon and the axe. An interpretative
history of Russian culture}. New York: Vintage Books, 1970.

Livro traz uma descrição abrangente da história cultural russa, indo da
era pré-Romanov até os tempos comunistas de Joseph Stalin.

\item\textsc{bushkovitch}, Paul. \textit{História Concisa da Rússia}. São Paulo:
Edipro, 2015.

Este livro de leitura acessível, traça um amplo panorama da história
russa desde o século IX, passando pelos principais acontecimentos das
artes e das ciências.

\item\textsc{figes}, Orlando. \textit{Rússia: uma história cultural}. Rio de Janeiro: Record, 2017.

O renomado historiador analisa os grandes artistas e as mais importantes
manifestações culturais russas, revelando o espírito dessa nação.

\item\textsc{segrillo}, Angelo. \textit{Os russos}. São Paulo: Contexto, 2012.

Especialista em história e literatura, o autor fala com propriedade
sobre esse povo complexo e com tantas contribuições importantes para a
humanidade, na pintura, na música, na política e outras áreas.

\item\textsc{tolstói}, Liev. \textit{Os últimos dias}. São Paulo: Penguin, 2011.

O livro reúne ensaios, cartas, parábolas e fragmentos de obras de
Tolstói, escritos a partir de 1882. Traduzido diretamente do russo.

\item\textsc{volkov}, S. \textit{São Petersburgo: uma história cultural}. Rio de
Janeiro: Record, 1997.

Volkov se aprofunda na história e simbologia da cidade imperial, que,
segundo ele, ganhou força mítica a partir do poema O Cavaleiro de
Bronze, de Puchkin.
\end{itemize}

\end{document}

