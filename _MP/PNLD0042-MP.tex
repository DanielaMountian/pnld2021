\documentclass[12pt]{extarticle}
\usepackage{manualdoprofessor}
\usepackage{fichatecnica}
\usepackage{lipsum,media9,graficos}
\usepackage[justification=raggedright]{caption}
\usepackage[one]{bncc}
\usepackage[madalena]{../edlab}



\begin{document}


\newcommand{\AutorLivro}{Liev Tolstói}
\newcommand{\TituloLivro}{A morte de Ivan Ílitch}
\newcommand{\Tema}{Ficção, mistério e fantasia}
\newcommand{\Genero}{Romance}
\newcommand{\imagemCapa}{./images/PNLD0042-01.png}
\newcommand{\issnppub}{---}
\newcommand{\issnepub}{---}
% \newcommand{\fichacatalografica}{PNLD0042-00.png}
\newcommand{\colaborador}{\textbf{Bruno Gradella e Vicente Castro}}


\title{\TituloLivro}
\author{\AutorLivro}
\def\authornotes{\colaborador}

\date{}
\maketitle

\baselineskip=1.15\baselineskip\par


\begin{abstract}\addcontentsline{toc}{section}{Carta ao professor}
Este Manual tem como objetivo fornecer subsídios para o trabalho com a
obra literária \emph{A Morte de Ivan Ílitch}, de Liev Tolstói.
Liev Tolstói, nascido na Rússia no dia 9 de setembro de 1828, foi um escritor amplamente 
reconhecido como um dos maiores de todos os tempos. Estudante de direito e línguas 
orientais na Universidade de Cazã, abandonou o curso para se juntar ao exército. 
A experiência no exército seguida de duas viagens pela Europa (em 1857 e 1860) 
foram muito marcantes para Tolstói, transformando-o em um anarquista pacifista. 
Foi a partir dessa experiência que ele começou a escrever alguns dos romances 
pelos quais ficou famoso, como \textit{Guerra e Paz} (1869) e \textit{Anna Karenina} (1877). 
 
\textit{A Morte de Ivan Ílitch}, publicado a primeira vez em 1886 quando 
Tolstói já tinha 50 anos, também é fruto de sua experiência no exército 
e de suas viagens pela Europa. O livro, na verdade uma novela, é uma narrativa 
leve que conta a história do senhor Ivan Ílitch Goloví, seu caminho até se tornar,
a exemplo de seu pai, um funcionário bem estabelecido do setor judiciário que, 
por decorrência de um pequeno acidente, uma queda mal cuidada, acaba morrendo. 
E do seu leito de morte a narrativa segue em forma de retrospectiva, numa das mais
propostas de reflexão sobre vida e morte da literatura mundial.

Não podemos deixar de reparar que este procedimento nos lembra o do nosso conhecido
e também importantíssimo Machado de Assis nas \textit{Memórias Póstumas de Brás Cubas},
ainda que haja diferenças significativas entre os dois textos, como o fato de que, aqui,
diferente de Brás Cubas, Ivan não toma consciência após a morte, mas ainda em vida. 
O narrador começa retratando a mediocridade da vida cotidiana, porém, a narrativa 
que conduz o caminho longo até a morte vai ganhando um tom mais sublime. 
Ivan sente, em seu leito de morte, um tipo de penitência inclusive pelo 
desleixo em relação ao acidente que o levou àquele estado. Mas por meio da 
constatação de ter levado a vida de forma digna, passa a um estado mais 
contemplativo e benevolente frente à sua finitude, ficando em paz consigo mesmo.

Esperamos que as indicações propostas aqui sejam úteis no trabalho em
sala de aula! 

\end{abstract}



\tableofcontents

\section{Atividades 1}

%\BNCC{EM13LP26}

\subsection{Pré-leitura}

%\BNCC{EM13LGG302}
%\BNCC{EM13LGG704}
%\BNCC{EM13LP10}
%\BNCC{EM13LP19}

\paragraph{Tema} Como a morte é vista em distintas culturas.

\paragraph{Conteúdo} Pesquisa e reflexão acerca do tema da morte, 
através de materiais impressos e digitais. Redação um texto
sobre o tratamento dado à morte em uma das tradições pesquisadas. 

\paragraph{Objetivo} Ampliar o conhecimento dos estudantes sobre o tema da morte
em diferentes culturas e religiões. Proporcionar uma reflexão acerca
de como distintos estilos literários se valem desse tema.

\paragraph{Justificativa} Em \emph{A Morte de Ivan Ílitch}, como o próprio título já
indica, a temática principal abordada na obra é a morte. Em específico,
como lidar com ela, como ela representa a finitude e como nos esquecemos
de sua inexorável chegada enquanto vivemos, muitas vezes nos prendendo e
despendendo energia em coisas pouco importantes. A temática da morte é
algo comum a todas as culturas, cada qual interpretando essa questão à
sua maneira. Por exemplo, no México, a morte é celebrada no Dia dos Mortos, em que
as famílias erguem altares com as fotos de seus familiares falecidos, com oferendas.
É um dia de festa, com música e danças tradicionais.

A morte é um tema sensível e muitas vezes não discutido abertamente. Com essa atividade,
os estudantes podem ter um espaço para conversar sobre e se valer de 
perspectivas de outras culturas e religiões para ampliar a sua visão sobre o assunto.


\begin{enumerate}
\paragraph{Metodologia} 

\item Sugira aos alunos que pesquisem, na biblioteca da escola ou
com materiais digitais, sobre a forma como distintas culturas e religiões
lidam com a morte.

\item Peça que eles escolham uma culturas ou religiões estudadas e
escrevam um breve texto sobre, apontando diferenças que encontraram
em relação à como o seu entorno lida com o tema.

\item Em seguida, deve ser feito um debate sobre como o tema da morte aparece
em obras literárias realistas.

\item Os estudantes podem fazer uma tabela comparativa de como o tema
é tratado em outros estilos literários, como por exemplo dentro da estética romântica.

\end{enumerate}

\paragraph{Tempo estimado} Duas aulas de 50 minutos.


\subsection{Leitura}

%\BNCC{EM13LGG103}
%\BNCC{EM13LP02}
%\BNCC{EM13LP48}


\paragraph{Tema} A trajetória de vida de Ivan Ílitch.

\paragraph{Conteúdo} Confecção de uma linha do tempo com os eventos mais
marcantes da vida da personagem.

\paragraph{Objetivo} Trazer uma maior atenção dos estudantes à estrutura
narrativa da obra.

\paragraph{Justificativa} Essa atividade poderá ajudar na análise 
da obra e no exercício de seleção de informações relevantes.

\paragraph{Metodologia}

\begin{enumerate}

\item Os estudantes devem produzir uma linha do
tempo, onde estarão marcados os eventos da trajetória de Ivan Ílitch.

\item No formato de um \textbf{mapa conceitual}, os alunos podem 
registrar uma lista de \textbf{personagens}, organizando nomes e suas
relações com a personagem principal. Cada personagem pode ser
identificada por um marcador adesivo contendo o nome e o sobrenome.

\item É interessante estimular a visualização da história e pedir aos estudantes
que desenhem ou pesquisem ilustrações para as personagens.

\item Conforme a leitura for avançando, podem anotar e fixar, tanto os eventos na linha do
tempo, quanto, ao lado de cada uma das personagens no quadro, adjetivos
que ajudem a caracterizar comportamentos, trechos de falas ou de
passagens emblemáticas.

\end{enumerate}

\paragraph{Tempo estimado} Um bimestre.


\subsection{Pós-leitura}

%\BNCC{EM13LGG102}
%\BNCC{EM13LGG303}
%\BNCC{EM13LGG402}
%\BNCC{EM13LGG703}
%\BNCC{EM13LP13}
%\BNCC{EM13LP14}
%\BNCC{EM13LP28}
%\BNCC{EM13LP29}
%\BNCC{EM13LP52}



\paragraph{Tema} A valorização da vida e da memória através de obituários.

\paragraph{Conteúdo} Reflexão acerca dos valores 
mais importantes da vida e o conceito de ``bem viver''.
Escrita de obituário da personagem principal.

\paragraph{Objetivo} Incentivar os estudantes a reconhecer
valores importantes da vida e como homenagear aqueles que partiram.

\paragraph{Justificativa} Após a leitura do livro e a reflexão acerca do tratamento dado à morte,
é interessante que os estudantes possam trabalhar a retórica do gênero obituário com uma homenagem à vida de Ivan Ílitch.

\paragraph{Metodologia} 

\begin{enumerate}

\item Proponha aos alunos um debate acerca da concepção
de bem viver e quais os valores mais importantes da vida de cada um.

\item É interessante oferecer
aos alunos uma bibliografia complementar, que pode incluir obras da
poesia árcade, bem como textos de filosofia, como de Sêneca, Epicuro,
Cícero, etc.

\item Por meio das informações colhidas, do
registro de análises e impressões de leitura, e da pesquisa de dados
contextuais, proponha a escrita coletiva de um obituário de
Ivan Ílitch.

\item Com a turma dividida em grupos, transforme a sala na
seção de um periódico do século XIX responsável por divulgar mortes de
cidadãos. Cada equipe redigirá uma nota apresentando as circunstâncias
de morte, a descrição do velório e do enterro do protagonista,
depoimentos imaginários de personagens, elogios irônicos à carreira e
à personalidade do defunto. 

\item Estimule a pesquisa, em \emph{sites} de
internet, de obituários de pessoas famosas. Incentive a
recuperação de episódios centrais da vida de Ivan Ílitch que possam
enriquecer a homenagem. 

\item Após a revisão da primeira versão do texto
pelo(a) professor(a) responsável, os alunos utilizarão o editor de
textos do computador para redigir a versão final, que poderá ser
publicada no \emph{site} da escola ou em um blog reservado às
experiências de leitura da obra.

\item Também podem recuperar as informações trazidas na
primeira atividade, uma vez que, sem sombra de dúvidas, aspectos
culturais e religiosos que tratam da morte, também indicam um parecer
sobre a vida.

\end{enumerate}

\paragraph{Tempo estimado} Uma aula de 50 minutos.

\section{Atividades 2}

%\BNCC{EM13CNT201}
%\BNCC{EM13CNT303}
%\BNCC{EM13CHS101}
%\BNCC{EM13CHS102}
%\BNCC{EM13CHS106}
%\BNCC{EM13CHS401}


A obra \emph{A Morte de Ivan Ílitch} possibilita trabalhos
interdisciplinares e integradores de diferentes campos do saber e áreas
de conhecimento. A seguir, propomos algumas atividades que podem ser
desenvolvidas conjuntamente com professores de outras áreas.

\subsection{Pré-leitura}

\paragraph{Tema} A importância da medicina preventiva como
estratégia de combate às doenças.

\paragraph{Conteúdo} Debate acerca do papel do Estado no combate às doenças.

\paragraph{Objetivo} Conscientizar os estudantes da importância de cuidar
de sua saúde e valorizar o papel da saúde pública.

\paragraph{Justificativa} Como preparação para a narrativa, a presente atividade
sugere debater com os alunos a importância da medicina preventiva como
estratégia de combate às doenças.

Na história, o fato de Ivan Ílitch ter ignorado
a lesão que sofreu com a queda pode ter potencializado o problema que o
levaria a óbito. A própria redação aplicada por Tolstói ao trecho, por
si só, já denota a indiferença com que Ivan Ílitch lidou com a situação.
Isso é algo muito comum entre as pessoas. Sentirem um mal estar, uma dor
e terem a crença de que aquilo irá passar em breve. Muitas das vezes,
sim, passa, porém, há situações em que esses incômodos são indícios de
uma mazela ainda pior por vir.

\paragraph{Metodologia}

\begin{enumerate}

\item Com o auxílio dos professores de ciências da natureza, traga
um panorama aos estudantes sobre o tratamento preventivo dado às doenças
por parte do Poder Público e até onde é possível a atuação do Estado.

\item Deve ser feita uma pesquisa, por parte dos estudantes,
acerca das diferenças entre o sistema de saúde no Brasil e em outros países.
Por ser usado como exemplo a questão da privatização do sistema de saúde nos Estados Unidos
e quais as consequências desse modelo.

\item Peça que os alunos apontem quais medidas podem ser feitas cotidianamente
para cuidar de sua saúde e quais são as estratégias que melhoram a qualidade de vida, 
prolongando o bem estar.

\end{enumerate}

\paragraph{Tempo estimado} Duas aulas de 50 minutos.


\subsection{Leitura}

\paragraph{Tema} Os limites e deveres da administração pública.

\paragraph{Conteúdo} Pesquisa e reflexão acerca do funcionamento dos órgãos do Estado,
a partir da obra \emph{A Morte de Ivan Ílitch} e com o auxílio dos professores
de humanidades. Apresentação de seminários em grupo acerca do conteúdo pesquisado.

\paragraph{Objetivo} Oferecer ao aluno subsídios para melhor compreensão do
funcionamento dos órgãos do Estado e de seus direitos e deveres enquanto cidadão.

\paragraph{Justificativa} Na obra \emph{A Morte de Ivan Ílitch}, 
há uma crítica por parte do autor acerca do Czarismo e da burocracia estatal russa.
A partir desse debate, é possível pensar na evolução do Estado Democrático
de Direito ao longo do tempo e visualizar os limites da atual administração
pública brasileira.

\paragraph{Metodologia}

\begin{enumerate}

\item Os professores de ciências humanas devem propor um debate sobre
a crítica de Tolstói ao excesso de burocracia estatal na Rússia Imperial.

\item Em seguida, deve ser feita uma explanação aos alunos sobre a
evolução do Estado Democrático de Direito. Devem ser abordadas questões
como a legislação tangente às funções da administração
pública e a divisão dos Três Poderes.

\item Recomenda-se que a sala seja dividida em grupos e que seja
feita uma pesquisa sobre como são preenchidos os cargos públicos
e quais os limites e deveres da administração pública.
A partir de textos constitucionais, os alunos devem investigar quais
são os seus direitos e deveres enquanto cidadãos.

\item Feito isso, os grupos podem apresentar seminários curtos expondo o
conteúdo pesquisado.

\end{enumerate}

\paragraph{Tempo estimado} Quatro aulas de 50 minutos.


\subsection{Pós-leitura}

\paragraph{Tema} Os percalços da saúde pública no Brasil.

\paragraph{Conteúdo} Debate acerca do funcionamento da saúde pública
e redação de texto sobre possíveis iniciativas para a melhoria dos atendimentos.

\paragraph{Objetivo} Incentivar os estudantes a reconhecer a efetividade
dos serviços de saúde pública em seu entorno e sobre o direito
constitucional à saúde.

\paragraph{Justificativa} Após a reflexão acerca do funcionamento do Estado Democrático de Direito
e do tratamento dado às doenças por parte do Poder Públicos, os estudantes 
têm em mãos ferramentas para redigir um texto argumentando quais são os 
possíveis mecanismos para melhorar o sistema de saúde.

\paragraph{Metodologia}

\begin{enumerate} 

\item Com o auxílio de um professor da área de ciências 
humanas e um da área de ciências da natureza, deve ser feito um debate
acerca do funcionamento da saúde pública no Brasil atualmente.

\item Podem ser propostas diversas questões para fomentar a discussão, como a
questão da universalidade de atendimentos; a capacidade financeira do
Estado em prestar o serviço, o direito constitucional à saúde, o dever
moral do Estado em garantir o bem-estar de seus administrados, as
vantagens e desvantagens que parcerias público-privadas podem gerar no
atendimento. Até que ponto é benéfico o papel de convênios, entre
outros.

\item A atividade também sugere que seja ponderada a burocracia
empregada no atendimento às pessoas adoentadas e às que vão
fazer exames.

\item Proponha que os estudantes façam uma dissertação
avaliando a real efetividade e qualidade dos serviços públicos prestados
na comunidade local e quais podem ser as iniciativas para melhorar o sistema de saúde.

\end{enumerate}

\paragraph{Tempo estimado} Duas aulas de 50 minutos.


\section{Aprofundamento}


\begin{comment}
Ao chegar ao Ensino Médio, é necessário que os estudantes se aprofundem
na compreensão das múltiplas linguagens e, sobretudo, da linguagem
literária. Em relação à literatura, a BNCC traz as seguintes
considerações:


\begin{quote}
{[}...{]} a leitura do texto literário, que ocupa o centro do trabalho
no Ensino Fundamental, deve permanecer nuclear também no Ensino Médio.
Por força de certa simplificação didática, as biografias de autores, as
características de épocas, os resumos e outros gêneros artísticos
substitutivos, como o cinema e as HQs, têm relegado o texto literário a
um plano secundário do ensino. Assim, é importante não só (re)colocá-lo
como ponto de partida para o trabalho com a literatura, como
intensificar seu convívio com os estudantes. Como linguagem
artisticamente organizada, a literatura enriquece nossa percepção e
nossa visão de mundo. Mediante arranjos especiais das palavras, ela cria
um universo que nos permite aumentar nossa capacidade de ver e sentir.
Nesse sentido, a literatura possibilita uma ampliação da nossa visão do
mundo, ajuda-nos não só a ver mais, mas a colocar em questão muito do
que estamos vendo/vivenciando. (Brasil, 2018, p. 491)
\end{quote}
\end{comment}

Nesta seção, desenvolvemos um trabalho de aprofundamento que, em diálogo
com a formação continuada de professores, oferece subsídios para a
abordagem do texto literário. A leitura em sala de aula de \emph{A Morte
de Ivan Ílitch} pode ser enriquecida pelo aprofundamento no universo
literário de Liev Tolstói e da literatura russa como um todo.


\Image{Tolstói em maio de 1908, quatro meses antes de seu aniversário de 80 anos (fotografado em Iasnaia Poliana por Sergei Mikhailovitch Prokudin-Gorski; a primeira foto colorida tirada oficialmente na Rússia) (Serguéi Mijáilovich Prokudin-Gorskii; Domínio Público)}{PNLD0042-03.png}


\subsection{A obra}

Publicado em 1886, o conto A Morte de Ivan Ílitch faz parte de uma série
de textos breves que um já consagrado Tolstói produzira por volta dos
seus cinquenta anos de vida. Em uma narrativa leve, mas nem por isso
menos profunda, Tolstói conta a história da vida e da morte do senhor
Ivan Ílitch Golovín, promovendo para o leitor uma reflexão acerca da
vida e da morte, enquanto narra a trajetória da personagem
principal, desde sua atividade junto ao judiciário, por meio da qual chega
num ponto alto de sucesso, até o seu adoecimento.


\Image{Folha de rosto da edição de 1895, no original russo. (Anakay; Domínio Público)}{PNLD0042-06.png}


A crítica sutil, salpicada com amaciada ironia, expõe as futilidades e
pequenezas de um universo de classe média na Rússia Imperial.
Curiosamente, alguns apontamentos do autor, ainda que claramente
indiquem ao que se remetem, são transcendentais, ultrapassando os limites
físicos e temporais daquela sociedade. Não obstante, os lampejos cada
vez mais humanizados que se presentificam nos pensamentos da personagem
principal a aproximam de qualquer um de nós, na condição de humanos, ao
acompanharmos sua áspera jornada.

\subsection{Uma personagem de muitos convívios sociais, mas essencialmente
sozinha.}

Logo que o conto se inicia, com o anúncio da morte de Ivan Ílitch
Golovin, funcionário público do Judiciário imperial russo, temos já
uma clara impressão de como foi a vida desse distinto cidadão.

A notícia havia saído em um periódico local, sendo lida por seus colegas de
trabalho. De certo modo consternados, eles debatem acerca da morte do
homem, ao mesmo tempo que refletem sobre as possíveis realocações -- em
razão do óbito -- de nomes nos cargos da repartição, bem como se é de
fato necessário ter de suportar o fardo social do comparecimento ao
velório.

Um desses homens, Piotr Ivanovitch, que havia, inclusive, estado na
convivência de Ivan Ílitch desde os tempos da escola de jurisprudência,
se sente compelido a comparecer ao velório, muito em razão dos fardos
dos costumes sociais. Entretanto, isso não o impediu de comentar com a
esposa sobre possíveis benesses que sua família poderia receber pela
realocação dos funcionários nos cargos do tribunal.

Deseja ser breve no velório, afinal marcara um carteado com outros
companheiros, mas, ao chegar lá, é interpelado pela viúva que o chama
para conversar, delongando assim seu passatempo lúdico. Ela deseja
indagar-lhe sobre a possibilidade do recebimento de uma pensão mais
gorda. Ao ouvir o pedido, Piotr Ivánovitch polidamente recrimina o
Estado por sua mesquinhez, indicando não crer ser possível o recebimento
de um benefício mais robusto.

Curioso é que tal diálogo se segue às reclamações da viúva das agruras
que ela vivera nos últimos momentos do defunto, posto que Ivan gritara incessantemente por três dias. É realizada a cerimônia, e, após uma breve conversa com um funcionário
da casa, Piotr Ivánovitch vê que ainda é tempo de participar da
jogatina, de modo que se dirige à casa do amigo onde ela ocorria.

\subsection{A trajetória de Ivan Ílitch -- crítica ao funcionalismo público}

Após esse trecho introdutório, Tolstói passa a narrar a história de
Ivan Ílitch. Desde o início, Ivan Ílitch é pintado como um sujeito
mediano, nascido em uma família de posses medianas. Seu pai atuava como
funcionário público, daqueles que, na definição de Tolstói, estão lá há
muito tempo para serem mandados embora e que são designados a um cargo, de atribuições fictícias.

Aliás, toda vez que o Estado se presentifica na obra, fica claro como a crítica do autor ao Estado como excessivo, sedentário, acomodado. Tolstói escancara a
falência da intenção de Pedro I com sua tabela de patentes. Observa-se
funcionários preguiçosos e corruptos. As relações se dão
por interesse e o tráfico de influência parece ser o melhor de todos os
planos de carreira.


\Image{Retrato de Pedro I (1672-1725) (Arkhangelskoye Palace; Domínio Público)}{PNLD0042-07.png}


\Image{Família imperial russa em 1913 (Boasson and Eggler St. Petersburg Nevsky 24.; Domínio Público)}{PNLD0042-08.png}


Ora, as críticas ao funcionalismo público, às burocracias ineficazes e
aos profissionais de qualidade duvidosa são lugar comum na literatura
russa desde os tempos de Gógol, sendo a questão também amplamente
abordada em Dostoiévski. Na literatura nacional, Nelson Rodrigues
explorará também essa temática.

Com tal quadro em mente, e observando o contexto em que a personagem
principal está inserida, vê-se que esta, sem grandes brilhos, cresce, estuda, e tal qual o pai ingressa no funcionalismo
público, observando rigorosamente uma cartilha social, desejando corresponder ao básico exigido. Em conversas banais, cumpria o protocolo
apresentando um descontentamento leve em relação ao governo, adotando um
tom de liberalismo moderado, adequando-se então aos modismos da época,
mas sem correr o risco de se indispor com o fornecedor de seu sustento.

\subsection{A vida familiar e social -- de um incômodo necessário a um fardo
insuportável}

Ivan Ílitch se casou porque era isso que se esperava de um homem
classificado como decente, mas, ao longo da história, é possível observar que seu
matrimônio como uma fonte de aborrecimento.

Apesar de um início aquecido, o casamento logo esfria, e o
relacionamento amoroso deu lugar a conflitos cotidianos por motivos
pequenos, o que fazia com que Ivan Ílitch evitasse a companhia dos
familiares, buscando sempre se refugiar no trabalho.

Nota-se a ausência de empatia e desejo pelo convívio familiar. Todos
ao seu redor, em verdade, parecem estranhos que, por uma determinação
social, é obrigado a conviver. Não existe envolvimento afetivo, a
família é como se fosse, perante à sociedade, o mesmo que a toga é para
o cargo de juiz, um uniforme, uma formalidade, um elemento tradicional
que confere distinção a seu possuidor. Trata-se do completo esvaziamento do significado das relações
familiares.

No que diz respeito à uma aparente dedicação ao trabalho, seu
verdadeiro prazer era durante os jogos de baralho, onde sentia que sobressaíam seus talentos. Em certo momento, um remanejamento das cartas no funcionalismo público em sua
província ameaçou seu conforto e possibilidade de promoção.

Frustrado, decide agir. Viaja à capital do Império, buscando realocação
e, por sorte, encontra no trem conhecidos dispostos a lhe auxiliar. Com
isso, consegue a transferência, para um cargo distante daqueles que lhe
opuseram, com um ordenado de cinco mil rublos e mais três mil e
quinhentos de abono pela mudança. O sucesso do novo posto faz com que até parentes, há muito sem interesse
em um contato, o procurem.

\subsection{Altos e baixos na vida de Ivan Ílitch}

Satisfeito e empolgado, muda-se adiante de sua família, com o intuito de
preparar um bom lar, que impressionasse a todos. Contudo, nos preparativos finais, Ivan Ílitch sofre uma queda, com a
qual acaba recebendo uma pancada no flanco esquerdo. Considerando esse
acidente um infortúnio sem importância, Ivan Ílitch segue sua vida,
recebe a esposa e filhos com felicidade e até brinca, imitando sua
queda. Assume o posto, ganhando bem, um ordenado do qual faltam apenas
quinhentos rublos para uma vida satisfatória.

Porém, a euforia advinda
do sucesso profissional e da mudança rapidamente acaba. Ele passa a
sentir uma dor constante, que o molestava na hora da alimentação, além
de ter a impressão de possuir um cheiro repugnante saindo de sua boca. A
harmonia conjugal também se deteriora, retornando ao cotidiano anterior.

O incômodo constante leva Ivan Ílitch a uma constante irritabilidade.
Ora, por que isso acontecia com ele? Por que a dor não cessava? A
perpetuação do incômodo amplificava seu mal estar físico, uma vez que
contaminava seu psicológico a todo instante.

Busca médicos renomados, famosos, alternativos. Cada qual parece trazer
um diagnóstico, uma cura milagrosa que ocorreria em breve. Ivan Ílitch
vê nos médicos a arrogância da academia, o olhar de
desprezo com seus pacientes, sem o menor interesse em ouvir suas
queixas, em uma postura similar com a qual ele tivera como juiz na
condução dos procedimentos jurídicos e na oitiva dos comuns com quem
cruzara.

O desespero vai lhe consumindo a consciência. Lembra-se dos silogismos
da época de estudo, em que aprendera que se todo homem é mortal, e Caio
é homem, logo Caio é mortal. Esse pensamento o entristece. Pouco importa
que Caio é mortal, o que importa é que ele, Ivan Ílitch, também o é.

Não desejando a morte, mas tendo que suportar o incômodo incessante,
Ivan Ílitch passa a utilizar ópio para se aliviar. Enquanto padece,
reflete sobre sua história. E nos caminhos sinuosos de seu pensamento
percebe que tem poucas lembranças de verdadeira felicidade.

Irrita-se ao ver sua família seguindo a vida normalmente, vendo que
tornou-se um fardo, e suspeita que talvez sempre o tenha sido. Observa
sua mulher interesseira, vê sua filha em um relacionamento com outro
carreirista e seu filho como estudante trilhando passos similares aos
seus.

Chegou a odiar a mulher, atribuindo-lhe a razão de seus malefícios. O
que era comum, sempre atribuía a terceiros as razões de seus
infortúnios. Ela também desejara sua morte, porém se assustou com seu
desejo, afinal, a morte do marido significaria o fim do ordenado.

\subsection{A redenção em um amigo inesperado}

No final da vida, tem um pouco de alívio ao ser cuidado pelo criado
Guerássim. Um mujique, jovem, de vistosa compleição. Ao contrário dos
médicos, e todos do convívio de Ivan Ílitch, tinha ânimo e boa vontade
para auxiliar um moribundo. Levantava-lhe as pernas, única posição em
que Ivan Ílitch sentia um pouco de conforto, ficando horas assim,
para o alívio de seu patrão, sem se importar ou emitir qualquer queixa.

Tinha bom espírito para todas as suas atitudes, opondo-se à perpétua
tonalidade \emph{blasé} vivida Ivan Ílitch em praticamente toda a sua
vida. Também fora o único que não cultivou falsas esperanças no patrão
moribundo. Sabia que ele iria morrer e lidava naturalmente com isso, não
vendo perda de tempo em seu cuidado, apesar de que a causa poderia ser
vista como perdida. Fazer bem ao próximo lhe bastava.

Nesse momento, já em seus instantes finais, Ivan Ílitch percebe o qual
insossa foi sua vida. Lamenta o tempo perdido com
frivolidades. Percebe que tudo em
sua vida se resumiu, com exceção de algumas recordações de sua juventude,
a uma rotina afetada.

Similarmente a Memórias Póstumas de Brás Cubas, temos em A Morte de Ivan
Ílitch uma história que começa a partir da morte da personagem principal
e, no desenrolar do livro, conhecemos com mais detalhes sua biografia. A
diferença resta no fato de Brás Cubas saber que está morto, e sua narrativa
consiste em rememorações de sua história já consciente de seu fim,
enquanto com Ivan Ílitch, embora saibamos de seu fim, percorremos sua
história com suas angústias e temores.

 Ao passar dessas linhas, somos apresentados à genialidade de Tolstói na condução da narrativa. No livro, saímos muito
sutilmente do universo cotidiano, mesquinho, ganancioso e pretensioso
das repartições, para irmos, paulatinamente, ao
mais íntimo psicológico de Ivan Ílitch.

Tolstói é tão delicado no conduzir de sua história que mal percebemos
que a causa dos malefícios do protagonista foi sua queda, tal como
quando sentimos uma dor inesperada e não nos lembramos de quando nos
ferimos em um momento anterior.

O livro começa retratando a mediocridade cotidiana, porém, a narrativa
que conduz o caminho longo até a morte vai se esvaindo do relato da
pequenez, até que a personagem principal atinge um algo mais sublime.

E tão delicadamente a personagem sai da sensação de penitência por ter
gastado prodigamente a vida, para um estado mais contemplativo e
benevolente frente à sua finitude.

Já em suas últimas horas, é visitado no cômodo que passara a habitar por
seu filho. Ao ver as olheiras do filho, Ivan Ílitch se compadece do
filho. É um momento de compaixão mútua, talvez um dos primeiros momentos
de sentimentos espontâneos narrados no oceano de afetação que até então
viviam.

Nutre compaixão também pela esposa, apesar de dias antes ter nutrido por
ela sentimentos tão negativos. Também se apieda da filha, de seu noivado com outro funcionário público,
tão medíocre quanto ele fora. Essa questão dominava seu pensamento sempre que aparecia.

A percepção de que nutria um bem
querer por sua família, um amor fundamental, permitiu a
paz a Ivan Ílitch, para que ele se permitisse ofuscar diante de uma
realidade maior que a dele próprio. É nessa hora que ele se liberta da dor,
do terror da morte e sutilmente, após dias de tormento, é levado em
seu último suspiro.

A plenitude atingida é o que marca seu último sopro. Ainda tentara dizer
perdão a mulher e aos filhos, porém as palavras não saíram. Mas isso
pouco importava, sua consciência estava agora harmônica.
Não à toa que os ``amigos'' de Ivan Ílitch que compareceram ao velório
notaram que seu cadáver guardava uma expressão mais significativa do que
a que tivera em vida.

Assim se encerra o conto. Somos confrontados com as reflexões de
um homem moribundo; observamos a penúria de um ser em confronto consigo
mesmo. Além disso, também somos convidados a olhar para dentro e pensarmos
sobre nós. Desse modo, Ivan Ílitch está em nós. Quando olhamos atônitos
a passagem do tempo e consternados nos debatemos a pensar se aquele
passo fora bem dado, se aquela resposta foi adequada, se há sentido na
escolha feita, se o propósito impresso foi realmente valoroso. E e com isso compreendemos a magnitude desse conto.

\subsection{Por que ler \emph{A morte de Ivan Ílitch?}}

Ainda que escrito no século XIX, A Morte de Ivan Ílitch é um livro
atual. Tanto em sua crítica ao funcionalismo público, com suas práticas
tão distantes de um ideal cívico, quanto por nos
ofertar uma reflexão acerca da vida, de sua efemeridade e fragilidade,
da certeza da morte e como com ela lidamos, dos valores empregados e das
opções feitas ao longo de nossa trajetória. A obra nos faz pensar sobre nossas
ambições e vaidades, sobre aquilo que almejamos, sobre o que nos traz
bem-estar.

Em outros termos, o conto demonstra que o homem, através dos tempos, não
é tão distinto quanto possa parecer a um primeiro olhar. Ao contrário, é
bem similar a nós. Em suas angústias mais íntimas, no imbricado de suas
relações familiares e sociais, nos anseios da vida profissional e no
temor à morte somos todos humanos. Assim, é possível nos vermos
refletidos na personagem, ainda que de outro tempo e outro mundo.

A grandeza da literatura russa está em ser capaz de descrever
visceralmente as situações mais cotidianas. No conto em questão, somos
colocados próximos de uma pessoa que nunca vimos e, ainda assim, nos identificamos e nos
compadecemos, amplificando aquilo que há de mais humano em todos nós.


\Image{O jovem Tolstói, em 1854 (Pavel Biryukov; Domínio Público)}{PNLD0042-04.png}




\section{Sugestões de atividades complementares: relações dialógicas e
intertextuais}

%\BNCC{EM13LP03}
%\BNCC{EM13LP04}
%\BNCC{EM13LP49}
%\BNCC{EM13LP51}

No Ensino Médio, da mesma forma que no Ensino Fundamental, a \textsc{bncc}
organiza o trabalho com as práticas de linguagem em cinco campos
de atuação social. São eles: campo da vida pessoal, campo da vida
pública, campo jornalístico"-midiático, campo artístico"-literário e campo
das práticas de estudo e pesquisa.

De acordo com essa divisão, propomos na sequência um trabalho
interdiscursivo e intertextual com a obra \emph{A Morte de Ivan Ílitch.}

\subsection{Campo da vida pessoal}

\begin{comment}\begin{quote}
O campo da vida pessoal pretende funcionar como espaço de articulações
e sínteses das aprendizagens de outros campos postas a serviço dos
projetos de vida dos estudantes. As práticas de linguagem privilegiadas
nesse campo relacionam"-se com a ampliação do saber sobre si, tendo em
vista as condições que cercam a vida contemporânea e as condições
juvenis no Brasil e no mundo.

Está em questão também possibilitar vivências significativas de práticas
colaborativas em situações de interação presenciais ou em ambientes
digitais e aprender, na articulação com outras áreas, campos e com os
projetos e escolhas pessoais dos jovens, procedimentos de levantamento,
tratamento e divulgação de dados e informações e o uso desses dados em
produções diversas e na proposição de ações e projetos de natureza
variada, para fomentar o protagonismo juvenil de forma
contextualizada. (\textsc{bncc}, p. 494)
\end{quote}
\end{comment}

\SideImage{Pintura de Tolstói vestido com roupas camponesas, por Ilya Repin (1901) (Ilya Repin; Domínio Público)}{PNLD0042-05.png}

Ivan Ílitch, protagonista de \emph{A Morte de Ivan Ílitch}, viveu uma
vida morna, chegando à conclusão disso apenas nos últimos dias de sua
vida, quando percebeu o quanto era infeliz, além de estranho e alheio
às pessoas que o cercavam. Essa situação não é tão incomum, afinal,
muitas pessoas até próximas desconhecem elementos muito íntimos de
nosso ser. O que é paradoxal, afinal, vivemos em um período de grande
exposição nas redes sociais. A situação torna-se tão \emph{sui
generis} que muitas vezes nós mesmos desconhecemos nossos potenciais.
Diante disso, sugere-se que o aluno monte um quadro pessoal, onde
apontará dez qualidades individuais, cinco no campo das relações
pessoais e cinco no universo escolar e do trabalho. Essa atividade não
precisa ser exposta aos demais, mas se conhecer é um bom ponto de
partida inclusive para o aprendizado.

\subsection{Campo de atuação na vida pública}

\begin{comment}
\begin{quote}
No cerne do campo de atuação na vida pública estão a ampliação da
participação em diferentes instâncias da vida pública, a defesa dos
direitos, o domínio básico de textos legais e a discussão e o debate de
ideias, propostas e projetos. {[}\ldots{}{]}

Ainda no domínio das ênfases, indica"-se um conjunto de habilidades que
se relacionam com a análise, discussão, elaboração e desenvolvimento de
propostas de ação e de projetos culturais e de intervenção social.
(\textsc{bncc}, p. 494)
\end{quote}
\end{comment}

Em \emph{A Morte de Ivan Ílitch}, as alusões a um sistema
jurisdicional burocrático, ineficiente e desumanizado são
constantes. Infelizmente, a impessoalidade do serviço público, que é
um princípio, quando pensada para tratar igualmente os
administrados, acaba por se tornar também uma mazela, quando
desumaniza estes, desconsiderando assim, até mesmo, a sua própria
humanidade. Reformular o sistema como está posto não é uma tarefa tão simples, mas pode ser pensada. Sugira aos alunos que redijam um projeto de
lei visando tornar mais fácil o acesso do administrado com seus
administradores. Por exemplo, pode-se criar uma sala virtual de
participação em decisões da Câmara Legislativa, entre outras
intersecções de atividades presenciais e a distância que podem
aproximar esses dois pólos da relação.

\subsection{Campo jornalístico"-midiático}

\begin{comment}
\begin{quote}
Em relação ao campo jornalístico"-midiático, espera"-se que os jovens
que chegam ao Ensino Médio sejam capazes de: compreender os fatos e
circunstâncias principais relatados; perceber a impossibilidade de
neutralidade absoluta no relato de fatos; adotar procedimentos básicos
de checagem de veracidade de informação; identificar diferentes pontos
de vista diante de questões polêmicas de relevância social; avaliar
argumentos utilizados e posicionar"-se em relação a eles de forma ética;
identificar e denunciar discursos de ódio e que envolvam desrespeito aos
Direitos Humanos; e produzir textos jornalísticos variados, tendo em
vista seus contextos de produção e características dos gêneros. Eles
também devem ter condições de analisar estratégias
linguístico"-discursivas utilizadas pelos textos publicitários e de
refletir sobre necessidades e condições de consumo.

No Ensino Médio, os jovens precisam aprofundar a análise dos interesses
que movem o campo jornalístico midiático, da relação entre informação e
opinião, com destaque para o fenômeno da pós"-verdade, consolidar o
desenvolvimento de habilidades, apropriar"-se de mais procedimentos
envolvidos na curadoria de informações, ampliar o contato com projetos
editoriais independentes e tomar consciência de que uma mídia
independente e plural é condição indispensável para a democracia.

Como já destacado, as práticas que têm lugar nas redes sociais têm
tratamento ampliado. (\textsc{bncc}, p. 494-495)
\end{quote}
\end{comment}

A leitura de todo livro transporta o leitor para outro universo.
Assim, como um expectador contemporâneo, ele observa os eventos
narrados, passando a organizá-los em sua mente, de acordo com suas
memórias e conhecimentos pretéritos. O intelectual Gilberto Freyre uma
vez teria dito que ``O Brasil é a Rússia Americana''. Essa frase se
pautou no fato de ambos os países serem de dimensões continentais,
possuírem clima extremo e terem abolido seus sistemas de trabalho
compulsório tardiamente. Entretanto, pairam ainda elementos muito
parecidos entre os dois países, em suas respectivas realidades. Peça
ao aluno para fazer um quadro comparativo, a partir do livro lido,
entre as similitudes e diferenças de Brasil e Rússia.

\subsection{Campo artístico"-literário}

<<<<<<< HEAD
% \begin{quote}
% No campo artístico"-literário busca"-se a ampliação do contato e a
% análise mais fundamentada de manifestações culturais e artísticas em
% geral. Está em jogo a continuidade da formação do leitor literário e do
% desenvolvimento da fruição. A análise contextualizada de produções
% artísticas e dos textos literários, com destaque para os clássicos,
% intensifica"-se no Ensino Médio. Gêneros e formas diversas de produções
% vinculadas à apreciação de obras artísticas e produções culturais
% (resenhas, vlogs e podcasts literários, culturais etc.) ou a formas de
% apropriação do texto literário, de produções cinematográficas e teatrais
% e de outras manifestações artísticas (remidiações, paródias,
% estilizações, videominutos, fanfics etc.) continuam a ser considerados
% associados a habilidades técnicas e estéticas mais refinadas.

% A escrita literária, por sua vez, ainda que não seja o foco central do
% componente de Língua Portuguesa, também se mostra rica em possibilidades
% expressivas. (\textsc{bncc}, p. 495-496).
% \end{quote}

Retome com os estudantes o painel de sala, cuja organização foi
proposta na Atividade 2 e cuja montagem foi enriquecida nas sucessivas
etapas de leitura da obra. Por meio das informações colhidas, do
registro de análises e impressões de leitura, e da pesquisa de dados
contextuais, proponha a escrita coletiva de um obituário de
Ivan Ílitch. Com a turma dividida em grupos, transforme a sala na
seção de um periódico do século XIX responsável por divulgar mortes de
cidadãos. Cada equipe redigirá uma nota apresentando as circunstâncias
de morte, a descrição do velório e do enterro do protagonista,
depoimentos imaginários de personagens, elogios irônicos à carreira e
à personalidade do defunto. Estimule a pesquisa, em \emph{sites} de
internet, de obituários de pessoas famosas, com o objetivo de
trabalhar a retórica do gênero obituário, e incentive a
recuperação de episódios centrais da vida de Ivan Ílitch que possam
enriquecer a homenagem. Após a revisão da primeira versão do texto
pelo(a) professor(a) responsável, os alunos utilizarão o editor de
textos do computador para redigir a versão final, que poderá ser
publicada no \emph{site} da escola ou em um blog reservado às
experiências de leitura da obra.

\begin{comment}
\begin{quote}
No campo artístico"-literário busca"-se a ampliação do contato e a
análise mais fundamentada de manifestações culturais e artísticas em
geral. Está em jogo a continuidade da formação do leitor literário e do
desenvolvimento da fruição. A análise contextualizada de produções
artísticas e dos textos literários, com destaque para os clássicos,
intensifica"-se no Ensino Médio. Gêneros e formas diversas de produções
vinculadas à apreciação de obras artísticas e produções culturais
(resenhas, vlogs e podcasts literários, culturais etc.) ou a formas de
apropriação do texto literário, de produções cinematográficas e teatrais
e de outras manifestações artísticas (remidiações, paródias,
estilizações, videominutos, fanfics etc.) continuam a ser considerados
associados a habilidades técnicas e estéticas mais refinadas.

A escrita literária, por sua vez, ainda que não seja o foco central do
componente de Língua Portuguesa, também se mostra rica em possibilidades
expressivas. (\textsc{bncc}, p. 495-496).
\end{quote}
\end{comment}


\subsection{Campo das práticas de estudo e pesquisa}

\begin{comment}
\begin{quote}
O campo das práticas de estudo e pesquisa mantém destaque para os
gêneros e habilidades envolvidos na leitura/escuta e produção de textos
de diferentes áreas do conhecimento e para as habilidades e
procedimentos envolvidos no estudo. Ganham realce também as habilidades
relacionadas à análise, síntese, reflexão, problematização e pesquisa:
estabelecimento de recorte da questão ou problema; seleção de
informações; estabelecimento das condições de coleta de dados para a
realização de levantamentos; realização de pesquisas de diferentes
tipos; tratamento dos dados e informações; e formas de uso e
socialização dos resultados e análises.

Além de fazer uso competente da língua e das outras semioses, os
estudantes devem ter uma atitude investigativa e criativa em relação a
elas e compreender princípios e procedimentos metodológicos que orientam
a produção do conhecimento sobre a língua e as linguagens e a formulação
de regras. (\textsc{bncc}, p. 495-496)
\end{quote}
\end{comment}

Proponha uma pesquisa para ampliação do repertório cultural e para
compreensão do lugar ocupado pelas personagens na obra lida. Para uma
maior compreensão da sociedade da época de Tolstói, é interessante que
o aluno faça uma pesquisa em que possa compreender o sistema de
ranques de Pedro I, bem como era dos mujiques (camponeses) no período.
Como foram as transformações sócio-econômicas com o fim da servidão e
como as agitações políticas -- o próprio Ivan Ilitich se definia como
uma espécie de reformista moderado -- do período podem ser vistas como
um prenúncio do que estava por vir. Os alunos podem se dividir em
grupos e apresentar um artigo a respeito das questões acima indicadas,
fazendo uma análise crítica dos acontecimentos históricos que
precederam, envolveram e sucederam os eventos narrados em \emph{A
Morte de Ivan Ílitch.}


\SideImage{Retrato do século XVIII de um mujique, camponês russo. (Dmitrismirnov; Domínio Público)}{PNLD0042-09.png}


\SideImage{Grupo de camponesas russas em 1917 (Francis Brewster Reeves; CC0)}{PNLD0042-10.png}


\section{Referências complementares}

\subsection{Livros}

\begin{itemize}
\item\textsc{batuman}, Elif. \textit{Os possessos}. São Paulo: Leya, 2012.

O leitor acompanha uma trama policial envolvendo Tolstói, visita a
imperial São Petersburgo, passa pelo Uzbequistão e segue os rastros de
Pushkin no Cáucaso.

\item\textsc{parini}, Jay. \textit{A última estação}. Rio de Janeiro: Record, 2009.

O autor combina verdade e ficção para recriar o último ano da vida de
Tolstói, acompanhando-o até seus últimos momentos, que sucederam sua
dramática fuga de casa.

\item\textsc{salerno}, Silvana (adapt.). \textit{Guerra e Paz}. São Paulo: Companhia
das Letras, 2010.

O livro é uma adaptação que resume as mais de mil páginas do clássico de
Tolstói, que trata a invasão da Rússia por Napoleão Bonaparte no começo
do século XIX.

\item\textsc{tolstói}, Liev. De quanta terra precisa o homem? São Paulo:
Companhia das Letrinhas, 2009.

O grande escritor russo recria neste conto a dramática história de um
homem obcecado pelo desejo de conseguir mais e mais terras.
\end{itemize}

\subsection{Filme}

\begin{itemize}
\item\textbf{Anna Karenina. Direção: Joe Wright (Reino Unido, 2013).}

Neste filme conceitual, Moscou é um grande palco onde é recriado o drama
de uma das mais poderosas personagens de Tolstói, Anna Karenina, que tem
seu pedido de divórcio negado e é privada do direito de ver seus filhos.

\item\textbf{A última estação. Direção: Michael Hoffman (Reino Unido, Rússia,
Alemanha, 2009).}

Filme dramatiza os últimos momentos da vida de Liev Tolstói, na estação
ferroviária de Astapovo. Também aborda a disputa entre a esposa de
Tolstói, Sophia, e seu discípulo Vladimir Chertkov pelos direitos de sua
obra.
\end{itemize}

\section{Bibliografia comentada}

\begin{itemize}
\item\textsc{bartlett}, Rosamund. \textit{Tolstói: a biografia}. São Paulo: Biblioteca Azul, 2013.

A autora se debruça sobre a vida e a obra de uma das figuras mais
emblemáticas da Rússia, conhecido não só por seus grandes escritos, mas
também pelo misticismo do homem que passou por uma profunda
transformação ao longo da vida.

\item\textsc{berlin}, I. \textit{Pensadores Russos}. São Paulo: Companhia das Letras, 1989.

Com aguçada capacidade crítica, o autor reúne nessa obra de fôlego
diversos ensaios sobre a Rússia do século XIX, do tempo dos czares.

\item\textsc{billington}, J. H. \textit{The icon and the axe. An interpretative
history of Russian culture}. New York: Vintage Books, 1970.

Livro traz uma descrição abrangente da história cultural russa, indo da
era pré-Romanov até os tempos comunistas de Joseph Stalin.

\item\textsc{bushkovitch}, Paul. \textit{História Concisa da Rússia}. São Paulo:
Edipro, 2015.

Este livro de leitura acessível, traça um amplo panorama da história
russa desde o século IX, passando pelos principais acontecimentos das
artes e das ciências.

\item\textsc{figes}, Orlando. \textit{Rússia: uma história cultural}. Rio de Janeiro: Record, 2017.

O renomado historiador analisa os grandes artistas e as mais importantes
manifestações culturais russas, revelando o espírito dessa nação.

\item\textsc{segrillo}, Angelo. \textit{Os russos}. São Paulo: Contexto, 2012.

Especialista em história e literatura, o autor fala com propriedade
sobre esse povo complexo e com tantas contribuições importantes para a
humanidade, na pintura, na música, na política e outras áreas.

\item\textsc{tolstói}, Liev. \textit{Os últimos dias}. São Paulo: Penguin, 2011.

O livro reúne ensaios, cartas, parábolas e fragmentos de obras de
Tolstói, escritos a partir de 1882. Traduzido diretamente do russo.

\item\textsc{volkov}, S. \textit{São Petersburgo: uma história cultural}. Rio de
Janeiro: Record, 1997.

Volkov se aprofunda na história e simbologia da cidade imperial, que,
segundo ele, ganhou força mítica a partir do poema O Cavaleiro de
Bronze, de Puchkin.
\end{itemize}

\end{document}

