\documentclass[11pt]{extarticle}
\usepackage{manualdoprofessor}
\usepackage{fichatecnica}
\usepackage{lipsum,media9,graficos}
\usepackage[justification=raggedright]{caption}
\usepackage[one]{bncc}
\usepackage[nmenosum]{../edlab}

\begin{document}


\newcommand{\AutorLivro}{Igor Mendes}
\newcommand{\TituloLivro}{Pequena prisão}
\newcommand{\Tema}{Protagonismo juvenil}
\newcommand{\Genero}{Diário; biografia; autobiografia; relatos; memórias}
\newcommand{\imagemCapa}{./images/PNLD0062-MP-C.png}
\newcommand{\issnppub}{---}
\newcommand{\issnepub}{---}
% \newcommand{\fichacatalografica}{PNLD0061-00.png}
\newcommand{\colaborador}{{Rebeca Martins de Souza}}


\title{\TituloLivro}
\author{\AutorLivro}
\def\authornotes{\colaborador}

\date{}
\maketitle


\begin{abstract}\addcontentsline{toc}{section}{Carta ao professor}

O presente manual busca subsidiar o estudo aprofundado da obra \emph{A
pequena prisão}, de Igor Mendes, nas suas linhas, entrelinhas e, por
isso mesmo, em seu contexto mais amplo. Com isto, buscamos mais do que a
compreensão e valorização da língua portuguesa, através da literatura,
mas a sua utilização como instrumento para interpretar o mundo que nos
cerca, e que não respeita fronteiras disciplinares rígidas, senão que as
integra e ressignifica.

Em \emph{A pequena prisão}, livro de estreia de Igor Mendes, o
autor"-personagem retrata talvez a forma mais brutal de intervenção do
espaço sobre a personalidade humana: o cárcere, exemplo acabado de
instituição total, cujos efeitos indeléveis sobre a alma têm fascinado e
horrorizado a atenção do público desde sempre. Prisioneiro político em
plena vigência da Nova República -- quando, supunha"-se, era pleno o
direito à livre expressão e manifestação -, Igor, antes de ser escritor,
era uma das vozes anônimas dentre as multidões que tomaram as ruas de
todo o país durante as históricas manifestações de junho de 2013. No ano
seguinte, no contexto da realização da Copa do Mundo sediada no Brasil,
ele e mais 22 pessoas (na sua maioria, jovens) foram processados e
presos, acusados de participar de ``protestos violentos''. Além da
perseguição judicial, os 23 foram também proibidos de frequentar
quaisquer manifestações políticas, num cerceamento explícito das
liberdades individuais sem precedentes até então. Acusado de violar esta
imposição arbitrária, Igor Mendes teve sua prisão preventiva decretada e
permaneceu encarcerado em diferentes presídios por sete meses, de 3 de
dezembro de 2014 a 22 de junho de 2015.

O livro conta esta história, na primeira pessoa do singular. Apesar
disso (ou por isso mesmo), nos apresenta um painel de diálogos e de
vozes, permeado por conflitos e solidariedade, nas profundezas da
consciência ou no contato com outros desafortunados. A linguagem crua,
não raro próxima da informalidade, busca situar os homens no seu meio e
no seu tempo, sem adornos ou julgamentos morais. Igor, estudante
universitário à época, utiliza"-se do choque que aquela realidade tão
diferente da sua produz para nos contar trajetórias desprovidas de
idealizações. Talvez por isso encontre mais pontos de contato entre
aquela pequena sociedade carcerária e a grande sociedade ``livre'' do
que nós, aqui fora, gostaríamos de imaginar.


Para interpretar esta obra de literatura brasileira contemporânea,
buscamos mobilizar uma série de conhecimentos, sociológicos,
antropológicos, históricos e geográficos, que transcendem o texto e nos
pintam um verdadeiro painel deste complicado Brasil de princípios do
século \textsc{xxi} -- sobretudo, das camadas mais marginalizadas da nossa
juventude. Guiou"-nos a busca do ``aprimoramento do educando como pessoa
humana, incluindo a formação ética e o desenvolvimento da autonomia
intelectual e do pensamento crítico'', como preconiza o Artigo nº 35 da
LDB 9394/96. Esperamos ser bem"-sucedidos neste intento, de modo a
suscitar entre os nossos jovens o gosto pela leitura e por valores
humanos tão caros como o respeito aos diferentes, o apreço pela justiça,
a solidariedade e a efetivação de uma liberdade que seja capaz de
conjugar os interesses coletivos e individuais.

\end{abstract}

\tableofcontents

\Image{O autor Igor Mendes (Bruna Freire; Acervo pessoal do autor)}{PNLD0062-03.jpg}

\section{Propostas de Atividades I}

\subsection{Pré"-leitura}

\paragraph{Tema} Esquecer ou lembrar das nossas histórias de dor? -- O
poder da literatura memorialística e de testemunho na ampliação da nossa
visão de mundo e concepção de história.

\paragraph{Conteúdo} Apresentação dos gêneros literatura de memória e
literatura de testemunho como discurso esteticamente elaborado sobre a
experiência vivida, resgatada ou testemunhada, a partir da exibição do
filme \emph{Doze anos de escravidão}, dirigido por Steve McQueen e vencedor
de várias premiações cinematográficas, seguido de debate.

\paragraph{Objetivo} Introduzir o público às noções de literatura de
memória e de testemunho através da linguagem cinematográfica, procurando
evidenciar suas especificidades formais, estéticas e éticas, seu impacto
histórico e sua contribuição para uma reinterpretação sobre o mundo e
reposicionamento dos sujeitos envolvidos e afetados pela narrativa.

\paragraph{Justificativa} Para introduzir o leitor neste tipo de
literatura, memorialística e de testemunho, de modo a enriquecer a sua
capacidade de compreensão e mesmo de produção sobre os gêneros
destacados, é fundamental que se chame a atenção para a sua relevância
em termos subjetivos e históricos, tanto quanto se debata os seus
aspectos específicos e formais. Ao longo da trajetória de vida do
estudante, é muito possível que ele já tenha tomado contato com esta
literatura informalmente, através de recursos musicais e/ou poéticos e
mesmo a partir de relatos orais. Cabe ao professor, neste momento,
desnaturalizar a sua reprodução, indo a fundo na discussão sobre o
sentido e forma destas formulações, no que ela importa ao narrador e de
que maneira ela pode afetar seus leitores. Neste sentido, a mediação
desta discussão através da exibição do filme \emph{Doze anos de escravidão},
que busca ser fidedigno ao relato das
memórias de Solomon Northup publicadas em 1853, pode ser profundamente
estimulante, seja pela forma quanto pelo conteúdo desta obra. A intenção
fundamental, neste momento, é criar condições de aproximação desta
experiência cinematográfica com elementos que serão analisados pelo
leitor durante a obra \emph{A pequena prisão}.

\paragraph{Metodologia} Esta aula divide"-se em quatro momentos: a
apresentação da proposta de leitura e dos gêneros literários, seguida da
exibição do filme, o debate sobre o filme e particularidades dos
gêneros em questão, e a sistematização dos conhecimentos formulados.

\begin{enumerate}
\item Para a primeira parte da aula, a apresentação da proposta de leitura e
dos gêneros literários, o/a educador/a pode proceder à leitura coletiva
de um trecho do livro \emph{A pequena prisão}, de Igor Mendes,
organizando simultaneamente um mapa mental (diagrama feito para
organizar informações) junto aos estudantes, em papel pardo para a
exposição permanente na sala de aula, ou quadro. Aqui, destacamos que o
trecho utilizado pode ser o da preferência do/da professor/a, mas
indicamos a leitura integral pós"-prefácio intitulada ``Advertência''.
\BNCC{EM13LP02}

Durante a leitura coletiva, em voz alta, indicamos que a cada parágrafo
seja alimentado este grande mapa mental coletivo. 

\begin{itemize}
\item No primeiro parágrafo,
por exemplo, o/a professor/a pode contextualizar as jornadas de junho de
2013 tanto quanto a ideia de criminalização de movimentos sociais, a
importância do protagonismo juvenil, e a relação deste contexto com a
obra e o autor. 

\item No segundo parágrafo, é importante o aprofundamento da
afirmação de Igor sobre seu texto tratar"-se de um ``depoimento engajado,
assumidamente parcial'', contrapondo a noção de imparcialidade defendida
em outros discursos, especialmente midiáticos. 

\item No terceiro parágrafo,
destacamos a relevância do início da ambientação de seu texto e
especialmente a necessidade de problematização da frase: ``Afinal, se
nos querem calar, não é ainda mais necessário que falemos?''. Levantar
questões relativas ao sujeito da frase, seu sentido e interpretações
pode garantir um debate que sirva de base ao aprendizado dos gêneros em
destaque. 

\item No quarto e quinto parágrafos, é importante ressaltar alguns elementos
próprios da literatura de testemunho, além do debate sobre denotação e
conotação dos termos. A afirmação ``E também porque tenho pressa'' pode
ser debatida entre os leitores, garantido o mistério sobre o que o
motiva a esta escrita. Retomar essa questão ao final da leitura integral
da obra pode ser interessante. 

\item No sexto parágrafo, discutir a relação
entre a pequena prisão e a ideia de liberdade. 

\item No sétimo, é importante
discutir a noção de grande prisão, bem como o ditado ``a cadeia é longa,
mas não é perpétua'', ressaltando"-se a perspectiva de \emph{mudanças
sociais} que o autor levanta. 

\item No oitavo e nono parágrafos, é importante
destacar as principais características de uma literatura memorialística
em perspectiva comparada à literatura de testemunho, ressaltando a
relevância histórica de conhecermos pontos de vista que geralmente fogem
à historiografia oficial ou reconhecida. 
\end{itemize}

% Tempo estimado para esta parte:
% uma aula de 50 minutos.

\Image{Advertência (Isabel Teixeira (Ateliê Fora de Esquadro); Direitos cedidos à n"-1)}{PNLD0062-06.png}

\item Finalizada a leitura indicada acima, propomos a exibição do filme \emph{Doze
anos de escravidão}, sendo contextualizada a
história da narrativa e o objetivo do/a professor/a para tal: apreender
as especificidades das literaturas de memória e de testemunho, bem como
compreender os efeitos historico"-sociais comuns a estas narrativas. 
\BNCC{EM13LP13}

Como
fonte de pesquisa sobre o filme, indicamos dois artigos que podem servir
ao planejamento do/a professor/a: 

\begin{itemize}
\item \emph{<<Doze anos de escravidão>> e o problema da representação das atrocidades humanas}, por Ana Lúcia Araújo, disponível no site do \href{https://www.scielo.br/scielo.php?script=sci_arttext&pid=S0002-05912014000200257}{\emph{Scielo}};

\item \emph{A luta de Solomon Northup pela liberdade}, por Maria Carolina Ramos, disponível no \href{https://canalcienciascriminais.jusbrasil.com.br/artigos/622104088/a-luta-de-solomon-northup-pela-liberdade?ref=amp}{Canal Ciências
Criminais}.
\end{itemize}

\item Para a terceira e quarta partes desta aula, a saber, a construção do
debate sobre o filme e particularidades dos gêneros em questão, além da
sistematização do conhecimento construído coletivamente, o/a educador/a
pode reposicionar a sala de aula para uma conformação em roda,
encaminhando uma pergunta motivadora: ``Seria melhor esquecer ou lembrar
das nossas histórias de dor, seja individualmente ou como sociedade?'' e
ainda ``Por que às vezes esquecemos também das nossas próprias vitórias?''
\BNCC{EM13LGG303}

A partir destas questões, é possível refletir sobre aspectos importantes
relativos a esta narrativa específica, tanto quanto é possível
relacioná"-la à leitura do livro \emph{A pequena prisão.} É possível
também compreender aquilo que caracteriza as narrativas memorialísticas
e de testemunho, como a comum narrativa em primeira pessoa, o retorno e
registro do passado, os recursos estilísticos empregados para aproximar
o público da obra, o discurso engajado que pode finalmente culminar em
salto, rumo à transformação real de situações de opressão e exploração
sociais vigentes. 

Destacamos, neste contexto, a relevância da
sistematização das principais informações estruturais e estilísticas
destes tipos de literatura preferencialmente no próprio mapa mental
construído pela turma.
%Tempo estimado de duração: uma aula de 50 minutos.
\end{enumerate}

\paragraph{Tempo estimado} Duas aulas de 50 minutos,
aproximadamente.

\Image{Ilustração a partir de foto de Bruna Freire (Isabel Teixeira (Ateliê Fora de Esquadro); Direitos cedidos à n"-1)}{PNLD0062-07.png}

\subsection{Leitura}

\paragraph{Tema} As muitas formas de falar --- o diálogo entre a memória,
o testemunho, a arte e o espaço da sala de aula.

\paragraph{Conteúdo} Aprofundamento dos elementos estruturais e
estilísticos das literaturas de memória e de testemunho, seguido da
proposição de trabalho de livre criação artística com os estudantes, e
exposição de suas obras no espaço da sala de aula.

\paragraph{Objetivo} Realizar um levantamento sobre os principais pontos
de interesse dos estudantes acerca da obra, dar seguimento ao estudo
sobre os gêneros literários memória e testemunho nas suas múltiplas
formas de expressão, e orientar a produção artística dos jovens (através
do desenho, da pintura, da fotografia, da charge, e mesmo da música e
poesia), que ilustre a sua compreensão sobre a leitura, crie vínculos
estéticos com esta e desenvolva clima de imersão literária.

\paragraph{Justificativa} Durante o período de leitura do livro, é
importante que o/a professor/a de língua não somente desenvolva
artifícios junto aos estudantes de compreensão sobre as especificidades
destes gêneros literários tal qual eles se apresentam em \emph{A pequena
prisão}, como também explore outras formas da memória e do testemunho se
expressarem. Neste movimento, é importante, ainda, mapear o conjunto dos
interesses dos leitores sobre a obra e mobilizar o seu engajamento
coletivo, sua capacidade de análise crítica e de produção acerca da
narrativa. Desta forma, garantir que o tempo da aula seja utilizado para
a livre expressão artística protagonizada pelos estudantes visa ampliar
a relação dialógica com texto, suas reflexões e seus personagens, bem
como aprofundar os aspectos estéticos e éticos presentes na literatura.

\paragraph{Metodologia} 

\begin{enumerate}
\item Para um primeiro momento, é fundamental que o/a
professor/a retome o conjunto das discussões travadas na primeira aula,
com o auxílio do mapa mental produzido coletivamente pelos estudantes;
que se avalie coletivamente o impacto destas primeiras reflexões na
compreensão da leitura do livro, e quais os elementos de interesse dos
leitores no mesmo. Este mapeamento inicial é fundamental para que você
possa fazer do presente manual um instrumento dinâmico, e, da literatura,
fonte viva de inspiração. 

\item Em seguida, visando aprofundar ainda mais os
gêneros literários memória e testemunho, é importante que o/a educador/a
siga trazendo elementos característicos deste tipo de obra, e suas
múltiplas formas de expressão e pontos de contato. Como sugestão,
indicamos que se aborde especialmente poemas e músicas que signifiquem a
memória e que possam suscitar ou não algum nível de engajamento do
narrador"-personagem a ponto de configurar pontos de aproximação com o
testemunho (possibilidade de se trabalhar com poemas de Alex Polari, em
\emph{Camarim de prisioneiro}, com músicas de Legião Urbana, a exemplo
de \emph{Faroeste caboclo}, e Racionais Mc's, \emph{Diário de um
detento}). 
\BNCC{EM13LP49}

Para a compreensão da aula, é importante que o/a professora
oriente a discussão a partir das seguintes questões
\begin{itemize}
\item ``Diante dos textos apresentados, o que você
acredita que tenha motivado o autor a escrever sobre este assunto?''

\item ``Quais são os argumentos e recursos estéticos/estilísticos utilizados
pelo autor para impactar o público leitor?''

\item ``Qual a relevância de sua
obra para a desnaturalização da sociedade em que vivemos?''
\end{itemize}

\item Por fim, a proposta de atividade a ser encaminhada junto aos jovens é a
de sua própria produção artística, seja por meio de desenho, pintura,
fotografia, esquete teatral, charge, e mesmo através da composição de
música e poesia, inspirando-se no seguinte comando: ``Ilustre, de maneira
criativa e autônoma, o que a pequena e/ou a grande prisão significam para
você''. 
\BNCC{EM13LP53}

Destacamos que esta atividade deve ser realizada individualmente e
ao menos iniciada/pré"-concebida no espaço da sala de aula. A exposição
das obras deve ser organizada pelos próprios estudantes, com a
orientação do/a educador/a, preferencialmente no espaço da sala de aula.
O mapa mental organizado coletivamente sobre a obra também pode ser
parte integrante da mesma. 

Todo este conjunto de produções pode
integrar, ainda, a organização de uma semana literária na escola. Um
projeto mais ambicioso, como este, depende sobretudo dos interesses e
possibilidades da própria comunidade escolar. Torcemos para que o
desenvolvam!
\BNCC{EM13LP47}
\end{enumerate}


\Image{Ilustração penitenciária (Isabel Teixeira (Ateliê Fora de Esquadro); Direitos cedidos à n"-1)}{PNLD0062-04.png}

\paragraph{Tempo estimado} Três aulas de 50 minutos, aproximadamente.


\subsection{Pós"-leitura}

\paragraph{Tema} Para compreender o gênero literário e seus recursos
narrativos também é necessário escrever -- A teoria e a prática das
literaturas memorialística e de testemunho.

\paragraph{Conteúdo} Sistematização de conhecimentos formais, estilísticos
e contextuais expressos na obra \emph{A pequena prisão}, e orientação
colaborativa para a produção de relatos memorialísticos pelos
estudantes.

\paragraph{Objetivo} Dada a conclusão da leitura do livro, o objetivo
desta aula é sistematizar, através de atividade em grupo, os principais
elementos formais, estilísticos e contextuais apresentados na obra em
questão, e orientar a produção de literatura de memória e de testemunho
pelos próprios estudantes. A realização de um balanço coletivo final da
atividade é recomendada.

\paragraph{Justificativa} Após a leitura do livro, é papel do/a educador/a
auxiliar os estudantes na compreensão dos artifícios estéticos e formais
utilizados pela literatura para afetar o público leitor, bem como seu
aspecto ético e conjunto de outras reflexões estimuladas pelo
narrador"-personagem. Isso representa um trabalho de conclusão sobre a
análise crítica da obra, interessando"-nos, ainda, que os estudantes
desenvolvam a capacidade argumentativa e de produção de literaturas de
memória e de testemunho com base no conjunto de recursos narrativos que
aprenderam e sobre os quais refletiram ao longo das últimas aulas.

 \Image{Ilustração passagem (Isabel Teixeira (Ateliê Fora de Esquadro); Direitos cedidos à n"-1)}{PNLD0062-05.png}

\paragraph{Metodologia} 

\begin{enumerate}
\item Para esta atividade, indicamos que o/a educador/a
desenvolva um estudo dirigido para ser trabalhado com antecedência em
casa e a partir da organização de grupos de quatro estudantes. O critério
para a formação do grupo deve ser o vínculo de confiança entre eles.

Este estudo dirigido deve estar dividido em duas partes. A primeira, 
contendo questões elementares referentes \emph{A pequena prisão}
como:

\begin{itemize}
\item ``Qual a temática central abordada pelo testemunho de Igor
Mendes?''

\item ``Qual o contexto de escrita do livro?''

\item ``Qual você acha que
é a motivação do autor para a produção desta obra?''

\item ``Qual o papel da
literatura na vida do autor"-personagem e suas principais influências
literárias?''

\item ``Como você caracterizaria o autor"-personagem desta
história antes, durante e depois da experiência prisional?''

\item ``De que
maneira o autor retrata o tempo na prisão e qual a relação disto com a
forma que o autor organiza o tempo em sua narrativa?''

\item ``Qual a crítica
levantada pelo autor quando afirma que alguns ainda acreditam nesta
sociedade que 'anos de chumbo são coisas do passado'?''

\item ``O autor
descreve o espaço da prisão não apenas como tortura e tragédia, mas
também como tentativa de sobrevivência e resistência. Descreva em poucas
linhas o que caracteriza este espaço, bem como os seus efeitos para os
sujeitos que ali permanecem''

\item ``A pequena prisão, afinal, cumpre qual
papel na sociedade, segundo o autor? Você acredita que poderia ser
diferente?''

\item ``Aponte ao menos quatro figuras de linguagem utilizadas
pelo autor ao longo de sua obra (quais sejam figuras sonoras, figuras de
palavras, figuras de sintaxe e figuras de pensamento), a fim de engajar
ou aproximar/sensibilizar o leitor''

\item ``Como você descreveria as demais
personagens deste testemunho, e seu estilo de vida?''

\item ``Como você
compararia os rituais vivenciados na prisão, com os rituais vivenciados
no espaço escolar?''

\item ``Qual o papel da literatura de testemunho e da
literatura de memória para as sociedades e qual o seu impacto para a
análise histórica?''

\item ``O autor tem um estilo de realizar boa parte de
suas narrativas descritivas, você saberia analisá"-lo?''

\item ``Quais os
principais elementos da narrativa de Igor que te afetaram ou despertaram
reflexões interessantes?''

\item ``No desfecho desta narrativa, as reflexões
apontadas pelo autor"-personagem indicam pessimismo ou otimismo com
relação a sua vida e a história mais ampla da sociedade em que vivemos?
Justifique sua resposta.'' 
\end{itemize}

Ressaltamos que nesta parte da atividade o
grupo de estudantes precisa debater as questões para registrar suas
respostas, cabendo a possibilidade de registro das divergências de
ideias entre os integrantes.
\BNCC{EM13LGG303}

\item A segunda parte deste material orienta aos estudantes para que,
individualmente, produzam relatos memorialísticos ou ``testemunhos'',
entre 20 e 30 linhas, baseando"-se na seguinte afirmação disparadora:
\emph{relatar a minha memória é também passar uma mensagem para o restante do
mundo}. O conjunto das respostas do estudo dirigido e a produção
memorialística dos estudantes será a base da aula presencial em questão.
\BNCC{EM13LGG101}

Presencialmente, a aula deverá ser iniciada com a apresentação da
discussão feita em cada grupo, resposta a resposta, para toda a turma.
Isso abrirá margem para debates interessantes, exercícios argumentativos
e trocas de percepções. 

\item Passada esta parte, indicamos que abra"-se um
momento para a troca de relatos memorialísticos/\,testemunhos feitos
individualmente, dentro do grupo de trabalho. Estas trocas cumprem dupla
finalidade que precisam ser orientadas aos jovens: 

\begin{itemize}
\item Que estes possam
realizar sugestões, contribuições estilísticas, para que o texto
individual se adeque bem a proposta
\item Que estes analisem pontos
convergentes e divergentes com relação as experiências de vida que o
grupo demonstrou ter a partir das produções escritas. 
\end{itemize}

\item Ao final da aula,
indicamos que os jovens exponham um pouco o debate que construíram
dentro do próprio grupo e apresentem um balanço do que aprenderam com a
atividade.

Todo o material redigido pode ser entregue imediatamente ao/à educador/a
ou, ainda, é possível que este garanta algum tempo para a devolutiva
final, revisada, dos estudantes. Esse material poderá constituir
instrumento avaliativo para a disciplina.
\end{enumerate}

\paragraph{Tempo estimado} Duas aulas de 50 minutos, aproximadamente.


\section{Propostas de Atividades II}

\Image{Identidade (Isabel Teixeira (Ateliê Fora de Esquadro); Direitos cedidos à n"-1)}{PNLD0062-08.png}

\subsection{Pré"-Leitura}

\paragraph{Tema} Direitos, crime e cárcere -- como a leitura do livro A
pequena prisão materializa estas noções e nos ajuda a compreender a
realidade do nosso país.

\paragraph{Conteúdo} Orientação de pesquisa protagonizada pelos estudantes
sobre as concepções de Direitos, Crime e Prisão/Cárcere, mobilizados por
questões problematizadoras, exibição do documentário norte"-americano
\emph{A 13ª emenda,} dirigido por Ava DuVernay e lançado em 2016, e
organização de debate.

\paragraph{Objetivo} Compreender o papel desempenhado pelos sistemas
prisionais no Brasil e no mundo, contextualizar e introduzir a leitura do
estudante para o livro \emph{A pequena prisão} através da produção
documentária destacada, mobilizar os estudantes para que de forma
autônoma pesquisem e debatam as noções de direitos, crime e prisão; e
garantir os instrumentos necessários para a desnaturalização dos jovens
sobre a realidade que nos cerca.

\paragraph{Justificativa} O ensino médio é o espaço privilegiado e
formalmente organizado para o debate e análise crítica dos mais diversos
assuntos que dizem respeito à realidade plural das juventudes com as
quais trabalhamos e aprendemos. De um lado, a necessidade de nos
aprofundarmos sobre a relação dialética entre os indivíduos e a
sociedade, com fins de conscientização humana, social e política. De
outro, uma realidade cruel nos bate a porta: no Brasil, temos a terceira
maior população carcerária do mundo, sendo a composição de mais da
metade desta formada por jovens, especialmente pobres e negros.
Compreender este fenômeno social do ponto de vista histórico e das
instituições que o cercam para em seguida analisarmos os efeitos sociais
da prisão sobre estas juventudes é tarefa urgente, caso a nossa posição
seja por garantir perspectiva de futuro a estes sujeitos, e mesmo
educá"-los na defesa de princípios democráticos e republicanos em nosso
país. É diante deste contexto que surge a proposta de trabalharmos com
\emph{A pequena prisão} nas escolas, texto de prazerosa e instigante
leitura, tanto quanto trabalharmos a origem e papel históricos
atribuídos às noções de Direito, de crime e de prisão.

\paragraph{Metodologia} 

\begin{enumerate}

\item Esta proposta de atividade se inicia com a
contextualização da proposta aos estudantes: dialogar com a abordagem em
línguas sobre a leitura do livro \emph{A pequena prisão}, e avançar na
reflexão sobre a relação entre direitos, crime e prisão, retratados
primorosamente pela obra. 

\item Dito isto, o/a professor/a pode escrever no
quadro ou utilizar uma cartolina para cada termo, seguido de uma
pergunta problematizadora que deverá ser apresentada aos estudantes
organizados em trios, como exemplos. 

\item Para o termo \emph{direitos}, é
possível perguntar se sempre existiu a noção de direito nas sociedades e
mesmo quais os tipos de direitos que encontramos e como se consolidaram
na realidade específica do nosso país. Para o termo \emph{crime}, é possível
perguntar como essa ideia surge ao longo da história, quem define o que
é crime e o que não é, e mesmo se toda violência é considerada crime ou
vice versa. Para o termo \emph{prisão} é possível perguntar se ao longo da
história da humanidade sempre existiram prisões, qual a sua função
social, se ela representa solução para problemas sociais ou mesmo porquê
esse espaço é tão invisibilizado e pouco debatido. 
\BNCC{EM13LGG304}

O/a professor/a pode
garantir cerca de dez minutos para os estudantes pesquisarem, seja por
meio de internet pelo celular, ou de livros acessíveis no momento,
informações sobre estes três termos. Assim, no retorno deste tempo, pode
pedir para que os jovens apresentem as informações que pesquisaram, tema
a tema, de modo a mobilizar o máximo de informações possíveis sobre
eles. A cada apresentação de informação e debate, o jovem pode preencher
o quadro ou cartolina com a síntese das mesmas.
\BNCC{EM13LP12}

\item Em seguida, o/a professor/a pode parar a dinâmica e prosseguir a
exibição do documentário norte-americano \emph{A 13ª emenda}, com 1h40
minutos de duração. É importante explicar que este aborda a realidade do
sistema prisional norte americano, que a sua constituição prevê pela 13ª
emenda que ``Não haverá, nos Estados Unidos ou em qualquer lugar sujeito
a sua jurisdição, nem escravidão, nem trabalhos forçados, salvo como
punição de um crime pelo qual o réu tenha sido devidamente condenado'', e
que esta realidade apresenta muitos pontos de aproximação com a
realidade do sistema penal brasileiro retrata no livro de Igor Mendes.
\BNCC{EM13LP13}

Para mais informações ao/à educador/educadora sobre este documentário, bem
como para o encaminhamento da discussão relacionada a realidade
brasileira, indicamos a leitura dos artigos de Sara de Araújo Pessoa e 
Fernanda da Silva Lima, \emph{Racismo e política criminal: uma análise a
partir do Documentário 13th -- 13ª Emenda} na Revista Thesis Juris, 
\href{https://doi.org/10.5585/rtj.v8i2.10763}{acessível aqui}. E o artigo de Vera Malaguti Batista, \emph{As tragédias dos bairros onde
moram}, na Revista Transversos, \href{https://www.e-publicacoes.uerj.br/index.php/transversos/article/view/33656}{acessível aqui}.

\item Terminada a exibição do documentário, os estudantes estarão munidos de
informações e reflexões capazes de garantir longos debates em sala de
aula, bem como associações possíveis com o livro. O fundamental deste
processo é destacar os pontos de aproximação e distinção entre a
realidade do documentário e a realidade brasileira. Questões derivadas
das anteriores sobre os termos, envolvendo uma relação entre os mesmos
pode surgir. Na medida do possível, procure registrar as principais
informações e argumentos que aparecerem neste último momento da aula,
seja no quadro ou cartolina. Caso registre em cartolina, procure
garantir um espaço na sala de aula para a exposição dos mesmos.
\end{enumerate}

\Image{Ilustração cela (Isabel Teixeira (Ateliê Fora de Esquadro); Direitos cedidos à n"-1)}{PNLD0062-09.png}

\paragraph{Tempo estimado} Quatro aulas de 50 minutos, aproximadamente.

\subsection{Leitura}

\paragraph{Tema} O conceito de prisionização abordado no livro ``A pequena
prisão'' e a música de Johnny Cash.

\paragraph{Conteúdo} Exposição da luta do músico Johnny Cash para atrair
visibilidade à discussão sobre o sistema prisional nos EUA, e orientação
para o trabalho de construção conceitual do termo prisionização
comparativamente ao conceito de assimilação, aplicável à realidade de
outras instituições.

\paragraph{Objetivos} Ampliar o repertório dos estudantes sobre o debate
prisional, através da abordagem biográfica de Johnny Cash e análise de
algumas de suas músicas. Fornecer instrumentos para que os mesmos
articulem ideias e desenvolvam com suas palavras o conceito de
prisionização e apontar distinções deste conceito com relação ao de
``assimilação'' cultural, comumente debatido em sociologia. Analisar até
que ponto a instituição e espaço escolar, assim como outros, mobilizam
processos de assimilação cultural nas juventudes.

\paragraph{Justificativa} Problematizar o processo de afetação dos
espaços, e de tudo o que suas relações sociais constroem, nos sujeitos é
descortinar as pequenas relações de poder que nos cercam e que também
exercemos sobre o meio. A consciência sobre essa ``via de mão dupla''
que mobiliza a história, permite aos jovens desnaturalizar suas
vivências, perceber as simbologias presentes em diferentes realidades e
por isso mesmo ajudam a aprofundar ainda mais as reflexões sobre a
necessidade de se encaminhar proposições que garantam solução ao
problema do encarceramento em massa no nosso país.

\paragraph{Metodologia} 

\begin{enumerate}
\item Esta atividade se iniciará com a apresentação aos
estudantes de sua proposta geral: formular coletivamente, com amparo nos
debates e leitura do livro \emph{A pequena prisão}, o conceito de
\emph{prisionização}, buscando diálogo com a noção de assimilação cultural e
refletindo como esta pode ser aplicada à realidade escolar. 

Para isso,
o/a professor/a pode perguntar se alguém já ouviu falar no cantor Johnny
Cash, conhecido pela organização de diversos shows e gravação de discos
ao vivo, no espaço da prisão. 

\item Em seguida, indicamos a exibição do vídeo
``Johnny Cash na Cadeia. {[}Antídoto\#84{]}'', presente no Youtube, onde
um narrador conta, em cerca de seis minutos, um pouco a história do
cantor, sua relação com a luta contra o encarceramento em massa e
tratamento dedicado aos presos, tanto quanto a experiência de outros
artistas e figuras políticas que viveram a realidade da prisão, mesmo
sem ter passado por um julgamento efetivamente justo (você, aliás, pode
pedir como trabalho complementar, a pesquisa sobre a trajetória de vida
de uma destas pessoas que apareceram no vídeo).

Um instrumento para
amparar o conhecimento do/a educador/a e consequentemente dos estudantes
pode ser também o artigo ``Johnny Cash e a Prisão Folson: a real
história de Glen Sherley'', no seguinte
\href{https://canalcienciascriminais.jusbrasil.com.br/artigos/409579800/johnny-cash-e-a-prisao-folsom-a-real-historia-de-glen-sherley}{link}.

\item Para seguir mobilizando o envolvimento dos jovens com relação a esta
história, o/a educador/a pode distribuir as letras em inglês,
acompanhadas com a devida tradução para o português de duas músicas do
cantor, e colocá"-las para tocar, de preferência por meio de
vídeo"-gravações no próprio espaço prisional. 
\BNCC{EM13LGG403}

As duas músicas que
destacamos para a atividade são: \emph{Folsom Prison Blues}, um de seus
primeiros sucessos, dialoga muito bem com \emph{Diário de um detento}, de
Racionais MC's, e \emph{San Quentin}. Uma terceira música possível, caso
os estudantes demonstrem interesse em suas canções chama"-se \emph{I Got
Stripes}. Seguem abaixo as letras das músicas compostas por Johnny Cash,
com suas traduções retiradas do site
\url{www.letras.mus.br}.

\Image{Ilustração do livro a partir de foto de Bruna Freire  (Isabel Teixeira (Ateliê Fora de Esquadro); Direitos cedidos à n"-1)}{PNLD0062-10.png}

\begin{verse}
\textbf{Blues da prisão de Folsom}\\!

Eu ouço o trem chegando\\
Está passando pela curva\\
E eu não tenho visto o sol desde\\
Eu nem me lembro quando\\
Eu estou preso na prisão de Folsom\\
E o tempo continua se arrastando\\*
Mas aquele trem continua passando\\*
Indo até San Anton\\!
Quando eu era só um bebê\\
Minha mãe me disse: Filho\\
Sempre seja um bom menino, nunca brinque com armas de fogo\\
Mas eu atirei num homem em Reno\\
Só para vê-lo morrer\\
Agora toda vez que escuto aquele apito\\*
Eu seguro minha cabeça e choro\\!
Eu aposto como existem ricaços jantando\\
Num luxuoso vagão restaurante\\
Eles provavelmente estão bebendo café\\
E fumando grandes charutos\\
Bem, eu sabia que isto aconteceria\\
Eu sei que não posso ser livre\\
Mas essas pessoas continuam vivendo\\*
E é isso o que me tortura\\!
Bem, se eles me libertassem desta prisão\\
Se aquele trem fosse meu\\
Eu aposto como eu iria só mais um pouco longe\\
Longe da prisão de Folsom\\
É onde quero ficar\\*
E deixaria aquele apito triste\\*
Levar minhas tristezas para longe
\end{verse}

\item Dada a primeira execução das músicas, indicamos que o/a educador/a leia
com o acompanhamento dos estudantes, em voz alta, a página 316 do livro, destaque para a leitura de rodapé, e o primeiro
parágrafo da página 317. Em seguida, mobilize o debate entre os
presentes a partir das questões:

\begin{itemize}
\item `Qual o impacto de rotular e
esteriotipar pessoas sem sequer conhecê"-las em profundidade para a vida
coletiva e individual?

\item`O que vocês compreenderam por
'prisionização'? Seria a mesma coisa que o conceito de 'assimilação
cultural', segundo o autor?''

\item ``A prisionização acontece invariavelmente ou
os sujeitos podem apresentar alguma margem de resistência sobre este
fenômeno?''

\item ``Você caracterizaria o fenômeno de prisionização enquanto
fenômeno violento? Por que?''
\end{itemize}

Durante o debate, procure retomar elementos das músicas de Johnny Cash
que tratam sobre a prisionização e de que maneira podemos identificá"-la
nas experiências individuais. Registre as respostas mais interessantes
no quadro ou em folha de cartolina para a turma.
\BNCC{EM13CHS502}

\item Por fim, você pode colocar para tocar mais uma vez alguma música de
Johnny Cash em destaque, de acordo com os interesses dos jovens a partir
do debate, e indique uma tarefa para casa com o seguinte comando:

``Vimos com os conceitos de \emph{assimilação cultural} e \emph{prisionização} o
poder de influência do espaço institucional, e do conjunto de suas
relações, na vida dos indivíduos. Agora é a sua vez de refletir: ``\emph{De que
maneira você avalia que se desenvolve o processo de assimilação cultural
no espaço escolar?} Redija um texto a partir do tema destacado, entre 10
e 20 linhas.''
\BNCC{EM13LGG302}

Para a próxima aula, certamente esta turma terá ``pano para manga'' para
argumentar sobre os efeitos positivos e negativos da convivência escolar
em suas vidas.
\end{enumerate}

\paragraph{Tempo estimado} Duas aulas de 50 minutos, aproximadamente.

\Image{Ilustração brasão penitenciária (Isabel Teixeira (Ateliê Fora de Esquadro); Direitos cedidos à n"-1)}{PNLD0062-11.png}

\subsection{Pós"-leitura}

\paragraph{Tema} A pequena e a grande prisão pelo olhar dos estudantes
-- oficina de criação e teatro.

\paragraph{Conteúdo} Organização de assembleia estudantil sobre \emph{A
pequena prisão} brasileira e proposição à produção de esquete teatral
protagonizada pelos leitores, que retrate criativamente a relação entre
a pequena e a grande prisão e registre, por meio da construção de um
painel, dados estatísticos acerca do sistema carcerário brasileiro e
sobre o contexto de escrita do livro.

\paragraph{Objetivos} Para ser feita junto a professores de outras áreas, em especial Ciências Humanas. Desenvolver a capacidade analítica e argumentativa
sobre a obra, seu contexto e sobre o sistema prisional em si. Estimular
o trabalho de pesquisa sobre o sistema prisional brasileiro em
comparação com outros sistemas no mundo. Desenvolver a autonomia, o
espírito de cooperação, a criticidade, o engajamento e a criação
artística dos jovens a partir da proposição de uma esquete teatral a ser
apresentada na escola.

\paragraph{Justificativa} O exercício de leitura de um livro costuma ser
um ato solitário, que nos garante envolvimento com os personagens e com
todas as suas elucubrações, sentimentos e experiências. Conhecer seus
recursos estéticos, estilísticos e a ética por detrás de suas linhas,
garante profundidade ao nosso olhar, criticidade e afastamento de uma
série de pré"-noções que carregávamos conosco antes da experiência. Tal
qual Heráclito de Éfeso nos aponta ``não se pode entrar duas vezes no
mesmo rio'', cada leitura nos representa uma vivência e uma elaboração
mental sobre ela, distintos. É justamente no coletivo, na troca, no
diálogo, a partir de uma perspectiva crítica no sentido de ir a fundo,
lidar com a raiz das questões, e empática que garantimos uma riqueza da
experiência literária ainda maior. Garantimos uma espécie de afetação
mútua, pelos elementos da obra, e por aqueles que interpretam a obra, os
``outsiders'' da experiência, no caso, prisional. Aí reside a relevância
dos debates abertos em sala, que devem sempre ser muito bem organizados,
mediados e registrados em seus aspectos principais. Associar esta
prática com a pesquisa de dados estatísticos e históricos sobre o
sistema prisional brasileiro em comparação com outros sistemas
prisionais no mundo é munir os estudantes de instrumentos teóricos,
argumentativos e propositivos para os futuros desafios que enfrentará ao
longo de sua escolarização (lembramos que as redações do ENEM e outros
vestibulares/exames demandam tais habilidades) bem como de sua vida.
Finalizar esse processo, mobilizando a produção artística destes jovens
por meio da criação de esquetes teatrais sobre a pequena e grande
prisões representa de fato elevar a altos níveis suas potencialidades e
aprendizados. Para além das demandas formais presentes na BNCC, e que
aqui estão sendo contempladas, esta experiência, em particular, permite
que o jovem formule encontros da obra com suas próprias experiências, e
torne"-se protagonista da ``afetação'' de tantas mais pessoas.

\Image{Ilustração relatos  (Isabel Teixeira (Ateliê Fora de Esquadro); Direitos cedidos à n"-1)}{PNLD0062-12.png}

\paragraph{Metodologia} 

\begin{enumerate}
\item Esta aula será dividida em dois momentos. O
primeiro é a conformação de um espaço de assembleia de estudantes para a
discussão sobre o livro \emph{A pequena prisão} e o sistema carcerário
brasileiro. Atentamos que a conformação deste espaço não é elemento
secundário da atividade, visto o que debatemos sobre o efeito dos
espaços nos sujeitos e vice"-versa. 
\BNCC{EM13LP25}

\item O tema de mobilização do debate pode
ser: ``O sistema penal e prisional brasileiros estão efetivamente
solucionando o problema do crime e da violência no país? A quais
interesses eles servem?''. É importante que nesta atividade se garanta, além
da mediação do/a professor/a coordenador da aula (educadores de outras áreas
devem estar presentes para contribuírem com as reflexões), a
preparação para a mediação de um ou dois jovens da turma. Eles podem
cumprir o papel de protagonizar a apresentação da assembleia e
alimentá"-la com informações provenientes de pesquisas científicas sobre
o tema e mesmo com a abordagem de trechos do livro. 

Outras perguntas
mobilizadoras da atividade podem ser:

\begin{itemize}
\item ``Cite uma palavra que caracterize a
pequena prisão e justifique''

\item ``Quem são os prisioneiros das pequenas
prisões brasileiras? São tão diferentes dos 'prisioneiros' mais
oprimidos da grande prisão denunciada por Igor Mendes?''

\item ``Como vocês
avaliam a ideia de vivermos numa grande prisão, como indica o
autor"-personagem?'' 
\end{itemize}

Destacamos que, para o encerramento, é fundamental que
se demande dos estudantes propostas de intervenção diante desta questão
social.

\item Realizada esta etapa da aula, indicamos que o/a educador/a que esteja na
coordenação da atividade, proponha a formação de grupos de estudantes
organizados preferencialmente por critérios de afinidade. Indicamos que
os grupos possam ter cerca de 10 integrantes, a fim de que dividam
tarefas internamente. Estes grupos deverão ser orientados a elaboração
de uma esquete teatral, ou seja, uma pequena apresentação que retrate a
pequena e grande prisões, e a busca humana permanente por liberdade.
\BNCC{EM13LGG201}

Esta atividade criativa pode retratar diretamente algum aspecto da obra
de Igor Mendes, ou pode ainda, influenciada por ela, retratar um outro
cenário, experiência esta que pode ser interessante para que se avalie
pontos de convergência e divergência entre a escrita e a interpretação
do leitor. Como pano de fundo da peça, é obrigatória a elaboração de um
painel de cartazes, retratando informações históricas e estatísticas
sobre o sistema prisional brasileiro e o contexto de produção do livro
\emph{A pequena prisão}. 

A criatividade e engajamento para a produção
dos diálogos, interpretação e montagem de cenário devem ser quesitos
avaliados. Garantindo"-se o devido tempo para esta produção em sala de
aula de maneira orientada pelo/a professor/a e mesmo fora dela,
indicamos que, do total de tempos estimados, um seja fragmentado e
dedicado à apresentação dos estudantes, seja para suas próprias turmas,
seja para o conjunto da escola.
\end{enumerate}

\paragraph{Tempo estimado} Quatro aulas de 50 minutos, aproximadamente.


\section{Aprofundamento}

Lançado em setembro de 2017, \emph{A
pequena prisão} é, segundo a
professora Vera Malaguti Batista, no prefácio da obra, ``talvez o mais
importante livro brasileiro de criminologia dos últimos tempos''. Isto
porque, para ela, ``tudo o que tentamos descrever como o sistema
penitenciário brasileiro aparece aqui como uma verdade encarnada nos
corpos dos seres humanos com que Igor conviveu em Bangu''. O relato,
escrito em primeira pessoa, num tom testemunhal, refaz o percurso deste
autor"-personagem por diferentes presídios cariocas entre dezembro de
2014 e junho de 2015.

\Image{``Em defesa dos presos e perseguidos políticos'' (Isabel Teixeira (Ateliê Fora de Esquadro); Direitos cedidos à n"-1)}{PNLD0062-13.png}

\emph{A
pequena prisão}, como literatura de testemunho, partilha da vocação
do gênero em imbricar questões éticas e estéticas, como nos lembra o
professor Jaime Ginzburg:

\begin{quote}
O estudo do testemunho articula estética e ética como campos
indissociáveis de pensamento. O problema do valor do texto, da
relevância da escrita, não se insere em um campo de autonomia da arte,
mas é lançado no âmbito abrangente da discussão de direitos civis, em
que a escrita é vista como enunciação posicionada em um campo social
marcado por conflitos, em que a imagem da alteridade pode ser
constantemente colocada em questão''. (Jaime Ginzburg \emph{apud}
Maria Isolina de Castro Soares. Link para acesso
\href{https://abralic.org.br/anais/arquivos/2018\_1547730007.pdf}{aqui}).
\end{quote}

Neste diapasão, \emph{A pequena prisão} dá vida e corpo a debates
caríssimos às juventudes sobre justiça, democracia, liberdade, dentre
outros, ao mesmo tempo em que esmiúça a relação dialética entre o
indivíduo, seus comportamentos e crenças, e o meio em que se insere.
Como lembram MACIEL et al. (2007), as memórias fazem parte da literatura
autobiográfica, e afirmam que ``as inexatidões da memória, capacidade
humana de armazenar dados, transformam os fatos em recordações por meio
da linguagem''. Os autores salientam que as memórias buscam as
recordações do narrador com o objetivo de evocar pessoas e
acontecimentos representativos num momento posterior, ou seja,
registra"-se uma realidade que poderá ser retomada. A importância do
gênero recai, portanto, no caráter histórico que possui, pois, mesmo que
não se pretenda contar os fatos de modo objetivo, calca-se a narrativa
em algo que já passou. Também ressaltam que o texto remete à volta do eu
passado para construir o presente. A partir da análise do passado,
pode"-se compreender e atribuir novos significados ao presente. Assim,
\emph{A
pequena prisão}, ao jogar luz sobre os invisibilizados, revela"-se
na verdade um texto inquietante, posto que nos leva a questionar sobre as
grades -- mais ou menos visíveis - que nos cercam aqui fora. E isso se
faz não implícita, mas declaradamente pelo autor:

\begin{quote}
Por que falo em pequena prisão? Exatamente porque, iludidos com
uma sociedade autoproclamada 'livre', vivemos na verdade em uma
imensa, cada vez maior, prisão. Não creio que possamos considerar
realmente livres os que têm de enfrentar a rotina de um trabalho
extenuante e embrutecedor, coagidos pela fome e pela ameaça de
desemprego. 'Livres' para ir ao supermercado e assistir televisão.
'Livres' para acordar ainda de madrugada, atravessar a cidade em
transportes caros e precários. 'Livres' nas nossas prisões
domiciliares, cheias de pequenos luxos desnecessários, pelos quais
pagamos o equivalente a uma vida inteira de trabalho -- isso quando
temos o 'privilégio' de ter um teto sob o qual nos abrigar. Desse
ponto de vista, o que chamamos de prisão, a cadeia, é apenas uma fração
da prisão maior em que vivemos -- um pouco mais pobre de vida, mais
descaradamente odiosa, é verdade, mas ainda assim uma fração, se
comparada ao grande presídio de povos em que se converte nossa sociedade
nesses princípios de século \textsc{xxi}. Não me julguem pessimista: há um ditado
penitenciário que diz: 'a cadeia é longa, mas não perpétua', e creio
firmemente que isso é válido tanto para a pequena quanto para a grande
prisão''. (\emph{A
pequena prisão}, ps. 34"-35)
\end{quote}

Para interpretar e traduzir esta obra, que transita além da literatura,
por diversos campos das ciências humanas, como por exemplo a sociologia,
filosofia e antropologia, nos valemos de uma série de instrumentos que
buscam engajar os estudantes na análise crítica da leitura, na reflexão
sobre ética e estética para além da literatura, na pesquisa e produção
de textos verbais e multissemióticos, na construção de estruturas de
argumentação e, o mais importante, na transformação de todo este
conjunto de reflexões, que certamente ampliarão sua visão de mundo, em
ações propagadoras de uma sociedade efetivamente justa, democrática e
inclusiva.

O interesse de obras sobre o sistema penitenciário -- seja na literatura
ou na dramaturgia -- tem sido crescente nos últimos anos. Talvez porque
estejamos em plena vigência do que alguns estudiosos têm chamado de
hiperencarceramento, fenômeno cuja compreensão deve se dar de modo
indissociável da ampla deterioração das condições de vida dos
trabalhadores e sucateamento dos serviços públicos, no âmbito do
neoliberalismo. Assim, segundo Wacquant,``à atrofia deliberada do
Estado social corresponde a hipertrofia distópica do Estado penal: a
miséria e a extinção de um tem como contrapartida direta e necessária a
grandeza e prosperidade insolente do outro'' (Loicq Wacquant em \emph{As
Prisões da miséria}''. Paris: Raisons d'Agir, 1999).

No Brasil, a população carcerária explodiu nos últimos trinta anos: de
90 mil pessoas privadas de liberdade em 1990, tínhamos no ano de 2020
860 mil pessoas nessas condições (a terceira maior população carcerária
do mundo, atrás somente dos Estados Unidos e da China). Embora seja
bastante precária a base de dados oficiais sobre o perfil desta
população, o último levantamento do Sistema Integrado de Informação
Penitenciária (InfoPen), vinculado ao Ministério da Justiça, relativo a
2014, apontava que os jovens representavam 54,8\% desta população 
(link disponível em 
\href{https://www.gov.br/mdh/pt-br/noticias\_seppir/noticias/junho/mapa-do-encarceramento-aponta-maioria-da-populacao-carceraria-e-negra-1}{gov.br}).
Não temos motivos para crer que este dado tenha sofrido alterações
sensíveis desde então.

\Image{Luta--luto (Isabel Teixeira (Ateliê Fora de Esquadro); Direitos cedidos à n"-1)}{PNLD0062-16.png}

Só isto já seria razão suficiente para avivar o interesse dos estudantes
pelo tema do cárcere. Mas, em \emph{A
pequena prisão}, encontramos ainda um
outro tema profundamente relacionado à juventude: a sua participação
política. Isto porque, como já se disse, o autor foi encarcerado em
decorrência de sua participação nas manifestações de 2013 e 2014, que
foram uma explosão de indignação popular em geral e de indignação
juvenil em particular. A luta contra o aumento das passagens, desatou um
amplo movimento de protesto em todo o país, associado à realização
próxima dos mega"-eventos, tendo como contrapartida do Estado uma ampla
repressão policial. Segundo relatório publicado pela ONG Artigo 19, nos
696 protestos registrados especificamente em Junho de 2013, ocorreram 8
mortes, 837 pessoas feridas, 2608 detidas e 117 jornalistas agredidos ou
feridos. (link: \url{protestos.artigo19.org/panorama.php}). O livro, portanto,
é uma espécie de junção de fios desencapados de alta voltagem: prisão e
protestos.

Segundo Leonardo Nascimento, analisando a obra no artigo ``O
compromisso assumido com os silenciados'':

\begin{quote}
Ao se reconstruir como
sujeito de uma experiência dolorosa, Igor nos oferece a possibilidade de
sermos `afetados' por suas memórias, no sentido em que o termo é
empregado por Jeanne Favret-Saada em seus postulados sobre o ofício de
etnógrafa. Para essa autora, `ser afetado' não seria o mesmo que ter uma
experiência de empatia, uma vez que não somos portadores da capacidade
de nos transformarmos em outros. O que é possível, portanto, é
estabelecer uma relação de afetação, uma prática em que saímos alterados
em nossos juízos e verdades. (Disponível no link: 
\href{https://suplementopernambuco.com.br/artigos/2032-o-compromisso-assumido-com-os-silenciados.html}{suplementopernambuco.com.br}).
\end{quote}

É, neste sentido, impossível não ser afetado diante da narrativa sobre
as dores, esperanças e misérias desta gente esquecida que habita os
porões escuros da sociedade brasileira.

E qual é o papel da literatura, senão o de despertar tais afetos? E de
apreender, numa realidade complexa, toda uma teia de sentidos? Antônio
Candido, comentando Memórias do Cárcere, de Graciliano Ramos, diz"-nos:

\begin{quote}
Parcela desse halo negativo, a prisão preocupa e fascina a
literatura moderna, desde os mestres do romance no século passado.
Atenuada em Dickens, terrível em Victor Hugo e Balzac, monstruosa em
Dostoievski. Para o romancista, é uma espécie de laboratório, donde
surgem as soluções mais inesperadas e contraditórias. Se de um lado
piora as relações humanas, ela as refaz ao seu modo, e neste processo,
fazendo descer ao máximo a humanidade do homem, pode extrair do bárbaro
novas leis de pureza e lealdade. É como se houvesse em nós um
joão-teimoso que precisa a qualquer preço, e em meio à degradação mais
profunda, estabelecer algumas leis de conduta para poder, através delas,
afirmar aspirações de limpeza. (Antônio Candido. \emph{Ficção e confissão},
ed. Ouro sobre Azul, Rio, 2006).
\end{quote}

Parece que Igor não foge a esta joão"-teimosia, uma vez que nos descreve
o submundo carcerário não do ponto de vista do pitoresco, mas do
\emph{quantum} humano que há nele. Por essa via, encontra mais
identidades do que diferenças entre os ``de dentro'' e os ``de fora''
dos muros. Neste sentido, Igor recusa qualquer maniqueísmo, do tipo que
tende umas vezes a romantizar, outras a desumanizar, as personagens
narradas, como faz questão de nos esclarecer logo na Advertência:

\begin{quote}
Não espere, caro leitor e cara leitora, uma descrição minuciosa
sobre lugares e objetos. Essa descrição, quando aparece, foi feita
sempre em função de desvendar o estado de espírito, o que pensavam e
como agiam aqueles que davam vida ao ambiente hostil, moviam a
engrenagem aparentemente monótona. O fator essencial da sociedade é o
Ser Humano, não as coisas, daí que minha atenção se voltou toda para os
personagens que encontrei, no caso, pessoas reais, assombrosamente
reais, tão complexas quanto cada um de nós''. (\emph{A
pequena prisão}, p.35).
\end{quote}

\Image{Carta para Igor Mendes (Isabel Teixeira (Ateliê Fora de Esquadro); Direitos cedidos à n"-1)}{PNLD0062-18.png}

\emph{A
pequena prisão} nos cativa, na verdade, porque é um livro sobre o
Brasil.

Atualmente, Igor Mendes trabalha como professor e escritor. O ``processo
dos 23'' tornou"-se verdadeiro imbróglio jurídico, sem desfecho à vista.
Após \emph{A
pequena prisão}, publicou \emph{Resistir é preciso}, um breve
ensaio sobre as manifestações de 2013"-2014 (n-1 edições, 2018) e seu
primeiro romance ficcional, \emph{Esta indescritível liberdade} (Faria e
Silva edições, 2020).

\section{Sugestões de referências complementares}

\subsection{Audiovisual}

\begin{itemize}
\item\textit{Carandiru}. Direção: Héctor Babenco (Brasil, 2003).

A adaptação para o cinema do livro do médico Dráuzio Varella narra suas 
impressões sobre a prisão, em visitas como voluntário em um programa 
de tratamento da AIDS.

\item\textit{O Dia em que Dorival Encarou a Guarda}. Direção: Jorge Furtado e José Pedro Goulart (Brasil, 1986).

O curta conta a história de Dorival, um homem preso em sua cela, 
sentindo muito calor, deseja poder tomar banho. No entanto, 
para isso, tem que desafiar os carcereiros e suas vontades 
individuais.

\item\textit{Olga}. Direção: Jayme Monjardim (Brasil, 2004).

O filme baseado na história de Olga Benário, companheira de Luís 
Carlos Prestes, líder da Intentona Comunista de 1935. Porém, durante 
o governo Vargas, ambos são presos e Olga é deportada para Alemanha 
nazista.

\item\textit{Parasita}. Direção: Bong Joon Ho (Coreia do Sul, 2019).

Uma família inteira desempregada e vivendo em condições limitadas tem 
seu destino alterado a partir do filho mais novo, que consegue um emprego 
como tutor para a filha de uma família burguesa. O filme retrata as relações 
de trabalho entre patrão e empregado e as nuances de cada uma das famílias, 
suas semelhanças e diferenças.

\item\textit{Quase dois irmãos}. Direção: Lúcia Murat (Brasil, 2005).

Amigos desde a infância, por conta de seus pais, quando presos durante 
a década de 70, se encontram na prisão da Ilha Grande, no Rio de Janeiro. 
Enquanto um estava preso por motivos políticos, o outro estava preso por 
ser um grande traficante de drogas.

\item Entrevista com Igor Mendes sobre a obra \emph{A pequena prisão}

Ciência \& Letras -- \emph{A
pequena prisão}. 
Apresentador: Renato Farias. Entrevistado: Igor Mendes. 
Programa Ciência e Letras, Canal Saúde. Exibição em 28 de novembro de 2017, \href{youtube.com/kwLs4aImRRM}{{Youtube}}.
\end{itemize}

\subsection{Musical}

\begin{itemize}
\item``Cálice'', em \emph{Chico Buarque} (1978), Chico Buarque.

Disponível no \href{https://www.youtube.com/watch?v=9y2xB90A0CY&ab_channel=ChicoBuarque-Topic}{Youtube}, a música de Chico Buarque, em uma brincadeira sonora com a palavra ``cálice'' e a conjugação do verbo calar, no imperativo, ``cale"-se'',
canta a censura, o tormento e o sofrimento físico, expressivos do período da ditadura militar.

%\item``Fábrica'', em \emph{Dois} (1986), Legião Urbana.

\item``Hey Joe'', em \emph{Rappa Mundi} (1996), O Rappa.

A versão brasileira da música homônima de Jimi Hendrix narrada pelo grupo O Rappa, disponível \href{https://www.youtube.com/watch?v=4yS9fXaeqss&ab_channel=Leandrocururu}{aqui} os rumos 
da vida de um homem em uma comunidade de poucos recursos e permeada pela violência. 

\item``Pequena memória para um tempo sem memória'', em \emph{De Volta Ao Começo} (1980), Gonzaguinha.

\href{https://www.youtube.com/watch?v=SJ_1pjnW2Lg&ab_channel=fdbb1}{Pequena memória para um tempo sem memória} de Gonzaguinha 
canta a história de um povo que sem liberdade para lutar pelos seus direitos, resiste.

\end{itemize}

\section{Bibliografia comentada}

\begin{itemize}
%\textsc{bastista}, Vera Malaguti. \textit{As tragédias dos bairros onde moram. Revista Transversos}. ``Dossiê: Grupo Tortura Nunca Mais do Rio de Janeiro: três décadas de Resistência''. Rio de Janeiro, nº. 12, pp. 154-167, Ano 05. abr. 2018.

%\textsc{brecht}, Bertold. \textit{Canção do Remendo e do casaco}. 

%\textsc{charriére}, Henri. \textit{Papillon}. Editora Laffont, 1969.

\item\textsc{das}, Veena. ``O ato de testemunhar: violência, gênero e subjetividade''. In
\emph{Cadernos Pagu}, no.37,Campinas, Jul./Dez., 2011. 

Neste artigo a autora debate o impacto da memória e do testemunho para o desenvolvimento 
de outras realidades históricas.

\item Fórum brasileiro de
segurança pública. \textit{Anuário Brasileiro de Segurança Pública 2020}. Endereço Eletrônico em
\href{https://forumseguranca.org.br/wp-content/uploads/2020/10/anuario-14-2020-v1-interativo.pdf}{forumseguranca.org.br} 

Publicação que aborda dados sobre a segurança pública no Brasil, em
destaque dados relevantes sobre a evolução do seu sistema prisional.

%\texsc{freitas}, Alípio de. \textit{Resistir é preciso}. Editora Record, 1981.

%\textsc{henley}, William Ernest Henley. \textit{Invictus}.

%\textsc{jocenir}. \textit{Diário de um detento: o livro}. Labortexto, 2001. 

\item\textsc{koche}, Vanilda Salton; \textsc{boff}, Odete Maria Benetti. ``Memórias literárias
como um gênero textual no ensino da escrita''. In \emph{V SIGET (Simpósio
Internacional de Estudos de Gêneros Textuais) O ensino em foco}, Rio
Grande do Sul, ago. 2009. 

O artigo aborda o reconhecimento do gênero memorialístico nas políticas educacionais 
nacionais, suas particularidades e relevância no ensino da produção textual e leitura 
da língua.

%\textsc{levi}, Primo. \textit{Shemà}. 

%\textsc{lobosco}, Fábio. \textit{Prisionização: múltiplos aspectos da assimilação
%prisional}. De Jure: revista jurídica do Ministério Público do Estado 
%de Minas Gerais, n.16, p.135-162, jan./jun., 2011.

\item\textsc{lobosco}, Fábio. \textit{Sobre um novo conceito de prisionização: o fenômeno da
assimilação prisional de acordo com a realidade prisional brasileira}.
Tese de doutorado. Universidade Presbiteriana Mackenzie, São Paulo,
2017. 

A publicação aprofunda o conceito sociológico de assimilação
cultural, buscando evidenciar as particularidades deste fenômeno numa
instituição total como a prisão, e sua aplicação em especial na
realidade prisional brasileira.

\item\textsc{marchuschi}, Beth. ``A escrita do gênero memórias literárias no espaço
escolar: desafios e possibilidades''. In \emph{Cadernos Cenpec}, São Paulo, v.2,
n.1, p.47"-73, julho 2012. 

Neste artigo a autora debate propostas sobre
como desenvolver o gênero memórias literárias na prática escolar.

%\textsc{mendes}, Igor. \textit{Resistir é preciso}. n-1 edições, 2018. 

%\textsc{mendes}, Igor. \textit{Esta indescritível liberdade. Faria e Silva edições,
%2020. 

%\textsc{milani}, Wilson. ``Resenha crítica -- A experiência de Igor Mendes no Cárcere''. 
%Jornal A Nova Democracia, ano XVI, Nº 199 -- 1ª quinzena de
%novembro de 2017. Endereço Eletrônico:
%\href{https://anovademocracia.com.br/no-199/7704-resenha-critica-a-experiencia-de-igor-mendes-no-carcere}{{anovademocracia.com.br}

%\textsc{nascimento}, Leonardo. ``O compromisso assumido com os silenciados''. PERNAMBUCO: Jornal Literário da Companhia Editora de Pernambuco, dez.2017. Endereço Eletrônico:
%\href{https://suplementopernambuco.com.br/artigos/2032-o-compromisso-assumido-com-os-silenciados.html}{suplementopernambuco.com.br}

%\textsc{polari}, Alex. Os primeiros tempos da totura.

%\texsc{ramos}, Graciliano. Memórias do cárcere. Editora Record, 2020.

\item\textsc{salgueiro}, Wilberth. 	``Linguagem e trauma na escrita do testemunho''. In
Revista Conexão Letras. Rio Grande do Sul. v.3, n.3, 2008. 

Neste artigo o autor investe no estudo sobre as especificidades do gênero
literário ``testemunho'' e sua relação com a ideia de ser entendido
enquanto gênero próprio das vozes deslegitimadas e invisibilizadas de
nossa sociedade.

\item\textsc{salgueiro}, Wilberth. ``O que é literatura de testemunho (e considerações
em torno de Graciliano Ramos, Alex Polari e André do Rap)''. In Matraga.
Rio de Janeiro. v.19, n. 31, jul./dez. 2012  

Neste artigo o autor, referenciado em importantes teóricos do campo não apenas 
da literatura, como da filosofia, antropologia, história e psicologia, desenvolve as
principais características e tensões literárias acerca do gênero
``testemunho'', buscando evidenciar sua relevância social e mesmo
desenvolvimento em três obras clássicas da narrativa de experiência
prisional.

%\textsc{soares}, Maria Isolina de Castro. A pequena prisão, de Igor Mendes: 
%prisão política no Brasil democrático. Congresso Internacional 2018 -- Associação Brasileira de 
%Literatura Comparada: Circulação, tramas \& sentidos na Literatura, p. 2755-2766, ago.2018. 
%Disponível em: \href{https://abralic.org.br/anais/arquivos/2018_1547730007.pdf}{abralic.org.br}

%\textsc{tolstói}, Liev. Ressurreição. Companhia das Letras, 2020.

\item\textsc{wacquant}, Loïc. \textit{As prisões da miséria}. Paris: Raisons d'Agir,
1999. 

Nesta obra, o autor debate a noção de um Estado Penal que para
administrar o desenvolvimento dos seus interesses econômicos e políticos
recorre a criminalização da pobreza e consequentemente, aumento da
população carcerária, a exemplo do que ocorre nos Estados Unidos e em
outros países do mundo.
\end{itemize}

\end{document}


