\documentclass[11pt]{extarticle}
\usepackage{manualdoprofessor}
\usepackage{fichatecnica}
\usepackage{lipsum,media9,graficos}
\usepackage[justification=raggedright]{caption}
\usepackage[one]{bncc}
\usepackage[hedra]{../edlab}



\begin{document}


\newcommand{\AutorLivro}{Tâmis Parron (org.)}
\newcommand{\TituloLivro}{Nascidos na escravidão}
\newcommand{\Tema}{Diálogos com a sociologia e com a antropologia}
\newcommand{\Genero}{Diário; biografia; autobiografia; relatos; memórias}
\newcommand{\imagemCapa}{./images/PNLD0007-01.png}
\newcommand{\issnppub}{978-65-89705-03-1}
\newcommand{\issnepub}{978-65-89705-00-0}
% \newcommand{\fichacatalografica}{PNLD0008-00.png}
\newcommand{\colaborador}{Eduardo Modesto de Carvalho, 
      Bruno Gradella e Vicente Castro}


\title{\TituloLivro}
\author{\AutorLivro}
\def\authornotes{\colaborador}

\date{}
\maketitle

\begin{abstract}\addcontentsline{toc}{section}{Carta ao professor}

Este manual tem por objetivo fornecer apoio pedagógico aos educadores
do ensino médio. Contamos com vocês para mediar este encontro com \emph{Nascidos na escravidão}, 
livro inédito no Brasil que certamente há de causar grande movimentação de 
ideias e emoções em sala de aula. 

Construído a partir de entrevistas a ex"-escravizados feitas pelo Projeto 
Federal de Escritores, durante a presidência de Franklin Roosevelt (1933--1945), 
esse volume apresenta passagens organizadas em oito seções. Cada capítulo 
enfoca um tema que esclarece um aspecto importante da escravidão. São eles: 
Trabalho, Condições de vida, Crueldade e castigos físicos, Famílias, Atitudes 
raciais, Cultura negra, Resistência e Emancipação. A diferença no material 
conforme a cor do entrevistador é nítida: ``não vou contar nada para os 
brancos, tenho medo de fazer inimigos'', diz um dos entrevistados. Outro homem 
explicou que ``muitos escravos velhos fecham a porta antes de contar a verdade 
sobre a época da escravidão. Quando a porta está aberta, eles contam como os 
seus senhores eram bonzinhos e como tudo era uma maravilha''. Os entrevistadores 
negros ouviram uma descrição mais sombria das realidades do cativeiro.

A partir do gênero textual da entrevista, utilizado na elaboração do 
livro, propomos, aqui no manual, a construção de perfis biográficos 
que desenvolverão as competências suscitadas pela leitura. Este 
material é um convite para o professor trabalhar aspectos sociais 
e históricos por meio de uma abordagem próxima do aluno, usando 
elementos do seu cotidiano e do seu interesse.

No final do século \textsc{xix} havia, nos Estados Unidos das Américas, 
quase 4 milhões de escravizados nos estados sulistas. Parte importante 
da sociedade, o seu trabalho produzia as riquezas imensas e seu ``valor 
monetário'' era maior que a soma de todos os investimentos do país. 
Por causa da mão de obra negra, os grandes senhores de escravos eram 70\% 
das pessoas mais ricas dos \textsc{eua}. Apesar disso, as vidas escravas 
não são bem documentadas. As leis e práticas sociais rígidas proibiam 
que eles aprendessem a ler e a escrever. Os proprietários de escravos 
e os brancos do sul deixaram uma imensa quantidade de documentos, mas 
os cativos não tinham permissão para se expressar da mesma forma.

Esperamos que o trabalho seja proveitoso!


\end{abstract}

\tableofcontents

\section{Proposta de Atividades I}


\SideImage{Paul D. Escott, autor do prefácio. (Wake Forest University; Acervo pessoal do autor)}{PNLD0007-03}


\subsection{Pré"-leitura}

\paragraph{Tema} Memória e documentação.

\paragraph{Conteúdo} Relação entre documentos, sobretudo escritos, e a memória da escravidão.

\paragraph{Objetivo} Habilitar os e as estudantes a perceber o papel dos documentos na 
construção de uma narrativa fidedigna com a história da escravidão, sobretudo para aqueles
que sofreram o processo de escravização.
\BNCC{EM13LP01} % Circulação 

\paragraph{Justificativa} A escravidão faz parte da história dos Estados Unidos das Américas.
Fator sem o qual grandes fortunas não teriam sido criadas, possibilitou o desenvolvimento
do sistema econômico hoje em voga: o capitalismo liberal. Ora, a escravidão não se fez sozinha, 
e os braços que realizavam o trabalho forçado eram os de pessoas traficadas na condição de escravizados
de regiões da Costa Oeste da África até os estados do sul dos \textsc{eua} para realizar serviços forçados,
sobretudo nas plantações de algodão, que caracterizavam 50\% das exportações do país no século \textsc{xix}. 

Mas esses são dados de fora. Quando aguçamos nossa curiosidade sobre as vivências dessas pessoas
nessa  condição nos deparamos com um problema: impedidas de acessar o sistema educacional,
escravizados em geral não sabiam ler e escrever e, portanto, não podiam registrar suas rotinas,
ou expressar como se sentiam pelo meio da escrita. Tirando alguns casos como Frederick Douglas,
William Wells Brown e Harriet Jacobs, os documentos escritos no século \textsc{xix} sobre a vida dos escravizados
eram todos de senhores brancos. 

Eis que em meio à crise dos anos 1930, já no século \textsc{xx}, o Projeto Federal de Escritores, a fim de 
empregar os escritores sofrendo pela falta de trabalho, promove uma série de entrevistas com a
última geração que viveu a escravidão no país. É graças a esta mobilização que temos, hoje, 
o presente material que nos permite trabalhar os aspectos da escravidão nos \textsc{eua}. 

\paragraph{Metodologia}

\begin{enumerate}

  \item
  Partindo de uma abordagem que traga a vida e as experiências dos e das estudantes para
  o seio da sala de aula, proponha as seguintes questões: Quem foram seus avôs e avós?
  E seus bisavós? De onde vieram? Qual profissão exerceram durante a vida? 
  Usando sites na internet, peça que busquem a origem de seus sobrenomes, um tipo
  de sinalizador obre a origem histórica da família. 
  \BNCC{EM13LP33} % Minha história; questionário

  \item
  Depois deste levantamento superficial em sala de aula, peça que conversem, em suas
  casas, com seus familiares mais velhos, a fim de conseguir informações mais apuradas
  sobre o passado. É possível que, a depender da natureza de composição dos e das estudantes, 
  haja uma diferença na quantidade e qualidade dos
  materiais encontrados. Famílias de imigrantes europeus ou asiáticos que vieram
  ao Brasil em geral têm um maior acerco de documentação da genealogia, o que
  permite um acesso mais fácil à história das pessoas de gerações passadas. 
  O que não acontece com descendentes de negros escravizados e indígenas, já
  que só recentemente, numa perspectiva histórica, estas populações tiveram acesso 
  aos direitos básicos que permitiriam primeiro o reconhecimento de suas
  individualidades. 

  \item
  Esta ambientação no tema pode ser desconfortável mas é necessária para 
  que se possa adentrar com honestidade na natureza do livro. Para refletir
  sobre a ausência de informações em alguns casos, articulando as habilidades de 
  línguas estrangeiras, o professor ou a professora pode mostrar à turma a canção 
  \href{https://www.youtube.com/watch?v=L5jI9I03q8E}{``Ain't got no, I got life''} da cantora, compositora e ativista negra norte"-americana 
  Nina Simone que diz, em tradução livre:

\BNCC{EM13LGG403}

\begin{verse}
Não tenho casa, não tenho sapatos\\
Não tenho dinheiro, não tenho classe\\
Não tenho amigos, não fui à escola\\
Não tenho roupas, não tenho emprego\\
Não tenho dinheiro, nem um lugar\\

\vspace{5mm}

Não tenho pai, não tenho mãe\\
Não tenho filhos, nem irmãs mais velhas\\
Não tenho terra, não tenho fé\\
Não tenho tato, não tenho deus\\
Não tenho amor\\
\end{verse}

\SideImage{Nina Simone foi uma artista que não separou a arte de seu papel político contestatório na sociedade racista de seu país, os \textsc{eua}. (Gerrit de Bruin ; CC BY 4.0)}{PNLD0007-18}

Depois, podem ouvir a canção \href{https://www.youtube.com/watch?v=KaZeU__e5q4}{``Chororô''} da 
cantora e compositora negra e baiana Luedji 
Luna que vai tratar do mesmo assunto: as faltas e perdas na história
dos e das descendentes da escravidão:

\begin{verse}Eu não tenho chão\\
Nem um teto que me queira\\
Nem parentes que me saibam\\
Nem família que me seja\\
\vspace{2mm}
(...)\\
\vspace{2mm}
E quase tudo que tenho\\
Levo guardado dentro\\
\end{verse}

\SideImage{Luedji Luna busca em seu trabalho reconstruir os laços destruídos pelo sistema escravagista. (Marcelo Nava; CC BY-NC-ND 2.0)}{PNLD0007-17}

\item 
Por fim, mostre aos estudantes a seguinte esquematização gráfica sobre o fluxo do tráfico negreiro
durante os séculos. Chame à atenção as diferenças entre o Brasil e os Estados Unidos das Américas.

\href{https://www.youtube.com/watch?v=SKo-_Xxfywk}{Tráfico negreiro no Atlântico}.



\end{enumerate}

\subsection{Leitura}


\paragraph{Tema} Vestígios do conflito na comunicação.


\paragraph{Conteúdo} Compreensão das diferenças na relação entrevistador"-entrevistado
a partir da cor daquele.
\BNCC{EM13LGG604} % Diferenças de discurso 
\BNCC{EM13LGG202} % Ideologia


\paragraph{Objetivo} Instigar os leitores e as leitoras a perceber como, a partir dos registros,
podemos verificar os conflitos demarcados pelo sistema escravagista na própria linguagem a partir
da diferença na comunicação dos escravizados quando entrevistados por brancos e por negros.

\paragraph{Justificativa} A obra \emph{Nascidos na escravidão} traz memórias e
comentários, em primeira mão, de pessoas que foram, em um momento de
suas vidas, reduzidas à condição de escravizadas. Conforme o próprio
livro problematiza, é possível que, em alguns casos, o relato tenha sido
até enviesado, em razão de o entrevistado ter se sentido coagido, de
alguma forma, pelo entrevistador. De todo modo, percebe"-se, durante a
leitura, que se trata de uma coletânea narrada a partir da perspectivas
de pessoas que raramente tiveram a oportunidade de se expressar a
respeito do que viveram.

\Image{Fuga em massa de escravos de Cambridge, Maryland, \textsc{eua}, 1857. (William Still; CC-BY-NC 2.0)}{PNLD0007-04}

\paragraph{Metodologia}

\begin{enumerate}

  \item A partir da leitura dos capítulos ``Condições de vida'' e ``Crueldade e castigos
  físicos'', respectivamente nas páginas 63 e 101 do livro, o professor ou a professora deve 
  pedir aos e às estudantes que anotem passagens onde os ex"-escravizados relatam
  positivamente e negativamente suas memórias. Exemplo:

\vspace{5mm}

\emph{Minha família e todos os escravos da nossa fazenda eram
bem tratados, quase não se açoitava naquelas bandas. O
senhor era o feitor dos seus negros e não usava mais
ninguém. Eu servia a mesa e desnatava na casa grande.
Eu comia à mesa com a minha senhora e a família
e nunca ninguém disse nada. A gente comia toicinho,
verdura, batata irlandesa e isso que se come agora}. (p. 93)

\vspace{3mm}

\emph{O senhor e a senhora
eram brancos muito bons e eram muito bons para os negros}. [Mas um pouco depois ele diz:] \emph{Às vezes a gente
levava chibatada} [e] \emph{quando se junta uma boiada para
vender os bezerros, como choram as vacas e os bezerrinhos, era assim que eram os escravos naqueles tempos.
Eles não sabiam nada dos parentes}. (p. 151)

\vspace{5mm}

Tomando este último caso como exemplo a ser apresentado à turma antes da atividade,
pergunte a eles como pode ser entendida a súbita mudança entre ``o senhor
e a senhora eram brancos muito bons'' e logo em seguida ``às vezes a gente 
levava chibatada [...] quando se junta uma boiada para
vender os bezerros.''
% Ideologia e poder

  \item
  Ainda utilizando o recurso de comparação entre falas, peça que,
  nos mesmos trechos, se observem semelhanças e diferenças
  nos discursos de homens e mulheres. A condição de escravizados
  igualada suas experiências ou estas eram diferentes?

\end{enumerate}

  \paragraph{Tempo estimado} Duas aulas de 50 minutos.

\subsection{Pós"-leitura}

\paragraph{Tema} A escravidão nos \textsc{eua} e no Brasil.

\paragraph{Conteúdo} Diferenças e semelhanças entre as duas experiências a partir da
literatura dos dois países. \textit{Pretuguês} e preconceito linguístico.
\BNCC{EM13LGG402}

\paragraph{Objetivo} Habilitar os e as estudantes a identificar obras e autores no contexto
da escravidão no Brasil. 
\BNCC{EM13LP50} % Intertextualizade : vários autores


\paragraph{Justificativa} As diferenças entre os sistemas escravagistas brasileiro e estadunidense são
muitas. Na atividade de pré"-leitura, pôde"-se ver que, aqui, o tráfico negreiro persistiu até
os fins do século \textsc{xix} enquanto já tinha se extinguido nos \textsc{eua} ao fim da Guerra Civil. 
Além disso, a ausência de narrativas autobiográficas de escravizados ou ex"-escravizados é outro
ponto que distancia as duas experiências. O mais próximo que temos da situação dos escravizados 
nas fazendas de \textit{plantation}, neste sentido, não é um relato brasileiro, mas cubano. 
Trata-se de \textit{A autobiografia do poeta"-escravo} de Juan Francisco Manzano, afro"-cubano que driblou
os empecilhos do sistema e aprendeu a ler e escrever. Ainda que não seja brasileiro,
a narrativa de Manzano é muito próxima de nós pois o esquema escravagista neste país é bem
mais próximo ao que foi o nosso do que em relação aos \textsc{eua}. Em Cuba como no Brasil, a 
\textit{plantation} foi a realidade que prevaleceu. Grupos de 200 escravizados trabalhavam numa
mesma fazenda de açúcar ou café, enquanto nos \textsc{eua}, eram grupos bem menores, e o produto
principal era o algodão. 
Um aspecto que chama a atenção na obra de Manzano é sua linguagem. Ainda que tenha aprendido
a faculdade da escrita, não foi em sua forma padrão culta que ele escreveu o tal livro.
O espanhol ``pobre'' com que relatou suas memórias foi diversas vezes ``consertado'' por
editores, sob a premissa de torná"-lo padrão e digno de leitura --- um claro exemplo
de preconceito linguístico.

Ainda que a escravidão não tenha sido tão documentada, temos escritos
importantes de autores e autoras negros que nos possibilitam vislumbrar criticamente
esta experiência. Entre os fins do século \textsc{xix} e o começo do \textsc{xx}, temos, por exemplo, Luiz
Gama, ex"-escravizado vendido pelo próprio pai que tornou"-se jurista
defensor da causa abolicionista; Lima Barreto, homem de letras que muito e com qualidade
artística escreveu sobre a condição da população negra no Brasil, ainda que pouco reconhecido
à sua época. No primeiro caso, temos um exemplo de uso da norma padrão culta do português,
e no segundo, um autor que privilegiava o jeito de falar do povo --- uma das possíveis
justificativas para o insucesso de sua obra por um tempo, o mesmo que foi
dito sobre Manzano. Ainda temos Machado de Assis, considerado um dos grandes nomes
da literatura universal por sua obra influenciada por clássicos europeus em diálogo
com a realidade brasileira, que escrevia numa linguagem considerada ``perfeita''
sob os parâmetros da norma culta. É interessante notar que só recentemente
o autor, negro, tenha sido reconhecido como tal. 

Percebemos, com isso, que a violência do sistema escravagista ultrapassa o corpo
físico destas pessoas, e lhes atinge na própria linguagem. A conquista da linguagem
e a legitimação da mesma é algo que vemos se repetir nas experiências de emancipação da
população negra em lugares onde houve escravidão.

\paragraph{Metodologia}

\begin{enumerate}
  \item
  Para contextualizar a turma na discussão da escravidão no Brasil e sua relação com
  a literatura, o professor ou a professora pode iniciar a aula perguntando 
  quais são os autores e autores brasileiros que eles conhecem. Então, quais destes
  são negros.

  \item
  A partir disso, dividida em grupos, a turma deve elaborar um seminário sobre escritores
  e escritoras negros no Brasil. Devem ser levados em conta: suas trajetórias biográficas,
  as principais obras, e o lugar da linguagem em seus textos --- escreviam na norma culta
  ou popular? 
\end{enumerate}

  \paragraph{Tempo estimado} Três aulas de 50 minutos.

\Image{Carolina Maria de Jesus foi uma escritora brasileira conhecida pelo livro ``Quarto de despejo'' onde narra sua vida num morro. (Arquivo Nacional; Domínio Público.)}{PNLD0007-21}


\section{Proposta de Atividades II}

A obra \emph{Nascidos na Escravidão} possibilita trabalhos
interdisciplinares e integradores de diferentes campos do saber e áreas
de conhecimento. A seguir, propomos algumas atividades que podem ser
desenvolvidas conjuntamente com professores de outras áreas. Além das
habilidades de Linguagens e suas Tecnologias e de Língua Portuguesa,
indicadas nas etapas da seção anterior e válidas também para esta,
listamos a seguir as habilidades de outras áreas, presentes na abordagem
interdisciplinar:

\subsection{Pré"-leitura}

\paragraph{Tema} O papel da escravidão na economia

\paragraph{Conteúdo} Imbricamento da escravidão no desenvolvimento do sistema capitalista.

\paragraph{Objetivo} Capacitar os e as estudantes do ensino médio a perceber o papel fundamental
do sistema escravagista na constituição e desenvolvimento do sistema capitalista moderno.
\BNCC{EM13LP01} % Contexto histórico
\BNCC{EM13CHS101} %Compreensão de ideias filosóficas % Economia, industrialização, modernização; 

\paragraph{Justificativa} O psiquiatra e filósofo martinicano Frantz Fanon, na obra \emph{Pele negra,
máscaras brancas}, defende que o racismo e o capitalismo estão imbricados na sociogênese 
do colonialismo, ou seja, são condições sociais que o sistema colonial alimentou e das 
quais precisou para que pudesse funcionar. Tendo funcionado muito bem durante séculos, 
era de se esperar estes efeitos se perpetuassem nestas sociedades ex"-colonizadas. 
Consideramos necessário que estudantes do ensino médio sejam capazes de perceber 
esta relação para que tenham uma visão crítica e profunda da realidade social
apresentada em obras como \emph{Nascidos na Escravidão}, já que se atentar à questão
racial sem levar em conta desigualdades sociais, por exemplo, é insuficiente. 

\SideImage{Frantz Fanon foi um dos primeiros a estudar os efeitos 
psíquicos do sistema colonial na população negra. 
(Pacha J. Willka ; CC BY-SA 3.0)} {PNLD0007-19}

\paragraph{Metodologia}

  Mediada pelos professores ou professoras de história e geografia, deve"-se mostrar a imagem abaixo,
  publicada no jornal Diário do Brazil em 1888, na ocasião do anúncio sobre o fim da escravidão no país.
  O mercado financeiro à época reage negativamente à notícia.
 
 \Image{Nota publicada no Diário do Brazil em 1888.``A desconfiança é geral. O capital se retrahe. O espirito de empreza desaparece''(Acervo digital da Biblioteca Nacional ; Domínio Público)}{PNLD0007-20}

  A partir desta imagem, promova uma discussão a partir da seguinte pergunta: Quais são
  os interessem do mercado financeiro na manutenção da escravidão? E hoje em dia,
  quais são os eventos que provocam uma reação similar do mesmo?  

  Os professores e professoras de história e geografia devem fazer uma contextualização,
  durante a discussão, e indicar quais podem ser estes motivos. É interessante que a conversa
  seja feita em roda, onde todos e todas sintam que estão fazendo parte, podendo
  sempre participar.

  \paragraph{Tempo estimado} Duas aulas de 50 minutos.


\subsection{Leitura}

\paragraph{Tema} O abolicionismo nos \textsc{eua} e no Brasil.

\paragraph{Conteúdo} Movimentos abolicionistas.

\paragraph{Objetivo} Capacitar os e as estudantes à discussão sobre os processos abolicionistas
tendo em vista a elaboração de um projeto interdisciplinar sobre o tema, partindo do livro. 

\paragraph{Justificativa} Na página 277 do livro, encontramos o seguinte depoimento
acerca do fim da Guerra Civil e a consequente conquista da liberdade pelos negros norte"-americanos:

\begin{quote}
\emph{Foi um dia feliz para nós escravos quando chegou a notícia que a guerra acabara 
e os brancos tinham que nos soltar. O senhor chamou seus negros até o pátio da casa
grande, mas eu não fiquei por lá para ver o que ele tinha
a dizer. Saí correndo daquele lugar, gritando o mais alto que podia}.
\end{quote}

A abolição nos Estados Unidos das Américas foi conquistada por meio de uma Guerra Civil na
primeira metade da década de 1860. No Haiti, houve uma revolução quase ao mesmo tempo
em que a Revolução Francesa, em 1789, que não só pôs fim à escravidão no território
como instituiu uma República. Em geral ligada aos interesses republicanos, 
o abolicionismo assumiu facetas diferentes em cada um dos lugares onde se 
passou. No caso do Brasil, último dos países das Américas a alcançá"-lo,
foi o caso, dentre outros, de uma forte pressão do Império Britânico sobre 
o governo brasileiro afim de expandir sua Indústria.

Estudar as diferenças entre os processos abolicionistas nas diversas
partes do continente americano é, além de se aprofundar nas multiplicidades
da escravidão, compreender os fatores supostamente externos que estavam em jogo 
em cada um dos casos.

\paragraph{Metodologia} A partir da leitura do capítulo ``Emancipação'', na página 273, 
o professor ou a professora deve incentivar uma discussão acerca da abolição, com o apoio
de professores da área de história. 
Quem foram seus protagonistas? Quem a defendia e quem era contra? 
Chamar à atenção o caso brasileiro e dar exemplos da autobiografia de Luiz Gama,
como exemplo de um negro abolicionista, e, por outro lado, as cartas do escritor 
José de Alencar ao Imperador D. Pedro II em defesa da escravidão. 
Quais as naturezas dos argumentos de Gama e de Alencar, respectivamente a 
favor e contra a escravidão? E em relação aos relatos dos ex"-escravizados no capítulo, 
como eles se localizam?
EM13LGG303 % Questões polêmicas

Sugerimos que os e as estudantes, divididos em grupos, preparem um seminário sobre o tema
das diferenças no processo abolicionista nos principais países das Américas. 
\BNCC{EM13LP16}

%\Image{Luiz Gama, conhecido como ``Libertador de Escravos'', foi ele mesmo escravizado e, quando criança, foi vendido pelo próprio pai para o tráfico. (Domínio Público)}{PNLD0007-22.png}

\subsection{Pós"-leitura}

\paragraph{Tema} As representações da abolição nas artes visuais.

\paragraph{Conteúdo} Representação artística e curadoria de eventos históricos.

\paragraph{Objetivo} Habilitar os e as estudantes a fazer uma análise
crítica de obras de arte que se propõem a representar eventos históricos.

\paragraph{Justificativa} 

\paragraph{Metodologia} 

\begin{enumerate}
  \item 
  Com o professor ou a professora de artes presente, deve ser feita uma apresentação
  das seguintes obras de arte que representam o fim da escravidão. 
  Numa delas temos a obra ``Incômodo'', obra de 2014 de Sidney do Amaral, artista afro"-brasileiro
  de importante contribuição para as artes visuais. Na outra, ``A liberdade dos escravos'', datada de
  1889, de Pedro Américo.


  \item
  A partir do que foi estudado sobre o processo abolicionista no Brasil e 
  nos Estados Unidos das
  Américas, quais as leituras podem ser feitas sobre estas duas obras?
  No caso do quadro de Pedro Américo, qual o papel das pessoas negras
  representadas? De quem é o protagonismo sobre o evento? 
  E no caso de Sidney do Amaral, como se dá a representação
  das pessoas negras na construção da abolição?
% Interpretação de obra de arte
  \Image{``Incômodo'', 2014. Sidney do Amaral. (Obra do acervo da Pinacoteca do Estado de São Paulo; Divulgação)}{PNLD0007-15}



  \item
  Seguida à discussão sobre as obras, o professor ou a professora de 
  artes deve apresentar o conceito de \textit{curadoria} e, em seguida,
  propor um exercício à turma: em grupos, devem organizar uma 
  curadoria sobre os temas apresentados nos capítulos do livro ``Nascidos
  na escravidão'' com obras de arte que dialoguem com eles 
  acompanhado de um texto de apresentação. 

  \Image{``A libertação dos escravos'', 1889. Pedro Américo. (Acervo do Palácio dos Bandeirantes, Domínio Público)}{PNLD0007-16}

\end{enumerate}


\section{Aprofundamento}

Nesta seção, desenvolvemos um trabalho de aprofundamento que, em diálogo
com a formação continuada de professores, oferece subsídios para a
abordagem do texto literário. A leitura em sala de aula \emph{Nascidos
na Escravidão} pode ser enriquecida pelo aprofundamento no universo
literário em que a obra está inserida.

\emph{Nascidos na escravidão} se organiza a partir da compilação cerca
de 204 falas de ex"-escravizados norte"-americanos, entrevistados pelo
Projeto Nacional de Escritores, entre os anos de 1936 e 1938. Essas
entrevistas foram recuperadas por uma historiógrafa estadunidense, entre
as décadas de 1950 e 1960, com o objetivo de compreender o sistema
escravista vigente no país, sob uma nova perspectiva: a dos negros
escravizados do século \textsc{xix}.


\Image{Geo Simmons, um dos entrevistados. (Library of Congress; Domínio Público)}{PNLD0007-11.png}


\Image{Clara Brim, uma das entrevistadas. (Library of Congress; Domínio Público)}{PNLD0007-12.png}


É interessante estabelecer relações entre a obra e o contexto em que ela
foi produzida. Para isso, é recomendável estabelecer parcerias com
professores da área de Ciências Humanas. Entre as estratégias de leitura
indicadas na \textsc{bncc}, estão aquelas que, ``por um lado, permitam a
compreensão dos modos de produção, circulação e recepção das obras
{[}\ldots{}{]} e o desvelamento dos interesses e dos conflitos que permeiam
suas condições de produção {[}\ldots{}{]}''.

A média de idade das pessoas ouvidas nessas entrevistas era, na época em
que foram realizadas, de cerca de oitenta anos. Por isso, temos de ter a
ideia de que grande parte dos entrevistados viveu a escravidão durante a
infância, ficando em cativeiro até por volta dos dez anos. Isso ocorreu
porque, em 1860, eclodiu a \textbf{Guerra Civil Norte"-Americana}
(1860--1865), responsável pelo fim do sistema escravagista no país.

Como grande parte dos entrevistados passou a infância sob a condição de
escravizados, a obra permite observar como funcionava o regime
escravista dos Estados Unidos, sob a perspectiva de testemunhas reais.


\SideImage{Sol Waltson, um dos entrevistados. (Library of Congress; Domínio Público)}{PNLD0007-13.png}


\SideImage{Ellen Butler, uma das entrevistadas. (Library of Congress; Domínio Público)}{PNLD0007-14.png}


Há um dado curioso sobre o contexto em que as entrevistas que deram
origem ao livro foram realizadas: quando o entrevistador era uma pessoa
branca, muitas informações sobre a brutalidade do regime eram omitidas.
Já quando o entrevistador era uma pessoa negra, os entrevistados falavam
toda a verdade e expressavam sentimentos e opiniões sobre o passado.
Além do mais, os anos de 1930 foram marcados por uma forte segregação
racial institucionalizada nos Estados Unidos, sobretudo nos estados do
sul do país.

\subsection{A escravidão nos Estados Unidos}

A obra permite observar o contexto vivido pelos escravos durante o
século \textsc{xix} estadunidense. O país, nos anos de 1850 e 1860, antes da
eclosão da Guerra Civil, tinha cerca de quatro milhões de negros
escravizados, concentrados nos estados do sul. Esse número de cativos só
é inferior ao índice registrado no Império Romano, na Antiguidade. Os
principais produtos cultivados nos Estados Unidos, pelos senhores de
escravos, e que estavam voltados para o mercado externo, eram o algodão,
açúcar, arroz e tabaco. O \textbf{algodão} foi o produto de maior
relevância porque, no contexto da Primeira Revolução Industrial, era
responsável por movimentar grande parte da economia do mundo
capitalista.

\SideImage{Em um campo de algodão na Carolina do Sul, \textsc{eua}. (Okinawa Soba/Flickr; CC-BY-NC 2.0)}{PNLD0007-09.png}

Nesse sistema de \emph{plantation}, a dinâmica de trabalho de um
escravizado na fazenda de seu senhor era exaustiva. A configuração do
mundo de trabalho, porém, não era homogênea para todas as localidades em
que vigorava a escravidão naquele país: além do trabalho, é preciso ter
em mente que esses escravizados eram agentes sociais que faziam
resistência ao sistema. Portanto, a falsa ideia de docilidade dessas
pessoas deve ser combatida. A fuga era um dos principais modos de
resistência dos cativos. Na obra \emph{Nascidos na escravidão}, o relato
de Thomas Cole, no contexto da Guerra Civil, torna possível observar as
condições que permitiram que ele conseguisse fugir e adentrar nas lutas
do \emph{front} pelo Norte.

Notamos também que ele menciona os cães. Os cachorros eram um dos
artifícios usados pelos senhores de escravos na caça aos cativos que
fugiam mata adentro. Isso ocorria porque a \emph{plantation}
norte"-americana era um vasto campo aberto, de propriedade do senhor de
escravos, o qual tinha grande domínio visual do espaço. Porém, quando um
escravizado fugia em direção à mata, esse domínio se perdia e capatazes
e cachorros seguiam trilhas e pistas para capturar os cativos fugitivos.
Em uma situação de guerra, essas rotas ganhavam um significado ainda
maior como símbolos de resistência.

A estrutura da escravidão no sul dos Estados Unidos estava consolidada e
não houve um desgaste gradual ao longo do tempo, como aconteceu em
território brasileiro, por exemplo. O escravismo norte"-americano acabou
de maneira abrupta com a eclosão da Guerra Civil e com a luta das
colônias do Norte contra as colônias do Sul.

Outro dado que nos ajuda a compreender o sistema escravista dos Estados
Unidos, em comparação com o brasileiro, é o fato de que, com o fim desse
regime no ano de 1865, houve algumas políticas públicas para a inserção
dos negros escravizados na sociedade estadunidense. Ainda assim, essa
experiência durou apenas doze anos. Com o fim da dita Reconstrução, no
ano de 1877, sobreveio a ascensão do segregacionismo racial e o
surgimento de grupos supremacistas brancos. Ao mesmo tempo, o trabalho,
a resistência e a emancipação dos ex"-escravizados, conforme demonstrado
na obra, são pontos que igualmente merecem destaque para a compreensão
do período.

A obra \emph{Nascidos na escravidão} traz relatos sobre como era a
escravidão, do ponto de vista dos próprios escravizados, e essa
perspectiva rompe com uma historiografia tradicional, que dava
importância somente para documentos e fontes exclusivas dos senhores de
escravos, homens brancos e da elite daquele país.

\Image{Antigo mercado de escravos, Flórida, \textsc{eua}. (Okinawa Soba/Flickr; CC-BY-NC 2.0)}{PNLD0007-10.png}

\subsection{Atividades para o aprofundamento da pesquisa}



\subsubsection{Escrever sobre o direito à liberdade}

  A leitura de relatos acerca das vivências de ex"-escravizados nos
  Estados Unidos do século \textsc{xix} convida à reflexão sobre temas ainda hoje
  atuais, mas também possibilita o trabalho no campo das competências
  socioemocionais. O registro de trajetórias marcadas pelo sofrimento e
  pela superação das adversidades estimula o desenvolvimento da empatia
  e da solidariedade em relação a dores alheias. Por isso, na esfera da
  subjetividade e dos projetos de vida, os testemunhos de vida podem
  servir como ponto de partida para debates e rodas de conversa sobre
  temas relevantes para o universo dos adolescentes. Em parceria com
  professores de Ciências Humanas, proponha a elaboração individual de
  \textbf{crônicas argumentativas} com base nos temas do respeito às
  diferenças e do direito à liberdade. 
\BNCC{EM13LGG103} % Escrita criativa

  Verifique as visões da turma
  sobre a noção de liberdade e observe que valores estão ligados, do
  ponto de vista deles, à existência dessa condição. Em seguida,
  relacione o estatuto do sujeito livre à necessidade de respeito e
  tolerância das diferenças em todos os níveis da existência, mas,
  sobretudo, no que tange a questões culturais e étnico"-raciais. Para
  fundamentar a argumentação, estimule a pesquisa virtual a passagens da
  Constituição que asseguram as liberdades individuais e coletivas.
  Oriente, depois, o registro dos argumentos e contra"-argumentos no
  caderno, para que seja possível articulá"-los em uma unidade
  argumentativa coesa e coerente. Reserve um momento para a produção das
  primeiras versões das crônicas, que devem ser preferencialmente
  digitadas. Incentive a inserção de exemplos extraídos da atualidade, a
  partir de novas consultas a \emph{sites} confiáveis de internet.
  Partindo do cotidiano, é possível que a argumentação seja construída
  com base em um olhar poético e sensível para a realidade vivida por
  muitas pessoas em todo o mundo. O combate à discriminação deve ser
  incentivado e argumentos que fujam do senso comum podem ser
  construídos a partir da proposição de formas de intervenção concreta
  sobre os problemas apontados. As produções podem ser compartilhadas
  com os colegas e professores envolvidos na atividade, para que
  comentários críticos e construtivos sejam feitos, de forma respeitosa
  e democrática. Ao final, as versões definitivas podem ser publicadas
  no \emph{site} do colégio, em redes sociais ou em um blog da turma,
  após um processo de revisão e edição das produções textuais.


\subsubsection{O preconceito racial no Brasil}


  A luta antirracista é uma das principais pautas dos movimentos negros
  em todo o mundo. O combate ao preconceito étnico"-racial deve ser
  estimulado no cotidiano da sala de aula, sobretudo porque diversos
  discursos veiculados socialmente perpetuam o chamado \emph{racismo
  estrutural}, que muitas vezes não é notado por grande parte dos
  falantes. 
\BNCC{EM13CHS502}

  Por isso, a leitura de relatos autobiográficos, produzidos
  no contexto da escravidão, pode fundamentar as discussões em sala de
  aula. A partir das experiências de leitura de narrativas originárias
  dos Estados Unidos, proponha a elaboração, individual ou em dupla, de
  um \textbf{artigo de opinião} sobre o preconceito racial no Brasil.
  Para isso, com auxílio de professores da área de Ciências Humanas, é
  importante estabelecer paralelos comparativos sobre o regime
  escravista nos dois países, bem como sobre as consequências
  histórico"-sociais da escravidão. Atitudes segregacionistas, racismo
  internalizado e estrutural, discriminação nas esferas pública e
  privada, violência e desigualdade são aspectos que podem ser
  explicados historicamente. Além disso, professores de Ciências da
  Natureza podem apresentar argumentos científicos, sobretudo ligados ao
  campo de estudo da Genética, para combater o preconceito e as teorias
  eugenistas que ganharam força nos séculos \textsc{xix} e \textsc{xx}. A partir dos
  debates interdisciplinares, os estudantes poderão articular os
  argumentos sob a forma de artigos que discutam, de forma coesa e
  coerente, as causas históricas, os impactos sociais e formas concretas
  de atuação social e intervenção quanto ao problema do racismo na
  sociedade brasileira. É importante incluir elementos trazidos das
  leituras, mas também é recomendável estimular a consulta a livros e
  \emph{sites} confiáveis sobre a luta antirracista. As primeiras
  versões dos textos podem ser compartilhadas, em sala de aula, com os
  colegas e professores envolvidos. Na sequência, as produções podem ser
  revisadas, digitadas e editadas com auxílio do computador. Ao final,
  os textos podem ser publicados no \emph{site} da escola, em redes
  sociais ou no blog da turma destinado ao registro das experiências de
  leitura.

\subsubsection{Discussão sobre as novas interpretações 
a respeito de determinadas estátuas e monumentos públicos}


  A derrubada de estátuas de figuras históricas associadas, sobretudo,
  ao racismo, foi uma expressão de protesto multiplicada em muitos
  lugares do mundo, no ano de 2020. O movimento foi disparado pelo
  assassinato de George Floyd, cidadão negro asfixiado por um oficial
  branco, durante uma abordagem policial na cidade de Minnesota, nos
  Estados Unidos. A partir desse fato, proponha a produção individual de
  um \textbf{ensaio} escrito, por meio do qual os estudantes analisem
  criticamente um aspecto da polêmica intervenção sobre estátuas e
  monumentos públicos.
\BNCC{EM13LGG305}

  Para isso, estimule a reflexão sobre aspectos
  ligados à questão, que podem servir de base para a formulação da tese
  e da argumentação desenvolvida no texto.

Verifique se, na opinião dos alunos, a derrubada das estátuas compõe um
gesto: (a) agressivo, irresponsável e de violência gratuita contra o
patrimônio histórico; ou (b) necessário para a conscientização acerca
das injustiças sofridas pelas populações negras ao longo da História;
(c) de vandalismo infundado, disfarçado em postura politicamente correta
ou (d) aceitável para expressar a revolta e tornar público o racismo
presente na sociedade; (e) de destruição da memória de uma nação e de
iconoclastia de objetos de arte ou (f) de reparação junto às vítimas
silenciadas pelo opróbrio da escravidão; (g) de repúdio à suposta
intocabilidade de figuras históricas, cúmplices e responsáveis pela
escravidão ou (h) de grito de dor, vindo como resposta à segregação
racial, há tanto tempo legitimada e hoje implícita e explícita nas
relações sociais, marcadas pelo racismo estrutural.

Esses tópicos tratam do tema sob duas perspectivas contrárias entre si.
Ao escolher um item para orientar a sua produção, o estudante tomará
partido de um lado ou de outro, mas não deverá desconsiderar os demais
pontos de vista: é preciso também leva"-los em conta ao formular os
argumentos e contra"-argumentos.

Para produzir um ensaio, é preciso mobilizar o repertório sociocultural
e fundamentar os argumentos com base em livros, filmes, letras de
música, obras de arte e textos históricos ou filosóficos, que podem ser
mencionados ao longo do texto. É interessante estimular a consulta a
\emph{sites} de periódicos confiáveis, ligados à divulgação de
atualidades.

O ensaísta é um autor que analisa diferentes aspectos das ideias,
observando cada um dos elementos como se fossem as faces de um prisma.
Lembre"-se, no entanto, de que é preciso defender uma tese central; ao
fazer um ensaio sobre o tema selecionado, é possível adotar diferentes
posturas para construir a argumentação: o aluno poderá acolher uma das
perspectivas, rejeitando a outra, em uma atitude com inclinação ao
formato \emph{ou\ldots{} ou}; também é possível escolher um lugar enunciativo
com inclinação ao formato \emph{e\ldots{} e}, apresentando, para tanto,
soluções alternativas.

Observe, em sala de aula, que as opiniões possíveis a respeito do tema
refletem as vozes às quais cada aluno responderá: elas são, de um lado,
aquelas vozes que comentaram, analisaram e discutiram esse tema na mídia
contemporânea; além disso, de outro lado, há também as vozes dos atores
sociais que, na vivência pretérita, presente e futura dos fatos ligados
à escravidão negra, constituem a História como arena de conflitos
operada discursivamente. Ao comentar, analisar e discutir o tema no
formato de um ensaio, o texto responderá por adesão a determinadas vozes
sociais e não a outras. Em parceria com professores de Ciências Humanas,
reflita profundamente com os alunos sobre o ponto de vista a ser
defendido, uma vez que ser responsivo no ato de criar um ensaio é ser
responsável socialmente.

Os alunos podem elaborar, no caderno, um projeto de texto para organizar
os principais argumentos sobre o tema. É interessante que elaborem mapas
de ideias, façam um \emph{brainstorming} a partir das múltiplas faces da
questão, listem tópicos que auxiliem a posterior elaboração dos
parágrafos e da orientação argumentativa coerente do texto. É importante
desenvolver uma reflexão sobre o aspecto selecionado, verificando se se
as intervenções sobre monumentos do patrimônio público constituem um
gesto agressivo, inconsequente e vão, ou se podem constituir uma
participação necessária, legítima e oportuna na História da humanidade.

O ensaio permite, ao longo do texto, marcas de 1ª pessoa que explicitem
a subjetividade do autor, por meio de expressões como ``Eu creio
que\ldots{}'', ``no meu ponto de vista, \ldots{}'', ``eu penso que\ldots{}'' etc. É
possível até mesmo estabelecer paralelos com experiências pessoais e
fazer breves relatos de situações pessoais que tenham ligação com o
tema. Ao mesmo tempo, esse gênero de texto acolhe modalizadores de
dúvida e incerteza, como ``talvez'', ``é provável que\ldots{}'', ``pode ser
que\ldots{}'', entre outros que captam o processo de construção do pensamento
autoral. Esse gênero textual não visa, portanto, como ocorre em outras
modalidades de argumentação, à construção de um efeito de objetividade,
uma vez que o ensaísta conduz o leitor por um passeio pelo universo das
ideias pessoais e, ao expor um ponto de vista, estimula a reflexão
independente e a formação de opiniões do público. O leitor, portanto,
deve desempenhar um papel ativo e acompanhar o raciocínio do autor,
investigando, completando a análise por meio da autorreflexão e
formulando conclusões consistentes.

A primeira versão escrita do ensaio pode ser compartilhada com os
colegas e com os professores envolvidos na atividade. É recomendável que
os alunos leiam as produções em voz alta e anotem os comentários, as
sugestões e as críticas construtivas. Os posicionamentos podem ser
rediscutidos, com respeito e liberdade de expressão, mas sempre
respeitando os direitos humanos. Após a apresentação, cada aluno fará as
alterações e reformulações necessárias. A segunda versão do texto, se possível, 
será digitada, utilizando o computador, de acordo com a norma padrão da
língua portuguesa. As versões finais podem ser publicadas nas redes
sociais, no \emph{site} da escola ou em um blog destinado à divulgação
dos trabalhos.

\subsubsection{Oficina de poesia sobre a tolerância}


  O texto literário oferece possibilidades de reflexão e conhecimento de
  situações da vida que podem se aproximar das experiências dos
  estudantes ou colocá"-los em contato com realidades vividas por outras
  pessoas. Com base na leitura de relatos autobiográficos de
  ex"-escravizados norte"-americanos, os alunos podem conhecer
  experiências de outros lugares e épocas, que permanecem atuais e
  dialogam diretamente com o Brasil da contemporaneidade. Apesar de a
  matéria básica das produções ser extraída da vida real, a elaboração
  por meio da palavra confere uma dimensão literária às narrativas. Além
  de documentos de época, revelam"-se olhares de sujeitos sensíveis,
  marcados por situações históricas em contextos de crise. Para
  aprofundar o trabalho com conflitos ligados à experiência juvenil,
  proponha um mergulho poético nas noções de liberdade e tolerância:
  desta vez, os estudantes produzirão \textbf{poemas líricos} para
  expressar ideias e sentimentos ligados a esses temas. Como ponto de
  partida, é possível retomar produções e discussões feitas ao longo da
  leitura, bem como letras de música, pertencentes a gêneros variados,
  que tratem igualmente desses valores. Utilizando uma \emph{playlist}
  construída coletivamente pelos alunos, privilegiando o repertório do
  \emph{hip hop}, reserve um momento para uma oficina de criação
  poética. 
\BNCC{EM13LP47} %Sarau

  Na sequência, proponha a leitura compartilhada das produções,
  comentando"-as e incentivando eventuais reformulações. Por fim, os
  textos podem ser digitados, de maneira a compor uma antologia poética
  acerca da liberdade e da tolerância, que pode ser publicada no
  \emph{site} da escola ou no blog da turma, destinado às atividades
  feitas com base nas experiências de leitura.

\subsubsection{Reportagem sobre o tráfico negreiro}


  As consequências da escravidão, em diferentes países do mundo,
  geraram, ao longo do tempo, experiências semelhantes de preconceito e
  discriminação. A leitura de relatos de ex"-escravizados estadunidenses
  enriquecem as discussões sobre o tema e permitem traçar paralelos com
  realidades diversas, incluindo a da História nacional. No Brasil
  contemporâneo, as desigualdades sociais se perpetuam também em
  decorrência do passado escravista. Em parceria com professores de
  Ciências Humanas, retome as discussões sobre as raízes históricas da
  escravidão, com ênfase no Brasil, e proponha uma pesquisa sobre o
  tráfico negreiro, desta vez concentrada no contexto brasileiro. 

  Por meio da pesquisa em livros e \emph{sites} de internet, a turma,
  dividida em pequenos grupos, buscará informações sobre números de
  pessoas trazidas do continente africano para o Brasil, as
  características do sistema social, político e econômico que
  sustentavam a escravidão, mapas de rotas do tráfico de escravizados,
  relatos acerca do cotidiano nos navios negreiros, as leis que
  procuraram modificar esse cenário e poemas produzidos sobre essas
  experiências, sobretudo pela vertente socialmente engajada do
  Romantismo (conhecida como \emph{condoreira} e praticada, no Brasil,
  sobretudo por Castro Alves). A partir dos dados coletados, oriente a
  produção de uma \textbf{reportagem} escrita sobre o tráfico negreiro
  no Brasil, em linguagem de divulgação, com o objetivo de promover
  reflexões sobre questões atuais do país, tanto nas relações pessoais
  quanto no mundo do trabalho. As versões finais dessas reportagens,
  digitadas e editadas pelos grupos com auxílio do computador, podem ser
  compartilhadas no \emph{site} da escola ou no blog da turma, destinado
  à publicação de produções ligadas às experiências de leitura.
\BNCC{EM13CHS605} % Direitos Humanos



\section{Sugestões de referências complementares}\label{sugestoes}


\subsection{Filmes}

\begin{itemize}
\item\emph{A cor púrpura}. Direção: Steven Spielberg (\textsc{eua}, 1985).

Após sofrer violências em casa, a jovem Celie enfrenta trajetórias de
dor e superação. Triste e solitária, ela escreve cartas para a irmã, até
que a chegada da amante do marido transforma seu destino.

\item\emph{Besouro}. Direção: João Daniel Tikhomiroff (Brasil, 2009).

Na Bahia dos anos 20, o pequeno Manoel é apresentado à capoeira pelo
Mestre Alípio. Ao crescer, Besouro, como passa a ser chamado, recebe a
missão de defender seus semelhantes da opressão e do racismo.

\item\emph{Django Livre}. Direção: Quentin Tarantino (\textsc{eua}, 2013).

O filme é construído em torno da improvável parceria entre Django, um
escravo liberto, e o Dr. Schultz, um caçador alemão de recompensas.
Juntos, irão atrás dos criminosos mais perigosos do sul dos \textsc{eua} e
tentarão resgatar a esposa de Django.

\item\emph{Doze anos de escravidão}. Direção: Steve McQueen (\textsc{eua}, 2013).

Vencedor de três estatuetas no Oscar de 2014, o filme conta a história
real de Solomon Northup, negro livre que foi escravizado por 12 anos no
sul dos \textsc{eua}, ao cair na armadilha de uma oferta de emprego.

\item\emph{Histórias cruzadas}. Direção: Tate Taylor (\textsc{eua}, Índia,
  Emirados Árabes Unidos, 2012).

Adaptação do livro \emph{A Resposta}, este longa conta a história de
Skeeter, jovem que quer ser escritora. Ela começa a entrevistar mulheres
negras que deixaram suas vidas para trabalhar na criação dos filhos da
elite branca. Sua iniciativa irá desagradar muita gente.

\item\emph{O sol é para todos}. Direção: Robert Mulligan (\textsc{eua}, 1963).

Baseado no romance homônimo, o filme conta a história de Tom Robinson,
jovem negro injustamente acusado de violentar uma mulher branca, e do
advogado Atticus Finch, que enfrentou a rejeição da cidade para defender
o réu.
\end{itemize}

\subsection{Lugares para visitar}

\begin{itemize}
\item\textbf{Museu Afro Brasil}

O museu, localizado na Vila Mariana, zona sul de São Paulo, reúne um
acervo contendo mais de 5 mil obras relacionadas à cultura africana e
afro"-brasileira. O site está disponível \href{http://www.museuafrobrasil.org.br/}{aqui}.
\end{itemize}

\section{Bibliografia comentada}

\begin{itemize}
\item\textsc{adichie}, Chimamanda Ngozi. \textbf{O perigo de uma história única}.
  São Paulo: Companhia das Letras, 2019.

A escritora nigeriana defende que nosso conhecimento é construído pelas
histórias que escutamos e que, quanto mais diversas e numerosas forem
essas narrativas, mais completa será nossa compreensão sobre o mundo.

Em busca de alternativas às universidades nigerianas, a jovem Ifemelu
emigra para os Estados Unidos. Enquanto se destaca no meio acadêmico,
ela depara com a questão racial e com as dificuldades da vida de mulher
negra e estrangeira.

\item\textsc{alexander}, Michelle. \textbf{A nova segregação}. São Paulo: Boitempo,
  2018.

Esta obra desafiou a opinião de que o governo Obama assinalava o advento
de uma nova era pós"-racial. A autora analisa o sistema prisional dos \textsc{eua}
e expõe como o racismo estrutural opera nas sociedades ocidentais.

\item\textsc{angelou}, Maya. \textbf{Eu sei por que o pássaro canta na gaiola}.
  Bauru: Astral Cultural, 2018.

Neste romance emocionante, a autora conta a história de Marguerite Ann
Johnson, garota negra criada no sul dos \textsc{eua} pela avó, e dá voz a jovens
que, como ela, enfrentam muitos preconceitos.

\item\textsc{evaristo}, Conceição. \textbf{Olhos d'água}. Rio de Janeiro: Pallas, 2014.

Com uma escrita lírica e sensível, a escritora mineira reúne breves
contos sobre o cotidiano, concentrando o foco de interesse sobre a vida
da população afro"-brasileira.

\item\textsc{lee}, Harper. \textbf{O sol é para todos}. São Paulo: José Olympio,
  2006.

No sul dos \textsc{eua} da década de 1930, uma garotinha esperta e observadora
relata a saga do pai, um advogado que arrisca tudo para defender um
homem negro, injustamente acusado de cometer um crime.

\item\textsc{horne}, Gerald. \textbf{O sul mais distante}. São Paulo: Companhia das
  Letras, 2010.

O autor vê a escravidão em termos hemisféricos e defende que o sul
escravista dos \textsc{eua} via, em uma aliança com o Brasil (o ``sul mais
distante''), uma forma de proteção contra um futuro embate com o norte
estadunidense, na Guerra de Secessão.

\item\textsc{marquese}, Rafael; \textsc{salles}; Ricardo (org.). \textbf{Escravidão e
  capitalismo histórico no século \textsc{xix}: Cuba, Brasil, Estados Unidos}.
  Rio de Janeiro: Civilização Brasileira, 2016.

O livro reúne ensaios de historiadores brasileiros e estrangeiros sobre
a escravização de negros nas Américas, ao longo do século \textsc{xix}.

\item\textsc{mccullers}, Carson. \textbf{A convidada do casamento}. São Paulo: Novo
  Século, 2008.

Frankie é uma menina cujos sonhos se chocam com sua rotina pacata,
compartilhada com seu primo e com Berenice, a cozinheira negra da casa.
Em um verão solitário nos \textsc{eua}, suas incertezas aumentam quando recebe a
notícia do casamento do irmão.

\item\textsc{morgan}, Edmund S. ``Escravidão e liberdade: o paradoxo americano''.
  \emph{In} \textbf{Estudos Avançados} 14 (38), p. 21-50. São Paulo:
  Universidade de São Paulo, 2000. São Paulo, 2000. (Disponível em:
  \href{http://www.revistas.usp.br/eav/article/view/9507}{\emph{revistas.usp.br}}.
  Acesso em: 8 de fev. de 2021.)

O autor busca compreender como o povo estadunidense pôde, desde o
princípio, desenvolver uma dedicação às ideias de liberdade e dignidade
humanas, e simultaneamente apoiar um sistema de trabalho que negava
diariamente esses valores.

\item\textsc{northup}, Solomon. \textbf{Doze anos de escravidão}. São Paulo:
  Penguin, 2014.

O livro é o relato real e assombroso de um negro livre que, atraído por
uma proposta de trabalho, abandona a segurança do Norte dos \textsc{eua} e acaba
sendo sequestrado e vendido como escravo no Sul.

\item\textsc{porter}, Regina. \textbf{Os viajantes}. São Paulo: Companhia das
  Letras, 2020.

Por meio das múltiplas perspectivas de personagens, a obra apresenta uma
trama que avança e volta no tempo. O livro traça um panorama da vida nos
Estados Unidos entre a década de 1950 e a eleição de Barack Obama para
presidente.

\item\textsc{stockett}, Kathryn. \textbf{A resposta}. Rio de Janeiro: Bertrand
  Brasil, 2012.

A obra conta a história de Eugenia, jovem que deseja ser escritora. Ela
tem um plano brilhante, mas perigoso: escrever um livro em que
empregadas domésticas negras relatem seus relacionamentos com patroas
brancas do Mississípi, nos anos 1960.

\item\textsc{stowe}, Harriet Beecher. \textbf{A cabana do pai Tomás}. São Paulo:
  Carambaia, 2018.

Um dos romances mais importantes da época da Guerra Civil Americana, a
obra conta a história do escravo Tom e influenciou intensamente as lutas
contra a escravidão.

\item\textsc{walker}, Alice. \textbf{A cor púrpura}. São Paulo: José Olympio, 2009.

Vencedora do Prêmio Pulitzer, a autora narra com sensibilidade a vida de
Celie, uma mulher negra no sul dos Estados Unidos, do começo do século
\textsc{xx}, que sofreu abusos do padastro e depois do marido.
\end{itemize}

\end{document}

