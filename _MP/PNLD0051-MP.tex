\documentclass[12pt]{extarticle}
\usepackage{manualdoprofessor}
\usepackage{fichatecnica}
\usepackage{lipsum,media9,graficos}
\usepackage[justification=raggedright]{caption}
\usepackage[one]{bncc}
\usepackage[lamparina]{../edlab}

\begin{document}


% Pensar em road movie por conta da viagem das crinças
% Pensar que o Schwob é um estudante de clássicas e devia saber 
% bem o gênero antigo que estava operando. Olhar por exemplo
% Hermógenes e talvez achar lá. Talvez tirar exercícios 
% práticos do próprio Hermogenes (não cite o Hermógenes!)
% https://en.wikipedia.org/wiki/Progymnasmata#Fable_(mythos)

\newcommand{\AutorLivro}{Maria Valéria Rezende}
\newcommand{\TituloLivro}{Rio de sonhos}
\newcommand{\Tema}{Ficção, mistério e fantasia}
\newcommand{\Genero}{Conto, crônica e novela}
\newcommand{\imagemCapa}{./images/PNLD0051-01.png}
\newcommand{\issnppub}{---}
\newcommand{\issnepub}{---}
% \newcommand{\fichacatalografica}{PNLD0051-00.png}
\newcommand{\colaborador}{\textbf{Raphaella Lira}}

\title{\TituloLivro}
\author{\AutorLivro}
\def\authornotes{\colaborador}

\date{}
\maketitle

\begin{abstract}\addcontentsline{toc}{section}{Carta ao professor}
% O livro que aqui apresentamos será indispensável para um trabalho

O livro conta a história de Paulo, um jovem pernabucano prestes a ingressar na universidade, 
que se vê obrigado a lidar 
com as pressões familiares em relação às suas escolhas profissionais.
Paulo, que nasceu em uma cidade pequena, decide passar uma noite às margens do rio São Francisco.
antes de prestar os exames vestibulares.

Durante a madrugada, o jovem acaba salvando a vida de Iara. 
O leitor perceberá porém, que Iara também é responsável pelo 
salvamento de Paulo, pois é ao
lado dela que Paulo recupera a ligação com seu lugar de origem
e descobre o custo do progresso para os povos tradicionais do
país.

Uma história que trata simultaneamente das escolhas pessoais 
a fazer quando deixamos a infância, sem perder no horizonte
o impacto que cada uma delas pode ter na vida que partilhamos
coletivamente. \emph{Rio de sonhos} é também uma delicada história de
companheirismo e descobertas, obra de uma das mais importantes autoras
da literatura brasileira em atividade. Esperamos que a jornada de Paulo
e Iara colabore para o desenvolvimento das aptidões e competências de
seus alunos.

Acreditamos que sua leitura pode
contribuir para que os seus educandos tenham contato com personagens com os
quais se identificar, cujas jornadas permitem entrar em
contato com um Brasil diverso, que olha para o futuro sem perder o
contato com o passado. A leitura obra pode ainda 
contribuir para uma formação múltipla, em contato com conhecimentos de
História, Geografia, além dos conhecimentos trazidos pelos povos
tradicionais de nosso país.

\end{abstract}

\tableofcontents

\section{Introdução}

O livro aborda as escolhas pessoais  que fazemos para sair da infância, e 
reflete sobre os impactos dessas escolhas na nossa vida coletiva.  

Maria Valéria Rezende é uma das mais importantes autoras da literatura brasileira  em atividade.

Ela nasceu em Santos,  no ano de 1942. 
O interesse pela população carente  começou aos sete anos de idade  nas andanças com o pai,
um médico conhecido na santa casa de misericórdia de Santos. 
Enquanto o pai examinava os doentes do morro do Saboó,  a pequena Maria Valéria lia 
livros para as crianças dessa região 

Essa experiencia marcou para sempre a vida da escritora.
Atuou em seguida na juventude estudantil católica e depois entrou 
para a Congregação de nossa senhora, cônegas de Santo Agostinho,  
em 1965, onde permanece até hoje.

Maria Valéria  passou grande parte de sua vida como missionária e educadora popular.
Nos anos 1960 e 1970  escreveu livros em prosa e verso  a respeito da igreja e das classes populares brasileiras.  
A estreia na literatura aconteceu mais tarde  em 2001,  com quase sessenta anos de idade.

Poucos anos depois Maria Valéria se consagrou  ganhando a mais importante e tradicional distinção literária do país:  o prêmio Jabuti, que ganhou pela primeira vez em 2009,  com a obra infantil  \textit{No risco do caracol}. 
Em 2013  tornou a ser premiada,  dessa vez pela obra juvenil  \textit{Ouro dentro da cabeça}.
Em 2015 ganhou pela terceira vez, na categoria ``livro do ano de ficção e romance''  por sua obra 
\textit{Quarenta dias}.

Maria Valéria Rezende  mora em joão pessoa, na Paraíba. 
Já publicou inúmeras obras  e foi traduzida para diversos idiomas.
Seus livros são conhecidos  e apreciados pelo público  e pela crítica.

A obra \textit{Rio de Sonhos}
é um livro leve e divertido, com linguagem acessível e descrições são claras.
O leitor tem a sensação de que está caminhando ao lado de Paulo e Iara. 

A obra pode ser classificada como uma novela que se utiliza de características do conto. 
A novela  se destaca pela pluralidade dramática,  com diversos conflitos e ações que se articulam,  formando um conjunto amplo de personagens em diversos espaços e em tempo mais extenso.
Já o conto  contém um só drama,  um só conflito,  uma ação,  com poucas personagens e restrição do tempo e do espaço essa unidade flui para um único efeito.

Em \textit{Rio de Sonhos},  por mais que a ação seja centrada na jornada de Paulo e Iara,  
várias características do texto  permitem que alguns o classifiquem como uma novela e não 
como um romance.
Essas características incluem o amplo espaço no qual a ação de desenvolve  como também a união de várias situações dramáticas  que levam à resolução final do texto.


\section{Atividades 1}
%\BNCC{EM13LP26}
\subsection{Pré-leitura}

%EM13LP06: Analisar efeitos de sentido decorrentes de usos expressivos
%da linguagem, da escolha de determinadas palavras ou expressões e da
%ordenação, combinação e contraposição de palavras, dentre outros, para
%ampliar as possibilidades de construção de sentidos e de uso crítico
%da língua.
%EM13LP01: Relacionar o texto, tanto na produção como na
%leitura/escuta, com suas condições de produção e seu contexto
%sócio-histórico de circulação (leitor/audiência previstos, objetivos,
%pontos de vista e perspectivas, papel social do autor, época, gênero
%do discurso etc.), de forma a ampliar as possibilidades de construção
%de sentidos e de análise crítica e produzir textos adequados a
%diferentes situações.
%EM13LP09: Comparar o tratamento dado pela gramática tradicional e
%pelas gramáticas de uso contemporâneas em relação a diferentes tópicos
%gramaticais, de forma a perceber as diferenças de abordagem e o
%fenômeno da variação linguística e analisar motivações que levam ao
%predomínio do ensino da norma-padrão na escola.
%EM13LP21: Produzir, de forma colaborativa, e socializar
%\emph{playlists} comentadas de preferências culturais e de
%entretenimento, revistas culturais, fanzines, \emph{e}-\emph{zines} ou
%publicações afins que divulguem, comentem e avaliem músicas,
%\emph{games}, séries, filmes, quadrinhos, livros, peças, exposições,
%espetáculos de dança etc., de forma a compartilhar gostos, identificar
%afinidades, fomentar comunidades etc.
%EM13LP31: Compreender criticamente textos de divulgação científica
%orais, escritos e multissemióticos de diferentes áreas do
%conhecimento, identificando sua organização tópica e a hierarquização
%das informações, identificando e descartando fontes não confiáveis e
%problematizando enfoques tendenciosos ou superficiais.

\paragraph{Tema} A presença indígena na literatura.


Antes de começar a ler o livro, é importante situar os estudantes a
respeito do tipo de leitura e discussão que os espera. Sensibilizá-los
para a representação dos indígenas, salientar aspectos históricos
envolvidos na representação, além de colocar em evidência a maneira como
esse grupo especial da população vai ser retratado na obra. Espera-se
que os fragmentos alimentem uma discussão coletiva interessante, a ser
conduzida pelo professor, e que culmine na feitura da atividade.

\paragraph{Conteúdo}  
Partindo de três fragmentos distintos, observar como é retratada a
figura do indígena, propondo ao educando que contraponha as diferentes
visões e observe semelhanças ou diferenças possíveis.

\paragraph{Objetivo}
Capacitar os alunos para reconhecer o discurso identificado
historicamente com a colonização, ao mesmo tempo em que chama a
atenção para uma outra visão possível do indígena que habita o
território nacional. Trabalhar também a questão da alteridade e de
como o outro, o desconhecido, é retratado.

\paragraph{Metodologia}
\begin{enumerate}
\item Apresente aos alunos os três fragmentos a seguir:


\begin{quote}
\begin{enumerate}
\item{[}\ldots{}{]} viu sair um rapaz novo e forte em cujos braços Iara
se aninhou e ficou, assim abraçada, falando no ouvido dele, ele
passando-lhe carinhosamente as mãos nos cabelos dela e
respondendo-lhe com expressão de imenso carinho.\footnote{REZENDE, Maria Valeria. \emph{Rio de sonhos}. Rio de Janeiro: Lamparina, 2021. p.\,48.}
\end{enumerate}
\end{quote}

\begin{quote}
\begin{enumerate}
\setcounter{enumii}{1}
\item Dissemos que essa pobre gente vive sem religião e sem lei, o que é
verdadeiro. Na realidade, não há criatura dotada de razão que seja
tão cega a ponto de, olhando para a ordem do céu, da terra, do sol e
da lua, ou para o mar e as coisas que se criam todos os dias, deixar
de considerar que tudo isso foi feito pela mão de algum artífice que
não o homem. {[}\ldots{}{]}


{[}\ldots{}{]} Os nossos selvagens fazem menção a um grande senhor,
que na língua deles se chama Tupã e que, morando no céu, faz chover
e trovejar. Mas não têm eles maneira nem hora de orar a esse deus ou
de cultuá-lo, assim tampouco há lugar próprio para isso.\footnote{OLIVIERI, Antonio Carlos; VILLA, Marco Antonio {[}organização{]}. \emph{Cronistas do descobrimento}. São Paulo: Ática, 2008. p.\,60--61.}
\end{enumerate}
\end{quote}


\begin{quote}
\begin{enumerate}
\setcounter{enumii}{2}
\item A feição deles é serem pardos, maneira de avermelhados, de bons
rostos e bons narizes, bem-feitos. Andam nus, sem nenhuma cobertura.
Nem estimam de cobrir ou de mostrar suas vergonhas; e nisso têm
tanta inocência como em mostrar o rosto. Ambos traziam os beiços de
baixo furados e metidos neles seus ossos brancos e verdadeiros, de
comprimento duma mão travessa, da grossura dum fuso de algodão,
agudos na ponta como um furador. Metem-nos pela parte de dentro do
beiço; e a parte que lhes fica entre o beiço e os dentes é feita
como roque de xadrez, ali encaixado de tal sorte que não os molesta,
nem os estorva no falar, no comer ou no beber {[}\ldots{}{]}\footnote{CAMINHA, Pero Vaz. \emph{A carta}. Disponível em:
\href{http://www.dominiopublico.gov.br/pesquisa/DetalheObraForm.do?select_action=\&co_obra=17424}{dominiopublico.gov.br},
acesso em 20 de fevereiro de 2021.}
\end{enumerate}
\end{quote}

A partir dessas três leituras, proponha uma reflexão sobre como são
retratados os indígenas, estimule que os alunos partilhem suas
experiências e conhecimentos acumulados. Chame a atenção para o
terceiro fragmento e pergunte a eles o que particulariza essa
descrição em contraponto às outras duas.

\item Revele aos alunos a autoria dos fragmentos e peça que analisem o
vocabulário empregado em cada uma das passagens para descrever os
indígenas.

\begin{itemize}
\item
  O que cada uma das passagens sugere?
\item
  Quais são os adjetivos utilizados para descrever os indígenas?
\item
  A partir da observação do terceiro fragmento, o que é possível afirmar
  sobre a representação do indígena brasileiro na contemporaneidade?
\end{itemize}

\item Solicite que os estudantes façam uma pesquisa sobre os povos
tradicionais que hoje habitam o território nacional e quais são os
legados desses povos para a cultura brasileira. Estimule que
pesquisem hábitos, vocabulário, culinária, expressões artísticas e
musicais que se fazem presentes e que relações com as diversas e
múltiplas culturas indígenas podem ser traçadas. É interessante
também estimular os estudantes a compreenderem que, frequentemente,
esses legados são referenciados como se tivessem sido herdados de
uma única e indistinta cultura, quando, na verdade, são frutos de
diferentes povos em diferentes tempos.
\end{enumerate}

\paragraph{Tempo estimado} Duas aulas de 50 minutos.

\subsection{Leitura}

%EM13LP45: Analisar, discutir, produzir e socializar, tendo em vista
%temas e acontecimentos de interesse local ou global, notícias,
%fotodenúncias, fotorreportagens, reportagens multimidiáticas,
%documentários, infográficos, \emph{podcasts} noticiosos, artigos de
%opinião, críticas da mídia, \emph{vlogs} de opinião, textos de
%apresentação e apreciação de produções culturais (resenhas, ensaios
%etc.) e outros gêneros próprios das formas de expressão das culturas
%juvenis (\emph{vlogs} e \emph{podcasts} culturais, \emph{gameplay}
%etc.), em várias mídias, vivenciando de forma significativa o papel de
%repórter, analista, crítico, editorialista ou articulista, leitor,
%vlogueiro e \emph{booktuber}, entre outros.
%EM13LP49: Perceber as peculiaridades estruturais e estilísticas de
%diferentes gêneros literários (a apreensão pessoal do cotidiano nas
%crônicas, a manifestação livre e subjetiva do eu lírico diante do
%mundo nos poemas, a múltipla perspectiva da vida humana e social dos
%romances, a dimensão política e social de textos da literatura
%marginal e da periferia etc.) para experimentar os diferentes ângulos
%de apreensão do indivíduo e do mundo pela literatura.
%EM13LP06: Analisar efeitos de sentido decorrentes de usos expressivos
%da linguagem, da escolha de determinadas palavras ou expressões e da
%ordenação, combinação e contraposição de palavras, dentre outros, para
%ampliar as possibilidades de construção de sentidos e de uso crítico
%da língua.

\paragraph{Tema} Análise formal do enredo e da experiência do protagonista
  
A atividade a seguir foi elaborada para ser feita ao longo da leitura
da obra, que poderá se dar tanto em sala de aula, quanto
individualmente pelos estudantes. Os debates podem ser conduzidos pelo
educador de forma flexível, sempre visando dar conta das problemáticas
apresentadas através da figura do jovem protagonista Paulo e sua
jornada de descoberta e conscientização.

\paragraph{Conteúdo}
A partir do fragmento selecionado da narrativa, orientar os educandos
a desenvolver uma análise formal sobre a experiência de Paulo como
nativo da zona rural e morador da capital.

\paragraph{Objetivo}
Estimular os educandos a descontruir visões aceitas socialmente, com
base em análises de textos diversos.

\paragraph{Metodologia}
\begin{enumerate}
\item Apresente os três fragmentos a seguir:

\begin{quote}
\begin{enumerate}
\item Assim que completou 14, o pai mandou-o para o Recife, estudar e
morar na casa do Tio Teodoro. Mergulhou então num mundo muito
diferente, onde descobriu logo que não havia lugar para essas
personagens que lhe povoavam memória e imaginação. \emph{A primeira
vez em que falou do rio adormecido}, \emph{das almas dos afogados e
da Mãe}-\emph{d}'\emph{água}, \emph{os colegas puseram}-\emph{se a
mangar dele}:

--- Você está falando sério? Mas isso é mesmo um \emph{matuto atrasado
e ignorante}!

Nunca mais abriu a boca para falar de sua terra e de seu rio, mas
sobreviveu, dividido entre a saudade dos assombros de sua infância e
o desafio de igualar-se aos meninos da cidade. Agora, estava ali de
novo, e a força de encantamento do São Francisco fazia reviver a
esperança de deslindar mistérios.\footnote{REZENDE, Maria Valeria. \emph{Rio de sonhos}. Rio de Janeiro: Lamparina, 2021. p.\,10. Grifos nossos.}
\end{enumerate}
\end{quote}

\begin{quote}
\begin{enumerate}
\setcounter{enumii}{1}
\item É o homem permanentemente fatigado. Reflete a preguiça invencível, a
atonia muscular perene, em tudo: na palavra remorada, no gesto
contrafeito, na cadência langorosa das modinhas, na tendência
constante à imobilidade e à inquietude.

Entretanto, toda esta aparência de cansaço ilude.

Nada é mais surpreendedor do que vê-la desaparecer de improviso.
Naquela organização combalida operam-se, em segundos, transmutações
completas. Basta o aparecimento de qualquer incidente exigindo-lhe o
desencadear das energias adormidas. O homem transfigura-se.\footnote{CUNHA, Euclides da. \emph{Os sertões}. Rio de Janeiro: Francisco Alves, 1950. p.\,115.}
\end{enumerate}
\end{quote}

\begin{quote}
\begin{enumerate}
\setcounter{enumii}{2}
\item É só a gente passá\\
Um fio de linha no meio,\\
pra mode enfia as continha,\\
Com jeito pra não erra\ldots{}\\
Quando o fio tivé bem cheio,\\
Cô'as conta tudo juntinha,\\
Também ta prontinho e feito\\
Um rosarinho prefeito\\
De contas de capiá\ldots{}\\
Mas como eu ia contando\\
As conta de capiá\\
Azurzinha, bonitinha,\\
Quando elas fica quietinha,\\
paradinha na varinha,\\
Tão vendo o córgo passá\footnote{BENTO, Nhô {[}OLIVEIRA, José Bento de{]}. \emph{Rosário de capiá}.
São Paulo: Graphicars F. Lanzara, 1946. \emph{Apud} SILVA, Shirley
Cabarite. ``Monteiro Lobato e a linguagem do Jeca Tatu''. In: VII
Congresso Nacional de Linguística e Filologia, 2003, Rio de Janeiro.
\emph{Cadernos do CNLF}, v. VII, 2003. Não paginado.}
\end{enumerate}
\end{quote}


\item Inicie a discussão do primeiro trecho, oriundo de \emph{Rio de
sonhos}, analisando coletivamente com a turma a reação dos colegas
de Paulo às suas histórias. Procure orientar a discussão no sentido
de observar as diferenças entre os hábitos do meio urbano e do modo
de vida da zona rural.

Após essa discussão, apresente o trecho de Euclides da Cunha e
analise com os estudantes como é a representação oferecida pelo
autor do sertanejo, e de que maneira essa figura se relaciona com
Paulo.

Por fim, apresente o poema de Nhô Bento e oriente os alunos a
construírem uma descrição coletiva do dialeto caipira que é
retratado.

Após esse momento inicial, proponha uma discussão sobre as
diferenças entre o espaço delimitado pelo campo ou zona rural e a
faixa litorânea do território brasileiro. Questione o que causou o
aprofundamento dessas distinções e como é retratado, em especial no
poema e no primeiro fragmento, o habitante do interior do Brasil. É
importante salientar que a poesia de Nhô Bento funciona como uma
ilustração de uma variante da própria língua portuguesa que
frequentemente aparece em regiões do Brasil -- sua presença na
atividade ajuda a elaborar um retrato. O foco deve ser o fragmento
da narrativa \emph{Rio de sonhos}.

\item Após essas discussões coletivas, oriente que os alunos formulem um
texto que responda às perguntas abaixo. Esse texto pode servir
também como guia para uma apresentação oral rápida.

\begin{itemize}
\item
  Por que Paulo é chamado especificamente de \emph{matuto}?
\item
  Como é descrito o sertanejo na passagem de Euclides da Cunha? Essa
  descrição vai de encontro ao que é sugerido pelos colegas de Paulo em
  suas ofensas ao rapaz?
\item
  Por que seus colegas desprezam seus conhecimentos e suas vivências? As
  memórias e as narrativas a que Paulo faz referência tem ligação
  especial com o fato de ele ser de origem interiorana?
\item
  Por que a fala de Nhô Bento é retratada dessa forma?
\item
  O que significa, para um brasileiro, nascer no interior do país hoje?
\end{itemize}
\end{enumerate}

\paragraph{Tempo necessário} Duas aulas de 50 minutos.

\subsection{Pós-leitura}

%EM13LP03: Analisar relações de intertextualidade e interdiscursividade
%que permitam a explicitação de relações dialógicas, a identificação de
%posicionamentos ou de perspectivas, a compreensão de paródias e
%estilizações, entre outras possibilidades.
%EM13LP05: Analisar, em textos argumentativos, os posicionamentos
%assumidos, os movimentos argumentativos e os argumentos utilizados
%para sustentá-los, para avaliar sua força e eficácia, e posicionar-se
%diante da questão discutida e/ou dos argumentos utilizados, recorrendo
%aos mecanismos linguísticos necessários.
%EM13LP13: Analisar, a partir de referências contextuais, estéticas e
%culturais, efeitos de sentido decorrentes de escolhas de elementos
%sonoros (volume, timbre, intensidade, pausas, ritmo, efeitos sonoros,
%sincronização etc.) e de suas relações com o verbal, levando-os em
%conta na produção de áudios, para ampliar as possibilidades de
%construção de sentidos e de apreciação.
%EM13LP14: Analisar, a partir de referências contextuais, estéticas e
%culturais, efeitos de sentido decorrentes de escolhas e composição das
%imagens (enquadramento, ângulo/vetor, foco/profundidade de campo,
%iluminação, cor, linhas, formas etc.) e de sua sequenciação
%(disposição e transição, movimentos de câmera, \emph{remix}, entre
%outros), das performances (movimentos do corpo, gestos, ocupação do
%espaço cênico), dos elementos sonoros (entonação, trilha sonora,
%sampleamento etc.) e das relações desses elementos com o verbal,
%levando em conta esses efeitos nas produções de imagens e vídeos, para
%ampliar as possibilidades de construção de sentidos e de apreciação.

\paragraph{Tema} Implicações do real na ficção

Após haver concluído a leitura da obra, é necessário propor uma
atividade que lide com a totalidade da narrativa, contemplando também
a jornada dos personagens. A discussão que normalmente se dá após a
conclusão da leitura faz parte de um importante ciclo compreendido
tanto pelo desenvolvimento das competências necessárias ao falante de
língua portuguesa, quanto pelo papel no letramento literário do
estudante.


\paragraph{Conteúdo}
Atividade de pós-leitura, elaborada para servir simultaneamente como
um exercício criativo, que permita ao aluno refletir sobre o texto que
acabou de ler, ao mesmo tempo em que se apresenta como uma
possibilidade de extrapolar o que é proposto pela própria ficção e
pensar nas implicações reais do que é sugerido na obra. É interessante
introduzir a apresentação dessa proposta com uma discussão sobre o fim
do livro, sobre a necessidade da preservação ambiental e quais os
meios que os estudantes julgam necessários para que isso se dê.
Proponha que os alunos pesquisem ainda mais sobre a região para
embasar seus textos.

\paragraph{Objetivos}
Sensibilizar os alunos para a necessidade da preservação ambiental e
também evidenciar a necessidade de proteger as comunidades
tradicionais presentes no território nacional. Essa atividade também
tem como objetivo fazer com que os alunos exercitem suas habilidades
argumentativas a partir de uma narrativa ficcional.

\paragraph{Metodologia}
Argumentação a partir da ficção.

\begin{quote}
Queria uma profissão que lhe permitisse explorar e refazer a história
daquela região na qual tinha suas raízes e, agora, seu grande amor.
Havia de ser \emph{geógrafo}, \emph{ou historiador}, \emph{ou
arqueólogo}, \emph{agrônomo ou ambientalista}, \emph{ou tudo isso
junto}\ldots{} Teria de pensar e tomar informações para poder saber
por onde começar esse novo caminho.\footnote{REZENDE, Maria Valeria. \emph{Rio de sonhos}. Rio de Janeiro: Lamparina, 2021. p. 54. Grifos nossos.}
\end{quote}

Após haver lido e debatido o livro em sala de aula, proponha aos
estudantes a seguinte produção textual. Imagine que o jovem Paulo,
imbuído de seu desejo de salvar as comunidades e tudo aquilo que teve
oportunidade de conhecer ao lado de Iara, acaba escolhendo alguma das
profissões citadas acima, cuja prática permitiria que fosse capaz de
interceder e defender os povos tradicionais, suas práticas, cultivos
e, principalmente, toda uma natureza que infelizmente não pode falar
por si mesma.

Assim, formado, Paulo decide começar a escrever artigos para a
imprensa local, denunciando o que a eventual inundação do território
para a construção da hidroelétrica poderia causar. Proponha aos
estudantes que se imaginem como o jovem protagonista da obra, querendo
defender não apenas os recursos naturais, as riquezas arqueológicas,
como também o lar dos povos tradicionais. Estimule que os alunos sejam
criativos em suas escolhas e que pesquisem as profissões que achem que
mais se relacionam com o que interpretaram do personagem, valorizando
também a relação que essa atividade pode ter com o desenvolvimento do
projeto de vida pessoal de cada um deles individualmente.

\paragraph{Tempo estimado} Duas aulas de 50 minutos.

\section{Atividades 2}

\subsection{Pré-leitura}

%EM13LP26: Relacionar textos e documentos legais e normativos de âmbito
%universal, nacional, local ou escolar que envolvam a definição de
%direitos e deveres -- em especial, os voltados a adolescentes e jovens
%-- aos seus contextos de produção, identificando ou inferindo
%possíveis motivações e finalidades, como forma de ampliar a
%compreensão desses direitos e deveres.
%EM13LP25: Participar de reuniões na escola (conselho de escola e de
%classe, grêmio livre etc.), agremiações, coletivos ou movimentos,
%entre outros, em debates, assembleias, fóruns de discussão etc.,
%exercitando a escuta atenta, respeitando seu turno e tempo de fala,
%posicionando-se de forma fundamentada, respeitosa e ética diante da
%apresentação de propostas e defesas de opiniões, usando estratégias
%linguísticas típicas de negociação e de apoio e/ou de consideração do
%discurso do outro (como solicitar esclarecimento, detalhamento, fazer
%referência direta ou retomar a fala do outro, parafraseando-a para
%endossá-la, enfatizá-la, complementá-la ou enfraquecê-la),
%considerando propostas alternativas e reformulando seu posicionamento,
%quando for caso, com vistas ao entendimento e ao bem comum.
%EM13LP41: Analisar os processos humanos e automáticos de curadoria que
%operam nas redes sociais e outros domínios da internet, comparando os
%\emph{feeds} de diferentes páginas de redes sociais e discutindo os
%efeitos desses modelos de curadoria, de forma a ampliar as
%possibilidades de trato com o diferente e minimizar o efeito bolha e a
%manipulação de terceiros.
%EM13CNT306: Avaliar os riscos envolvidos em atividades cotidianas,
%aplicando conhecimentos das Ciências da Natureza, para justificar o
%uso de equipamentos e recursos, bem como comportamentos de segurança,
%visando à integridade física, individual e coletiva, e socioambiental,
%podendo fazer uso de dispositivos e aplicativos digitais que
%viabilizem a estruturação de simulações de tais riscos.
%EM13CNT309: Analisar questões socioambientais, políticas e econômicas
%relativas à dependência do mundo atual em relação aos recursos não
%renováveis e discutir a necessidade de introdução de alternativas e
%novas tecnologias energéticas e de materiais, comparando diferentes
%tipos de motores e processos de produção de novos materiais.

\paragraph{Tema} Progresso e preservação ambiental

Como um dos pontos mais importantes da obra a ser lida é justamente a
construção de uma hidroelétrica no rio São Francisco e como isso
afetaria diretamente a vida das comunidades tradicionais que vivem à
sua margem, é importante começar familiarizando os alunos com algumas
questões que ainda hoje pautam as discussões sobre progresso e
preservação ambiental. As atividades aqui propostas pretendem
mobilizar competências múltiplas, tanto do campo de Língua Portuguesa
como dos campos de Ciências da Natureza e de Ciências Humanas.

\paragraph{Conteúdo}
A atividade propõe familiarizar o aluno com a questão que será
trabalhada pela obra, através da discussão sobre a construção da Usina
de Belo Monte.

\paragraph{Objetivos}
Trabalhar com os alunos questões relativas à preservação ambiental,
pensando sempre sobre quais as necessidades geradas pelo estilo de
vida contemporâneo, problematizando o uso de energia e seus impactos
que, muitas vezes, são invisíveis para os moradores dos centros
urbanos e irreversíveis para os moradores do interior.

\paragraph{Metodologia}
  \begin{enumerate}
\item Leia juntamente com os estudantes a matéria abaixo, publicada pela
    \emph{Agência Pública} em dezembro de 2019:

\begin{quote}
\textbf{No Xingu, finalização da última turbina de Belo Monte pode
significar ``o fim do rio''}

\emph{Com os impactos da hidroelétrica}, \emph{pescadores recorrem
ao Bolsa Família e a} ``\emph{bicos}''; \emph{vídeos mostram
pescadores arrastando barcos onde antes havia água em abundância.}

Com os custos estimados em mais de R\$ 40 bilhões, a Usina
Hidroelétrica de Belo Monte (UHE Belo Monte) teve a última turbina
inaugurada no último dia 27 de novembro, em cerimônia que contou com
a presença do presidente Jair Bolsonaro.

Desde que a usina barrou um pedaço do rio Xingu, no final de 2015,
mais de 200 famílias de pescadores que habitam o trecho de 100
quilômetros entre as cidades de Altamira, Anapu, Senador José
Porfírio e Vitória do Xingu viram sua fonte de renda e de
alimentação diminuir.

``Quando a gente chegou aqui, pra pegar um peixe, bastava uma vara,
uma malhadeira e já podia dizer: `Vou ali buscar um peixe pra gente
almoçar.' Era rapidinho, pegava até 100 quilos. Hoje, você passa o
dia todinho e não pega um peixe, talvez durante a noite você consiga
pegar um'', conta o pescador Francisco Fernandes da Silva, de 57
anos. Morador da Volta Grande do Xingu há 18 anos, ele reclama que
os impactos provocados por Belo Monte desde o início da obra se
agravaram com o barramento definitivo do rio.

{[}\ldots{}{]}

Um estudo realizado entre 2014 e 2017 constatou que a produção total
do pacu -- espécie mais comum da região -- caiu nos meses de pico da
pesca (janeiro e fevereiro) de 929,8 kg em 2015, para 396,3 kg no
primeiro ano de barramento, em 2016, situação também influenciada
pelo impacto climático do El Niño.

Para a usina funcionar, foi construída uma barragem principal no rio
Xingu, que desviou a água necessária na geração de energia para dois
reservatórios -- com isso, criou-se um ``trecho de vazão reduzida''.
Nesse trecho, onde está localizada a Volta Grande, a quantidade de
água que passa -- a vazão -- não é mais a natural do rio, mas a
artificialmente liberada pela concessionária de Belo Monte, a Norte
Energia S. A.

Com a inauguração da última turbina, Belo Monte agora opera com sua
capacidade máxima, e a Norte Energia passará a adotar o chamado
``hidrograma de consenso'', que estabelece valores mínimos de água
que devem chegar ao trecho de vazão reduzida ao longo de cada mês do
ano. A ser testada por seis anos, a partir de janeiro de 2020, a
medida tem como objetivo mitigar os danos provocados pelo desvio de
água para a geração de energia.

Criticado por especialistas por causa de aspectos ambientais, o
``hidrograma de consenso'' não é suficiente sequer para garantir a
viabilidade energética da usina ao longo de todo o ano. Segundo
reportagem publicada do \emph{O Estado de S}. \emph{Paulo} na última
sexta-feira, a Norte Energia quer erguer uma usina térmica para
compensar a baixa produção de energia e já consultou a Agência
Nacional de Energia Elétrica (Aneel) sobre a possibilidade.
{[}\ldots{}{]}\footnote{OLIVERIA, Rafael. ``No Xingu, finalização da última turbina de Belo
Monte pode significar `o fim do rio'''. \emph{Agência Pública}, 16
de dezembro de 2019. Disponível em
\href{https://apublica.org/2019/12/no-xingu-finalizacao-da-ultima-turbina-de-belo-monte-pode-significar-o-fim-do-rio/}{apublica.org},
acesso em 20 de fevereiro de 2021.}
\end{quote}

\item Após a leitura da atividade, conduza um debate com a turma sobre o
impacto das atividades produtoras de energia, em especial as
hidroelétricas, uma vez que essas são as mais comuns no território
nacional. É importante também fazer com que os estudantes se
conscientizem, nesse primeiro momento, do que significa em larga
escala esse processo e quais são os impactos de sua presença, tanto
em suas vidas individualmente, quanto no progresso do país.

\begin{itemize}
\item
  Em larga escala, qual é a relação da energia elétrica com o
  desenvolvimento?
\item
  O que significa a diminuição da pesca e alteração do volume de água no
  rio Xingu?
\item
  É possível conciliar as necessidades impostas pelo desenvolvimento com
  o bem-estar das populações ribeirinhas ou tradicionais?
\item
  Qual é o impacto causado pela construção de uma usina hidroelétrica de
  grande porte e de que maneira isso afeta a interação populacional das
  espécies da região?
\end{itemize}

\item Proponha que os estudantes organizem o resultado de suas
considerações em forma de apresentação oral, em duplas ou trios, de
modo que todos tenham algum tempo de se dirigir ao grupo como um
todo. É importante evidenciar também como os conhecimentos de outras
disciplinas pertencentes ao grupo de Ciências da Natureza podem
ajudar a refletir sobre as necessidades sociais geradas pelo
progresso e a preservação, que se faz cada dia mais urgente.
Estimule os grupos a pensarem em soluções criativas e faça com que
eles exponham suas ideias e que se interessem pelas propostas dos
colegas.
  \end{enumerate}

\paragraph{Tempo estimado} Três aulas de 50 minutos.

\subsection{Leitura}

%EM13CHS103: Elaborar hipóteses, selecionar evidências e compor
%argumentos relativos a processos políticos, econômicos, sociais,
%ambientais, culturais e epistemológicos, com base na sistematização de
%dados e informações de diversas naturezas (expressões artísticas,
%textos filosóficos e sociológicos, documentos históricos e
%geográficos, gráficos, mapas, tabelas, tradições orais, entre outros).
%EM13CHS104: Analisar objetos da cultura material e imaterial como
%suporte de conhecimentos, valores, crenças e práticas que singularizam
%diferentes sociedades inseridas no tempo e no espaço.
%EM13CHS304: Analisar os impactos socioambientais decorrentes de
%práticas de instituições governamentais, de empresas e de indivíduos,
%discutindo as origens dessas práticas, e selecionar aquelas que
%respeitem e promovam a consciência e a ética socioambiental e o
%consumo responsável. práticas, e selecionar aquelas que respeitem e
%promovam a consciência e a ética socioambiental e o consumo
%responsável.
%EM13LP27: Engajar-se na busca de solução para problemas que envolvam a
%coletividade, denunciando o desrespeito a direitos, organizando e/ou
%participando de discussões, campanhas e debates, produzindo textos
%reivindicatórios, normativos, entre outras possibilidades, como forma
%de fomentar os princípios democráticos e uma atuação pautada pela
%ética da responsabilidade, pelo consumo consciente e pela consciência
%socioambiental.
%EM13LP29: Resumir e resenhar textos, por meio do uso de paráfrases, de
%marcas do discurso reportado e de citações, para uso em textos de
%divulgação de estudos e pesquisas.

\paragraph{Tema} Pinturas rupestres e a arte no Ocidente

Uma das propostas a ser explorada nesse ponto, no que diz respeito à
abordagem interdisciplinar, é a riqueza arqueológica presente na
região. Na obra, além dos conhecimentos representados pelos povos
tradicionais, temos a representação de um encontro dos jovens com uma
pintura rupestre.

\paragraph{Conteúdo}
Apresentar a arte rupestre e abrir caminho para uma pesquisa sobre a
história da arte no Ocidente e permitir não apenas o acesso às formas
artísticas que fazem parte da nossa história enquanto território, como
aquelas que fazem parte da história mundial.

\paragraph{Objetivos}
Familiarizar o aluno com expressões artísticas, com a História da
Arte, e conscientizar os estudantes sobre a necessidade da fruição
artística. É importante também evidenciar, nesse contato, sobre a
necessidade da expressão artística e sua relevância na cultura, não só
do Brasil, como do mundo.

\paragraph{Metodologia}
Releia com os estudantes o fragmento abaixo:

\begin{quote}
  Galoparam até o Letreiro do Sobrado. Encontraram sem dificuldade,
  quase à beira do rio, o pequeno abrigo naturalmente escavado na pedra,
  cujas \emph{paredes eram cobertas de pinturas e gravações rupestres},
  \emph{belas e misteriosas}, \emph{uma verdadeira escrita de milhares
  de anos atrás}, que só Vô Batista acreditava poder ler. Os jovens
  tinham a esperança de encontrá-lo lá, tentando decifrar mensagens dos
  antepassados sobre a crise que o rio e seu povo estavam vivendo agora.

  Ambos conheciam bem o local. Paulo o descobrira trazido por seu pai,
  que adorava arqueologia, e tinha voltado muitas vezes por conta
  própria, nas suas andanças de adolescente sonhador. Iara tinha vindo
  ali muitas vezes, na garupa de Telegrama, acompanhando seu avô,
  atraídos ambos pelos segredos escritos naquelas paredes.

  Com emoção, desmontaram e correram para a pequena gruta, cheios de
  esperança. Seus olhos custaram alguns segundos a acostumar-se à pouca
  luz do local, até que puderam ver que estava completamente vazio, a
  não ser pelos misteriosos sinais nas paredes.

  Ficaram ambos um tempo parados e silenciosos, em parte pela decepção,
  mas também porque outros sentimentos iam tomando conta de cada um.

  Paulo lembrava que a beleza do que vira ali, na primeira visita,
  despertara nele o desejo de ser artista. Algumas vezes tinha voltado
  sozinho, com seus cadernos e lápis de cor, para inspirar-se.\footnote{REZENDE, Maria Valeria. \emph{Rio de sonhos}. Rio de Janeiro: Lamparina, 2021. p.\,23--24. Grifos nossos.}
\end{quote}

Proponha um debate com a turma sobre as pinturas rupestres,
  apresentando o conceito, situando historicamente a expressão artística
  e sensibilizando a turma para a importância histórica, social e
  arqueológica dessas pinturas.

  Apesar de as pinturas rupestres mais famosas do Brasil estarem no
  Piauí, a narrativa faz referência a uma região onde podem ser
  encontradas pinturas e expressões, hoje consideradas artísticas, que
  datam de 3.000 a 10.000 anos atrás.

  Estimule a turma a refletir, primeiro de maneira individual, depois
  coletivamente, sobre o que significam essas pinturas e como elas
  servem para que entremos em contato com o cotidiano dos indivíduos que
  viveram nessas épocas, e como eles retratavam seus cotidianos, já que,
  muitas vezes, essas imagens eram representações de caças, animais,
  objetos domésticos ou mesmo de trabalhos que então eram feitos. A
  relação do ser humano com a expressão artística -- que engloba a
  literatura, o cinema e todas as formas de arte que hoje conhecemos --
  começou com essas imagens pintadas com os dedos e obtidas a partir de
  carvão ou mesmo cera de abelha.

Observe a imagem abaixo:

%  PNLD0051-03.png, ``Reprodução do teto da caverna de Altamira''

  A imagem é de uma pintura rupestre famosa, da caverna de Altamira, na
  Espanha. Observe como os traços são semelhantes aos da pintura
  encontrada na caverna brasileira. As cores, e mesmo o fato de a
  pintura representar um animal remetem a um universo mais objetivo.
  Será que as razões que levaram esses sujeitos pré-históricos a pintar
  são as mesmas que ainda hoje levam os pintores contemporâneos a
  produzir arte?

  É interessante discutir com a turma as possibilidades representadas
  pelas artes visuais, e seu papel indispensável na formação da relação
  do indivíduo com o mundo e como a representação foi, ao longo dos
  séculos, uma ferramenta importante na construção de subjetividade, ou
  seja, na complexa arquitetura pessoal do eu.

Agora, veja a série \emph{O touro} {[}\emph{Le taureu}{]} de
  litografias de Pablo Picasso, especialmente o \emph{Estado III}, de
  1945 (Moma, Nova York. Disponível em:
  \href{https://www.moma.org/collection/works/62951}{moma.org},
  acesso em 20 de fevereiro de 2021), \emph{Estado IV}, de 1945 (Moma,
  Nova York. Disponível em:
  \href{https://www.moma.org/collection/works/62968}{moma.org},
  acesso em 20 de fevereiro de 2021), \emph{Estado VII}, de 1945 (Moma,
  Nova York. Disponível em:
  \href{https://www.moma.org/collection/works/62986}{moma.org},
  acesso em 20 de fevereiro de 2021), \emph{Estado XI}, de 1946 (Moma,
  Nova York. Disponível em:
  \href{https://www.moma.org/collection/works/63046}{moma.org},
  acesso em 20 de fevereiro de 2021) e o \emph{Estado XIV}, de 1946
  (Moma, Nova York. Disponível em:
  \href{https://www.moma.org/collection/works/65356}{moma.org},
  acesso em 20 de fevereiro de 2021). Essas obras, de autoria do pintor
  moderno espanhol, claramente se inspiram nos traços rupestres
  encontrados na caverna. Quais são as diferenças, no entanto? Por que o
  pintor se inspiraria nesse tipo de expressão tão antiga e
  aparentemente primitiva, uma vez que já estava pintando no século XX?

  Estimule que os estudantes pesquisem sobre questões relativas à arte e
  sua história com base nos questionamentos que emergirem durante o
  debate conduzido em sala. É interessante também perguntar o que eles
  acharam das imagens, o que preferiram, se apreciaram mais uma ou
  outra, e por quê. Faça com que fundamentem suas respostas e dê espaço
  para que falem de suas predileções estéticas, valorizando o que
  disserem, e sempre mostrando a importância que a arte tem.

  Ao fim, proponha que pesquisem sobre arte rupestre, sobre o cubismo de
  Pablo Picasso e outros pintores que foram representantes desse estilo,
  e mesmo sobre questões que possam surgir durante a atividade. Para que
  o gosto pela fruição artística seja construído, é necessário que o
  estudante veja que seu olhar sobre uma obra importa, na mesma medida
  em que é importante conhecer sobre os movimentos estéticos e o
  contexto em que determinadas obras foram produzidas.

  Proponha que o resultado da pesquisa seja transformado em uma produção
  textual.

\paragraph{Tempo estimado} Duas aulas de 50 minutos.

\subsection{Pós-leitura}


%EM13LP15: Planejar, produzir, revisar, editar, reescrever e avaliar
%textos escritos e multissemióticos, considerando sua adequação às
%condições de produção do texto, no que diz respeito ao lugar social a
%ser assumido e à imagem que se pretende passar a respeito de si mesmo,
%ao leitor pretendido, ao veículo e mídia em que o texto ou produção
%cultural vai circular, ao contexto imediato e sócio-histórico mais
%geral, ao gênero textual em questão e suas regularidades, à variedade
%linguística apropriada a esse contexto e ao uso do conhecimento dos
%aspectos notacionais (ortografia padrão, pontuação adequada,
%mecanismos de concordância nominal e verbal, regência verbal etc.),
%sempre que o contexto o exigir.
%EM13LP16: Produzir e analisar textos orais, considerando sua adequação
%aos contextos de produção, à forma composicional e ao estilo do gênero
%em questão, à clareza, à progressão temática e à variedade linguística
%empregada, como também aos elementos relacionados à fala (modulação de
%voz, entonação, ritmo, altura e intensidade, respiração etc.) e à
%cinestesia (postura corporal, movimentos e gestualidade significativa,
%expressão facial, contato de olho com plateia etc.).
%EM13LP26: Relacionar textos e documentos legais e normativos de âmbito
%universal, nacional, local ou escolar que envolvam a definição de
%direitos e deveres -- em especial, os voltados a adolescentes e jovens
%-- aos seus contextos de produção, identificando ou inferindo
%possíveis motivações e finalidades, como forma de ampliar a
%compreensão desses direitos e deveres.
%EM13CNT206: Discutir a importância da preservação e conservação da
%biodiversidade, considerando parâmetros qualitativos e quantitativos,
%e avaliar os efeitos da ação humana e das políticas ambientais para a
%garantia da sustentabilidade do planeta.
%EM13CNT301: Construir questões, elaborar hipóteses, previsões e
%estimativas, empregar instrumentos de medição e representar e
%interpretar modelos explicativos, dados e/ou resultados experimentais
%para construir, avaliar e justificar conclusões no enfrentamento de
%situações-problema sob uma perspectiva científica.
%EM13CHS504: Analisar e avaliar os impasses ético-políticos decorrentes
%das transformações científicas e tecnológicas no mundo contemporâneo e
%seus desdobramentos nas atitudes e nos valores de indivíduos, grupos
%sociais, sociedades e culturas.
%EM13CHS605: Analisar os princípios da declaração dos Direitos Humanos,
%recorrendo às noções de justiça, igualdade e fraternidade, identificar
%os progressos e entraves à concretização desses direitos nas diversas
%sociedades contemporâneas e promover ações concretas diante da
%desigualdade e das violações desses direitos em diferentes espaços de
%vivência, respeitando a identidade de cada grupo e de cada indivíduo.

\paragraph{Tema} Gravação de \emph{podcasdt} com os temas desenvolvidos.  

Após haver concluído a leitura da obra, uma das questões
interdisciplinares que mais saltam aos olhos diz respeito à
preservação, tanto de povos e comunidades, como de recursos naturais e
arqueológicos. Unindo os conhecimentos legados por outras áreas, como
é possível discutir preservação de maneira realista, é a pergunta que
os alunos devem se fazer nesse ponto.

\paragraph{Conteúdo}
  A atividade consiste em orientar os alunos a gravar um \emph{podcast}
  sobre preservação ambiental e questões relativas à proteção dos povos
  tradicionais e populações ribeirinhas, que podem ter suas existências
  e cultivos afetados pelo alagamento de territórios.

\paragraph{Objetivos}
  Conscientizar os estudantes das urgências ambientais e das inegáveis
  exigências feitas em nome do progresso, sempre questionando destacando
  que, todas as decisões tomadas em larga escala pelo governo e por
  instâncias governamentais acabam afetando enormes quantitativos de
  cidadãos. Os estudantes devem ser orientados a pensar sobre inclusão e
  exclusão. Nesse sentido, uma vez que, se todos são cidadãos, por que
  alguns podem ser diretamente afetados pelas políticas públicas, em
  prol de um suposto progresso?

\paragraph{Metodologia}
  Após concluir a leitura e debater o livro, a atividade propõe que os
  alunos reúnam suas reflexões em forma de \emph{podcast}. Programa de
  áudio semelhante aos programas de rádio, o \emph{podcast} se
  caracteriza por sintetizar informações que, muitas vezes, não estão
  tão evidentes ou acessíveis para o grande público e apresentá-las
  normalmente no formato mesa-redonda, em que vários participantes
  conversam sobre um determinado tópico. Os estudantes podem ser
  divididos em grupos de até 10 participantes, sendo que as funções a
  serem executadas devem englobar todas as fases da elaboração do
  programa, desde a pesquisa até o desenvolvimento do roteiro que será
  seguido pelos participantes até a edição. O professor deve
  supervisionar essas atividades e contar com o auxílio da estrutura
  escolar, incluindo colegas de outras áreas que possam ajudar a
  enriquecer o debate e tirar eventuais dúvidas.

  Após concluídas as etapas, os programas podem ser ouvidos em sala,
  seguidos de debates e depois compartilhados com a comunidade escolar,
  para ampliar a penetração das questões nessa esfera.

  \section{Aprofundamento}

  \emph{Rio de sonhos} é uma obra juvenil. Isso quer dizer que é uma
  obra destinada ao público jovem, pensada para dialogar diretamente com
  seus anseios e questões, situando esses em um lugar de importância e
  destaque. Ler literatura de qualidade na escola pode colaborar
  diretamente na formação de leitores, que é sempre um dos pontos
  cardinais da atividade leitora em sala de aula.

  Muito embora a obra se passe no ano de 1988 e dê conta de
  acontecimentos fictícios, a narrativa compreendida pelas aventuras de
  Paulo e Iara ao longo do rio São Francisco é verossímil e conta um
  fragmento de história do país que, recentemente, voltou a ser pauta
  nos principais jornais e revistas, justamente com a construção da
  usina hidroelétrica de Belo Monte. A novela, porém, dá uma dimensão do
  que é o microcosmo afetado por uma construção largamente anunciada
  como feita em nome do progresso.

  Em determinado momento da narrativa, os jovens buscam abrigo em uma
  aldeia pankararu e conhecem Juca, filho mais novo dos avós de Iara. A
  página ``Pankararu'' (ISA, \emph{Povos Indígenas no Brasil.}
  Disponível em:
  \href{https://www.indios.org.br/pt/Povo:Pankararu}{indios.org.br},
  acesso em 20 de fevereiro de 2021) reúne textos, fotografias, mapas e
  notícias sobre a realidade desse povo e seus territórios. Atualmente
  mantido pelo Instituto Socioambiental (ISA), o \emph{site}
  \href{https://www.indios.org.br/}{indios.org.br}
  foi criado pelo Cedi (Centro Ecumênico de Documentação e Informação)
  com o propósito de reunir verbetes com informações e análises de todos
  os povos indígenas que habitam o território nacional. Durante a noite
  que os jovens passam na aldeia, o povo se reune na igreja e o pajé
  anima a oração. O artista Aislan Santos, da etnia pankararu, registra
  seu olhar sobre tradições e culturas das aldeias nas obras de sua
  exposição ``Yeposanóng'' no Memorial dos Povos Indígenas, em Brasília
  (CASTRO, Milena. ``Rituais da etnia Pankararu são retratados em mostra
  no Memorial dos Povos Indígenas, em Brasília'', \emph{G1}, 26 de
  fevereiro de 2021. Disponível em:
 \href{https://g1.globo.com/google/amp/df/distrito-federal/o-que-fazer-no-distrito-federal/noticia/2021/02/26/rituais-da-etnia-pankararu-sao-retratados-em-mostra-no-memorial-dos-povos-indigenas-em-brasilia.ghtml}{g1.globo.com},
  acesso em 27 de fevereiro de 2021).

  O livro ainda conta com a possibilidade de trabalhar referências
  clássicas da literatura brasileira, uma vez que o jovem Paulo é
  retratado como um leitor e chega mesmo a evocar Castro Alves. A
  intertextualidade presente também colabora para um acesso mais
  orgânico a outras obras clássicas, possibilitando versatilidade e
  criatividade na abordagem.

  Um outro ponto que deve ser mencionado é a maneira como a narrativa é
  conduzida. Por se tratar de uma novela, gênero muito mais popular na
  América hispânica do que no Brasil, o que por si só já permite um
  debate interessante e produtivo sobre os gêneros literários e as
  fronteiras geográficas, a história contada em \emph{Rio de sonhos}
  consegue passar ao leitor a sensação de que, apesar de os personagens
  irem cruzando o caminho de Paulo e Iara e saindo de cena, a sensação é
  de que os jovens se movem num cenário vivo, em que os acontecimentos
  continuam tomando seus rumos após a passagem deles.

  O trabalho com gêneros pode render muitos frutos no que diz respeito à
  elaboração de categorias distintivas que visam elencar
  particularidades de cada tipo de texto e qual seu contexto adequado.
  Já que podemos afirmar que um dos objetivos principais da
  escolarização, no que tange ao ensino de Língua Portuguesa e
  Literatura, é transformar o estudante em ``poliglota de seu próprio
  idioma'', para retomar aqui as palavras de Evanildo Bechara, ou seja,
  fazer com que o aluno domine com tranquilidade seu próprio idioma e
  saiba diferenciar seus contextos de produção, não apenas dominando a
  norma culta e sua variante falada. \emph{Rio de sonhos} também
  comporta a possibilidade de trabalho através de mitos e lendas
  indígenas, já que a cena em que Paulo resgata Iara é repleta de
  referências dessa natureza.

  O \emph{site} da autora
  \href{https://www.mariavaleriarezende.com/}{mariavaleriarezende.com}
  reúne informações, resenhas, entrevistas, vídeos, e notícias sobre
  Maria Valéria Rezende e destaca suas obras premiadas -- romances,
  contos, crônicas, ensaios, coletâneas, obras infantis e juvenis.

  \section{Atividades para ir além}

  Como \emph{Rio de sonhos} é uma obra que convida a um mergulho mais
  intenso na história e geografia da região do rio São Francisco e suas
  riquezas, vamos propor mais duas atividades pós-leitura para os
  estudantes, uma tratando questões de língua portuguesa e linguagem, e
  uma outra sobre a geografia física da região. Elas podem ser feitas
  após o término das atividades propostas ou em outra ocasião durante o
  trabalho em sala de aula com a obra. Pensadas para enriquecer a
  proposta interdisciplinar com o texto também, as duas podem ser
  trabalhadas tanto pelo professor de Língua Portuguesa, quanto pelo
  professor de outras disciplinas, e mesmo fazer parte de um trabalho
  duplo coordenado por dois ou mais professores, que tome lugar em
  distintas aulas.

  A literatura, mais do que uma ferramenta para o ensino da língua ou
  alicerce da cultura nacional, é, sem dúvida, uma expressão artística
  na qual convivem vários saberes, e nada melhor que sua leitura deixe
  também evidente para o estudante essa multiplicidade.

\subsection{Atividade I}

%EM13LP02: Estabelecer relações entre as partes do texto, tanto na
%produção como na recepção, considerando a construção composicional e o
%estilo do gênero, usando/reconhecendo adequadamente elementos e
%recursos coesivos diversos que contribuam para a coerência, a
%continuidade do texto e sua progressão temática, e organizando
%informações, tendo em vista as condições de produção e as relações
%lógico-discursivas envolvidas (causa/efeito ou consequência;
%tese/argumentos; problema/solução; definição/exemplos etc.).
%EM13LP06: Analisar efeitos de sentido decorrentes de usos expressivos
%da linguagem, da escolha de determinadas palavras ou expressões e da
%ordenação, combinação e contraposição de palavras, dentre outros, para
%ampliar as possibilidades de construção de sentidos e de uso crítico
%da língua.
%EM13LP08: Analisar elementos e aspectos da sintaxe do Português, como
%a ordem dos constituintes da sentença (e os efeitos que causam sua
%inversão), a estrutura dos sintagmas, as categorias sintáticas, os
%processos de coordenação e subordinação (e os efeitos de seus usos) e
%a sintaxe de concordância e de regência, de modo a potencializar os
%processos de compreensão e produção de textos e a possibilitar
%escolhas adequadas à situação comunicativa.

\paragraph{Objetivo}
  Mostrar as possibilidades de trabalho amplas sugeridas pela obra,
  tanto na direção de uma abordagem de análise gramatical, como será
  feito a seguir na primeira atividade, quanto na análise das
  consequências ambientais causadas pelo alagamento artificial de uma
  hidroelétrica. Ambas as atividades foram pensadas para servir como
  modelo para novas atividades e inspirar tanto os docentes quanto os
  estudantes a irem além.

\paragraph{Conteúdo}
  Análise dos elementos gramaticais e exercício de reflexão sobre a o
  processo de referenciação empregado na forma escrita da norma culta da
  Língua Portuguesa e subsequente análise dos impactos ambientais no
  semiárido após a construção de uma barragem.

\paragraph{Metodologia}
\begin{enumerate}
\item Releia com os estudantes o fragmento abaixo:

\begin{quote}
\emph{Descalçou} os tênis, \emph{amarrou}-os na cintura e
\emph{começou} a descer o paredão do cânion, \emph{agarrando}-se às
reentrâncias que a lua mostrava, até junto da correnteza.
\emph{Encontrou} a pedra saliente que procurava, \emph{sentou}-se e
\emph{apoiou} as costas na parede de rocha, \emph{sentindo} a água
respingar-lhe as pernas nuas. \emph{Sentindo} a solidão daquele
lugar, ficou imóvel e \emph{fechou} os olhos, para retomar o fôlego
e acalmar as batidas do coração.\footnote{REZENDE, Maria Valeria. \emph{Rio de sonhos}. Rio de Janeiro: Lamparina, 2021. p.\,8. Grifos nossos.}
\end{quote}

\item Pergunte aos alunos quem é o sujeito dos verbos destacados acima.

\item Depois observe a colocação dos pronomes oblíquos e pergunte a quem se referem.

\item Após ouvir as respostas, compartilhe o gabarito e discuta com eles a natureza do que está acontecendo no trecho.

\item Apesar de o fragmento acima não possuir um sujeito explícito, sabemos
que Paulo é agente de todas as ações citadas. Analise com os
estudantes os efeitos que a elipse do sujeito confere ao fragmento e a
função exercida pelos pronomes oblíquos na coesão textual. É
interessante debater que, por mais que sujeito possa parecer uma
categoria inquestionável no que diz respeito à sintaxe da língua, sua
classificação abre brechas para uma discussão sobre a sintaxe e a
semântica das palavras, já que é muito comum que ele seja definido
como ``aquele que pratica a ação''. Já os oblíquos vão funcionar como
complementos verbais, diretos ou indiretos, e todos vão, de algum
modo, retomar o sujeito. É interessante pautar esse debate em sala de
aula, pois o conhecimento linguístico intuitivo muitas vezes acaba
impossibilitando uma análise mais detalhada dos fenômenos da língua,
indispensável para uma formação completa dos estudantes. É possível
tornar esse tipo de atividade, quando baseada em uma leitura feita em
sala ou mesmo supervisionada pelo educador, em uma reflexão profunda
sobre os usos da língua e seus processos de referenciação, tão
importantes para a articulação do discurso, tanto na escrita quanto na
fala.
\end{enumerate}

\subsection{Atividade II}

%EM13CNT203: Avaliar e prever efeitos de intervenções nos ecossistemas,
%e seus impactos nos seres vivos e no corpo humano, com base nos
%mecanismos de manutenção da vida, nos ciclos da matéria e nas
%transformações e transferências de energia, utilizando representações
%e simulações sobre tais fatores, com ou sem o uso de dispositivos e
%aplicativos digitais (como \emph{softwares} de simulação e de
%realidade virtual, entre outros).
%EM13CNT304: Analisar e debater situações controversas sobre a
%aplicação de conhecimentos da área de Ciências da Natureza (tais como
%tecnologias do DNA, tratamentos com células-tronco, produção de
%armamentos, formas de controle de pragas, entre outros), com base em
%argumentos consistentes, éticos e responsáveis, distinguindo
%diferentes pontos de vista.
%EM13CNT310: Investigar e analisar os efeitos de programas de
%infraestrutura e demais serviços básicos (saneamento, energia
%elétrica, transporte, telecomunicações, cobertura vacinal, atendimento
%primário à saúde e produção de alimentos, entre outros) e identificar
%necessidades locais e/ou regionais em relação a esses serviços, a fim
%de promover ações que contribuam para a melhoria na qualidade de vida
%e nas condições de saúde da população.

\paragraph{Conteúdo}
  O episódio narrado na obra \emph{Rio de sonhos} compreende uma jornada
  que tem duração de um dia inteiro às margens do rio São Francisco.
  Inspirada por acontecimentos reais à época da construção da
  hidroelétrica Luiz Gonzaga, que alagou Petrolândia, município que hoje
  se encontra debaixo d'água. A obra de Maria Valéria Rezende leva o
  leitor, necessáriamente, a se questionar sobre os impactos ambientais
  de tais acontecimentos. Por mais que hoje saibamos que a região foi,
  tanto social quanto economicamente beneficiada pela construção da
  usina, o que significou para a natureza da região? A região em questão
  possui clima semiárido e vegetação característica da caatinga
  hiperxerófila, formada por espécies vegetais com alta capacidade de
  retenção de água. Durante a estação mais quente do ano, suas folhas
  caem e o seu metabolismo vegetal fica bastante reduzido, de modo a
  facilitar sua sobrevivência.

\paragraph{Metodologia}
\begin{enumerate}
\item Discuta sobre as questões geográficas da região com os estudantes,
chamando atenção para as características do clima e do solo da
região, formado pela bacia sedimentar do Jatobá e oriente-os a
pesquisar sobre as questões geológicas da região também.

\item Peça que respondam às questões abaixo em forma de produção textual.
\begin{itemize}
\item
  O alagamento da região causa que tipo de consequência ambiental?
\item
  O que acontece com os peixes de um determinado fluxo de água quando é
  construída uma barragem?
\item
  Qual é o impacto gerado na fauna e na flora do semiárido após um
  alagamento de grandes proporções?
\item
  Uma vez que construções desse tipo acabam afetando diretamente
  populações que se relacionam com a terra de maneira completamente
  distinta das populações das grandes cidades, o que uma alteração
  ambiental de grande porte significa de fato nessa região?
\end{itemize}

Após a conclusão da redação, é importante que os estudantes
interessados compartilhem a leitura de seus textos com a turma e o
professor, promovendo assim um debate coletivo voltado para a
conscientização ainda maior do preço do progresso de um país.
\end{enumerate}

\section{Obras complementares}

Obras literárias, cinematográficas, pesquisas acadêmicas, vídeos,
\emph{sites}, páginas e canais do \emph{Youtube} que vão ampliar o
debate propiciado pela obra de Maria Valéria Rezende.

    \begin{quote}
    \emph{A CAVERNA dos sonhos esquecidos}. Direção de Werner Herzog.
    França: Creative Differences, History Films, Ministère de la Culture
    et de la Communication, ARTE (co-produção), Werner Herzog
    Filmproduktion (co-produção), More4 (associação), 2011. 90 minutos
    \end{quote}

    A partir da descoberta de uma caverna em Chauvet, na França, o
    cineasta Werner Herzog faz um filme sobre um conjunto de desenhos
    rupestres que deve ter em torno de 30 mil anos, e que perfaz o
    conjunto de registros imagéticos dos mais antigos da humanidade. Com
    depoimento de especialistas em arte, museologia, geólogos e
    biólogos, o filme investiga a caverna, seu significado, e apresenta
    uma série de interpretações interessantes sobre arte rupestre a
    partir de um olhar contemporâneo e criativo.

    \begin{quote}
    ESBELL, Jaider. Disponível em:
    \href{http://www.jaideresbell.com.br/site/}{jaideresbell.com.br},
    acesso em 20 de fevereiro de 2021
    \end{quote}

    Jaider Esbell é artista, escritor e produtor cultural indígena da
    etnia Makuxi. Seu \emph{site} contem imagens de suas obras, vídeos e
    entrevistas com o artista, assim como seu \emph{blog}.

    \begin{quote}
    KRENAK, Ailton. \emph{Ideias para adiar o fim do mundo}. São Paulo:
    Companhia das Letras, 2020
    \end{quote}

    Famoso por ter sido o líder indígena que pintou o rosto com jenipapo
    durante a Assembleia Constituinte de 1987, Krenak é hoje considerado
    um dos maiores porta-vozes de seu povo, além de ser amplamente visto
    como um importante pensador indígena. Suas considerações em
    \emph{Ideias para adiar o fim do mundo} criticam diretamente o modo
    de vida da sociedade contemporânea e sua relação com o desastre
    socioambiental que estamos experimentando. É importante mencionar
    que a etnia de Krenak é natural do Vale do Rio Doce, território que
    foi recentemente afetado por gravíssimos acidentes ambientais.

    \begin{quote}
    MACHADO, Alex Moreira. ``Carrancas de carvão: ressignificações das
    figuras de proas do rio São Francisco''. Dissertação de mestrado em
    Artes Visuais, Escola de Belas Artes da Bahia, Programa de
    Pósgraduação em Artes Visuais da Universidade Federal da Bahia.
    Salvador: UFBA, 2020. Disponível em:
    \href{http://www.ppgav.eba.ufba.br/sites/ppgav.eba.ufba.br/files/2020_-_alex_moreira_machado.pdf}{ppgav.eba.ufba.br},
    acesso em 20 de fevereiro de 2021
    \end{quote}

    Dissertação de mestrado sobre o processo de criação artística do
    autor que explora a poética visual de carrancas de carvão vegetal
    como matéria-prima. Inspirado nas carrancas do rio São Francisco,
    denuncia a situação degradante em que o Rio se encontra.

    \begin{quote}
    ``Mulherio das Letras: Maria Valéria Rezende'', \emph{YouTube},
    canal \emph{Mulheres de Luta}. Sem data. 2 minutos. Disponível em:
    \href{https://www.youtube.com/watch?v=rytdrJToUSY\&feature=emb_logo}{youtube.com/rytdrJToUSY},
    acesso em 20 de fevereiro de 2021
    \end{quote}

    A autora de \emph{Rio de sonhos} é uma das organizadoras do coletivo
    Mulherio das Letras. Neste vídeo, Maria Valéria Rezende conta sobre
    as origens do grupo -- um movimento nacional que reúne mulheres
    envolvidas com a literatura.

    \begin{quote}
    \emph{Mulherio das Letras}. Disponível em:
    \href{https://pt-br.facebook.com/mulheriodasletras/}{fb.com/mulheriodasletras},
    acesso em 20 de fevereiro de 2021
    \end{quote}

    Página oficial no \emph{Facebook} do movimento nacional Mulherio das
    Letras.

    \begin{quote}
    \emph{Mulherio das Letras}, canal do \emph{Youtube}, diponível em:
    \href{https://www.youtube.com/channel/UCzOwcQ9fdT5GmD8IJR_21SA/videos}{youtube.com},
    acesso em 20 de fevereiro de 2021
    \end{quote}

    Canal do \emph{Youtube} que reúne vídeos, filmes, programas e
    atividades realizadas pelo coletivo feminista literário além de
    grupos regionais do Brasil e do exterior.

    \begin{quote}
    MUNDURUKU, Daniel. \emph{O banquete dos deuses}. São Paulo: Global,
    2015
    \end{quote}

    Uma obra essencial para desconstruir os estereótipos e preconceitos
    que ainda perduram em relação à cultura indígena, sua importância e
    o papel de sua ancestralidade na cultura brasileira.

    \begin{quote}
    REZENDE, Maria Valéria. \emph{Outros cantos}. São Paulo: Alfaguara,
    2016
    \end{quote}

    Obra importante da mesma autora de \emph{Rio de sonhos}, que, de
    maneira análoga, chama a atenção para realidade do interior do país
    e a necessidade de repensar o projeto civilizacional do Brasil.

    \begin{quote}
    SILVEIRA, Maria José. \emph{Maria Altamira}. São Paulo: Instante,
    2020
    \end{quote}

    Romance de Maria José Silveira, \emph{Maria Altamira} conta a saga
    de Aleli, sobrevivente do terremoto de Yungay, no Peru, em 1970. Ela
    sai caminhando por vários países da América do Sul até chegar ao
    Xingu, no Brasil. Seu percurso é o destino de milhões de viventes
    sem terra para viver, como de sua filha, Maria Altamira que luta
    para preservar o rio Xingu da atualidade.

    \begin{quote}
    VASCONCELOS, Leonardo. ``Petrolândia, a Atlântida brasileira''.
    \emph{NE10}, \emph{blog Mochileo}. 1 de abril de 2019. Disponível
    em:
    \href{https://m.blogs.ne10.uol.com.br/mochileo/2019/04/01/petrolandia-atlantida-brasileira/}{<<Petrolandia atlântida brasileira>>},
    acesso em 20 de fevereiro de 2021
    \end{quote}

    Contem imagens e vídeos impressionantes da cidade de Petrolândia, a
    459 km de Recife, às margens do rio São Francisco, que foi inundada
    para a contrução da Usina Hidroelétrica Luiz Gonzaga em 1988.


\section{Bibliografia comentada}

    As referências aqui elencadas foram pensadas para que o perfil comum
    às atividades focasse no desenvolvimento de competências leitoras,
    envolvendo escrita e interpretação, já que essa é uma das áreas na
    qual os estudantes brasileiros seguem sistematicamente performando
    abaixo do esperado, tanto nos exames internacionais, quanto nos
    unificados. As questões interdisciplinares, muito embora exijam
    conhecimentos específicos de outras áreas, também focam nesse
    trabalho.

    CAVALCANTE, Paula. ``Igreja submersa volta a aparecer por causa da
    estiagem, em Petrolândia'', \emph{G1}, 24 de novembro de 2014,
    diponível em:
    \href{http://g1.globo.com/pe/caruaru-regiao/noticia/2014/11/igreja-submersa-volta-aparecer-por-causa-da-estiagem-em-petrolandia.html}{<<Igreja submersa volta aparecer em Petrolandia>>},
    acesso em 20 de fevereiro de 2021.

    Matéria que fala sobre a cidade de Petrolândia e mostra como o
    município ficou após a inundação.

    DALVI, Maria Amélia; REZENDE, Neide Luzia de; JOVER-FALEIROS, Rita
    (organização). \emph{Leitura de literatura na escola}. São Paulo:
    Parábola, 2017.

    Coletânea de ensaios sobre o papel indispensável da leitura de
    textos literários na escola.

    DAUSTER, Tania; FERREIRA, Lucelena (organização). \emph{Por que
    ler}?: \emph{perspectivas culturais do ensino da leitura}. Rio de
    Janeiro: Lamparina, Faperj, 2010

    Interessa a professores, pesquisadores da área de Educação e todos
    que acreditam no ato de ler como ferramenta de aprendizagem e fonte
    de crescimento pessoal. Recebeu o selo de Altamente Recomendável --
    FNLIJ 2011.

    GOMES, Suzana dos Santos. \emph{Práticas de leitura e capacidades de
    linguagem na escola}. Belo Horizonte: Editora UFMG, 2011.

    Leitura essencial sobre a realidade da capacidade leitora dos alunos
    brasileiros.

    GOMPERTZ, Will. \emph{Isso é arte}?: \emph{150 anos de arte moderna
    do impressionismo até hoje}. Tradução: Maria Luiza X. de A. Borges.
    Rio de Janeiro: Zahar, 2012.

    Leitura leve e divertida sobre a arte, que acaba abordando tanto o
    contemporâneo quanto questões que atravessam historicamente o campo.

    MARCHUSCHI, Luiz Antônio\textsc{.} \emph{Produção textual},
    \emph{análise de gêneros e compreensão}. São Paulo: Parábola, 2018.

    Obra indispensável para repensar atividades de produção textual e as
    necessidades, tanto do aluno quanto da sociedade, no processo de
    escolarização.

    MOTA, José Aroudo. ``Os limites da transposição do rio São
    Francisco'', \emph{Desafios do desenvolvimento}: \emph{a revista de
    informações do Instituto de Pesquisa Econômica Aplicada}, Ano 2,
    Edição 6, 1 de janeiro de 2005. Disponível em:
    \href{https://www.ipea.gov.br/desafios/index.php?option=com_content\&view=article\&id=723:os-limites-da-transposicao-do-rio-sao-francisco\&catid=29:artigos-materias\&Itemid=34}{ipea.gov.br},
    acesso em 20 de fevereiro de 2021.

    Artigo do Ipea sobre a transposição do São Francisco, dá uma ideia
    interessante sobre a importância do rio e seu papel de destaque na
    bacia hidrográfica tanto do Brasil quanto do Nordeste,
    especificamente.

    ROJO, Roxane. \emph{Letramentos múltiplos}, \emph{escola e inclusão
    social}. São Paulo: Parábola, 2018.

    Obra importante para refletir sobre o papel da escola nas atividades
    de letramento literários e letramentos múltiplos existentes hoje em
    dia.

    \textsc{SCHWARCZ,} Lilia M.; \textsc{STARLING}, Heloisa M.
    \emph{Brasil}: \emph{uma biografia}. São Paulo: Companhia das
    Letras, 2016.

\end{document}


