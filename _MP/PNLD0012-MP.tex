\documentclass[12pt]{extarticle}
\usepackage{manualdoprofessor}
\usepackage{fichatecnica}
\usepackage{lipsum,media9,graficos}
\usepackage[justification=raggedright]{caption}
\usepackage{bncc}
\usepackage[fenix]{../edlab}

 

\begin{document}


\newcommand{\AutorLivro}{Oliveira de Panelas}
\newcommand{\TituloLivro}{Vocação de cantador}
\newcommand{\Tema}{Ficção, mistério e fantasia}
\newcommand{\Genero}{Poema}
% \newcommand{\imagemCapa}{PNLD0012-01.png}
\newcommand{\issnppub}{---}
\newcommand{\issnepub}{---}
% \newcommand{\fichacatalografica}{PNLD0012-00.png}
\newcommand{\colaborador}{\textbf{Mariana Barrile, Bruno Gradella e Vicente Castro} é uma pessoa incrível e vai fazer um bom serviço.}


\title{\TituloLivro}
\author{\AutorLivro}
\def\authornotes{\colaborador}

\date{}
\maketitle
\tableofcontents

\pagebreak

\section{Carta aos professores}

Caro educador / Cara educadora,\\\bigskip

Este Manual tem como objetivo fornecer subsídios para o trabalho com a
obra literária \emph{Vocação de cantador}, de Oliveira de Panelas.

Neste material, são propostas atividades de leitura do texto literário,
em perspectiva interdisciplinar, a partir de propostas de abordagem
envolvendo a integração de diferentes áreas do conhecimento. Todas as
etapas de pré"-leitura, leitura e pós"-leitura atribuem aos professores o
papel de mediadores entre os estudantes e a obra literária. Aos alunos,
por sua vez, é conferido um lugar de protagonismo e autonomia na
construção do conhecimento, a partir do emprego de metodologias ativas e
estratégias de aprendizagem criativa.

Seguindo as competências e habilidades indicadas na nova Base Nacional
Comum Curricular (\textsc{bncc}), o trabalho com o texto literário é desenvolvido
no âmbito dos diferentes campos de atuação social. Para isso, levam"-se
em consideração os campos da vida pessoal, de atuação na vida pública,
das práticas de estudo e pesquisa, jornalístico"-midiático e, sobretudo,
artístico"-literário.

Cada uma das seções do Manual sugere atividades e apresenta informações
complementares para enriquecer a experiência de leitura do texto
literário. A partir de uma proposta dialógica de ensino de literatura,
procura desenvolver habilidades de leitura e produção de textos, visando
à formação do sujeito leitor"-autor competente, capaz de interagir com o
mundo e de atribuir sentido às próprias vivências.

Reforçando o caráter formativo e informativo da literatura, o material
procura articular a formação de leitores à elaboração de projetos de
vida, a partir da ampliação do repertório artístico"-cultural dos
estudantes. A leitura crítica da obra literária é concebida em sentido
amplo e envolve o estabelecimento de relações entre textos pertencentes
a esferas variadas de comunicação e a gêneros discursivos diversos.

Ao mesmo tempo, a releitura de clássicos literários traz, para a
contemporaneidade, novas possibilidades de significação para obras que
permanecem atuais. Para contribuir com essa aproximação entre jovens
leitores e obras consagradas, são propostas atividades que empregam as
novas tecnologias de informação e comunicação, fundamentais para a
formação de leitores inseridos na cultura digital. Valorizam"-se,
portanto, estratégias de leitura e produção textuais no âmbito da
hipertextualidade e do multiletramento.

Para a formação continuada de professores, apresentamos sequências que
estimulam a criatividade e a inovação, com possibilidades de adaptação
às diferentes realidades de ensino. Orientações de caminhos possíveis
são apresentadas como sugestões para as atividades de leitura em sala de
aula, sempre permeáveis aos diferentes perfis de grupos e às
especificidades de gostos e repertórios culturais.

Ao trabalhar com Literatura de Cordel é indispensável que tenhamos
consciência da importância desse gênero literário na história do nosso
país. O Cordel é considerado Patrimônio Cultural Brasileiro, visto que
ele é um meio de disseminar a história, o linguajar e os costumes do
povo nordestino. Por muito tempo, inclusive, foi utilizado como meio de
disseminação de notícias; os cordelistas eram responsáveis por espalhar
as notícias da cidade através do repente, fazendo com que as informações
chegassem à todas as classes, em um linguajar simples e popular.

Trabalhar com Literatura de Cordel em sala de aula é trabalhar com
estratégias de aprendizagem criativa que envolvem as mais diversas áreas
de ensino e de comunicação. É uma grande oportunidade de incentivar os
alunos a desenvolverem seu lado artístico, além de coloca"-los em contato
com um rico material cultural. A leitura de uma literatura tão rica
historicamente traz possibilidades de ressignificação das obras,
aproximando os jovens de realidades que eles, muitas vezes, nunca
imaginaram conhecer.

Conforme a \textsc{bncc} assinala, ``no Ensino Médio, os jovens intensificam o
conhecimento sobre seus sentimentos, interesses, capacidades
intelectuais e expressivas; ampliam e aprofundam vínculos sociais e
afetivos; e refletem sobre a vida e o trabalho que gostariam de ter.
Encontram"-se diante de questionamentos sobre si próprios e seus projetos
de vida, vivendo juventudes marcadas por contextos socioculturais
diversos'' (Brasil, 2018, p.481).

Nada mais adequado, portanto, que oferecer textos literários capazes de
estimular reflexões sobre a vida pessoal e rotas possíveis para os
projetos de vida. Para você, professor(a), abrem"-se novas trilhas para o
contato sempre renovado com obras que são, a um só tempo, atuais e
atemporais.

Boa jornada!

\section{Atividades 1}

\BNCC{EM13LP26}

\subsection{Pré"-leitura}

\BNCC{EM13LGG302}
\BNCC{EM13LGG704}
\BNCC{EM13LP10}
\BNCC{EM13LP19}

Antes de se iniciar a leitura, é interessante que os alunos
entendam o universo para o qual estão enveredando. A literatura de
cordel é um patrimônio da cultura brasileira. Fabricada em folhetos,
geralmente trazem para o universo escrito casos e histórias já comuns na
oralidade popular. Também é comum que seu texto seja construído por meio
de rimas. Além disso, uma das características típicas do cordel são as
gravuras, que possuem um tracejado muito próprio. Não obstante, por
serem uma expressão muito típica da cultura de uma região do país, os
diversos títulos produzidos em cordel trazem elementos básicos que são
características intrínsecas muito próprias. Essa atividade sugere que os
alunos façam uma pesquisa com vários títulos de cordel. Procurem olhar a
sinopse da obra, folheiem seu conteúdo e busquem temas e personagens que
frequentemente aparecem nessas obras. Isso posto, é interessante que
cada aluno escreva um relatório com seus pareceres iniciais, colocando
suas suposições do porquê da repetição desses eixos temáticos e
personagens. Posteriormente, o professor pode coletar essas informações
junto aos alunos e abordar as questões durante as aulas sobre o livro.

\subsection{Leitura}

\BNCC{EM13LGG103}
\BNCC{EM13LP02}
\BNCC{EM13LP48}

Já mais
familiarizados com as características do cordel, os alunos devem passar
então à produção de um texto nesse formato. É possível trabalhar com a
redação de uma autobiografia do aluno, contando sua história nos moldes
de um cordel. Entretanto, é possível que alguns alunos não se sintam
confortáveis em relatar experiências próprias. Diante disso é possível
propor que seja uma biografia com elementos romanceados, elementos do
fantástico, ou mesmo a biografia de uma terceira pessoa, pela qual o
aluno nutre admiração. Em certo sentido, os cordéis promovem isso,
personagens admiráveis, de modo que escrever sobre um ídolo ainda é algo
muito produtivo nesta atividade. Além do mais, só a produção no formato
já ajudará os alunos a terem maior familiaridade com o gênero. As rimas
devem ser estimuladas, mas não são uma obrigatoriedade, posto que podem
gerar uma trava produtiva a alguns estudantes.

\subsection{Pós"-leitura}

\BNCC{EM13LGG102}
\BNCC{EM13LGG303}
\BNCC{EM13LGG402}
\BNCC{EM13LGG703}
\BNCC{EM13LP13}
\BNCC{EM13LP14}
\BNCC{EM13LP28}
\BNCC{EM13LP29}
\BNCC{EM13LP52}

Com o material
produzido na atividade anterior, é possível a realização de uma grande
feira de cordéis na escola, junto ao professor de artes, sugere"-se que
os alunos preparem uma decoração adequada, valorizando as cores e os
traços presentes nesse tipo de material tão distinto. Além da exposição
dos cordéis produzidos, sugere"-se também a promoção de encenações de
trechos de obras lidas, ou produzidas pelos alunos, a declamação de
poemas, competição de repentes. Também se estimula a aproximação deste
gênero com outras produções artísticas mundiais. Nada impede que, em uma
dessas encenações ou leituras dramáticas, os alunos podem buscar trechos
de peças, filmes ou obras da literatura mundial e encená"-los, valendo"-se
da estética e do vocabulário típico aos cordéis.


\section{Atividades 2}

A obra \emph{Oliveira de Panelas} possibilita trabalhos
interdisciplinares e integradores de diferentes campos do saber e áreas
de conhecimento. A seguir, propomos algumas atividades que podem ser
desenvolvidas conjuntamente com professores de outras áreas. Além das
habilidades de Linguagens e suas Tecnologias e de Língua Portuguesa,
indicadas nas etapas da seção anterior e válidas também para esta,
listamos a seguir as habilidades de outras áreas, presentes na abordagem
interdisciplinar:

\BNCC{EM13CNT201}
\BNCC{EM13CNT303}
\BNCC{EM13CHS101}
\BNCC{EM13CHS102}
\BNCC{EM13CHS106}
\BNCC{EM13CHS401}

\subsection{Pré"-leitura}

Antes da
leitura, é possível discutir a situação da seca no Nordeste. Professores
de ciências humanas podem discutir a história desse fenômeno e suas
implicações sociais, bem como as relações que o homem travou com essa
situação ao longo de toda a ocupação da região. Por outro lado,
professores de ciências da natureza podem auxiliar os alunos na
compreensão dos fenômenos geológicos e meteorológicos que contribuem
para a perpetuação da aridez local, bem como explicar o bioma da
caatinga e sua importância para o equilíbrio ecológico. Munidos dessa
informação os alunos podem produzir um jornal, tanto nos moldes
escritos, como em vídeo, construindo uma reportagem sobre todas as
questões aventadas. O material produzido pode ser divulgado no site da
escola. Como sugestão, indica"-se utilizar aplicativos gratuitos de
gravação e edição de vídeos, disponíveis em dispositivos digitais


\subsection{Leitura}

Recomenda"-se, para esta atividade a leitura do poema
\emph{Gemidos da Terra.} A partir dela, sugere"-se, com auxílio do
professor de humanidades, a proposição de um debate acerca da
superpopulação, da ocupação de áreas, até então, desabitadas pelo homem,
e todos os impactos que a humanidade tem gerado na natureza desde o
processo de industrialização. Sugere"-se, para isso, estabelecer diálogos
com pesquisas sobre o aquecimento global, a poluição dos mares, extinção
de espécies e, a partir disso, propor coletivamente ações possíveis para
serem tomadas dentro da escola, em casa e no bairro. É interessante
incentivar a pesquisa a iniciativas tomadas por jovens ao redor do Globo
para tentar sanar os problemas ambientais.

\subsection{Pós"-leitura}

Para essa atividade, já com os alunos bem versados no
universo do cordel, sugere"-se a pesquisa sobre as características das
pelejas ou desafios, tradicionais tenções dos trovadores ibéricos. As
tenções são um modelo poético fixo que consiste em uma discussão entre
dois ou mais trovadores: um deles sustentava uma tese, contrária à do
adversário, sobre uma determinada questão. Grosso modo, poder"-se"-ia
dizer que se trata de um processo dialético, digno das obras de Platão,
porém, no universo da música, da poesia, da rima e do repente.

Dito isso, instrua os alunos a se valerem do resultado da pesquisa e
observarem como isso aparece na obra de Oliveira de Panelas. É possível,
também, fazer um paralelo com a manifestação cultural contemporânea, o
\textsc{slam}.

Ao final, proponha aos alunos a formulação de um sarau com produções
autorais dos estudantes, onde serão apresentadas intervenções, no
formato de \textsc{slam}, a partir do diálogo de fontes construído nesta
atividade. É possível se utilizar das mídias para captar e compartilhar
as produções, montando um acervo que pode ser disponibilizado no portal
do colégio.

\section{Aprofundamento}

O texto literário, presente no currículo escolar desde o Ensino
Fundamental, deve também permanecer como base de formação no Ensino
Médio, para que os alunos possam se aprofundar na compreensão não só da
linguagem literária, mas também das múltiplas linguagens, estimulando a
criatividade e imaginação. Segundo Paulo Freire (1989)\footnote{~\textsc{freire},
  Paulo. A importância do ato de ler: em três artigos que se completam,
  23 ed, São Paulo: Autores Associados: Cortez, 1989.}, o ato de ler
implica na percepção crítica, interpretação, reescrita e reelaboração do
que lemos, agindo assim, diretamente, com diferentes esferas de
aprendizagem.

Este manual oferece subsídios para a abordagem do texto literário em
diálogo com a formação continuada de professores, apresentando
propostas, com diferentes abordagens, para o trabalho de \emph{Vocação
de Cantador, de Oliveira de Panelas,} dentro e fora sala, incentivando
os alunos a adquirirem o hábito de leitura e pensamento crítico.

\subsection{Importância da Literatura de Cordel em sala}

Com uma produção simples e de grande abrangência, a Literatura de Cordel
ganhou espaço e prestígio na cultura nordestina brasileira, tornando"-se,
em 2018, patrimônio cultural imaterial do nosso país -- reconhecido pelo
Conselho Consultivo do Instituto do Patrimônio Histórico e Artístico
Nacional. Assim, o Cordel como gênero do discurso contribui na formação
do aluno possibilitando o domínio de outros conteúdos, além da
descentralização do ensino. O estudo da Literatura de Cordel como forma
de expressão da cultura popular contribui também no aprimoramento das
habilidades de oralidade, escrita, leitura, interpretação, linguagens
artísticas e até na dramatização de peças, auxiliando na
interdisciplinaridade dos temas.

A Educação Literária aparece entre as competências gerais da \textsc{bncc} (Base
Nacional Comum Curricular), enfatizando a importância vivência do aluno
no aprendizado da literatura e demais manifestações artísticas --
``\emph{Valorizar e fruir as diversas manifestações artísticas e
culturais, das locais às mundiais, e também participar de práticas
diversificadas da produção artístico"-cultural}\footnote{~\textsc{brasil}.
  Ministério da Educação. Base Nacional Comum Curricular. Brasília,
  2018.}'' --, logo, o Cordel ganha espaço nesse cenário, pois, a
partir de seu estudo, é possível despertar nos alunos o interesse por
diversos campos artísticos.

Entretanto, o uso do Cordel em sala de aula, bem como nos livros
didáticos, ainda é muito restrito, por não ser tão prestigiado quanto os
demais gêneros literários. Diante disso, cabe a nós resgatarmos essa
parte da nossa identidade nacional, contextualizando o aluno no meio
social e cultural de seu país.

\subsection{Variação linguística e oralidade}

O Cordel nasceu da oralidade e da linguagem popular; uma leitura
silenciosa limita seu poder de comunicação, impedindo que seu potencial
seja trabalhado como um todo. O gênero, se bem explorado, pode auxiliar
no aprendizado e desenvoltura dos alunos nessa modalidade de expressão,
devido ao seu ritmo cadenciado e seu linguajar comum, próximo ao
cotidiano do aluno. Para isso, o professor deve promover atividades que
possibilitem a verbalização do aluno, que estimulem a livre expressão,
para que ele possa, a partir dos exercícios, identificar seu local de
fala, além de desenvolver respeito e empatia pela fala do outro,
aprimorando, também, a convivência social.

Ao utilizar a Literatura de Cordel, o professor poderá abordar a questão
do preconceito linguístico da língua portuguesa, ao estimular a leitura
de poemas que fogem do padrão gramaticalmente institucionalizado. É
possível mostrar aos alunos que a linguagem popular é muitas vezes
discriminada, mesmo fazendo parte de uma cultura rica e diversificada,
quebrando a ideia de que o ideal é necessariamente o padrão.

\subsection{Variação cultural e geográfica}

Como explicitado anteriormente, é sabido que a Literatura de Cordel faz
parte de nossa cultura e tradição. Antes mesmo da chegada das grandes
mídias e meios de comunicação, o Cordel funcionou como instrumento de
disseminação de valores, lendas e conhecimento popular da tradição
nordestina. Levá"-lo à escola é uma maneira de resgate da nossa cultura,
motivando o aluno a conhecer mais sobre nosso país e seus diferentes
povos e regiões, além de nossa história religiosa, econômica e política,
vez que muitos cordéis abordam realisticamente essas questões.

\subsection{Campo artístico e literário }

A ilustração com xilogravura (gravura em madeira) é uma característica
marcante dos folhetos de Cordel, usada para decorar e dar mais vida aos
poemas, além de oferecer material para as mais variadas interpretações
das obras. O uso da técnica deu"-se graças ao baixo custo de produção e
foi fundamental para disseminar a cultura do Nordeste em outras partes
do Brasil. Os traços marcantes da xilogravura de cordel em composição
com os poemas se transformam em uma expressão de linguagem, registrando
a história do nosso povo.

Levar componentes artísticos para a sala de aula é uma forma de chamar à
atenção do aluno, além de proporcionar maior pensamento crítico e
incentivo à expressão artística e literária.

\subsection{Sobre a obra Vocação de Cantador}

\subsubsection{A estrutura da métrica}

A Literatura de Cordel sofreu, estruturalmente, diversas modificações
com o passar dos anos, por se tratar de uma linguagem oral que foi sendo
transformada, também, em escrita. No início, os repentistas não tinham
compromisso com número de versos ou métrica, entretanto a rima sempre
esteve presente nos poemas -- instrumento utilizado para favorecer a
memorização e facilitar a articulação dos repentistas. Entretanto, a
simplicidade não está atrelada apenas à oralidade, mas também ao alcance
social que uma linguagem acessível pode fornecer.

Vocação de Cantador apresenta cordéis de diversos estilos: desde poemas
de improviso, sem nenhuma métrica fixa -- como ``Banquete não é jejum''
--, até as clássicas sextilhas de setissílabos com as rimas xaxaxa, como
em ``O Cantador''.

\subsubsection{Síntese da obra}

Publicada em 2016, a obra \emph{Vocação de Cantador} traz um compilado
de algumas composições do poeta Oliveira de Panelas, desde sua
apresentação como ``homem das nuvens'' -- aquele capaz de voar alto e
levar seus leitores junto --, até poemas que retratam a morte e a
existência humana, através de sua narrativa divertida e criativa.

Por se tratar de uma antologia, a obra reúne cordéis das mais diferentes
fases da vida de Oliveira de Panelas. No início, através da tradicional
sextilha de setissílabos, o poeta se apresenta aos leitores, seguido por
poemas que relatam histórias do povo e da cultura nordestina, envolvendo
uma gama de assuntos -- a vida no sertão, contos amorosos, anedotas
humorísticas, relatos de guerra, até reforma agrária e política --, nos
mais diversos metros. Ao final, o poeta se despede dos leitores,
agradecendo"-os e se desculpando por aqueles que não gozaram de seus
cordéis.

\subsubsection{Sobre o autor}

Oliveira Francisco de Melo, mais conhecido como Oliveira de Panelas --
por ter nascido no município de Panelas (\textsc{pe}) em 24 de maio de 1946 --, é
cordelista e repentista desde a sua infância. Seu talento e dedicação ao
Cordel fez com que ele viajasse o mundo nos últimos 40 anos levando a
cultura nordestina a vários países, como Cuba, Portugal e França, além
de ganhar o apelido de ``Pavarotti dos Sertões'', graças à sua maestria
e notável erudição, apresentando"-se a todo tipo de público.

Pertencente à ``nova'' geração de cordelistas, virou uma figura popular
na \textsc{tv} e nos rádios -- e hoje também na internet --, disseminando a
Literatura de Cordel pelo Brasil e mundo afora. Apesar de fazer parte
dessa nova geração, o poeta conservou os costumes dos cordelistas,
resultando, assim, nesse grande sucesso. Oliveira de Panelas coleciona
títulos e premiações ao falar, principalmente, sobre os problemas
enfrentados pelo povo nordestino, divulgando informações e
proporcionando entretenimento a quem não tem acesso.

\subsection{Papel do leitor}

O hábito de leitura auxilia na evolução de diversas habilidades, em
especial se desenvolvido durante a infância e adolescência. Ele trabalha
diretamente com o aprimoramento do vocabulário, criatividade,
imaginação, além das habilidades socioemocionais, como maior habilidade
para estabelecer diálogos, lidar com desafios e sentimentos, ajudando na
formação do indivíduo. Esses são aspectos fundamentais para uma melhor
interação social, além de proporcionar sensações ímpares.

É de extrema importância que o professor incentive o aluno a adquirir o
hábito de leitura pois, através dela, é possível que ele aprenda, viaje
e descubra sobre novos lugares e povos sem sair de casa.

\section{Referências complementares}

\subsection{Livros}

\begin{itemize}
\item\textsc{andrade}, Cláudio Henrique Salles; \textsc{silva}, João Melquíades Ferreira;   \textsc{barros} Leandro Gomes de. \textit{Feira de versos: poesia de cordel}.
  São Paulo: Ática, 2019.
  
Este livro é uma coleção de pérolas do cordel nacional. A obra reúne
textos de três importantes cordelistas: Leandro Gomes de Barros, o
primeiro a editar cordel no Brasil no século \textsc{xix}, João Melquíades, que
apresenta \emph{O pavão misterioso} e Patativa do Assaré, cujos textos
vêm ganhando reconhecimento internacional.

\item\textsc{haurélio}, Marco. \textit{Antologia do cordel brasileiro.} São Paulo:
  Global, 2012.

Nesta antologia, o leitor tem acesso a um leque variado de cordéis,
desde aqueles inspirados nos contos fantásticos e nos contos de fadas,
até outros em que predominam mitos da Grécia Antiga ou que deitam raízes
nas histórias de animais do fabulário mundial.

\item\textsc{suassuna}, Ariano. \textit{Romance da Pedra do Reino e o Príncipe do
  Sangue do Vai"-e"-volta}. Rio de Janeiro: Nova Fronteira, 2017.


O romance de Ariano Suassuna, publicado originalmente em 1971, narra a
história de Dom Pedro Dinis Ferreira, o Quaderna, apresentando seu
memorial de defesa perante o corregedor, com ressonâncias da tradição
literária do cordel.
\end{itemize}

\subsection{Filmes}
\begin{itemize}
\item\textit{Auto da Compadecida}. Direção: Guel Arraes (Brasil, 2000).

Baseado na peça de Ariano Suassuna, o filme evoca o imaginário popular e
religioso do Nordeste para contar as aventuras de João Grilo e Chicó,
uma dupla de malandros que sobrevive de trapaças.

\item\textit{Patativa do Assaré -- Ave Poesia}. Direção: Rosemberg Cariny (Brasil, 2007).

O documentário apresenta a trajetória da vida e da obra do poeta
cearense Patativa do Assaré, que explorou em sua obra a riqueza das
tradições populares. Sua história é contada por meio de depoimentos de
amigos, familiares e admiradores que destacam a relevância do artista
para a cultura brasileira.
\end{itemize}

\subsection{\emph{Site}}

\begin{itemize}
\item\textit{Cordel: Literatura Popular em Verso}

(\url{http://www.casaruibarbosa.gov.br/cordel/acervo.html})

No \emph{site} da Fundação Casa de Rui Barbosa, há informações e
materiais diversos sobre o acervo da instituição, com diversos
exemplares representativos da literatura nacional em cordel.
\end{itemize}

\section{Bibliografia comentada}

\begin{itemize}
\item\textsc{farias}, Pedro Américo de. \textit{Nordestinos: coletânea poética do
  Nordeste brasileiro}. Lisboa: Fragmentos, 1994.

A obra apresenta uma recolha da poesia popular nordestina, com temas
oriundos do folclore e da matéria histórica.

\item\textsc{haurélio}, Marco. \textit{Literatura de cordel: do sertão à sala de
  aula.} São Paulo: Paulus, 2013.

Declamados ou cantados, os cordéis levaram ao público, da tradição oral
ao contexto escolar, as façanhas dos cangaceiros Lampião e Antônio
Silvino, os milagres do Padre Cícero e outras narrativas populares.

\item\_\_\_\_\_. \textit{Breve história da literatura de cordel.} São
  Paulo: Claridade, 2018.

Esta obra apresenta as origens do Cordel, destaca seus principais
expoentes e mostra o leque de influências dessa tradição na cena
cultural brasileira.

\item\textsc{nascimento}, Lourgeny Damasceno do. \textit{A importância da literatura
  de cordel no cotidiano dos alunos da \textsc{eja}}. Monografia apresentada ao
  Departamento de Artes da \textsc{unb}. Brasília: 2011. (Disponível em:
  \url{https://bdm.unb.br/bitstream/10483/4463/1/2011_LourgenyDamascenodoNascimento.pdf}.
  Acesso em 18 de fevereiro de 2021.)

A autora destaca a relevância do trabalho com poemas em cordel no
trabalho com jovens e adultos, a partir do resgate das tradições
populares e da valorização dos saberes regionais.

\item\textsc{tavares}, Braulio. \textit{Contando histórias em versos: poesia e
  romanceiro popular no Brasil.} São Paulo: Editora 34, 2009.

O autor apresenta os principais recursos expressivos da linguagem
poética popular, enquanto introduz os leitores a rimas, ritmos, temas,
formas literárias e modos narrativos tipicamente brasileiros.
\end{itemize}

\end{document}

