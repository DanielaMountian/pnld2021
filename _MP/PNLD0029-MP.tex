\documentclass[12pt]{extarticle}
\usepackage{manualdoprofessor}
\usepackage{fichatecnica}
\usepackage{lipsum,media9,graficos}
\usepackage[justification=raggedright]{caption}
\usepackage[one]{bncc}
\usepackage[nmenosum]{../edlab}

% \Image{Anne Ballester Soares (Autor Desconhecido; Acervo da Autora)}{PNLD0029-03.png}
% \Image{Xapono, casa comunal habitada pelo Yanomami (Wikimedia Commons; Domínio Público)}{PNLD0029-04.png}
% \Image{Mapa do território Yanomami no Brasil e Venezuela, 2012 (Wikimedia Commons; CC-BY-SA 3.0)}{PNLD0029-05.png}
% \Image{Mulheres Yanomami do lado amazônico (Wikimedia Commons; CC-BY-SA 3.0)}{PNLD0029-06.png}
% \Image{Mulher Yanomami tecendo cesta, 1999 (Cmacauley; CC-BY-SA 3.0)}{PNLD0029-07.png}

\begin{document}


\newcommand{\AutorLivro}{Anne Ballester Soares (Org.)}
\newcommand{\TituloLivro}{O surgimento da noite: mitologias Yanomami*}
\newcommand{\Tema}{Ficção, mistério e fantasia}
\newcommand{\Genero}{Mitologia indígena}
\newcommand{\imagemCapa}{./images/PNLD0029-01.png}
\newcommand{\issnppub}{---}
\newcommand{\issnepub}{---}
% \newcommand{\fichacatalografica}{PNLD0029-00.png}
\newcommand{\colaborador}{\textbf{Alessandra Cabral, Vicente Castro e Bruno Gradella}}


\title{\TituloLivro}
\author{\AutorLivro}
\def\authornotes{\colaborador}

\date{}
\maketitle




\begin{abstract}\addcontentsline{toc}{section}{Carta ao professor}
Este Manual tem como objetivo fornecer subsídios para o trabalho com a
obra literária \emph{O surgimento da noite.}

Este livro foi organizado por Anne Balester Soares, francesa nascida em 1955 
que viveu por 24 anos com os Yanomami.  Enquanto ativista, trabalhou como agente 
de saúde no combate à malária, foi alfabetizadora em língua yanomami e professora 
de português para jovens e adultos em posições de liderança indígena.

\emph{O Surgimento da Noite} apresenta narrativas sobre os mitos de criação segundo 
os Yanomami. O livro conta não somente com as narrativas coletadas, como também 
apresenta a versão bilíngue, uma vez que esse material foi organizado para o uso 
das comunidades Yanomami também. A obra contém um pequeno material introdutório 
com a descrição dos fonemas indígenas. Ele é extremamente importante para nos 
aproximarmos dessa cultura e entendermos os sons que deverão ser pronunciados 
verdadeiramente durante a leitura. 

Apesar do título do livro fazer referência ao surgimento da noite, as narrativas
encontradas ao decorrer da leitura abordam o surgimento de outros elementos
também. Encontraremos o surgimento do tabaco, do cipó e da banana através
das aventuras do personagem Horonamɨ. Horonamɨ é um grande pajé que surgiu de si mesmo, 
assim como é relatado na narrativa. Surgiu junto com as florestas e ensinou aos Yanomami 
como morar nelas. Além de compartilhar os conhecimentos com o povo, ele também
compartilhou suas histórias com os estrangeiros.

A valorização da natureza e de seus elementos é a peça fundamental para entender 
a cultura Yanomami. O modo de narrar apresentado é muito semelhante as aventuras 
de Macunaíma, obra de Mario de Andrade. O autor modernista mergulhou nas histórias 
brasileiras para construir seu herói e suas aventuras.

Esperamos que as indicações propostas aqui sejam muito úteis no trabalho em
sala de aula! 


\end{abstract}


\tableofcontents
\section{Proposta de atividades I}

\subsection{Pré-leitura}

% EM13LGG302
% EM13LGG704
% EM13LP10
% EM13LP19
%
\paragraph{Tema} Cosmologia indígena brasileira.

\paragraph{Conteúdo} Compreensão das multiplas narrativas antigas ameríndias
e suas diferenças com as narrativas ocidentais. 

\paragraph{Objetivo} Aproximar a perspectiva indígena ao universo acadêmico, 
compartilhando as narrativas de maneira didática.

\paragraph{Justificativa} Todos os povos detêm, em sua cultura e folclore,
histórias e personagens lendárias, compondo suas identidades por meio de
narrativas. 

Conhecer e preservar as narrativas é uma forma de
cultivar a cultura e, nesse sentido, as noções e espacialidades outras 
que a cosmologia ameríndia nos revela coloca-nos em condições de fronteiras 
entre mundos possíveis, possibilitando, por meio delas, expandir nosso 
próprio mundo e nossas próprias percepções.

Refletir sobre os múltiplos modos de habitar e se relacionar com a terra e 
a floresta, a partir de perspectivas que as entendem como um entidade
viva e povoada também por outros seres que não somente humanos, possibilita
olhar criticamente a forma de habitar ocidental, seus ecos e catástrofes.

A floresta, as casas tradicionais e os espíritos, são alicerces no pensamento 
indígena e suas muitas formas de habitar o território. Aproximar-se dessa outra
forma de ver a terra, em seus devidos termos, permite perceber que a cosmologia 
está constantemente vinculada à identidade social coletiva e individual dos povos
indígenas, Yanomami e outros. 

Como diria Bruce Albert e Davi Kopenawa em \textit{A queda do céu: palavras de 
um xamã yanomami}: ``Não pensem que a floresta é vazia. Embora os brancos não os 
vejam, vivem nela multidões de espíritos''.

``Tampouco pensem que as montanhas estão postas na floresta à toa, sem nenhuma razão. 
São casas de espíritos; casas de ancestrais''.

\paragraph{Metodologia} 

% EM13LGG302
% EM13LGG704
% EM13LP10
% EM13LP19

Para se iniciar os trabalhos com \emph{O Surgimento
da Noite}, sugere-se orientar os alunos à pesquisa de mitos e lendas
indígenas brasileiros. Para uma maior riqueza de detalhes, selecione
etnias específicas e, dividindo a sala em grupos, indique cada uma delas
a um grupo. 

Para introduzir a atividade mobilize a turma para pesquisar sobre Ailton 
Krenak e Davi Kopenawa Yanomami, do povo Krenak, em Minas Gerais e Yanonami,
no Amazonas, respectivamente. Essas duas lideranças indígenas têm grande valia 
para a disseminação do pensamento indígena ameríndio e suas cosmologias particulares.

Proponha aos grupos que busquem as semelhanças e diferenças nos significados que são 
atribuídos à floresta e outros-que-humanos entre as etnias que foram designados a pesquisar,
produzindo, em conjunto, um organograma representativo de parte da cosmologia ameríndia.  

\SideImage{Anne Ballester Soares (Autor Desconhecido; Acervo da Autora)}{PNLD0029-03.png}


\paragraph{Tempo estimado} Duas aulas de 50 minutos. 


\subsection{Leitura}

% EM13LGG103
% EM13LP02
% EM13LP48

\paragraph{Tema} Notícias falsas e comunidades indígenas.

\paragraph{Conteúdo} Conhecimento e articulação de uma linguagem jornalística
comprometida com a segurança das informações e projeção dos direitos e vozes
indígenas. 

\paragraph{Objetivo} Estimular e habilitar os estudantes a procurarem fontes
confiáveis e não compartilhar a desinformação.

\paragraph{Justificativa} Atualmente vivemos em uma sociedade que possui acesso 
à informação de maneira dinâmica e somos arrebatados a todo instante por alguma 
nova notícia. Nesse sentido, estar bem-informado é muito importante e o acesso à
informação de forma clara e segura é ainda mais, porém vivemos uma onda de falsas 
notícias e devemos tomar muito cuidado com o compartilhamento de informações enganosas.

A comunidade indígena também foi atingida pela desinformação e a criação de boatos.
No entanto, diferentemente do passado, a população da Terra Indígena Yanomami possui seu
próprio sistema de radiofonia, possibilitando a comunicação entre as aldeias e as cidades 
do entorno, ampliando o acesso à informação.

Além desse sistema, conectados aos celulares e em contato com outros parentes, são outras as
estratégias para estancar a desinformação, que por meio das redes sociais ecoam suas vozes e 
denúncias, mobilizando uma rede de apoiadores, aliados e defensores dos povos da floresta. 

Perceber e distinguir as notícias falsas faz-se urgente, através de um processo de conscientização
coletiva, para que os povos indígenas possam ter seus direitos respeitados e vozes escutadas. 

\paragraph{Metodologia} 

Para introduzir essa atividade, proponha aos alunos a pesquisa de notícias falsas e relacionadas
aos povos indígenas e/ou seu território. Com o intuito de desmentir, sugira a escrita de textos
com as verdadeiras notícias, em formato jornalístico e, se possível, peça que divulguem no
portal online da escola – caso não seja viável, convide os alunos a apresentarem entre si os textos, 
simulando a redação de um jornal comprometido com a transmissão de informações seguras.  

Como sugestão de temática para as notícias, poderia se pensar a situação garimpeira na Terras
Indígenas Yanomami. Com o enfraquecimento da legislação ambiental e indigenista, falsas promessas
viabilizam a exploração mineral e desmatamento nessas Terras.

Faça uma roda de conversa com a sala antes da produção escrita individual de cada aluno, a fim de
estimular uma discussão frente às responsabilidades de se tornar aliado dos povos indígenas, possibilitando
a expulsão de interferências externas em suas Terras e permitindo viver em paz e com saúde.  


\paragraph{Tempo estimado} Duas aulas de 50 minutos.


\subsection{Pós-leitura}

% EM13LGG102
% EM13LGG303
% EM13LGG402
% EM13LGG703
% EM13LP13
% EM13LP14
% EM13LP28
% EM13LP29
% EM13LP52

\paragraph{Tema} Transposição no eixo narrativo.

\paragraph{Conteúdo} Exercícios de escrita buscando mudar a pessoa verbal, 
do texto original, reposicionando o leitor como novo narrador, aproximando-o
do discurso.

\paragraph{Objetivo} A partir da experiência de leitura e do exercício proposto
de escrita, observar a compreensão da obra lida, por meio do resultado dessa 
transposição de narradores, ofertando uma nova perspectiva sobre \textit{O Surgimento da Noite}. 

\paragraph{Justificativa} A mudança do eixo narrativo de terceira para primeira
pessoa, transforma os leitores em protagonistas da obra. Não sendo mais possível
o elemento da onisciência, passa a existir uma maior dificuldade na criação da 
atmosfera sentimental, sensorial e psicológica do indivíduo, exigindo dos alunos
um esforço de tradução.

Ao mesmo tempo que seja um desafio colocar-se na posição de narrador de mitos que 
pareçam, ao olho desatento, muito distantes da condição ocidental, é uma forma de 
verificar a apreensão do conteúdo da obra e as reverberações dessa leitura. 

Reposicionar-se como narrador de mitos que remontam o mundo como ele é hoje em dia, 
permitiria, como afirma Eduardo Viveiros de Castro, pensar que ``o mito não é apenas 
o repositório de eventos originários que se perderam na aurora dos tempos, ele orienta 
e justifica constantemente o presente.'' 

Assim, mbilizando os alunos para além das informações contidas na obra, essa atividade 
estimularia a pensar o mito e sua narrativa como meio conceitual e temporal que se destina
a múltiplos receptores.

\paragraph{Metodologia}

Nessa atividade de pós-leitura é interessante sugerir aos estudantes que
produzam textos, a partir das narrativas lidas na obra, presentando aos leitores a 
perspectiva dos protagonistas, por meio da troca das pessoas do discurso.

A obra em terceira pessoa, passaria a ser contada em primeira pessoa do singular,
reposicionando os alunos, não mais como leitores, e sim como emissores.

Antes de realizarem o exercício, proponha à turma uma leitura coletiva do trecho abaixo:

``Quando Horonamɨ flechou o mutum da noite, apesar de estar perto da sua moradia e de retornar 
correndo, ele também sofreu, porque anoiteceu de uma vez. Depois de ter soprado a noite em todos
os cantos, e de ter corrido, ele adormeceu. Naquela noite, os Yanomami também sofreram. Não anoiteceu
devagar. Até Horonamɨ passou fome, pois não tinha como fazer fogo. Ele acabou ficando na escuridão, 
apesar de estar perto do seu xapono. Como foi assim que aconteceu, a mãe dele também sofreu, todos 
ficaram tontos de fome à noite. A escuridão perseguiu Horonamɨ bem de perto, e ele estava com fome.

Depois de a noite apagar o dia, os que moravam com ele morreram de fome, pois comiam somente terra,
comiam terra vorazmente e sofriam. Não sobreviveram. Até seu próprio cunhado sofreu e quase morreu. 
Horonamɨ ficou angustiado.

Havia então três pajés: o avô, o avô mais novo e o cunhado, e eles esquartejaram a noite, fazendo 
reaparecer a luz do dia.

Para as pessoas não comerem mais terra, Horonamɨ foi caçar. Ele nos ensinou a caçar. Ele tinha uma 
zarabatana, que alguns Yanomami usam para soprar, era isso que ele usava. Ele soprava os animais, tinha 
um sopro forte, e foi assim que ele nos ensinou a matar a caça com veneno.

É assim, é a própria história dos antepassados. É a história daquele que se apossou da floresta, é o 
início de tudo, a história do primeiro dono da floresta, Horonamɨ.''

Sugira que o texto seja lido já pensando na transposição do narrador e depois da leitura, em roda,
conversem sobre as reverberações que a passagem teve em cada um dos alunos e como isso pode ser
retomado na produção escrita.


\paragraph{Tempo estimado} Duas aulas de 50 minutos.


\section{Proposta de atividades II}


A obra \emph{O Surgimento da Noite} possibilita trabalhos
interdisciplinares e integradores de diferentes campos do saber e áreas
de conhecimento. A seguir, propomos algumas atividades que podem ser
desenvolvidas conjuntamente com professores de outras áreas. Além das
habilidades de Linguagens e suas Tecnologias e de Língua Portuguesa,
indicadas nas etapas da seção anterior e válidas também para esta,
listamos a seguir as habilidades de outras áreas, presentes na abordagem
interdisciplinar:


% EM13CNT201
% EM13CNT303
% EM13CHS101
% EM13CHS102
% EM13CHS106
% EM13CHS401

\subsection{Pré-leitura}

\paragraph{Tema} Demarcação de Terras Indígenas.

\paragraph{Conteúdo} Introdução e contextualização das Terras Indígenas, por meio
do levantamento de elementos e técnicas socioantropológicas.

\paragraph{Objetivo} Ambientalizar os estudantes nas questões que permeiam a homologação
do território brasileiro pelos povos indígenas e a importância desse direito à manutenção
de suas culturas e tradições.
Identificar as diferenças nas demarcações das Terras Indígenas em relação ao período
atual e as mudanças no decorrer do tempo.

\paragraph{Justificativa} A Terra Indígena Yanomami, entre os estados de Roraima e Amazonas, 
é a maior Terra Indígena do Brasil e ela foi demarcada e homologada pelo governo brasileiro 
através de um decreto presidencial assinado em 25 de maio de 1992.

Observar como o conhecimento desses pormenores é importante, pois, por meio deles é
possível se chegar a mecanismos mais contundentes de preservação dos direitos indígenas. 

A constituição de 1988 assegura o direito de existir como povo, cultura, território e modo de vida.
No entanto, atualmente, esse direito está novamente ameaçado pela destruição acelerada da floresta.

Em uma entrevista, a fala de Ailton Krenak é representativa da importância da segurança desse direito:
``Quando os índios falam que a Terra é nossa mãe, dizem ‘Eles são tão poéticos, que imagem mais bonita’. 
Isso não é poesia, é a nossa vida. 
Estamos colados no corpo da Terra. 
Somos terminal nervoso dela. 
Quando alguém fura, machuca ou arranha a Terra, desorganiza o nosso mundo''. 

\paragraph{Metodologia}

Para essa atividade de pré-leitura, em parceria com professores de ciências humanas, 
proponha uma interlocução com outras disciplinas, como a antropologia social, suas ferramentas 
e técnicas, para que se possa compreender como se dá a manutenção das culturas e tradições dos 
grupos indígenas brasileiros. 

Partindo do mapa etno-histórico do Brasil, de 1944, adaptado do mapa de Curt Nimuendajú,
proponha aos alunos refletirem também sobre os dados históricos e linguísticos de cada povo
a ser estudado e as mudanças que ocorreram por meio das políticas de delimitação das Terras 
Indígenas. 

Sugira que cada um dos grupos, no final do trabalho, retorne ao mapa de 1944, refletindo 
sobre o papel do Estado no manejo do território e como a cosmologia ameríndia estabelece 
conexões com a Terra para além do posicionamento geográfico e seu significado para esses povos.

Como produção, sugere-se a produção de um mapa etnolinguístico das populações indígenas do Brasil.

\Image{Mapa do território Yanomami no Brasil e Venezuela, 2012 (Wikimedia Commons; CC-BY-SA 3.0)}{PNLD0029-05.png}


\paragraph{Tempo estimado} Duas aulas de 50 minutos.


\subsection{Leitura}

\paragraph{Tema} O pensamento mítico e o pensamento científico.

\paragraph{Conteúdo} Compreensão da importância de outros-que-humanos na cosmologia
Yanomami e as relações entre as espécies apresentadas na obra. 

\paragraph{Objetivo} Apresentar o pensamento mítico ao lado do pensamento científico
como formas de olhar o mundo que não se anulam, mas se complementam, quando situadas e 
lidas em seus próprios termos. 

Familiarizar os estudantes com as classificações taxonômicas da biologia em reino, filo, 
classe, ordem, família, gênero e espécie, entendendo também as particulares do bioma 
Amazônico: o conjunto de animais e plantas dessa região.

Os estudantes devem identificar passagens na obra literária que mencionem essas espécies e
propor agrupamentos considerando tanto a cosmologia ameríndia quanto a classficação biológica
de filos de organismos com características semelhantes. 

\paragraph{Justificativa} O território ocupado pelos Yanomami, atualmente, abrange 
aproximadamente 192 mil km2 de floresta tropical contínua, no oeste do maciço 
guianense, em ambos os lados da fronteira entre o Brasil e a Venezuela. A população 
total é estimada em 36.000 pessoas, distribuídas em  mais de 600 aldeias. 
No Brasil, os Yanomami somam aproximadamente 22.000 pessoas, distribuídas 
nas mais de 250 aldeias da Terra Indígena Yanomami – a maior Terra Indígena 
do país –, com área aproximada de 96.650 km2, entre os estados de Roraima e 
Amazonas. Yanomami, enquanto grupo indígena e a família linguística, conforma
um conjunto cultural e linguístico composto de cinco subgrupos de línguas 
aparentadas e parcialmente inteligíveis entre si, com diferentes costumes, 
conhecimentos, concepções e construções.

O vínculo e conexão com os animais, águas e montanha dos Yanomami com a 
Terra, faz parte da forma como veem a origem do mundo e remontam seus mitos. 
As narrativas míticas yanomami remetem sua origem à copulação do demiurgo 
Omama com a filha do monstro aquático, dono das plantas cultivadas. À Omama
é atribuída a origem das regras da sociedade e da cultura yanomami atual,
bem como a criação dos espíritos auxiliares dos pajés: os \emph{xapiri}, 
para curá-los das doenças e protege-los da morte. Os \emph{xapiri} são os 
espíritos das águas, espíritos da floresta e espíritos animais, eles descem 
à floresta como uma multiplicidade de imagens de seres brilhantes que cantam 
e dançam, sendo sua aparência como pequenos humanoides.

Para os Yanomami, a floresta é viva, ela pensa e transmite palavras inesgotáveis,
que são a fonte de conhecimento dos \emph{xapiri}. Estes espíritos escutam as 
árvores e são, portanto, o veículo dos saberes da floresta para os xamãs. Os xamãs,
por sua vez, conseguem equilibrar no peito uma multiplicidade de \emph{xapiris}, 
escutam o seu canto e traduzem estes saberes para o resto da sua comunidade.

O filho de Omama foi o primeiro xamã. O irmão ciumento e malvado de Omama,
Yoasi, é a origem da morte e dos males do mundo.

Nas palavras de Davi Kopenawa: ``Omama  tinha muita sabedoria. Ele soube criar a 
floresta, as montanhas e os rios, o céu e o sol, a noite, a lua e as estrelas. 
Foi ele que, no primeiro tempo, nos deu a existência e estabeleceu nossos costumes. 
Ele também era muito bonito. Seu irmão Yoasi, ao contrário, tinha a pele coberta de 
manchas esbranquiçadas e só fazia coisas ruins.''

Quanto ao surgimento da noite, mobiliza-se uma série de espécies que indicam essa 
relação intrínseca entre a floresta e o pensamento indígena: 

``A noite estava empoleirada em cima de uma árvore não muito distante. Parecia com 
um mutum empoleirado, cuja cauda repousava na parte alta de um galho inclinado de 
uma árvore paikawa. Assim era a escuridão. Apesar de a noite parecer um mutum, 
Horonamɨ conseguiu encontrá-la. A noite também cantava como um mutum.

Nessa época, os animais — como arara, mutum, queixada, anta, veado, caiarara, maitaca, 
irara, tamanduá-bandeira, papagaio e jabuti — eram Yanomami e, como os Yanomami, moravam 
em xapono. Horonamɨ designou cada espécie de animal e deu-lhes seus nomes. Naquela época, 
ele procurou pela terra firme sem descanso, quando não havia xaponos espalhados pela selva; 
havia somente o xapono dele. Os animais também viviam em xapono.

Quando Horonamɨ soprou a escuridão com sua zarabatana para nós dormirmos, ele queria que 
anoitecesse. Ele encontrou a escuridão e soprou. Depois de fazer cair a escuridão, ao mesmo 
tempo se desenhou um pequeno círculo no chão, embaixo do lugar onde estava empoleirado o 
dono da escuridão.''

Aproximar essa outra lógica de compreensão dos seres da lógica científica que classifica,
organiza e separa os seres entre humanos, vivos e outros, revela-se de grande importância
para a entender como isso se reverbera na forma que cada um dos povos lida com a terra e os 
espaços que habitam. 

\paragraph{Metodologia}

Durante a leitura, professores de ciências da natureza podem abordar os mitos 
presentes no livro, posicionando o pensamento mítico junto ao pensamento científico.

Uma narrativa mítica ensina que os estrangeiros devem também sua
existência aos poderes demiúrgicos de Omama. Conta-se que foram criados
a partir da espuma do sangue de um grupo de ancestrais yanomami, que devorados por jacarés 
e ariranhas e afogados em uma grande enchente que ocorreu na aldeia ancestral Hayowari, entre 
o rio Parima e o alto Orinoco, reconhecida por ser onde Omama teria perfurado o solo de sua 
roça e com a água que saia do chão dado origem a todos os rios da terra e os céus. Os estrangeiros 
teriam sido conduzidos por Omama às terras europeias e teria feito com que essa gente se parecesse
com os Yanomami.

Tendo retomado essa passagem, sugira aos alunos que, a partir do
substrato oferecido pelos professores de ciências da natureza seja
procedida uma reflexão do porque tais elementos foram priorizados pelos
índios Yanomami em seus mitos que remontam a origem de seus mundos.

Proponha que de antemão a turma crie hipóteses, com base nas leituras, para que 
depois possem refutá-las ou não, dentro dos termos científicos, considerando
sempre a importância simbólica dos mitos. 

Como produto final e processual, crie em conjunto com os alunos um mapa afetivo, relacionando as 
espécies mencionadas nas passagens da obra com os espaços de uma aldeia Yanomami e depois agrupe-as 
de acordo com as classificações taxonômicas.

\Image{Xapono, casa comunal habitada pelo Yanomami (Wikimedia Commons; Domínio Público)}{PNLD0029-04.png}


\paragraph{Tempo estimado} Duas aulas de 50 minutos.


\subsection{Pós-leitura}

\paragraph{Tema} Vídeo e cinema indígena.

\paragraph{Conteúdo} Produção de escrita em formato de roteiro pensando possíveis
formatos de mídia de transmissão, além da articulação de capturas de imagem e som 
para remontar o que foi aprendido com a leitura da obra \textit{O Surgimento da Noite}.

\paragraph{Objetivo} Estimular e habilitar os estudantes a pensarem paralelos 
entre produções escritas e o mundo, enquanto imagem e paisagem sonora e física,
para produção audiovisual. 

Exercitar a capacidade de tradução da cosmologia indígena para situações imagéticas
e sonoras que não necessariamente tenham relação direta com a passagem da obra, mas 
que ainda assim sejam representativas.

\paragraph{Justificativa} A produção, formação e discussão audiovisual como ferramentas 
de produção de narrativas de reinvindicação, luta e imaginação dos povos indígenas e para 
outros povos é de extrema relevância por situarem o discurso indígena em seus próprios 
termos, contando suas próprias lutas e também fazendo com que representem os não-indígenas 
de acordo com seus próprios códigos.

A organizadora de \textit{O Surgimento da Noite}, Anne Ballester Soares vive entre os Yanomami 
desde 1994, organizando publicações em seu idioma, ela acredita que será melhor para os Yanomami 
se comunicarem de forma autônoma.

``Eles não são livres para se expressar e ainda assim têm muito a contar. A saúde indígena está 
piorando devido ao uso indevido de recursos. Seus próprios métodos de cura não são respeitados e 
há novas ameaças de prospectos de mineração, assim eles querem poder contar para resistir'', disse 
ela em uma entrevista.

Anne é co-fundadora de um centro de mídia que promove oficinas para ensinar a comunidade indígena a 
usar a captura e edição de áudio e vídeo, de maneira a melhor a linguagem, apresentação e análise, 
bem como escrita de roteiro, pós-produção, formatos de compressão e mídia de transmissão.

Os Yanomami lidam com a produção de imagens principalmente pela função política de sua circulação.
Nós, não indígenas tendemos a acreditar nas coisas somente quando vemos as imagens, como provas de 
existência. Nesse sentido, o exercício de tradução de passagens lidas na obra para o universo imagético 
aproxima do léxico cultural não-indígena a cosmologia ameríndia e permite que os alunos expressem
suas interpretações, enquanto estrangeiros a cultura Yanomami. 

\paragraph{Metodologia}

Após a leitura, sugere-se que os estudantes produzam um
documentário em vídeo sobre a cosmologia, a língua e os valores de um dos
povos indígenas mapeados anteriormente. 

Sugere-se, para isso, inicialmente, a produção de um roteiro, onde restarão 
determinados os temas a serem tratados, bem como esses terão divididos os 
tempos para cada assunto. A introdução de músicas, vinhetas, momentos de fala,
entrevistas, narrativas com imagens também devem ser pensados.

Sugere-se, para essa atividade a utilização de obras que tratam da
cosmologia e história indígena. 

Também indica-se a transmissão de filmes
da Anauê Mostra de Cinema Indígena e do forumdoc.bh – Festival do Filme 
Documentário e Etnográfico, fórum de antropologia e cinema –, onde os filmes
são produzidos pelos povos indígenas, o que conferirá ao estudante uma 
oportunidade de compreender o mundo sob a ótica dessas populações. Além da própria 
plataforma criada pela organizadora da obra, Anne Ballester Soares, chamada 
``Xapono - Núcleo Audiovisual Yanomami'', onde é possível observar a produção 
audiovisual indígena das aldeias. 

Caso não seja possível a produção de vídeo, proponha que os alunos façam 
uma série de desenhos expressando o que, posteriormente, seria a progressão 
de um vídeo ou animação.


\paragraph{Tempo estimado} Duas aulas de 50 minutos.


\section{Aprofundamento}

Ao chegar ao Ensino Médio, é necessário que os estudantes se aprofundem
na compreensão das múltiplas linguagens e, sobretudo, da linguagem
literária. Em relação à literatura, a BNCC traz as seguintes
considerações:

\begin{quote}
``{[}...{]} a leitura do texto literário, que ocupa o centro do trabalho
no Ensino Fundamental, deve permanecer nuclear também no Ensino Médio.
Por força de certa simplificação didática, as biografias de autores, as
características de épocas, os resumos e outros gêneros artísticos
substitutivos, como o cinema e as HQs, têm relegado o texto literário a
um plano secundário do ensino. Assim, é importante não só (re)colocá-lo
como ponto de partida para o trabalho com a literatura, como
intensificar seu convívio com os estudantes. Como linguagem
artisticamente organizada, a literatura enriquece nossa percepção e
nossa visão de mundo. Mediante arranjos especiais das palavras, ela cria
um universo que nos permite aumentar nossa capacidade de ver e sentir.
Nesse sentido, a literatura possibilita uma ampliação da nossa visão do
mundo, ajuda-nos não só a ver mais, mas a colocar em questão muito do
que estamos vendo/vivenciando.'' (Brasil, 2018, p. 491)
\end{quote}

Nesta seção, desenvolvemos um trabalho de aprofundamento que, em diálogo
com a formação continuada de professores, oferece subsídios para a
abordagem do texto literário.

\subsection{A obra}

\emph{O Surgimento da Noite} é um dos livros da Coleção Mundo Indígena.
O trabalho de pesquisa feito com diferentes povos indígenas nos
apresenta narrativas sobre os mitos de criação sob a perspectiva
Yanomami. O livro conta não somente com as narrativas coletadas, como
também apresenta a versão bilingue, uma vez que esse material foi
organizado para o uso das comunidades Yanomami também.

A obra contém um pequeno material introdutório com a descrição dos
fonemas indígenas. Ele é extremamente importante para nos aproximarmos
dessa cultura e entendermos os sons que deverão ser pronunciados
verdadeiramente durante a leitura.

Apesar do título do livro fazer referência ao surgimento da noite, as
narrativas encontradas ao decorrer da leitura abordam o surgimento de
outros elementos também. Encontraremos o surgimento do tabaco, do cipó e
da banana através das aventuras do personagem \emph{Horonamɨ.}

\emph{Horonamɨ} é um grande pajé que surgiu dele mesmo, assim como é
relatado na narrativa. Surgiu junto com as florestas e ensinou aos
Yanomami como morar nelas. Além de compartilhar os conhecimentos com os
Yanomami, ele também compartilhou suas histórias com os estrangeiros.

``Horonamɨ procurou aquilo que nos permite dormir. Ele fez aquilo que nos 
fará dormir. Aconteceu em toda a floresta. Ele procurou sem desistir, procurou, 
procurou e acabou encontrando essa coisa perto da sua moradia. A cauda da coisa 
já estava visível, pendurada em um galho, mas Horonamɨ pensava que a coisa estaria 
sentada na raiz de uma árvore e continuou procurando longe, em todas as direções.

Não foi a noite que surgiu sozinha, de repente, para nós dormirmos. Assim, quem fez 
não foi outro. Não foi outro que fez anoitecer: foi Horonamɨ, e apenas Horonamɨ, quem 
soprou nosso sono — somente ele.''


\subsection{A importância linguística das línguas indígenas}

O Brasil é composto por uma diversidade linguística muito grande. Há um
total de 154 línguas indígenas faladas por todo o território brasileiro,
mas antes do Brasil colônia, tínhamos entre 600 e 1000 línguas. O
trabalho de registro feito pelos pesquisadores em conjunto com as
comunidades indígenas é de extrema importância para conseguir conservar
a história desses idiomas. Não somente para consultas posteriores, mas
para a organização da própria comunidade.

O livro \emph{O Surgimento da Noite} é a tradução do trabalho de coleta
de histórias do imaginário dos Yanomami. Visando não somente a promoção
e distribuição dessa obra em português, a obra possui a versão original
para utilização da comunidade Yanomami também.

\Image{Mulheres Yanomami do lado amazônico (Wikimedia Commons; CC-BY-SA 3.0)}{PNLD0029-06.png}

É importante lembrarmos que o português se sobressaiu como a língua
oficial do Brasil por se tratar do idioma do colonizador, mas muitas
palavras indígenas foram assimiladas pelos falantes do português.
Infelizmente, algumas línguas indígenas sofreram uma extinção tanto de
seus falantes, como do seu próprio código linguístico.

A família linguística Yanomami, segundo estudos recentes, conforma um 
conjunto cultural e linguístico composto de, pelo menos, cinco subgrupos 
de línguas aparentadas e parcialmente inteligíveis entre si: Yanomam, 
Yanomamɨ, Sanumá, Ninam e Ỹaroamë. 

De acordo com estudos e avanços nas documentações das línguas Yanomami, essa
subdivisão em cinco grupos tem se transformado. Ao longo dos primeiros anos 
de estudos linguísticos sobre esta família, durante a década de 1970, as variedades 
eram agrupadas em quatro subgrupos. Essa classificação, considerada tradicional,
considerava o Ỹaroamë como uma variedade do Ninam. 

Os movimentos de aglutinação e desvinculação acerca das línguas Yanomami são 
característicos dos estudos linguísticos e o fato algumas regiões nunca terem 
sido estudadas e tampouco suas variantes, reafirmam a condição não permanente 
dessas subdivisões. 

\subsection{A lógica cristã e o mito de criação indígena}

Assim que se iniciou a colonização do Brasil, os portugueses acreditavam
que tudo aquilo que não se pareciam com eles poderia ser denominado de
selvagem. A Coroa Portuguesa sempre foi muito conhecida por ser
extremamente católica, logo a sua história é repleta de perseguições e
intolerância em relação as novas religiões.

As missões jesuíticas são as provas de como a intolerância cultural era
altamente disseminada no século XVI e XVII. A metrópole envia padres que
tinham como missão catequizar a comunidade indígena e os fazerem
acreditar na existência de um Deus, em conceitos maniqueístas e a
supremacia da cultura europeia.

Sob a perspectiva cristã, o mundo foi criado em sete dias por um Deus
único e onipresente. Após os sete dias, a humanidade foi criada a partir
de Adão e Eva. Tudo o que existe no mundo foi criado exclusivamente por
um único ser. Quando acompanhamos as narrativas dos Yanomami, não temos
a figura de um Grande Criador e único, mas temos a criação dos elementos
e o surgimento dos fenômenos físicos, como a noite ou o dia, a partir de
aventuras vividas em conjunto pela comunidade.

O cuidado conceitual terminológico ao se utilizar a palavra “criação” para 
narrativas míticas yanomami está vinculada, por certo,  ao fato de não se 
basearem no Gênesis bíblico, na qual o mundo teria sido criado por um certo 
Deus a partir do nada. Omama e seu irmão atuavam mais como reparadores de um 
cosmos dilacerado, enfraquecido e necessitado de reparos, do que como criadores. 
Novos espaços, humanos e outros-que-humanos são produtos de operações de 
transformação e diferenciação sobre algo já existente. 

A ideia de coletivo e a concepção do surgimento de elementos a partir de
ações ou reações é a linha de raciocínio das comunidades. Não existe uma
figura única que deva ser respeitada, mas existe um conjunto de ações da
natureza e todos os seus elementos e quando uma comunidade trabalha em
conjunto, a natureza poderá fornecer novos conceitos e outros novos
elementos.

\subsection{Por que ler \emph{O Surgimento da Noite}}

\emph{O Surgimento da Noite} é uma leitura rápida, mas recheada de
histórias fantásticas. A valorização da natureza e de seus elementos é a
peça fundamental para entender a cultura Yanomami. O modo de narrar
apresentado é muito semelhante as aventuras de Macunaíma, obra de Mario
de Andrade. O autor modernista mergulhou nas histórias brasileiras para
construir seu herói e suas aventuras.

É importante conhecermos e valorizarmos os relatos e as culturas que já
existiam no Brasil antes da colonização. Os registros fonéticos e
fonológicos realizados pelos pesquisadores são como um tesouro que devem
ser compartilhados e bem cuidados. Ler as narrativas Yanomami é poder
se conectar com o Brasil anterior aquele que aprendemos nas aulas de
história.

\Image{Mulher Yanomami tecendo cesta, 1999 (Cmacauley; CC-BY-SA 3.0)}{PNLD0029-07.png}

No Ensino Médio, da mesma forma que no Ensino Fundamental, a BNCC
organiza o trabalho com as práticas de linguagem em cinco \textbf{campos
de atuação social}. São eles: campo da vida pessoal, campo da vida
pública, campo jornalístico-midiático, campo artístico-literário e campo
das práticas de estudo e pesquisa.

De acordo com essa divisão, propomos na sequência um trabalho
interdiscursivo e intertextual com a obra \emph{O Surgimento da Noite} e o \textbf{campo da vida pessoal}

``O campo da vida pessoal pretende funcionar como espaço de articulações
e sínteses das aprendizagens de outros campos postas a serviço dos
projetos de vida dos estudantes. As práticas de linguagem privilegiadas
nesse campo relacionam-se com a ampliação do saber sobre si, tendo em
vista as condições que cercam a vida contemporânea e as condições
juvenis no Brasil e no mundo.

Está em questão também possibilitar vivências significativas de práticas
colaborativas em situações de interação presenciais ou em ambientes
digitais e aprender, na articulação com outras áreas, campos e com os
projetos e escolhas pessoais dos jovens, procedimentos de levantamento,
tratamento e divulgação de dados e informações e o uso desses dados em
produções diversas e na proposição de ações e projetos de natureza
variada, para fomentar o protagonismo juvenil de forma
contextualizada.'' (BNCC, p. 494)

\begin{itemize}
\item
  Apesar do grande número de diversidade linguística e comunidades
  indígenas, essa população ainda vive marginalizada e é vista muitas
  vezes de forma negativa. Durante a pandemia do coronavírus em 2020, a
  população indígena também sofreu grandes impactos, uma vez que essa
  comunidade não vive isolada do mundo -- estereótipo criado para
  estimular o preconceito. O isolamento social das comunidades foi
  vivido de forma diferente das populações das grandes cidades. Sugira a
  leitura do texto \emph{O Amanhã não está à venda}, de Ailton Krekak, e a
  pesquisa sobre os impactos das enfermidades advindas de outros países
  que prejudicaram a comunidade indígena.
\end{itemize}


\subsection{Campo de atuação na vida pública}

``No cerne do campo de atuação na vida pública estão a ampliação da
participação em diferentes instâncias da vida pública, a defesa dos
direitos, o domínio básico de textos legais e a discussão e o debate de
ideias, propostas e projetos. {[}...{]}

Ainda no domínio das ênfases, indica-se um conjunto de habilidades que
se relacionam com a análise, discussão, elaboração e desenvolvimento de
propostas de ação e de projetos culturais e de intervenção social.''
(BNCC, p. 494)

\begin{itemize}
\item
  As narrativas indígenas da comunidade Yanomami sobre a criação do
  mundo não contempla um único ser onipresente, capaz de criar a
  infinidade de elementos da natureza. Encontramos personagens capazes
  de criar elementos através da resolução de conflitos e estão sujeitos
  a mortalidade do mundo. Assim que se iniciou a colonização do Brasil,
  a Coroa Portuguesa enviou os jesuítas para o projeto de catequização
  da comunidade indígena, não respeitando a crença daqueles que já
  habitam o território. Proponha aos alunos o exercício de se tornarem
  um personagem ocular na história da colonização e que tenha recebido a
  missão de argumentar, em uma carta que deverá ser enviada ao Rei de
  Portugal, sobre o quão negativo poderá ser a catequização da
  comunidade indígena e da importância da tolerância religiosa. A ideia
  é estimular aos alunos a importância de respeitar as diversas crenças
  e trabalhar o gênero textual da escrita de uma carta.
\end{itemize}


\subsection{Campo jornalístico-midiático }

``Em relação ao campo jornalístico-midiático, espera-se que os jovens
que chegam ao Ensino Médio sejam capazes de: compreender os fatos e
circunstâncias principais relatados; perceber a impossibilidade de
neutralidade absoluta no relato de fatos; adotar procedimentos básicos
de checagem de veracidade de informação; identificar diferentes pontos
de vista diante de questões polêmicas de relevância social; avaliar
argumentos utilizados e posicionar-se em relação a eles de forma ética;
identificar e denunciar discursos de ódio e que envolvam desrespeito aos
Direitos Humanos; e produzir textos jornalísticos variados, tendo em
vista seus contextos de produção e características dos gêneros. Eles
também devem ter condições de analisar estratégias
linguístico-discursivas utilizadas pelos textos publicitários e de
refletir sobre necessidades e condições de consumo.

No Ensino Médio, os jovens precisam aprofundar a análise dos interesses
que movem o campo jornalístico midiático, da relação entre informação e
opinião, com destaque para o fenômeno da pós-verdade, consolidar o
desenvolvimento de habilidades, apropriar-se de mais procedimentos
envolvidos na curadoria de informações, ampliar o contato com projetos
editoriais independentes e tomar consciência de que uma mídia
independente e plural é condição indispensável para a democracia.

Como já destacado, as práticas que têm lugar nas redes sociais têm
tratamento ampliado.'' (BNCC, p. 494-495)

\begin{itemize}
\item
  O Brasil possui muitas comunidades indígenas espalhadas pelo
  território. Algumas comunidades possuem acesso à energia elétrica e
  outros equipamentos de infraestrutura advindos da industrialização, ainda que de forma
  precária. Algumas outras comunidades ainda são de difícil acesso, mas
  possuem um grande contato com a cidade e continuaram mantendo suas
  tradições ,precisando lutar para proteger o seu espaço territorial por
  direito. O professor poderá propor a montagem de um pequeno
  documentário em que os alunos deverão relatar os costumes de uma
  comunidade indígena próxima da sua cidade, região ou estado. A ideia é
  conseguir entender qual a importância da preservação das comunidades
  indígenas e como elas estão se inserindo no mundo globalizado. Lembrando sempre que 
  a ontologia ameríndia não considera a transformação como negativa, mas sim indelével, assim, 
  poderia se dizer que a tradição, para esses povos, se mantém por meio da capacidade de seguir
  se transformando e se restabelecendo.
\end{itemize}

\subsection{Campo artístico-literário }

``No campo artístico-literário busca-se a ampliação do contato e a
análise mais fundamentada de manifestações culturais e artísticas em
geral. Está em jogo a continuidade da formação do leitor literário e do
desenvolvimento da fruição. A análise contextualizada de produções
artísticas e dos textos literários, com destaque para os clássicos,
intensifica-se no Ensino Médio. Gêneros e formas diversas de produções
vinculadas à apreciação de obras artísticas e produções culturais
(resenhas, vlogs e podcasts literários, culturais etc.) ou a formas de
apropriação do texto literário, de produções cinematográficas e teatrais
e de outras manifestações artísticas (remidiações, paródias,
estilizações, videominutos, fanfics etc.) continuam a ser considerados
associados a habilidades técnicas e estéticas mais refinadas.

A escrita literária, por sua vez, ainda que não seja o foco central do
componente de Língua Portuguesa, também se mostra rica em possibilidades
expressivas.'' (BNCC, p. 495-496)

\begin{itemize}
\item
  A obra \emph{O Surgimento da Noite} apresenta narrativas recheadas de
  detalhes sobre as aventuras daqueles que surgiram antes da comunidade
  e situações que acarretaram a criação de algum elemento da natureza ou
  dos seres humanos. O professor de língua portuguesa em parceria com os
  professores de Educação Artística, poderão propor a criação de um
  painel composto com a pintura de uma ou mais narrativa e a
  apresentação em forma de teatro. O painel poderá ser feito em conjunto
  com as outras turmas e a apresentação de teatro poderá ser apresentada
  para os pais. Algumas narrativas possuem um conteúdo que poderá ser
  mal interpretado pelos pais e pelos próprios alunos, dessa forma é
  importante que os professores criem um ambiente de debate anterior a
  montagem do painel e da peça de teatro.
\end{itemize}

\subsection{Campo das práticas de estudo e pesquisa }

``O campo das práticas de estudo e pesquisa mantém destaque para os
gêneros e habilidades envolvidos na leitura/escuta e produção de textos
de diferentes áreas do conhecimento e para as habilidades e
procedimentos envolvidos no estudo. Ganham realce também as habilidades
relacionadas à análise, síntese, reflexão, problematização e pesquisa:
estabelecimento de recorte da questão ou problema; seleção de
informações; estabelecimento das condições de coleta de dados para a
realização de levantamentos; realização de pesquisas de diferentes
tipos; tratamento dos dados e informações; e formas de uso e
socialização dos resultados e análises.

Além de fazer uso competente da língua e das outras semioses, os
estudantes devem ter uma atitude investigativa e criativa em relação a
elas e compreender princípios e procedimentos metodológicos que orientam
a produção do conhecimento sobre a língua e as linguagens e a formulação
de regras.'' (BNCC, p. 495-496)

\begin{itemize}
\item
  A coletânea de narrativas da comunidade Yanomami se concretizou
  através do trabalho de coleta de dados que os pesquisadores da área de
  linguística empenharam em conjunto com a própria comunidade. A língua
  possui um caráter muito maior do que apenas o exercício físico do
  aparelho fonador, carregando consigo experiencias e visões de mundo.
  Proponha aos alunos a pesquisa sobre os troncos linguísticos indígenas
  existentes no Brasil e, a partir da pesquisa, elaborar uma pequena
  monografia que contemple a importância da língua enquanto fenômeno
  social. O professor poderá sugerir a leitura da reportagem com as
  professora Ana Cabral e Ana Muller, ambas responsáveis por projetos de
  pesquisa linguística indígena, deram para o site UOL falando sobre as
  especificidades de uma determinada língua.
  https://tab.uol.com.br/noticias/redacao/2020/10/11/saudade-ansiedade-aconchego-como-a-lingua-influencia-cultura-e-sensacoes.htm\textbf{\\
  }
\end{itemize}

\subsection{Novos caminhos:} Referências complementares

\textbf{CARVALHO, Bernardo. Nove noites. São Paulo: Companhia de Bolso,
2006.}

Fruto de profunda pesquisa, o livro narra a história do antropólogo
americano Buell Quain, que se matou em 1939, aos 27 anos, enquanto
tentava voltar para a civilização, vindo de uma aldeia indígena no
interior do Brasil.

\textbf{CHATWIN, Bruce. O rastro dos cantos. São Paulo: Companhia das
Letras, 1996.}

O escritor vai atrás do rastro dos cantos, ligado aos mitos de
aborígines da Austrália Central sobre seres legendários que atravessaram
o continente no tempo da criação, cantando o que viam e dando existência
ao mundo através do canto.

\textbf{KRENAK, Ailton. Ideias para adiar o fim do mundo. São Paulo:
Companhia das Letras, 2019.}

O líder indígena critica a ideia de humanidade como algo separado da
natureza e recusa a ideia do humano como superior aos demais seres.

\textbf{MUNDURUKU, Daniel. Contos indígenas brasileiros. São Paulo:
Global Editora, 2004.}

Os oito contos selecionados pelo autor, a partir de um critério
linguístico, retratam através de seus mitos a caminhada de povos
indígenas de norte a sul do Brasil.

\subsection{filmes: }

\textbf{Como fotografei os Yanomami. Direção: Otavio Cury (Brasil,
2018).}

O documentário mostra como é a vida de enfermeiros e técnicos que
atendem os Yanomami em pequenos abrigos isolados.

\textbf{Xingu. Direção: Cao Hamburger (Brasil, 2012).}

O filme conta a história dos três irmãos Villas Bôas, que resolveram
trocar o conforto da cidade grande pelas aventuras na floresta e
acabaram se tornando personagens centrais na criação do atual Parque
Nacional do Xingu.

\textbf{O abraço da Serpente. Direção: Ciro Guerra (Colômbia, 2016).}

Situado durante o período da Febra da Borracha, em 1909 e 1940, o filme 
retrata dois momentos da história de Karamakate, um xamã amazônico e último 
sobrevivente de seu povo, e sua experiência de contato com dois cientistas, 
um alemão Theodor Koch-Grünberg e um americano Richard Evans Schultes, que 
vão até ele em busca da yakruna, uma planta sagrada. 

\textbf{Ex-Pajé. Direção: Luiz Bolognesi (Brasil, 2018).}

O filme conta a história do povo Paiter Suruí, da terra indígena Sete de 
Setembro, em Rondônia. O protagonista, Perpera, tinha 20 anos quando seu povo 
teve oprimeiro contato com não-indígenas, em 1969. Até esse momento, Perpera 
era o pajé de seu povo, no entanto, com o contato, veio também um pastor 
evangélico que condenava o xamanismo, vendo-se obrigado a abandonar sua ancentralidade.

\textbf{A febre. Direção: Maya Da-Rin (Brasil, 2019).}

Em Manaus, no Amazonas, o filme ``A Febre'' acompanha a personagem Justino, um
indígena que há bastante tempo vive na cidade, trabalhando como segurança no 
porto local. A filha de Justino, Vanessa trabalha em um posto de saúde e consegue 
ingressar na Faculdade de Medicina da Universidade de Brasília. 
A trama gira em torno da dúvida de Vanessa seguir seu sonho e abadonar seu pai, com 
ainda o surgimento de uma febra estranha que aparece no local, paralelamente a uma 
série de ataques a animais.

\textbf{Chuva é cantoria na aldeira dos mortos. Direção: Renée Nader (Brasil, 2019).}

O filme conta a história de Ihjãc, um jovem da etnia Krahô, que mora na aldeia Pedra 
Branca, no Tocantins. Após a morte de seu pai, recusando a ideia de se tornar um xamã, 
ele busca refúgio na cidade. Distante de seu povo e da sua cultura, com uma fluidez 
entre momentos ficcionais e documentais, Ihjãc passa a enfrentear as dificuldades de 
ser um indígena habitando os espaços das cidades no Brasil contemporâneo.

\subsection{para visitar: }\subsection{museu do índio}
(\href{http://www.museudoindio.gov.br/}{{museudoindio.gov.br}})

O museu possui um acervo com milhares de peças, além de biblioteca,
galeria de arte e espaços agradáveis para receber os visitantes. Fica no
bairro do Botafogo, no Rio de Janeiro.

\subsection{\emph{site}:}

\href{https://vimeo.com/xapono?fbclid=IwAR2dao6qx16C1AwtQMf4HGTWCCsQ7ItIUtpq1R8ZiNB4EACiVNuJ-mp_gu8}{Xapono – Núcleo Audiovisual Yanomami}

Fundado em parceria com a Associação Yanomami Kurikama, a Fábrica de 
Cinema, a Escola Xapomi e a ONG Rios Profundos (co-fundada por Anne 
Ballester Soares), o Xapono - Núcleo Audiovisual Yanomami é um centro 
de produção, formação e discussão sobre vídeo e cinema enquanto 
ferramentas de produção de narrativas de reinvindicação, luta e 
imaginação de e para os povos indígenas.

\href{http://www.gov.br/funai}{{gov.br/funai}}

O site da Fundação Nacional do Índio é um importante banco de dados
sobre povos e terras indígenas. Também informa sobre meio ambiente e
direitos sociais.

\href{https://www.socioambiental.org/pt-br}{{socioambiental.org/pt-br}}

O site do Instituto Socioambiental – ISA é outro importante banco de dados 
sobre povos e terras indígenas, fornecendo monitoramento e proposições de 
alternativas às políticas públicas, a fim de se assegurar a defesa dos direitos
socioambientais. 

\subsection{Bibliografia comentada}

\textbf{BARAZAL, Neusa Romero. Yanomami: um povo em luta pelos direitos
humanos. São Paulo: Edusp, 2001.}

O livro trata a história de resistência dos Yanomami, considerados
o mais numeroso grupo indígena da floresta a chegar ao século XIX isolada
dos brancos.

\textbf{BARLOW, Genevieve. Stories from Latin America/Historias de
Latinoamerica. Blacklick: McGraw-Hill Education, 2010.}

Este livro bilíngue narra em inglês e em espanhol 16 contos da
Argentina, Bolívia, Colômbia, Guatemala, Paraguai, Peru e outros países
do continente americano.

\textbf{COSTA, da. Os Melhores Contos da América Latina. Rio de Janeiro: Agir,
2008.}


O livro é uma grande antologia de contos da América Latina, reunindo
textos de todos os países da região, das mais diversas épocas e escolas
literárias.

\textbf{ELIADE, Mircea. Cosmos e história: o mito do eterno retorno. São
Paulo: Mercúryo, 2004.}

Este trabalho fundador da história das religiões aborda as expressões e
atividades de uma grande variedade de culturas religiosas arcaicas e
``primitivas''.

\textbf{KOPENAWA, Davi; ALBERT, Bruce. A queda do céu: palavras de um
xamã Yanomami. São Paulo: Companhia das Letras, 2015.}

Xamã e porta-voz dos Yanomami, o autor dá seu testemunho autobiográfico
neste volume, que é também um manifesto xamânico e libelo contra a
destruição da Amazônia.

\textbf{LÉVI-STRAUSS, Claude. Tristes trópicos. São Paulo: Companhia das
Letras, 1996.}

Com um texto que se posiciona entre o ensaio e a narrativa de viagem, o
renomado antropólogo desloca parâmetros consagrados, questionando
viajantes e cientistas.

\textbf{NOGUEIRA, Thyago. Claudia Andujar: a luta Yanomami. São Paulo:
IMS, 2019.}

O catálogo da exposição homônima reúne imagens do trabalho da fotógrafa
dedicado aos Yanomami, retomando aspectos pouco conhecidos de sua
trajetória e luta pela demarcação de terras indígenas.

\textbf{STORTO, Luciana. Línguas indígenas: tradição, universais e
diversidade. Campinas: Mercado de Letras, 2019.}

A autora traça um painel das línguas indígenas atualmente faladas no
Brasil, unindo reflexões da Antropologia e da Linguística.

\textbf{WERÁ, Kaká. A terra dos mil povos: história indígena do Brasil.
São Paulo: Editora Peirópolis, 2020.}

Nesta obra, diversos antropólogos se debruçam sobre a questão de quem
eram e como pensavam os primeiros habitantes do Brasil.

\end{document}

