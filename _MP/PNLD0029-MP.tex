\documentclass[12pt]{extarticle}
\usepackage{manualdoprofessor}
\usepackage{fichatecnica}
\usepackage{lipsum,media9,graficos}
\usepackage[justification=raggedright]{caption}
\usepackage{bncc}
\usepackage[nmenosum]{../edlab}

% \Image{Anne Ballester Soares (Autor Desconhecido; Acervo da Autora)}{PNLD0029-03.png}
% \Image{Xapono, casa comunal habitada pelo Yanomami (Wikimedia Commons; Domínio Público)}{PNLD0029-04.png}
% \Image{Mapa do território Yanomami no Brasil e Venezuela, 2012 (Wikimedia Commons; CC BY-SA 3.0)}{PNLD0029-05.png}
% \Image{Mulheres Yanomami do lado amazônico (Wikimedia Commons; CC BY-SA 3.0)}{PNLD0029-06.png}
% \Image{Mulher Yanomami tecendo cesta, 1999 (Cmacauley; CC BY-SA 3.0)}{PNLD0029-07.png}

\begin{document}


\newcommand{\AutorLivro}{Anne Ballester Soares (Org.)}
\newcommand{\TituloLivro}{O surgimento da noite: mitologias Yanomami*}
\newcommand{\Tema}{Ficção, mistério e fantasia}
\newcommand{\Genero}{Mitologia indígena}
\newcommand{\imagemCapa}{./images/PNLD0029-01.png}
\newcommand{\issnppub}{---}
\newcommand{\issnepub}{---}
% \newcommand{\fichacatalografica}{PNLD0029-00.png}
\newcommand{\colaborador}{\textbf{Fulano de Tal} é uma pessoa incrível e vai fazer um bom serviço.}


\title{\TituloLivro}
\author{\AutorLivro}
\def\authornotes{\colaborador}

\date{}
\maketitle




\begin{abstract}
Este Manual tem como objetivo fornecer subsídios para o trabalho com a
obra literária \emph{O surgimento da noite.}

Este livro foi organizado por Anne Balester Soares, francesa nascida em 1955 
que viveu por 24 anos com os Yanomami.  Enquanto ativista, trabalhou como agente 
de saúde no combate à malária, foi alfabetizadora em língua yanomami e professora 
de português para jovens e adultos em posições de liderança indígena.

\emph{O Surgimento da Noite} apresenta narrativas sobre os mitos de criação segundo 
os Yanomami. O livro conta não somente com as narrativas coletadas, como também 
apresenta a versão bilíngue, uma vez que esse material foi organizado para o uso 
das comunidades Yanomami também. A obra contém um pequeno material introdutório 
com a descrição dos fonemas indígenas. Ele é extremamente importante para nos 
aproximarmos dessa cultura e entendermos os sons que deverão ser pronunciados 
verdadeiramente durante a leitura. 

Apesar do título do livro fazer referência ao surgimento da noite, as narrativas
encontradas ao decorrer da leitura abordam o surgimento de outros elementos
também. Encontraremos o surgimento do tabaco, do cipó e da banana através
das aventuras do personagem Horonamɨ. Horonamɨ é um grande pajé que surgiu de si mesmo, 
assim como é relatado na narrativa. Surgiu junto com as florestas e ensinou aos Yanomami 
como morar nelas. Além de compartilhar os conhecimentos com o povo, ele também
compartilhou suas histórias com os estrangeiros.

A valorização da natureza e de seus elementos é a peça fundamental para entender 
a cultura Yanomami. O modo de narrar apresentado é muito semelhante as aventuras 
de Macunaíma, obra de Mario de Andrade. O autor modernista mergulhou nas histórias 
brasileiras para construir seu herói e suas aventuras.

Esperamos que as indicações propostas aqui sejam muito úteis no trabalho em
sala de aula! 


\end{abstract}


\tableofcontents
\section{Proposta de atividades I}

\subsection{Pré-leitura}


% EM13LGG302
% EM13LGG704
% EM13LP10
% EM13LP19

Para se iniciar os trabalhos com \emph{O Surgimento
da Noite}, sugere-se orientar os alunos à pesquisa de mitos e lendas
indígenas brasileiros. Para uma maior riqueza de detalhes, selecione
etnias específicas e, dividindo a sala em grupos, indique cada uma delas
a um grupo. O objetivo é conseguir compartilhar as narrativas de maneira
didática. A educação indígena no Brasil ainda é muito precária e se faz
necessário programas de ação afirmativa, como o programa de cotas nas
universidades, para que possam garantir a inserção dessa população no
meio acadêmico. Todos os povos detêm, em sua cultura e folclore,
histórias e personagens lendários, compondo essas narrativas suas
identidades. Assim, conhecer e preservar as narrativas é uma forma de
cultivar a cultura e, nesse sentido, as noções e espacialidades outras 
que a cosmologia indígena nos revela coloca-nos em condições de fronteiras 
entre mundos possíveis, possibilitando, por meio delas, expandir nosso 
próprio mundo e nossas próprias percepções.

\SideImage{Anne Ballester Soares (Autor Desconhecido; Acervo da Autora)}{PNLD0029-03.png}

\subsection{Leitura}

% EM13LGG103
% EM13LP02
% EM13LP48

Atualmente vivemos em uma sociedade que possui acesso à
informação de maneira dinâmica. Somos arrebatados a todo instante por
alguma nova notícia. Estar bem-informado é muito importante e o acesso à
informação de forma clara e segura é ainda mais, porém vivemos uma onda
de faltas notícias e devemos tomar muito cuidado com o compartilhamento
de informações falsas. A comunidade indígena também foi atingida pela
desinformação e a criação de boatos. Proponha aos alunos a pesquisa das
notícias falsas com o intuito de desmentir. A ideia é escrever textos
com as verdadeiras notícias, em formato jornalístico, e divulgar no
portal online da escola. O objetivo é criar a consciência nos jovens de
sempre procurar por fontes confiáveis e não compartilhar a
desinformação.

\subsection{Pós-leitura}
% EM13LGG102
% EM13LGG303
% EM13LGG402
% EM13LGG703
% EM13LP13
% EM13LP14
% EM13LP28
% EM13LP29
% EM13LP52

Na pós-leitura é interessante sugerir aos educandos que
produzam textos, a partir das narrativas lidas na obra, só que, dessa
vez, apresentando aos leitores a perspectiva dos protagonistas. Para
isso, é necessária a mudança do eixo narrativo de terceira para primeira
pessoa. Assim, perde-se o elemento da onisciência e há uma maior
dificuldade na criação da atmosfera sentimental e psicológica do
indivíduo. O desafio é justamente observar a compreensão da obra lida
por meio do resultado dessa transposição de narradores, ofertando uma
nova perspectiva sobre o texto lido.


\section{Proposta de atividades II}


A obra \emph{O Surgimento da Noite} possibilita trabalhos
interdisciplinares e integradores de diferentes campos do saber e áreas
de conhecimento. A seguir, propomos algumas atividades que podem ser
desenvolvidas conjuntamente com professores de outras áreas. Além das
habilidades de Linguagens e suas Tecnologias e de Língua Portuguesa,
indicadas nas etapas da seção anterior e válidas também para esta,
listamos a seguir as habilidades de outras áreas, presentes na abordagem
interdisciplinar:


% EM13CNT201
% EM13CNT303
% EM13CHS101
% EM13CHS102
% EM13CHS106
% EM13CHS401

\subsection{Pré-leitura}

Para a atividade de pré-leitura, recomenda-se buscar uma
parceria com os professores de ciências humanas para uma conversa,
levantando elementos e técnicas socioantropológicas que auxiliam na
condução da preservação dos grupos indígenas brasileiros. Observem como
o conhecimento desses pormenores é importante, pois, por meio deles é
possível se chegar a mecanismos mais contundentes de preservação. Ao
final, sugere-se a produção de um mapa etnolinguístico das populações
indígenas do Brasil.

\Image{Mapa do território Yanomami no Brasil e Venezuela, 2012 (Wikimedia Commons; CC BY-SA 3.0)}{PNLD0029-05.png}

\subsection{Leitura}

Durante a leitura, professores de ciências da natureza podem
abordar os mitos presentes no livro, posicionando o pensamento mítico junto ao
pensamento científico.

O território ocupado pelos Yanomami, atualmente, abrange aproximadamente 
192 mil km2 de floresta tropical contínua, no oeste do maciço guianense, 
em ambos os lados da fronteira entre o Brasil e a Venezuela. A população 
total é estimada em 36.000 pessoas, distribuídas em  mais de 600 aldeias. 
No Brasil, os Yanomami somam aproximadamente 22.000 pessoas, distribuídas 
nas mais de 250 aldeias da Terra Indígena Yanomami – a maior Terra Indígena 
do país –, com área aproximada de 96.650 km2, entre os estados de Roraima e 
Amazonas. Yanomami, enquanto grupo indígena e a família linguística, conforma
um conjunto cultural e linguístico composto de cinco subgrupos de línguas 
aparentadas e parcialmente inteligíveis entre si, com diferentes costumes, 
conhecimentos, concepções e construções.

O vínculo e conexão com os animais, águas e montanha dos Yanomami com a 
Terra, faz parte da forma como veem a origem do mundo e remontam seus mitos. 
As narrativas míticas yanomami remetem sua origem à copulação do demiurgo 
Omama com a filha do monstro aquático, dono das plantas cultivadas. À Omama
é atribuída a origem das regras da sociedade e da cultura yanomami atual,
bem como a criação dos espíritos auxiliares dos pajés: os \emph{xapiri}, 
para curá-los das doenças e protege-los da morte. Os \emph{xapiri} são os 
espíritos das águas, espíritos da floresta e espíritos animais, eles descem 
à floresta como uma multiplicidade de imagens de seres brilhantes que cantam 
e dançam, sendo sua aparência como pequenos humanoides.

Para os Yanomami, a floresta é viva, ela pensa e transmite palavras inesgotáveis,
que são a fonte de conhecimento dos \emph{xapiri}. Estes espíritos escutam as 
árvores e são, portanto, o veículo dos saberes da floresta para os xamãs. Os xamãs,
por sua vez, conseguem equilibrar no peito uma multiplicidade de \emph{xapiris}, 
escutam o seu canto e traduzem estes saberes para o resto da sua comunidade.

O filho de Omama foi o primeiro xamã. O irmão ciumento e malvado de Omama,
Yoasi, é a origem da morte e dos males do mundo.
Uma narrativa mítica ensina que os estrangeiros devem também sua
existência aos poderes demiúrgicos de Omama. Conta-se que foram criados
a partir da espuma do sangue de um grupo de ancestrais yanomami levado
por uma enchente e devorado por jacarés e ariranhas.

Tendo retomado essas passagens, sugere-se que, a partir do
substrato oferecido pelos professores de ciências da natureza seja
procedida uma reflexão do porque tais elementos foram priorizados pelos
índios Yanomami em seus mitos de criação.

\Image{Xapono, casa comunal habitada pelo Yanomami (Wikimedia Commons; Domínio Público)}{PNLD0029-04.png}

\subsection{Pós-Leitura}

 Após a leitura, sugere-se que os estudantes produzam um
documentário em vídeo sobre a cosmologia, a língua e os valores de um dos
povos indígenas mapeados anteriormente. Sugere-se, para isso,
inicialmente, a produção de um roteiro, onde restarão determinados os
temas a serem tratados, bem como esses terão divididos os tempos para
cada assunto. A introdução de músicas, vinhetas, momentos de fala,
entrevistas, narrativas com imagens também devem ser pensados.
Sugere-se, para essa atividade a utilização de obras que tratam da
cosmologia e história indígena. Também indica-se a transmissão de filmes
da Anauê Mostra de Cinema Indígena e do forumdoc.bh – Festival do Filme 
Documentário e Etnográfico, fórum de antropologia e cinema –, onde os filmes
são produzidos pelos povos indígenas, o que conferirá ao estudante uma 
oportunidade de compreender o mundo sob a ótica dessas populações. 


\section{Aprofundamento}

Ao chegar ao Ensino Médio, é necessário que os estudantes se aprofundem
na compreensão das múltiplas linguagens e, sobretudo, da linguagem
literária. Em relação à literatura, a BNCC traz as seguintes
considerações:

\begin{quote}
``{[}...{]} a leitura do texto literário, que ocupa o centro do trabalho
no Ensino Fundamental, deve permanecer nuclear também no Ensino Médio.
Por força de certa simplificação didática, as biografias de autores, as
características de épocas, os resumos e outros gêneros artísticos
substitutivos, como o cinema e as HQs, têm relegado o texto literário a
um plano secundário do ensino. Assim, é importante não só (re)colocá-lo
como ponto de partida para o trabalho com a literatura, como
intensificar seu convívio com os estudantes. Como linguagem
artisticamente organizada, a literatura enriquece nossa percepção e
nossa visão de mundo. Mediante arranjos especiais das palavras, ela cria
um universo que nos permite aumentar nossa capacidade de ver e sentir.
Nesse sentido, a literatura possibilita uma ampliação da nossa visão do
mundo, ajuda-nos não só a ver mais, mas a colocar em questão muito do
que estamos vendo/vivenciando.'' (Brasil, 2018, p. 491)
\end{quote}

Nesta seção, desenvolvemos um trabalho de aprofundamento que, em diálogo
com a formação continuada de professores, oferece subsídios para a
abordagem do texto literário.

\subsection{A obra}

\emph{O Surgimento da Noite} é um dos livros da Coleção Mundo Indígena.
O trabalho de pesquisa feito com diferentes povos indígenas nos
apresenta narrativas sobre os mitos de criação sob a perspectiva
Yanomami. O livro conta não somente com as narrativas coletadas, como
também apresenta a versão bilingue, uma vez que esse material foi
organizado para o uso das comunidades Yanomami também.

A obra contém um pequeno material introdutório com a descrição dos
fonemas indígenas. Ele é extremamente importante para nos aproximarmos
dessa cultura e entendermos os sons que deverão ser pronunciados
verdadeiramente durante a leitura.

Apesar do título do livro fazer referência ao surgimento da noite, as
narrativas encontradas ao decorrer da leitura abordam o surgimento de
outros elementos também. Encontraremos o surgimento do tabaco, do cipó e
da banana através das aventuras do personagem \emph{Horonamɨ.}

\emph{Horonamɨ} é um grande pajé que surgiu dele mesmo, assim como é
relatado na narrativa. Surgiu junto com as florestas e ensinou aos
Yanomami como morar nelas. Além de compartilhar os conhecimentos com os
Yanomami, ele também compartilhou suas histórias com os estrangeiros.

\subsection{A importância linguística das línguas indígenas}

O Brasil é composto por uma diversidade linguística muito grande. Há um
total de 154 línguas indígenas faladas por todo o território brasileiro,
mas antes do Brasil colônia, tínhamos entre 600 e 1000 línguas. O
trabalho de registro feito pelos pesquisadores em conjunto com as
comunidades indígenas é de extrema importância para conseguir conservar
a história desses idiomas. Não somente para consultas posteriores, mas
para a organização da própria comunidade.

O livro \emph{O Surgimento da Noite} é a tradução do trabalho de coleta
de histórias do imaginário dos Yanomami. Visando não somente a promoção
e distribuição dessa obra em português, a obra possui a versão original
para utilização da comunidade Yanomami também.

\Image{Mulheres Yanomami do lado amazônico (Wikimedia Commons; CC BY-SA 3.0)}{PNLD0029-06.png}

É importante lembrarmos que o português se sobressaiu como a língua
oficial do Brasil por se tratar do idioma do colonizador, mas muitas
palavras indígenas foram assimiladas pelos falantes do português.
Infelizmente, algumas línguas indígenas sofreram uma extinção tanto de
seus falantes, como do seu próprio código linguístico.

A família linguística Yanomami, segundo estudos recentes, conforma um 
conjunto cultural e linguístico composto de, pelo menos, cinco subgrupos 
de línguas aparentadas e parcialmente inteligíveis entre si: Yanomam, 
Yanomamɨ, Sanumá, Ninam e Ỹaroamë. 

\subsection{A lógica cristã e o mito de criação indígena}

Assim que se iniciou a colonização do Brasil, os portugueses acreditavam
que tudo aquilo que não se pareciam com eles poderia ser denominado de
selvagem. A Coroa Portuguesa sempre foi muito conhecida por ser
extremamente católica, logo a sua história é repleta de perseguições e
intolerância em relação as novas religiões.

As missões jesuíticas são as provas de como a intolerância cultural era
altamente disseminada no século XVI e XVII. A metrópole envia padres que
tinham como missão catequizar a comunidade indígena e os fazerem
acreditar na existência de um Deus, em conceitos maniqueístas e a
supremacia da cultura europeia.

Sob a perspectiva cristão, o mundo foi criado em sete dias por um Deus
único e onipresente. Após os sete dias, a humanidade foi criada a partir
de Adão e Eva. Tudo o que existe no mundo foi criado exclusivamente por
um único ser. Quando acompanhamos as narrativas dos Yanomami, não temos
a figura de um Grande Criador e único, mas temos a criação dos elementos
e o surgimento dos fenômenos físicos, como a noite ou o dia, a partir de
aventuras vividas em conjunto pela comunidade.

O cuidado conceitual terminológico ao se utilizar a palavra “criação” para 
narrativas míticas yanomami está vinculada, por certo,  ao fato de não se 
basearem no Gênesis bíblico, na qual o mundo teria sido criado por um certo 
Deus a partir do nada. Omama e seu irmão atuavam mais como reparadores de um 
cosmos dilacerado, enfraquecido e necessitado de reparos, do que como criadores. 
Novos espaços, humanos e outros-que-humanos são produtos de operações de 
transformação e diferenciação sobre algo já existente. 

A ideia de coletivo e a concepção do surgimento de elementos a partir de
ações ou reações é a linha de raciocínio das comunidades. Não existe uma
figura única que deva ser respeitada, mas existe um conjunto de ações da
natureza e todos os seus elementos e quando uma comunidade trabalha em
conjunto, a natureza poderá fornecer novos conceitos e outros novos
elementos.

\subsection{Por que ler \emph{O Surgimento da Noite}}

\emph{O Surgimento da Noite} é uma leitura rápida, mas recheada de
histórias fantásticas. A valorização da natureza e de seus elementos é a
peça fundamental para entender a cultura Yanomami. O modo de narrar
apresentado é muito semelhante as aventuras de Macunaíma, obra de Mario
de Andrade. O autor modernista mergulhou nas histórias brasileiras para
construir seu herói e suas aventuras.

É importante conhecermos e valorizarmos os relatos e as culturas que já
existiam no Brasil antes da colonização. Os registros fonéticos e
fonológicos realizados pelos pesquisadores são como um tesouro que devem
ser compartilhados e bem cuidados. Ler as narrativas Yanomami é poder
se conectar com o Brasil anterior aquele que aprendemos nas aulas de
história.

\Image{Mulher Yanomami tecendo cesta, 1999 (Cmacauley; CC BY-SA 3.0)}{PNLD0029-07.png}

No Ensino Médio, da mesma forma que no Ensino Fundamental, a BNCC
organiza o trabalho com as práticas de linguagem em cinco \textbf{campos
de atuação social}. São eles: campo da vida pessoal, campo da vida
pública, campo jornalístico-midiático, campo artístico-literário e campo
das práticas de estudo e pesquisa.

De acordo com essa divisão, propomos na sequência um trabalho
interdiscursivo e intertextual com a obra \emph{O Surgimento da Noite} e o \textbf{campo da vida pessoal}

``O campo da vida pessoal pretende funcionar como espaço de articulações
e sínteses das aprendizagens de outros campos postas a serviço dos
projetos de vida dos estudantes. As práticas de linguagem privilegiadas
nesse campo relacionam-se com a ampliação do saber sobre si, tendo em
vista as condições que cercam a vida contemporânea e as condições
juvenis no Brasil e no mundo.

Está em questão também possibilitar vivências significativas de práticas
colaborativas em situações de interação presenciais ou em ambientes
digitais e aprender, na articulação com outras áreas, campos e com os
projetos e escolhas pessoais dos jovens, procedimentos de levantamento,
tratamento e divulgação de dados e informações e o uso desses dados em
produções diversas e na proposição de ações e projetos de natureza
variada, para fomentar o protagonismo juvenil de forma
contextualizada.'' (BNCC, p. 494)

\begin{itemize}
\item
  Apesar do grande número de diversidade linguística e comunidades
  indígenas, essa população ainda vive marginalizada e é vista muitas
  vezes de forma negativa. Durante a pandemia do coronavírus em 2020, a
  população indígena também sofreu grandes impactos, uma vez que essa
  comunidade não vive isolada do mundo -- estereótipo criado para
  estimular o preconceito. O isolamento social das comunidades foi
  vivido de forma diferente das populações das grandes cidades. Sugira a
  leitura do texto \emph{O Amanhã não está à venda}, de Ailton Krekak, e a
  pesquisa sobre os impactos das enfermidades advindas de outros países
  que prejudicaram a comunidade indígena.
\end{itemize}


\subsection{Campo de atuação na vida pública}

``No cerne do campo de atuação na vida pública estão a ampliação da
participação em diferentes instâncias da vida pública, a defesa dos
direitos, o domínio básico de textos legais e a discussão e o debate de
ideias, propostas e projetos. {[}...{]}

Ainda no domínio das ênfases, indica-se um conjunto de habilidades que
se relacionam com a análise, discussão, elaboração e desenvolvimento de
propostas de ação e de projetos culturais e de intervenção social.''
(BNCC, p. 494)

\begin{itemize}
\item
  As narrativas indígenas da comunidade Yanomami sobre a criação do
  mundo não contempla um único ser onipresente, capaz de criar a
  infinidade de elementos da natureza. Encontramos personagens capazes
  de criar elementos através da resolução de conflitos e estão sujeitos
  a mortalidade do mundo. Assim que se iniciou a colonização do Brasil,
  a Coroa Portuguesa enviou os jesuítas para o projeto de catequização
  da comunidade indígena, não respeitando a crença daqueles que já
  habitam o território. Proponha aos alunos o exercício de se tornarem
  um personagem ocular na história da colonização e que tenha recebido a
  missão de argumentar, em uma carta que deverá ser enviada ao Rei de
  Portugal, sobre o quão negativo poderá ser a catequização da
  comunidade indígena e da importância da tolerância religiosa. A ideia
  é estimular aos alunos a importância de respeitar as diversas crenças
  e trabalhar o gênero textual da escrita de uma carta.
\end{itemize}


\subsection{Campo jornalístico-midiático }

``Em relação ao campo jornalístico-midiático, espera-se que os jovens
que chegam ao Ensino Médio sejam capazes de: compreender os fatos e
circunstâncias principais relatados; perceber a impossibilidade de
neutralidade absoluta no relato de fatos; adotar procedimentos básicos
de checagem de veracidade de informação; identificar diferentes pontos
de vista diante de questões polêmicas de relevância social; avaliar
argumentos utilizados e posicionar-se em relação a eles de forma ética;
identificar e denunciar discursos de ódio e que envolvam desrespeito aos
Direitos Humanos; e produzir textos jornalísticos variados, tendo em
vista seus contextos de produção e características dos gêneros. Eles
também devem ter condições de analisar estratégias
linguístico-discursivas utilizadas pelos textos publicitários e de
refletir sobre necessidades e condições de consumo.

No Ensino Médio, os jovens precisam aprofundar a análise dos interesses
que movem o campo jornalístico midiático, da relação entre informação e
opinião, com destaque para o fenômeno da pós-verdade, consolidar o
desenvolvimento de habilidades, apropriar-se de mais procedimentos
envolvidos na curadoria de informações, ampliar o contato com projetos
editoriais independentes e tomar consciência de que uma mídia
independente e plural é condição indispensável para a democracia.

Como já destacado, as práticas que têm lugar nas redes sociais têm
tratamento ampliado.'' (BNCC, p. 494-495)

\begin{itemize}
\item
  O Brasil possui muitas comunidades indígenas espalhadas pelo
  território. Algumas comunidades possuem acesso à energia elétrica e
  outros equipamentos de infraestrutura advindos da industrialização, ainda que de forma
  precária. Algumas outras comunidades ainda são de difícil acesso, mas
  possuem um grande contato com a cidade e continuaram mantendo suas
  tradições ,precisando lutar para proteger o seu espaço territorial por
  direito. O professor poderá propor a montagem de um pequeno
  documentário em que os alunos deverão relatar os costumes de uma
  comunidade indígena próxima da sua cidade, região ou estado. A ideia é
  conseguir entender qual a importância da preservação das comunidades
  indígenas e como elas estão se inserindo no mundo globalizado. Lembrando sempre que 
  a ontologia ameríndia não considera a transformação como negativa, mas sim indelével, assim, 
  poderia se dizer que a tradição, para esses povos, se mantém por meio da capacidade de seguir
  se transformando e se restabelecendo.
\end{itemize}

\subsection{Campo artístico-literário }

``No campo artístico-literário busca-se a ampliação do contato e a
análise mais fundamentada de manifestações culturais e artísticas em
geral. Está em jogo a continuidade da formação do leitor literário e do
desenvolvimento da fruição. A análise contextualizada de produções
artísticas e dos textos literários, com destaque para os clássicos,
intensifica-se no Ensino Médio. Gêneros e formas diversas de produções
vinculadas à apreciação de obras artísticas e produções culturais
(resenhas, vlogs e podcasts literários, culturais etc.) ou a formas de
apropriação do texto literário, de produções cinematográficas e teatrais
e de outras manifestações artísticas (remidiações, paródias,
estilizações, videominutos, fanfics etc.) continuam a ser considerados
associados a habilidades técnicas e estéticas mais refinadas.

A escrita literária, por sua vez, ainda que não seja o foco central do
componente de Língua Portuguesa, também se mostra rica em possibilidades
expressivas.'' (BNCC, p. 495-496)

\begin{itemize}
\item
  A obra \emph{O Surgimento da Noite} apresenta narrativas recheadas de
  detalhes sobre as aventuras daqueles que surgiram antes da comunidade
  e situações que acarretaram a criação de algum elemento da natureza ou
  dos seres humanos. O professor de língua portuguesa em parceria com os
  professores de Educação Artística, poderão propor a criação de um
  painel composto com a pintura de uma ou mais narrativa e a
  apresentação em forma de teatro. O painel poderá ser feito em conjunto
  com as outras turmas e a apresentação de teatro poderá ser apresentada
  para os pais. Algumas narrativas possuem um conteúdo que poderá ser
  mal interpretado pelos pais e pelos próprios alunos, dessa forma é
  importante que os professores criem um ambiente de debate anterior a
  montagem do painel e da peça de teatro.
\end{itemize}

\subsection{Campo das práticas de estudo e pesquisa }

``O campo das práticas de estudo e pesquisa mantém destaque para os
gêneros e habilidades envolvidos na leitura/escuta e produção de textos
de diferentes áreas do conhecimento e para as habilidades e
procedimentos envolvidos no estudo. Ganham realce também as habilidades
relacionadas à análise, síntese, reflexão, problematização e pesquisa:
estabelecimento de recorte da questão ou problema; seleção de
informações; estabelecimento das condições de coleta de dados para a
realização de levantamentos; realização de pesquisas de diferentes
tipos; tratamento dos dados e informações; e formas de uso e
socialização dos resultados e análises.

Além de fazer uso competente da língua e das outras semioses, os
estudantes devem ter uma atitude investigativa e criativa em relação a
elas e compreender princípios e procedimentos metodológicos que orientam
a produção do conhecimento sobre a língua e as linguagens e a formulação
de regras.'' (BNCC, p. 495-496)

\begin{itemize}
\item
  A coletânea de narrativas da comunidade Yanomami se concretizou
  através do trabalho de coleta de dados que os pesquisadores da área de
  linguística empenharam em conjunto com a própria comunidade. A língua
  possui um caráter muito maior do que apenas o exercício físico do
  aparelho fonador, carregando consigo experiencias e visões de mundo.
  Proponha aos alunos a pesquisa sobre os troncos linguísticos indígenas
  existentes no Brasil e, a partir da pesquisa, elaborar uma pequena
  monografia que contemple a importância da língua enquanto fenômeno
  social. O professor poderá sugerir a leitura da reportagem com as
  professora Ana Cabral e Ana Muller, ambas responsáveis por projetos de
  pesquisa linguística indígena, deram para o site UOL falando sobre as
  especificidades de uma determinada língua.
  https://tab.uol.com.br/noticias/redacao/2020/10/11/saudade-ansiedade-aconchego-como-a-lingua-influencia-cultura-e-sensacoes.htm\textbf{\\
  }
\end{itemize}

\subsection{Novos caminhos:} Referências complementares

\textbf{CARVALHO, Bernardo. Nove noites. São Paulo: Companhia de Bolso,
2006.}

Fruto de profunda pesquisa, o livro narra a história do antropólogo
americano Buell Quain, que se matou em 1939, aos 27 anos, enquanto
tentava voltar para a civilização, vindo de uma aldeia indígena no
interior do Brasil.

\textbf{CHATWIN, Bruce. O rastro dos cantos. São Paulo: Companhia das
Letras, 1996.}

O escritor vai atrás do rastro dos cantos, ligado aos mitos de
aborígines da Austrália Central sobre seres legendários que atravessaram
o continente no tempo da criação, cantando o que viam e dando existência
ao mundo através do canto.

\textbf{KRENAK, Ailton. Ideias para adiar o fim do mundo. São Paulo:
Companhia das Letras, 2019.}

O líder indígena critica a ideia de humanidade como algo separado da
natureza e recusa a ideia do humano como superior aos demais seres.

\textbf{MUNDURUKU, Daniel. Contos indígenas brasileiros. São Paulo:
Global Editora, 2004.}

Os oito contos selecionados pelo autor, a partir de um critério
linguístico, retratam através de seus mitos a caminhada de povos
indígenas de norte a sul do Brasil.

\subsection{filmes: }

\textbf{Como fotografei os Yanomami. Direção: Otavio Cury (Brasil,
2018).}

O documentário mostra como é a vida de enfermeiros e técnicos que
atendem os Yanomami em pequenos abrigos isolados.

\subsection{xingu. direção: cao hamburger (brasil, 2012).}

O filme conta a história dos três irmãos Villas Bôas, que resolveram
trocar o conforto da cidade grande pelas aventuras na floresta e
acabaram se tornando personagens centrais na criação do atual Parque
Nacional do Xingu.

\subsection{para visitar: }\subsection{museu do índio}
(\href{http://www.museudoindio.gov.br/}{{museudoindio.gov.br}})

O museu possui um acervo com milhares de peças, além de biblioteca,
galeria de arte e espaços agradáveis para receber os visitantes. Fica no
bairro do Botafogo, no Rio de Janeiro.

\subsection{\emph{site}:}

\href{http://www.gov.br/funai}{{gov.br/funai}}

O site da Fundação Nacional do Índio é um importante banco de dados
sobre povos e terras indígenas. Também informa sobre meio ambiente e
direitos sociais.

\href{https://www.socioambiental.org/pt-br}{{socioambiental.org/pt-br}}

O site do Instituto Socioambiental – ISA é outro importante banco de dados 
sobre povos e terras indígenas, fornecendo monitoramento e proposições de 
alternativas às políticas públicas, a fim de se assegurar a defesa dos direitos
socioambientais. 

\subsection{Bibliografia comentada}

\textbf{BARAZAL, Neusa Romero. Yanomami: um povo em luta pelos direitos
humanos. São Paulo: Edusp, 2001.}

O livro trata a história de resistência dos Yanomami, considerados
o mais numeroso grupo indígena da floresta a chegar ao século XIX isolada
dos brancos.

\textbf{BARLOW, Genevieve. Stories from Latin America/Historias de
Latinoamerica. Blacklick: McGraw-Hill Education, 2010.}

Este livro bilíngue narra em inglês e em espanhol 16 contos da
Argentina, Bolívia, Colômbia, Guatemala, Paraguai, Peru e outros países
do continente americano.

\textbf{COSTA, da. Os Melhores Contos da América Latina. Rio de Janeiro: Agir,
2008.}


O livro é uma grande antologia de contos da América Latina, reunindo
textos de todos os países da região, das mais diversas épocas e escolas
literárias.

\textbf{ELIADE, Mircea. Cosmos e história: o mito do eterno retorno. São
Paulo: Mercúryo, 2004.}

Este trabalho fundador da história das religiões aborda as expressões e
atividades de uma grande variedade de culturas religiosas arcaicas e
``primitivas''.

\textbf{KOPENAWA, Davi; ALBERT, Bruce. A queda do céu: palavras de um
xamã Yanomami. São Paulo: Companhia das Letras, 2015.}

Xamã e porta-voz dos Yanomami, o autor dá seu testemunho autobiográfico
neste volume, que é também um manifesto xamânico e libelo contra a
destruição da Amazônia.

\textbf{LÉVI-STRAUSS, Claude. Tristes trópicos. São Paulo: Companhia das
Letras, 1996.}

Com um texto que se posiciona entre o ensaio e a narrativa de viagem, o
renomado antropólogo desloca parâmetros consagrados, questionando
viajantes e cientistas.

\textbf{NOGUEIRA, Thyago. Claudia Andujar: a luta Yanomami. São Paulo:
IMS, 2019.}

O catálogo da exposição homônima reúne imagens do trabalho da fotógrafa
dedicado aos Yanomami, retomando aspectos pouco conhecidos de sua
trajetória e luta pela demarcação de terras indígenas.

\textbf{STORTO, Luciana. Línguas indígenas: tradição, universais e
diversidade. Campinas: Mercado de Letras, 2019.}

A autora traça um painel das línguas indígenas atualmente faladas no
Brasil, unindo reflexões da Antropologia e da Linguística.

\textbf{WERÁ, Kaká. A terra dos mil povos: história indígena do Brasil.
São Paulo: Editora Peirópolis, 2020.}

Nesta obra, diversos antropólogos se debruçam sobre a questão de quem
eram e como pensavam os primeiros habitantes do Brasil.

\end{document}

