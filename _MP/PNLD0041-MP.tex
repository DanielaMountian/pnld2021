\documentclass[11pt]{extarticle}
\usepackage{manualdoprofessor}
\usepackage{fichatecnica}
\usepackage{lipsum,media9,graficos}
\usepackage[justification=raggedright]{caption}
\usepackage[one]{bncc}
\usepackage[maisemelhores]{../edlab}

\begin{document}


\newcommand{\AutorLivro}{Johann Ludwig Tieck}
\newcommand{\TituloLivro}{Feitiço de amor e outros contos}
\newcommand{\Tema}{Ficção, mistério e fantasia}
\newcommand{\Genero}{Conto, crônica e novela}
\newcommand{\imagemCapa}{./images/PNLD0041-01.png}
\newcommand{\issnppub}{---}
\newcommand{\issnepub}{---}
% \newcommand{\fichacatalografica}{PNLD0041-00.png}
\newcommand{\colaborador}{\textbf{Michelle Etienne Florence, Bruno Gradella e Vicente Castro}}


\title{\TituloLivro}
\author{\AutorLivro}
\def\authornotes{\colaborador}

\date{}
\maketitle

\baselineskip=1.2\baselineskip\par

\begin{abstract}
Este Manual tem como objetivo fornecer subsídios para o trabalho com a
obra literária \emph{Feitiços de Amor e Outros Contos}, de Ludwig Tieck.

Johann Ludwig Tieck, nascido em Berlim no ano de 1773, foi um dos mais 
importantes fundadores do movimento romântico alemão. Durante trinta anos, 
rivalizou com Goethe: muitos consideravam-no o verdadeiro centro da
literatura alemã. 

Entre suas primeiras narrativas estão os romances \emph{A história de William 
Lovell}, 1795–96 e \emph{Peregrinações de Franz Sternbald}, 1798. Além de poeta 
e escritor, destacou-se como editor e tradutor. Em 1834, consagrou a Camões
a novela histórica \emph{Morte do poeta}, onde reconta o destino do poeta 
maior da língua portuguesa. 

Inédito em português, \textbf{Feitiço de amor e outros contos} reúne seis das 
narrativas macabras ou fantásticas de \textit{Phantasus} (1812--16), coletânea 
de "contos maravilhosos artísticos", composições literárias de características 
populares com algum estilo. Esse título, "Fântaso", faz referência ao deus grego 
dos sonhos com seres inanimados, pois é ele o guia que inicia o poeta pelas
manifestações assustadoras da natureza no universo: o medo, a tolice, o gracejo, 
o amor, todos temas presentes nos contos deste livo. 

No conto que dá nome à coletânea, "Feitiço de amor", veremos como Emil, rapaz
sensível e melancólico, herdeiro de uma fortuna, antagoniza com seu companheiro 
Roderich em relação às atitudes perante a vida. Completamente contrário ao jeito
sociável e festivo de Roderich, Emil vive enclausurado na busca de experiências 
sublimes, inclusive de amor. É assim que acaba se apaixonando por sua vizinha, 
uma bela jovem. Mas no momento em que a presencia praticando um ato de magia negra
junto a uma bruxa terrível, o horror toma conta de sua vida. 

Esperamos que as indicações propostas aqui sejam muito úteis no trabalho em
sala de aula! 





\end{abstract}

\tableofcontents

\section{Introdução}

Johann Ludwig Tieck (1773--1853) estudou história, filologia e literatura moderna.
Conheceu o filósofo Karl W. F.\,Schlegel (1772--1829) e se envolveu nas discussões do primeiro romantismo. 
Conviveu também com escritores como 
Novalis (1772--1881), 
os filósofos 
Johann Gottlieb Fichte (1762--1814) e 
Friedrich W. J.\,Schelling (1775--1854). 
Ludwig Tieck foi um dos pioneiros do romantismo alemão e uma das figuras mais conhecidas de seu tempo. 
Seus textos eram populares na Alemanha e tiveram forte impacto na Grã-Bretanha, na Espanha e nos Estados Unidos.

Depois de sua morte, caiu no esquecimento, mas teve sua obra revalorizada a partir de meados do século XX. 
Hoje está sendo recuperado como um dos autores da literatura romântica que associa elementos do maravilhoso a traços góticos.

Além de poeta e escritor, Tieck se destacou como editor e tradutor.
Traduziu o \textit{Dom Quixote}, de Miguel de Cervantes (1547--1616) e boa parte da obra de 
W.\,Shakespeare (1564--1616), dando continuidade às traduções iniciadas por Schlegel.
Consagrou a Luis de Camões (1524–1580) uma novela histórica sobre a morte 
do escritor português, na qual reconta o destino do poeta maior da língua portuguesa.

Estabelecido em Berlim, participou da sociedade literária de então.

A ideia de reunir em uma coletânea as macabras ou fantásticas de 
\textit{Phantasus}, concebida em três partes, cada uma com sete textos,
numa composição intermediada por diálogos, remete à série de contos italianos \textit{Decameron}, do escritor italiano Giovanni Boccaccio (1313--1375). 

A seleção dos contos desta publicação privilegia o horror nas narrativas de “phantasus”. 

As narrativas acolhem o fantástico de maneira tão natural que é possível 
se deparar com o maravilhoso como se fosse o mero cotidiano. 
Em narrativas como o conto “O cálice”, o macabro predomina e permeia toda a história
Assim como nos contos populares, a trama nos “contos artísticos” de Tieck nem sempre conduz a um final macabro. 

As traduções desta obra representam o primeiro esforço para apresentar um autor chave do romantismo alemão ao público brasileiro. Sobre Tieck, o crítico Otto Maria Carpeaux escreveu: “nos 22 volumes das obras completas de Tieck, muita coisa boa e bela está enterrada e esquecida. Durante trinta anos, Tieck rivalizou com Goethe. 
Muitos o consideravam o verdadeiro centro da literatura alemã.”

No contexto do movimento literário “tempestade e ímpeto” (\textit{Sturm und Drang}), que prenuncia o romantismo alemão, a pesquisa de Johann Gottfried Herder se baseava em canções, baladas e contos populares. Foi Herder quem, 
pela primeira vez, empregou o conceito de “canções populares”, que 
buscava reabilitar o espírito literário vital e vigoroso, mas também o “nacional”, 
o que gerou interpretações controversas. A polêmica em torno do conceito “contos populares” 
desencadeia uma cisão dentro do romantismo alemão. 

A pesquisa sobre as raízes e heranças dessa literatura é complexa e remete a canções de trovadores medievais e à literatura inglesa, que o escritor admirava. 
A influência do escrito escocês James Macpherson baseadas em baladas gálicas.
Percebemos semelhante fecundidade da fantasia e o mundo de heróis em infindáveis batalhas.
As pesquisas do folclore alemão, nortearam  por exemplo as pesquisas dos Irmãos Grimm, célebres pelos “contos maravilhosos para o lar e as crianças”, histórias extraídas da tradição oral. 
Outros românticos, entre eles Ludwig Tieck e Novalis, transformaram o conteúdo popular num significado individual e subjetivo. 
A polêmica em torno do conceito “contos populares” desencadeia uma cisão dentro do romantismo alemão. 
O escritor alemão, portanto, adaptou as trovas medievais, colocando-as nos contos de inspiração popular.

O conto é o gênero literário praticado por Ludwig Tieck e nele o autor valoriza a fantasia e a imaginação.
Com a narrativa breve que segue diretamente para o final, tieck introduz o “maravilhoso”, efeito que 
admirou e elogiou em peças como “A tempestade” e “Sonho de uma noite de verão”, de Shakespeare. 
Nos desfechos dos contos maravilhosos de Tieck, prevalece o tom sombrio.

A literatura dedicada a temas sinistros e sobrenaturais representou uma ruptura com o racionalismo clássico. 
Entre os vários gêneros literários, o romance gótico foi o mais popular e causou o maior impacto sobre a imaginação popular. Muitas características desse gênero estão presentes nas histórias macabras de autores como 
Wilkie Collins (1824--1889), H.\,P.\,Lovecraft (1890--1937) e Stephen King (1947--\,), 
assim como no gênero dos romances policiais. 


\section{Atividades 1}

%\BNCC{EM13LP26}

\subsection{Pré-leitura}

%\BNCC{EM13LGG302}
%\BNCC{EM13LGG704}
%\BNCC{EM13LP10}
%\BNCC{EM13LP19}

% \paragraph{Tema}
% \paragraph{Conteúdo}
% \paragraph{Objetivo}
% \paragraph{Justificativa}
% \paragraph{Metodologia}
% 	\begin{enumerate}
% 	\end{enumerate}
% \paragraph{Tempo estimado}

Na pré-leitura, promova uma roda de conversas para
compartilhar oralmente alguns contos maravilhosos europeus, sobretudo os
coletados pelos irmãos Grimm.

Mais familiares para muitos alunos que os contos de Tieck, essas
narrativas trazem elementos fantásticos e folclóricos que poderão
aproximar os estudantes da prosa artística de Tieck.

\subsection{Leitura}

%\BNCC{EM13LGG103}
%\BNCC{EM13LP02}
%\BNCC{EM13LP48}

% \paragraph{Tema}
% \paragraph{Conteúdo}
% \paragraph{Objetivo}
% \paragraph{Justificativa}
% \paragraph{Metodologia}
% 	\begin{enumerate}
% 	\end{enumerate}
% \paragraph{Tempo estimado}

Durante a leitura, analise as características românticas
presentes na obra e estabeleça comparações com a literatura medieval de
língua portuguesa, sobretudo as cantigas trovadorescas e as novelas de
cavalaria.
\Image{Retrato de Johann Ludwig Tieck, 1828. (Vogel, C. ; Domínio Público)}{PNLD0041-03.png}

O conto ``O Loiro Eckbert'', por exemplo, está entre os textos da
literatura alemã que mais vezes foram analisados e comentados, tendo
sido submetido às mais variadas correntes teóricas. Paradoxalmente, as
primeiras linhas da narrativa delineiam uma situação de banalidade,
introduzindo aos olhos do leitor o cotidiano da vida simples, ordeira e
reclusa de um casal de meia idade --- Eckbert e Bertha ---, que,
aparentemente, nada tem de complexo ou extraordinário. O caráter pacato
desse ponto de partida potencializa o contraste frente aos eventos
seguintes, que comprometem tanto Bertha quanto Eckbert em mentiras,
traições, atos de impiedade e violências, instituindo como fio condutor
do conto o tema da culpa.

Escrevendo ``o loiro Eckbert'' em 1796 (publicado no ano seguinte),
Tieck produziu um texto que, na Alemanha, constrói a passagem do gótico
do século XVIII ao fantástico do século XIX (na acepção de Todorov).

\SideImage{Casa Ludwig Tieck de 1819 a 1842, Saxônia, Alemanha (Lupus in Saxonia; CC-BY-SA 4.0)}{PNLD0041-05.png}

A intersecção se dá com contos de fadas dotados de aspectos terríveis,
que vão do canibalismo a assassinato e outras formas de violência. No
entanto, se nesses contos de fadas mencionados os heróis superam as
forças maléficas, Eckbert é engolido pelo turbilhão de revelações
escabrosas com que a velha o confronta. Em Tieck, a bruxa é a vencedora.
Mais do que isso: ela surge como uma instância ineludível que, qual uma
moira do destino, controla as etapas do fio da vida e, por fim, corta-o.

Em seguida, utilize exemplos da prosa romântica europeia, que retoma a
idade média no âmbito do projeto de evasão temporal.

Selecione, por exemplo, trechos de Walter Scott, em Ivanhoé, ou
Alexandre Herculano, em romances portugueses como ``O Bobo'' e ``Eurico,
O Presbítero''.

\subsection{Pós-leitura}

%\BNCC{EM13LGG102}
%\BNCC{EM13LGG303}
%\BNCC{EM13LGG402}
%\BNCC{EM13LGG703}
%\BNCC{EM13LP13}
%\BNCC{EM13LP14}
%\BNCC{EM13LP28}
%\BNCC{EM13LP29}
%\BNCC{EM13LP52}

% \paragraph{Tema}
% \paragraph{Conteúdo}
% \paragraph{Objetivo}
% \paragraph{Justificativa}
% \paragraph{Metodologia}
% 	\begin{enumerate}
% 	\end{enumerate}
% \paragraph{Tempo estimado}

Depois da leitura, promova uma oficina de escrita literária
em que os alunos possam compor contos com atmosfera gótica, aproveitando
elementos sombrios presentes nos contos de Tieck ou mesmo em contos,
filmes e séries atuais.

Cada aluno deverá desenvolver um conto, dando o devido destaque à
personagem criada ou selecionada de alguma das fontes mencionadas,
desenvolvendo-a. A partir disso. Por fim, com os textos escritos, é
possível a realização de uma leitura dinâmica diante da turma,
apresentando o produto final.

\Image{O gazebo, 1861 (MDZ München; Domínio Público)}{PNLD0041-06.png}

\section{Atividades 2}

%\BNCC{EM13CNT201}
%\BNCC{EM13CNT303}
%\BNCC{EM13CHS101}
%\BNCC{EM13CHS102}
%\BNCC{EM13CHS106}
%\BNCC{EM13CHS401}



A obra \emph{Feitiços de Amor e Outros Contos} possibilita trabalhos
interdisciplinares e integradores de diferentes campos do saber e áreas
de conhecimento. A seguir, propomos algumas atividades que podem ser
desenvolvidas conjuntamente com professores de outras áreas. Além das
habilidades de Linguagens e suas Tecnologias e de Língua Portuguesa,
indicadas nas etapas da seção anterior e válidas também para esta,
listamos a seguir as habilidades de outras áreas, presentes na abordagem
interdisciplinar:

\subsection{Pré-leitura}

% \paragraph{Tema}
% \paragraph{Conteúdo}
% \paragraph{Objetivo}
% \paragraph{Justificativa}
% \paragraph{Metodologia}
% 	\begin{enumerate}
% 	\end{enumerate}
% \paragraph{Tempo estimado}

Antes da leitura, é interessante se aprofundar no universo
no qual a história será construída. Para isso, em parceria com
professores de ciências humanas, proponha pesquisa e debate sobre a
presença da Idade Média na ficção. Sugira a reflexão sobre o porquê essa
época é tão presente no imaginário, seja por meio de filmes, séries,
livros, jogos, etc. E solicite aos alunos que busquem dados que
comprovem ou desmistifiquem elementos comuns do nosso imaginário quando
pensamos em Idade Média.

\subsection{Leitura}

% \paragraph{Tema}
% \paragraph{Conteúdo}
% \paragraph{Objetivo}
% \paragraph{Justificativa}
% \paragraph{Metodologia}
% 	\begin{enumerate}
% 	\end{enumerate}
% \paragraph{Tempo estimado}

Durante a leitura, os alunos podem pesquisar o universo
folclórico alemão, presente nas obras do autor, em diálogo com elementos
herdados da Inglaterra de Shakespeare ou da literatura medieval
europeia.

Nos contos de Tieck, o sobrenatural --- quer ele tenha mais carregadas
as cintilações do maravilhoso, quer esteja mais mergulhado nas penumbras
do gótico e horripilante --- traduz metaforicamente a experiência de
transpor um umbral e vislumbrar aquilo que se oculta por debaixo da
superfície. O encontro com o sobrenatural é produzido pelo olhar que
roça o desconhecido, misterioso e proibido, dando margem então ao efeito
de estranhamento. Os protagonistas de Tieck --- assim como depois os de
E. T. A. Hoffmann --- são indivíduos que se destacam por serem mais
sensíveis e instáveis (do que a maioria), e menos adaptados às
convenções de seu ambiente. Esses protagonistas transgridem as
fronteiras --- geográficas, assim como sociais --- em busca do que está
além de sua terra natal e que transcende o permitido.

\Image{Ilustração para a peça lírica ``Vida e Morte de Santa Genoveva'' (Gbi.bytos/Wikimedia Commons; CC-BY-SA 4.0)}{PNLD0041-07.png}

\subsection{Pós-leitura}

% \paragraph{Tema}
% \paragraph{Conteúdo}
% \paragraph{Objetivo}
% \paragraph{Justificativa}
% \paragraph{Metodologia}
% 	\begin{enumerate}
% 	\end{enumerate}
% \paragraph{Tempo estimado}

Depois da leitura, os alunos podem recuperar os contos
maravilhosos e fantásticos produzidos na atividade um e apresentar um
sarau, para criar o clima da apresentação, é possível organizar uma
playlist com músicas de compositores do tempo de ludwig Tieck, que
recuperem a atmosfera fantástica e misteriosa das narrativas.

Ludwig Tieck lançou mão de uma gama de recursos estéticos que
solidificam sua obra no cenário literário. Assim como Mary Shelley na
Inglaterra, Nathaniel Hawthorne nos Estados Unidos, Álvaro do Carvalhal,
em Portugal, Álvares de Azevedo, no Brasil, etc., Tieck utilizou o
gótico como instrumento poético inovador, capaz de fertilizar a
expressão de linguagem artística.

Sugerimos também alguns filmes e séries que dialogam com o universo das
narrativas góticas e populares, como ``O Labirinto Do Fauno'', de
Guillermo Del Toro, ou ``Grimm'', série policial que revive
criativamente o imaginário dos folcloristas alemães.

\Image{``Gato de Botas. Contos de fadas infantis em três atos'', Berlim, 1797, por Ludwig Tieck. (Klaus Günzel: Die deutschen Romantiker. Artemis, Zürich, 1995.; Domínio Público)}{PNLD0041-08.png}

\section{Aprofundamento}

Ao chegar ao Ensino Médio, é necessário que os estudantes se aprofundem
na compreensão das múltiplas linguagens e, sobretudo, da linguagem
literária. Em relação à literatura, a BNCC traz as seguintes
considerações:

\begin{quote}
{[}...{]} a leitura do texto literário, que ocupa o centro do trabalho
no Ensino Fundamental, deve permanecer nuclear também no Ensino Médio.
Por força de certa simplificação didática, as biografias de autores, as
características de épocas, os resumos e outros gêneros artísticos
substitutivos, como o cinema e as HQs, têm relegado o texto literário a
um plano secundário do ensino. Assim, é importante não só (re)colocá-lo
como ponto de partida para o trabalho com a literatura, como
intensificar seu convívio com os estudantes. Como linguagem
artisticamente organizada, a literatura enriquece nossa percepção e
nossa visão de mundo. Mediante arranjos especiais das palavras, ela cria
um universo que nos permite aumentar nossa capacidade de ver e sentir.
Nesse sentido, a literatura possibilita uma ampliação da nossa visão do
mundo, ajuda-nos não só a ver mais, mas a colocar em questão muito do
que estamos vendo/vivenciando. (Brasil, 2018, p. 491)
\end{quote}

Nesta seção, desenvolvemos um trabalho de aprofundamento que, em diálogo
com a formação continuada de professores, oferece subsídios para a
abordagem do texto literário.

\subsection{A Obra}

Feitiço de Amor e Outros Contos, coletânea de contos de Ludwig Tieck,
expõe as fantasmagorias do sujeito diante da estreiteza da realidade, da
incompletude e da sua incapacidade em abarcar a verdade. Bem ao estilo
romântico alemão e imbuído de entusiasmo pelo irracional, o autor
resgata elementos das canções medievais e dos contos populares, para
compor, com a literatura tradicional, o gênero fantástico. A riqueza e a
liberdade das imagens fantasiosas, também dotadas de terror e ironia,
impregnam de subjetividade a narrativa e dão voz ao inconsciente. O
universo onírico é de alguma forma inatingível e a sua conciliação com o
real não é impunemente alcançada. O vislumbre do sobrenatural faz o
homem a confrontar-se com o incognoscível em si e no mundo.

\subsection{Por que ler ``Feitiço de Amor e Outros Contos''}

Ludwig Tieck foi um dos pioneiros do Romantismo Alemão e dos escritores
mais lidos à época. Ficou muitos anos esquecido e foi redescoberto já no
século XX. Alguns dos contos desta coletânea foram publicados
originalmente em ''Phantasus'' entre 1812 e 1816. Dos 21 textos que
comporiam os três volumes inicialmente concebidos, apenas 13 foram
concluídos, e, destes, seis narrativas fantásticas são apresentadas
nessa obra.

A visitação do maravilhoso e do macabro, em esferas existenciais
aprofundadas, revelam o sujeito cindido entre o mundo real e projeções
de sua alma insondável, características que levam alguns especialistas a
considerar o autor o criador do herói romântico. Figuras e personagens
fabulosos personificam o desejo do autor pelo novo, pelo extraordinário,
pelo misticismo e, sobretudo, pela dúvida, uma vez que não é possível
conhecer completamente a verdade.

\Image{Ilustração para ``A vida do famoso imperador Abraham Tonelli'', 1920 (Rolf von Hoerschelmann; Domínio Público)}{PNLD0041-09.png}

\subsection{Literatura histórica e supranacional}

Ludwig Tieck foi admirador de Calderon de la Barca e de Shakespeare.
Traduziu ao alemão algumas obras do dramaturgo inglês e de Cervantes.
Influenciou-se pelas narrativas ossiânicas, a partir de poemas gálicos,
de James Macpherson. Absorveu da cultura popular os contos de fadas, as
lendas, as superstições. Resgatou os poemas trovadorescos medievais
alemães no conteúdo e forma, escrevendo, ele próprio, versos
heptassílabos em redondilha maior. E ainda assimilou as correntes
góticas e de horror que pairavam no cenário literário desde meados do
século XVIII.

\subsection{O Movimento Romântico}

O movimento romântico nasce nas últimas décadas do século XVIII, tendo
durado boa parte do século XIX. É o momento em que a Europa assiste à
Revolução Francesa e seus desdobramentos. Nesse contexto, ocorre a
ruptura com o sistema estamental da sociedade, levando à igualdade
jurídica e oferecendo ao indivíduo uma unicidade específica de seu
próprio ser. Ainda que libertadoras, essas mudanças gerarão conflitos
internos e reflexões, com percussão na arte. Assim, o Romantismo
interessa-se pelo individual, ou ao menos, a um ente individualizado,
rompendo com traços universalistas do Iluminismo. Também, focado nas
forças do âmago, opõe-se ao Racionalismo. É um movimento que oferta
substrato, ao mesmo tempo que se alimenta dos acontecimentos políticos,
sendo totalmente imbricada sua relação com o nacionalismo. Isso, somado
ao destaque dado a utopias, passados idealizados ou porvires
fantásticos, e, de certo modo, também com o escapismo que caracterizou o
estilo, geraram grandes influências nos eventos históricos vividos na
Europa do período.

\subsection{Contos de Fadas e de Bruxas}

A valorização da literatura popular e medieval, ocorrida durante o
romantismo alemão, teve grandes adeptos, incluindo os irmãos Grimm e
Ludwig Tieck. Tais autores recriaram o gênero conto a partir dos contos
maravilhosos.

Os pontos em comum desta obra com os contos de fadas são inúmeros,
principalmente em relação: aos cenários, como bosques, aldeias, pontes,
cabanas; às personagens como crianças, jovens, casais apaixonados,
bruxas, velhos sábios, animais encantados; e às circunstâncias como
conquista do ser amado, evasão da terra natal, fuga de maus tratos,
descoberta de um mundo mágico. Entretanto, nesses, os heróis,
geralmente, confrontam-se com o mal e o vencem.

Já nos contos fantásticos em questão, os protagonistas traem os ideais
românticos, sofrem metamorfoses, são desviados do bom caminho e, muitas
vezes, levados à perdição sem recompensas. As bruxas e os velhos acabam
por vencer ao final, decidir os destinos e executar as sentenças de
morte. Alguns enfoques, dentro destas temáticas, foram muito caros a
Tieck, como o pecado e expulsão do paraíso, a nostalgia da infância, a
solidão, e o determinismo do passado.

\subsection{Racional e Sobrenatural}

O empirismo pós iluminismo refutou os mistérios, que, alienados,
migraram para a dimensão artística, em especial para a literatura. Os
padrões estéticos do equilíbrio e da razão apontam para a objetividade e
a verossimilhança, e entram em conflito com as necessidades do herói
romântico, que enxerga a realidade transfigurada em projeções de sua
subjetividade. Assim, a racionalidade da época não foi capaz de
erradicar sua contraparte irracional, que intuitivamente demonstrava as
lacunas imponderáveis pela ciência. As personagens conseguem então
vislumbrar o que está oculto por meio do contato com o sobrenatural. No
entanto, as narrativas ganharam um toque realista ao abrigar locais
reais, personagens de famílias existentes, e descrições detalhadas,
quase ``científicas'', dos fatos ficcionais em prol da respeitabilidade,
com fundamentos alquímicos e astrológicos.

\subsection{O maravilhoso e o sinistro}

O misterioso e o desconhecido habitam cenários transcendentais
longínquos ou paralelos, que remontam a infância ou o sonho. Nesse
sentido, paisagens lúgubres ou solitárias podem encerrar universos
encantadores. Tais locais de harmonia, plenitude ou felicidade
desencadeiam sentimentos nostálgicos, pois o herói não pode regressar
aos mesmos após sua partida e tampouco trazê-los à sua realidade. São
seres algo inocentes e puros que contracenam com criaturas fabulosas ou
funestas, de sorte que acabam participando de crimes ou feitiçarias por
almejarem desvendar a existência. Quando seus pecados são revelados, o
protagonista entrecruza os limites entre o mundo fantástico e o real,
com a constatação inequívoca da existência desses universos colaterais e
incompatíveis. A aproximação a um horizonte de eventos pode levá-lo à
morte. Estamos diante de um homem vulnerável que não consegue conhecer a
si mesmo.

\Image{Ilustração d'une Berceuse de Tieck (canção de ninar) por Ludwig Richter (Gbi.bytos/Wikimedia Commons; CC-BY-SA 4.0)}{PNLD0041-10.png}

\subsection{Utopia ou simbologia}

No romantismo, a natureza harmônica almejada não é compatível com o
mundo objetivo, pois a realidade está sempre a desconstruí-la. Tieck
aponta que a pretensão iluminista de felicidade é utópica. Tal utopia
deve permanecer no campo imaginário artístico, como bem demonstrado por
suas personagens, cuja rebelião a esta máxima custou-lhes, muitas vezes,
a vida.

Segundo Tieck, ``tudo o que nos envolve só é verdade até certo ponto''.
O homem é imprevisível e mutável, fazendo com que tenha leituras
conflitantes acerca do cotidiano. No campo simbólico, o sinistro e o
maravilhoso, presentes nos elementos narrativos dos contos fantásticos,
podem expressar o interior cindido da personagem. O homem, angustiado em
busca de sua própria identidade, encerra esses duplos e, ao
aprofundar-se, reconhece a sordidez e a magia entranhadas na existência
humana, sob o manto de normalidade.

\section{Sugestões de atividades complementares: relações dialógicas e
intertextuais}

%\BNCC{EM13LP03}
%\BNCC{EM13LP04}
%\BNCC{EM13LP49}
%\BNCC{EM13LP51}

No Ensino Médio, da mesma forma que no Ensino Fundamental, a \textsc{bncc}
organiza o trabalho com as práticas de linguagem em cinco \textbf{campos
de atuação social}. São eles: campo da vida pessoal, campo da vida
pública, campo jornalístico"-midiático, campo artístico"-literário e campo
das práticas de estudo e pesquisa.

De acordo com essa divisão, propomos na sequência um trabalho
interdiscursivo e intertextual com a obra \emph{Feitiço de Amor e Outros
Contos.}

\subsection{Campo da vida pessoal}

\begin{quote}
O campo da vida pessoal pretende funcionar como espaço de articulações
e sínteses das aprendizagens de outros campos postas a serviço dos
projetos de vida dos estudantes. As práticas de linguagem privilegiadas
nesse campo relacionam"-se com a ampliação do saber sobre si, tendo em
vista as condições que cercam a vida contemporânea e as condições
juvenis no Brasil e no mundo.

Está em questão também possibilitar vivências significativas de práticas
colaborativas em situações de interação presenciais ou em ambientes
digitais e aprender, na articulação com outras áreas, campos e com os
projetos e escolhas pessoais dos jovens, procedimentos de levantamento,
tratamento e divulgação de dados e informações e o uso desses dados em
produções diversas e na proposição de ações e projetos de natureza
variada, para fomentar o protagonismo juvenil de forma
contextualizada. (\textsc{bncc}, p. 494)
\end{quote}

No "fantástico", a imaginação sem entraves é capaz de expressar, no
objeto, o ser humano misterioso, para além do pensamento lógico, e,
portanto, possibilita a exposição de suas divergências. Considerando o
exposto, sugerimos a pesquisa de obras fantásticas nas artes plásticas
e na literatura e a produção de uma criação artística individual
baseada no tema, em que o aluno esteja representado.


\subsection{Campo de atuação na vida pública}

\begin{quote}
No cerne do campo de atuação na vida pública estão a ampliação da
participação em diferentes instâncias da vida pública, a defesa dos
direitos, o domínio básico de textos legais e a discussão e o debate de
ideias, propostas e projetos. {[}\ldots{}{]}

Ainda no domínio das ênfases, indica"-se um conjunto de habilidades que
se relacionam com a análise, discussão, elaboração e desenvolvimento de
propostas de ação e de projetos culturais e de intervenção social.
(\textsc{bncc}, p. 494)
\end{quote}

A importância de reconhecimento do contraditório no homem e na vida,
com o auxílio da arte fantástica, possibilita acomodar as
divergências, e, assim, contribuir para o crescimento pessoal e
comunitário e a realizar leituras mais aprofundada dos desfechos da
vida. Propor aos alunos que pesquisem, em grupos, linhas de
investigações científicas sobre a COVID-19, como dados sobre a doença
e o vírus, as vacinas, os tratamentos e a profilaxia. Além disso,
pesquisar no imaginário popular o significado simbólico e emocional da
pandemia. Construir uma narrativa em grupos que abarque os elementos
encontrados.

\subsection{Campo jornalístico"-midiático}

\begin{quote}
Em relação ao campo jornalístico"-midiático, espera"-se que os jovens
que chegam ao Ensino Médio sejam capazes de: compreender os fatos e
circunstâncias principais relatados; perceber a impossibilidade de
neutralidade absoluta no relato de fatos; adotar procedimentos básicos
de checagem de veracidade de informação; identificar diferentes pontos
de vista diante de questões polêmicas de relevância social; avaliar
argumentos utilizados e posicionar"-se em relação a eles de forma ética;
identificar e denunciar discursos de ódio e que envolvam desrespeito aos
Direitos Humanos; e produzir textos jornalísticos variados, tendo em
vista seus contextos de produção e características dos gêneros. Eles
também devem ter condições de analisar estratégias
linguístico"-discursivas utilizadas pelos textos publicitários e de
refletir sobre necessidades e condições de consumo.

No Ensino Médio, os jovens precisam aprofundar a análise dos interesses
que movem o campo jornalístico midiático, da relação entre informação e
opinião, com destaque para o fenômeno da pós"-verdade, consolidar o
desenvolvimento de habilidades, apropriar"-se de mais procedimentos
envolvidos na curadoria de informações, ampliar o contato com projetos
editoriais independentes e tomar consciência de que uma mídia
independente e plural é condição indispensável para a democracia.

Como já destacado, as práticas que têm lugar nas redes sociais têm
tratamento ampliado. (\textsc{bncc}, p. 494-495)
\end{quote}

Aqui se segure observar e reconhecer discursos cativantes e
moralizantes muito comuns na atualidade, nas redes sociais e mídias,
porém distanciados, na prática, de fatos comprobatórios de seu caráter
benigno e de atitudes éticas dos próprios grupos que os defendem.

\subsection{Campo artístico"-literário}

\begin{quote}
No campo artístico"-literário busca"-se a ampliação do contato e a
análise mais fundamentada de manifestações culturais e artísticas em
geral. Está em jogo a continuidade da formação do leitor literário e do
desenvolvimento da fruição. A análise contextualizada de produções
artísticas e dos textos literários, com destaque para os clássicos,
intensifica"-se no Ensino Médio. Gêneros e formas diversas de produções
vinculadas à apreciação de obras artísticas e produções culturais
(resenhas, vlogs e podcasts literários, culturais etc.) ou a formas de
apropriação do texto literário, de produções cinematográficas e teatrais
e de outras manifestações artísticas (remidiações, paródias,
estilizações, videominutos, fanfics etc.) continuam a ser considerados
associados a habilidades técnicas e estéticas mais refinadas.

A escrita literária, por sua vez, ainda que não seja o foco central do
componente de Língua Portuguesa, também se mostra rica em possibilidades
expressivas. (\textsc{bncc}, p. 495-496).
\end{quote}

Dito isso, sugerimos a leitura dos contos maravilhosos populares, com
destaque aos dos Irmãos Grimm, e posterior discussão sobre os traços
românticos e simbólicos presentes nessas obras.

\Image{Wilhelm Grimm e Jacob Grimm em 1847, Daguerreótipo. (Hermann Blow; Domínio Público)}{PNLD0041-04.png}

\subsection{Campo das práticas de estudo e pesquisa}

\begin{quote}
O campo das práticas de estudo e pesquisa mantém destaque para os
gêneros e habilidades envolvidos na leitura/escuta e produção de textos
de diferentes áreas do conhecimento e para as habilidades e
procedimentos envolvidos no estudo. Ganham realce também as habilidades
relacionadas à análise, síntese, reflexão, problematização e pesquisa:
estabelecimento de recorte da questão ou problema; seleção de
informações; estabelecimento das condições de coleta de dados para a
realização de levantamentos; realização de pesquisas de diferentes
tipos; tratamento dos dados e informações; e formas de uso e
socialização dos resultados e análises.

Além de fazer uso competente da língua e das outras semioses, os
estudantes devem ter uma atitude investigativa e criativa em relação a
elas e compreender princípios e procedimentos metodológicos que orientam
a produção do conhecimento sobre a língua e as linguagens e a formulação
de regras. (\textsc{bncc}, p. 495-496)
\end{quote}

As correntes racionalistas, como o positivismo e o historicismo,
destituídos da metafísica e das discussões éticas, correm o risco de
se tornarem neutras em relação aos valores morais e esvaziarem de
sentido a alma humana. Os regimes totalitários, na figura de um líder
messiânico e acolhedor, podem preencher este vácuo, e angariar
simpatizantes por meio de um discurso carismático de igualdade, amor
patriótico e progresso, entre outros. Diante do ganho social
grandioso, os meios para alcançar a sociedade perfeita se torna algo
secundário. Considerando o exposto, sugerimos a discussão sobre os
insights que o gênero fantástico pode fornecer para a compreensão da
mentalidade do homem à época, e a reflexão sobre sua contribuição
positiva e negativa para tais desdobramentos histórico-políticos.

\section{Referências complementares}

\begin{itemize}
\item\textsc{andrade}, Carlos Drummond de. \textit{Amar se aprende amando: Poesia de
convívio e de humor}. São Paulo: Companhia das Letras, 2018.

O que encontramos nos 68 poemas que compõem este volume é o fruto do
esforço do poeta em conciliar sentimento e experiência, em um percurso
pelas diferentes formas que o sentimento amoroso assume em sua poética.

\item\textsc{carpeaux}, Otto Maria. \textit{A história Concisa da Literatura Alemã}.
Alphaville: Faro Editorial, 2013.

O autor faz uma síntese dos grandes momentos, livros e autores da
literatura alemã, e conta com uma avaliação crítica de sua importância
para a cultura e o desenvolvimento do país e sua influência nos
principais movimentos culturais do mundo contemporâneo.

\item\textsc{carpeaux}, Otto Maria. \textit{A história Literatura Ocidental}. São
Paulo: Leya, 2019.

Da literatura grega à contemporânea, Carpeaux analisa e critica as obras
com seu arcabouço teórico. Seu conhecimento nos leva ao encontro dos
mais importantes autores da literatura ocidental.
\end{itemize}

\section{Bibliografia comentada}

\begin{itemize}
\item\textsc{bauman,} \textit{Zygmunt. Amor líquido: Sobre a fragilidade dos
laços humanos}. Rio de Janeiro: Zahar, 2004.

A modernidade líquida, "um mundo repleto de sinais confusos, propenso a
mudar com rapidez e de forma imprevisível" em que vivemos, traz consigo
uma misteriosa fragilidade dos laços humanos, um amor líquido.

\item\textsc{campbell}, Joseph. \textit{O Poder do Mito}. São Paulo: Palas Athena,
2014.

O livro é o fruto de uma série de conversas mantidas entre Joseph
Campbell e o jornalista Bill Moyers, numa combinação de sabedoria e
humor, sobre o casamento, os nascimentos virginais, a trajetória do
herói e o sacrifício ritual.

\item\textsc{paz}, Octavio. \textit{A Dupla Chama}. São Paulo: Mandarim Editora, 1999.

O livro fornece uma história social e literária do amor e do erotismo,
comparando as manifestações modernas com as épocas anteriores, embora
observando a relação especial entre erotismo e poesia.

\item\textsc{rougemont}, Denis De. \textit{O Amor e o Ocidente}. Rio de Janeiro:
Guanabara, 1988.

Denis de Rougemont acreditava que o casamento vivia uma crise no início
do século XX e em seu livro pretende apontar os culpados por ela. Sua
tese é sobre a antítese amor e paixão.

\item\textsc{volobuef}, Karin. Ludwig Tieck: meandros góticos. In \emph{Ilha
do Desterro}. Florianópolis, n. 62, jan/jun 2012, p. 153-172.
(Disponível em:
\url{https://repositorio.unesp.br/bitstream/handle/11449/73913/2-s2.0-84872003268.pdf?sequence=1\&isAllowed=y}.
Acesso em 18 de fevereiro de 2021.)

A autora analisa as ressonâncias da prosa gótica e do conto maravilhoso
na produção romântica de Ludwig Tieck.
\end{itemize}


\end{document}



