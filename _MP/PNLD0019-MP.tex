\documentclass[12pt]{extarticle}
\usepackage{manualdoprofessor}
\usepackage{fichatecnica}
\usepackage{lipsum,media9,graficos}
\usepackage[justification=raggedright]{caption}
\usepackage{bncc}
\usepackage[indico]{../edlab}
 

\begin{document}


\newcommand{\AutorLivro}{Patativa do Assaré}
\newcommand{\TituloLivro}{Uma voz do Nordeste}
\newcommand{\Tema}{Ficção, mistério e fantasia}
\newcommand{\Genero}{Poema}
% \newcommand{\imagemCapa}{PNLD0019-01.png}
\newcommand{\issnppub}{---}
\newcommand{\issnepub}{---}
% \newcommand{\fichacatalografica}{PNLD0019-00.png}
\newcommand{\colaborador}{\textbf{Mariana Barrile, Bruno Gradella e Vicente Castro} é uma pessoa incrível e vai fazer um bom serviço.}


\title{\TituloLivro}
\author{\AutorLivro}
\def\authornotes{\colaborador}

\date{}
\maketitle
\tableofcontents

\pagebreak

\section{Carta aos professores}

Caro educador / Cara educadora,\\\bigskip

Este Manual tem como objetivo fornecer subsídios para o trabalho com a
obra literária \emph{Uma voz do Nordeste}, de autor homônimo.

Neste material, são propostas atividades de leitura do texto literário,
em perspectiva interdisciplinar, a partir de propostas de abordagem
envolvendo a integração de diferentes áreas do conhecimento. Todas as
etapas de pré"-leitura, leitura e pós"-leitura atribuem aos professores o
papel de mediadores entre os estudantes e a obra literária. Aos alunos,
por sua vez, é conferido um lugar de protagonismo e autonomia na
construção do conhecimento, a partir do emprego de metodologias ativas e
estratégias de aprendizagem criativa.

Seguindo as competências e habilidades indicadas na nova Base Nacional
Comum Curricular (\textsc{bncc}), o trabalho com o texto literário é desenvolvido
no âmbito dos diferentes campos de atuação social. Para isso, levam"-se
em consideração os campos da vida pessoal, de atuação na vida pública,
das práticas de estudo e pesquisa, jornalístico"-midiático e, sobretudo,
artístico"-literário.

Cada uma das seções do Manual sugere atividades e apresenta informações
complementares para enriquecer a experiência de leitura do texto
literário. A partir de uma proposta dialógica de ensino de literatura,
procura desenvolver habilidades de leitura e produção de textos, visando
à formação do sujeito leitor"-autor competente, capaz de interagir com o
mundo e de atribuir sentido às próprias vivências.

Reforçando o caráter formativo e informativo da literatura, o material
procura articular a formação de leitores à elaboração de projetos de
vida, a partir da ampliação do repertório artístico"-cultural dos
estudantes. A leitura crítica da obra literária é concebida em sentido
amplo e envolve o estabelecimento de relações entre textos pertencentes
a esferas variadas de comunicação e a gêneros discursivos diversos.

Ao mesmo tempo, a releitura de clássicos literários traz, para a
contemporaneidade, novas possibilidades de significação para obras que
permanecem atuais. Para contribuir com essa aproximação entre jovens
leitores e obras consagradas, são propostas atividades que empregam as
novas tecnologias de informação e comunicação, fundamentais para a
formação de leitores inseridos na cultura digital. Valorizam"-se,
portanto, estratégias de leitura e produção textuais no âmbito da
hipertextualidade e do multiletramento.

Para a formação continuada de professores, apresentamos sequências que
estimulam a criatividade e a inovação, com possibilidades de adaptação
às diferentes realidades de ensino. Orientações de caminhos possíveis
são apresentadas como sugestões para as atividades de leitura em sala de
aula, sempre permeáveis aos diferentes perfis de grupos e às
especificidades de gostos e repertórios culturais.

Ao trabalhar com Literatura de Cordel é indispensável que tenhamos
consciência da importância desse gênero literário na história do nosso
país. O Cordel é considerado Patrimônio Cultural Brasileiro, visto que
ele é um meio de disseminar a história, o linguajar e os costumes do
povo nordestino. Por muito tempo, inclusive, foi utilizado como meio de
disseminação de notícias; os cordelistas eram responsáveis por espalhar
as notícias da cidade através do repente, fazendo com que as informações
chegassem à todas as classes, em um linguajar simples e popular.

Trabalhar com Literatura de Cordel em sala de aula é trabalhar com
estratégias de aprendizagem criativa que envolvem as mais diversas áreas
de ensino e de comunicação. É uma grande oportunidade de incentivar os
alunos a desenvolverem seu lado artístico, além de coloca"-los em contato
com um rico material cultural. A leitura de uma literatura tão rica
historicamente traz possibilidades de ressignificação das obras,
aproximando os jovens de realidades que eles, muitas vezes, nunca
imaginaram conhecer.

Conforme a \textsc{bncc} assinala, ``no Ensino Médio, os jovens intensificam o
conhecimento sobre seus sentimentos, interesses, capacidades
intelectuais e expressivas; ampliam e aprofundam vínculos sociais e
afetivos; e refletem sobre a vida e o trabalho que gostariam de ter.
Encontram"-se diante de questionamentos sobre si próprios e seus projetos
de vida, vivendo juventudes marcadas por contextos socioculturais
diversos'' (Brasil, 2018, p.481).

Nada mais adequado, portanto, que oferecer textos literários capazes de
estimular reflexões sobre a vida pessoal e rotas possíveis para os
projetos de vida. Para você, professor(a), abrem"-se novas trilhas para o
contato sempre renovado com obras que são, a um só tempo, atuais e
atemporais.

Boa jornada!

\section{Atividades 1}

\BNCC{EM13LP26}

\subsection{Pré"-leitura}

\BNCC{EM13LGG302}
\BNCC{EM13LGG704}
\BNCC{EM13LP10}
\BNCC{EM13LP19}

Antes de se iniciar a leitura, é interessante que os alunos
entendam o universo para o qual estão enveredando. A literatura de
cordel é um patrimônio da cultura brasileira. Fabricada em folhetos,
geralmente trazem para o universo escrito casos e histórias já comuns na
oralidade popular. Também é comum que seu texto seja construído por meio
de rimas. Além disso, uma das características típicas do cordel são as
gravuras, que possuem um tracejado muito próprio. Não obstante, por
serem uma expressão muito típica da cultura de uma região do país, os
diversos títulos produzidos em cordel trazem elementos básicos que são
características intrínsecas muito próprias. Essa atividade sugere que os
alunos façam uma pesquisa com vários títulos de cordel. Procurem olhar a
sinopse da obra, folheiem seu conteúdo e busquem temas e personagens que
frequentemente aparecem nessas obras. Isso posto, é interessante que
cada aluno escreva um relatório com seus pareceres iniciais, colocando
suas suposições do porquê da repetição desses eixos temáticos e
personagens. Posteriormente, o professor pode coletar essas informações
junto aos alunos e abordar as questões durante as aulas sobre o livro.

\subsection{Leitura}

\BNCC{EM13LGG103}
\BNCC{EM13LP02}
\BNCC{EM13LP48}

Já mais familiarizados com as características do cordel, os
alunos devem passar então à produção de um texto nesse formato. É
possível trabalhar com a redação de uma autobiografia do aluno, contando
sua história nos moldes de um cordel. Entretanto, é possível que alguns
alunos não se sintam confortáveis em relatar experiências próprias.
Diante disso é possível propor que seja uma biografia com elementos
romanceados, elementos do fantástico, ou mesmo a biografia de uma
terceira pessoa, pela qual o aluno nutre admiração. Em certo sentido, os
cordéis promovem isso, personagens admiráveis, de modo que escrever
sobre um ídolo ainda é algo muito produtivo nesta atividade. Além do
mais, só a produção no formato já ajudará os alunos a terem maior
familiaridade com o gênero. As rimas devem ser estimuladas, mas não são
uma obrigatoriedade, posto que podem gerar uma trava produtiva a alguns
estudantes.

\subsection{Pós"-leitura}

\BNCC{EM13LGG102}
\BNCC{EM13LGG303}
\BNCC{EM13LGG402}
\BNCC{EM13LGG703}
\BNCC{EM13LP13}
\BNCC{EM13LP14}
\BNCC{EM13LP28}
\BNCC{EM13LP29}
\BNCC{EM13LP52}

Com o material produzido na atividade anterior, é possível a
realização de uma grande feira de cordéis na escola, junto ao professor
de artes, sugere"-se que os alunos preparem uma decoração adequada,
valorizando as cores e os traços presentes nesse tipo de material tão
distinto. Além da exposição dos cordéis produzidos, sugere"-se também a
promoção de encenações de trechos de obras lidas, ou produzidas pelos
alunos, a declamação de poemas, competição de repentes. Também se
estimula a aproximação deste gênero com outras produções artísticas
mundiais. Nada impede que, em uma dessas encenações ou leituras
dramáticas, os alunos podem buscar trechos de peças, filmes ou obras da
literatura mundial e encená"-los, valendo"-se da estética e do vocabulário
típico aos cordéis.

\section{Atividades 2}

\BNCC{EM13CNT201}
\BNCC{EM13CNT303}
\BNCC{EM13CHS101}
\BNCC{EM13CHS102}
\BNCC{EM13CHS106}
\BNCC{EM13CHS401}

A obra \emph{Uma voz do Nordeste} possibilita trabalhos
interdisciplinares e integradores de diferentes campos do saber e áreas
de conhecimento. A seguir, propomos algumas atividades que podem ser
desenvolvidas conjuntamente com professores de outras áreas. Além das
habilidades de Linguagens e suas Tecnologias e de Língua Portuguesa,
indicadas nas etapas da seção anterior e válidas também para esta,
listamos a seguir as habilidades de outras áreas, presentes na abordagem
interdisciplinar:

\subsection{Pré"-leitura}

Antes da leitura, é possível discutir a situação da seca no
Nordeste. Professores de ciências humanas podem discutir a história
desse fenômeno e suas implicações sociais, bem como as relações que o
homem travou com essa situação ao longo de toda a ocupação da região.
Por outro lado, professores de ciências da natureza podem auxiliar os
alunos na compreensão dos fenômenos geológicos e meteorológicos que
contribuem para a perpetuação da aridez local, bem como explicar o bioma
da caatinga e sua importância para o equilíbrio ecológico. Munidos dessa
informação os alunos podem produzir um jornal, tanto nos moldes
escritos, como em vídeo, construindo uma reportagem sobre todas as
questões aventadas. O material produzido pode ser divulgado no site da
escola. Como sugestão, indica"-se utilizar aplicativos gratuitos de
gravação e edição de vídeos, disponíveis em dispositivos digitais

\subsection{Leitura}

Entre os poemas do livro, está o intitulado \emph{História
de Aladim e a Lâmpada Maravilhosa}, baseada na famosa história que
compõe as \emph{Mil e Uma Noites}, já tendo recebido várias adaptações
para filmes e desenhos. O que poucas pessoas sabem, é que essa história
passou a integrar o cômputo das \emph{Mil e Uma Noites}, apenas no
século \textsc{xviii}, após ter aparecido em uma tradução francesa. O tradutor,
Antoine Galland, afirmava que ele havia ouvido a narrativa de um
contador de histórias Sírio. Em geral, estudiosos entendem que a
história de Aladim é de fato uma história árabe, produzida entre os
séculos \textsc{vii} e \textsc{xiii}, tendo permanecido a maior parte de sua existência na
oralidade. Entretanto, há alguns que entendem que se trata de uma
fabricação europeia, imbuída do que o intelectual Edward Said definiu
como Orientalismo. Com auxílio do professor de ciências humanas, peça
para que os alunos levantem informações e debatam acerca desse conceito
e significado.

É importante que alguns pontos sejam levantados, como por exemplo, o que
Said define como caráter generalizante das observações ocidentais em
relação ao Oriente, que colocam culturas distintas como se fossem
homogêneas apenas por uma localização geográfica. Entre esses elementos,
Said indica que os ocidentais vislumbram o Oriente como o lugar dos
déspotas, do esplendor, das especiarias, das sabedorias misteriosa, da
sensualidade, de modo que, muitos, ao viajarem para o Oriente, se
frustram, porque a realidade é totalmente distinta da expectativa. Aqui
se frisa que, apesar de alguns pontos dessa visão serem positivos, ela
ainda é estereotipada.

Com isso em mãos, acrescido da pesquisa dos alunos, promova uma reflexão
e debate acerca da questão, transpondo"-a para a realidade nacional.
Indague aos alunos se no próprio Brasil existe, ou não, uma visão
estereotipada de uma região frente a outra.

\subsection{Pós"-leitura}

Na obra lida, é feita a oposição entre campo e cidade, e as
características das pessoas que habitam esses dois ambientes. Nesta
temática, do homem rural em oposição ao homem urbano, proponha aos
alunos, contando com o auxílio do professor de ciências humanas, a
criação de um mapa. Nele, inicialmente divida as regiões em
majoritariamente urbanas e rurais. Feito isso, aconselha"-se buscar
notícias e estudos sobre as grandes desigualdades e os fatores do
deslocamento populacional para as grandes cidades. Com essa informação,
é possível indicar de maneira gráfica, valendo"-se de setas por exemplos,
caso seja um painel físico, ou por recursos animados, caso o mapa esteja
sendo produzido virtualmente, os fluxos migratórios estudados. Como
fonte, também, sugere"-se orientar a pesquisa por meio da obra de
Patativa do Assaré e de outros autores nordestinos.

Para complementar a obra, sugere"-se criar xilogravuras a partir da
confecção de carimbos artesanais.

\section{Aprofundamento}

Ao chegar ao Ensino Médio, é necessário que os estudantes se aprofundem
na compreensão das múltiplas linguagens e, sobretudo, da linguagem
literária. Em relação à literatura, a \textsc{bncc} traz as seguintes
considerações:

\begin{quote}
{[}\ldots{}{]} a leitura do texto literário, que ocupa o centro do trabalho
no Ensino Fundamental, deve permanecer nuclear também no Ensino Médio.
Por força de certa simplificação didática, as biografias de autores, as
características de épocas, os resumos e outros gêneros artísticos
substitutivos, como o cinema e as \textsc{hq}s, têm relegado o texto literário a
um plano secundário do ensino. Assim, é importante não só (re)colocá"-lo
como ponto de partida para o trabalho com a literatura, como
intensificar seu convívio com os estudantes. Como linguagem
artisticamente organizada, a literatura enriquece nossa percepção e
nossa visão de mundo. Mediante arranjos especiais das palavras, ela cria
um universo que nos permite aumentar nossa capacidade de ver e sentir.
Nesse sentido, a literatura possibilita uma ampliação da nossa visão do
mundo, ajuda"-nos não só a ver mais, mas a colocar em questão muito do
que estamos vendo/vivenciando. (Brasil, 2018, p. 491)
\end{quote}

Nesta seção, desenvolvemos um trabalho de aprofundamento que, em diálogo
com a formação continuada de professores, oferece subsídios para a
abordagem do texto literário. A leitura em sala de aula de
\emph{Uma voz do Nordeste} pode ser enriquecida pelo aprofundamento no
universo literário.

\subsection{O romance da maturidade}

O texto literário, presente no currículo escolar desde o Ensino
Fundamental, deve também permanecer como base de formação no Ensino
Médio, para que os alunos possam se aprofundar na compreensão não só da
linguagem literária, mas também das múltiplas linguagens, estimulando a
criatividade e imaginação. Segundo Paulo Freire (1989),\footnote{\textsc{freire}, Paulo. \textit{A importância do ato de ler: em três artigos que se completam}, 23 ed, São Paulo: Autores Associados: Cortez, 1989.} o ato de
ler implica na percepção crítica, interpretação, reescrita e
reelaboração do que lemos, agindo assim, diretamente, com diferentes
esferas de aprendizagem.

Este manual oferece subsídios para a abordagem do texto literário em
diálogo com a formação continuada de professores, apresentando
propostas, com diferentes abordagens, para o trabalho de \emph{Vocação
de Cantador, de Oliveira de Panelas,} dentro e fora sala, incentivando
os alunos a adquirirem o hábito de leitura e pensamento crítico.

\subsection{Importância da Literatura de Cordel em sala}

Com uma produção simples e de grande abrangência, a Literatura de Cordel
ganhou espaço e prestígio na cultura nordestina brasileira, tornando"-se,
em 2018, patrimônio cultural imaterial do nosso país -- reconhecido pelo
Conselho Consultivo do Instituto do Patrimônio Histórico e Artístico
Nacional. Assim, o Cordel como gênero do discurso contribui na formação
do aluno possibilitando o domínio de outros conteúdos, além da
descentralização do ensino. O estudo da Literatura de Cordel como forma
de expressão da cultura popular contribui também no aprimoramento das
habilidades de oralidade, escrita, leitura, interpretação, linguagens
artísticas e até na dramatização de peças, auxiliando na
interdisciplinaridade dos temas.

A Educação Literária aparece entre as competências gerais da \textsc{bncc} (Base
Nacional Comum Curricular), enfatizando a importância vivência do aluno
no aprendizado da literatura e demais manifestações artísticas --
``\emph{Valorizar e fruir as diversas manifestações artísticas e
culturais, das locais às mundiais, e também participar de práticas
diversificadas da produção artístico"-cultural\footnote{\textsc{brasil}. Ministério da Educação. Base Nacional Comum Curricular.
Brasília, 2018.}'' --},
logo, o Cordel ganha espaço nesse cenário, pois, a partir de seu estudo,
é possível despertar nos alunos o interesse por diversos campos
artísticos.

Entretanto, o uso do Cordel em sala de aula, bem como nos livros
didáticos, ainda é muito restrito, por não ser tão prestigiado quanto os
demais gêneros literários. Diante disso, cabe a nós resgatarmos essa
parte da nossa identidade nacional, contextualizando o aluno no meio
social e cultural de seu país.

\subsection{Variação linguística e oralidade}

O Cordel nasceu da oralidade e da linguagem popular; uma leitura
silenciosa limita seu poder de comunicação, impedindo que seu potencial
seja trabalhado como um todo. O gênero, se bem explorado, pode auxiliar
no aprendizado e desenvoltura dos alunos nessa modalidade de expressão,
devido ao seu ritmo cadenciado e seu linguajar comum, próximo ao
cotidiano do aluno. Para isso, o professor deve promover atividades que
possibilitem a verbalização do aluno, que estimulem a livre expressão,
para que ele possa, a partir dos exercícios, identificar seu local de
fala, além de desenvolver respeito e empatia pela fala do outro,
aprimorando, também, a convivência social.

Ao utilizar a Literatura de Cordel, o professor poderá abordar a questão
do preconceito linguístico da língua portuguesa, ao estimular a leitura
de poemas que fogem do padrão gramaticalmente institucionalizado. É
possível mostrar aos alunos que a linguagem popular é muitas vezes
discriminada, mesmo fazendo parte de uma cultura rica e diversificada,
quebrando a ideia de que o ideal é necessariamente o padrão.

\subsection{Variação cultural e geográfica}

Como explicitado anteriormente, é sabido que a Literatura de Cordel faz
parte de nossa cultura e tradição. Antes mesmo da chegada das grandes
mídias e meios de comunicação, o Cordel funcionou como instrumento de
disseminação de valores, lendas e conhecimento popular da tradição
nordestina. Levá"-lo à escola é uma maneira de resgate da nossa cultura,
motivando o aluno a conhecer mais sobre nosso país e seus diferentes
povos e regiões, além de nossa história religiosa, econômica e política,
vez que muitos cordéis abordam realisticamente essas questões.

\subsection{Campo artístico e literário}

A ilustração com xilogravura (gravura em madeira) é uma característica
marcante dos folhetos de Cordel, usada para decorar e dar mais vida aos
poemas, além de oferecer material para as mais variadas interpretações
das obras. O uso da técnica deu"-se graças ao baixo custo de produção e
foi fundamental para disseminar a cultura do Nordeste em outras partes
do Brasil. Os traços marcantes da xilogravura de cordel em composição
com os poemas se transformam em uma expressão de linguagem, registrando
a história do nosso povo.

Levar componentes artísticos para a sala de aula é uma forma de chamar à
atenção do aluno, além de proporcionar maior pensamento crítico e
incentivo à expressão artística e literária.

\subsection{Sobre a obra \textit{Uma voz do Nordeste}}

\emph{Uma voz no Nordeste} reúne cinco poemas emblemáticos de Patativa
do Assaré, passando por diferentes temas e versificações, apresentando
as principais características da poesia popular -- poesia essa que,
muitas vezes, é deixada de lado por se distanciar do erudito. No
entanto, Patativa do Assaré contribuiu ativamente para a construção e
divulgação da identidade nordestina, utilizando, em seus poemas, imagens
marcantes da tradição popular e uma grande variedade de personagens que
simbolizam o nordeste brasileiro.

Publicada em 2016, essa coletânea apresenta a identidade sertaneja
vivida e retratada por Patativa do Assaré com tamanha fidelidade, que
fez com que ele recebesse o título de poeta da oralidade, inspirando
poetas, cantadores e músicos.

\subsection{A estrutura da métrica}

A Literatura de Cordel sofreu, estruturalmente, diversas modificações
com o passar dos anos, por se tratar de uma linguagem oral que foi sendo
transformada, também, em escrita. No início, os repentistas não tinham
compromisso com número de versos ou métrica, entretanto a rima sempre
esteve presente nos poemas -- instrumento utilizado para favorecer a
memorização e facilitar a articulação dos repentistas. Entretanto, a
simplicidade não está atrelada apenas à oralidade, mas também ao alcance
social que uma linguagem acessível pode fornecer.

Em \emph{Uma voz do Nordeste}, estão reunidos cinco poemas dos mais
diversos tipos de métricas e rimas, representando a expressão do canto
do poeta e desvencilhando"-se da forma de saber erudita.

\subsection{A Voz do Nordeste}

Em décadas de cantador, grande parte da obra de Patativa do Assaré ficou
apenas na oralidade, sem ser transcrita. Esse é um dos grandes aspectos
da poesia oral e popular: feita pensando em atender a todos os públicos,
sem empecilhos linguísticos ou sociais, transmitindo a tradição e os
acontecimentos da vida cotidiana. É importante ressaltar também a
temática das obras de Assaré que, sob aparente sutilidade e ingenuidade,
aborda temas relevantes para a população nordestina -- população essa
que, sem os cordelistas, nem sempre conseguiam ter acesso à informação.

Além de representar a voz do povo nordestino, Patativa do Assaré também
inspirou muitos escritores e músicos, contribuindo para o crescimento da
cultura popular nordestina.

\subsubsection{Sobre o autor}

Antônio Gonçalves da Silva, mais conhecido como Patativa do Assaré,
nasceu em 5 de março de 1909 em Assaré -- Ceará. Dono de um dos maiores
legados da Literatura de Cordel nordestina, Patativa do Assaré carrega
em suas obras a essência da cultura de seu povo, relatando suas alegrias
e sofrimentos.

Em 1995, o atual presidente do Brasil, Fernando Henrique Cardoso,
prestou uma homenagem pública ao poeta, dedicando"-lhe a medalha José de
Alencar, por suas grandes contribuições em favor do desenvolvimento
cultural do estado do Ceará.

Sua popularidade é notável. Além da premiação recebida por Fernando
Henrique Cardoso, Patativa do Assaré também sempre foi tratado com muito
carinho pelos demais sertanejos e cordelistas, que, frequentemente,
escrevem poemas e canções em sua homenagem.

\subsubsection{Papel do leitor}

O hábito de leitura auxilia na evolução de diversas habilidades, em
especial se desenvolvido durante a infância e adolescência. Ele trabalha
diretamente com o aprimoramento do vocabulário, criatividade,
imaginação, além das habilidades socioemocionais, como maior habilidade
para estabelecer diálogos, lidar com desafios e sentimentos, ajudando na
formação do indivíduo. Esses são aspectos fundamentais para uma melhor
interação social, além de proporcionar sensações ímpares.

É de extrema importância que o professor incentive o aluno a adquirir o
hábito de leitura pois, através dela, é possível que ele aprenda, viaje
e descubra sobre novos lugares e povos sem sair de casa.


\section{Referências complementares}

\subsection{Livros}
\begin{itemize}
\item\textsc{andrade}, Cláudio Henrique Salles; \textsc{silva}, João Melquíades Ferreira; \textsc{barros} Leandro Gomes de. \textit{Feira de versos: poesia de cordel}. São Paulo: Ática, 2019.

Este livro é uma coleção de pérolas do cordel nacional. A obra reúne
textos de três importantes cordelistas: Leandro Gomes de Barros, o
primeiro a editar cordel no Brasil no século \textsc{xix}, João Melquíades, que
apresenta \emph{O pavão misterioso} e Patativa do Assaré, cujos textos
vêm ganhando reconhecimento internacional.

\item\textsc{haurélio}, Marco. \textit{Antologia do cordel brasileiro.} São Paulo:
  Global, 2012.

Nesta antologia, o leitor tem acesso a um leque variado de cordéis,
desde aqueles inspirados nos contos fantásticos e nos contos de fadas,
até outros em que predominam mitos da Grécia Antiga ou que deitam raízes
nas histórias de animais do fabulário mundial.

\item\textsc{suassuna}, Ariano. \textit{Romance da Pedra do Reino e o Príncipe do
Sangue do Vai"-e"-volta}. Rio de Janeiro: Nova Fronteira, 2017.

O romance de Ariano Suassuna, publicado originalmente em 1971, narra a
história de Dom Pedro Dinis Ferreira, o Quaderna, apresentando seu
memorial de defesa perante o corregedor, com ressonâncias da tradição
literária do cordel.
\end{itemize}

\subsection{Filmes}

\begin{itemize}
\item\textit{Auto da Compadecida}. Direção: Guel Arraes (Brasil, 2000).

Baseado na peça de Ariano Suassuna, o filme evoca o imaginário popular e
religioso do Nordeste para contar as aventuras de João Grilo e Chicó,
uma dupla de malandros que sobrevive de trapaças.

\item\textit{Patativa do Assaré -- Ave Poesia}. Direção: Rosemberg Cariny (Brasil, 2007).

O documentário apresenta a trajetória da vida e da obra do poeta
cearense Patativa do Assaré, que explorou em sua obra a riqueza das
tradições populares. Sua história é contada por meio de depoimentos de
amigos, familiares e admiradores que destacam a relevância do artista
para a cultura brasileira.
\end{itemize}

\subsection{Site}

\begin{itemize}
\item\textit{Cordel: Literatura Popular em Verso}\\
(\url{http://www.casaruibarbosa.gov.br/cordel/acervo.html})

No \emph{site} da Fundação Casa de Rui Barbosa, há informações e
materiais diversos sobre o acervo da instituição, com diversos
exemplares representativos da literatura nacional em cordel.
\end{itemize}

\section{Bibliografia comentada}

\begin{itemize}
\item\textsc{alencar}, Maria Silvana Militão de. \textit{A linguagem regional
  popular na obra de Patativa do Assaré}. Fortaleza: Universidade
  Federal do Ceará, 1997.

Este estudo, central para compreender o emprego da variedade regional
nos cordéis de Patativa do Assaré, aproxima a literatura e a análise
sociolinguística.

\item\textsc{barroso}, Oswald; \textsc{barbalho}, Alexandre. \textit{Letras ao sol -- antologia da literatura cearense.} Fortaleza: Fundação Demócrito
  Rocha, 1998.

A obra apresenta uma amostra representativa da poesia popular
nordestina, com ênfase na literatura do Ceará.

\item\textsc{farias}, Pedro Américo de. \textit{Nordestinos: coletânea poética do
  Nordeste brasileiro}. Lisboa: Fragmentos, 1994.

A obra apresenta uma recolha da poesia popular nordestina, com temas
oriundos do folclore e da matéria histórica.

\item\textsc{figueiredo} \textsc{filho}, J. de. \textit{Patativa do Assaré: novos poemas
  comentados}. Ceará: Museu do Ceará, 2005.

A obra apresenta uma antologia comentada de poemas populares de Patativa
do Assaré, com rica contextualização sobre o autor e seu universo.

\item\textsc{haurélio}, Marco. \textit{Literatura de cordel: do sertão à sala de
  aula.} São Paulo: Paulus, 2013.

Declamados ou cantados, os cordéis levaram ao público, da tradição oral
ao contexto escolar, as façanhas dos cangaceiros Lampião e Antônio
Silvino, os milagres do Padre Cícero e outras narrativas populares.

\item \_\_\_\_\_. \textit{Breve história da literatura de cordel.} São
  Paulo: Claridade, 2018.

Esta obra apresenta as origens do Cordel, destaca seus principais
expoentes e mostra o leque de influências dessa tradição na cena
cultural brasileira.

\item\textsc{nascimento}, Lourgeny Damasceno do. \textit{A importância da literatura
  de cordel no cotidiano dos alunos da \textsc{eja}}. Monografia apresentada ao
  Departamento de Artes da \textsc{unb}. Brasília: 2011. (Disponível em:
  \url{https://bdm.unb.br/bitstream/10483/4463/1/2011_LourgenyDamascenodoNascimento.pdf}.
  Acesso em 18 de fevereiro de 2021.)


A autora destaca a relevância do trabalho com poemas em cordel no
trabalho com jovens e adultos, a partir do resgate das tradições
populares e da valorização dos saberes regionais.

\item\textsc{tavares}, Braulio. \textit{Contando histórias em versos: poesia e
  romanceiro popular no Brasil.} São Paulo: Editora 34, 2009.

O autor apresenta os principais recursos expressivos da linguagem
poética popular, enquanto introduz os leitores a rimas, ritmos, temas,
formas literárias e modos narrativos tipicamente brasileiros.
\end{itemize}

\end{document}

