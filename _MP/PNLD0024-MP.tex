\documentclass[11pt]{extarticle}
\usepackage{manualdoprofessor}
\usepackage{fichatecnica}
\usepackage{lipsum,media9,graficos}
\usepackage[justification=raggedright]{caption}
\usepackage[one]{bncc}
\usepackage[circuito]{../edlab}

% \Image{Monteiro Lobato em meados de 1920 (Publicado na coleção "Nosso Século" (1980) da Editora Abril - volume relativo a 1910-1930, página 186; Domínio Público )}{PNLD0024-03.png}
% \Image{Selos comemorativos de 1973 com as personagens do Sítio do Pica-Pau Amarelo: Emília, Tia Nastácia, Narizinho, Pedrinho, Quindim, Visconde de Sabugosa, Rabicó, Burro Falante, Dona Benta (Correios do Brasil; Domínio Público )}{PNLD0024-04.png}
% \Image{O livro "Cidades Mortas", originalmente publicado em 1919, reúne os primeiros escritos do autor (Internet Archive ; Creative Commons CC0)}{PNLD0024-05.png}
% \Image{Considerada obra-prima do autor, publicada em 1920 (Internet Archive ; Domínio Público )}{PNLD0024-06.png}
% \Image{Primeiro livro da série de histórias do autor dedicada às crianças (Biblioteca Brasiliana Guita e José Mindlin; Domínio Público )}{PNLD0024-07.png}
% \Image{O autor traduziu e adaptou diversas obras, dentre as quais "Por quem os sinos dobram" de Ernest Hemingway (Hemeroteca Digital – Biblioteca Nacional; Domínio Público )}{PNLD0024-08.png}
% \Image{Por meio dessa personagem, passava-se às crianças noções de higiene e saneamento (Biblioteca Brasiliana Guita e José Mindlin; Domínio Público )}{PNLD0024-09.png}
% \Image{Por meio dessa personagem, passava-se às crianças noções de higiene e saneamento (Biblioteca Brasiliana Guita e José Mindlin; Domínio Público )}{PNLD0024-10.png}
% \Image{Por meio dessa personagem, passava-se às crianças noções de higiene e saneamento (Biblioteca Brasiliana Guita e José Mindlin; Domínio Público )}{PNLD0024-11.png}

\begin{document}


\newcommand{\AutorLivro}{Monteiro Lobato}
\newcommand{\TituloLivro}{Um suplício moderno e outros contos}
\newcommand{\Tema}{Ficção, mistério e fantasia}
\newcommand{\Genero}{Conto, crônica e novela}
\newcommand{\imagemCapa}{./images/PNLD0024-01.png}
\newcommand{\issnppub}{---}
\newcommand{\issnepub}{---}
% \newcommand{\fichacatalografica}{PNLD0024-00.png}
\newcommand{\colaborador}{\textbf{Ieda Lebensztayn}}


\title{\TituloLivro}
\author{\AutorLivro}
\def\authornotes{\colaborador}

\date{}
\maketitle

\begin{abstract}\addcontentsline{toc}{section}{Carta ao professor}

Este manual estabelecerá pontos de contato entre a matéria literária
apresentada neste livro e você, que é quem de fato coloca em prática o
processo educativo. O presente volume é uma seleção de contos escritos
por Monteiro Lobato, entre 1900 e 1939, muitos deles publicados
primeiramente na imprensa, na \emph{Revista do Brasil}, que ele adquiriu
e passou a editar em 1918. É incrível, você verá: muitas dessas
narrativas já completaram cem anos, mas ainda são atuais, engraçadas,
comoventes!

Certamente os alunos, se ainda não leram Lobato, já ouviram falar das
personagens do \emph{Sítio do Pica-Pau Amarelo}, que ganharam vida nos
livros e tempos depois nas séries de TV. Mas sempre é tempo de conhecer,
por meio do estilo da escrita de Monteiro Lobato, as aventuras e
diálogos vividos pela boneca-gente Emília, pelo sábio espiga de milho
Visconde de Sabugosa, pelas crianças Pedrinho e Narizinho, pelas boas
senhoras Tia Nastácia e Dona Benta, pelo Marquês de Rabicó, o Burro
Falante, o Quindim.

Então, lembrando aqui a riqueza da obra infantojuvenil de Lobato, fica
reforçado o convite para agora conhecer a obra voltada para o público
adulto, a que não faltam o espírito crítico, como o da Emília, o humor,
o suspense, reviravoltas e a sabedoria do desejo de compreender os
conflitos humanos. Monteiro Lobato publicou os volumes de contos
\emph{Urupês} (1918), \emph{Cidades mortas} (1919), \emph{Negrinha}
(1920) e \emph{O macaco que se fez homem} (1923). À edição original de
\emph{Negrinha} Lobato depois acrescentou textos.

Aqui será fornecida uma série de questões, propostas de atividades e
materiais suplementares que permitirão conhecer a realidade social, os
dramas e momentos poéticos de personagens criadas por Lobato, como o
comprador de fazendas, o Jeca Tatu, o estafeta, Negrinha, o jardineiro
Timóteo, o galo Peva, o menino Pedrinho. Aqui, com os textos de Monteiro
Lobato e as atividades propostas, você terá elementos para entender
melhor a nossa realidade atual, rir ou chorar dela, por meio da arte de
contos e também de suas adaptações para filmes.

Esperamos que ache útil e divertido para o bom trabalho em sala de aula!
\end{abstract}

\tableofcontents


\section{Atividades 1}
%\BNCC{EM13LP26}

\subsection{Tema 1}

O significado de conto e seu potencial para o cinema: a representação
social de homens parasitas, a expressão de conflitos subjetivos e a
construção de reviravoltas no entrecho, entre humor e fim trágico.

%\textbf{(Habilidades BNCC:} EM13LP45, EM14LP48, EM13LP51, EM13CHS101,EM13CHS104, EM13CHS105)

\paragraph{Conteúdo} ``O comprador de fazendas''.

\paragraph{Objetivos e justificativa}
Para conhecer a técnica de Monteiro Lobato como contista, vale ler ``O
comprador de fazendas'', conto de 1917, publicado no ano seguinte no
volume \emph{Urupês}. Ele mostra como Lobato é capaz de criar situações
engraçadas de fundo trágico, fincadas na realidade brasileira, vividas
por personagens muito bem caracterizadas em sua posição social e em sua
condição de parasitas.
\SideImage{Monteiro Lobato em meados de 1920 (Publicado na coleção ``Nosso Século'' (1980) da Editora Abril - volume relativo a 1910-1930, página 186; Domínio Público)}{PNLD0024-03.png}

Acompanhe o entrecho: O fazendeiro Davi Moreira de Sousa, quando comprou
a propriedade chamada Espigão, imaginou que tinha feito um negócio da
China. Porém, aquilo era uma espiga, quer dizer, aquela fazenda não ia
para a frente, e Davi só fazia coçar a cabeça grisalha.

Então, ele resolveu disfarçar os problemas da fazenda do Espigão e
fingir que ela era uma maravilha, para espigar, ou seja, enganar um
trouxa rico, que comprasse a propriedade.

Mas quem sabe se o pretendente a comprador da fazenda, o jovem e
bem-apessoado Pedro Trancoso de Carvalhais Fagundes, tem muito dinheiro
mesmo? Quem sabe é Lobato, e você logo saberá, ao ler o conto. Mas
sobretudo quem precisa saber da situação real de Pedro Trancoso é Zilda,
a filha do fazendeiro arruinado. Ela ficou com o ``coração aos pinotes''
diante dos elogios de Trancoso, que gostou dela, bem como da
hospitalidade da família Moreira.

Tamanha a força desse conto, inspirou a adaptação para dois filmes,
curiosamente ambos dirigidos por Alberto Pieralisi. O primeiro,
realizado em 1951 pela Companhia Cinematográfica Maristela, em São
Paulo, foi estrelado pelos veteranos atores Procópio Ferreira e
Henriette Morineau como marido e mulher, Hélio Souto como Trancoso, e
Margot Bittencourt como Zilda. Composta para o filme, a música ``Festa
no Arraiá'' é executada pelo autor Luiz Gonzaga. A outra versão do filme
sobre o conto é de 1974, tendo Jorge Dória e Lélia Abramo como o casal
de fazendeiros, Agildo Ribeiro no papel do comprador, e Eliane Martins
como a filha do fazendeiro.

\paragraph{Metodologia}

\subsubsection{Pré-leitura}

Pedir que os alunos procurem no dicionário e anotem os significados das
palavras \emph{espiga}, \emph{espigão} e \emph{espigar}.

Promover um bate-papo com os alunos, perguntando como definiriam o
gênero conto e pedindo para se recordarem de obras literárias adaptadas
para o cinema. As questões se multiplicam a partir daí: Como a
construção das personagens, a delimitação do espaço e do tempo da
narrativa, o pendor realista e ou fantasioso e o estilo podem garantir
um bom conto? Quais os desafios, limites e conquistas ao se adaptarem
contos ou romances para a linguagem cinematográfica? Preferem as versões
originais ou os filmes? Lembram-se de atrizes e de atores que
representaram de forma marcante personagens literárias? O modo como você
imaginava a caracterização de certa personagem era muito diferente da
atriz escolhida para o papel? Se você fosse o diretor de uma adaptação
cinematográfica de obra literária, quais atores convidaria?

Vale retomar o sentido de vitalidade com que Monteiro Lobato concebe o
conto, apresentado em ``O resto de onça'', de \emph{Cidades mortas}:
``Quero conto que conte coisas; conto donde eu saia podendo contar a um
amigo o que aconteceu, como o fulano morreu, se a menina casou, se o mau
foi enforcado ou não. Contos, em suma, como os de Maupassant ou
Kipling\ldots{}''.

\Image{O livro ``Cidades Mortas'', originalmente publicado em 1919, reúne os primeiros escritos do autor (Internet Archive; Creative Commons CC-BY)}{PNLD0024-05.png}

\subsubsection{Leitura}

Observar a caracterização das personagens e seu significado como
representação da realidade social brasileira, em que sobressaem os
parasitas, desejosos de se aproveitarem do alheio. Vejam-se o fazendeiro
arruinado Davi Moreira de Sousa, sua mulher Isaura, o filho Zico, a
filha Zilda e o arrivista Pedro Trancoso, que se arroga rico e candidato
a dono da fazenda e do amor de Zilda. Cumpre perceber como, junto com a
representação social, se desenham a construção do entrecho e o efeito
humorístico de fundo trágico do conto. Enfatize-se o jogo de interesses,
disfarces, enganos e desenganos que constituem os conflitos entre as
personagens, com seus desejos, afetos e necessidades econômicas e
sociais. Interessa apontar para os alunos os efeitos da escolha de
palavras como \emph{espiga}: tem sentido aumentativo irônico no nome da
fazenda, afinal o Espigão é decadente; na frase ``Espiga é o que aquilo
é!'', a palavra tem o sentido de contratempo, o aborrecimento contido na
história do Espigão; e, como verbo, \emph{espigar} significa enganar,
prejudicar, que é o cerne do movimento do conto, de todos os
negociadores da fazenda.

\subsubsection{Pós-leitura}

Propor que os alunos assistam às duas versões de filmes intitulados
\emph{O comprador de fazendas}, inspirados no conto de Lobato. A
primeira, de 1950, e a segunda, de 1974.

Pedir que eles observem as soluções encontradas pelo diretor nas duas
adaptações: um bom exercício crítico é apontarem semelhanças e
diferenças entre o conto e os filmes, e depois entre as duas versões de
filme. Então, poderão refletir sobre a opção por um final feliz ou
trágico e os efeitos que criam no público. Como decorrência, os alunos
podem pensar sobre as possíveis estratégias de uma arte de formação de
consciência crítica, com humor ou não, ou voltada sobretudo para o
entretenimento.

\paragraph{Tempo estimado} Duas aulas de 50 minutos.

\subsection{Tema B}

A construção literária da ironia, do riso, da compaixão e do instante
poético como fundamentos da moral e da atitude antirracista.
%\textbf{(Habilidades BNCC:} EM13LP45, EM13LP51, EM13CHS102, EM13CHS104,EM13CHS105)

\paragraph{Conteúdo} ``Negrinha''.

\paragraph{Objetivos e justificativa}
Capacitar os alunos a compreenderem a ironia como figura de linguagem
articulada à intenção autoral de representar criticamente as iniquidades
da realidade brasileira, de origem escravocrata, e expressar compaixão
pelos que sofrem.

``Negrinha'' é um conto dolorosamente belo, em que a perspectiva crítica
do narrador e o efeito poético se configuram à medida que ele acompanha
os sofrimentos da menina negra órfã, a crueldade de dona Inácia para com
ela, vinda da naturalização da violência escravocrata e travestida de
benevolência aos olhos da sociedade e da igreja, e o riso das meninas
ricas ante a ingenuidade de Negrinha, que se maravilhara ao ver pela
primeira vez uma boneca.

A leitura do conto ``Negrinha'' possibilita desenvolver-se a consciência
dos estudantes quanto às desigualdades sociais e às injustiças de origem
histórica e formar-se a sensibilidade estética e a compreensão para com
o outro, de maneira a combaterem a naturalização de atitudes racistas,
preconceituosas.

\Image{Considerada obra-prima do autor, publicada em 1920 (Internet Archive ; Domínio Público )}{PNLD0024-06.png}

\subsubsection{Pré-leitura}

Abrir um bate-papo com os alunos, perguntando como eles definiriam
ironia e pedindo que se lembrem de contextos em que se usa essa figura
de linguagem.

Na sequência, propor que pensem e formulem hipóteses para explicar o que
é o riso, o que o provoca, qual o seu significado e suas consequências,
seus possíveis efeitos, para quem ri e, talvez, para quem é objeto do
riso.

Então, perguntar aos alunos o que entendem por compaixão, misericórdia,
se já se compadeceram de alguém, em que circunstâncias.

Aproveitando as respostas dos estudantes, importa explicar a eles os
significados de ironia, riso e compaixão, baseando-se principalmente nas
obras de Muecke, Bergson e Schopenhauer aqui referidas e nas acepções
apresentadas pelo dicionário (pode ser \emph{Aulete} digital,
\emph{Houaiss}). Indicar para que leiam tais obras.

Pedir que os estudantes pesquisem sobre a origem colonial escravocrata
brasileira e promover um primeiro bate-papo com eles a respeito das
consequências dessa formação histórica para a realidade de desigualdades
sociais do país.

\subsubsection{Leitura}

Mostrar como a ironia serve para a caracterização das personagens,
contrastando a fragilidade da menina Negrinha e a impiedade com
aparência de benignidade da senhora.

Logo o narrador nos coloca diante de uma pobre órfã de sete anos,
nascida na senzala, de mãe escrava. Depois de afirmar que a patroa não
gostava de crianças, o que nos causa distanciamento, ele se esmera em
elogiá-la como ``excelente''; porém, ao escancarar que a virtude da
mulher, agradável aos padres, era a riqueza, ``entaladas'' suas banhas
na cadeira de balanço, ``trono'' de quem se considerava dona do mundo,
ressalta a ironia aos olhos do leitor:

\begin{quote}
Excelente senhora, a patroa. Gorda, rica, dona do mundo, amimada dos
padres, com lugar certo na igreja e camarote de luxo reservado no céu.
Entaladas as banhas no trono (uma cadeira de balanço na sala de jantar),
ali bordava, recebia as amigas e o vigário, dando audiências, discutindo
o tempo. Uma virtuosa senhora, em suma --- ``dama de grandes virtudes
apostólicas, esteio da religião e da moral'', dizia o reverendo.

Ótima, a dona Inácia.
\end{quote}

Vale apontar para os alunos o modo como o narrador configura, a um
tempo, a crueldade de dona Inácia e a reificação de Negrinha. Flagra a
inveja e o egoísmo da mulher: viúva sem filhos, não suportava choro de
criança. Enquanto ainda era viva a mãe da menina, tinha de calar com
beliscões o choro da criança, de fome e frio -- vejam que horror. À
pergunta retórica e insensível da senhora -- ``Quem é a peste que está
chorando aí?'' --, o narrador devolve como resposta o desnudamento da
coisificação a que dona Inácia submetia a ``criminosa'' Negrinha,
tratada como ``diabo'' e ``peste'': ``Quem havia de ser? A pia de lavar
pratos? O pilão? O forno?''.

Lobato, que se consagraria criando um mundo literário para as crianças,
revela, em especial no conto ``Negrinha'', grande sensibilidade para com
as emoções e percepções do universo infantil: observa que, órfã aos
quatro anos e tratada a pontapés, sempre imóvel, com os olhos assustados
e a tremer, Negrinha ``não compreendia a ideia dos grandes. Batiam-lhe
sempre, por ação ou omissão. A mesma coisa, o mesmo ato, a mesma palavra
provocava ora risadas, ora castigos''. Sua única alegria do tempo era
ver o relógio cuco. Lobato mostra como a menina internalizara o fato de
não ter direito a nada, tanto que até seu sorriso era reprimido:
``Sorria-se então por dentro, feliz um instante''.

Ao perguntar ``Que ideia faria de si essa criança que nunca ouvira uma
palavra de carinho?'', o narrador partilha sua compreensão quanto à
importância do afeto para a formação da autoconsciência, criando empatia
do leitor para com Negrinha. O uso do diminutivo, na afirmação de que
ela não teria ``gostinho'' algum na vida, nos aproxima dramaticamente da
menina. Ela era chamada pelo diminutivo, mas não por afeto, e o narrador
ironiza que a ``mimoseassem'' com apelidos terríveis: ``Pestinha, diabo,
coruja, barata descascada, bruxa, pata-choca, pinto gorado, mosca-morta,
sujeira, bisca, trapo, cachorrinha, coisa-ruim, lixo''. Crueldade
impensável, a pandemia de então levou a que a chamassem de bubônica,
suspendendo-se o apelido porque a sonoridade da palavra agradara à órfã.

Chamar a atenção dos estudantes para a forma como, segundo mostra o
narrador, dona Inácia trazia naturalizada, introjetada em seus gestos de
dar cocres, puxões de orelha, beliscões e ovo fervente na boca da órfã,
a violência dos senhores contra escravos: ``Vinha da escravidão, fora
senhora de escravos -- e daquelas ferozes, amigas de ouvir cantar o bolo
e estalar o `bacalhau'. Nunca se afizera ao regime novo -- essa
indecência de negro igual a branco e qualquer coisinha: a polícia!''.
Sobressai a consciência histórica de Lobato quanto ao mal da violência e
do preconceito contra os africanos, que permanece na sociedade após a
Abolição da escravatura: ``O Treze de Maio tirou-lhe das mãos o
azorrague, mas não lhe tirou da alma a gana. Conservava Negrinha em casa
como remédio para os frenesis. Inocente derivativo''.

Acompanhar com os alunos a percepção de que a chegada das duas sobrinhas
de ``Santa'' Inácia faz se desenrolarem os momentos mais trágicos e
poéticos do conto. Depois de enlevada pela ilusão de que não seria crime
brincar, Negrinha é ``chicoteada'' pela ``dura lição da desigualdade
humana'', ``angústia moral'' a se somar às conhecidas dores físicas:
sim, aquelas meninas ricas e loiras eram crianças como ela, mas por que
bem tratadas?

Em seguida, os alunos notarão a força com que o conto configura o
instante em que, ao presenciar a alegria das sobrinhas diante do
maravilhamento de Negrinha com a Boneca nos braços, Dona Inácia ``pela
primeira vez na vida foi mulher. Apiedou-se''. Aí importa retomar o
sentido da compaixão como ``fundamento da moral'', conforme
Schopenhauer, e então o teor antirracista do conto. A ética, o
imperativo do respeito ao outro, toma forma ao se acompanhar o momento
em que a menina, vivendo a experiência de brincar com a Boneca,
reconhece que tem uma alma; já não suportaria a coisificação a que a
``caridosa'' Dona Inácia a submetia:

\begin{quote}
Negrinha, coisa humana, percebeu nesse dia da boneca que tinha uma alma.
Divina eclosão! Surpresa maravilhosa do mundo que trazia em si e que
desabrochava, afinal, como fulgurante flor de luz. Sentiu-se elevada à
altura de ente humano. Cessara de ser coisa -- e doravante ser-lhe-ia
impossível viver a vida de coisa. Se não era coisa! Se sentia! Se
vibrava!

Assim foi -- e essa consciência a matou.
\end{quote}

\subsubsection{Pós-leitura}

Propor que os alunos leiam os contos de Machado de Assis ``O alienista''
e ``Caso da vara''. Neles, representam-se personagens negros que vivem
instantes de felicidade para dentro, porque lhes é barrada a expansão, o
que ocorre também com Negrinha. Pedir que os alunos identifiquem tais
personagens, comentem sua caracterização e reflitam sobre os efeitos que
provocam nos leitores.

Sugerir que pesquisem livros a respeito da colonização escravocrata e de
suas consequências para a sociedade brasileira, como \emph{O
abolicionismo}, de Joaquim Nabuco; \emph{Casa-grande \& senzala}, de
Gilberto Freyre; \emph{Raízes do Brasil}, de Sérgio Buarque de Holanda;
\emph{Dialética da colonização}, de Alfredo Bosi; \emph{Escravidão}, de
Laurentino Gomes. E promover um debate, incitando os alunos a
relembrarem as passagens mais marcantes de ``Negrinha'', articularem-nas
com essas leituras e reavaliarem as questões tratadas no bate-papo
inicial.

Propor a leitura dos textos de Graciliano Ramos ``Baleia'', capítulo de
\emph{Vidas secas}, e ``Minsk'', conto incluído em \emph{Insônia}, no
sentido de os alunos observarem como os textos literários constroem
poeticamente a representação social de injustiças junto com a expressão
de sofrimentos das personagens, com destaque para a imagem da neblina.

\paragraph{Tempo estimado} Duas aulas de 50 minutos.

\subsection{Tema C}

O amor invisível: o ponto de vista narrativo que cria um efeito de
aproximação com personagens sofridos, em especial crianças.

%\textbf{(Habilidades BNCC:} EM13LP45, EM13LP51, EM13CHS102, EM13CHS104,EM13CHS105)

\paragraph{Conteúdo} ``O fisco (Conto de Natal)'' e ``Duas cavalgaduras''.

\paragraph{Objetivos e justificativa}
O objetivo é que os estudantes percebam a necessidade de lutar e
resistir contra o predomínio dos interesses materiais e imediatistas e
suas visões estereotipadas, o qual impede as pessoas de perceberem a
existência de um universo de afeto que também pode e deve presidir as
ações.

Tanto ``O fisco (Conto de Natal)'', de 1918, publicado no volume
\emph{Negrinha}, quanto ``Duas cavalgaduras'' , de 1923, incluído
originalmente em \emph{O macaco que se fez homem}, são contos em que o
ponto de vista do narrador se destaca pela habilidade de prender a
atenção do leitor para desvendar injustiças, aproximando-se de
personagens sofridas, em especial crianças.

\Image{Primeiro livro da série de histórias do autor dedicada às crianças (Biblioteca Brasiliana Guita e José Mindlin; Domínio Público )}{PNLD0024-07.png}

\subsubsection{Pré-leitura}

Dividir a turma em grupos, alguns dedicados à leitura de ``O fisco
(Conto de Natal)'', outros se detêm em ``Duas cavalgaduras''.

Para os leitores de ``O fisco'', abrir um bate-papo, perguntando sobre o
significado das palavras \emph{fisco} e \emph{imposto} e pedindo que
contem histórias sobre cobranças de impostos. Vale observar a
materialidade da origem das palavras: conforme o dicionário
\emph{Houaiss}, fisco, do latim \emph{fiscus}, era o ``saco de guardar
dinheiro; as rendas do príncipe, providas pelo erário público''.

Na sequência, pedir que se recordem de contos natalinos que já ouviram e
ou leram.

Propor aos estudantes que se centrarão em ``Duas cavalgaduras'' as
seguintes pesquisas: sobre o significado e a origem da palavra
\emph{belchior}; sobre os sebos no Brasil; sobre a trajetória de Lobato
como editor; sobre o escritor Ribeiro Couto, em especial sobre o livro
de contos \emph{O crime do estudante Batista} (1922), publicado pela
editora de Monteiro Lobato; sobre antissemitismo.

\subsubsection{Leitura}


Pedir que os alunos observem como ``O fisco (Conto de Natal)'', por meio
de uma construção ardilosa, cinematográfica, se compõe de mais de um
olhar voltado à situação de miséria e injustiças que recai sobre
Pedrinho e sua família. A princípio seguimos o ponto de vista cruel do
fiscal e do policial, dirigido contra a criança que desobedece à lei e
trabalha na rua sem atender à fiscalização. As observações do narrador
nos levam à compreensão dos fatos, distanciando-se daquelas
pseudoautoridades que tratavam o menino como cachorro e o acusavam de
cometer um crime e fingir ``arzinho de inocência'': ``A miserável
criança evidentemente não entendia, não sabia que coisa era aquela de
licença, tão importante, reclamada assim a empuxões brutais''.

Depois, o narrador permite-nos conhecer o ponto de vista de Pedrinho, a
generosidade e a pureza de sua intenção de ajudar no sustento da
família. Atando-se o início e o fim da narrativa, sobressai o olhar do
menino para o ignorado e ignorante fiscal: ele imagina ter diante de si
um potencial freguês, que o ajudará a realizar o sonho de surpreender os
pais levando dinheiro para casa do trabalho de engraxate. Como se
observou quanto ao conto ``Negrinha'', aqui também a compaixão
despertada no leitor é fundamento da moral e de força poética: ``Encarou
o homem já a cinco passos e sorriu com infinita ternura nos olhos, num
agradecimento antecipado em que havia tesouros de gratidão''. E, por
fim, da insensibilidade dos representantes do poder à situação precária
daquela família, da incapacidade de ver o amor do menino pela família e
sua capacidade de sacrifício, advém o acirramento de sua tragédia
econômica e afetiva.

Propor que os alunos reflitam sobre o subtítulo entre parênteses --
``Conto de Natal'' --, buscando entender a originalidade do conto de
Lobato e seu teor crítico contra uma sociedade que, fincada no lucro e
conservada por um aparato policial e burocrático, é indiferente a
pessoas como Pedrinho e sua família.

Indicar que os alunos percebam como o conto ``Duas cavalgaduras''
envolve um mistério que nada mais é que o da compreensão do outro: ao
prejulgarem o vendedor do sebo, o narrador e a personagem do conto de
Ribeiro Couto cometiam uma injustiça -- daí o título irônico do conto,
por fim elucidado. Retomar com os estudantes as reflexões sobre
estereótipos, feitas desde a leitura de outros contos lobatianos, e
refletir sobre as relações entre arte e vida, com base em comentários do
narrador acerca dos vínculos entre ficção e realidade.

\subsubsection{Pós-leitura}

Promover um evento de apresentação e debate dos contos, com base na
leitura e nas reflexões realizadas pelos grupos.

Nesse sentido, os alunos decidem se narram os entrechos dos contos ou os
representam, depois apresentam aos colegas análises sobre as estratégias
que permitem ao escritor criar o suspense e o sentido crítico e ético,
de respeito às crianças e às famílias exploradas como a de Pedrinho, de
respeito às histórias pessoais como a do dono do sebo. E por fim
refletem acerca de como resistir à naturalização da desigualdade social,
alimentada por impostos, e como combater as visões estereotipadas,
preconceituosas. Incluído na mesma seção da antologia (``O amor
invisível''), o conto ``Fatia de vida'' pode ser escolhido também por um
grupo para a apresentação e ou articulação com os demais.

A fim de ampliar o repertório dos alunos para o debate, vale recomendar
leituras como: \emph{Vidas secas}, de Graciliano Ramos, que traz também
a questão de ser Fabiano oprimido pelo fiscal da prefeitura, pelo
soldado amarelo, pelo patrão e pelo governo; \emph{Os vendedores de
cigarros da praça Três Cruzes}, de Joseph Ziemian, sobre antissemitismo
e o holocausto judaico; e o \emph{Guia dos sebos das cidades do Rio de
Janeiro e de São Paulo: acrescido de um anexo com alguns dos principais
sebos de Belo Horizonte, Brasília, Curitiba, Goiânia, Porto Alegre,
Recife, Salvador e São Luís do Maranhão}, de Antonio Carlos Secchin.

\paragraph{Tempo estimado} Duas aulas de 50 minutos.


\section{Atividades 2}

\subsection{Tema A}

A representação do parasitismo: os aproveitadores donos, os decadentes e
os olhodaruáveis.

%\textbf{(Habilidades BNCC:} EM13LP45, EM13LP47,EM14LP48, EM13LP51, EM13CHS101, EM13CHS102, EM13CHS104, EM13CHS105)

\paragraph{Conteúdo} ``Velha praga'', ``Urupês'' e ``Um suplício moderno''.

\paragraph{Objetivos e justificativa}
Feita a leitura de ``O comprador de fazendas'', agora pode ampliar-se o
objetivo de compreender a representação de personagens parasitas como o
fazendeiro arruinado e o falso pretendente a comprador da fazenda.
Destaca-se então a figura do Jeca Tatu: em ``Velha praga'', de 1914,
Monteiro Lobato o critica por não se dedicar ao cultivo da terra e fazer
queimadas, que estragam o solo e só permitem depois, e por pouco tempo,
alguma plantação de subsistência. Por isso, o caboclo sobrevive como
nômade, sempre é expulso das terras, que pertencem aos outros. O conto
``Urupês'', também de 1914, ironiza que o ``grande cuidado'' do Jeca
Tatu seja ``espremer todas as consequências da lei do menor esforço''. O
escritor denuncia a realidade de apatia a que o Jeca está preso,
levando-nos a perceber também o parasitismo dos donos da terra e dos
políticos, exploradores do trabalho e da miséria alheios.

\Image{Por meio dessa personagem, passava-se às crianças noções de higiene e saneamento (Biblioteca Brasiliana Guita e José Mindlin; Domínio Público )}{PNLD0024-11.png}

Justamente o contraste entre os parasitas poderosos e os trabalhadores
``olhodaruáveis'', segundo o expressivo neologismo criado por Monteiro
Lobato, é o centro do conto ``Um suplício moderno''. Embora escrito em
1916, fala de uma realidade semelhante à atual, de políticos que fazem
tramoias em nome de interesses particulares, e de condutores explorados,
os estafetas do correio, que nos lembram os motoristas de aplicativos
como \emph{Uber} \emph{e iFood} de hoje.

Biriba, o protagonista, é um estafeta, ou seja, tipo de carteiro,
funcionário público encarregado de fazer entregas. Recebeu esse cargo
como uma honra, depois de ter-se empenhado, ``fósforo eleitoral'' a
vender votos, pela eleição do coronel. Porém, o trabalho do estafeta era
estafante, eternamente montado numa égua para levar aqui e ali cartas e
mercadorias, e sujeito à demissão por qualquer atraso. Então, exausto de
seu trabalho, com a intenção de ser demitido, Biriba deixa de entregar
um documento que era fundamental para o coronel se conservar no poder.
Eis que toma corpo o humor criado por Lobato: apesar de o estafeta haver
falhado em sua função, o novo coronel, ao assumir o poder, não o demite,
porque afinal foi beneficiado pela negligência de Biriba.

Assim, Lobato nos faz rir do estafeta, por fim desiludido, afinal o
apoio ao político não lhe permitiu fazer parte da ``falange gorda dos
carrapatos orçamentívoros que pacientemente devoram o país'', e sim
permanecer explorado diariamente.

\paragraph{Metodologia}

\subsubsection{Pré-leitura}

Pedir aos alunos que pesquisem a respeito da personagem Jeca Tatu,
presente nos textos de Lobato ``Velha praga'' e ``Urupês''.

Abrir um bate-papo perguntando se sabem o que significam as palavras
\emph{estereótipo}, \emph{estafeta} e \emph{suplício}. Solicitar que
procurem os significados no dicionário (pode ser no \emph{Aulete}
digital). Indicar que etimologicamente \emph{suplício} equivale a ``ato
de dobrar os joelhos'', o que interessará para o sentido de exploração
do estafeta, centro do conto.

Na sequência, a fim de que os estudantes compreendam o alcance da
reflexão sobre \emph{estereótipo}, vale indicar os textos ``O enigma do
olhar'', de Alfredo Bosi (de \emph{Machado de Assis: o enigma do
olhar}), e ``Entre a opinião e o estereótipo'', de Ecléa Bosi, com
destaque para estes trechos:

O estereótipo nos é transmitido com tal força e autoridade que pode
parecer um fato biológico. {[}\ldots{}{]}

O repouso no estereótipo, nas explicações dadas pelo poder, conduz a uma
capitulação da percepção e a um estreitamento do campo mental.
{[}\ldots{}{]}

A consciência se enfraquece quando se dobra à realidade sem tensão; é
preciso despregar a verdade das coisas por um esforço. {[}\ldots{}{]}

Mas só merece de nós um esforço aquilo que amamos.

{[}\ldots{}{]} Tudo começa numa afinidade, numa simpatia do sujeito da
percepção e da ação pelo seu objeto.

Para alcançar esse alto grau de tomada de consciência da vida em si, há
um momento de recusa do que foi estabelecido sem a nossa aquiescência e
experiência. Isto se dá sempre que nós queremos habitar plenamente as
coisas do mundo. (pp. 117, 118, 123, 125).

Então, a proposta é ler ``Velha praga'', ``Urupês'' e ``Um suplício
moderno'', tendo por horizonte compreender personagens como o Jeca e o
estafeta entre estereótipos e pessoas, quer dizer, em seus traços comuns
de tipos sociais e em suas singularidades individuais. E, por
consequência, refletir sobre as possibilidades e limitações da realidade
social, em especial brasileira.

\Image{Por meio dessa personagem, passava-se às crianças noções de higiene e saneamento (Biblioteca Brasiliana Guita e José Mindlin; Domínio Público )}{PNLD0024-10.png}

\subsubsection{Leitura}

Observar a caracterização do Jeca Tatu em ``Velha praga'' e buscar
entender por que o escritor lhe dirige uma crítica forte. Ele se refere
ao caboclo como ``funesto parasita da terra'', um piolho, ressaltando
que ele seria como um selvagem, incapaz de se fixar à terra: ``espécie
de homem baldio, seminômade, inadaptável à civilização''. De maneira
realista, o autor aponta que, com a vida das águas em setembro, o
caboclo consegue plantar na terra ``em cinzas'' algum milho, feijão e
arroz, porém essa produção é insuficiente, e ele estragou a terra com a
queimada. Daí a conclusão de ser o Jeca de ``dolorosa memória para a
natureza circunvizinha''.

Pedir para os alunos atentarem para o modo como, em ``Urupês'', Lobato
acompanha a paralisia do personagem Jeca Tatu diante dos fatos
históricos do país, como a Abolição da escravatura e a proclamação da
República:

\begin{quote}
Pelo Treze de Maio, mal esvoaça o florido decreto da Princesa e o negro
exausto larga num \emph{uf!} o cabo da enxada, o caboclo olha, coça a
cabeça, imagina e deixa que do velho mundo venha quem nele pegue de
novo.

Em 15 de Novembro troca-se um trono vitalício pela cadeira quadrienal. O
país bestifica-se ante o inopinado da mudança. O caboclo não dá pela
coisa.

Vem Floriano; estouram as granadas de Custódio; Gumercindo bate às
portas de Roma; Incitatus derranca o país. O caboclo continua de
cócoras, a modorrar... Nada o esperta. Nenhuma ferrotoada o põe de pé.
Social, como individualmente, em todos os atos da vida, Jeca, antes de
agir, acocora-se.
\end{quote}

A crítica lobatiana assume a forma da ironia contra o parasitismo, a
inação do caboclo: ``Seu grande cuidado é espremer todas as
consequências da lei do menor esforço''. E ressalta a inconsciência do
Jeca em relação à própria realidade social e política de exploração: ele
vota sem saber em quem, sempre no Governo mesmo.

\begin{quote}
O fato mais importante de sua vida é sem dúvida votar no Governo. Tira
nesse dia da arca a roupa preta do casamento, sarjão furadinho de traça
e todo vincado de dobras; entala os pés num alentado sapatão de bezerro;
ata ao pescoço um colarinho de bico e, sem gravata, ringindo e mancando,
vai pegar o diploma de eleitor às mãos do chefe Coisada, que lho retém
para maior garantia da fidelidade partidária.

Vota. Não sabe em quem, mas vota. Esfrega a pena no livro eleitoral,
arabescando o aranhol de gatafunhos e que chama ``sua graça''.
\end{quote}

Propor aos estudantes que tracem um paralelo entre a caracterização do
Jeca e a do estafeta: o primeiro é sempre tocado das terras, que não
cultiva, num nomadismo miserável, e vota no governo, sem saber em quem;
o segundo também é ``olhodaruável'', mas trabalha sem descanso e tem
interesse por política, no entanto limitado pela crença inútil de que
usufruirá do parasitismo como os poderosos; tal crença gera a revolta e
a traição contra o coronel explorador, daí o desfecho cômico da
narrativa.

Sugerir que os alunos identifiquem como se constrói ``Um suplício
moderno'': a princípio é apresentado o tipo social do estafeta como um
``olhodarruável''; depois, vem a história pessoal do estafeta Biriba,
que escancara, com humor, a exploração sofrida por ele.

Assim, o propósito é, refletindo sobre o estereótipo do caboclo
preguiçoso, do Jeca, e sobre as dificuldades como a de Biriba de ter um
bom emprego, buscar sempre entender várias faces dos problemas e fatores
históricos e econômicos da realidade social, prestando atenção aos
limites éticos e políticos ultrapassados pela exploração do trabalho
alheio.

\subsubsection{Pós-leitura}

Consultar o \emph{site} oficial de Monteiro Lobato
(\url{http://www.monteirolobato.com/miscelania/jeca-tatuzinho}), para
saber sobre o livro \emph{Jeca Tatuzinho}: lançado em 1924, ensinava
para as crianças noções de higiene e saneamento, por meio do personagem
criado por Lobato. Foi um sucesso também o Biotônico Fontoura.

Assistir ao filme \emph{Jeca Tatu}, pesquisar quem foi Mazzaropi e
avaliar em que medida a atuação dele cria empatia com a personagem de
Lobato.

\Image{Por meio dessa personagem, passava-se às crianças noções de higiene e saneamento (Biblioteca Brasiliana Guita e José Mindlin; Domínio Público )}{PNLD0024-09.png}

Pesquisar como era na época de Lobato a realidade das fazendas e dos
trabalhadores rurais no vale do Paraíba e ou em outras regiões do país,
e como é hoje. Refletir sobre a origem histórica das desigualdades
sociais e econômicas da realidade brasileira, país de matriz colonial e
escravocrata.

Considerando a representação do estafeta, funcionário do correio que se
esfalfava no transporte a cavalo, pesquisar e refletir sobre as mudanças
nos transportes e nas comunicações, conforme as inovações tecnológicas.
Pesquisar o caminho de transformações nas comunicações, desde cartas,
telegramas, telefone, fax, até os \emph{smartfones}, que possibilitam a
comunicação \emph{on-line} e também viabilizam o transporte, por meio de
aplicativos como o uber.

Chamar a atenção dos alunos para o adjetivo ``olhodarruável'', criado
por Monteiro Lobato. Então, propor que pesquisem o significado de
\emph{neologismo} e, inspirados na palavra concebida pelo escritor,
inventem outras. Paralelamente, pedir uma pesquisa a respeito da
história do modernismo brasileiro, inclusive da polêmica de Lobato com a
pintora Anita Malfatti, de maneira a perceber a modernidade da obra de
Lobato e o modo como a historiografia pode fazer generalizações e deixar
de lado especificidades das obras.

Cabe refletir, portanto, sobre a realidade representada nos contos de
Lobato, de parasitas donos e ``olhodarruáveis'', e as mudanças e
permanências verificadas no momento atual. Então, a proposta é escrever
um conto procurando representar essas questões e, em grupos, partilhar
as narrativas e depois escolher uma e transformá-la num vídeo.

Outra tarefa divertida é a leitura do conto lobatiano ``Nuvem de
gafanhotos'', de aproveitadores folgadíssimos que nos fazem rir,
representando também o parasitismo social.

\paragraph{Tempo estimado} Duas aulas de 50 minutos.

\subsection{Tema B}

Os povos originários brasileiros e os africanos como protagonistas da história.
%EM13LP45, EM13LP47, EM14LP48,EM13LP51, EM13LP53, EM13CHS102, EM13CHS104, EM13CHS105)

\paragraph{Conteúdo} ``O jardineiro Timóteo'', ``Marabá'' e ``Os negros''.

\paragraph{Objetivos e justificativa}
A leitura dos contos ``O jardineiro Timóteo'', ``Marabá'' e ``Os
negros'', cada um na singularidade de sua efabulação que prende a
atenção dos leitores, desperta melhor compreensão quanto à formação
colonial brasileira, marcada pela violência contra os povos originários
e os africanos, sobretudo em termos da exploração da mão de obra,
visando aos lucros dos colonizadores e, continuamente, à conservação do
poder nas mãos da elite dirigente.

\paragraph{Metodologia}

\subsubsection{Pré-leitura}

Com base em \emph{Reflexões sobre a arte}, de Alfredo Bosi, abrir uma
conversa com os alunos sobre as três dimensões da obra de arte -- a de
representação social, de expressão subjetiva e de construção formal.

Pedir que façam uma pesquisa sobre marcas da colonização na realidade
brasileira, em especial sobre o indianismo e consequências da Abolição
da escravidão.

Propor que leiam as seções ``Os antigos senhores'' e ``Os antigos
escravos'' da \emph{Pequena história da República}, de Graciliano Ramos.
Nelas, o escritor apresenta a situação de desamparo em que ficaram
antigos senhores e ex-escravos após a Abolição. Depois de ganhar o
mundo, a ``preta velha'' retornou arrependida, mas encontrou mudanças:
``os brancos arriados, murchos, bambos; as plantações murchas, bambas,
arriadas; a fazenda quase deserta. A autoridade soberba do patriarca
encolhera. Tudo encolhera -- e nesse encolhimento, nessa conformação, os
ombros caíam resignados, os braços moles se cruzavam, os olhos espiavam
no fogo as panelas escassas. Pobreza, devastação, indícios de miséria.
Desalento, rugas e cabelos grisalhos''.

Graciliano desnuda a miséria da sociedade brasileira, que, constituída
sobre as iniquidades da base escravocrata, não ofereceu aos ex-escravos
condições para se manterem:

\begin{quote}
A alegria tumultuosa dos negros foi substituída por uma vaga
inquietação. Escravos, tinham a certeza de que não lhes faltaria um
pedaço de bacalhau, uma esteira na senzala e a roupa de baeta com que se
vestiam; livres, necessitavam prover-se dessas coisas --- e não se
achavam aptos para obtê-las.
\end{quote}

\subsubsection{Leitura}

Considerando, conforme expõe Alfredo Bosi, as referidas três dimensões
da obra de arte -- mimética, existencial e formal --, perceber como elas
estão configuradas nos três contos.

Em ``O jardineiro Timóteo'', atentar para a construção poética da pureza
do homem que, conhecendo a linguagem das flores, apesar de ex-escravo e
criado, se dedicava com afeto ao cultivo do jardim da casa alheia,
insciente de que a propriedade sequer pertenceria àqueles donos para
sempre.

Em ``Os negros'', observar como se combina a construção da atmosfera
fantasmagórica e trevosa com a representação da realidade histórica de
violência contra os escravos e com a expressão de conflitos, dada a
paixão entre o homem pobre e a filha do fazendeiro.

Em ``Marabá'', destacar a originalidade moderna da construção, que evoca
a literatura indianista de Alencar e abusa da autoironia
metalinguística, num entrecho com suspense que traz a tragédia das
relações amorosas entre um moço branco e uma índia.

\subsubsection{Pós-leitura}

Pedir que os alunos se organizem em grupos e escolham um dos três contos
como base para escreverem um roteiro e gravarem um vídeo representando a
história narrada. Eles releem o conto, conversam sobre os fatos centrais
do enredo e sobre sua construção, para poderem elaborar o roteiro.
Apresentam os vídeos para os colegas. Depois, promove-se uma conversa,
para avaliarem: a importância do texto literário e de sua adaptação em
outras formas artísticas para se compreender melhor a realidade; os
momentos mais prazerosos do trabalho de criação do vídeo, as
dificuldades de adaptar a história e preparar o vídeo; o aprendizado
para a elaboração de próximas atividades semelhantes.

Em especial, a partir de ``O jardineiro Timóteo'', fazer uma pesquisa a
respeito da diversidade de flores referidas no conto, como crisandálias,
crisântemos, esporinhas, periquitos, cinerárias.

\paragraph{Tempo estimado} Duas aulas de 50 minutos.

\subsection{Tema C}

A disponibilidade afetiva para o outro, rara na sociedade humana.
%EM13LP45, EM13LP50, EM13CHS101, EM13LP51,EM13CHS104, EM13CHS105)

\paragraph{Conteúdo} ``Tragédia dum capão de pintos'', ``Quero ajudar o Brasil''

\paragraph{Objetivos e justificativa}
Despertar a imaginação e a sensibilidade e respeito para com os outros,
sejam animais ou pessoas de origens sociais, étnicas diferentes.

Num mundo imediatista, de primazia de interesses individuais e
consumismo, é fundamental que os alunos conheçam a disponibilidade
afetiva do galo Peva, que adota as aves órfãs e sofre enormemente por
elas, e a do homem capaz de sacrificar, pelo bem do país, o pouco
dinheiro de que dispunha.

\paragraph{Metodologia}

\subsubsection{Pré-leitura}

Dividir a turma em grupos: alguns se dedicam à leitura de ``Tragédia dum
capão de pintos''; outros se detêm em ``Quero ajudar o Brasil''.

Apresentar ao primeiro grupo a antologia de contos das várias regiões do
país que o escritor Graciliano Ramos organizou nos anos 1940, depois de
ter pesquisado durante dois meses na Academia Brasileira de Letras e
outros dois na Biblioteca Nacional. Compõe-se de Três volumes: I, Norte
e Nordeste; II, Leste; III, Sul e Centro-Oeste. São um convite para o
leitor se embrenhar na ficção curta brasileira, incluindo escritores
como Machado de Assis, Lima Barreto, Mário de Andrade, Carlos Drummond
de Andrade, Fernando Sabino e alguns pouco conhecidos hoje. Sendo a
edição posterior à morte de Graciliano, seu amigo Aurélio Buarque de
Holanda nela incluiu o conto ``Minsk'', antes publicado em
\emph{Insônia} (1947). De Monteiro Lobato, o conto escolhido por
Graciliano foi ``Tragédia dum capão de pintos''.

Propor que os estudantes, ao lerem esse conto, façam anotações
procurando entender quais seriam os critérios para a escolha desse
texto.

Para os alunos do segundo grupo, pedir que, considerando o título
``Quero ajudar o Brasil'', escrevam um texto indicando problemas do
Brasil e o que seriam capazes de fazer para ajudar o país. E sugerir uma
pesquisa sobre a atuação de Monteiro Lobato a favor do petróleo
brasileiro.


\subsubsection{Leitura}

Pedir para os alunos do primeiro grupo observarem como o narrador
configura a singela história do amor do galo Peva pelos três filhotes
órfãos, que termina em revolta e tragédia, dada a indiferença dos homens
e da própria natureza por aquele afeto puro.

Atentar para a forma criativa como Lobato traduz o olhar dos animais em
relação aos homens, que inclui desconfiança, incompreensão, submissão e
medo. E perceber a crítica à falta de atenção dos homens, cegos às
manifestações de afeto dos animais e à dor que lhes causam.

Solicitar que os estudantes centrados em ``Quero ajudar o Brasil''
percebam como a linguagem, inclusive a utilizada por Lobato, pode ter um
teor racista. Porém, cumpre atentar para o sentido antirracista do
texto, que ressalta o amor impensável do homem afrodescendente pelo
país, a ponto de desejar sacrificar seu dinheiro. Ele tem a ``sublime
serenidade'', pois é capaz de exercer um ``dever de consciência''.

Interessa perceber a forma como o narrador expressa a comoção que a
atitude do homem negro lhe causa. Combinam-se o gesto de esconder as
lágrimas e uma formulação eufemística para o choro: ``Em certas ocasiões
só mesmo derrubando uma caneta e custando a achá-la, porque há umas tais
glândulas que nos turvam os olhos com umas aguinhas
impertinentes\ldots{}''.

A força daquela atitude inacreditável, de sacrificar-se pelo Brasil,
leva a refletir-se sobre o limite ético que deve frear os interesses
comerciais e as propagandas: embora o propósito do narrador fosse
defender a campanha pelo petróleo, não era digno prejudicar aquele homem
singular que, mesmo pobre, priorizava o bem comum.

É preciso realçar a atualidade de certas frases do texto de Lobato, que
desmascaram a estrutura iníqua da sociedade brasileira, desde a origem
colonial e escravocrata: ``A calúnia é a rainha da técnica''; ``Os
salários no Brasil são a miséria que sabemos''; ``mas no Brasil não há
negros ricos''; ``Essa coisa chamada Brasil, que é de vender, que até os
ministros vendem, ele queria ajudar\ldots{}''.

\subsubsection{Pós-leitura}

Pedir que os dois grupos de alunos criem formas de apresentar aos
colegas suas leituras dos contos.

A ``Tragédia dum capão de pintos'' pode inspirar um debate sobre a
relação entre os homens e os animais e sua configuração literária. Nesse
sentido, vale retomar os textos ``Baleia'' e ``Minsk'', de Graciliano
Ramos, que também figuram poeticamente a morte de animais. E, de
Monteiro Lobato mesmo, recomenda-se a leitura do conto ``Era no
Paraíso''.

O debate propiciado pela história do galo Peva e seus filhotes tão
queridos pode incluir questões importantes como a adoção e o veganismo.
Quanto a ``Quero ajudar o Brasil'', leva a um debate sobre o problema do
racismo: algumas frases de Lobato deixam ver um racismo arraigado na
sociedade, embora o sentido do conto seja antirracista. Ele provoca
também uma reflexão sobre os limites e possibilidades de ações
individuais e coletivas, voltadas ao bem do país. E sobre as
ambiguidades ligadas ao conceito de patriotismo. A reflexão pode
ampliar-se com o estudo sobre a luta de Lobato a favor do petróleo
nacional, e a consideração sobre a situação atual da Petrobras. Leiam-se
também, de \textbar{}Lobato: \emph{O escândalo do petróleo} (1936) e
\emph{O Poço do Visconde} (1937). Confira-se:
\url{http://www.monteirolobato.com/linha-do-tempo/1931-1939-a-luta-de-lobato-pelo-petroleo-e-ferro}.

Por fim, retomando o sentido da antologia de contos das regiões do
Brasil organizada por Graciliano Ramos, sugerir aos alunos que comentem
sobre os contos de Lobato de que mais gostaram e por quê. Lançar a
questão: qual conto escolheriam para uma antologia como a de Graciliano?

\paragraph{Tempo estimado} Duas aulas de 50 minutos.

\section{Aprofundamento}

%\subsection{A vida de Monteiro Lobato e a riqueza temática de seus contos}

Fazendeiro, escritor para crianças e adultos, editor, empresário,
defensor do petróleo nacional: a intensidade com que Lobato experienciou
as várias faces de sua vida transparece na vitalidade de seus contos,
frutos de sua sensibilidade, observação crítica, conhecimentos
literários e trabalho intelectual e artístico.

José Bento Renato MONTEIRO LOBATO nasceu em Taubaté, São Paulo, a 18 de
abril de 1882, que ficou consagrado como Dia Nacional do Livro Infantil,
e faleceu em São Paulo, a 4 de julho de 1948.

Escritor de literatura infantojuvenil, contista, jornalista, editor,
tradutor, pintor e fotógrafo. Aos onze anos, mudou seu nome para José
Bento, por causa das iniciais gravadas no castão da bengala do pai,
J.B.M.L. Apesar de sua inclinação para as artes plásticas, cursou a
Faculdade de Direito do Largo São Francisco, em São Paulo, por imposição
do avô, o Visconde de Tremembé. Formado em 1904, voltou a Taubaté, onde
foi nomeado promotor público interino, transferido, em 1907, para
Areias, São Paulo. Enviou artigos para \emph{A Tribuna}, de Santos,
traduções para o jornal \emph{O Estado de S. Paulo} e caricaturas para a
revista \emph{Fon-Fon!}, do Rio de Janeiro. Em 1911 herdou, com as duas
irmãs, a fazenda do avô. Publicou, em 1914, os artigos ``Velha praga'' e
``Urupês'' em \emph{O Estado de S. Paulo}, criando o personagem Jeca
Tatu. Em 1917, vendeu a fazenda e se mudou para São Paulo.

Escreveu em \emph{O Estado de S. Paulo} o artigo ``A propósito da
Exposição de Malfatti'' (``Paranoia ou mistificação?''), de crítica
contra as vanguardas, abrindo polêmica com os modernistas. Em 1918,
estreou com o livro de contos \emph{Urupês}, que esgotou 30 mil
exemplares entre 1918 e 1925, e comprou a \emph{Revista do Brasil},
lançando as bases da indústria editorial no país. \emph{Cidades mortas},
originalmente publicado em 1919, numa edição da \emph{Revista do
Brasil}, reúne os primeiros escritos de Lobato, ainda estudante em
Taubaté, e contos que escreveu antes de viajar a Nova York para ocupar
um posto no Consulado brasileiro.

Criando uma rede de distribuição, com vendedores autônomos e
consignatários, revolucionou o mercado livreiro. Em 1920, fundou a
editora Monteiro Lobato \& Cia, que publicou obras de Lima Barreto, Léo
Vaz, Oswald de Andrade, Ribeiro Couto, Menotti del Picchia, Guilherme de
Almeida, Oliveira Viana e Amadeu Amaral, entre muitos outros. No mesmo
ano, lançou \emph{A menina do Narizinho Arrebitado}, primeira da série
de histórias com que Lobato criou a literatura brasileira dedicada às
crianças, formando gerações de leitores. Em 1924, com capital ampliado e
nova denominação, Companhia Gráfico-Editora Monteiro Lobato, sua editora
monta o maior parque gráfico da América Latina. Porém, no ano seguinte,
dificuldades financeiras o levam a vender a \emph{Revista do Brasil} e
liquidar a editora. Mudou-se para o Rio de Janeiro e fundou a Companhia
Editora Nacional.

Adido comercial em Nova York de 1927 até 1930, voltou ao Brasil com
ideias para a exploração de ferro e petróleo. Fundou empresas de
prospecção, mas, contrariando interesses multinacionais e fazendo
oposição, em artigos e entrevistas, ao governo Vargas, foi preso por
seis meses em 1941. Recebeu indulto depois de cumprir metade da pena,
mas o governo mandou apreender e queimar seus livros infantis.

Em 1944, Lobato recusou indicação para a Academia Brasileira de Letras.
Em 1946, tornou-se sócio da editora Brasiliense. Embarcou para a
Argentina e fundou em Buenos Aires a Editorial Acteon, retornando no ano
seguinte a São Paulo.

Principais obras: a) Livros para crianças: \emph{O Saci} (1921);
\emph{Fábulas} (1922); \emph{Reinações de Narizinho} (1931);
\emph{Viagem ao céu} (1932); \emph{Caçadas de Pedrinho} (1933);
\emph{História do Mundo para as} \emph{Crianças} (1933); \emph{Emília no
País da Gramática} (1934); \emph{Aritmética da Emília} (1935);
\emph{Memórias da Emília} (1936); \emph{O Poço do Visconde} (1937);
\emph{O Picapau Amarelo} (1939); \emph{A Reforma da Natureza} (1941);
\emph{A Chave do Tamanho} (1942); \emph{Os doze trabalhos de Hércules},
dois volumes (1944); b) Livros para adultos: \emph{Urupês} (1918);
\emph{Cidades} \emph{mortas} (1919); \emph{Ideias de Jeca Tatu} (1919);
\emph{Negrinha} (1920); \emph{Mundo da lua} (1923); \emph{O Presidente
Negro/O choque das raças} (1926); \emph{Ferro} (1931); \emph{América}
(1932); \emph{O escândalo do petróleo} (1936); \emph{A barca de Gleyre}
(1944).

\Image{Selos comemorativos de 1973 com as personagens do Sítio do Pica-Pau Amarelo: Emília, Tia Nastácia, Narizinho, Pedrinho, Quindim, Visconde de Sabugosa, Rabicó, Burro Falante, Dona Benta (Correios do Brasil; Domínio Público )}{PNLD0024-04.png}

Lobato traduziu e adaptou diversas obras, entre as quais: \emph{Da
história da filosofia}, de Will Durand; \emph{Memórias}, de André
Maurois; \emph{Por quem os sinos dobram}, de Ernest Hemingway;
\emph{Crepúsculo dos ídolos e Anticristo}, de Friedrich Nietzsche;
\emph{Robinson Crusoé}, de Daniel Defoe; \emph{Mogli, o menino lobo}, de
Rudyard Kipling; \emph{Aventuras de Tom Sawyer}, de Mark Twain;
\emph{Pollyana}, de Eleanor H. Porter; \emph{Moby Dick}, de Herman
Melville; \emph{Tarzan}, de Edgar Rice Burroughs.

\Image{O autor traduziu e adaptou diversas obras, dentre as quais ``Por quem os sinos dobram'' de Ernest Hemingway (Hemeroteca Digital – Biblioteca Nacional; Domínio Público )}{PNLD0024-08.png}

Tendo em vista a riqueza dos contos lobatianos, mostrou-se necessária a
organização deste livro segundo um critério temático: para além de
selecionar os textos e ordená-los apenas seguindo os volumes originais,
o reagrupamento temático visa a chamar a atenção do leitor para a força
do estilo de Lobato, capaz de dar forma a uma variedade de aspectos de
questões semelhantes.

Esta edição organiza os contos em quatro seções: ``Os protagonistas da
história'' traz narrativas que nos levam a compreender a história de
africanos e seus descendentes, bem como de senhores e fazendeiros
decadentes, num país de origem escravocrata, cuja marca continua sendo a
violência; ``Amor invisível'' concentra a sensibilidade de Monteiro
Lobato para com os seres desamparados, especialmente crianças e animais,
em uma sociedade que privilegia o dinheiro e o consumismo; ``Os
parasitas donos, os decadentes e os olhodarruáveis'' tece a
representação de personagens como o caboclo Jeca Tatu e o estafeta
Biriba, bem diferentes, mas todos explorados pelos poderosos. ``Terra
para rir, ou chorar'' reúne contos que revelam insuficiências da
realidade, em especial brasileira, por meio de um efeito cômico,
provocando a consciência crítica e a sensibilidade estética do leitor.

Obra-prima de Lobato, o conto ``Negrinha'', publicado em 1920 no volume
homônimo, desperta comoção e consciência histórica dos leitores: tece a
representação crítica da naturalização da violência, causada por uma
senhora falsamente caridosa, contra uma menina órfã descendente de
escravos; e, a um tempo, expressa com poesia o sofrimento da criança,
que, habituada brutalmente à privação de tudo, só sorria, e para dentro,
diante do relógio cuco e morreu de tristeza depois de haver conhecido a
maravilha que é brincar de boneca e saber que tem uma alma.

Considerando, com Schopenhauer, ser a compaixão o fundamento da moral,
sobressai a lição ética do conto, em que o narrador flagra o momento
único em que a senhora cruel foi gente e se apiedou da criança,
permitindo-lhe brincar com as outras. Tragicamente, no conto, o momento
raro de plenitude da órfã redundará na percepção do vazio de sua vida e
na sua morte. Mas tal lição ética -- o avesso da violência, do racismo,
de preconceitos, da insensibilidade para com o outro -- fica para o
leitor. E assume dimensão política em contos como o atualíssimo ``Um
suplício moderno'', de 1916, em que, depois de conhecermos o tipo social
do estafeta, do encarregado de fazer entregas para os outros,
acompanhamos a história pessoal do estafeta Biriba. É uma história
cômica e trágica: segundo o neologismo criado por Lobato, ele não passa
de um ``olhodarruável'', vergado aos donos do poder, apesar de
interesseiro, desejoso de ser parasita como os políticos que apoia.

\section{Referências complementares}

AZEVEDO, Carmen Lucia de; CAMARGOS, Marcia Mascarenhas de Rezende \&
SACCHETTA, Vladimir. \emph{Monteiro Lobato: furacão na Botocúndia}. 3
ed. São Paulo: Ed. Senac, 2001.

BERGSON, Henri. \emph{O riso: ensaio sobre a significação da
comicidade}. Tradução de Ivone Castilho Benedetti. São Paulo: Martins
Fontes 2007.

BOSI, Alfredo. \emph{Dialética da colonização}. 4 ed. São Paulo:
Companhia das Letras, 2003.

BOSI, Alfredo. ``O enigma do olhar''. In: \emph{Machado de Assis: o
enigma do olhar}. São Paulo: Ática, 1999.

BOSI, Ecléa. ``Entre a opinião e o estereótipo''. \emph{O tempo vivo da
memória: ensaios de psicologia social}. São Paulo: Ateliê, 2003.

CARVALHO FRANCO, Maria Sylvia de. \emph{Homens livres na ordem
escravocrata}. São Paulo: Ática, 1974.

CAVALHEIRO, Edgard. \emph{Monteiro Lobato, vida e obra}. 2 tomos. São
Paulo: Companhia Editora Nacional, 1955.

CHAUVIN, Jean Pierre (org.). \emph{O Alienista, ``O Imortal'' \& ``A
Cartomante''}, de Machado de Assis. São Paulo: Hedra, 2021.

\emph{Dicionário Aulete digital}: \url{http://www.aulete.com.br/}.

FREYRE, Gilberto. \emph{Casa-grande \& senzala}. 12 ed. Brasília: Ed.
UnB, 1963.

GOMES, Laurentino. \emph{Escravidão}. Rio de Janeiro: Globo, 2019.

HOLANDA, Sérgio Buarque de. \emph{Raízes do Brasil}. Prefácio de Antonio
Candido. 4 ed. Brasília: Ed. UnB, 1963.

KOSHIYAMA, Alice Mitika. \emph{Monteiro Lobato: intelectual, empresário,
editor}. São Paulo: Edusp, 2006. Coleção Memória Editorial, 4.

LAJOLO, Marisa. \emph{Monteiro Lobato, livro a livro: obra adulta}. São
Paulo: Editora Unesp, 2014.

LAJOLO, Marisa \& CECCANTINI, João Luís (orgs.). \emph{Monteiro Lobato,
livro a livro: obra infantil}. São Paulo: Editora Unesp; Imprensa
Oficial do Estado de São Paulo, 2008.

LOBATO, Monteiro \& RANGEL, Godofredo. \emph{A barca de Gleyre: quarenta
anos de correspondência literária}. São Paulo: Brasiliense, 1968.

MUECKE, D. C. \emph{Ironia e o irônico}. Tradução de Geraldo Gerson de
Souza. São Paulo: Perspectiva, 1995.

MONTEIRO LOBATO, \emph{site} oficial:
\url{http://www.monteirolobato.com/}.

NABUCO, Joaquim. \emph{O abolicionismo}. Rio de Janeiro: Nova Fronteira;
São Paulo: Publifolha, 2000 (Grandes Nomes do Pensamento Brasileiro).

\emph{O COMPRADOR DE FAZENDAS}, Filme adaptado do conto de Monteiro
Lobato, do volume \emph{Urupês} (1918). Direção de Alberto Pieralisi.
São Paulo, Companhia Cinematográfica Maristela, 1950. Comédia, P\&B:
\url{https://www.youtube.com/watch?v=LcdfdfD9_Bs}.

\emph{O COMPRADOR DE FAZENDAS}, Filme adaptado do conto de Monteiro
Lobato, do volume \emph{Urupês} (1918). Direção de Alberto Pieralisi.
Rio de Janeiro, Embrafilme -- Empresa Brasileira de Filmes S.A., 1972:
\url{https://www.youtube.com/watch?v=C9OrDOQWm5o}.

RAMOS, Graciliano. \emph{Pequena história da República} (1940). In:
\emph{Alexandre e outros heróis}. 3. ed. São Paulo: Martins, {[}1962{]}
1966.

RAMOS, Graciliano. \emph{Vidas secas}. 140 ed. Rio de Janeiro: Record,
{[}1938{]} 2019.

SCHOPENHAUER, Arthur. \emph{Sobre o fundamento da moral}. Tradução de
Maria Lúcia Mello Oliveira Cacciola. 2 ed. São Paulo: Martins Fontes,
2001 (Coleção Clássicos).

SECCHIN, Antonio Carlos. \emph{Guia dos sebos das cidades do Rio de
Janeiro e de São Paulo: acrescido de um anexo com alguns dos principais
sebos de Belo Horizonte, Brasília, Curitiba, Goiânia, Porto Alegre,
Recife, Salvador e São Luís do Maranhão}. Rio de Janeiro: Fundação
Biblioteca Nacional; Sociedade de Amigos da Biblioteca Nacional; Nova
Fronteira, 2002.

VILLAÇA, Alcides. ``Querer, poder, precisar: `O caso da vara'''.
\emph{Teresa: Revista de Literatura Brasileira}, São Paulo, USP; Editora
34; Imprensa Oficial, n. 6-7, pp. 17-30, 2006.

ZIEMIAN, Joseph. \emph{Os vendedores de cigarros da Praça Três Cruzes}.
Tradução de Jacob Lebensztayn. São Paulo: Três Estrelas, 2019.

\section{Bibliografia comentada}

ANDRADE, Mário de. ``Contos e contistas'' {[}1938{]}. In: \emph{O
empalhador de passarinho}. 3. ed. São Paulo; Brasília: Martins; INL,
1972, pp. 5-8. Motivado por uma pesquisa da \emph{Revista Acadêmica} em
busca dos dez melhores contos brasileiros, o artigo reflete sobre esse
gênero literário.

BOSI, Alfredo (org.). \emph{O conto brasileiro contemporâneo}. 16 ed.
São Paulo: Cultrix, {[}1975{]} 2015. Conforme o organizador indica no
prefácio, ``Situação e formas do conto brasileiro contemporâneo'' (pp.
7-22), a antologia possibilita aos leitores conhecerem a variedade do
gênero, entre realismo, fantasia e experimentação, na produção
brasileira posterior ao modernismo, caminho pavimentado por Lobato.
Notas biobibliográficas acerca de seus autores acompanham os textos
selecionados.

BOSI, Alfredo. \emph{Reflexões sobre a arte}. 4. ed. São Paulo: Ática,
1991. Obra que apresenta o sentido da arte como combinação de três
dimensões: a vertente de construção formal, a de representação social e
a de expressão subjetiva, às quais se soma a vertente de transitividade
com o leitor. A observação de como essas dimensões se articulam nas
obras de arte constitui o exercício crítico, conforme Bosi empreendeu em
relação à fortuna crítica de Machado de Assis, no livro \emph{Brás Cubas
em três versões}, publicado pela Companhia das Letras em 2006. Boa
síntese se encontra no artigo-conferência ``Machado de Assis na
encruzilhada dos caminhos da crítica'' (\emph{Machado de Assis em
Linha}, ano 2, n. 4, dez. 2009, disponível em:
\url{http://machadodeassis.net/download/numero04/num04artigo02.pdf}).

CARPEAUX, Otto Maria. ``Obras-primas desconhecidas do conto
brasileiro'', \emph{A Manhã}, ``Letras e Artes'', Rio de Janeiro, 10
abr. 1949; \emph{Folha da Manhã}, Quarto Caderno, São Paulo, 15 maio
1949, pp. 14-5. In: RAMOS, Graciliano. \emph{Conversas}. Organização de
Thiago Mio Salla e Ieda Lebensztayn. Rio de Janeiro: Record, 2014, p.
207-213. A conversa entre o crítico e o escritor convida os leitores a
conhecerem não só diversos contistas, como também critérios de
construção artística para avaliar os textos preferidos.

FERREIRA, Aurélio Buarque de Holanda \& RÓNAI, Paulo (organizadores e
tradutores). \emph{Mar de histórias: antologia do conto mundial}. 4 ed.
Rio de Janeiro: Nova Fronteira, {[}1978{]} 1999, 10 vols. Reunindo
narrativas antigas e mais recentes, muitas célebres, outras traduzidas
pela primeira vez para a língua portuguesa, constitui a mais completa
panorâmica do conto universal.

GOTLIB, Nádia Battella. \emph{Teoria do conto}. São Paulo: Ática, 1985.
Com base nas reflexões teóricas de Vladimir Propp, Edgar Allan Poe,
Anton Tchekhov e Julio Cortázar, e na análise de textos curtos porém
plenos de significado de grandes contistas, a professora Nádia,
livre-docente em Literatura Brasileira pela Universidade de São Paulo,
leva os leitores a conhecerem e compreenderem a especificidade do conto
como gênero literário.

POE, Edgar A. ``A filosofia da composição'' {[}\emph{The Philosophy of
Composition}, 1846{]}. In: \emph{Ficção completa, poesia \& ensaios}.
Organização, tradução e notas por Oscar Mendes, com a colaboração de
Milton Amado. Rio de Janeiro: Aguilar, 1981, pp. 911-20; POE, Edgar A.
\emph{A filosofia da composição}. Prefácio de Pedro Süssekind. Tradução
de Léa Viveiros de Castro. 2 ed. Rio de Janeiro: 7Letras, 2011. Texto
fundamental para o estudo do conto: analisando ``O corvo'', seu mais
famoso poema, Poe apresenta seu processo consciente de composição e
desenvolve uma teoria baseada no princípio da unidade de efeito.

PROPP, Vladimir. \emph{Morfologia do conto maravilhoso}. Tradução de
Jasna Paravich Sarhan. Organização e prefácio de Boris Schnaiderman. 2.
ed. Rio de Janeiro: Forense Universitária, 2006. Propp se dedica à
descrição de contos populares russos, formados por esquemas narrativos
constantes, em busca de conhecer sua estrutura e de definir o conto
maravilhoso.

RAMOS, Graciliano. \emph{Contos e novelas}. Rio de Janeiro:
Livraria-Editora da Casa do Estudante do Brasil, 1957, 3 vols.: Norte e
Nordeste; Leste; Sul e Centro-Oeste; \emph{Seleção de contos
brasileiros}. Rio de Janeiro: Edições de Ouro, 1966. 3 vols.: Norte e
Nordeste; Leste; Sul e Centro-Oeste. Antologia organizada pelo escritor
Graciliano Ramos nos anos 1940, com base em pesquisa realizada na
Academia Brasileira de Letras e na Biblioteca Nacional: segue um
critério geográfico, incluindo escritores antigos e modernos de todo o
país. O conto de Lobato escolhido por Graciliano é ``Tragédia dum capão
de pintos'', de 1923, publicado em \emph{O macaco que se fez homem}.


\end{document}

