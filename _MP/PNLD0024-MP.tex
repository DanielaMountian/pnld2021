\documentclass{article}
\usepackage{manualdoprofessor}
\usepackage{fichatecnica}
\usepackage{lipsum,media9,graficos}
\usepackage[justification=raggedright]{caption}
\usepackage{bncc}
\usepackage[circuito]{logoedlab}

\begin{document}


\newcommand{\AutorLivro}{Monteiro Lobato}
\newcommand{\TituloLivro}{Um suplício moderno e outros contos}
\newcommand{\Tema}{Ficção, mistério e fantasia}
\newcommand{\Genero}{Conto, crônica e novela}
% \newcommand{\imagemCapa}{PNLD0024-01.png}
\newcommand{\issnppub}{---}
\newcommand{\issnepub}{---}
% \newcommand{\fichacatalografica}{PNLD0024-00.png}
\newcommand{\colaborador}{\textbf{Fulano de Tal} é uma pessoa incrível e vai fazer um bom serviço.}


\title{\TituloLivro}
\author{\AutorLivro}
\def\authornotes{\colaborador}

\date{}
\maketitle
\tableofcontents

\pagebreak

\section{Carta aos professores}

Caro educador / Cara educadora,\\\bigskip

\reversemarginpar
\marginparwidth=5cm

Este manual estabelecerá pontos de contato entre a matéria literária
apresentada neste livro e você, que é quem de fato coloca em prática o
processo educativo. O presente volume é uma seleção de contos escritos
por Monteiro Lobato, entre 1900 e 1939, muitos deles publicados
primeiramente na imprensa, na \emph{Revista do Brasil}, que ele adquiriu
e passou a editar em 1918. É incrível, você verá: muitas dessas
narrativas já completaram cem anos, mas ainda são atuais, engraçadas,
comoventes!

Certamente os alunos, se ainda não leram Lobato, já ouviram falar das
personagens do \emph{Sítio do Pica-Pau Amarelo}, que ganharam vida nos
livros e tempos depois nas séries de TV. Mas sempre é tempo de conhecer,
por meio do estilo da escrita de Monteiro Lobato, as aventuras e
diálogos vividos pela boneca-gente Emília, pelo sábio espiga de milho
Visconde de Sabugosa, pelas crianças Pedrinho e Narizinho, pelas boas
senhoras Tia Nastácia e Dona Benta, pelo Marquês de Rabicó, o Burro
Falante, o Quindim.

Então, lembrando aqui a riqueza da obra infantojuvenil de Lobato, fica
reforçado o convite para agora conhecer a obra voltada para o público
adulto, a que não faltam o espírito crítico, como o da Emília, o humor,
o suspense, reviravoltas e a sabedoria do desejo de compreender os
conflitos humanos. Monteiro Lobato publicou os volumes de contos
\emph{Urupês} (1918), \emph{Cidades mortas} (1919), \emph{Negrinha}
(1920) e \emph{O macaco que se fez homem} (1923). À edição original de
\emph{Negrinha} Lobato depois acrescentou textos.

Aqui será fornecida uma série de questões, propostas de atividades e
materiais suplementares que permitirão conhecer a realidade social, os
dramas e momentos poéticos de personagens criadas por Lobato, como o
comprador de fazendas, o Jeca Tatu, o estafeta, Negrinha, o jardineiro
Timóteo, o galo Peva, o menino Pedrinho. Aqui, com os textos de Monteiro
Lobato e as atividades propostas, você terá elementos para entender
melhor a nossa realidade atual, rir ou chorar dela, por meio da arte de
contos e também de suas adaptações para filmes.

Esperamos que ache útil e divertido para o bom trabalho em sala de aula!

\section{Atividades 1}
%\BNCC{EM13LP26}

\subsection{Pré-leitura}


\subsection{Leitura}
\subsection{Pós-leitura}



\section{Atividades 2}

\subsection{Pré-leitura}
\subsection{Leitura}
\subsection{Pós-leitura}

\lipsum
\end{document}

