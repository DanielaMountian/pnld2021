\documentclass[12pt]{extarticle}
\usepackage{manualdoprofessor}
\usepackage{fichatecnica}
\usepackage{lipsum,media9,graficos}
\usepackage[justification=raggedright]{caption}
\usepackage[one]{bncc}
\usepackage[nexus]{../edlab}

   

\begin{document}


\newcommand{\AutorLivro}{Harriet Ann Jacobs}
\newcommand{\TituloLivro}{Incidentes da vida de uma escrava}
\newcommand{\Tema}{Diálogos com a sociologia e com a antropologia}
\newcommand{\Genero}{Diário, biografia, autobiografia, relatos, memórias}
\newcommand{\imagemCapa}{./images/PNLD0020-01.png}
\newcommand{\issnppub}{---}
\newcommand{\issnepub}{---}
% \newcommand{\fichacatalografica}{PNLD0020-00.png}
\newcommand{\colaborador}{\textbf{Eduardo Modesto de Carvalho, Bruno Gradella e Vicente Castro} é uma pessoa incrível e vai fazer um bom serviço.}


\title{\TituloLivro}
\author{\AutorLivro}
\def\authornotes{\colaborador}

\date{}
\maketitle

\begin{abstract}

Este Manual tem como objetivo fornecer subsídios para o trabalho com a
obra literária \emph{Incidentes da Vida de uma Escrava -- Escritos por
ela mesma}, de Harriet Jacobs.

Harriet Ann Jacobs nasceu em 1813, em Edenton, na Carolina do Norte.
Filha de escravizados, Harriet já nasce na mesma condição. Ainda criança, 
foi submetida ao seu dono legal, Dr. James Norcom, com quem sofreu diversos
abusos e de quem se libertou em 1835. Durante todo o seu período de liberdade, Harriet
Jacobs foi extremamente ativa no movimento abolicionista norte-americano e participou 
de um grupo, associado ao jornal North Star, de Frederick Douglass, que defendia a
liberdade dos negros. A primeira obra publicada por Harriet foi uma carta destinada ao jornal
The New York Tribune, no ano de 1853. Esse escrito inicial foi uma resposta ao ensaio de
Julia Tyler, que defendia com veemência a escravidão.

\emph{Incidentes da vida de uma escrava} é uma autobiografia publicada originalmente
no ano de 1861, onde a autora relata principalmente sobre os anos vividos sob o domínio
de seu então dono, James Norcom. 
Dois temas se destacam durante a narrativa: meninice e maternidade. ``Mesmo a criancinha, 
acostumada a atender sua senhora e os filhos, aprende antes dos doze anos de idade por que 
sua senhora odeia esse escravo ou aquele''. As meninas se tornam ``conhecedora{[}s{]} precoce{[}s{]} 
da maldade''. No instante em que o senhor se interessa por elas sexualmente, entendem que 
não são mais crianças, e certamente não aos olhos dele. Para as meninas escravizadas, 
a beleza é a pior das maldições. Além da violência sexual, outro problema é a incapacidade 
de poder oferecer cuidado e proteção aos próprios filhos. A maternidade vai ser uma das 
maiores batalhas da vida de Jacobs, e seria impossível ignorar o amor pelos seus filhos 
dentro da narrativa. Os filhos foram a razão para ela ter sobrevivido. As vozes dos seus 
filhos e até conseguir vê-los enquanto estava escondida eram uma fonte de vida para ela.

A meninice e a maternidade são histórias que outros autores que escreveram sobre a 
escravidão podiam testemunhar, mas nunca viver. Durante toda a narrativa, a autora
resiste ao desejo de Norcom de transformar sua feminidade e sua maternidade em arma 
contra ela mesma e, com isso, faz de sua obra um verdadeiro ato de coragem. 

Esperamos que as indicações propostas aqui sejam muito úteis no trabalho em
sala de aula! 

\end{abstract}

\tableofcontents


\section{Atividades 1}

%\BNCC{EM13LP26}

\subsection{Pré"-leitura}

\paragraph{Tema} ``E eu não sou uma mulher?''

\paragraph{Conteúdo} Introdução à discussão sobre o lugar das mulheres negras
numa sociedade escravista, bem como a particularidades do gênero autobiográfico.

\paragraph{Objetivo} Familiarizar os alunos com as premissas do gênero com o qual
irão se deparar na leitura do livro e proporcionar um primeiro contato com um texto
de uma mulher negra ex"-escravizada.

\paragraph{Justificativa} Narrativa retrospectiva em primeira pessoa sobre os anos 
que viveu na condição de escravizada, \emph{Incidentes da vida de uma escrava} 
preenche todos os requesitos de uma obra autobiográfica. Uma raridade entre os 
documentos sobre a escravidão, o livro foi escrito por uma mulher ex"-escravizada 
que não só acessou o universo da leitura e escrita, como relatou sob sua própria perspectiva 
a condição dos negros nos Estados Unidos das Américas. 
Numa sociedade onde os negros não tinham voz, a condição de mulher dificultava ainda mais
o acesso aos lugares. Neste sentido, encontramos neste livro um exemplo daquilo que alguns anos
depois estudiosos chamaram de Interseccionalidade: o estudo da sobreposição ou intersecção de 
identidades sociais e sistemas relacionados de opressão, dominação ou discriminação.

\paragraph{Metodologia}

   \begin{enumerate}
    \item 
    Para este primeiro momento, propomos que o professor leia o seguinte trecho
    de um discurso de Sojourner Truth:

    \textit{``Aquele homem lá diz que as mulheres precisam de ajuda para entrar em carruagens
    e atravessar as valas, e sempre ter os melhores lugares não importa onde. Nunca ninguém
    me ajudou a entrar em carruagens ou a pessar pelas poças, nem nunca me deram o melhor
    lugar. E eu não sou uma mulher? Olhem para mim! Olhem o meu braço! Eu arei a terra, 
    plantei e juntei toda a colheira nos celeiros; não havia um homem páreo para mim! 
    E eu não sou uma mulher?''}

    Após a leitura, o professor deve suscitar o debate acerca dos principais argumentos 
    levantados pela oradora: o que ela defende? 

    \item
    Num segundo momento da aula, após a discussão, o professor deve oferecer aos alunos algumas 
    definições sobre o texto autobiográfico. Sugerimos a seguinte definição
    proposta pelo ensaísta francês Philippe Lejeune em seu livro \emph{O pacto autobiográfico}:

    ``narrativa retrospectiva em prosa que uma pessoa real faz de sua própria existência, 
    quando focaliza sua história individual, em particular a história de sua personalidade'' 

    O professor deve atentar aos alunos de que, ainda que o discurso de Truth não seja uma 
    narrativa em prosa, mas um discurso apresentado oralmente, ele guarda algumas características
    do gênero: apresentado em primeira pessoa, parte da história individual e real de quem fala
    e de sua personalidade. Neste caso, com um claro objetivo: chamar à atenção do público
    as injustiças sofridas pelas mulheres negras nos Estados Unidos.

    \item
    Os alunos devem elaborar um resumo com as principais características do gênero autobiográfico,
    além de um ensaio sobre o tema discutido.

   \end{enumerate}

 \paragraph{Tempo estimado} Duas a quatro aulas de 50 minutos.

\Image{Harriet Ann Jacobs em 1894 (Jean Fagan Yellin; Domínio Público)}{PNLD0020-03.png}

\subsection{Leitura}

 \paragraph{Tema} O escravizado e o amor.

 \paragraph{Conteúdo} Compreensão do tema e produção no gênero discursivo 
 argumentarivo artigo de opinião.

 \paragraph{Objetivo} Proporcionar uma reflexão crítica acerca do tema do amor
 na vida dos escravizados e em nossas vidas, seguida da criação de um artigo de opinião
 sobre o assunto.

 \paragraph{Justificativa} O tema do amor aparece no presente livro mais explicitamente
 no capítulo ``O amado'', onde a autora narra sua experiência com um 
 homem negro livre que propunha comprá"-la para viverem juntos. As dificuldades 
 encontradas pelo casal para ficar juntos ultrapassam os corriqueiros problemas
 encontrados por amantes apaixonados. Percebemos, com esta narrativa, que a escravidão
 tem seus efeitos inclusive no que se considera mais íntimo dos indivíduos: a vida
 afetiva.

 \paragraph{Metodologia}
   \begin{enumerate}
    \item
    O professor deve pedir, antes da aula, que os alunos leiam o capítulo ``O amado'', 
    na página 62 do livro, onde a autora narra uma experiência amorosa durante 
    sua vida de escravizada.
    A seguir, deve se propor uma roda de conversa norteada por perguntas como:
    ``Quais os principais empecilhos encontrados por Jacobs em sua jornada afetiva?'',
    ``O que significa nesse contexto um escravizado ter a liberdade de escolher
    com quem irá se relacionar?''
   
   \item 
   Ainda na discussão, o professor deve apresentar à turma o seguinte trecho do capítulo
   ``Atitudes raciais'' do livro \emph{Nascidos na escravidão}, onde um ex"-escravizado 
   fala do quotidiano das relações afetivas entre senhores e mulheres escravizadas:

   ``O senhor não tinha filhos com mulher branca. Ele tinha
suas namoradas entre as escravas. Não sou homem de
contar mentiras. Eu conto a verdade e essa é a verdade.
Naquele tempo era difícil encontrar um senhor que não
tivesse uma mulher sua entre os escravos. Era uma coisa
geral entre os donos de escravos.
Uma das escravas na fazenda vizinha da nossa foi até
a sua senhora e contou que o senhor estava forçando ela a
deixá-lo ter relações com ela, e a senhora respondeu: ‘Ora,
pois sim, você pertence a ele’.'' (p. 190)

    A discussão deve ser estimulada a partir destes novos dados. Em contraste com a 
    impossibilidade de amor vista no capítulo de Truth, temos aqui uma realidade
    violenta da obrigação que estas mulheres tinham de satisfazer seus senhores com seus
    corpos.  

    \item
    Para finalizar, o professor deve pedir que os alunos, individualmente,
    produzam um artigo de opinião que deve, orbigatoriamente, partir
    dos textos trabalhados em aula, mas que também pode articular argumentos
    de outras áreas de conhecimento do aluno. As produções poderão ser
    compartilhadas num jornal coletivo criado pelos próprios alunos com
    o fim de divulgar os textos com toda a escola.

   \end{enumerate}

 \paragraph{Tempo estimado} Quatro aulas de 50 minutos.


\subsection{Pós"-leitura}

 \paragraph{Tema} Ouvindo e fazendo ouvir. 

 \paragraph{Conteúdo} Elaboração de \emph{podcasts} a partir de entrevistas com
 pessoas reais tendo por base temas suscitados pelas leituras feitas em classe.

 \paragraph{Objetivo} Habilitar os alunos a realizar um processo de coleta de dados
 para a formação de um perfil biográfico. Guiados pelas discussões realizadas durante as 
 últimas aulas, tanto sobre a forma quanto o conteúdo do presente livro, agora 
 eles serão convidados a tomar a posição de criadores de um material a ser divulgado
 em forma de \emph{podcast}.

 \paragraph{Justificativa} A obra traz memórias e comentários, em primeira mão, de uma
pessoa que passou parte da vida reduzida à condição de escravizada.
Dessa forma, o leitor tem acesso a um universo que, muitas vezes, não é
abordado na escola, uma vez que o relato individual é comumente
sobreposto por documentos oficiais e textos produzidos por pessoas
brancas, por exemplo. Essa abordagem acaba, por vezes, construindo
apenas uma visão superficial ou parcial das narrativas, e é por isso que
nos últimos anos tem"-se valorizado mais a utilização de diários,
memórias autobiográficas, entrevistas e depoimentos para reconstruir os
contextos de produção e circulação dos textos. 
Para aprofundamento acerca das questões
  referentes à escrita biográfica, sugere"-se a consulta a um dos mais
  completos estudos sobre o tema, publicado em língua portuguesa:
  \emph{O desafio biográfico: escrever uma vida}, de François Dosse (São
  Paulo: \textsc{edusp}, 2009). Nessa obra, fundamental para as aulas de produção
  de texto, observa"-se que o discurso biográfico permite refletir, de
  maneira aprofundada, sobre os procedimentos empregados para narrar os
  acontecimentos da vida de sujeitos com existência histórica
  comprovada. Dosse permite considerar a biografia e o perfil biográfico
  como gêneros textuais híbridos, ancorados sempre em duas dimensões
  complementares -- a histórica e a ficcional --, uma vez que a
  reprodução de eventos reais, vividos por alguém no passado, não é
  realizada apenas objetivamente, mas envolve o emprego da imaginação e
  da visão de mundo de um autor, que seleciona, filtra, organiza e
  interpreta os dados.

 \paragraph{Metodologia}
   \begin{enumerate}
    \item
    Solicite aos alunos que
entrevistem pessoas mais velhas -- familiares, vizinhos ou membros da
comunidade escolar -- e coletem informações a respeito do cotidiano e
das relações sociais e de trabalho. Sugere"-se que se peça para os
entrevistados fazerem uma comparação da fase atual com outros da vida. Sugira 
temas como o amor, trabalhado na última aula, dentro da história
de vida do entrevistado.

  \item
  A primeira versão deverá ser digitada no computador, com o auxílio de um
programa para edição de textos, e compartilhada com a turma. Em seguida,
os estudantes retomam os textos produzidos e verificam se eles atendem
aos objetivos centrais, verificando se a entrevista foi adequadamente
realizada e permitiu respostas detalhadas para a posterior construção do
perfil biográfico. Posteriormente, releia a sua produção e faça os
ajustes e as reformulações que julgar necessárias.

  \item
A série de \emph{podcasts} divulgará os perfis biográficos, no formato
de arquivos de áudio que permitirão conhecer a trajetória de diferentes
integrantes da comunidade. As gravações serão transmitidas pela
internet, no formato de uma sequência de programas digitais de rádio. O
material poderá ser facilmente acessado pelo público, a partir de
celulares, \emph{tablets} ou computadores, e poderão ser ouvidos em
qualquer lugar e momento do dia. Os áudios deverão ser gravados e
publicados em uma plataforma de streaming. Incentive os alunos a criar
um nome para a série de programas produzida pela turma. Na produção dos
\emph{podcasts}, recomende atenção às marcas de oralidade.


  \item
  Ao final, reserve um momento para apresentar a gravação em sala de aula
e avalie a qualidade do áudio, corrigindo pausas ou informações
desnecessárias, além de ruídos ou chiados. As fotografias dos indivíduos
biografados -- tanto os retratos feitos durante as conversas quanto as
fotos disponibilizadas -- formarão um mural na sala de aula ou no
corredor do colégio, e permitirão divulgar a série de \emph{podcasts}.
Durante a divulgação, os estudantes poderão organizar uma campanha
criativa pela escola e/ ou pelo bairro, assim como nas redes sociais,
com o objetivo de anunciar o lançamento da série. É interessante criar,
por exemplo, cartazes com \textsc{qr}-Codes de acesso aos \emph{podcasts} e, após
a estreia, anunciem a postagem de novos episódios. Procurem não lançar
todos os áudios de uma vez e criem uma periodicidade de publicação. Os
programas podem ser compartilhados e armazenados em páginas de redes
sociais, plataformas de streaming para distribuição digital de dados, ou
mesmo no \emph{site} ou em um blog da escola.

   \end{enumerate}

 \paragraph{Tempo estimado} Quatro aulas de 50 minutos.



\section{Atividades 2}

%\BNCC{EM13CNT201}
%\BNCC{EM13CNT303}
%\BNCC{EM13CHS101}
%\BNCC{EM13CHS102}
%\BNCC{EM13CHS106}
%\BNCC{EM13CHS401}


\subsection{Pré"-leitura}

\paragraph{Tema} A escravidão e seus impactos sociais.

\paragraph{Conteúdo} Contextualização geral sobre a escravidão nos Estados Unidos e 
no restante das Américas.

\paragraph{Objetivo} Habilitar os alunos a serem capazes de identificar
as principais características do sistema escravista nos Estados Unidos e 
no restante das Américas.

\paragraph{Justificativa} Entender todos os contextos que perpassam
a escravidão é de suma importância para a compreensão de um livro
como \emph{Incidentes da vida de uma escrava}. São as particularidades
do sistema escravista que desencadeiam as experiências, sobretudo negativas,
vividas pela autora e contadas em seu livro, que é ao mesmo tempo
um documento de denúncia à barbárie da escravidão, quanto
um registro de sobrevivência dentor do mesmo. 
O livro \emph{Nascidos na escravidão}, coletânea de entrevistas coletadas
 pelo Projeto Federal de Escritores estadunidenses com ex"-escravizados 
 na década de 1930, será uma ótima fonte de acesso a essas informações,
 já que neste livro, os alunos terão contato com relatos em primeira
 pessoa das experiências vividas na escravidão. 

\paragraph{Metodologia}

   \begin{enumerate}
    \item
    Com a presença de um professor da área de Humanidades, a aula pode se iniciar 
    com a exibição do seguinte vídeo animado
    que apresenta as movimentações do Tráfico Negreiro no Oceano Atlântico
    entre os anos de 1560 e 1860:
    \href{https://www.youtube.com/watch?v=aMyg2Cmukmo}{O Tráfico Negreiro no Atlântico}.

    \item
    Em seguida, o referido professor, deve apresentar, à lousa, alguns dados da escravidão,
    como: número total de indivíduos traficados (ou uma média deste número, devido a problemas
    na documentação, já que o tráfico continuou mesmo depois das abolições), regiões africanas
    de onde etes indivíduos eram capturados, principais impérios europeus que patrocinaram
    o tráfico, atividades realizadas nas colônias e a vida quotidiana nas colônias. Além disso, 
    é importante deixar claro aos alunos algumas diferenças entre o sistema escravista
    nas diferentes regiões das Américas: nos Estados Unidos, a situação era diferente daquela
    encontrada no Brasil ou em Cuba, por exemplo. 

    \item
    Como se trata de uma introdução ao tema, o professor não precisa esmiuçar todos os detalhes.
    A ideia é que os próprios alunos, divididos em grupos, realizem pesquisas sobre um destes
    temas em sites da Internet, livros de História, catálogos de exposições sobre o tema,
    acervos de museus virtuais ou presenciais, a depender das condições sanitárias do país. 
    Uma fonte obrigatória, no entanto, deve ser o livro \emph{Nascidos na escravidão}, dividido
    em capítulos sobre os principais aspectos da vida dos escravizados. Esta obrigatoriedade
    se deve ao fato de que este, junto à presente obra de Truth, é um dos raros
    documentos escritos onde encontramos relatos em primeira pessoa de 
    sobreviventes do sistema escravista. Conforme o tema de cada grupo, o professor deve
    indicar um ou mais capítulos do livro onde se deve focalizar a pesquisa.

    \item
    Os alunos terão uma semana para realizar as pesquisas. Os resultados deverão ser apresentados 
    em forma de seminário, além de uma parte escrita com os principais resultados.

   \end{enumerate}

 \paragraph{Tempo estimado} Quatro aulas de 50 minutos.


\subsection{Leitura}

 \paragraph{Tema} O (ex"--)escravizado e o amor.

 \paragraph{Conteúdo} Compreensão do tema do amor na vida dos escravizados a partir
 da leitura de um capítulo do livro e obras de artes visuais.

 \paragraph{Objetivo} Proporcionar uma reflexão crítica acerca do tema do amor
 na vida dos escravizados e em nossas vidas a partir da leitura de um capítulo sobre
 o tema no presente livro e a apreciação crítica de obras de arte sobre o tema.

 \paragraph{Justificativa} Retomando o tema do amor presente no capítulo
 ``O amado'' da presente obra, propomos um diálogo com o pensamento da feminista negra 
 estadunidense Bell Hooks que defende o poder revolucionário do amor. 
 Se pensarmos na condição do escravizado, é evidente que este
 sentimento não poderia fazer parte de seu quotidiano. 
 Partindo do pressuposto de que a arte é uma forma de documentar a realidade,
 mas também de sugerir novas possibilidades, um estudo sobre as representações
 do amor nas artes visuais deve ser muito produtivo na construção de um conhecimento
 acerca deste aspecto da vida de pessoas ex"-escravizadas.

 \paragraph{Metodologia}
   \begin{enumerate}
    \item
      O professor deve pedir, antes da aula, que os alunos leiam o capítulo ``O amado'', 
    na página 62 do livro, onde a autora narra uma experiência amorosa durante 
    sua vida de escravizada. Guiados pela concepção revolucionária do sentimento 
    amoroso, proponha uma discussão após a leitura. 
    ``Quais os principais empecilhos encontrados por Jacobs em sua jornada afetiva?''

    \item
    Num segundo momento, o professor de Artes deve apresentar algumas obras
    de artes visuais clássicas que retratam o amor. Sugerimos, por exemplo,
    ``Dança no campo'', de Pierre"-Auguste Renoir e "O beijo" de Gustav Klimt.
    O professor deve passar um tempo chamando atenção aos detalhes de cada 
    obra, sem esquecer da realação entre as duas pessoas nelas: em ambos
    os casos, um homem acaricia uma mulher, demonstrando afeto.

    \item
    Em seguida, o professor deve apresentar as seguintes obras: 
    ``Pequeno senhor que eu amo'' de Julien de Villeneuve, e ``Namorados'',
    de Ismael Nery --- ambas na página seguinte. Na primeira obra, o pintor francês, em 1840, representa 
    uma mulher escravizada que acaricia seu senhor, e na segunda, o pintor 
    brasileiro representa um casal de namorados negros. 

    \item
    Relacionando os conteúdos estudados com a leitura do capítulo, 
    os alunos devem produzir: um comentário crítico acerca destas imagens 
    e sua relação com a narrativa de Truth, e, então, com o auxílio do
    professor de artes, produzir uma obra de artes visuais que represente o amor. 

    \item
    Os resultados, os comentários bem como as imagens, podem ser divulgadas
    numa revista coletiva criada pelos próprios alunos para circular na escola. 
\end{enumerate}

 \paragraph{Tempo estimado} Quatro aulas de 50 minutos.

\Image{``Dança no campo'' de Pierre"-Auguste Renoir, 1833. (Pintura localizada no Museu de Orsay, Paris)}{PNLD0020-11}
\Image{``O beijo'' de Gustav Klimt, 1908. (Pintura localizada no museu do Palácio Belvedere, Viena.)}{PNLD0020-12}
\Image{``Pequeno senhor que eu amo'', de Julien de Villeneuve, 1840. (Pintura localizada no Museu de Aquitaine, Bordéus.)}{PNLD0020-13}
\Image{``Namorado'', de Ismael Nery, 1927. (Coleção particular, Rio de Janeiro)}{PNLD0020-14}


\subsection{Pós"-leitura}

 \paragraph{Tema} A escravidão nos dias de hoje.

 \paragraph{Conteúdo} Compreensão acerca do tema da escravidão pós"-abolição.

 \paragraph{Objetivo} Habilitar os alunos a compreender a persistência 
 do sistema escravista nas sociedades globalizadas mesmo após o decreto
 de seu fim.

 \paragraph{Justificativa} Oficialmente, a Mauritânia foi, em 1981, o último país do
mundo a abolir a escravidão. Entretanto, não é raro esse país aparecer
em jornais, com denúncias de fatos que revelam que escravidão persiste
por lá até os dias de hoje. Questões econômicas e culturais, bem como
negligência governamental com o problema, fazem com que esse mal
persista no país. Ainda que pareça distante, a \textbf{escravidão
moderna} é um problema existente em muitos países e, inclusive, aqui no
Brasil. E mais: ela pode ocorrer tanto em locais afastados, como em
fazendas isoladas, quanto em grandes centros urbanos, onde pessoas,
muitas vezes estrangeiros ou migrantes, são confinadas em seus ambientes
de trabalho, obrigadas a viver isoladas, recebendo pouco mais que a
alimentação e sendo impedidas de sair para tentar algo melhor.

 \paragraph{Metodologia}
   \begin{enumerate}
    \item
    Com o
auxílio de professores de Ciências Humanas, proponha uma pesquisa, em
\emph{sites} de internet, acerca da escravidão moderna. Para isso,
aconselha"-se que a turma leia o texto constitucional brasileiro, além de
reportagens e notícias, nacionais e internacionais, acerca da temática.
Sugere"-se também que sejam realizadas pesquisas em \emph{sites}
confiáveis, como o portal: \url{https://www.freedomunited.org/}, que tem
o objetivo de investigar, denunciar e combater tentativas de redução de
indivíduos à condição de escravizados.
    \item
    Com o material encontrado pelos
grupos de trabalho, organizem coletivamente um portfólio de matérias
jornalísticas sobre a escravidão no mundo contemporâneo e montem um
painel no mural da sala.
    \item
    Em seguida, proponha a escrita de um
\textbf{artigo de opinião}, produzido coletivamente pelas equipes,
combatendo a escravidão nos dias atuais e atuando propostas concretas de
intervenção sobre o problema, com base na legislação e nos textos
jornalísticos pesquisados. É importante haver conexões com situações
apresentadas na obra lida em sala de aula. Os argumentos apresentados na
primeira versão devem ser debatidos com os colegas e com os professores
envolvidos na atividade.
  \item
  Por fim, os grupos digitarão a versão final do
artigo, bem fundamentada e coerente, e publicarão no \emph{site} da
escola, em redes sociais ou no blog da turma.
   \end{enumerate}
 \paragraph{Tempo estimado} Quatro aulas de 50 minutos.


\section{Aprofundamento}

Ao chegar ao Ensino Médio, é necessário que os estudantes se aprofundem
na compreensão das múltiplas linguagens e, sobretudo, da linguagem
literária. Em relação à literatura, a \textsc{bncc} traz as seguintes
considerações:

\begin{quote}
{[}\ldots{}{]} a leitura do texto literário, que ocupa o centro do trabalho
no Ensino Fundamental, deve permanecer nuclear também no Ensino Médio.
Por força de certa simplificação didática, as biografias de autores, as
características de épocas, os resumos e outros gêneros artísticos
substitutivos, como o cinema e as \textsc{hq}s, têm relegado o texto literário a
um plano secundário do ensino. Assim, é importante não só (re)colocá"-lo
como ponto de partida para o trabalho com a literatura, como
intensificar seu convívio com os estudantes. Como linguagem
artisticamente organizada, a literatura enriquece nossa percepção e
nossa visão de mundo. Mediante arranjos especiais das palavras, ela cria
um universo que nos permite aumentar nossa capacidade de ver e sentir.
Nesse sentido, a literatura possibilita uma ampliação da nossa visão do
mundo, ajuda"-nos não só a ver mais, mas a colocar em questão muito do
que estamos vendo/vivenciando. (Brasil, 2018, p. 491)
\end{quote}

Nesta seção, desenvolvemos um trabalho de aprofundamento que, em diálogo
com a formação continuada de professores, oferece subsídios para a
abordagem do texto literário. A leitura em sala de aula \emph{Incidentes
da Vida de uma Escrava - Escritos por ela mesma} pode ser enriquecida
pelo aprofundamento no universo literário em que a obra está inserida.

\subsection{Uma obra marcante}

\emph{Incidentes da vida de uma escrava -- narrados por ela mesma} é a
autobiografia da ex"-escravizada norte"-americana Harriet Ann Jacobs,
publicada originalmente no ano de 1861. A obra representou um marco na
tradição de escrita biográfica sobre a temática da escravidão nos
Estados Unidos.


\Image{Publicação original do livro, em 1861. (Documenting the American South; Domínio Público)}{PNLD0020-09.png}


\subsection{Incidentes da vida}

Harriet Ann Jacobs nasceu em 1813, em Edenton, na Carolina do
Norte. Filha de Elijah Knox e Delilah Horniblow, Harriet já nasce na
condição de escravizada devido ao princípio do partus sequitur ventrem,
que estabelecia que filhos de escravos já nasceriam em condição de
escravidão. Ainda criança, Harriet foi submetida ao seu dono legal, Dr.
James Norcom. Esse senhor de escravos foi abusivo com a garota desde a
infância dela, durante a qual eram frequentes violações e torturas. Para
tentar escapar das mãos de Norcom, Harriet se uniu ao advogado Samuel
Sawyer, com quem teve dois filhos, Joseph e Louisa, ambos também
nascidos em condição de escravizados.


\Image{Retrato do senhor de escravos James Norcom (State Archives of North Carolina Raleigh, NC; CC0)}{PNLD0020-07.png}



Em 1835, Harriet consegue fugir definitivamente da posse de James Norcom
e, após sete anos de percalços, se estabelece na Filadélfia, seguindo
depois para a cidade de Nova Iorque. Durante todo o seu período de
liberdade, Harriet Jacobs foi extremamente ativa no movimento
abolicionista norte"-americano e participou de um grupo, associado ao
jornal North Star, de Frederick Douglass, que defendia a liberdade dos
negros.


\Image{Anúncio feito pelo senhor de Harriet, Norcom, com uma recompensa de 100 dólares para quem a capturasse. No cartaz, há uma descrição de sua aparência física e a explicação de ela havia fugido. Há também o aviso de que poderão sofrer penalidades os que a ajudarem em sua fuga. (State Archives of North Carolina Raleigh, NC; CC0)}{PNLD0020-05.png}


A primeira obra publicada por Harriet foi uma carta destinada ao jornal
\emph{The New York Tribune}, no ano de 1853. Esse escrito inicial foi
uma resposta ao ensaio de Julia Tyler, que defendia com veemência a
escravidão. Além de diversas outras cartas publicadas por Harriet, sua
principal obra foi a autobiografia intitulada Incidentes da Vida de uma
Escrava, autorizada para publicação no ano de 1861, pela editora Thayer
and Eldridge. A obra narra toda a sua vitoriosa luta contra a escravidão
e em defesa da liberdade. Harriet Ann Jacobs viveu 84 anos, vindo a
falecer na cidade de Washington no ano de 1897.


\Image{Casa onde morou Harriet Ann Jacobs, em Cambridge, Massachusetts. (Midnightdreary; CC-BY-SA-3.0)}{PNLD0020-04.png}


\Image{Foto de 1864 da escola fundada por Harriet Jacobs, que oferecia ensino gratuito a crianças negras. (Robert Langmuir African American Photograph Collection; Domínio Público)}{PNLD0020-08.png}


\subsection{Uma voz feminina}

Durante muito tempo, os relatos e documentos sobre a escravidão
estadunidense vinham a partir de falas e escritos predominantemente
masculinos, fossem da parte dos senhores ou mesmo dos escravizados.
Dessa maneira, \emph{Incidentes da vida de uma escrava} rompe com um
paradigma, trazendo"-nos a \emph{perspectiva de uma mulher
escravizada}, no Sul dos Estados Unidos do século \textsc{xix}.


\Image{Retrato de Harriet, desenhado com lápis grafite por Keith White em 1994. (State Archives of North Carolina Raleigh, NC; CC0)}{PNLD0020-06.png}


Harriet Ann Jacobs é a primeira mulher negra a escrever uma
autobiografia sobre a sua condição de escravizada e uma das primeiras a
relatar, de forma precisa, a diferença entre ser homem e ser mulher no
sistema escravista dos Estados Unidos. Ela conta como era viver diante
de um senhor de escravos que praticava diversas violências físicas e
emocionais. Revela também toda a dificuldade de ser mãe dentro de um
contexto de escravidão. O livro deu origem a um intenso debate político,
histórico e social acerca da feminidade negra nos Estados Unidos.

\subsection{A escravidão nos Estados Unidos}

É interessante estabelecer relações entre a obra e o contexto em que ela
foi produzida. Para isso, é recomendável estabelecer parcerias com
professores da área de Ciências Humanas. Entre as estratégias de leitura
indicadas na \textsc{bncc}, estão aquelas que, ``por um lado, permitam a
compreensão dos modos de produção, circulação e recepção das obras
{[}\ldots{}{]} e o desvelamento dos interesses e dos conflitos que permeiam
suas condições de produção {[}\ldots{}{]}''.

A época retratada na obra apresenta a vida no Sul do país durante a
infância, a adolescência e a juventude de Harriet Jacobs. Essa região
estava fortemente estruturada sob o sistema de plantantion, produzindo
arroz, tabaco, açúcar e algodão. Este, aliás, era o principal produto
norte"-americano no período, em razão da crescente industrialização do
setor têxtil na Europa. Desse modo, podemos perceber que a
industrialização europeia gerou consequências mundiais; no caso dos
Estados Unidos, intensificou a escravidão e a prolongou por muitos anos.

Cerca de 4 milhões de escravos viviam nas fazendas dos estados sulistas,
sendo que quase metade dessa população era feminina. O número de
escravizadas foi um fator essencial para a solidificação do regime
escravista no país por dois motivos: em primeiro lugar, o tráfico
negreiro no Atlântico Norte mingou muito mais rapidamente do que seu par
no Atlântico Sul, tornando"-se diminuto na primeira metade do século \textsc{xix};
em segundo lugar, deve"-se destacar que o recém"-nascido seguia a condição
escravizada de sua mãe. Portanto, se a mãe fosse escrava, seu filho,
consequentemente, também o seria.

Por esses motivos, o corpo das mulheres escravizadas era o grande motor
da escravatura norte"-americana. As meninas da faixa etária entre 12 e 15
anos eram as mais suscetíveis aos abusos físicos, sexuais e psicológicos
por parte dos seus senhores. Também era nessa idade que as escravas eram
tidas, por seus proprietários, como apropriadas para a reprodução,
conforme Jacobs comprova ao relatar as principais brutalidades sofridas
por ela.


\Image{"Eu era propriedade da filha deles" - Harriet Jacobs. Essa foto, de uma menina escravizada cuidando da filha do senhor de escravos, mostra como crianças eram obrigadas a cuidar de outras crianças. (State Archives of Florida, Florida Memory; )}{PNLD0020-10.png}


\subsection{Uma obra para os tempos atuais}

A obra \emph{Incidentes da vida de uma escrava} ajuda permite
compreender o terrível cotidiano escravista estadunidense e mostra como
a violência e os abusos mexiam com a mente e as emoções dos escravizados
e escravizadas. Tais sofrimentos podem ser equiparados às dores físicas
dos castigos e punições. A autobiografia possibilita observar também
como o abuso sexual de mulheres negras escravizadas era algo cotidiano.

Sobre essa questão, a autora assinala como as meninas negras
escravizadas perdiam muito cedo sua inocência. Frequentemente, ela
afirma que a beleza das meninas poderia ser algo que se voltava contra
elas: a beleza, segundo ela, era a maior maldição para uma menina negra
escravizada nos Estados Unidos do século \textsc{xix}.

A partir de Harriet Ann Jacobs, outras mulheres buscaram também relatar
suas experiências no contexto do escravismo e, graças também às suas
narrativas, é possível enveredar por um rico campo de pesquisa histórica
sobre a trajetória dos negros nos Estados Unidos.

\section{Sugestões de atividades complementares: relações dialógicas e
intertextuais}

%\BNCC{EM13LP03}
%\BNCC{EM13LP04}
%\BNCC{EM13LP49}
%\BNCC{EM13LP51}

No Ensino Médio, da mesma forma que no Ensino Fundamental, a \textsc{bncc}
organiza o trabalho com as práticas de linguagem em cinco \textbf{campos
de atuação social}. São eles: campo da vida pessoal, campo da vida
pública, campo jornalístico"-midiático, campo artístico"-literário e campo
das práticas de estudo e pesquisa.

De acordo com essa divisão, propomos na sequência um trabalho
interdiscursivo e intertextual com a obra \emph{Incidentes da vida de uma escrava.}

\subsection{Campo da vida pessoal}

\begin{quote}
O campo da vida pessoal pretende funcionar como espaço de articulações
e sínteses das aprendizagens de outros campos postas a serviço dos
projetos de vida dos estudantes. As práticas de linguagem privilegiadas
nesse campo relacionam"-se com a ampliação do saber sobre si, tendo em
vista as condições que cercam a vida contemporânea e as condições
juvenis no Brasil e no mundo.

Está em questão também possibilitar vivências significativas de práticas
colaborativas em situações de interação presenciais ou em ambientes
digitais e aprender, na articulação com outras áreas, campos e com os
projetos e escolhas pessoais dos jovens, procedimentos de levantamento,
tratamento e divulgação de dados e informações e o uso desses dados em
produções diversas e na proposição de ações e projetos de natureza
variada, para fomentar o protagonismo juvenil de forma
contextualizada. (\textsc{bncc}, p. 494)
\end{quote}

A leitura de relatos acerca das vivências de ex"-escravizados nos
Estados Unidos do século \textsc{xix} convida à reflexão sobre temas ainda hoje
atuais, mas também possibilita o trabalho no campo das competências
socioemocionais. O registro de trajetórias marcadas pelo sofrimento e
pela superação das adversidades estimula o desenvolvimento da empatia
e da solidariedade em relação a dores alheias. Por isso, na esfera da
subjetividade e dos projetos de vida, os testemunhos de vida podem
servir como ponto de partida para debates e rodas de conversa sobre
temas relevantes para o universo dos adolescentes. Em parceria com
professores de Ciências Humanas, proponha a elaboração individual de
\textbf{crônicas argumentativas} com base nos temas do respeito às
diferenças e do direito à liberdade. Verifique as visões da turma
sobre a noção de liberdade e observe que valores estão ligados, do
ponto de vista deles, à existência dessa condição. Em seguida,
relacione o estatuto do sujeito livre à necessidade de respeito e
tolerância das diferenças em todos os níveis da existência, mas,
sobretudo, no que tange a questões culturais e étnico"-raciais. Para
fundamentar a argumentação, estimule a pesquisa virtual a passagens da
Constituição que asseguram as liberdades individuais e coletivas.
Oriente, depois, o registro dos argumentos e contra"-argumentos no
caderno, para que seja possível articulá"-los em uma unidade
argumentativa coesa e coerente. Reserve um momento para a produção das
primeiras versões das crônicas, que devem ser preferencialmente
digitadas. Incentive a inserção de exemplos extraídos da atualidade, a
partir de novas consultas a \emph{sites} confiáveis de internet.
Partindo do cotidiano, é possível que a argumentação seja construída
com base em um olhar poético e sensível para a realidade vivida por
muitas pessoas em todo o mundo. O combate à discriminação deve ser
incentivado e argumentos que fujam do senso comum podem ser
construídos a partir da proposição de formas de intervenção concreta
sobre os problemas apontados. As produções podem ser compartilhadas
com os colegas e professores envolvidos na atividade, para que
comentários críticos e construtivos sejam feitos, de forma respeitosa
e democrática. Ao final, as versões definitivas podem ser publicadas
no \emph{site} do colégio, em redes sociais ou em um blog da turma,
após um processo de revisão e edição das produções textuais.

\subsection{Campo de atuação na vida pública}

\begin{quote}
No cerne do campo de atuação na vida pública estão a ampliação da
participação em diferentes instâncias da vida pública, a defesa dos
direitos, o domínio básico de textos legais e a discussão e o debate de
ideias, propostas e projetos. {[}\ldots{}{]}

Ainda no domínio das ênfases, indica"-se um conjunto de habilidades que
se relacionam com a análise, discussão, elaboração e desenvolvimento de
propostas de ação e de projetos culturais e de intervenção social.
(\textsc{bncc}, p. 494)
\end{quote}

A luta antirracista é uma das principais pautas dos movimentos negros
em todo o mundo. O combate ao preconceito étnico"-racial deve ser
estimulado no cotidiano da sala de aula, sobretudo porque diversos
discursos veiculados socialmente perpetuam o chamado \emph{racismo
estrutural}, que muitas vezes não é notado por grande parte dos
falantes. Por isso, a leitura de relatos autobiográficos, produzidos
no contexto da escravidão, pode fundamentar as discussões em sala de
aula. A partir das experiências de leitura de narrativas originárias
dos Estados Unidos, proponha a elaboração, individual ou em dupla, de
um \textbf{artigo de opinião} sobre o preconceito racial no Brasil.
Para isso, com auxílio de professores da área de Ciências Humanas, é
importante estabelecer paralelos comparativos sobre o regime
escravista nos dois países, bem como sobre as consequências
histórico"-sociais da escravidão. Atitudes segregacionistas, racismo
internalizado e estrutural, discriminação nas esferas pública e
privada, violência e desigualdade são aspectos que podem ser
explicados historicamente. Além disso, professores de Ciências da
Natureza podem apresentar argumentos científicos, sobretudo ligados ao
campo de estudo da Genética, para combater o preconceito e as teorias
eugenistas que ganharam força nos séculos \textsc{xix} e \textsc{xx}. A partir dos
debates interdisciplinares, os estudantes poderão articular os
argumentos sob a forma de artigos que discutam, de forma coesa e
coerente, as causas históricas, os impactos sociais e formas concretas
de atuação social e intervenção quanto ao problema do racismo na
sociedade brasileira. É importante incluir elementos trazidos das
leituras, mas também é recomendável estimular a consulta a livros e
\emph{sites} confiáveis sobre a luta antirracista. As primeiras
versões dos textos podem ser compartilhadas, em sala de aula, com os
colegas e professores envolvidos. Na sequência, as produções podem ser
revisadas, digitadas e editadas com auxílio do computador. Ao final,
os textos podem ser publicados no \emph{site} da escola, em redes
sociais ou no blog da turma destinado ao registro das experiências de
leitura.

\subsection{Campo jornalístico"-midiático}

\begin{quote}
Em relação ao campo jornalístico"-midiático, espera"-se que os jovens
que chegam ao Ensino Médio sejam capazes de: compreender os fatos e
circunstâncias principais relatados; perceber a impossibilidade de
neutralidade absoluta no relato de fatos; adotar procedimentos básicos
de checagem de veracidade de informação; identificar diferentes pontos
de vista diante de questões polêmicas de relevância social; avaliar
argumentos utilizados e posicionar"-se em relação a eles de forma ética;
identificar e denunciar discursos de ódio e que envolvam desrespeito aos
Direitos Humanos; e produzir textos jornalísticos variados, tendo em
vista seus contextos de produção e características dos gêneros. Eles
também devem ter condições de analisar estratégias
linguístico"-discursivas utilizadas pelos textos publicitários e de
refletir sobre necessidades e condições de consumo.

No Ensino Médio, os jovens precisam aprofundar a análise dos interesses
que movem o campo jornalístico midiático, da relação entre informação e
opinião, com destaque para o fenômeno da pós"-verdade, consolidar o
desenvolvimento de habilidades, apropriar"-se de mais procedimentos
envolvidos na curadoria de informações, ampliar o contato com projetos
editoriais independentes e tomar consciência de que uma mídia
independente e plural é condição indispensável para a democracia.

Como já destacado, as práticas que têm lugar nas redes sociais têm
tratamento ampliado. (\textsc{bncc}, p. 494-495)
\end{quote}

A derrubada de estátuas de figuras históricas associadas, sobretudo,
ao racismo, foi uma expressão de protesto multiplicada em muitos
lugares do mundo, no ano de 2020. O movimento foi disparado pelo
assassinato de George Floyd, cidadão negro asfixiado por um oficial
branco, durante uma abordagem policial na cidade de Minnesota, nos
Estados Unidos. A partir desse fato, proponha a produção individual de
um \textbf{ensaio} escrito, por meio do qual os estudantes analisem
criticamente um aspecto da polêmica intervenção sobre estátuas e
monumentos públicos. Para isso, estimule a reflexão sobre aspectos
ligados à questão, que podem servir de base para a formulação da tese
e da argumentação desenvolvida no texto.

Verifique se, na opinião dos alunos, a derrubada das estátuas compõe um
gesto: (a) agressivo, irresponsável e de violência gratuita contra o
patrimônio histórico; ou (b) necessário para a conscientização acerca
das injustiças sofridas pelas populações negras ao longo da História;
(c) de vandalismo infundado, disfarçado em postura politicamente correta
ou (d) aceitável para expressar a revolta e tornar público o racismo
presente na sociedade; (e) de destruição da memória de uma nação e de
iconoclastia de objetos de arte ou (f) de reparação junto às vítimas
silenciadas pelo opróbrio da escravidão; (g) de repúdio à suposta
intocabilidade de figuras históricas, cúmplices e responsáveis pela
escravidão ou (h) de grito de dor, vindo como resposta à segregação
racial, há tanto tempo legitimada e hoje implícita e explícita nas
relações sociais, marcadas pelo racismo estrutural.

Esses tópicos tratam do tema sob duas perspectivas contrárias entre si.
Ao escolher um item para orientar a sua produção, o estudante tomará
partido de um lado ou de outro, mas não deverá desconsiderar os demais
pontos de vista: é preciso também leva"-los em conta ao formular os
argumentos e contra"-argumentos.

Para produzir um ensaio, é preciso mobilizar o repertório sociocultural
e fundamentar os argumentos com base em livros, filmes, letras de
música, obras de arte e textos históricos ou filosóficos, que podem ser
mencionados ao longo do texto. É interessante estimular a consulta a
\emph{sites} de periódicos confiáveis, ligados à divulgação de
atualidades.

O ensaísta é um autor que analisa diferentes aspectos das ideias,
observando cada um dos elementos como se fossem as faces de um prisma.
Lembre"-se, no entanto, de que é preciso defender uma tese central; ao
fazer um ensaio sobre o tema selecionado, é possível adotar diferentes
posturas para construir a argumentação: o aluno poderá acolher uma das
perspectivas, rejeitando a outra, em uma atitude com inclinação ao
formato \emph{ou\ldots{} ou}; também é possível escolher um lugar enunciativo
com inclinação ao formato \emph{e\ldots{} e}, apresentando, para tanto,
soluções alternativas.

Observe, em sala de aula, que as opiniões possíveis a respeito do tema
refletem as vozes às quais cada aluno responderá: elas são, de um lado,
aquelas vozes que comentaram, analisaram e discutiram esse tema na mídia
contemporânea; além disso, de outro lado, há também as vozes dos atores
sociais que, na vivência pretérita, presente e futura dos fatos ligados
à escravidão negra, constituem a História como arena de conflitos
operada discursivamente. Ao comentar, analisar e discutir o tema no
formato de um ensaio, o texto responderá por adesão a determinadas vozes
sociais e não a outras. Em parceria com professores de Ciências Humanas,
reflita profundamente com os alunos sobre o ponto de vista a ser
defendido, uma vez que ser responsivo no ato de criar um ensaio é ser
responsável socialmente.

Os alunos podem elaborar, no caderno, um projeto de texto para organizar
os principais argumentos sobre o tema. É interessante que elaborem mapas
de ideias, façam um \emph{brainstorming} a partir das múltiplas faces da
questão, listem tópicos que auxiliem a posterior elaboração dos
parágrafos e da orientação argumentativa coerente do texto. É importante
desenvolver uma reflexão sobre o aspecto selecionado, verificando se se
as intervenções sobre monumentos do patrimônio público constituem um
gesto agressivo, inconsequente e vão, ou se podem constituir uma
participação necessária, legítima e oportuna na História da humanidade.

O ensaio permite, ao longo do texto, marcas de 1ª pessoa que explicitem
a subjetividade do autor, por meio de expressões como ``Eu creio
que\ldots{}'', ``no meu ponto de vista, \ldots{}'', ``eu penso que\ldots{}'' etc. É
possível até mesmo estabelecer paralelos com experiências pessoais e
fazer breves relatos de situações pessoais que tenham ligação com o
tema. Ao mesmo tempo, esse gênero de texto acolhe modalizadores de
dúvida e incerteza, como ``talvez'', ``é provável que\ldots{}'', ``pode ser
que\ldots{}'', entre outros que captam o processo de construção do pensamento
autoral. Esse gênero textual não visa, portanto, como ocorre em outras
modalidades de argumentação, à construção de um efeito de objetividade,
uma vez que o ensaísta conduz o leitor por um passeio pelo universo das
ideias pessoais e, ao expor um ponto de vista, estimula a reflexão
independente e a formação de opiniões do público. O leitor, portanto,
deve desempenhar um papel ativo e acompanhar o raciocínio do autor,
investigando, completando a análise por meio da autorreflexão e
formulando conclusões consistentes.

A primeira versão escrita do ensaio pode ser compartilhada com os
colegas e com os professores envolvidos na atividade. É recomendável que
os alunos leiam as produções em voz alta e anotem os comentários, as
sugestões e as críticas construtivas. Os posicionamentos podem ser
rediscutidos, com respeito e liberdade de expressão, mas sempre
respeitando os direitos humanos. Após a apresentação, cada aluno fará as
alterações e reformulações necessárias. A segunda versão do texto será
digitada, utilizando o computador , de acordo com a norma padrão da
língua portuguesa. As versões finais podem ser publicadas nas redes
sociais, no \emph{site} da escola ou em um blog destinado à divulgação
dos trabalhos.

\subsection{Campo artístico"-literário}

\begin{quote}
No campo artístico"-literário busca"-se a ampliação do contato e a
análise mais fundamentada de manifestações culturais e artísticas em
geral. Está em jogo a continuidade da formação do leitor literário e do
desenvolvimento da fruição. A análise contextualizada de produções
artísticas e dos textos literários, com destaque para os clássicos,
intensifica"-se no Ensino Médio. Gêneros e formas diversas de produções
vinculadas à apreciação de obras artísticas e produções culturais
(resenhas, vlogs e podcasts literários, culturais etc.) ou a formas de
apropriação do texto literário, de produções cinematográficas e teatrais
e de outras manifestações artísticas (remidiações, paródias,
estilizações, videominutos, fanfics etc.) continuam a ser considerados
associados a habilidades técnicas e estéticas mais refinadas.

A escrita literária, por sua vez, ainda que não seja o foco central do
componente de Língua Portuguesa, também se mostra rica em possibilidades
expressivas. (\textsc{bncc}, p. 495-496).
\end{quote}

O texto literário oferece possibilidades de reflexão e conhecimento de
situações da vida que podem se aproximar das experiências dos
estudantes ou colocá"-los em contato com realidades vividas por outras
pessoas. Com base na leitura de relatos autobiográficos de
ex"-escravizados norte"-americanos, os alunos podem conhecer
experiências de outros lugares e épocas, que permanecem atuais e
dialogam diretamente com o Brasil da contemporaneidade. Apesar de a
matéria básica das produções ser extraída da vida real, a elaboração
por meio da palavra confere uma dimensão literária às narrativas. Além
de documentos de época, revelam"-se olhares de sujeitos sensíveis,
marcados por situações históricas em contextos de crise. Para
aprofundar o trabalho com conflitos ligados à experiência juvenil,
proponha um mergulho poético nas noções de liberdade e tolerância:
desta vez, os estudantes produzirão \textbf{poemas líricos} para
expressar ideias e sentimentos ligados a esses temas. Como ponto de
partida, é possível retomar produções e discussões feitas ao longo da
leitura, bem como letras de música, pertencentes a gêneros variados,
que tratem igualmente desses valores. Utilizando uma \emph{playlist}
construída coletivamente pelos alunos, privilegiando o repertório do
\emph{hip hop}, reserve um momento para uma oficina de criação
poética. Na sequência, proponha a leitura compartilhada das produções,
comentando"-as e incentivando eventuais reformulações. Por fim, os
textos podem ser digitados, de maneira a compor uma antologia poética
acerca da liberdade e da tolerância, que pode ser publicada no
\emph{site} da escola ou no blog da turma, destinado às atividades
feitas com base nas experiências de leitura.

\subsection{Campo das práticas de estudo e pesquisa}

\begin{quote}
O campo das práticas de estudo e pesquisa mantém destaque para os
gêneros e habilidades envolvidos na leitura/escuta e produção de textos
de diferentes áreas do conhecimento e para as habilidades e
procedimentos envolvidos no estudo. Ganham realce também as habilidades
relacionadas à análise, síntese, reflexão, problematização e pesquisa:
estabelecimento de recorte da questão ou problema; seleção de
informações; estabelecimento das condições de coleta de dados para a
realização de levantamentos; realização de pesquisas de diferentes
tipos; tratamento dos dados e informações; e formas de uso e
socialização dos resultados e análises.

Além de fazer uso competente da língua e das outras semioses, os
estudantes devem ter uma atitude investigativa e criativa em relação a
elas e compreender princípios e procedimentos metodológicos que orientam
a produção do conhecimento sobre a língua e as linguagens e a formulação
de regras. (\textsc{bncc}, p. 495-496)
\end{quote}

As consequências da escravidão, em diferentes países do mundo,
geraram, ao longo do tempo, experiências semelhantes de preconceito e
discriminação. A leitura de relatos de ex"-escravizados estadunidenses
enriquecem as discussões sobre o tema e permitem traçar paralelos com
realidades diversas, incluindo a da História nacional. No Brasil
contemporâneo, as desigualdades sociais se perpetuam também em
decorrência do passado escravista. Em parceria com professores de
Ciências Humanas, retome as discussões sobre as raízes históricas da
escravidão, com ênfase no Brasil, e proponha uma pesquisa sobre o
tráfico negreiro, desta vez concentrada no contexto brasileiro. Por
meio da pesquisa em livros e \emph{sites} de internet, a turma,
dividida em pequenos grupos, buscará informações sobre números de
pessoas trazidas do continente africano para o Brasil, as
características do sistema social, político e econômico que
sustentavam a escravidão, mapas de rotas do tráfico de escravizados,
relatos acerca do cotidiano nos navios negreiros, as leis que
procuraram modificar esse cenário e poemas produzidos sobre essas
experiências, sobretudo pela vertente socialmente engajada do
Romantismo (conhecida como \emph{condoreira} e praticada, no Brasil,
sobretudo por Castro Alves). A partir dos dados coletados, oriente a
produção de uma \textbf{reportagem} escrita sobre o tráfico negreiro
no Brasil, em linguagem de divulgação, com o objetivo de promover
reflexões sobre questões atuais do país, tanto nas relações pessoais
quanto no mundo do trabalho. As versões finais dessas reportagens,
digitadas e editadas pelos grupos com auxílio do computador, podem ser
compartilhadas no \emph{site} da escola ou no blog da turma, destinado
à publicação de produções ligadas às experiências de leitura.

\section{Referências complementares}

\subsection{Livros}
\begin{itemize}
\item\textsc{adichie}, Chimamanda Ngozi. \textit{Americanah}. São Paulo: Companhia
  das Letras, 2014.

Em busca de alternativas às universidades nigerianas, a jovem Ifemelu
emigra para os Estados Unidos. Enquanto se destaca no meio acadêmico,
ela depara com a questão racial e com as dificuldades da vida de mulher
negra e estrangeira.

\item\textsc{angelou}, Maya. \textit{Eu sei por que o pássaro canta na gaiola}.
  Bauru: Astral Cultural, 2018.

Neste romance emocionante, a autora conta a história de Marguerite Ann
Johnson, garota negra criada no sul dos \textsc{eua} pela avó, e dá voz a jovens
que, como ela, enfrentam muitos preconceitos.

\item\textsc{evaristo}, Conceição. \textit{Olhos d'água}. Rio de Janeiro: Pallas,
  2014.

Com uma escrita lírica e sensível, a escritora mineira reúne breves
contos sobre o cotidiano, concentrando o foco de interesse sobre a vida
da população afro"-brasileira.

\item\textsc{lee}, Harper. \textit{O sol é para todos}. São Paulo: José Olympio,
  2006.

No sul dos \textsc{eua} da década de 1930, uma garotinha esperta e observadora
relata a saga do pai, um advogado que arrisca tudo para defender um
homem negro, injustamente acusado de cometer um crime.

\item\textsc{northup}, Solomon. \textit{Doze anos de escravidão}. São Paulo:
  Penguin, 2014.

O livro é o relato real e assombroso de um negro livre que, atraído por
uma proposta de trabalho, abandona a segurança do Norte dos \textsc{eua} e acaba
sendo sequestrado e vendido como escravo no Sul.

\item\textsc{stockett}, Kathryn. \textit{A resposta}. Rio de Janeiro: Bertrand
  Brasil, 2012.

A obra conta a história de Eugenia, jovem que deseja ser escritora. Ela
tem um plano brilhante, mas perigoso: escrever um livro em que
empregadas domésticas negras relatem seus relacionamentos com patroas
brancas do Mississipi, nos anos 1960.

\item\textsc{stowe}, Harriet Beecher. \textit{A cabana do pai Tomás}. São Paulo:
  Carambaia, 2018.

Um dos romances mais importantes da época da Guerra Civil Americana, a
obra conta a história do escravo Tom e influenciou intensamente as lutas
contra a escravidão.

\item\textsc{walker}, Alice. \textit{A cor púrpura}. São Paulo: José Olympio, 2009.

Vencedora do Prêmio Pulitzer, a autora narra com sensibilidade a vida de
Celie, uma mulher negra no sul dos Estados Unidos, do começo do século
\textsc{xx}, que sofreu abusos do padastro e depois do marido.
\end{itemize}

\subsection{Filmes}

\begin{itemize}
\item
  \textbf{A cor púrpura}. Direção: Steven Spielberg (\textsc{eua}, 1985).

Após sofrer violências em casa, a jovem Celie enfrenta trajetórias de
dor e superação. Triste e solitária, ela escreve cartas para a irmã, até
que a chegada da amante do marido transforma seu destino.

\item
  \textbf{Besouro}. Direção: João Daniel Tikhomiroff (Brasil, 2009).

Na Bahia dos anos 20, o pequeno Manoel é apresentado à capoeira pelo
Mestre Alípio. Ao crescer, Besouro, como passa a ser chamado, recebe a
missão de defender seus semelhantes da opressão e do racismo.

\item
  \textbf{Django Livre}. Direção: Quentin Tarantino (\textsc{eua}, 2013).

O filme é construído em torno da improvável parceria entre Django, um
escravo liberto, e o Dr. Schultz, um caçador alemão de recompensas.
Juntos, irão atrás dos criminosos mais perigosos do sul dos \textsc{eua} e
tentarão resgatar a esposa de Django.

\item
  \textbf{Doze anos de escravidão}. Direção: Steve McQueen (\textsc{eua}, 2013).

Vencedor de três estatuetas no Oscar de 2014, o filme conta a história
real de Solomon Northup, negro livre que foi escravizado por 12 anos no
sul dos \textsc{eua}, ao cair na armadilha de uma oferta de emprego.

\item
  \textbf{Histórias cruzadas}. Direção: Tate Taylor (\textsc{eua}, Índia,
  Emirados Árabes Unidos, 2012).

Adaptação do livro \emph{A Resposta}, este longa conta a história de
Skeeter, jovem que quer ser escritora. Ela começa a entrevistar mulheres
negras que deixaram suas vidas para trabalhar na criação dos filhos da
elite branca. Sua iniciativa irá desagradar muita gente.

\item
  \textbf{O sol é para todos}. Direção: Robert Mulligan (\textsc{eua}, 1963).

Baseado no romance homônimo, o filme conta a história de Tom Robinson,
jovem negro injustamente acusado de violentar uma mulher branca, e do
advogado Atticus Finch, que enfrentou a rejeição da cidade para defender
o réu.
\end{itemize}

\subsection{Lugar para visitar}

\begin{itemize}
\item\textbf{Museu Afro Brasil}\\
(\href{http://www.museuafrobrasil.org.br/}{museuafrobrasil.org.br}).

O museu, localizado na Vila Mariana, zona sul de São Paulo, reúne um
acervo contendo mais de 5 mil obras relacionadas à cultura africana e
afro"-brasileira.
\end{itemize}

\section{Bibliografia comentada}

\begin{itemize}
\item\textsc{adichie}, Chimamanda Ngozi. \textit{O perigo de uma história única}.
  São Paulo: Companhia das Letras, 2019.

A escritora nigeriana defende que nosso conhecimento é construído pelas
histórias que escutamos e que, quanto mais diversas e numerosas forem
essas narrativas, mais completa será nossa compreensão sobre o mundo.

\item\textsc{alexander}, Michelle. \textit{A nova segregação}. São Paulo: Boitempo,
  2018.

Esta obra desafiou a opinião de que o governo Obama assinalava o advento
de uma nova era pós"-racial. A autora analisa o sistema prisional dos \textsc{eua}
e expõe como o racismo estrutural opera nas sociedades ocidentais.

\item\textsc{horne}, Gerald. \textit{O sul mais distante}. São Paulo: Companhia das
  Letras, 2010.

O autor vê a escravidão em termos hemisféricos e defende que o sul
escravista dos \textsc{eua} via, em uma aliança com o Brasil (o ``sul mais
distante''), uma forma de proteção contra um futuro embate com o norte
estadunidense, na Guerra de Secessão.

\item\textsc{marquese}, Rafael; \textsc{salles}; Ricardo (org.). \textit{Escravidão e
  capitalismo histórico no século \textsc{xix}: Cuba, Brasil, Estados Unidos}.
  Rio de Janeiro: Civilização Brasileira, 2016.

O livro reúne ensaios de historiadores brasileiros e estrangeiros sobre
a escravização de negros nas Américas, ao longo do século \textsc{xix}.

\item\textsc{mccullers}, Carson. \textit{A convidada do casamento}. São Paulo: Novo
  Século, 2008.

Frankie é uma menina cujos sonhos se chocam com sua rotina pacata,
compartilhada com seu primo e com Berenice, a cozinheira negra da casa.
Em um verão solitário nos \textsc{eua}, suas incertezas aumentam quando recebe a
notícia do casamento do irmão.

\item\textsc{morgan}, Edmund S. ``Escravidão e liberdade: o paradoxo americano''.
  \emph{In} \textit{Estudos Avançados} 14 (38), p. 21-50. São Paulo:
  Universidade de São Paulo, 2000. São Paulo, 2000. (Disponível em:
  \href{http://www.revistas.usp.br/eav/article/view/9507}{revistas.usp.br}.
  Acesso em: 8 de fev. de 2021.)

O autor busca compreender como o povo estadunidense pôde, desde o
princípio, desenvolver uma dedicação às ideias de liberdade e dignidade
humanas, e simultaneamente apoiar um sistema de trabalho que negava
diariamente esses valores.

\item\textsc{porter}, Regina. \textit{Os viajantes}. São Paulo: Companhia das
  Letras, 2020.

Por meio das múltiplas perspectivas de personagens, a obra apresenta uma
trama que avança e volta no tempo. O livro traça um panorama da vida nos
Estados Unidos entre a década de 1950 e a eleição de Barack Obama para
presidente.
\end{itemize}


\end{document}

