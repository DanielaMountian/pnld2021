\documentclass[11pt]{extarticle}
\usepackage{manualdoprofessor}
\usepackage{fichatecnica}
\usepackage{lipsum,media9,graficos}
\usepackage[justification=raggedright]{caption}
\usepackage[one]{bncc}
\usepackage[ubu]{../edlab}



\begin{document}


\newcommand{\AutorLivro}{Josely V.~Baptista (org.)}
\newcommand{\TituloLivro}{Popol Vuh}
\newcommand{\Tema}{Ficção, mistério e fantasia}
\newcommand{\Genero}{Mitologia indígena}
\newcommand{\imagemCapa}{./images/PNLD0059-01.png}
\newcommand{\issnppub}{978-65-86497-32-8}
\newcommand{\issnepub}{978-85-71260-23-8}
% \newcommand{\fichacatalografica}{PNLD0059-00.png}
\newcommand{\colaborador}{Vicente Castro e Bruno Gradella}
% Sofia Boldrini (edição)}


%\baselineskip=1.20\baselineskip\par


\title{\TituloLivro}
\author{\AutorLivro}
\def\authornotes{\colaborador}

\date{}
\maketitle

\begin{abstract}\addcontentsline{toc}{section}{Carta ao professor}
Este Manual tem como objetivo fornecer subsídios para o trabalho com a
obra literária \emph{Popol Vuh}, obra de autoria anônima, compilada pelo
frei dominicano Francisco Ximénez.

O \emph{Popol Vuh} é considerado o mais importante documento poético-político
da antiguidade das Américas. Poema épico escrito da tradição mitológica maia, 
encontramos nele a narrativa maia da formação do universo e do surgimento do 
ser humano, permeada por outras aventuras. Tendo sido produzido após a conquista 
espanhola, supõe"-se que o \emph{Popol Vuh} tenha sido escrito por um mestre da 
palavra maia"-quiché na tentativa de preservá"-lo da violência dos colonizadores. 
Esta escrita teria sido feita no próprio idioma maio, mas com grafia latina, 
por volta de 1550. Devido à perda do manuscrito o que nos restou foi a cópia e 
tradução ao castelhano feita pelo frei dominicano Francisco Ximénez, já no século \textsc{xviii}.

Sobretudo em razão da violência com que se deu a cultura espanhola, muito da 
cultura maia se perdeu. Dessa forma, o \emph{Popol Vuh} é um dos poucos 
fragmentos que nos restam desse povo, oferecendo ao leitor moderno a possibilidade de perceber 
um mundo distinto e muito rico, com os indicativos de suas divindades, sua relação 
com o cosmos, da atuação de heróis, do lugar que cada animal ocupava na criação, 
da importância dos elementos. 

Num mundo onde temos contato com grandes poemas épicos fundadores como a \emph{Ilíada} 
dos gregos, a \emph{Eneida} dos latinos, ou a \emph{Epopeia de Gilgamesh} dos sumérios, 
não há nada mais justo do que termos a oportunidade de apresentar aos estudantes do ensino 
médio o poema fundador da grande civilização maia"-quiché. 

Esperamos que as indicações propostas aqui sejam muito úteis no trabalho em
sala de aula!



\end{abstract}

\pagebreak \tableofcontents

\section{Introdução}

%# abertura resumo da obra

Os povos mesoamericanos ficaram famosos por sua riqueza cultural e construções monumentais. 
Pirâmides e templos chamam a atenção até hoje mas esses são apenas alguns elementos das culturas desses povos.
As narrativas de criação do mundo foram imortalizadas em suportes como os códices ou ainda em uma variedade de objetos de cerâmica e placas de pedra.
Eles são fundamentais para compreender a vida dos povos mesoamericanos no passado e no presente.

Os povos indígenas da atual Guatemala descendem do tronco comum dos maias, que desenvolveram sua maravilhosa civilização na parte norte do país e no atual território da Península de Iucatã.
As características físicas da população e a semelhança que existe entre suas línguas demonstram o parentesco entre eles.

\textit{Popol Vuh} é considerado o mais importante documento poético"-político da antiguidade das Américas e faz parte da cultura dos povos que habitavam e ainda habitam a região da Guatemala. O livro reúne mitologias que apresentam a cosmogonia, o amanhecer da natureza e da humanidade, a mitologia heroica dos primeiros guerreiros e a história e a genealogia dos Maia"-Quiché da Guatemala. Os quichés representam atualmente cerca de 11,28\% da população guatemalteca\footnote{Segundo o Instituto Nacional de Estadística Guatemala/2020.} e sua língua é o segundo idioma do país, após o espanhol.

Seu legado milenar, vivo na tradição oral desses povos, foi escrito em hieroglifos mais. O original quiché foi transcrito no começo da colonização para o alfabeto latino e um manuscrito foi copiado e traduzido pelo frei dominicano Francisco Ximénez (1666--1729). No entanto, pelo desconhecimento que se tem do sistema de escrita ameríndia antes da conquista, não se sabe se o livro quiché foi um documento de forma fixa ou se teve outras versões escritas.
Mas é provável supor que o original era um livro de pinturas que os sacerdotes interpretavam para preservar a memória ancestral.


Publicado em 1550, trata-se de um poema épico que narra o mito de criação do mundo do povo Maia e suas formas de vida.
Seus temas principais envolvem, a cosmogonia, a mitologia, as lendas de formação e criação. O livro consiste no registro escrito do mito de criação Maia, que até então era transmitido por meio de poesia oral.
O mito é uma das poucas obras da cultura maia que chegou até nós.

Com a invasão espanhola, houve a perda de muitos elementos das culturas mesoamericanas. Terminada a conquista do México pelos Espanhóis, Hernán Cortés, recebendo a notícia da existência de terras habitadas por inúmeros povos da Guatemala, decidiu enviar o mais intrépido dos seus capitães, Pedro de Alvarado, para submetê-los à coroa.
Várias nações indígenas, descendentes dos antigos maias, ocupavam o território da Guatemala no século \textsc{xvi}, entre elas, uma das mais importantes, a etnia Quiché. Quando Alvarado cruzou as fronteiras do território Quiché, os indígenas resistiram vigorosamente. Porém, após sangrentas batalhas, acabaram se rendendo aos espanhóis.
\SideImage{Frontispício do manuscrito de Ximénez em que se lê <<[aqui] começam as histórias da origem dos índios desta província da Guatemala. Traduzido da língua Quiché para o castelhano, para conveniência dos ministros do Santo Evangelho, pelo R[everend] P[adre] F[riar] Francisco Ximénez, sacerdote doutrinal do conselho real de Santo Tomás Chilá.>> (Ohio State University; Domínio Público)}{PNLD0059-21}

Como recurso de negociação, os líderes Quiché propuseram a Alvarado a paz imediata e a sua acolhida em território Quiché. O acordo foi aceito pelo capitão, que uma vez dentro da cidade, os traiu se apoderando militarmente do poder, e condenando os negociadores Quiché à morte diante de uma população apavorada.
Em seguida, o capitão espanhol arrasou as construções da cidade e seus habitantes se dispersaram.





Escrito por uma liderança Quiché que conhecia profundamente das tradições mitológicas e poéticas mesoamericanas, o texto revela uma amplitude entre história, mito e filosofia, e essa união o aproxima dos grandes livros sagrados e de obras clássicas como a \emph{Ilíada} e a \emph{Odisseia}.

\textit{Popol Vuh} une o interesse e a beleza do romance com a austeridade da história pintada com as mais vivas cores a vida e a mentalidade de um grande povo.

O gênero literário da poesia épica está ligado a narrativas amplas, geralmente subdivididas em cantos menores, cujo conteúdo gira em torno de grandes acontecimentos ou ações heroicas, com elementos do fantástico.

Vale lembrar que, embora a obra tenha sido organizada nos primeiros anos após a conquista hispânica e é considerada um registro de elementos da cultura Maia, alguns especialistas argumentam que há no texto elementos culturais europeus. 
Na época em que o \textit{Popol Vuh} foi escrito, a população indígena da Guatemala já entrara em contato com pinturas e cânticos dos missionários espanhóis, havendo uma possível influência da bíblia na descrição da criação e suas primeiras frases podem parecer uma transcrição do livro do Gênesis, por exemplo. 
\Image{Fragmento de tecido pré-colombiano pertencente à coleção Landman em posse do MASP-SP, que foi exposta ao público em 2019. Divulgação.}{PNLD0059-22}

Os povos indígenas do México e da Guatemala também conservavam suas histórias e outros escritos em pinturas feitas em pano, alguns deles se salvaram da destruição geral de seus livros e documentos.\footnote{A maior coleção de tecidos pré-colombianos do Brasil, cujas obras podem ser vistas nesse \href{https://masp.org.br/exposicoes/acervo-em-transformacao-comodato-masp-landmann-texteis-pre-colombianos}{link}, foi exposta no Museu de Arte Moderna de São Paulo-MASP, em 2019.}

Além disso, os mitos e lendas se inscrevem em outros tempos e lugares e o 
\textit{Popol Vuh} está relacionado à cultura material desses povos.

Também faz"-se necessário indicar ainda uma peculiaridade do texto: alguns dos heróis tem a característica de \textit{tricksters}, ou seja, são personagens parecidos com nosso Macunaíma, descrito por Mário de Andrade, e não têm as mesmas características geralmente associadas ao heróis épicos. Os \textit{tricksters} são personagens espertos e trapaceiros, que transgridem ordens preestabelecidas, cujos comportamentos muitas vezes são contrários às regras sociais convencionais.
Heróis com essa característica são comuns em todo o mundo ameríndio, como demonstrou o antropólogo Claude Lévi-Strauss, baseado em narrativas amazônicas.

Nesta edição, o tradutor intercalou à narrativa em prosa trechos dispostos em versos, principalmente quando há invocações, cantos, conjuros, exortações e marcas de oralidade.


\section{Proposta de atividades I}

\subsection{Pré-Leitura}

\paragraph{Tema} Cosmologia Maia"-Quiché

\paragraph{Conteúdo} Compreensão das múltiplas narrativas antigas ameríndias, por meio da pesquisa 
de um mito de criação de seus mundos e do compartilhamento da narrativa com a turma.
\BNCC{EM13LP52}

\paragraph{Objetivo} Aproximar a perspectiva ameríndia ao universo acadêmico, 
compartilhando as narrativas de maneira didática.

\paragraph{Justificativa} A narrativa da obra \emph{Popol Vuh} é o registro de mitos da
cultura Maia, provavelmente conhecido por todos os pertencentes a essa
cultura, e que possivelmente tenham existido outras versões e outros registros
escritos.

No entanto, o formato do texto que temos hoje é talvez
resultado de um trabalho oriundo de uma mistura de influências. Isso é
devido ao fato de a obra ter sido registrada da maneira como chegou até
nós já no período colonial, onde as terras e populações maias haviam
sido subjugadas pela coroa de Madri. Para alguns pesquisadores, isso
explicaria o porquê da existência de similitudes entre as narrativas
descritas no \emph{Popol Vuh} com narrativas do universo greco"-latino e
judaico"-cristão. O fato é que nunca saberemos como eram os contos Maias
em seu formato original, pois os mesmos não chegaram até os dias de
hoje. De todo modo, diante do exposto, cabe frisar que alguns estudiosos
consideram o \emph{Popol Vuh} uma obra híbrida. De todo modo, está
claro que a maior parte do texto é advinda da cultura maia, em razão de
seus simbolismos e especificidades.

Conhecer e preservar as narrativas é uma forma de
cultivar a cultura e, nesse sentido, as noções e espacialidades outras 
que a cosmologia ameríndia nos revela coloca-nos em condições de fronteiras 
entre mundos possíveis, possibilitando, por meio delas, expandir nosso 
próprio mundo e nossas próprias percepções.

\paragraph{Metodologia} 

\begin{enumerate}
\item
Antes do mergulhar na obra, como atividade de pré"-leitura, proponha um debate 
com a turma sobre o que é um mito de criação, atentando para o fato de que é um 
comportamento humano típico, da necessidade que indivíduos e coletividades têm de
explicar o universo ao seu entorno.

\item
Em seguida, sugira que cada aluno realize uma pesquisa acerca de um mito de criação ameríndia 
e reitere que os mitos dialogam com o meio em que as pessoas estão 
inseridas, de modo que sempre surgem elementos do cotidiano, por vezes 
ganhando tons do fantástico, nas narrativas mitológicas.

\item
Atente os alunos à escolha de palavras e aos símbolos aplicados 
aos mitos. Aponte também que, frequentemente, os mitos estão imbuídos de conotação 
religiosa, social e política.

\item
Como produto para essa atividade, espera"-se um organograma que reúna e 
apresente as narrativas pesquisadas, indicando as semelhanças e diferenças
entre as mesmas.
\BNCC{EM13LP28}

\end{enumerate}

\paragraph{Tempo estimado} Duas aulas de 50 minutos. 

\subsection{Leitura}

\paragraph{Tema} Coro coletivo

\paragraph{Conteúdo} Exercícios de tradução e interpretação de um trecho da obra, 
reposicionando o leitor como novo orador, aproximando"-o do discurso.

\paragraph{Objetivo} Realização de uma leitura dramatizada de um trecho de \emph{Popol Vuh} 
e por meio do resultado dessa transposição de narradores, oferecer aos estudantes 
uma nova perspectiva sobre a obra. 
\BNCC{EM13LP02}

\paragraph{Justificativa} Pensando que a obra era de conhecimento comum dos maias, é
bem provável que a história fosse contada frequentemente. Por avós
carinhosos a seus netos, por guerreiros antes de uma batalha, entre
amigos antes de atividades esportivas, nas conversas cotidianas, no
plantio, no comércio, por vezes comparando a performance de um ou outro
indivíduo à narrativa do mito.

Deste modo, não é difícil imaginar que algumas dessas narrativas
ganhassem nuances diferentes de acordo com quem a contasse.

Assim, essa atividade estimularia a pensar o mito e sua narrativa como meio conceitual 
e temporal que se destina a múltiplos receptores.

\paragraph{Metodologia} 

\begin{enumerate}
\item
Para o início dessa atividade recomenda"-se que os alunos sejam agrupados e que
seja selecionado um trecho do livro \emph{Popol Vuh}, para que cada grupo fique 
com uma estrofe. Nossa sugestão é um trecho de grande plasticidade sonora, em que 
os pássaros guardiães voam alvoroçadamente, transcrito abaixo, da página 110 do livro:

\begin{verse}
Esses dois guardiães do jardim,\\
do jardim de Hun"-Camé e Vucub"-Camé,\\
nem notaram as formigas\\
roubando o que eles deviam vigiar,\\
as formigas pululando, carregando flores,\\
cortando flores nas árvores,\\
juntando"-as com as flores\\
que já estavam sob as árvores.\\


Esses dois guardiães, naquela cantoria\\
também não perceberam\\
que suas próprias caudas, suas próprias asas,\\
estavam sendo mordiscadas.\\


E as flores choviam,\\
caíam lá, ajuntavam"-se aqui,\\
eram cortadas acolá, e assim\\
logo se encheram de flores\\
as quatro cuias, que transbordavam,\\
ao alvorecer.
\end{verse}

\item
Proponha uma leitura coletiva desse trecho, valendo"-se dos termos utilizados pelo 
tradutor, mantendo o passo da narrativa e discutam como esse trecho ecoaria a forma como 
os mitos eram originalmente transmitidos.

\item
Sugira a cada um dos grupos a adaptação das estrofes a um linguajar mais próximo dos alunos, mas
ainda assim mantendo a estrutura narrativa oferecida na obra.

\item
Como produto final, proponha a produção de uma pequena cena, como uma cerimônia coletiva, 
que pode ou não ser encenada, abordando a situação desse trecho, cruzando 
elementos da obra com personagens análogos do cotidiano dos alunos.
\BNCC{EM13LP53}

\end{enumerate}

\paragraph{Tempo estimado} Duas aulas de 50 minutos. 

\subsection{Pós-Leitura}

\paragraph{Tema} Dia ecumênico 

\paragraph{Conteúdo} Conhecimento e articulação dos recursos utilizados na obra e 
e aprofundamento nas narrativas mitológicas ameríndias.

\paragraph{Objetivo} Estimular e habilitar os estudantes a perceber a 
correlação entre os diferentes mitos e a obra \emph{Popol Vuh}.
\BNCC{EM13LGG201}

\paragraph{Justificativa} Todos os povos detêm, em sua cultura e folclore,
histórias e personagens lendárias, compondo suas identidades por meio de
narrativas. 

Refletir sobre os múltiplos modos de habitar e se relacionar com a terra e 
a floresta, a partir de perspectivas que as entendem como entidades vivas e povoada 
também por outros seres que não somente humanos, possibilita
olhar criticamente a forma de habitar ocidental, seus ecos e catástrofes.

E reposicionar"-se como narrador/orador de mitos que remontam o mundo como ele é hoje em dia, 
permitiria, ainda que como desafio, é uma forma de 
verificar a apreensão do conteúdo da obra e as reverberações dessa leitura para os estudantes. 

\paragraph{Metodologia} 

\begin{enumerate}
\item
Como atividade pós"-leitura e a fim de se promover um caráter ecumênico, é
interessante a separação de um dia para a realização desse exercício. Nesse
dia, é interessante a produção de quadros murais, que podem ser
colocados ao redor do local em que as apresentações forem se dar, de
modo a fomentar um museu aberto da mitologia ameríndia no colégio. 

\item
A partir da retomada dos mitos pesquisados na atividade de pré"-leitura proponha aos 
alunos a composição de uma versão, em forma de poema narrativo, do mito selecionado, 
articulando os recursos estudados por meio da obra \emph{Popol Vuh}.

Gerando, assim, um caminho a ser percorrido e um senso de unicidade ao se trabalhar com o
livro. 

\item
Em seguida, estimule a turma a realizarem pequenas encenações,
monólogos, leituras dramatizadas etc., dos mitos escolhidos na primeira
atividade, seguindo os mesmos métodos da atividade de leitura.

\item
Sugira à turma que o registro das versões finais componham o memorial do colégio e
dos alunos ou até mesmo integrar o site da escola como parte de seu conteúdo ou 
publicar em formato de artigo em uma rede social como \url{medium.com}.
\BNCC{EM13LP54}

\end{enumerate}

\paragraph{Tempo estimado} Duas a quatro aulas de 50 minutos. 

\section{Proposta de atividades II}

A obra \emph{Popol Vuh} possibilita trabalhos interdisciplinares e
integradores de diferentes campos do saber e áreas de conhecimento. A
seguir, propomos algumas atividades que podem ser desenvolvidas
conjuntamente com professores de outras áreas. Além das habilidades de
Linguagens e suas Tecnologias e de Língua Portuguesa, indicadas nas
etapas da seção anterior e válidas também para esta, listamos a seguir
as habilidades de outras áreas, presentes na abordagem interdisciplinar:


\subsection{Pré-Leitura}

\paragraph{Tema} Georreferenciamento Maia

\paragraph{Conteúdo} Introdução e contextualização da obra \emph{Popol Vuh}
a partir do diálogo interdisciplinar que amplie o entendimento do mundo
do povo que compôs a narrativa mitológica.

\paragraph{Objetivo} Ambientar os estudantes no universo Maia"-Quiché.
Identificar as diferenças em relação ao período atual e as mudanças no decorrer
do tempo.


\paragraph{Justificativa} Para contextualização da obra a ser lida, revela"-se de 
grande valia um conjunto de atividades interdisciplinares, que perpassem diversos campos
do conhecimento sobre a realidade americana antes da chegada dos
colonizadores europeus. 
\BNCC{EM13CHS203}


A função dessas atividades é coletar a maior
quantidade de informações sobre a vida nas Américas antes do encontro
entre Velho e Novo Mundos.
\BNCC{EM13LP30}

Dessa maneira, aproximar a lógica vinculada a cosmologia da lógica científica que classifica,
organiza e separa os seres entre humanos, vivos e outros, revela"-se de grande importância
para a entender como isso se reverbera na forma que cada um dos povos lida com a terra e os 
espaços que habitam. 

\paragraph{Metodologia} 

\begin{enumerate}
\item 
Recomenda"-se para essa atividade a utilização de métodos como sala de
aula invertida, ou então a produção de seminários breves, de modo a
tornar mais próximo o universo americano antes da conquista europeia, o
que muito facilitará a compreensão do contexto da obra.

Parta dos seguintes questionamentos com a turma:

\item
Fauna e flora locais. Como era o meio-ambiente em que essas pessoas
viviam? E como elas se valiam dele para sua sobrevivência? Como a
flora e fauna impactaram na narrativa mitológica -- nessa atividade é
possível integrar conhecimentos dos professores de Humanidades e Biologia.

\item
Questões etno"-linguísticas, observar como se deu a chegada de humanos
nas Américas, ainda no Neolítico, e como ocorreu sua dispersão? --
essa atividade sugere a integração das disciplinas de Ciências Humanas 
e Ciências da Natureza.
\item
Quais tipos de sociedade surgiram? Como elas estavam internamente
organizadas politicamente? E como era a relação interpovos? -- nessa
atividade é possível a integração de conhecimentos das disciplinas de 
humanidades e idiomas.

\item
Como dividiam o tempo? As relações de trabalho? O plantio? -- essa 
atividade sugere"-se a confluência dos conhecimentos dos professores de 
humanidades e ciências exatas.

\end{enumerate}

\paragraph{Tempo estimado} Duas a quatro aulas de 50 minutos. 

\subsection{Leitura} 

\paragraph{Tema} Mito x Realidade

\paragraph{Conteúdo} Compreensão da importância de outros"-que"-humanos na cosmologia
Maia"-Quiché e as relações entre os elementos que compõem as paisagens apresentadas na obra. 

\paragraph{Objetivo} Apresentar o pensamento mítico e a visão de mundo ao lado do pensamento científico
como formas de olhar o mundo que não se anulam, mas se complementam, quando situadas e 
lidas em seus próprios termos. 
\BNCC{EM13LP52}

\paragraph{Justificativa} O vínculo e conexão com a natureza e a terra faz parte da forma como 
os Maias"-Quiché veem a origem de seu mundo e remontam seus mitos. 

Quanto a origem de seu mundo, em \emph{Popol Vuh}, mobiliza"-se uma série de elementos que indicam essa 
relação intrínseca entre a natureza e o pensamento ameríndio (pp.27): 

\begin{verse}
Que assim se faça: que o vazio se preencha.\\
Que essa água seja removida, escoada,\\
para que a terra nasça e possa formar seu prato.\\
E que depois venha a semeadura, o amanhecer\\
do arco do céu, do leito da terra.\\
Nossa obra, porém, nossa criação,\\
não ganhará dias de sagração nem de louvor\\
sem que se dê à luz o ser humano, a forma humana.\\
\end{verse}

As três entidades que formam o Coração do Céu (Raio Huracán, Raio Pequenino e Raio Repentino) 
foram responsáveis pela origem da vida, por meio de sua palavra. 
Para que a terra surgisse bastava dizer: ``Terra!''.

Posicionar esse pensamento junto do pensamento científico e empírico, faz"-se necessário 
para ampliar outras formas de ver o mundo.


\paragraph{Metodologia} 

\begin{enumerate}
\item
Para essa atividade de leitura, proponha um primeiro debate com os alunos, 
para que cruzem as informações obtidas nas atividades anteriores e passem, 
com isso a realizar um cruzamento de dados aglutinando o lido na obra
com possíveis razões para a narrativa se dar dessa ou daquela maneira.

\item
Aponte que a presença constante, no livro, de passagens na selva está vinculada ao fato de que esse era 
o meio"-ambiente em que os Maias"-Quiché estavam inseridos, por exemplo.
Para isso, recomenda"-se o acompanhamento dos professores de ciências da
natureza e de geografia, que são capazes de orientar os estudantes
quanto a pormenores de fatores geológicos, biológicos, meteorológicos,
esmiuçando as razões de tais fenômenos físico"-químicos, e seus impactos.
Como sugestão de leitura, para apoiar esse debate, proponha aos alunos o trecho das página 91 a 96 do livro.
\BNCC{EM13CNT206}

\item
Com esses dados colhidos, é interessante que os alunos, divididos em
grupos, utilizando os elementos apreendidos, produzam um texto, no qual
promoveram o traçar de um paralelo entre o que se extraiu da análise da
realidade e cultura maia"-quiché e a realidade e cultura de outro povo,
observando-se assim o impacto que o meio"-ambiente pode ter.

\end{enumerate}

\paragraph{Tempo estimado} Duas aulas de 50 minutos. 

\subsection{Pós-Leitura}

\paragraph{Tema} Em defesa dos povos ameríndios

\paragraph{Conteúdo}  Exercício e articulação de uma linguagem jornalística
comprometida com a segurança das informações e projeção dos direitos e vozes
ameríndias. 

\paragraph{Objetivo} Estimular e habilitar os estudantes, por meio da escrita, 
a refletir acerca do patrimônio cultural Maia-Quiché e indígena brasileiro.

\paragraph{Justificativa} Através de um processo de conscientização
coletiva, posicionar"-se como aliado/aliada, entendendo a importância cultural ameríndia, 
faz"-se urgente para que os povos indígenas possam ter seus direitos respeitados e vozes escutadas. 
\BNCC{EM13CHS601}

Além disso, não estando essa cosmologia milenar oculta em outro tempo"-espaço ou super distante 
das discussões atuais, é de grande valia perceber as reverberações da obra \emph{Popol Vuh}, quando 
se pensa a realidade brasileira. Ou seja, esse exercício possibilitaria refletir acerca do
patrimônio cultural dos povos indígenas no território brasileiro que, poucas vezes valorizado e compreendido, 
corre o risco de se tornar extinto para sempre, assim como a maior
parte dos textos Maias.

\paragraph{Metodologia} 

\begin{enumerate}
\item
Como atividade de pós"-leitura, proponha para cada aluno, individualmente, a produção de um artigo 
de opinião acerca da importância do respeito à diversidade de culturas e a valorização do
patrimônio cultural dos povos ameríndios. 

Nessa redação os alunos devem tomar o
livro lido como ponto de partido para desenvolver seus argumentos. 

\item
Para esse exercício, o aluno deverá ir além de um relato meramente de
constatação, indicando também os possíveis caminhos de políticas públicas --
podendo requerer auxílio dos professores das humanidades para isso --,
para a preservação desse patrimônio imaterial, que compõe nossa história
e está também arraigado na cultura macro"-brasileira, embora nem sempre
seja assim reconhecido.

\item
O produto final, se possível, pode ser impresso e circulado entre os alunos e 
professores como um jornal, ou até mesmo integrado no site da escola como parte das 
produções feitas pelos estudantes. 
\BNCC{EM13LGG103}

\end{enumerate}

\paragraph{Tempo estimado} Duas aulas de 50 minutos. 

\section{Aprofundamento}

\Image{Fragmento de tecido pré"-colombiano exposto no MASP-SP. Divulgação}{PNLD0059-24}

Nesta seção, desenvolvemos um trabalho de aprofundamento que, em diálogo
com a formação continuada de professores, oferece subsídios para a
abordagem do texto literário. A leitura em sala de aula de \emph{Popol
Vuh} pode ser enriquecida pelo aprofundamento no universo literário em
que a obra está inserida.


\subsection{A obra}

O \textit{Popol Vuh} é o registro escrito da tradição mitológica maia, no qual
encontra"-se a narração da formação do universo e do surgimento do homem,
permeada com outras venturas.

Tendo sido produzido após a conquista espanhola, supõe"-se que o Popol
Vuh tenha sido escrito pelos maias"-quiché em numa iniciativa não só para
registrar sua história, mas também para suplantar um reconhecimento e
legitimação cultural. Essa iniciativa está de acordo com os costumes
mesoamericanos, nos quais o propagar de narrativas mitológicas compunham
uma espécie de promoção política.

Importante destacar que o documento original foi produzido por um
maia"-quiché que aprendeu o idioma castelhano -- tanto que fontes revelam
que o manuscrito original fora escrito na língua maia, mas com a
utilização de caracteres latinos --, por volta do ano de 1550, sendo
esse manuscrito infelizmente perdido. O que nos restou, foi a cópia e
tradução ao castelhano feita pelo frei Francisco Ximénez, já no século
\textsc{xviii}, sendo ela, conhecida como o Manuscrito de Chichicastenango, a
versão mais antiga do \textit{Popol Vuh} que temos hoje.


\Image{Manuscrito de \textit{Popol Vuh}, compilado pelo frei dominicano Francisco Ximénez.(Francisco Ximénez; Domínio Público)}{PNLD0059-03.png}
 
 
\subsection{Por que ler \textit{Popol Vuh}?}

Dotados de famigerado conhecimento de astronomia, matemática e de
fenômenos naturais, muito da cultura maia se perdeu, principalmente em
razão da violência com que se deu a cultura espanhola. Dessa forma, o
Popol Vuh é um dos poucos fragmentos que nos restam desse povo,
oferecendo ao leitor moderno um perceber um mundo distinto e muito rico,
com os indicativos de suas divindades, sua relação com o cosmos, da
atuação de heróis, do lugar que cada animal ocupava na criação, da
importância dos elementos. Nas proximidades e distinções que nós, em
razão de nossa matriz cultural, possamos sentir em relação a cultura
expressa no texto é certo que a riqueza dos detalhes, a criatividade, a
beleza, ritmo e intensidade das passagens revela o quão único é o
potencial das criações humanas.

Trata"-se de um texto divertido, em geral leve, empolgante em suas
aventuras e emocionante no expressar das relações entre os seres dessa
cosmogonia. Ao leitor restarão prazerosos momentos no folhear das páginas
e a certeza de um arcabouço mais rico pela apresentação de toda uma
grandiosa e distinta visão de mundo.

\subsection{Um mito de criação}

Entre as suas narrativas, o \textit{Popol Vuh} traz a lenda de criação do mundo
segundo a mitologia dos maias-quiché.

Como em outras obras cosmogônicas, a história começa com a constatação
da existência unicamente de um nada, um vazio, um deserto.

E como em outros mitos de criação a contagem do tempo principia não pelo
surgimento de uma primeira coisa, mas pelo pronunciar da primeira
palavra. E, no caso, as divindades primordiais denotam que: "Tudo ainda
em suspenso, ainda silente. Tudo sereno, ainda em sossego. Tudo em
silêncio, vazio também o ventre do céu".


\Image{Exemplo de uma data do sistema de contagem de tempo dos maias pré-hispânicos conhecido como Conta Longa. (Francisco França; Domínio Público)}{PNLD0059-08.png}


\Image{Cerimônia para a comemoração de um novo período na contagem do tempo dos antigos maias. (Francisco França; Domínio Público)}{PNLD0059-06.png}


Na lenda maia, trata"-se de um mar, um oceano parado, um único elemento
primordial imperturbado.

As divindades criadoras, então, juntando pensamentos e palavras, decidem
preencher o vazio. A água deve ser realocada, para que sejam nascidos
outros elementos, para que o céu se separe da terra e para que, enfim,
haja lugar às criaturas que adorarão aos deuses.

E nisso as divindades claudicaram em sua construção. Primeiro esperaram
dos animais a adoração. Mas esses não podiam entoar seus nomes, nem
detinham a boa fala para prestar"-lhes a devida reverência, então os
deuses deles prescindiram, determinando que sua carne passaria a servir
de alimento. E então tentaram, em nova iniciativa, criar o homem.
Inicialmente com barro, mas seu corpo careceu de firme robustez. Então,
tentaram madeira, lograram sucesso, mas parecia que a mente dos homens
de madeira restara oca. Esses homens de madeira não adoravam aos deuses.
Então esses decidiram exterminá"-los.

Nessa parte da história, o texto traz a seguinte passagem:

\begin{quote}
Dizem que os macacos"-aranha que existem hoje na floresta descendem
deles. Restaram como um vestígio daqueles cujas carnes foram feitas,
pelo Criador, pelo Formador, só de madeira.
\end{quote}

É curioso que um mito do continente que inspirou Darwin a chegar às suas
conclusões sobre a teoria da evolução, já reconhecia a similitude entre
humanos e macacos.

O texto também revela que, apesar de um aparente insucesso, as
divindades primeiras agem como um artesão -- ou um inventor -- que
arrisca; que concebe a ideia, mas carece das técnicas para
desenvolvê"-la. Então segue quase como um instinto, por empirismo,
modelando, com tentativa e erro, atuando no forjar materializador do que
fora idealizado.

A história se desenrola e, por fim, é feito o homem, a partir do milho.
Mas este ainda terá que batalhar por seu lugar no mundo.


\Image{Deus do milho emergindo da carapaça de uma tartaruga. (Francisco França; Domínio Público)}{PNLD0059-10.png}



\subsection{O distante próximo: similitudes com outras narrativas}

Ao longo da saga da criação do homem e de sua busca por seu lugar no
mundo, o desenrolar da história do \textit{Popol Vuh} traz similitude com outras
produções conhecidas da cultura ocidental. Ao tomar a narrativa
greco-latina como contraponto, é como se a obra maia iniciasse
dialogando com Hesíodo. Este nos narra a formação do cosmos, o
surgimento das coisas por meio de gerações de divindades e o grande
esforço de Zeus por vencer a contenda com seres bestiais no intuito de
dar ordem ao mundo. Enquanto a produção mesoamericana conta como as
divindades perceberam o nada e decidiram preenche"-lo, criando elementos
da natureza e, por tentativa e erro, formaram o homem e, mesmo depois de
realizada sua criação, resolveram fazer"-lhe ajustes. Tal qual os deuses
do Olimpo, desejavam os deuses maias se distanciarem da humanidade. Se
por meio de pandora, Zeus outorgou os fardos da humanidade, explicitando
seus padeceres e limitações, Huracán turvou os olhos dos homens,
impedindo que eles vissem muito longe de seu entorno.


\Image{Huracán, o Coração do Céu. (Francisco França; Domínio Público)}{PNLD0059-05.png}


Na narrativa, há a tentativa de se criar homens do barro, ocorre um
dilúvio e é contada a travessia de heróis por lugares inóspitos. A
despeito dos nomes que aparentam difícil pronúncia num primeiro contato
-- para os quais o tradutor teve a delicadeza de nos oferecer uma
orientação de pronúncia --, o mito traz a nós histórias muito
familiares.

A diferença é que, nos mitos das tradições judaica e greco"-latina, o
homem surge já em um universo mais organizado. Na mitologia maia, o
homem surge ainda num mundo em formação, sendo exposto às angústias do
mesmo.

Na história do \textit{Popol Vuh}, narra"-se a migração, a busca dos primeiros
homens por seu lugar, o reunir com outras tribos e, de certo modo, uma
luta por sobrevivência, valendo-se da astúcia e força física.

Nesse sentido, é possível dizer que se conseguimos construir um diálogo
com Hesíodo no início do \textit{Popol Vuh}, no desenrolar da história, os
interlocutores do épico maia passam a ser Virgílio e Tito Lívio.

Há também uma certa semelhança com outros mitos, e também com as
fábulas, especialmente ao final da primeira parte e ao longo de toda a
segunda parte.

Animais ganham vozes e se tornam parte da história, revelando ao leitor
indicativos da moral da população quiché.

A segunda parte narra a aventura dos jovens Hunahpú e Ixbalanqué no
enfrentamento contra Vucub"-Caquix, a ave que se considerava uma
divindade, valendo"-se de alta astúcia para isso, numa repreensão aos
perigos de pouca modéstia e do desejo de se colocar próximo aos deuses,
sem na verdade sê"-lo.


\Image{Gêmeos enfrentam Vucub"-Caquix (desenho da imagem presente no Prato Blom). (Francisco França; Domínio Público)}{PNLD0059-04.png}


Essa história em particular nos remete a muitas narrativas familiares,
tanto no protagonismo de gêmeos, como Castor e Pólux, ou Rômulo e Remo,
como também em suas ações, que quebram as regras e se imbuem de sublime
malandragem para atingir os devidos desenlaces de suas aventuras, sendo
assim, perfeitos \textit{tricksters}, tal qual Loki ou Prometeu.

\subsection{Voz coletiva: um elemento dramático na narrativa épica}

Como já dito, trata"-se o \textit{Popol Vuh} de uma narrativa épica. Similar a
\textit{Gilgamesh}, à \textit{Ilíada} e à \textit{Teogonia}, e mesmo à história do Gênesis, conta a
história da formação do cosmos, sua posterior organização e o papel que
o homem detém nesse universo.

Entretanto, o texto não se organiza apenas na associação do texto
narrativo com diálogos entre as personagens. Há um terceiro elemento,
uma presença similar a uma conjunção de vozes que compõem a história,
enriquecendo"-a e contextualizando"-a, sendo peça chave no desenvolvimento
da história.

Nesse sentido, essa conjunção de vozes é muito similar ao coro do teatro
grego.

Como na dramaturgia helênica, esse coro mesoamericano simboliza uma
personagem coletiva, que adentra a trama compondo a história,
intermediando os elementos da narrativa, denotando sentimentos,
pensamentos ou indicando ações por vir.

Frequentemente, apresenta"-se como a voz dos deuses, um verbalizar
divino, destacado, indicando a vontade imanente, que ao acontecimento se
cria, apesar de perene; chama a atenção dos leitores e verbaliza aquilo
que apenas uma expressão facial ou um suspirar onomatopeico seriam
capazes de dizer.

E, além de tudo isso, confere à obra uma aura etérea, digna de um mundo
das deidades formadoras, acima de tudo, torna"-se um elemento que convida
o leitor a ambientar"-se na atmosfera da narrativa.

\subsection{Um pacto de sangue}

No texto épico, chama atenção que muitas vezes aparece o termo
``sangue''. É indicado logo no início que os homens de madeira não
possuíam sangue. Hun"-Hunahpú e Vucub"-Hunahpú tiveram que cruzar um rio
de sangue para chegar a Xibalbá. A seiva da árvore era vermelha como o
sangue e também coagulava. E talvez os trechos que mais levantam as
sobrancelhas, quando Tohil pede sangue aos sacrificadores e
autossacrificadores, para que o sangue fosse bebido pelos deuses, ou
mesmo quando destacam que os sacrifícios cresceram, indicando que os
cativos chegavam sendo sacrificados.

Ainda que a constante menção, e por vezes a forma como é feita, seja um
tanto visceral para o nosso paladar literário, é preciso enxergar nisso
um fundamento da própria cosmogonia maia. Isto é, dessa forma, além de
ser algo que conecta os homens aos seus deuses e ao seu meio ambiente, é
o que promove aquilo que mais lhe dá a sua característica humana.

Talvez, para os maias, o sangue tenha a mesma conotação semântica e
simbólica que a alma possui para nós, sendo o que confere a emoção, a
vitalidade, a parte que faz o homem transcender a condição de mero
participante no universo para protagonista. Na mitologia maia, o sangue
divino conferiu esses predicados aos homens, então o sacrifício de
sangue seria a retribuição, seria devolver aos deuses tal benesse e, em
certa medida, era uma atitude contributiva que reconectava os povos à
natureza.


\Image{Cena de uma prática de autossacrifício, com personagem à direita, extraindo sangue da língua. (Francisco França; Domínio Público)}{PNLD0059-09.png}


\subsection{Atividades para o aprofundamento da pesquisa}

No Ensino Médio, da mesma forma que no Ensino Fundamental, a BNCC
organiza o trabalho com as práticas de linguagem em cinco \textbf{campos
de atuação social}. São eles: campo da vida pessoal, campo da vida
pública, campo jornalístico"-midiático, campo artístico-literário e campo
das práticas de estudo e pesquisa.

De acordo com essa divisão, propomos na sequência um trabalho
interdiscursivo e intertextual com a obra \emph{Popol Vuh}.

\subsubsection{Redação em forma de correspondência}


  Escrever é a forma como o homem perpetua a si próprio, por meio do
  registro de suas ideias. Sabe"-se que muito da cultura maia se perdeu
  em razão da destruição de seus livros, o que é uma pena para a
  humanidade. E, assim como em outras culturas, além de textos de
  contabilidades e que narravam a vida de grandes reis, havia também
  cartas, coisas escritas na mera frugalidade. Outrossim, esse material
  também é importante, pois seria capaz de revelar o dia a dia da
  cultura desse povo. Por isso é importante a valorização da escrita e
  do registro das coisas cotidianas de modo a identificar afinidades. Dito isso, 
  a atividade propõe que os alunos troquem correspondências, podendo ser cartas ou e"-mails, em que
  podem contar assuntos triviais, mas cujo objeto principal deva ser
  algum fato de repercussão nacional recente. Nessa troca de
  correspondências, os alunos verão que haverá partes em que seus textos
  serão bem similares, outros serão bem distintos. De modo que ficará
  muito claro como esse tipo de documento, apesar de ser particular,
  pode servir a um historiador futuro para o conhecimento de um
  determinado período.
\BNCC{EM13LP21}


\subsubsection{Pesquisa sobre esportes pouco conhecidos}


  Em \emph{Popol Vuh}, há a menção a um jogo de bola. Os maias chamavam
  esse jogo de \emph{pitz}, e podemos compará-lo a um futebol, ou basquete
  moderno. São duas equipes que perseguem uma bola, a qual deve ser
  atravessada em um determinado espaço para a contagem de pontos. O
  esporte sempre esteve presente na história humana. Nossa espécie é
  feita para a corrida, sendo uma grande adaptação do ser humano, a
  corrida de longa duração. Com o surgir das sociedades complexas, os
  esportes passam a ser utilizados para treinamento militar, recreação
  social e mesmo para reforçar o pertencimento a determinada classe
  social. Dito isso, essa atividade sugere que os alunos, divididos em
  grupo, realizem uma pesquisa sobre esportes pouco conhecidos no
  Brasil. O resultado dessa pesquisa pode ser divulgado por meio da
  apresentação de seminários, ou então em pequenos vídeos que podem vir
  a ser divulgados nos sites do colégio.
\BNCC{EM13LP34}

\subsubsection{Pesquisas em notícias de periódicos}

  Na leitura de \emph{Popol Vuh}, o leitor é exposto às aventuras dos
  jovens gêmeos que conseguem grandes feitos em sua jornada pela selva
  maia, em uma época mítica. Eles mesmo participam do jogo de
  \emph{pitz} acima mencionado. Abordando as duas questões,
  frequentemente é visto o impacto negativo que a fama em tenra idade
  pode ocasionar em jovens. Desde astros da música, do esporte, etc.,
  pessoas que ainda são muito jovens são expostas a pressões que mesmo
  adultos não conseguiriam lidar bem. Isso posto, e levando em conta que
  é dos jovens exigido um alto desempenho para situações cotidianas
  desde muito cedo, esta atividade sugere que seja pedido ao aluno a
  redação de um artigo de opinião acerca dessa circunstância
  problemática. Recomenda"-se que sejam feitas pesquisas em notícias de
  periódicos, artigos científicos, pareceres de médicos e psicólogos
  sobre o tema. É possível que sejam realizadas entrevistas com
  profissionais da área, para o embasamento do conteúdo do artigo.
\BNCC{EM13LP33}
\BNCC{EM13LGG704}

\subsubsection{Encenação de uma sketch}

  Uma passagem com tom de comicidade na obra \textit{Popol Vuh} é a que envolve a
  peça que os gêmeos pregaram no Sete"-Arara. Situações como essa, em que
  uma personagem anda por aí tomada de soberba e é ludibriada por outros
  atuantes da aventura são comuns. Dito isso, a atividade sugere que
  seja apresentado aos alunos, ou que esses realizem uma pesquisa sobre,
  outras situações em que a astúcia acaba vencendo antagonistas, muitas
  vezes mais fortes e poderosos. 
\BNCC{EM13LP30}

  Escolhida a passagem, os alunos deverão
  encenar uma sketch da situação. Isso pode ser feito tanto
  presencialmente, quanto à distância, cada um assumindo um papel e,
  posteriormente, por meio da edição se dar fluidez à filmagem. O
  produto final poderá ser exibido no colégio, podendo ser colocado
  também no site da escola.

\subsubsection{Pesquisa sobre diversidade cultural indígena brasileira}

  Proponha uma pesquisa para ampliação do repertório cultural
  solicitando aos alunos que façam um levantamento dos grupos indígenas
  brasileiros. Estimule a reflexão dos alunos acerca dos seguintes
  aspectos: Quais condições que tais grupos se encontram? Será que há
  iniciativas para preservar sua cultura? E de que maneira as nações
  indígenas foram encaminhadas para a situação que se encontram hoje?
  Aconselha-se que os alunos, além da parte textual, realizem também a
  produção de um material gráfico, o qual poderá ser divulgado
  fisicamente em um painel, ou virtualmente, no site da escola.
\BNCC{EM13LP18}
\BNCC{EM13LP19}

\section{Sugestões de referências complementares}\label{sugestoes}

\subsection{Filmes:}

\begin{itemize}
\item\textit{Apocalypto}. Direção: Mel Gibson (Estados Unidos, 2006).

O filme se passa na península de Iucatã, durante o período da civilização Maia, antes da colonização 
espanhola, revelando as dinâmicas entre os impérios maias e seus governantes.

\item\textit{Distancia}. Direção: Sergio Ramírez (Guatemala, 2012)

Vinte anos após o sequestro de sua filha pelo exército guatemalteco, o pai parte em busca de seu 
encontro, quando descobre que ela está vivendo apenas 240km de distância.

\item\textit{Onde nasce o sol -- Donde Nace el Sol}. Direção: Elías Jiménez (Guatemala, 2013)

O filme narra a história de Maya, uma mulher indígena que, durante sua infância, tem sua comunidade 
invadida, precisando fugir. Em contato com sua cultura ancestral e guias espirituais, a personagem 
reconecta"-se com a floresta e passa a viver na selva. Entendendo que precisa trilhar seu caminho 
sozinha, Maya toma conhecimento das dificuldades da vida e de seu povo.

\item\textit{Guatemala: Coração do mundo Maia}. Direção: Ignacio Juansolo e Luis Ara (Guatemala, 2019)

O documentário percorre as belezas culturais e geomorfológicas da Guatemala, desde Sierra de las Minas --
próximo ao Lago de Izabal -- até Esquipulas -- quase na fronteira com Honduras e El Salvador--, ressaltando 
as antigas cidades maias e demais riquezas naturais.
\end{itemize}

\subsection{\emph{Sites}:}

\begin{itemize}
\item \href{https://artsandculture.google.com/project/exploring-the-maya-world}{O Mundo dos Maias}

A exposição virtual, desenvolvida pela plataforma Google Arts \& Culture, que simula digitalmente 
a experiência de um museu, disponibilizou e organizou um acervo de artefatos, obras de arte e 
histórias da Civilização Maia, intitulada ``O Mundo dos Maias''.

\item\href{https://www.learner.org/series/invitation-to-world-literature/popol-vuh/popol-vuh-map-timeline}{Linha do Tempo de \textit{Popol Vuh}}

Ainda que em inglês, o site da organização \textit{Learner} posiciona, em uma linha do tempo, as 
passagens da história, política e região Maia até os tempos atuais, apoiando"-se em mapas que mostram o território da 
Civilização Maia"-Quiché, além de trechos de \textit{Popol Vuh}, vídeos e outras imagens interativas. 
\end{itemize}

\subsection{Museus}

\section{Bibliografia comentada}

\begin{itemize}
\item\textsc{cereja}, William Roberto. \textit{Ensino de literatura: uma proposta
dialógica para o trabalho com literatura}. São Paulo: Atual, 2019.

O livro discute as práticas de ensino da literatura no Brasil a partir
de relatos de professores e alunos, de reflexões sobre vestibular e
livros didáticos, da análise da legislação e outras fontes.

\item\textsc{coelho}, Nelly Novaes. \textit{Literatura infantil: teoria, análise,
didática}. São Paulo: Moderna, 2002.

Este livro consiste em um amplo painel das possíveis abordagens,
leituras e análises da literatura infantil e juvenil.

\item\textsc{eliade}, Mircea. \textit{Cosmos e história: o mito do eterno retorno}.
São Paulo: Mercúryo, 2004.

Este trabalho fundador da história das religiões aborda as expressões e
atividades de uma grande variedade de culturas religiosas arcaicas e
``primitivas''.

\item\textsc{lajolo}, Marisa. \textit{Literatura: ontem, hoje, amanhã}. São Paulo:
Editora Unesp, 2018.

Estas páginas leves convidam leitores de todas as idades a refletirem
sobre o que é literatura, quais são seus limites e como a ideia de
literatura mudou com o tempo.

\item\textsc{lajolo}, Marisa; \textsc{zilberman}, Regina. \textit{A formação da leitura no
Brasil}. São Paulo: Editora Unesp, 2019.

As autoras fazem um traçado do nascimento, da consolidação e das
transformações das práticas de leitura da sociedade brasileira,
possibilitando que se acompanhe o amadurecimento do leitor.

\item\textsc{lévi-strauss}, Claude. \textit{Mito e significado}. Coimbra: Edições 70,
2007.

Nestas cinco conferências, o renomado antropólogo interpreta alguns
mitos e seus significados para a compreensão da natureza humana. Analisa
a relação da mitologia com a ciência, a história, a música e outros
temas.
\end{itemize}

\end{document}


