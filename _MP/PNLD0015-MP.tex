\documentclass[12pt]{extarticle}
\usepackage{manualdoprofessor}
\usepackage{fichatecnica}
\usepackage{lipsum,media9,graficos}
\usepackage[justification=raggedright]{caption}
\usepackage[one]{bncc}
\usepackage[piseagrama]{../edlab}



\begin{document}

\newcommand{\AutorLivro}{Gabriel Soares de Sousa}
\newcommand{\TituloLivro}{Tratado descritivo do Brasil}
\newcommand{\Tema}{Diálogos com a sociologia e com a antropologia}
\newcommand{\Genero}{Diário, biografia, autobiografia, relatos, memórias}
\newcommand{\imagemCapa}{./images/PNLD0015-01.png}
\newcommand{\issnppub}{978-65-89833-01-7}
\newcommand{\issnepub}{978-65-89833-00-0}
% \newcommand{\fichacatalografica}{PNLD0015-00.png}
\newcommand{\colaborador}{Bruno Gradella, Vicente Castro} 


\title{\TituloLivro}
\author{\AutorLivro}
\def\authornotes{\colaborador}

\date{}
\maketitle

\baselineskip=1.20\baselineskip\par


\begin{abstract}\addcontentsline{toc}{section}{Carta ao professor}
Este manual tem como objetivo fornecer subsídios para o trabalho com a
obra literária \emph{Tratado descritivo do Brasil}, de Gabriel Soares
de Sousa.

\textbf{Gabriel Soares de Sousa} nasceu em Portugal, provavelmente na
década de 1540, e chegou ao Brasil, proveniente do Reino e a caminho das
Índias Orientais, em 1569. Não se sabe o que fez este viajante e
explorador desistir de seguir viagem com o restante da tripulação e
desembarcar no litoral da capitania da Bahia. Gabriel Soares colocou no papel 
as lembranças dos dezessete anos em que viveu no Brasil, produzindo dois
textos, \textit{Roteiro geral com largas informações de toda a costa do
Brasil} e \textit{Memorial e declaração das grandezas da Bahia de Todos
os Santos}, que enviou ao valido do rei, D.~Cristóvão de Moura, em 1º
de março de 1587.

Inéditos e anônimos até o século \textsc{xix}, quando foram recuperados, 
reunidos e publicados integralmente, com sua autoria restituída, pelo 
historiador brasileiro Francisco Adolfo de Varnhagen, os dois textos do 
\emph{Tratado descritivo do Brasil} têm despertado grande interesse 
dos estudiosos do início da colonização do Brasil e é considerada por muitos 
o mais importante texto quinhentista sobre o assunto. Além disso, é fonte 
indispensável a diferentes áreas do conhecimento, como botânica, geografia, 
história e antropologia, pois as minuciosas descrições apresentadas pelo autor 
fornecem preciosas informações a respeito da fauna, flora, acidentes geográficos, 
povos nativos e engenhos da costa do Brasil no século \textsc{xvi}, sobretudo da 
Bahia de Todos os Santos.

A obra \emph{Tratado descritivo do Brasil} é fruto do
esforço de restaurar ao leitor contemporâneo um códice alternativo de
fins do século \textsc{xvi} ou início do \textsc{xvii}. Preservado na Biblioteca Brasiliana Guita e José Mindlin, o manuscrito contém a cópia dos dois textos de Gabriel
Soares de Sousa que vêm precedidos pelo traslado da carta enviada a
Cristóvão de Moura e pela breve apresentação intitulada ``Descrição do
que se contém neste caderno''. O códice traz como referência na lombada
a inscrição “\textit{Descripción y noticias del Brasil. Sin autor. Anno 1587}”,
o que sugere tratar"-se de uma cópia espanhola.

Esperamos que as indicações propostas aqui sejam muito úteis no trabalho em
sala de aula!

\end{abstract}

\tableofcontents

\Image{Selos comemorativos de 1987 com imagem de Gabriel Soares de Sousa (Wikipedia Commons; Domínio Público )}{PNLD0015-03.png}

\section{Propostas de Atividades I}

\subsection{Pré"-leitura}

\paragraph{Tema} As palavras e o tempo. 

\paragraph{Conteúdo} Pesquisa de vocábulos da língua portuguesa em 
dicionários antigos da Biblioteca Guita e José Mindlin. 

\paragraph{Objetivo} Instigar nos alunos a curiosidade acerca da mudança
de significado das palavras no decorrer do tempo. 
\BNCC{EM13LP01}

\paragraph{Justificativa} Junto com Fernão Cardim, Pero de Magalhães de 
Gandavo, e a carta do Pero Vaz, este é um dos principais textos do século 
\textsc{xvi}. A presente edição dos textos que compõem o \emph{Tratado} 
é especial por oferecer a amostra de um documento legítimo deste período
preservado na \href{https://digital.bbm.usp.br/view/?45000009023\#page/1/mode/2up}{Biblioteca Brasiliana Guita e José Mindlin}. O manuscrito 
de que se serviram os editores contém a cópia dos dois textos de Gabriel 
Soares, que vêm precedidos pelo traslado da carta enviada a Cristóvão de 
Moura e pela breve apresentação intitulada ``Descrição do que se contém 
neste caderno''. 
A leitura de um texto deve ser sempre acompanhada de um dicionário,
sobretudo quando se trata de um texto antigo. No caso de um texto do 
século \textsc{xvi}, o mais adequado é que se faça uso de um dicionário 
da língua em questão de um momento histórico próximo para se chegar mais 
próximo do que o autor buscava expressar. Para isso indicamos o dicionário 
mais antigo da língua portuguesa, aquele organizado por Raphael Bluteau e 
disponível para acesso livre na Biblioteca Guita e José Mindlin.

\paragraph{Metodologia}

\begin{enumerate}

\item 
A turma deve ser informada de que nas próximas aulas irão trabalhar
com a leitura de um livro do século \textsc{xvi}. Para começar a se
ambientar no universo vocabular deste período, o professor deve indicar
palavras"-chave do texto para que os alunos realizem, em grupos, uma 
pesquisa no dicionário ``Vocabulario portuguez e latino'', do século 
\textsc{xviii}, disponível para consulta online na Biblioteca Guita e 
José Mindlin. Sugerimos os seguintes tópicos: gentio, monstro, índio, 
escravo.
\BNCC{EM13LP01}
O link para acesso ao dicionário online pode ser encontrado \href{http://dicionarios.bbm.usp.br/pt-br/dicionario/edicao/1}{aqui}.

\item
Uma vez realizada a pesquisa, proponha que os alunos leiam o seguinte 
trecho dessa entrevista realizada com Daniel Munduruku:

\begin{quote} Uma palavra muda tudo? Sim, uma palavra muda muito. Nos 
meus vídeos e palestras, eu tenho sempre feito uma separação fundamental 
entre ``índio'' e ``indígena''. As pessoas ainda pensam que índio e 
indígena é a mesma coisa. Não é. O próprio dicionário diz isso.

A palavra ``indígena'' diz muito mais a nosso respeito do que a palavra 
``índio''. A palavra índio gera uma imagem distorcida. Já indígena quer 
dizer originário, aquele que está ali antes dos outros.\end{quote}

A entrevista completa está presente no site G1, em \href{https://g1.globo.com/educacao/noticia/2019/04/19/dia-do-indio-e-data-folclorica-e-preconceituosa-diz-escritor-indigena-daniel-munduruku.ghtml}{g1.globo.com/educacao}. 

\item
Após a leitura dos verbetes do dicionário de Bluteau e do trecho da 
entrevista de Daniel Munduruku, o professor de português deve propor 
aos alunos que pensem em palavras que mudaram no decorrer do tempo e que
um dia foram amplamente utilizadas e hoje em dia não mais. Exemplos: 
denegrir, indiada, judiar, mulato etc. Ao fim da discussão, deve ser
ser feito um levantamento das palavras e a apresentação de outras a ser
utilizadas em seu lugar. 

\end{enumerate} 

\paragraph{Tempo estimado} Duas a três aulas de 50 minutos.


\subsection{Leitura}

\paragraph{Tema} A baía de Guanabara.

\paragraph{Conteúdo} Escrita poética a partir de relatos sobre a 
baía de Guanabara.

\paragraph{Objetivo} Habilitar os alunos a se posicionar sobre o imaginário
do Brasil a partir da criação de um poema sobre um dos principais pontos
turísticos do país.

\paragraph{Justificativa} Não é de hoje que o Rio de Janeiro causa 
diferentes comoções naqueles que se deparam com sua paisagem montanhosa
em frente ao mar. Ao olhar estrangeiro, é a paisagem carioca a primeira
que vem à mente quando se pensa no Brasil. 
A presente obra apresenta talvez a mais detalhada descrição desta região
do período colonial, o que muito pode ter ajudado na divulgação de sua 
fama mundo afora. 
No decorrer do tempo, no entanto, não foram só boas as impressões que se 
tiveram sobre a baía de Guanabara. Em ``O Estrangeiro'', Caetano Veloso
cita a reação negativa do antropólogo francês Claude Lévi"-Strauss em sua 
presença, e o próprio compositor enxerga nela as contradições que 
constituem o Brasil, afastando"-a, assim, do ideal de beleza absoluta 
que alguns lhe associam. 
A partir disso, é importante que o aluno do ensino médio possa se inteirar
nesta discussão sobre uma paisagem que costuma representar o seu próprio
país e propor, segundo seu ponto de vista, uma contribuição para a 
construção de imaginário do mesmo. 

\paragraph{Metodologia}
\begin{enumerate}

\item Os capítulos 50, 51 e 52 do \textit{Tratado} se comprometem a fazer
uma apurada descrição da baía de Guanabara na cidade do Rio de Janeiro. 
O professor deve pedir que os alunos os leiam em sala de aula, chamando
à atenção trechos como o seguinte:

\begin{quote} É tamanha coisa o Rio de Janeiro da boca para dentro, que 
nos obriga a gastar o tempo em o declarar neste lugar, para que se veja 
como é capaz de se fazer mais conta dele do que se faz. E comecemos do 
Pão de Açúcar, que está da banda de fora da barra, que é um pico de
pedra mui alto, da feição do nome que tem, do qual, à ponta da barra, 
que se diz de Cara de Cão, há pouco espaço; e a terra, que fica entre 
esta ponta e o Pão de Açúcar, é baixa e chã; e virando"-se desta ponta 
para dentro da barra se chama Cidade Velha, onde se ela fundou primeiro.
\end{quote}

\item
Após a leitura das descrições esmiuçadas de Gabriel Soares sobre a 
paisagem do Rio de Janeiro, o professor deve apresentar aos alunos 
a canção \href{https://www.youtube.com/watch?v=uIMZLr7LXVE}{``O Estrangeiro''} do cantor e compositor baiano Caetano Veloso.
\BNCC{EM13LP14}

Nesta canção, além de uma descrição poética da baía de Guanabara, o 
cantor faz referência a figuras célebres que já fizeram referência ao
lugar. 

\begin{verse}
O pintor Paul Gauguin amou a luz na Baía de Guanabara
O compositor Cole Porter adorou as luzes na noite dela
A Baía de Guanabara
O antropólogo Claude Lévi"-Strauss detestou a Baía de Guanabara
Pareceu"-lhe uma boca banguela
E eu menos a conhecera, mais a amara
Sou cego de tanto vê"-la, de tanto tê"-la estrela
O que é uma coisa bela
[...]
Uma baleia, uma telenovela, um alaúde, um trem, uma arara
Mas era ao mesmo tempo bela e banguela a Guanabara
Em que se passara, passa, passará o raro pesadelo
Que aqui começo a construir, sempre buscando o belo e o amaro
Eu não sonhei
A praia de Botafogo era uma esteira rolante de areia branca e óleo diesel sob meus tênis
E o Pão de Açúcar menos óbvio possível à minha frente
Um Pão de Açúcar com umas arestas insuspeitadas
[...]
\end{verse}

\item 
A partir das duas produções, os alunos devem ser convidados, agora, a 
produzir um texto poético sobre a baía de Guanabara. Caso já a tenham 
visitado, podem se valer da lembrança da experiência, e caso não, devem 
partir das referências apresentadas e do imaginário comum brasileiro 
sobre o Rio de Janeiro. É importante se levar em conta que este foi 
durante muito tempo e de certa forma ainda é um dos principais cartões 
postais do Brasil para os estrangeiros. Em sua canção, Caetano Veloso 
propõe um olhar crítico em relação a esta posição, situando o espaço 
entre o ``belo'' e o ``banguelo'', ou seja, o bonito e o feio. Qual será 
a postura assumida pelo eu lírico do poema em relação à baía de Guanabara? 
\BNCC{EM13LP54} 

\item 
Ao fim da produção, os poemas devem ser compartilhados num jornal virtual 
da escola.

\end{enumerate}
\paragraph{Tempo estimado} Duas a três aulas de 50 minutos.

\subsection{Pós"-leitura}

\paragraph{Tema} O \emph{Tratado} e os jornais.  

\paragraph{Conteúdo} Elaboração de matéria jornalística a partir
de pesquisa diacrônica de reportagens e da leitura do \emph{Tratado descritivo do Brasil.}

\paragraph{Objetivo} Habilitar os alunos a perceber, dentro das 
delimitações dos gêneros jornalísticos e tratadista, como se dá,
sobretudo no primeiro, a construção de uma imagem negativa dos povos 
indígenas no território brasileiro.
\BNCC{EM13CHS304}

\paragraph{Justificativa} O tratado, segundo o Novo Dicionário da Língua
Portuguesa, é um estudo formal, científico, de caráter acadêmico, 
fundamentado e sistemático sobre um determinado assunto. Exemplos
clássicos remontam à Antiguidade romana com obras como a \emph{História Natural}
de Plínio, o Velho, um conjunto de trinta e sete livros sobre as
mais diversas áreas do conhecimento humano.
Já o gênero jornalístico é difícil de ser descrito sucintamente
já que abarca uma diversidade de subgêneros. O texto pode ser informativo,
opinativo, entretenimento, propaganda etc. E, às vezes, todos estes 
elementos podem estar presentes numa mesma reportagem. 
\BNCC{EM13LP49}

A partir da capacidade de formação de opinião de jornais e revistas
em relação a seu público, é interessante notar o possível papel deste
veículo de informação para a difusão de preconceitos relacionados aos
povos indígenas brasileiros. 

\SideImage{Propaganda da década de 1970 incentivando o adentramento do 
capital no território amazônico. A que se refere as ``lendas'' 
que devem acabar?. (Revista \emph{Realidade}, 1972; Reprodução/Acervo Ricardo Cardim)}{PNLD0015-11}
%\BNCC{EM13CHS201}

\paragraph{Metodologia}
\begin{enumerate}

	\item
	Divididos em grupos, os alunos devem desenvolver uma pesquisa
	em fontes abertas e confiáveis sobre reportagens, antigas e recentes, 
	que se refiram aos povos indígenas no Brasil. 

	\item
	Articulando também as informações obtidas sobre o assunto durante a 
	leitura do \emph{Tratado descritivo do Brasil}, os grupos devem 
	elaborar uma matéria jornalística sobre o tema. É importante que o 
	tom seja entre informativo e opinativo: devem ser apresentados dados 
	que desdigam os discursos pejorativos sobre estes povos, por um lado, 
	e por outro, deve haver uma argumentação acerca da escolha dos 
	conteúdos que entraram na matéria.
	\BNCC{EM13LGG303}

	O trabalho pode ser feito tanto por escrito, no formato de um jornal
	impresso, como por meio de recursos audiovisuais. Em ambos os casos,
	sugerimos que os resultados das produções sejam compartilhadas no blog
	da escola para divulgação entre a comunidade. 
\end{enumerate}

\paragraph{Tempo estimado} Quatro aulas de 50 minutos.

\Image{Vista de Olinda, Frans Post, 1647 (Acervo digital da Biblioteca Nacional; Domínio Público )}{PNLD0015-04.png}


\section{Propostas de Atividades II}

\subsection{Pré"-leitura}

\paragraph{Tema} As línguas faladas pelos ameríndios. 

\paragraph{Conteúdo} Apreciação de materiais audiovisuais em línguas
indígenas e, depois, pesquisa de vocábulos de origem indígenas na língua
portuguesa falada atualmente no Brasil.

\paragraph{Objetivo} Apresentar aos alunos o universo linguístico do Brasil
Colônia a partir da exibição do filme \emph{Hans Staden} que, além de 
ter boa parte dos diálogos em tupi antigo, representa o contexto colonial
no século \textsc{xvi}.
\BNCC{EM13LP01}

\paragraph{Justificativa} Até a proibição imposta por Marquês de Pombal
em 1758, com intuito de enfraquecer o poder da Igreja Católica na Colônia,
e consolidar o do Império, a língua majoritariamente falada no território 
brasileiro era a língua geral, o \emph{nheengatu}, literalmente, a 
``língua boa'', uma espécie de junção entre vocabulário do tupi antigo e 
a sintaxe do português. Utilizada por todos, desde os colonos portugueses 
até os negros escravizados e os indígenas de diversas etnias, era o
código comum no território e importante instrumento de influência dos
jesuítas que, por meio dela, promoviam a evangelização dos não"-cristãos.

Seu uso caiu em grande queda após a interdição imposta pelo primeiro
ministro português que visava reduzir ao máximo a influência da organização
militar"-religiosa da Companhia de Jesus. Hoje em dia, a língua é falada
na região do vale do Rio Negro, no estado do Amazonas e é língua oficial 
no município de São Gabriel da Cachoeira. 

Como a língua é um importante fator de reconhecimento de um povo, 
alguns povos indígenas que perderam o contato com suas línguas nativas
buscam hoje no nheengatu ou no tupi antigo um instrumento de retomada
simbólica e material de seus valores. Neste sentido, filmes como \emph{Hans Staden},
cujos diálogos são majoritariamente em tupi antigo, são uma importante 
fonte de acesso dos alunos a este universo linguístico.  

\paragraph{Metodologia}

\begin{enumerate}

\item 
Em horário de aula, o professor deve exibir o filme \emph{Hans Staden},
que conta a história de um mercenário alemão capturado por tupinambás 
no Brasil do século \textsc{xvi}. O filme contém muitos diálogos em 
tupi antigo, língua então muito falada na costa brasileira, e servirá
como porta de entrada aos alunos em relação a esta expressão linguística.
\BNCC{EM13LGG602}

\item 
Após a exibição, pode ser feita uma roda de conversa para se
compartilhar impressões e comentários sobre o filme. O professor de 
História pode ressaltar dados sobre o contexto do Brasil Colônia no 
século \textsc{xvi} bem como os interesses da Coroa portuguesa em proibir
o uso da língua indígena no século \textsc{xviii}.

\item 
Em seguida, divididos em grupos, os alunos devem realizar uma pesquisa 
acerca de palavras da língua portuguesa falada hoje em dia no Brasil 
que tenham origem nas línguas indígenas, sobretudo o tupi antigo. 
Muito será encontrado na área da toponímia, os nomes de cidades, rios etc.
Uma vez finalizada, os grupos devem fazer uma apresentação com os 
resultados encontrados, elencando as principais palavras. 
\BNCC{EM13CHS601}
\end{enumerate}

\paragraph{Tempo estimado} Quatro aulas de 50 minutos.


\subsection{Leitura}

\paragraph{Tema} Os monstros marinhos nas narrativas do ``descobrimento''.

\paragraph{Conteúdo} Leitura de trechos de textos e imagens descritivos
sobre monstros marinhos à época das expansão marítima europeia e 
elaboração de seminário sobre o tema articulando conhecimentos das 
Ciências Humanas e Naturais.  
\BNCC{EM13LP16}

\paragraph{Objetivo} Habilitar os alunos a compreender a função dos 
monstros marinhos nas narrativas de viajantes, em diferentes linguagens, 
dos primeiros séculos da expansão marítima europeia.

\paragraph{Justificativa} Um dos pontos altos do \emph{Tratado} é sem
dúvida a narrativa sobre o monstro marinho. Presente em outros textos 
dos navegantes do século \textsc{xvi}, estes seres estão diretamente
ligados aos perigos enfrentados durante a aventura nos mares e à natureza
desconhecida deste novo ambiente. É interessante perceber como os relatos
são parecidos em diferentes autores, o que nos faz pensar que se tinha 
um conceito comum sobre o que viria a ser um monstro marinho. 
Um professor de Ciências Humanas deve trazer à luz a compreensão de que
os primeiros viajantes europeus acabavam de sair de um contexto cultural
da Idade Média, onde os perigos da vida eram todos explicados por um
viés mítico"-religioso cristão. Os perigos do mar aberto e das grandes 
viagens, até então desconhecidos, eram ampliados e personificados nas
figuras dos monstros marinhos. No entanto, conforme as navegações 
tornavam"-se um importante recurso econômico, as referências a estes seres
passavam a se tornar cada vez mais raras, a fim justamente de construir
um imaginário positivo sobre estas atividades. 
Sobre este assunto, mais informações podem ser encontradas \href{https://www2.ufjf.br/noticias/2015/12/16/monstros-na-historia-relatos-de-viagens-expedicionarias-e-brasil-colonia/}{neste artigo}.

\paragraph{Metodologia}

\begin{enumerate}

\item Peça para os alunos lerem o seguinte trecho da presente obra:

\begin{quote} 
\emph{Que trata dos homens marinhos}
Não há dúvida senão que se encontram na Bahia e nos recôncavos dela 
muitos homens marinhos, a que os índios chamam pela sua língua upupiara,
\footnote{Upupiara, Ipupiara oi Igpupiara. Do tupi \textit{îpupi'ara}, 
aquele que vive nas águas ou homem marinho.} os quais andam pelo rio de 
água doce pelo tempo do verão, onde fazem muito dano aos índios
pescadores e mariscadores que andam em jangadas, onde os tomam, e aos que 
andam pela borda da água, metidos nela; a uns e outros apanham, e metem"-nos 
debaixo da água, onde os afogam; os quais saem à terra com a maré vazia 
afogados e mordidos na boca, narizes e na sua natura; e dizem outros índios pescadores que viram tomar estes mortos, que viram sobre água uma cabeça 
de homem lançar um braço fora dela e levar o morto; e os que isso viram 
se recolheram fugindo à terra assombrados, do que ficaram tão atemorizados 
que não quiseram tornar a pescar daí a muitos dias; o que também aconteceu 
a alguns negros de Guiné; os quais fantasmas ou homens marinhos mataram 
por vezes cinco índios meus; e já aconteceu tomar um monstro destes dois 
índios pescadores de uma jangada e levarem um, e salvar"-se outro tão 
assombrado que esteve para morrer; e alguns morrem disto. E um 
mestre"-de"-açúcar do meu engenho afirmou que olhando da janela do engenho 
que está sobre o rio, e que gritavam umas negras, uma noite, que estavam 
lavando umas formas de açúcar, viu um vulto maior que um homem à borda da 
água, mas que se lançou logo nela; ao qual mestre"-de"-açúcar as negras 
disseram que aquele fantasma vinha para pegar nelas, e que aquele era o 
homem marinho, as quais estiveram assombradas muitos dias; e destes 
acontecimentos acontecem muitos no verão, que no inverno não falta nunca 
nenhum negro.
\end{quote}

\item Em seguida, apresente a figura reproduzida abaixo de uma gravura do 
monstro marinho, ilustração do livro \emph{História da Província de Santa~
Cruz}, de Pero de Magalhães de Gandavo. Junto a ela, peça que os alunos 
leiam um trecho do mesmo livro sobre o monstro:
\BNCC{EM13LP48}

\begin{quote} Foi coisa tão nova e tão desusada aos olhos humanos a 
semelhança daquele fero e espantoso monstro marinho que nesta província 
se matou no ano de 1564, que ainda que por muitas partes do mundo se 
tenha já notícia dele, não deixarei todavia de a dar aqui outra vez de 
novo, relatando por extenso tudo o que acerca disto passou. Porque na 
verdade a maior parte dos retratos, ou quase todos, em que querem mostrar 
a semelhança de seu horrendo aspecto, andam errados, e além disso,
conta"-se o sucesso de sua morte por diferentes maneiras, sendo a
verdade uma só, a qual é a seguinte. Na capitania de São Vicente, sendo				\
já alta noite a horas em que todos começavam de se entregar ao sono,
acertou de sair fora de casa uma índia escrava do capitão; a qual
lançando os olhos a uma várzea que está pegada com o mar, e com a
povoação da mesma capitania, viu andar nela este monstro, movendo"-se de
uma parte para outra, com passos e meneios desusados, e dando alguns urros
de quando em quando tão feios, que como pasmada e quase fora de si, se
veio ao filho do mesmo capitão, cujo nome era Baltasar Ferreira, e lhe
deu conta do que vira, parecendo"-lhe que era alguma visão diabólica.
Mas como ele fosse homem não menos sisudo que esforçado, e esta gente
da terra seja digna de pouco crédito, não lho deu logo muito a suas
palavras, e deixando"-se estar na cama, a tornou outra vez a mandar
fora dizendo"-lhe que se afirmasse bem no que era. E obedecendo a índia
a seu mandado foi; e tornou mais espantada, afirmando"-lhe e
repetindo"-lhe uma vez e outra que andava ali uma coisa tão feia, que não
podia ser senão o demônio. Então se levantou ele mui depressa, e
lançou mão a uma espada que tinha junto de si, com a qual botou somente
em camisa pela porta fora, tendo para si (quando muito) que seria algum
tigre, ou outro animal da terra conhecido, com a vista do qual se
desenganasse do que a índia lhe queria persuadir. E pondo os olhos
naquela parte que ela lhe assinalou, viu confusamente o vulto do
monstro ao longo da praia, sem poder divisar o que era, por causa da
noite lho impedir e o monstro também ser coisa não vista, e fora do
parecer de todos os outros animais. E chegando"-se um pouco mais a ele
para que melhor se pudesse ajudar da vista, foi sentido do mesmo
monstro, o qual em levantando a cabeça, tanto que o viu, começou de
caminhar para o mar donde viera. Nisto conheceu o mancebo que
era aquilo coisa do mar, e antes que nele se metesse, acudiu com muita
presteza a tomar"-lhe a dianteira, e vendo o monstro que ele lhe embargava 
o caminho, levantou"-se direito para cima como um homem, fincado sobre as 
barbatanas do rabo, e estando assim a par com ele, deu"-lhe uma estocada 
pela barriga, e dando"-lha no mesmo instante se desviou para uma parte 
com tanta velocidade, que não pôde o monstro levá"-lo debaixo de si; 
porém não pouco afrontado, porque o grande torno
de sangue que saiu da ferida, lhe deu no rosto com tanta força que
quase ficou sem nenhuma vista.
\end{quote}

\SideImage{Gravura reproduzindo o monstro marinho de Gandavo e a forma 
como foi capturado.}{PNLD0015-10}

\item Agora os alunos devem fazer uma tabela comparativa entre as três 
representações do monstro: a do \emph{Tratado} e a de Gandavo, escritas,
a visual, na ilustração acima. 
\BNCC{EM13LP50}

\item 
Em seguida, deve ser feita uma pesquisa por ilustrações de mapas do
período da expansão marítima europeia onde encontram outros monstros
marinhos, principalmente os supostamente descobertos durante as viagens.
\BNCC{EM13CHS106} 

Após a recolhida das imagens, os alunos devem fazer uma nova comparação 
com as descrições já feitas. O que muda e o que é similar? 
Será que os ``seres do mar'' de que falavam os indígenas possuíam de fato
aquela aparência ou os europeus projetaram aquilo com o que estavam 
acostumados em seu imaginários sobre as criaturas marítimas? 
Um seminário final deve ser apresentado sobre o tema. 

\end{enumerate}

\paragraph{Tempo estimado} Três a quatro aulas de 50 minutos.

\Image{Primeira edição, organizada e revisada pelo historiador brasileiro Adolfo de Varnhagen (1816--1878), 1851 (Biblioteca Brasiliana Guita e José Mindlin; Domínio Público)}{PNLD0015-05.png}

\subsection{Pós"-leitura}

\paragraph{Tema} O que ficou dessa época?

\paragraph{Conteúdo} Leitura da carta de dedicatória do livro, exibição de 
filme sobre a época da colonização, discussão e elaboração de texto 
dissertativo"-argumentativo a este respeito.

\paragraph{Objetivo} Proporcionar aos alunos elementos para suscitar
uma discussão acerca dos mecanismos que possibilitaram o processo 
colonizatório nos territórios americanos e das possíveis reverberações
no presente.

\paragraph{Justificativa} O primeiro texto do \emph{Tratado} é uma carta
endereçada ao então vice"-rei de Portugal, D.~Cristóvão de Moura, onde 
se dedica a presente obra à majestade portuguesa. Muito comum em escritos
da época, estas dedicatórias ressaltam o contexto histórico da monarquia 
onde o poder estava totalmente centralizado na figura do rei.
A relação entre monarquia e colonização é muito representada, também,
no filme ``Aguirre: a cólera dos deuses'', de Werner Herzog, onde, em
meio a uma expedição em busca de ouro nas Américas espanholas, um grupo 
deserta"-se da Coroa e se autoproclama uma nova monarquia a fim de não ter 
que prestar contas nem dividir os lucros a serem conquistados em forma
de ouro e terras.
Pensando na monarquia portuguesa, o desenvolvimento na história de seus
brasões ou bandeiras durante o domínio sobre o território brasileiro
suscitam uma discussão acerca desta relação. Sobretudo quando percebemos
que as cores da atual bandeira do Estado do Brasil remetem às das famílias
imperiais europeias da época colonial. 

\paragraph{Metodologia}
\begin{enumerate}
\item 
Os alunos devem fazer uma pesquisa sobre a evolução das bandeiras da Coroa
portuguesa no decorrer da história, prestando a atenção ao que significam
seus símbolos e cores. Então, devem fazer o mesmo com as bandeiras do 
Estado brasileiro. Há alguma relação entre elas no simbolismo de suas
cores e elementos visuais? Sugerimos uma roda de conversa para se 
compartilhar as impressões acerca do tema pesquisado e estudado durante
as aulas. Durante esta roda, pode ser feita a exibição do filme ``Aguirre: 
a cólera dos deuses'', disponível \href{https://www.youtube.com/watch?v=weuYp-XFxAo}{aqui}, baseado nos diários do Frei Gaspar de Carvajal 
de meados do século \textsc{xvi}, que conta a história de uma expedição 
espanhola na Amazônia em busca da cidade mítica de Eldorado.
\BNCC{EM13LGG602}

\item 
Na sequência da discussão e da possível exibição do filme, os alunos devem 
preparar um texto dissertativo"-argumentativo, amparados nos argumentos 
surgidos nas discussões, nas pesquisas e em todas as atividades 
precedentes, acerca da colonização no território brasileiro e nas Américas. 
É esperado que os alunos articulem seus principais aspectos:
a relação dos colonizadores com os povos indígenas, a importância e o papel
da Coroa portuguesa neste processo, a escravidão decorrida do modelo 
econômico, e, por fim, os possíveis efeitos que esta experiência podem 
ter causado na atual sociedade brasileira onde vivemos.
\BNCC{EM13LP05} 

Os textos deverão ser publicados num jornal virtual da escola para a 
socialização com as outras turmas e a comunidade.

\end{enumerate}

\paragraph{Tempo estimado} Duas aulas de 50 minutos.


\section{Aprofundamento}


Nesta seção, desenvolvemos um trabalho de aprofundamento que, em diálogo
com a formação continuada de professores, oferece subsídios para a
abordagem do texto literário.

O \textit{Tratado descritivo do Brasil} está dividido em dois livros: 
\textit{Roteiro geral com largas informações de toda a costa do Brasil} e \textit{Memorial e declaração das grandezas da Bahia de todos os Santos, de sua fertilidade e das notáveis partes que tem}. O primeiro livro contém um proêmio e 74 capítulos. A segunda parte, por sua vez, abrange 196 capítulos. Tanto a edição de 1851 quanto a de 1879 são finalizadas com 270 comentários e observações de Francisco Adolpho de Varnhagen.

O termo ``roteiro'' refere"-se a um gênero recorrente, sobretudo na época dos descobrimentos, utilizado para descrever em detalhes uma viagem ou estabelecer uma rota ou guia para os navegantes, apontando precisamente os cabos, baixios, ilhas, portos ou rios navegáveis; enfim, tudo o que poderia servir para orientar um navegante conquistador. O memorial da Bahia faz parte de um gênero também corrente à época, o registro de lembranças.

As informações prestadas por Gabriel Soares de Sousa em seus livros empregam diversas áreas do conhecimento, tais como a náutica, a botânica, a zoologia e a memorialística, narrando as vivências, memórias e impressões do autor sobre os lugares, animais, plantas e gentes do Brasil. Gabriel tinha o poder da escrita, capaz de agradar qualquer tipo de público leitor.

É possível afirmar ainda que Gabriel, mesmo não tendo conhecido pessoalmente todos os cantos da costa brasileira, não deixou de descrevê"-los em seu tratado, muito provavelmente recorrendo a outras fontes para isso. Por certo percorreu toda a circunvizinhança da cidade de Salvador, por terra e por mar. Mas, provavelmente, recorreu a outras fontes para se referir com tanta minudência ao restante do território.

Boa parte do que Gabriel Soares escreveu deve ter sido compilado de informações encontradas na própria península durante o longo período em que lá esteve, possivelmente recebendo cartas remetidas de suas fazendas. Ao que tudo indica não percorreu toda costa brasileira, do rio de Vicente Pinzon, acima do rio das Amazonas, até algumas léguas depois da Baía de São Matias, na atual Argentina. 

Na primeira parte do livro, o \emph{Roteiro geral, com largas informações de toda a costa do Brasil}, Sousa descreve, de maneira bem detalhada, tudo o que se poderia encontrar na região costeira do Brasil, do rio Amazonas, passando pelo rio São Francisco e a baía do Rio de Janeiro, até o Rio da Prata.

Conta quem foram os primeiros povoadores das capitanias e os heróis colonizadores, fala quais são as atividades e potencialidades econômicas e como é a navegabilidade de cada região costeira, onde é possível adentrar com caravelões, como são os portos, o relevo, o clima, a vegetação, os frutos, a fauna, os rios, a fertilidade do solo, a localização e condições dos povoados, engenhos, plantações e áreas com pau"-brasil, onde é possível haver fluxo de pessoas e de mercadorias e, por fim, onde é possível encontrar os tão cobiçados metais e pedras preciosas.

Para ganhar a confiança do rei espanhol e mostrar credibilidade, o colono português fornece todo o tipo de informações estratégicas, coordenadas exatas de como chegar nos lugares, quais as vantagens econômicas de tudo que pode ser encontrado na colônia, bem como quais são as necessidades e perigos encontrados na mesma.

Em tom de crítica ou de conselho, dá sugestões de administração ao rei, pede mais recursos e investimentos, dizendo quais lugares necessitam de fortificação contra corsários estrangeiros, ou contra ``maus selvagens'', e quais localidades são boas para se conquistar, povoar e explorar, implantar engenhos, criar animais ou estabelecer plantações, assim como fez ao descrever o rio São Francisco.
Além disso, pelo viés econômico e exploratório, Gabriel Soares mostra o quanto a colônia não deixa nada a desejar, ou é superior, em comparação com o que pode ser encontrado em Portugal, Espanha, Índias ou nos demais territórios sob o domínio das coroas ibéricas, como ao falar das vacas criadas nas capitanias de São Vicente e Santo Amaro. 

\Image{Viagens marítimas de Hans Staden feitas em vários momentos em 1547 de Portugal/Espanha para o Rio de Janeiro e outros lugares do Brasil. (Acervo digital da Biblioteca Nacional; Domínio Público )}{PNLD0015-08.png}

Em uma pequena apresentação que precedia o \textit{Roteiro geral} e o
\textit{Memorial}, dedicada a Sua Majestade, sob o título ``Descrição do
que se contém neste caderno'', Soares de Sousa justifica os manuscritos a
Filipe 	\textsc{ii}. Nesse breve texto, conhecido como ``Proêmio'', por assim ter
sido nomeado por Varnhagen, o autor explica, como leal vassalo, que sua
``pretensão é manifestar a grandeza, fertilidade e outras grandes partes
que têm a Bahia de Todos os Santos e demais Estados do Brasil'', levando
ao conhecimento do rei as condições em que se encontrava a colônia
portuguesa, cujas terras tinham sido deixadas em estado de abandono
pelos monarcas anteriores, justamente pela pouca notícia que dali lhes
chegava. Continua sua justificativa argumentando que ``a el''"-rei nosso
senhor convém, e ao bem do seu serviço, que lhe mostre, por estas
lembranças, os grandes merecimentos deste seu Estado, as qualidades e
estranhezas dele, etc., para que lhe ponha os olhos e bafeje com seu
poder''. Gabriel Soares estava defendendo que apenas quando detentora de
tal conhecimento é que a Coroa poderia proteger aquelas possessões e
explorá"-las adequadamente.\footnote{Esse texto foi publicado pela
primeira vez por Varnhagen, precedendo o <<Roteiro geral>> e o
<<Memorial>>, na edição de 1851 e posteriormente na edição espanhola,
organizada por Cláudio Gans. O documento encontra"-se publicado na presente edição
sob o título <<Descrição do que se contém neste caderno>>.}

\SideImage{A edição de 1851 e a de 1879 de <<Tratado Descritivo do Brazil em 1581>> são finalizadas com 270 comentários e observações de Francisco Adolpho de Varnhagen (Acervo digital da Biblioteca Nacional; Domínio Público )}{PNLD0015-06.png}


Em 1590, após pelo menos quatro anos de espera, as solicitações de
Gabriel Soares de Sousa foram atendidas por Filipe \textsc{ii}. Entre as
principais concessões reais estavam o título de capitão"-mor e
governador da conquista do Rio São Francisco, o direito de nomear seu
sucessor em caso de falecimento e a permissão para prover por três anos
todos os ofícios de justiça e de fazenda nas terras que fossem por ele
ocupadas. Além disso, o rei incentivou esse empreendimento colonial por
meio da distribuição de honras e mercês aos primeiros participantes da
expedição e da permissão ao governador"-geral do Estado do Brasil, D.~Francisco de Sousa,
\footnote{ D.~Francisco de Sousa (c.1540--1611),
fidalgo português, residia na corte filipina, ocasião que o fez
conhecer Gabriel Soares de Sousa, ao ser nomeado pelo rei Filipe \textsc{ii}
como sétimo governador"-geral do Brasil, cargo que passou a exercer, na
Bahia, a partir de 1591. Onze anos depois, retornou à Corte, onde
iniciou negociações para voltar ao Brasil em busca de metais e pedras
preciosas nas capitanias ao sul da Bahia. Permaneceu no reino até 1609,
quando foi novamente instituído de um importante cargo na colônia, o de
governador das capitanias do sul} para que cedesse duzentos índios
flecheiros à empreitada.\footnote{ Os alvarás que concedem a Gabriel
Soares de Sousa essas honras e mercês estão publicados em \textit{
Pauliceae Lusitana Monumenta Historica}. Lisboa: Real Gabinete Portugal
de Leitura do Rio de Janeiro, 1956, tomo \textsc{i}, p.\,407} Em posse
desses privilégios reais, Gabriel Soares organizou uma armada com cerca
de 360 homens para retornar ao Brasil em uma urca
flamenga fretada pela Fazenda Real. Partiu de Lisboa em sete de abril
de 1591, mas, antes que alcançasse seu objetivo final, a embarcação
naufragou na enseada de Vazabarris,\footnote{ Esse topônimo é derivado
da expressão portuguesa “dar em vaza"-barris”, que significava perder"-se
sem esperanças de salvação, pois aquela era uma região em que ocorriam
frequentes naufrágios.} no litoral sergipano, o que levou à morte de
alguns tripulantes e à perda de parte do armamento. O nomeado
capitão"-mor e governador da conquista, contudo, não desistiu e caminhou
até a Bahia com os sobreviventes para lá reorganizar a expedição com
auxílio de D.~Francisco de Sousa. Naquele mesmo ano de 1591 partiu para
o sertão em direção à foz do Rio São Francisco em busca das tão
sonhadas minas, mas a ele sucederia o mesmo que a seu irmão anos antes,
vindo a falecer no início da viagem. As circunstâncias de sua morte não
são claras, se causada por doença ou por vingança de índios
aprisionados; o que se sabe é que em seguida à fatalidade seus ossos
foram levados para a Capela do Mosteiro de São Bento na cidade de
Salvador. Acredita"-se que seu corpo tenha sido sepultado no lugar onde
ainda hoje se encontra, no interior da capela, uma lápide com a
inscrição ``aqui jaz um pecador''.\footnote{Em seu
testamento, Gabriel Soares de Sousa pedia que fosse enterrado na 
capela"-mor do Mosteiro de São Bento e que sobre sua sepultura fosse colocada
aquela mesma inscrição.}

Parece, portanto, que os manuscritos \textit{Roteiro geral} e
\textit{Memorial}, endereçados a um dos mais influentes ministros de
Filipe 	\textsc{ii}, D.~Cristóvão de Moura, para que chegassem ao conhecimento do
rei, surtiram o efeito esperado por Gabriel Soares, que conseguiu as
extraordinárias concessões reais para realizar sua expedição na
colônia. E, apesar de suas ricas informações (é provavelmente a fonte
documental mais completa a respeito do primeiro século de colonização
do Brasil), não admira que esses textos não tenham sido publicados até
o século \textsc{xix}. Ainda que não fossem impressos, os escritos sobre o
ultramar dos séculos \textsc{xvi} e \textsc{xvii} não permaneceram completamente
desconhecidos dos leitores contemporâneos nem das gerações posteriores.
Muitos textos correlatos e coevos aos manuscritos de Gabriel Soares
tiveram a mesma sorte, desde a Carta de Pero Vaz de Caminha a D.~Manuel~\textsc{i}, 
datada de 1500, que só veio a ser publicada em 1817;\footnote{A
carta de Pero Vaz ao rei permaneceu desconhecida por mais de dois
séculos, conservada no Arquivo Nacional da Torre do Tombo, em Lisboa.
Foi encontrada pelo secretário de Estado português José de Seabra da
Silva, em 1773, noticiada pelo historiador espanhol Juan Bautista Muñoz
e publicada pela primeira vez pelo padre Manuel Ayres de Cazal em sua
\textit{Corografia Brasílica}.} passando pelo
\textit{Diário de Navegação}, de Pero Lopes de Sousa, divulgado por
Varnhagen em edição de 1839;\footnote{ Cf. Pero Lopes de Sousa, \textit{Diário da navegação da armada que 
foi à terra do Brasil em 1530 sob a capitania"-mor de Martim Afonso de Sousa,
escrito por seu irmão Pero Lopes de Sousa}. Publicado por Francisco Adolfo de Varnhagen. 
Lisboa: Typ. da Sociedade Propagadora dos Conhecimentos Uteis, 1839.} e pelos textos do jesuíta Fernão Cardim
escritos entre 1583 e 1601, reunidos e publicados 
sob o título de \textit{Tratados da terra e gente do Brasil} apenas em
1925;\footnote{ Cf. Fernão Cardim, \textit{Tratados da terra e gente do Brasil}. 
Introdução e notas de Batista Caetano, Capistrano de Abreu e Rodolfo Garcia. 
Rio de Janeiro: J.~Leite e Cia., 1925. 
A obra reúne três manuscritos do autor: \textit{Do clima e terra do Brasil}, \textit{Do princípio e Origem dos índios do Brasil}
e \textit{Narrativa Epistolar}, este último havia sido publicado por Varnhagen em 1847.} até os manuscritos do frei franciscano Vicente do Salvador,
\textit{História do Brasil 1500--1627}, publicados na íntegra em 1888.\footnote{ Cf. Vicente do Salvador, \textit{História do Brasil 1500--1627}.
Introdução de Capistrano de Abreu. Rio de Janeiro: Anais da Biblioteca Nacional, vol.\,13, 1888.}
Uma das poucas exceções a essa prática é a \textit{História da
província de Santa Cruz}, de autoria do gramático Pero de
Magalhães Gandavo, que foi escrita e impressa em língua portuguesa no
próprio século \textsc{xvi}.\footnote{ Cf. Pero de Magalhães Gandavo,
\textit{História da Província de Santa Cruz} \& \textit{Tratado da
Terra do Brasil}. Lisboa: Officina de Antônio Gonsalves, 1576.}

No caso do \textit{Tratado descritivo do Brasil}, o considerável
circuito de sua distribuição e consumo, ainda que sob a forma anônima
ou apócrifa, parece notório ao se verificar, por um lado, que cópias
dos manuscritos são encontradas hoje em arquivos públicos e
particulares de Portugal, Espanha, França, Inglaterra e Áustria e, por
outro, que muitos autores fizeram referências a Gabriel Soares ou a seu
texto, com autoria equivocada ou anônima, antes mesmo de sua primeira
publicação. Entre eles destacam"-se Pedro de Mariz, já no próprio século
\textsc{xvi}, Frei Vicente de Salvador, Antônio Leon Pinelo e Simão de
Vasconcelos, no século \textsc{xvii}, Frei Antônio de Santa Maria Jaboatão, no
século \textsc{xviii}, e Pedro Manuel Ayres de Cazal e Von Martius, no século
\textsc{xix}.\footnote{ Cf. Pedro de Mariz, \textit{Diálogos de Vária História}.
Coimbra: Officina de Antonio de Mariz, 1594; Frei Vicente do
Salvador,\textit{ História do Brasil} [1627]. Rio de Janeiro: Anais da
Biblioteca Nacional, vol.\,13, 1888; Antônio Leon Pinelo, \textit{Epítome
de la Biblioteca oriental y occidental, náutica y geográfica}. Madri:
por Juan Gonzales, 1629;  Simão de Vasconcelos, \textit{Crônica
da Companhia de Jesus no Estado do Brasil}. Lisboa: 1663, e
\textit{Notícias Curiosas e Necessárias das Cousas do Brasil}. Lisboa:
por João da Costa, 1668; Frei Antonio de Santa Maria Jaboatão,
\textit{Novo Orbe Sefárico Brasílico ou Crônica dos Frades Menores do
Brasil}. Lisboa: Officina de Antonio Vicente da Silva, 1761; 
Pedro Manuel Ayres de Cazal, \textit{Corografia Brazílica, ou Relação
Histórico"--Geográfica do Reino do Brasil}. Rio de Janeiro: Impressão
Régia, 1817; e Karl Friedrich Von Martius,
\textit{Nova Genera et Species Plantarum
Brasiliensium}. Munique: 1823--1832, 3 vols., e \textit{Herbarium
Florae Brasiliensis}. Munique: 1837.} Era comum esses tratados,
memórias e relatos circularem apenas em cópias manuscritas, podendo ser
alterados pelos copistas ou plagiados por outros autores. Além das
referências a seu texto em obras posteriores, as solicitações de
Gabriel Soares e as concessões reais foram evocadas por outros súditos
que negociaram com a Coroa expedições mineradoras no ultramar. Seu
processo de petições e concessões passou, portanto, a servir de modelo
ou exemplo a outros exploradores coloniais.\footnote{ O acordo entre
Gabriel Soares de Sousa e o rei Filipe~\textsc{ii} serviu de base não apenas
às negociações de expedições em busca de pedras e metais preciosos
referentes ao Estado do Brasil, como aos pedidos feitos pelo 
governador"-geral D.~Francisco de Sousa, que havia acompanhado as
solicitações de Soares e o malogro de sua empreitada na Bahia, e até
para expedições em Angola. (Cf. Rodrigo Ricupero. \textit{Honras e Mercês.
Poder e patrimônio nos primórdios do Brasil}. Tese de doutoramento apresentada
na Faculdade de Filosofia, Letras e Ciências Humanas/\textsc{usp}, 2006, pp.\,61--64).}

\section{Sugestões de referências complementares}\label{sugestoes}

\subsection{Filmes} 
\begin{itemize}

	\item \emph{Aguirre, a cólera dos deuses}. Direção de Werner Herzog. 
	(Alemanha, 1972).

	Baseado nos diários do Frei Gaspar de Carvajal de meados do século 
	\textsc{xvi}, conta a história de uma expedição espanhola na Amazônia 
	em busca da cidade mítica de Eldorado. Em resposta à tentativa de 
	desistência do então líder do grupo, um dos	soldados, Aguirre, dá 
	início a um motim, coroa um novo rei e toma o controle da expedição. 
	O filme é um interessante documento histórico as funciona sobretudo 
	como um estudo sobre os efeitos psíquicos nos indivíduos envolvidos 
	no projeto colonizatório.

	\item \emph{Desmundo}. Direção de Alain Fresnot. (Brasil, 2003).

	A jovem Oribela chega em um navio de órfãs mandadas para desposarem os
	primeiros colonizadores. Insatisfeita com o futuro marido, ela está
	determinada a fugir.

	\item \emph{Hans Staden}. Direção de Luiz Alberto Pereira. (Brasil e 
	Portugal, 1999).

	Inspirado no livro \emph{Duas viagens ao Brasil}, o filme narra a história
	do mercenário alemão Hans Staden que, numa batalha contra os franceses,
	é aprisionado por tupinambás por quem quase é morto num ritual antropofágico.
	O filme é importante por ser consideravelmente fiel à narrativa de
	Staden, uma das mais difundidas sobre o assunto no século \textsc{xvi},
	além de conter boa parte dos diálogos em tupi antigo.

	\item \emph{O povo brasileiro}. Direção de Isa Grinspum Ferraz. (Brasil, 2000).

	Quem são os brasileiros? Que matrizes nos alimentaram? Que traços nos
	distinguem? Nesta série de documentários, o antropólogo Darcy Ribeiro
	tenta responder questões como essas.

	\end{itemize}

\subsection{Lugares para visitar}

\begin{itemize}
\item\textbf{Coleção Brasiliana, Espaço Olavo Setúbal -- Itaú Cultural, São Paulo.}

Esta coleção permanente conta com quase mil itens, entre pinturas,
mapas, manuscritos, caricaturas e tantos outros documentos que ajudam a
contar a história do Brasil.
\end{itemize}

\subsection{\emph{Sites}}

\begin{itemize}
\item Biblioteca Brasiliana

Neste portal, é possível acessar o acervo digital da \href{http://www.bbm.usp.br/}{Biblioteca
Brasiliana}. Localizada na cidade de São Paulo, é uma das maiores do
Brasil e abriga cerca de 32 mil títulos, reunidos por Guita e José
Mindlin ao longo de 80 anos.

 \item Amazônia

 Neste \href{https://amazonia.org.br}{site}, é possível achar textos acerca dos principais temas 
 relacionados à vida na Amazônia, como os direitos dos povos indígenas e
 a preservação ambiental.
\end{itemize}

\section{Bibliografia comentada}

\begin{itemize}
\item\textsc{bosi}, Alfredo. \textit{Dialética da colonização}. São Paulo: Companhia
das Letras, 1992.

Em capítulos que vão de Anchieta à indústria cultural, o consagrado
pesquisador persegue as formas históricas que enlaçaram colonização,
culto e cultura.

\item\textsc{boxer}, Charles. \textit{O império marítimo português}. Coimbra: Edições 70, 2011.

O professor traça a evolução do império marítimo português desde as
primeiras viagens, no início do século \textsc{xv}, até a independência do
Brasil.

\item\textsc{candido}, Antonio. \textit{Iniciação à literatura brasileira}. Rio de
Janeiro: Ouro Sobre Azul, 2015.

Neste breve livro, um dos maiores críticos literários do país faz um
resumo histórico da literatura brasileira das origens até o século \textsc{xx}.

\item\textsc{holanda}, Sérgio Buarque de. \textit{Visão do paraíso}. São Paulo:
Companhia das Letras, 2010.

Esta obra inaugurou o ensaísmo sobre o imaginário do colonizador e
antecipou a historiografia das mentalidades ao estudar os mitos que
acompanharam as narrativas dos descobrimentos e da colonização da
América.

\item\textsc{miranda}, Ana. \textit{Desmundo}. São Paulo: Companhia das Letras, 1996.

Neste romance, a jovem Oribela cruza o Atlântico com um grupo de outras
órfãs que se são mandadas pela rainha de Portugal para se casarem com
cristãos que aqui moravam.

\item\textsc{navarro}, Eduardo de Almeida. \textit{Método moderno de tupi~
antigo: a língua do Brasil nos primeiros séculos}. São Paulo, Global: 2005.

Junto às lições muito didáticas da língua tupi, o professor Navarro 
oferece em cada unidade um aspecto da cultura e socialização indígena, 
o que faz deste livro um importante manual para se compreender o Brasil.

\item\textsc{priore}, Mary del. \textit{Histórias da gente brasileira -- colônia}. São Paulo: Leya, 2017.

No primeiro desta série de livros, a escritora narra a história do
Brasil no período colonial não sob a ótica de reis, guerras e grandes
feitos, mas pela voz do povo, de seus costumes e tradições.

\item\textsc{queiroz}, Dinah Silveira de. \textit{A muralha}. São Paulo: Instante,
2020.

Este best"-seller narra as paixões, os desafios e a violência daqueles
que desbravaram o interior do Brasil no início do século \textsc{xviii}, dando
papel de destaque às personagens femininas.

\item\textsc{roncari}, Luiz. \textit{Literatura Brasileira: dos primeiros cronistas
aos últimos românticos}. São Paulo: Edusp, 2014.

Obra destinada ao ensino de literatura. Com uma abordagem histórica,
fornece aos estudantes o aparato crítico necessário para compreender as
produções nacionais no campo das letras.

\item\textsc{schwarcz}, Lilia Moritz; \textsc{starling}, Heloisa Murgel. \textit{Brasil: uma biografia}. São Paulo: Companhia das Letras, 2015.

Juntando rigor científico, vasta documentação original e uma poderosa
iconografia, as autoras propõem uma nova e nada convencional história do
Brasil.

\item\textsc{souza}, Laura de Mello e (org.). \textit{História da vida privada no
Brasil}, vol. 1. São Paulo: Companhia das Letras, 1997.

O primeiro dos quatro volumes dessa série focaliza detalhes do dia a dia
dos primeiros colonizadores: o que e como comiam, onde dormiam, como
namoravam etc.
\end{itemize}

\end{document}

