\documentclass[12pt]{extarticle}
\usepackage{manualdoprofessor}
\usepackage{fichatecnica}
\usepackage{lipsum,media9,graficos}
\usepackage[justification=raggedright]{caption}
\usepackage[one]{bncc}
\usepackage[piseagrama]{../edlab}


\begin{document}

\newcommand{\AutorLivro}{Gabriel Soares de Sousa}
\newcommand{\TituloLivro}{Tratado descritivo do Brasil}
\newcommand{\Tema}{Diálogos com a sociologia e com a antropologia}
\newcommand{\Genero}{Diário, biografia, autobiografia, relatos, memórias}
\newcommand{\imagemCapa}{./images/PNLD0015-01.png}
\newcommand{\issnppub}{---}
\newcommand{\issnepub}{---}
% \newcommand{\fichacatalografica}{PNLD0015-00.png}
\newcommand{\colaborador}{\textbf{Bruno Gradella, Vicente Castro} e Ana Lancman (edição).}


\title{\TituloLivro}
\author{\AutorLivro}
\def\authornotes{\colaborador}

\date{}
\maketitle

\baselineskip=1.20\baselineskip\par


\begin{abstract}
Este Manual tem como objetivo fornecer subsídios para o trabalho com a
obra literária \emph{Tratado descritivo do Brasil}, de Gabriel Soares
de Sousa.

\textbf{Gabriel Soares de Sousa} nasceu em Portugal, provavelmente na
década de 1540, e chegou ao Brasil, proveniente do Reino e a caminho das
Índias Orientais, em 1569. Não se sabe o que fez este viajante e
explorador desistir de seguir viagem com o restante da tripulação e
desembarcar no litoral da capitania da Bahia. Gabriel Soares colocou no papel 
as lembranças dos dezessete anos em que viveu no Brasil, produzindo dois
textos, \textit{Roteiro geral com largas informações de toda a costa do
Brasil} e \textit{Memorial e declaração das grandezas da Bahia de Todos
os Santos}, que enviou ao valido do rei, D.~Cristóvão de Moura, em 1º
de março de 1587.

Inéditos e anônimos até o século \textsc{xix}, quando foram recuperados, 
reunidos e publicados integralmente, com sua autoria restituída, pelo 
historiador brasileiro Francisco Adolfo de Varnhagen, os dois textos do 
\emph{Tratado descritivo do Brasil} têm despertado grande interesse 
dos estudiosos do início da colonização do Brasil e é considerada por muitos 
o mais importante texto quinhentista sobre o assunto. Além disso, é fonte 
indispensável a diferentes áreas do conhecimento, como botânica, geografia, 
história e antropologia, pois as minuciosas descrições apresentadas pelo autor 
fornecem preciosas informações a respeito da fauna, flora, acidentes geográficos, 
povos nativos e engenhos da costa do Brasil no século \textsc{XVI}, sobretudo da 
Bahia de Todos os Santos.

A obra \emph{Tratado descritivo do Brasil} é fruto do
esforço de restaurar ao leitor contemporâneo um códice alternativo de
fins do século \textsc{xvi} ou início do \textsc{xvii}. Preservado na Biblioteca Guita e
José Mindlin, o manuscrito contém a cópia dos dois textos de Gabriel
Soares de Sousa que vêm precedidos pelo traslado da carta enviada a
Cristóvão de Moura e pela breve apresentação intitulada ``Descrição do
que se contém neste caderno''. O códice traz como referência na lombada
a inscrição “\textit{Descripcion y noticias del Brasil. Sin autor. Anno 1587}”,
o que sugere tratar"-se de uma cópia espanhola.

Esperamos que as indicações propostas aqui sejam muito úteis no trabalho em
sala de aula!

\end{abstract}

\tableofcontents

\section{Atividades 1}

%\BNCC{EM13LP26}

\subsection{Pré"-leitura}

%\BNCC{EM13LGG302}
%\BNCC{EM13LGG704}
%\BNCC{EM13LP10}
%\BNCC{EM13LP19}

\paragraph{Tema} O gênero diário.

\paragraph{Conteúdo} Escrita de um diário pessoal e seleção de trechos a serem compartilhados com o coletivo.

\paragraph{Objetivo} Familiarizar os estudantes com o gênero diário e fomentar o exercício de escrita cotidiana. 

\paragraph{Justificativa} Antes de adentrar na leitura da obra, é interessante que os
alunos estejam familiarizados com o gênero diário. Na contemporaneidade,
a prática de produção de diários pessoais se inscreve na fronteira entre
os formatos analógico e digital. Convivem simultaneamente as formas de
registro manuscritas, que adotam tradicionais cadernos e agendas como
suporte, e as novas tecnologias digitais de comunicação e informação,
baseadas no uso de editores de texto e compartilhamento de dados em
blogs e sites.

O gênero textual diário pertence ao grupo das experiências que envolvem
a escrita de si, no âmbito das quais se inserem também, por exemplo, a
autobiografia e o memorial.

A escrita de um diário é uma oportunidade para conhecer a
própria personalidade e, no caso de ter acesso a diários de outras
pessoas, descobrir os gostos, interesses e vivências de colegas e
familiares. Os textos serão escritos individualmente, mas a versão
digital será compartilhada com a turma.

\paragraph{Metodologia}

\begin{enumerate}

\item Proponha aos estudantes que escrevam, ao longo de uma semana, 
um diário pessoal, com pensamentos, anotações e relatos de seu dia a dia.

\item Na aula seguinte, os alunos deverão selecionar alguns trechos de seu diário e redigir em formato digital, que poderá ser veiculado junto ao portal do colégio. Indique que os
registros manuscritos poderão incluir todas as experiências que se
desejar anotar, mas as entradas do blog deverão ser filtradas e conter
apenas informações que possam ser lidas e acessadas abertamente por
outros.

\item  Sugere"-se dar continuidade aos diários, caso os estudantes tenham interesse, com a realização de um exercício de
\emph{journaling} -- como é conhecida, em inglês, a prática de produção
frequente de diários pessoais.

\end{enumerate} 

\paragraph{Tempo estimado} Duas aulas de 50 minutos.


\subsection{Leitura}

%\BNCC{EM13LGG103}
%\BNCC{EM13LP02}
%\BNCC{EM13LP48}


\paragraph{Tema} O manuscrito e a obra.

\paragraph{Conteúdo} Debate e escrita de texto a partir da comparação entre o manuscrito original da obra de Gabriel Soares de Sousa e o livro publicado.

\paragraph{Objetivo} Proporcionar aos estudantes o contato com o manuscrito original, para que possam observar a adaptação feita para o livro e as mudanças na língua portuguesa ao longo do tempo.

\paragraph{Justificativa} No livro, observa"-se com certa frequência a descrição de detalhes das
novas terras, conjuntamente com a análise de seu potencial econômico. Nessa atividade, os estudantes poderão refletir sobre quais as intenções do autor no incentivo à exploração de terras brasileiras.

\paragraph{Metodologia}

\begin{enumerate}

\item Os estudantes devem acessar o manuscrito original da obra presente no site 
\href{https://digital.bbm.usp.br/view/?45000009023#page/1/mode/2up}%
{Biblioteca Brasiliana Guita e José Mindlin}:

\item Peça que os alunos encontrem o proêmio na primeira parte do manuscrito. Primeiramente, devem realizar a leitura do manuscrito e em seguida do trecho adaptado para o livro:

\begin{quote}\emph{Declaração do que se contém neste caderno}

Como todas as coisas têm fim, convém que tenham princípio, e como o de minha pretensão é
manifestar a grandeza, fertilidade e outras grandes partes que tem a Bahia de Todos os
Santos e demais Estados do Brasil, do que os reis passados tanto se descuidaram, a el"-rei
nosso senhor convém, e ao bem do seu serviço, que lhe mostre, por estas lembranças, os
grandes merecimentos deste seu Estado, as qualidades e estranhezas dele etc., para que lhe
ponha os olhos e bafeje com o seu poder, o qual se engrandeça e estenda a felicidade, com
que se engrandeceram todos os Estados que reinam debaixo de sua proteção, porque está
muito desamparado depois que el"-rei D.~João~\textsc{iii} passou desta vida para a
eterna,\footnote{ O rei português D. João \textsc{iii} (1502-1557), filho do rei
D. Manuel, governou entre 1521 e 1557.} o qual principiou com tanto zelo, que para o
engrandecer meteu nisso tanto cabedal, como é notório, o qual se vivera mais dez anos
deixara nele edificadas muitas cidades, vilas e fortalezas mui populosas, o que se não
efetuou depois do seu falecimento, antes se arruinaram algumas povoações que em seu tempo
se fizeram, em cujo reparo e acrescentamento estará bem empregado todo o cuidado que Sua
Majestade mandar ter deste novo reino, para se edificar nele um grande império, o
qual com pouca despesa destes reinos se fará tão soberano que seja um dos Estados do mundo
porque terá de costa mais de mil léguas, como se verá por este \textit{Tratado} no tocante
à cosmografia dele, cuja terra é quase toda muito fértil, mui sadia, fresca e lavada de
bons ares e regada de frescas e frias águas. Pela qual costa tem muitos, mui seguros e
grandes portos, para nele entrarem grandes armadas, com muita facilidade, para as quais
tem mais quantidade de madeira que nenhuma parte do mundo, e outros muitos aparelhos para
se poderem fazer.
\end{quote} 

\item Agora, proponha que os alunos façam uma tabela comparativa com as diferenças entre os termos que encontraram no texto do manuscrito e o do livro, desde a ortografia até a mudança nos nomes de pessoas e países.

\item Em seguida, proponha um debate com os estudantes sobre as possíveis intenções de Gabriel Soares de Sousa ao escrever esses relatos. Quais elementos o autor aponta nessas declarações? Para quem esses escritos se destinam?

\item Ao final, peça para que os alunos formulem um texto
opinativo a respeito dos possíveis impactos gerados na exploração das
terras brasileiras durante o período da colonização, valendo"-se, para
isso, da terminologia e método de análise do trecho lido.

\end{enumerate}
\paragraph{Tempo estimado} Quatro aulas de 50 minutos.

\SideImage{Selos comemorativos de 1987 com imagem de Gabriel Soares de Sousa (Wikipedia Commons; Domínio Público )}{PNLD0015-03.png}

\Image{Vista de Olinda, Frans Post, 1647 (Acervo digital da Biblioteca Nacional; Domínio Público )}{PNLD0015-04.png}

\subsection{Pós"-leitura}

%\BNCC{EM13LGG102}
%\BNCC{EM13LGG303}
%\BNCC{EM13LGG402}
%\BNCC{EM13LGG703}
%\BNCC{EM13LP13}
%\BNCC{EM13LP14}
%\BNCC{EM13LP28}
%\BNCC{EM13LP29}
%\BNCC{EM13LP52}

\paragraph{Tema} A importância de documentos nas decisões do poder público.

\paragraph{Conteúdo} Pesquisa de textos e materiais legais sobre petições públicas.

\paragraph{Objetivo} Ressaltar a importância da confecção de documentos por parte
dos cidadãos para participar das decisões do Estado.

\paragraph{Justificativa} Gabriel Soares de Sousa endereçou seus manuscritos 
a um dos mais influentes ministros do Rei Filipe
\textsc{ii}, D.~Cristóvão de Moura e conseguiu as
extraordinárias concessões reais para realizar sua expedição na
colônia.

Após a análise dos manuscritos originais na atividade anterior e do 
debate acerca da intenção do autor ao endereçá-los, é interessante que os alunos possam debater acerca do
efeito causado pelos documentos nas decisões do poder público.


\paragraph{Metodologia}
\begin{enumerate}

\item Sugere"-se a pesquisa de textos e materiais legais, onde conste os caminhos para
que o cidadão possa realizar um peticionamento a um órgão do Poder
Público. 

\item Com esse material em mãos, proponha que os alunos exponham o
material encontrado, abrindo o espaço para que a turma se manifeste
acerca das características do documento. A partir disso, devem os
alunos, individualmente ou em grupo, produzir uma petição, requerendo ao
Poder Público a solução para um problema da comunidade.

\item Por fim, convém apresentar aos alunos o formato de
documentos oficiais, em específico, os documentos utilizados pelo poder
público na realização de seus trâmites cotidianos.

\end{enumerate}

\paragraph{Tempo estimado} Duas aulas de 50 minutos.



\section{Atividades 2}

As obras dos viajantes quinhentistas possibilitam trabalhos
interdisciplinares e integradores de diferentes campos do saber e áreas
de conhecimento. A seguir, propomos algumas atividades que podem ser
desenvolvidas conjuntamente com professores de outras áreas.

%\BNCC{EM13CNT201}
%\BNCC{EM13CNT303}
%\BNCC{EM13CHS101}
%\BNCC{EM13CHS102}
%\BNCC{EM13CHS106}
%\BNCC{EM13CHS401}

\subsection{Pré"-leitura}

\paragraph{Tema} As gravuras e desenhos históricos de viajantes.

\paragraph{Conteúdo} Pesquisa e estudo comparativo de gravuras de viajantes do século \textsc{xvi} e desenhos do século \textsc{xix}, 
com o auxílio professores da área de ciências humanas.

\paragraph{Objetivo} Incentivar os estudantes a analisar obras artísticas com viés histórico e documental.

\paragraph{Justificativa} A análise de gravuras e desenhos históricos pode trazer informações importantes para melhor contextualizar a obra que será lida.

\paragraph{Metodologia}

\begin{enumerate}

\item Deve ser proposto aos alunos uma atividade de pesquisa,
onde cada estudante deverá procurar gravuras realizadas por viajantes,
no século \textsc{xvi}.

\item Em seguida, peça para que os alunos realizem outra pesquisa de
desenhos realizados por viajantes, mas, dessa vez referentes ao século
\textsc{xix}.

\item  Quando os alunos tiverem já montado um acervo
considerável, convém a análise dessas imagens. Junto dos professores
de ciências humanas, conduza uma investigação, com a observação de elementos comuns desenhados. 
Proponha um debate acerca das razões pelas quais algumas opções artísticas foram feitas.

\item  Realize um exercício de comparação entre as imagens. Neste momento,
convém indagar aos alunos as mudanças que percebem e por que entendem
que tais mudanças ocorreram.

\end{enumerate}

\paragraph{Tempo estimado} Duas aulas de 50 minutos.


\subsection{Leitura}

\paragraph{Tema} O monstro marinho representado através da escrita e da gravura.

\paragraph{Conteúdo} Comparação entre diferentes linguagens artísticas enquanto documentos históricos.

\paragraph{Objetivo} Trazer instrumentos para que os estudantes possam realizar uma leitura aprofundada da obra 
com uma abordagem interdisciplinar.

\paragraph{Justificativa} Após a pesquisa e estudo de gravuras realizadas por viajantes do século \textsc{xix},
os estudantes estarão mais preparados para aguçar sua percepção acerca dos relatos trazidos por Gabriel
Soares de Sousa. Nessa atividade, poderão fazer um estudo comparativo de distintas linguagens, com o auxílio de professores
da área de ciências humanas.

\paragraph{Metodologia}

\begin{enumerate}

\item Peça para os alunos lerem o seguinte trecho da obra:


\begin{quote} 
\emph{Que trata dos homens marinhos}
Não há dúvida senão que se encontram na Bahia e nos recôncavos dela muitos homens
marinhos, a que os índios chamam pela sua língua upupiara,\footnote{Upupiara, Ipupiara oi
Igpupiara. Do tupi \textit{îpupi'ara}, aquele que vive nas águas ou homem marinho.} os
quais andam pelo rio de água doce pelo tempo do verão, onde fazem muito dano aos índios
pescadores e mariscadores que andam em jangadas, onde os tomam, e aos que andam pela borda
da água, metidos nela; a uns e outros apanham, e metem"-nos debaixo da água, onde os
afogam; os quais saem à terra com a maré vazia afogados e mordidos na boca, narizes e na
sua natura; e dizem outros índios pescadores que viram tomar estes mortos, que viram sobre
água uma cabeça de homem lançar um braço fora dela e levar o morto; e os que isso viram se
recolheram fugindo à terra assombrados, do que ficaram tão atemorizados que não quiseram
tornar a pescar daí a muitos dias; o que também aconteceu a alguns negros de Guiné; os
quais fantasmas ou homens marinhos mataram por vezes cinco índios meus; e já aconteceu
tomar um monstro destes dois índios pescadores de uma jangada e levarem um, e salvar"-se
outro tão assombrado que esteve para morrer; e alguns morrem disto. E um mestre"-de"-açúcar
do meu engenho afirmou que olhando da janela do engenho que está sobre o rio, e que
gritavam umas negras, uma noite, que estavam lavando umas formas de açúcar, viu um vulto
maior que um homem à borda da água, mas que se lançou logo nela; ao qual mestre"-de"-açúcar
as negras disseram que aquele fantasma vinha para pegar nelas, e que aquele era o homem
marinho, as quais estiveram assombradas muitos dias; e destes acontecimentos acontecem
muitos no verão, que no inverno não falta nunca nenhum negro.
\end{quote}

\item Em seguida, apresente a figura \begin{comment}%número? 
\end{comment} aqui reproduzida da gravura do monstro marinho, ilustração do livro História da Província de Santa Cruz, de Gândavo.
Com o auxílio dos professores de ciências humanas, contextualize as duas obras e suas específicas históricas.


\SideImage{Gravura reproduzindo o monstro marinho de Gândavo e a forma como foi capturado.}{PNLD0015-09}

\item Peça que os estudantes
façam uma tabela comparativa com semelhanças e diferenças entre o monstro
marinho descrito na obra e a ilustração do livro de Gândavo.

\item Proponha um debate acerca da representação do \emph{índio} e do \emph{negro} no trecho lido. Os professores de ciências humanas
podem trazer uma contextualização histórica sobre as diferenças entre a escravidão de indígenas e negros nesse período.

\end{enumerate}
\paragraph{Tempo estimado} Duas aulas de 50 minutos.

\Image{Primeira edição, organizada e revisada pelo historiador brasileiro Adolfo de Varnhagen (1816-1878), 1851 (Biblioteca Brasiliana Guita e José Mindlin; Domínio Público )}{PNLD0015-05.png}

\subsection{Pós"-leitura}

\paragraph{Tema} \emph{Índio} ou \emph{indígena}?

\paragraph{Conteúdo} Debate acerca das mudanças na descrição feita sobre os povos indígenas ao longo do tempo.

\paragraph{Objetivo} Ampliar o conhecimento dos estudantes acerca do reconhecimento dos povos indígenas hoje e proporcionar uma visão crítica da descrição feita pela colônia portuguesa.

\paragraph{Justificativa} Dando continuidade à atividade anterior e com a colaboração dos professores
da área de ciências humanas, a atividade de pós-leitura pretende trazer uma maior quantidade de elementos para aprimorar a visão crítica dos estudantes.

\paragraph{Metodologia}

\begin{enumerate}

\item Os estudantes devem consultar o dicionário mais antigo da língua portuguesa, composto por Rafael Bluteau. Devem acessar o verbete
\emph{índio}, presente em \href{http://dicionarios.bbm.usp.br/pt-br/dicionario/1/indio}. Faça uma leitura conjunta desse verbete em sala de aula.

\item Em seguida, peça que procurem o verbete \emph{indígena} e faça novamente uma leitura conjunta.

\item Agora, proponha que os alunos leiam o seguinte trecho dessa entrevista realizada com Daniel Munduruku:

\begin{quote} Uma palavra muda tudo? Sim, uma palavra muda muito. Nos meus vídeos e palestras, eu tenho sempre feito uma separação fundamental entre "índio" e "indígena". As pessoas ainda pensam que índio e indígena é a mesma coisa. Não é. O próprio dicionário diz isso.

A palavra indígena diz muito mais a nosso respeito do que a palavra índio. A palavra índio gera uma imagem distorcida. Já indígena quer dizer originário, aquele que está ali antes dos outros.\end{quote}

A entrevista completa está presente no site G1, em \href{https://g1.globo.com/educacao/noticia/2019/04/19/dia-do-indio-e-data-folclorica-e-preconceituosa-diz-escritor-indigena-daniel-munduruku.ghtml}.

\item Proponha aos alunos que escrevam um texto argumentativo sobre o uso da palavra \emph{índio} ou \emph{indígena} hoje em dia. É importante que os professores de ciências humanas contribuam com uma contextualização histórica do tratamento dado aos povos indígenas ao longo do tempo e qual o espaço que autores indígenas ocupam hoje em dia.

\end{enumerate}

\paragraph{Tempo estimado} Uma aula de 50 minutos.


\section{Aprofundamento}


\begin{comment}
Ao chegar ao Ensino Médio, é necessário que os estudantes se aprofundem
na compreensão das múltiplas linguagens e, sobretudo, da linguagem
literária. Em relação à literatura, a \textsc{bncc} traz as seguintes
considerações:

\begin{quote}
{[}\ldots{}{]} a leitura do texto literário, que ocupa o centro do trabalho
no Ensino Fundamental, deve permanecer nuclear também no Ensino Médio.
Por força de certa simplificação didática, as biografias de autores, as
características de épocas, os resumos e outros gêneros artísticos
substitutivos, como o cinema e as \textsc{hq}s, têm relegado o texto literário a
um plano secundário do ensino. Assim, é importante não só (re)colocá"-lo
como ponto de partida para o trabalho com a literatura, como
intensificar seu convívio com os estudantes. Como linguagem
artisticamente organizada, a literatura enriquece nossa percepção e
nossa visão de mundo. Mediante arranjos especiais das palavras, ela cria
um universo que nos permite aumentar nossa capacidade de ver e sentir.
Nesse sentido, a literatura possibilita uma ampliação da nossa visão do
mundo, ajuda"-nos não só a ver mais, mas a colocar em questão muito do
que estamos vendo/vivenciando. (Brasil, 2018, p. 491)
\end{quote}
\end{comment}


Nesta seção, desenvolvemos um trabalho de aprofundamento que, em diálogo
com a formação continuada de professores, oferece subsídios para a
abordagem do texto literário.

O \textit{Tratado descritivo do Brasil} está dividido em dois livros: 
\textit{Roteiro geral com largas informações de toda a costa do Brasil} e \textit{Memorial e declaração das grandezas da Bahia de todos os Santos, de sua fertilidade e das notáveis partes que tem}. O primeiro livro contém um proêmio e 74 capítulos. A segunda parte, por sua vez, abrange 196 capítulos. Tanto a edição de 1851 quanto a de 1879 são finalizadas com 270 comentários e observações de Francisco Adolpho de Varnhagen.

O termo ``roteiro'' refere-se a um gênero recorrente, sobretudo na época dos descobrimentos, utilizado para descrever em detalhes uma viagem ou estabelecer uma rota ou guia para os navegantes, apontando precisamente os cabos, baixios, ilhas, portos ou rios navegáveis; enfim, tudo o que poderia servir para orientar um navegante conquistador. O memorial da Bahia faz parte de um gênero também corrente à época, o registro de lembranças.

As informações prestadas por Gabriel Soares de Sousa em seus livros empregam diversas áreas do conhecimento, tais como a náutica, a botânica, a zoologia e a memorialística, narrando as vivências, memórias e impressões do autor sobre os lugares, animais, plantas e gentes do Brasil. Gabriel tinha o poder da escrita, capaz de agradar qualquer tipo de público leitor.

É possível afirmar ainda que Gabriel, mesmo não tendo conhecido pessoalmente todos os cantos da costa brasileira, não deixou de descrevê-los em seu tratado, muito provavelmente recorrendo a outras fontes para isso. Por certo percorreu toda a circunvizinhança da cidade de Salvador, por terra e por mar. Mas, provavelmente, recorreu a outras fontes para se referir com tanta minudência ao restante do território.

Boa parte do que Gabriel Soares escreveu deve ter sido compilado de informações encontradas na própria península durante o longo período em que lá esteve, possivelmente recebendo cartas remetidas de suas fazendas. Ao que tudo indica não percorreu toda costa brasileira, do rio de Vicente Pinzon, acima do rio das Amazonas, até algumas léguas depois da Baía de São Matias, na atual Argentina. 

Na primeira parte do livro, o \emph{Roteiro geral, com largas informações de toda a costa do Brasil}, Sousa descreve, de maneira bem detalhada, tudo o que se poderia encontrar na região costeira do Brasil, do rio Amazonas, passando pelo rio São Francisco e a baía do Rio de Janeiro, até o Rio da Prata.

Conta quem foram os primeiros povoadores das capitanias e os heróis colonizadores, fala quais são as atividades e potencialidades econômicas e como é a navegabilidade de cada região costeira, onde é possível adentrar com caravelões, como são os portos, o relevo, o clima, a vegetação, os frutos, a fauna, os rios, a fertilidade do solo, a localização e condições dos povoados, engenhos, plantações e áreas com pau-brasil, onde é possível haver fluxo de pessoas e de mercadorias e, por fim, onde é possível encontrar os tão cobiçados metais e pedras preciosas.

Para ganhar a confiança do rei espanhol e mostrar credibilidade, o colono português fornece todo o tipo de informações estratégicas, coordenadas exatas de como chegar nos lugares, quais as vantagens econômicas de tudo que pode ser encontrado na colônia, bem como quais são as necessidades e perigos encontrados na mesma.

Em tom de crítica ou de conselho, dá sugestões de administração ao rei, pede mais recursos e investimentos, dizendo quais lugares necessitam de fortificação contra corsários estrangeiros, ou contra ``maus selvagens'', e quais localidades são boas para se conquistar, povoar e explorar, implantar engenhos, criar animais ou estabelecer plantações, assim como fez ao descrever o rio São Francisco.
Além disso, pelo viés econômico e exploratório, Gabriel Soares mostra o quanto a colônia não deixa nada a desejar, ou é superior, em comparação com o que pode ser encontrado em Portugal, Espanha, Índias ou nos demais territórios sob o domínio das coroas ibéricas, como ao falar das vacas criadas nas capitanias de São Vicente e Santo Amaro. 

\SideImage{A edição de 1851 e a de 1879 de <<Tratado Descritivo do Brazil em 1581>> são finalizadas com 270 comentários e observações de Francisco Adolpho de Varnhagen (Acervo digital da Biblioteca Nacional; Domínio Público )}{PNLD0015-06.png}


\Image{Viagens marítimas de Hans Staden feitas em vários momentos em 1547 de Portugal/Espanha para o Rio de Janeiro e outros lugares do Brasil. (Acervo digital da Biblioteca Nacional; Domínio Público )}{PNLD0015-08.png}

Em uma pequena apresentação que precedia o \textit{Roteiro geral} e o
\textit{Memorial}, dedicada a Sua Majestade, sob o título ``Descrição do
que se contém neste caderno'', Soares de Sousa justifica os manuscritos a
Filipe 	\textsc{ii}. Nesse breve texto, conhecido como ``Proêmio'', por assim ter
sido nomeado por Varnhagen, o autor explica, como leal vassalo, que sua
``pretensão é manifestar a grandeza, fertilidade e outras grandes partes
que têm a Bahia de Todos os Santos e demais Estados do Brasil'', levando
ao conhecimento do rei as condições em que se encontrava a colônia
portuguesa, cujas terras tinham sido deixadas em estado de abandono
pelos monarcas anteriores, justamente pela pouca notícia que dali lhes
chegava. Continua sua justificativa argumentando que ``a el''-rei nosso
senhor convém, e ao bem do seu serviço, que lhe mostre, por estas
lembranças, os grandes merecimentos deste seu Estado, as qualidades e
estranhezas dele, etc., para que lhe ponha os olhos e bafeje com seu
poder''. Gabriel Soares estava defendendo que apenas quando detentora de
tal conhecimento é que a Coroa poderia proteger aquelas possessões e
explorá"-las adequadamente.\footnote{Esse texto foi publicado pela
primeira vez por Varnhagen, precedendo o “Roteiro geral” e o
“Memorial”, na edição de 1851 e posteriormente na edição espanhola,
organizada por Cláudio Gans. O documento encontra"-se publicado na presente edição
sob o título “Descrição do que se contém neste caderno”.}

Em 1590, após pelo menos quatro anos de espera, as solicitações de
Gabriel Soares de Sousa foram atendidas por Filipe \textsc{ii}. Entre as
principais concessões reais estavam o título de capitão"-mor e
governador da conquista do Rio São Francisco, o direito de nomear seu
sucessor em caso de falecimento e a permissão para prover por três anos
todos os ofícios de justiça e de fazenda nas terras que fossem por ele
ocupadas. Além disso, o rei incentivou esse empreendimento colonial por
meio da distribuição de honras e mercês aos primeiros participantes da
expedição e da permissão ao governador"-geral do Estado do Brasil, D.
Francisco de Sousa,\footnote{ D. Francisco de Sousa (c.1540--1611),
fidalgo português, residia na corte filipina, ocasião que o fez
conhecer Gabriel Soares de Sousa, ao ser nomeado pelo rei Filipe 	\textsc{ii}
como sétimo governador"-geral do Brasil, cargo que passou a exercer, na
Bahia, a partir de 1591. Onze anos depois, retornou à Corte, onde
iniciou negociações para voltar ao Brasil em busca de metais e pedras
preciosas nas capitanias ao sul da Bahia. Permaneceu no reino até 1609,
quando foi novamente instituído de um importante cargo na colônia, o de
governador das capitanias do sul} para que cedesse duzentos índios
flecheiros à empreitada.\footnote{ Os alvarás que concedem a Gabriel
Soares de Sousa essas honras e mercês estão publicados em\textit{
Pauliceae Lusitana Monumenta Historica}. Lisboa: Real Gabinete Portugal
de Leitura do Rio de Janeiro, 1956, tomo I, p. 407 e ss.} Em posse
desses privilégios reais, Gabriel Soares organizou uma armada com cerca
de 360 homens para retornar ao Brasil em uma urca
flamenga fretada pela Fazenda Real. Partiu de Lisboa em sete de abril
de 1591, mas, antes que alcançasse seu objetivo final, a embarcação
naufragou na enseada de Vazabarris,\footnote{ Esse topônimo é derivado
da expressão portuguesa “dar em vaza"-barris”, que significava perder"-se
sem esperanças de salvação, pois aquela era uma região em que ocorriam
frequentes naufrágios.} no litoral sergipano, o que levou à morte de
alguns tripulantes e à perda de parte do armamento. O nomeado
capitão"-mor e governador da conquista, contudo, não desistiu e caminhou
até a Bahia com os sobreviventes para lá reorganizar a expedição com
auxílio de D.~Francisco de Sousa. Naquele mesmo ano de 1591 partiu para
o sertão em direção à foz do Rio São Francisco em busca das tão
sonhadas minas, mas a ele sucederia o mesmo que a seu irmão anos antes,
vindo a falecer no início da viagem. As circunstâncias de sua morte não
são claras, se causada por doença ou por vingança de índios
aprisionados; o que se sabe é que em seguida à fatalidade seus ossos
foram levados para a Capela do Mosteiro de São Bento na cidade de
Salvador. Acredita"-se que seu corpo tenha sido sepultado no lugar onde
ainda hoje se encontra, no interior da capela, uma lápide com a
inscrição ``aqui jaz um pecador''.\footnote{ Em seu
testamento, Gabriel Soares de Sousa pedia que fosse enterrado na 
capela"-mor do Mosteiro de São Bento e que sobre sua sepultura fosse colocada
aquela mesma inscrição.}

Parece, portanto, que os manuscritos \textit{Roteiro geral} e
\textit{Memorial}, endereçados a um dos mais influentes ministros de
Filipe 	\textsc{ii}, D.~Cristóvão de Moura, para que chegassem ao conhecimento do
rei, surtiram o efeito esperado por Gabriel Soares, que conseguiu as
extraordinárias concessões reais para realizar sua expedição na
colônia. E, apesar de suas ricas informações (é provavelmente a fonte
documental mais completa a respeito do primeiro século de colonização
do Brasil), não admira que esses textos não tenham sido publicados até
o século \textsc{xix}. Ainda que não fossem impressos, os escritos sobre o
ultramar dos séculos \textsc{xvi} e \textsc{xvii} não permaneceram completamente
desconhecidos dos leitores contemporâneos nem das gerações posteriores.
Muitos textos correlatos e coevos aos manuscritos de Gabriel Soares
tiveram a mesma sorte, desde a Carta de Pero Vaz de Caminha a D.~Manuel~\textsc{i}, 
datada de 1500, que só veio a ser publicada em 1817;\footnote{ A
carta de Pero Vaz ao rei permaneceu desconhecida por mais de dois
séculos, conservada no Arquivo Nacional da Torre do Tombo, em Lisboa.
Foi encontrada pelo secretário de Estado português José de Seabra da
Silva, em 1773, noticiada pelo historiador espanhol Juan Bautista Muñoz
e publicada pela primeira vez pelo padre Manuel Ayres de Cazal em sua
\textit{Corografia Brasílica}.} passando pelo
\textit{Diário de Navegação}, de Pero Lopes de Sousa, divulgado por
Varnhagen em edição de 1839;\footnote{ Cf. Pero Lopes de Sousa, \textit{Diário da navegação da armada que 
foi à terra do Brasil em 1530 sob a capitania"-mor de Martim Afonso de Sousa,
escrito por seu irmão Pero Lopes de Sousa}. Publicado por Francisco Adolfo de Varnhagen. 
Lisboa: Typ. da Sociedade Propagadora dos Conhecimentos Uteis, 1839.} e pelos textos do jesuíta Fernão Cardim
escritos entre 1583 e 1601, reunidos e publicados 
sob o título de \textit{Tratados da terra e gente do Brasil} apenas em
1925;\footnote{ Cf. Fernão Cardim, \textit{Tratados da terra e gente do Brasil}. 
Introdução e notas de Batista Caetano, Capistrano de Abreu e Rodolfo Garcia. 
Rio de Janeiro: J.~Leite e Cia., 1925. 
A obra reúne três manuscritos do autor: \textit{Do clima e terra do Brasil}, \textit{Do princípio e Origem dos índios do Brasil}
e \textit{Narrativa Epistolar}, este último havia sido publicado por Varnhagen em 1847.} até os manuscritos do frei franciscano Vicente do Salvador,
\textit{História do Brasil 1500--1627}, publicados na íntegra em 1888.\footnote{ Cf. Vicente do Salvador, \textit{História do Brasil 1500-1627}.
Introdução de Capistrano de Abreu. Rio de Janeiro: Anais da Biblioteca Nacional, vol. 13, 1888.}
Uma das poucas exceções a essa prática é a \textit{História da
província de Santa Cruz}, de autoria do gramático Pero de
Magalhães Gandavo, que foi escrita e impressa em língua portuguesa no
próprio século \textsc{xvi}.\footnote{ Cf. Pero de Magalhães Gandavo,
\textit{História da Província de Santa Cruz} \& \textit{Tratado da
Terra do Brasil}. Lisboa: Officina de Antônio Gonsalves, 1576.}

No caso do \textit{Tratado descritivo do Brasil}, o considerável
circuito de sua distribuição e consumo, ainda que sob a forma anônima
ou apócrifa, parece notório ao se verificar, por um lado, que cópias
dos manuscritos são encontradas hoje em arquivos públicos e
particulares de Portugal, Espanha, França, Inglaterra e Áustria e, por
outro, que muitos autores fizeram referências a Gabriel Soares ou a seu
texto, com autoria equivocada ou anônima, antes mesmo de sua primeira
publicação. Entre eles destacam"-se Pedro de Mariz, já no próprio século
\textsc{xvi}, Frei Vicente de Salvador, Antônio Leon Pinelo e Simão de
Vasconcelos, no século \textsc{xvii}, Frei Antônio de Santa Maria Jaboatão, no
século \textsc{xviii}, e Pedro Manuel Ayres de Cazal e Von Martius, no século
\textsc{xix}.\footnote{ Cf. Pedro de Mariz, \textit{Diálogos de Vária História}.
Coimbra: Officina de Antonio de Mariz, 1594; Frei Vicente do
Salvador,\textit{ História do Brasil} [1627]. Rio de Janeiro: Anais da
Biblioteca Nacional, vol.13, 1888; Antônio Leon Pinelo, \textit{Epítome
de la Biblioteca oriental y occidental, náutica y geográfica}. Madri:
por Juan Gonzales, 1629;  Simão de Vasconcelos, \textit{Crônica
da Companhia de Jesus no Estado do Brasil}. Lisboa: 1663, e
\textit{Notícias Curiosas e Necessárias das Cousas do Brasil}. Lisboa:
por João da Costa, 1668; Frei Antonio de Santa Maria Jaboatão,
\textit{Novo Orbe Sefárico Brasílico ou Crônica dos Frades Menores do
Brasil}. Lisboa: Officina de Antonio Vicente da Silva, 1761; 
Pedro Manuel Ayres de Cazal, \textit{Corografia Brazílica, ou Relação
Histórico"-Geográfica do Reino do Brasil}. Rio de Janeiro: Impressão
Régia, 1817; e Karl Friedrich Von Martius,
\textit{Nova Genera et Species Plantarum
Brasiliensium}. Munique: 1823--1832, 3 vols., e \textit{Herbarium
Florae Brasiliensis}. Munique: 1837.} Era comum esses tratados,
memórias e relatos circularem apenas em cópias manuscritas, podendo ser
alterados pelos copistas ou plagiados por outros autores. Além das
referências a seu texto em obras posteriores, as solicitações de
Gabriel Soares e as concessões reais foram evocadas por outros súditos
que negociaram com a Coroa expedições mineradoras no ultramar. Seu
processo de petições e concessões passou, portanto, a servir de modelo
ou exemplo a outros exploradores coloniais.\footnote{ O acordo entre
Gabriel Soares de Sousa e o rei Filipe~\textsc{ii} serviu de base não apenas
às negociações de expedições em busca de pedras e metais preciosos
referentes ao Estado do Brasil, como aos pedidos feitos pelo 
governador"-geral D. Francisco de Sousa, que havia acompanhado as
solicitações de Soares e o malogro de sua empreitada na Bahia, e até
para expedições em Angola. (Cf. Rodrigo Ricupero. \textit{Honras e Mercês.
Poder e patrimônio nos primórdios do Brasil}. Tese de doutoramento apresentada
na Faculdade de Filosofia, Letras e Ciências Humanas/\textsc{usp}, 2006, pp. 61--64).}

\section{Referências complementares}

\subsection{Livros}

\begin{itemize}
\item\textsc{miranda}, Ana. \textit{Desmundo}. São Paulo: Companhia das Letras, 1996.

Neste romance, a jovem Oribela cruza o Atlântico com um grupo de outras
órfãs que se são mandadas pela rainha de Portugal para se casarem com
cristãos que aqui moravam.

\item\textsc{priore}, Mary del. \textit{Histórias da gente brasileira -- colônia}. São Paulo: Leya, 2017.

No primeiro desta série de livros, a escritora narra a história do
Brasil no período colonial não sob a ótica de reis, guerras e grandes
feitos, mas pela voz do povo, de seus costumes e tradições.

\item\textsc{queiroz}, Dinah Silveira de. \textit{A muralha}. São Paulo: Instante,
2020.

Este best"-seller narra as paixões, os desafios e a violência daqueles
que desbravaram o interior do Brasil no início do século \textsc{xviii}, dando
papel de destaque às personagens femininas.

\item\textsc{schwarcz}, Lilia Moritz; \textsc{starling}, Heloisa Murgel. \textit{Brasil: uma biografia}. São Paulo: Companhia das Letras, 2015.

Juntando rigor científico, vasta documentação original e uma poderosa
iconografia, as autoras propõem uma nova e nada convencional história do
Brasil.
\end{itemize}

\subsection{Filmes}

\begin{itemize}
\item\textbf{Desmundo. Direção: Alain Fresnot. (Brasil, 2003).}

A jovem Oribela chega em um navio de órfãs mandadas para desposarem os
primeiros colonizadores. Insatisfeita com o futuro marido, ela está
determinada a fugir.

\item\textbf{Hans Staden. Direção: Luis Alberto Pereira (Brasil, 1999). }

Naufragado em Santa Catarina, o alemão passa a ter um escravo da tribo
Carijó. Diante de sua fuga, Staden sai a seu encalço, até ser encontrado
por índios Tupinambás, inimigos dos portugueses, que o prendem no
intuito de matá"-lo e devorá"-lo.

\item\textbf{O povo brasileiro. Direção: Isa Grinspum Ferraz (Brasil, 2000).}

Quem são os brasileiros? Que matrizes nos alimentaram? Que traços nos
distinguem? Nesta série de documentários, o antropólogo Darcy Ribeiro
tenta responder questões como essas.
\end{itemize}

\subsection{Lugar para visitar}

\begin{itemize}
\item\textbf{Coleção Brasiliana, Espaço Olavo Setúbal -- Itaú Cultural, São Paulo.}

Esta coleção permanente conta com quase mil itens, entre pinturas,
mapas, manuscritos, caricaturas e tantos outros documentos que ajudam a
contar a história do Brasil.
\end{itemize}

\subsection{Site}

\begin{itemize}
\item\url{http://www.bbm.usp.br/}.

Neste portal, é possível acessar o acervo digital da Biblioteca
Brasiliana. Localizada na cidade de São Paulo, é uma das maiores do
Brasil e abriga cerca de 32 mil títulos, reunidos por Guita e José
Mindlin ao longo de 80 anos.
\end{itemize}

\section{Bibliografia comentada}

\begin{itemize}
\item\textsc{bosi}, Alfredo. \textit{Dialética da colonização}. São Paulo: Companhia
das Letras, 1992.

Em capítulos que vão de Anchieta à indústria cultural, o consagrado
pesquisador persegue as formas históricas que enlaçaram colonização,
culto e cultura.

\item\textsc{boxer}, Charles. \textit{O império marítimo português}. Coimbra: Edições 70, 2011.

O professor traça a evolução do império marítimo português desde as
primeiras viagens, no início do século \textsc{xv}, até a independência do
Brasil.

\item\textsc{candido}, Antonio. \textit{Iniciação à literatura brasileira}. Rio de
Janeiro: Ouro Sobre Azul, 2015.

Neste breve livro, um dos maiores críticos literários do país faz um
resumo histórico da literatura brasileira das origens até o século \textsc{xx}.

\item\textsc{holanda}, Sérgio Buarque de. \textit{Visão do paraíso}. São Paulo:
Companhia das Letras, 2010.

Esta obra inaugurou o ensaísmo sobre o imaginário do colonizador e
antecipou a historiografia das mentalidades ao estudar os mitos que
acompanharam as narrativas dos descobrimentos e da colonização da
América.

\item\textsc{roncari}, Luiz. \textit{Literatura Brasileira: dos primeiros cronistas
aos últimos românticos}. São Paulo: Edusp, 2014.

Obra destinada ao ensino de literatura. Com uma abordagem histórica,
fornece aos estudantes o aparato crítico necessário para compreender as
produções nacionais no campo das letras.

\item\textsc{souza}, Laura de Mello e (org.). \textit{História da vida privada no
Brasil}, vol. 1. São Paulo: Companhia das Letras, 1997.

O primeiro dos quatro volumes dessa série focaliza detalhes do dia a dia
dos primeiros colonizadores: o que e como comiam, onde dormiam, como
namoravam etc.
\end{itemize}

\end{document}

