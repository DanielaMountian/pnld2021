\documentclass[12pt]{extarticle}
\usepackage{manualdoprofessor}
\usepackage{fichatecnica}
\usepackage{lipsum,media9,graficos}
\usepackage[justification=raggedright]{caption}
\usepackage{bncc}
\usepackage[piseagrama]{../edlab}

 

\begin{document}

\newcommand{\AutorLivro}{Gabriel Soares}
\newcommand{\TituloLivro}{Tratado descritivo do Brasil (Seleta)}
\newcommand{\Tema}{Diálogos com a sociologia e com a antropologia}
\newcommand{\Genero}{Diário, biografia, autobiografia, relatos, memórias}
% \newcommand{\imagemCapa}{PNLD0015-01.png}
\newcommand{\issnppub}{---}
\newcommand{\issnepub}{---}
% \newcommand{\fichacatalografica}{PNLD0015-00.png}
\newcommand{\colaborador}{\textbf{Bruno Gradella e Vicente Castro} é uma pessoa incrível e vai fazer um bom serviço.}

{}
\title{\TituloLivro}
\author{\AutorLivro}
\def\authornotes{\colaborador}

\date{}
\maketitle

\begin{abstract}
Este Manual tem como objetivo fornecer subsídios para o trabalho com as
obras literárias \emph{Tratado descritivo do Brasil}, de Gabriel Soares
de Souza.
\end{abstract}

\tableofcontents

	
\section{Atividades 1}

\BNCC{EM13LP26}

\subsection{Pré"-leitura}

\BNCC{EM13LGG302}
\BNCC{EM13LGG704}
\BNCC{EM13LP10}
\BNCC{EM13LP19}

Antes de adentrar na leitura da obra, é interessante que os
alunos estejam familiarizados com o gênero diário. Na contemporaneidade,
a prática de produção de diários pessoais se inscreve na fronteira entre
os formatos analógico e digital. Convivem simultaneamente as formas de
registro manuscritas, que adotam tradicionais cadernos e agendas como
suporte, e as novas tecnologias digitais de comunicação e informação,
baseadas no uso de editores de texto e compartilhamento de dados em
blogs e sites.

O gênero textual diário pertence ao grupo das experiências que envolvem
a escrita de si, no âmbito das quais se inserem também, por exemplo, a
autobiografia e o memorial.

Na presente atividade, os alunos irão redigir um diário digital, que
poderá ser veiculado junto ao portal do colégio.

Para esta atividade, sugere"-se a realização de um exercício do
\emph{journaling} -- como é conhecida, em inglês, a prática de produção
frequente de diários pessoais. É uma oportunidade para conhecer a
própria personalidade e, no caso de ter acesso a diários de outras
pessoas, descobrir os gostos, interesses e vivências de colegas e
familiares. Nesta atividade, oriente os alunos a produzirem textos sobre
seus cotidianos. Os textos serão escritos individualmente, mas a versão
digital será compartilhada com a turma. Desse modo, indique que os
registros manuscritos poderão incluir todas as experiências que se
desejar anotar, mas as entradas do blog deverão ser filtradas e conter
apenas informações que possam ser lidas e acessadas abertamente por
outros.

\subsection{Leitura}

\BNCC{EM13LGG103}
\BNCC{EM13LP02}
\BNCC{EM13LP48}

No livro, observa"-se com certa frequência a descrição de detalhes das
novas terras, conjuntamente com a análise de seu potencial econômico.
Até hoje, debate"-se muito, dentro da lógica econômica, a melhor maneira
de utilização de áreas florestadas. Para uma boa compreensão de como é
possível que uma área de mata passe a integrar um ciclo ou lógica
econômica, é necessário, de antemão, que se faça um estudo ambiental do
local. Portanto, esta atividade sugere a pesquisa de um laudo técnico
sobre o potencial de impacto ambiental, em estudo para a implantação de
uma indústria, como uma usina hidrelétrica, por exemplo em um local que
anteriormente preservava suas características naturais. A atividade
fomenta a comparação entre o conteúdo do laudo e o que o texto
desenvolve. Ao final, peça para que os alunos formulem um texto
opinativo a respeito dos possíveis impactos gerados na exploração das
terras brasileiras durante o período da colonização, valendo"-se, para
isso, da terminologia e método de análise do laudo lido.

\subsection{Pós"-leitura}

\BNCC{EM13LGG102}
\BNCC{EM13LGG303}
\BNCC{EM13LGG402}
\BNCC{EM13LGG703}
\BNCC{EM13LP13}
\BNCC{EM13LP14}
\BNCC{EM13LP28}
\BNCC{EM13LP29}
\BNCC{EM13LP52}

Por fim, convém apresentar aos alunos o formato de
documentos oficiais, em específico, os documentos utilizados pelo poder
público na realização de seus trâmites cotidianos. Feito isso, sugere"-se
a pesquisa de textos e materiais legais, onde conste os caminhos para
que o cidadão possa realizar um peticionamento a um órgão do Poder
Público. Com esse material em mãos, proponha que os alunos exponham o
material encontrado, abrindo o espaço para que a turma se manifeste
acerca das características do documento. A partir disso, devem os
alunos, individualmente ou em grupo, produzir uma petição, requerendo ao
Poder Público a solução para um problema da comunidade.

\section{Atividades 2}

As obras dos viajantes quinhentistas possibilitam trabalhos
interdisciplinares e integradores de diferentes campos do saber e áreas
de conhecimento. A seguir, propomos algumas atividades que podem ser
desenvolvidas conjuntamente com professores de outras áreas. Além das
habilidades de Linguagens e suas Tecnologias e de Língua Portuguesa,
indicadas nas etapas da seção anterior e válidas também para esta,
listamos a seguir as habilidades de outras áreas, presentes na abordagem
interdisciplinar:

\BNCC{EM13CNT201}
\BNCC{EM13CNT303}
\BNCC{EM13CHS101}
\BNCC{EM13CHS102}
\BNCC{EM13CHS106}
\BNCC{EM13CHS401}

\subsection{Pré"-leitura}

Para melhor contextualizar a obra que será lida, esta
atividade sugere que seja proposta aos alunos uma atividade de pesquisa,
onde cada estudante deverá procurar gravuras realizadas por viajantes,
no século \textsc{xvi}. Quando os alunos tiverem já montado um acervo
considerável, convém a análise dessas imagens. Conduza uma investigação,
sendo conveniente observar elementos comuns desenhados e procure
observar as razões pelas quais algumas opções artísticas foram feitas.
Posteriormente, peça para que os alunos realizem outra pesquisa de
desenhos realizados por viajantes, mas, dessa vez referentes ao século
\textsc{xix}. Realize um exercício de comparação entre as imagens. Neste momento,
convém indagar aos alunos as mudanças que percebem e por que entendem
que tais mudanças ocorreram. É possível convidar o professor de ciências
humanas para essa atividade para oferecer subsídios para a interpretação
das gravuras.

\subsection{Leitura}

Durante a atividade de leitura, sugere"-se que seja proposto ao aluno que
esse pense e escolha um alimento de seu cotidiano. Com isso em mente,
cada um dos alunos deve realizar uma pesquisa acerca desse produto
alimentício, buscando informações a respeito dele. Com auxílio dos
professores de ciências humanas e de ciências da natureza, sugere"-se que
sejam os alunos convidados a pensar em aspectos como, de que parte do
globo esse alimento é nativo? Por que ele se tornou base da alimentação
da região? Como ele é cultivado? Como é beneficiado? Quais são sua
composição e características nutricionais? Sua produção causa impacto ao
meio"-ambiente? Sua produção é majoritariamente feita por meio de
aparatos tecnológicos, ou é rudimentar? Etc. Ao Final, é possível a
construção de um quadro mural com as informações obtidas, podendo esse
ser físico ou virtual, ou mesmo a construção de um guia informativo que
pode compor o acervo da biblioteca do colégio.

\subsection{Pós"-leitura}

Na pós"-leitura, sugere"-se, já equipados com os devidos
instrumentos e conhecimentos obtidos na atividade anterior, que os
alunos procurem informações e imagens de itens do cotidiano utilizados
no período colonial. Para essa atividade, indica"-se, inclusive, a visita
a algum museu, podendo esta ser presencial ou virtual, em que se
encontrem objetos e ferramentas antigas. A partir disso, indica"-se
propor aos alunos a reflexão acerca dos elementos cotidianos, como eles
eram, como se tornaram e, eventualmente, como podem ser melhorados. Por
fim, aconselha"-se orientar os alunos a produção de uma crônica, cujo
tema deve partir dos objetos observados, devendo o aluno imaginar o
cotidiano em que eles eram e/ou são utilizados.

\section{Aprofundamento}

Ao chegar ao Ensino Médio, é necessário que os estudantes se aprofundem
na compreensão das múltiplas linguagens e, sobretudo, da linguagem
literária. Em relação à literatura, a \textsc{bncc} traz as seguintes
considerações:

\begin{quote}
{[}\ldots{}{]} a leitura do texto literário, que ocupa o centro do trabalho
no Ensino Fundamental, deve permanecer nuclear também no Ensino Médio.
Por força de certa simplificação didática, as biografias de autores, as
características de épocas, os resumos e outros gêneros artísticos
substitutivos, como o cinema e as \textsc{hq}s, têm relegado o texto literário a
um plano secundário do ensino. Assim, é importante não só (re)colocá"-lo
como ponto de partida para o trabalho com a literatura, como
intensificar seu convívio com os estudantes. Como linguagem
artisticamente organizada, a literatura enriquece nossa percepção e
nossa visão de mundo. Mediante arranjos especiais das palavras, ela cria
um universo que nos permite aumentar nossa capacidade de ver e sentir.
Nesse sentido, a literatura possibilita uma ampliação da nossa visão do
mundo, ajuda"-nos não só a ver mais, mas a colocar em questão muito do
que estamos vendo/vivenciando. (Brasil, 2018, p. 491)
\end{quote}

Nesta seção, desenvolvemos um trabalho de aprofundamento que, em diálogo
com a formação continuada de professores, oferece subsídios para a
abordagem do texto literário.

O \textit{Tratado descritivo do Brasil em 1587} está dividido em dois livros:
\textit{Roteiro geral com largas informações de toda a costa do Brasil} e
\textit{Memorial e declaração das grandezas da Bahia de Todos os Santos, de sua
fertilidade e das notáveis partes que tem}. O primeiro livro contém um
proêmio e 74 capítulos. A segunda parte, por sua vez, abrange 196
capítulos. Tanto a edição de 1851 quanto a de 1879 são finalizadas com
270 comentários e observações de francisco adolpho de varnhagen:

\begin{quote}
O termo ``roteiro'' refere"-se a um gênero recorrente, sobretudo na época
dos descobrimentos, utilizado para descrever em detalhes uma viagem ou
estabelecer uma rota ou guia para os navegantes, apontando precisamente
os cabos, baixios, ilhas, portos ou rios navegáveis; enfim, tudo o que
poderia servir para orientar um navegante conquistador. O \textit{Memorial da
Bahia} faz parte de um gênero também corrente à época, o registro de
lembranças.
\end{quote}

As informações prestadas por Gabriel em seus livros empregam diversas
áreas do conhecimento, tais como a náutica, a botânica, a zoologia e a
memorialística, narrando as vivências, memórias e impressões do autor
sobre os lugares, animais, plantas e gentes do brasil. gabriel tinha o
poder da escrita, capaz de agradar qualquer tipo de público leitor.

\begin{quote}
Sem dúvida alguma, é possível reconhecer atrativos literários em seu
discurso: uma capacidade descritiva e persuasiva capaz de conquistar
leitores especialistas ou não (\textsc{azevedo}, 2015, p 28).
\end{quote}

A pesquisadora afirma ainda que Gabriel, mesmo não tendo conhecido
pessoalmente todos os cantos da costa brasileira, não deixou de
descrevê"-los em seu tratado, muito provavelmente recorrendo a outras
fontes para isso:

\begin{quote}
Por certo percorreu toda a circunvizinhança da cidade de Salvador, por
terra e por mar. Mas, provavelmente, bebeu em outras fontes para se
referir com tanta minudência ao restante do território.
\end{quote}

Boa parte do que Soares escreveu deve ter sido compilado de informações
encontradas na própria Península durante o longo período em que lá
esteve possivelmente recebendo letras remetidas de suas fazendas. Ao que
tudo indica não percorreu toda costa brasileira, do ``rio de Vicente
Pinzon'', acima do ``rio das Amazonas'', até algumas léguas depois da
baía de são matias, na atual Argentina.

Na primeira parte do livro, o roteiro geral, com largas informações de
toda a costa do Brasil, Sousa descreve, de maneira bem detalhada, tudo o
que se poderia encontrar na região costeira do Brasil, do rio Amazonas,
passando pelo rio São Francisco e a baía do Rio de Janeiro, até o rio da
Prata.

Conta quem foram os primeiros povoadores das capitanias e os heróis
colonizadores, fala quais são as atividades e potencialidades econômicas
e como é a navegabilidade de cada região costeira, onde é possível
adentrar com caravelões, como são os portos, o relevo, o clima, a
vegetação, os frutos, a fauna, os rios, a fertilidade do solo, a
localização e condições dos povoados, engenhos, plantações e áreas com
pau"-brasil, onde é possível haver fluxo de pessoas e de mercadorias e,
por fim, onde é possível encontrar os tão cobiçados metais e pedras
preciosas

Para ganhar a confiança do rei espanhol e mostrar credibilidade, o
colono português fornece todo o tipo de informações estratégicas,
coordenadas exatas de como chegar nos lugares, quais as vantagens
econômicas de tudo que pode ser encontrado na colônia, bem como quais
são as necessidades e perigos encontrados na mesma.

Em tom de crítica ou de conselho, dá sugestões de administração ao rei,
pede mais recursos e investimentos, dizendo quais lugares necessitam de
fortificação contra corsários estrangeiros, ou contra ``maus
selvagens'', e quais localidades são boas para se conquistar, povoar e
explorar, implantar engenhos, criar animais ou estabelecer plantações,
assim como fez ao descrever o rio São Francisco.

Além disso, pelo viés econômico e exploratório, Gabriel mostra o quanto
a colônia não deixa nada a desejar, ou é superior, em comparação com o
que pode ser encontrado em Portugal, Espanha, índias ou nos demais
territórios sob o domínio das coroas ibéricas, como ao falar das vacas
criadas nas capitanias de São Vicente e Santo Amaro.

Desde o proêmio (prefácio) do livro, o colono português já deixou claro
um de seus maiores objetivos com a redação de seu tratado.
Soares de Sousa pede ao rei filipe \textsc{ii} que a colônia seja fortificada,
frisando as perdas materiais e de vidas que poderiam ser geradas por uma
iminente invasão de corsários estrangeiros, além dos enormes gastos que
o monarca teria para conseguir expulsar os invasores.

O senhor de engenho ainda alerta o rei espanhol, há poucos anos no
cargo, sobre a existência de pedras e metais preciosos a serem
explorados e chama a atenção para todas as riquezas e potencialidades do
Brasil, que teriam sido alvo de descuido por parte dos reis anteriores.

\section{Referências complementares}

\subsection{Livros}

\begin{itemize}
\item\textsc{miranda}, Ana. \textit{Desmundo}. São Paulo: Companhia das Letras, 1996.

Neste romance, a jovem Oribela cruza o Atlântico com um grupo de outras
órfãs que se são mandadas pela rainha de Portugal para se casarem com
cristãos que aqui moravam.

\item\textsc{priore}, Mary del. \textit{Histórias da gente brasileira -- colônia}. São Paulo: Leya, 2017.

No primeiro desta série de livros, a escritora narra a história do
Brasil no período colonial não sob a ótica de reis, guerras e grandes
feitos, mas pela voz do povo, de seus costumes e tradições.

\item\textsc{queiroz}, Dinah Silveira de. \textit{A muralha}. São Paulo: Instante,
2020.

Este best"-seller narra as paixões, os desafios e a violência daqueles
que desbravaram o interior do Brasil no início do século \textsc{xviii}, dando
papel de destaque às personagens femininas.

\item\textsc{schwarcz}, Lilia Moritz; \textsc{starling}, Heloisa Murgel. \textit{Brasil: uma biografia}. São Paulo: Companhia das Letras, 2015.

Juntando rigor científico, vasta documentação original e uma poderosa
iconografia, as autoras propõem uma nova e nada convencional história do
Brasil.
\end{itemize}

\subsection{Filmes}

\begin{itemize}
\item\textbf{Desmundo. Direção: Alain Fresnot. (Brasil, 2003).}

A jovem Oribela chega em um navio de órfãs mandadas para desposarem os
primeiros colonizadores. Insatisfeita com o futuro marido, ela está
determinada a fugir.

\item\textbf{Hans Staden. Direção: Luis Alberto Pereira (Brasil, 1999). }

Naufragado em Santa Catarina, o alemão passa a ter um escravo da tribo
Carijó. Diante de sua fuga, Staden sai a seu encalço, até ser encontrado
por índios Tupinambás, inimigos dos portugueses, que o prendem no
intuito de matá"-lo e devorá"-lo.

\item\textbf{O povo brasileiro. Direção: Isa Grinspum Ferraz (Brasil, 2000).}

Quem são os brasileiros? Que matrizes nos alimentaram? Que traços nos
distinguem? Nesta série de documentários, o antropólogo Darcy Ribeiro
tenta responder questões como essas.
\end{itemize}

\subsection{Lugar para visitar}

\begin{itemize}
\item\textbf{Coleção Brasiliana, Espaço Olavo Setúbal -- Itaú Cultural, São Paulo.}

Esta coleção permanente conta com quase mil itens, entre pinturas,
mapas, manuscritos, caricaturas e tantos outros documentos que ajudam a
contar a história do Brasil.
\end{itemize}

\subsection{Site}

\begin{itemize}
\item\url{http://www.bbm.usp.br/}.

Neste portal, é possível acessar o acervo digital da Biblioteca
Brasiliana. Localizada na cidade de São Paulo, é uma das maiores do
Brasil e abriga cerca de 32 mil títulos, reunidos por Guita e José
Mindlin ao longo de 80 anos.
\end{itemize}

\section{Bibliografia comentada}

\begin{itemize}
\item\textsc{bosi}, Alfredo. \textit{Dialética da colonização}. São Paulo: Companhia
das Letras, 1992.

Em capítulos que vão de Anchieta à indústria cultural, o consagrado
pesquisador persegue as formas históricas que enlaçaram colonização,
culto e cultura.

\item\textsc{boxer}, Charles. \textit{O império marítimo português}. Coimbra: Edições 70, 2011.

O professor traça a evolução do império marítimo português desde as
primeiras viagens, no início do século \textsc{xv}, até a independência do
Brasil.

\item\textsc{candido}, Antonio. \textit{Iniciação à literatura brasileira}. Rio de
Janeiro: Ouro Sobre Azul, 2015.

Neste breve livro, um dos maiores críticos literários do país faz um
resumo histórico da literatura brasileira das origens até o século \textsc{xx}.

\item\textsc{holanda}, Sérgio Buarque de. \textit{Visão do paraíso}. São Paulo:
Companhia das Letras, 2010.

Esta obra inaugurou o ensaísmo sobre o imaginário do colonizador e
antecipou a historiografia das mentalidades ao estudar os mitos que
acompanharam as narrativas dos descobrimentos e da colonização da
América.

\item\textsc{roncari}, Luiz. \textit{Literatura Brasileira: dos primeiros cronistas
aos últimos românticos}. São Paulo: Edusp, 2014.

Obra destinada ao ensino de literatura. Com uma abordagem histórica,
fornece aos estudantes o aparato crítico necessário para compreender as
produções nacionais no campo das letras.

\item\textsc{souza}, Laura de Mello e (org.). \textit{História da vida privada no
Brasil}, vol. 1. São Paulo: Companhia das Letras, 1997.

O primeiro dos quatro volumes dessa série focaliza detalhes do dia a dia
dos primeiros colonizadores: o que e como comiam, onde dormiam, como
namoravam etc.
\end{itemize}

\end{document}

