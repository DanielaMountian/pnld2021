\documentclass[12pt]{extarticle}
\usepackage{manualdoprofessor}
\usepackage{fichatecnica}
\usepackage{lipsum,media9,graficos}
\usepackage[justification=raggedright]{caption}
\usepackage[one]{bncc}
\usepackage[escoladepais]{../edlab}



\begin{document}


\newcommand{\AutorLivro}{Marcel Schwob}
\newcommand{\TituloLivro}{Vidas imaginárias e outros textos}
\newcommand{\Tema}{Ficção, mistério e fantasia}
\newcommand{\Genero}{Conto, crônica e novela}
\newcommand{\imagemCapa}{./images/PNLD0036-01.png}
\newcommand{\issnppub}{---}
\newcommand{\issnepub}{---}
% \newcommand{\fichacatalografica}{PNLD0036-00.png}
\newcommand{\colaborador}{\textbf{Clarisse Assalim, Vicente Castro e Bruno Gradella}}


\title{\TituloLivro}
\author{\AutorLivro}
\def\authornotes{\colaborador}

\date{}
\maketitle

\baselineskip=1.2\baselineskip\par


\begin{abstract}\addcontentsline{toc}{section}{Carta ao professor}

Este manual tem o objetivo de auxiliá-lo no desenvolvimento de subsídios e 
práticas pedagógicas que estabeleçam diálogo entre \emph{Vidas Imaginárias e 
outros textos}, de Marcel Schwob, e os estudantes, de modo a ampliar não 
apenas a leitura do livro, mas também a sua relação com o mundo.

\textbf{Marcel Schwob} foi um ficcionista, ensaísta e tradutor nascido em 
1867 em Chaville, região de Champagne, na França, e morto em 1905, em Paris. 
Ocupou, ao final do século XIX, um lugar de referência nos círculos literários parisienses, 
onde conviveu intimamente com escritores como Paul Claudel, 
Guy de Maupassant, Jules Renard, Alfred Jarry, 
entre outros. Com formação erudita, traduziu autores latinos como Luciano de Samósata, 
Catulo e Petrônio, mas tinha especial predileção por escritores 
de língua inglesa, como Defoe, Stevenson, Meredith e
Whitman. 

Entre suas obras mais importantes estão \textit{Le Livre de Monelle} 
(O livro de Monelle), \textit{La Croisade des enfants} (A cruzada das crianças) e 
\textit{Vies imaginaires} (Vidas imaginárias), presentes no livro que 
temos em mãos. Apresentaremos brevemente cada uma delas abaixo.

\textbf{A cruzada das crianças}, de 1896, tem como ponto de partida as crônicas 
medievais do século \textsc{xiii} sobre um grupo de crianças alemãs e francesas que teriam 
se reunido em torno de um jovem profeta para marchar rumo a Jerusalém. 
A narrativa é composta de oito relatos, que trazem pontos de vista independentes 
sobre o acontecimento. Publicada no mesmo ano, \textbf{Vidas imaginárias}, reúne 
narrativas divertidas que têm como protagonistas personagens históricos mais ou 
menos conhecidos. Schwob reconstitui à sua maneira a trajetória de filósofos, 
escritores, escravos, soldados, piratas e criminosos, seja a partir de biografias 
já existentes, documentação histórica, ou fontes literárias.

Já \textbf{O livro de Monelle}, de 1894, é uma das obras mais conhecidas de 
Schwob e foi várias vezes adaptado para teatro e rádio. Além disso fascinou, 
à época de sua primeira publicação, nomes como Mallarmé e Anatole France. 
O autor cria uma mistura sutil de gêneros: conto, poema em prosa e texto 
profético sob forma de versículos. Cada uma de 
suas partes se organiza em fragmentos unidos, no entanto, por um fio 
condutor: Monelle, misterioso personagem feminino.

Não poderíamos deixar de citar, por fim, a intersecção tênue entre literatura 
e realidade em seus textos. Escritor de ficção, Schwob cria em sua obra uma 
dimensão intrinsecamente inconclusiva — e, sobretudo, imaginativa. Em suma, 
ele não vê a literatura como lugar de eleição e partilha, mas como um horizonte 
em que as insignificâncias da vida dos homens — suas ``esquisitices'' —, vindas 
da imaginação do escritor, podem existir em toda a sua misteriosa efemeridade, 
estimulando, por sua vez, a imaginação do leitor.

Aproveite bastante este material. Ele foi feito com muita dedicação e
carinho para você! Boa aula!


\end{abstract}


\tableofcontents

\section{Proposta de atividades I}

\subsection{Pré"-Leitura}

 \paragraph{Tema} A ficção a partir da realidade.

 \paragraph{Conteúdo} Compreensão dos principais elementos de um texto
 anedótico e exemplos deste recurso no decorrer da história da literatura.

 \paragraph{Objetivo} Introduzir os alunos à questão da escrita ficcional a partir
 do recurso clássico da anedota utilizado pelo autor do livro por meio de pesquisa
 feita em sala de aula.

 \paragraph{Justificativa} Estudante dos autores clássicos greco"-latinos que foi,
 é bem provável que Schwob conhecesse os recursos retóricos comuns à época clássica.
 A anedota é um destes. Em poucas palavras, trata"-se de partir de dados reais de 
 uma personagem célebre, como ações praticadas por elas ou frases ditas e, daí, 
 desenvolver uma história, uma anedota acerca desta pessoa. Exercício básico
 presente nos livros de retórica, foi por muito tempo um lugar comum 
 na formação escolar durante a Antiguidade clássica e em outros momentos inspirados por ela. 
 Praticar este exercício é, por isso, uma forma de se introduzir à tradição que deu origem ao
 presente livro.

 \paragraph{Metodologia}
 	\begin{enumerate}
	\item
	Proponha aos alunos uma pesquisa em fontes abertas e confiáveis a respeito de biografias
	imaginárias existentes na literatura, como por exemplo o livro \emph{Cidades invisíveis} 
	(1972), escrito por Ítalo Calvino, outro conhecedor dos aparatos utilizados pelos clássicos
	greco"-latinos. Nesta obra, Calvino extrapola os fatos possíveis 
	e imagina um diálogo fantástico entre Marco Polo, ``o maior viajante de todos os tempos'', 
	e Kublai Khan, o famoso imperador dos tártaros. A história se desenrola a partir 
	dos relatos de viagens do explorador veneziano através dos domínios do 
	Grande Khan, uma espécie de ``telescópio'' que reporta as maravilhas de seu império.
	Segue um trecho da abertura da obra que pode ser mostrado aos alunos:

``Não se sabe se Kublai Khan acredita em tudo o que diz Marco Polo quando este lhe descreve as cidades visitadas em suas missões diplomáticas, mas o imperador dos tártaros certamente continua a ouvir o jovem veneziano com maior curiosidade e atenção do que a qualquer outro de seus enviados ou exploradores.''

	\item
	A partir da pesquisa, peça aos estudantes para produzirem um texto similar, 
	sobre um local ou situação de escolha própria, descrito 
	através de elementos parecidos com os que Ítalo Calvino usa 
	em \emph{Cidades invisíveis}: subjetivos, imaginários ou poéticos.
	\item
	Uma vez finalizados os textos, os alunos devem publicá"-los num jornal virtual
	criados por eles mesmo para divulgação com o restante da comunidade interna e 
	externa à escola.
 	\end{enumerate}

 \paragraph{Tempo estimado} Duas a três aulas de 50 minutos.

\SideImage{Retrato de Marcel Schwob em 1902. (Paul Boyer; Domínio Público)}{PNLD0036-03.png}


\subsection{Leitura}

 \paragraph{Tema} O universo medieval em ``A cruzada das crianças''.

 \paragraph{Conteúdo} Compreensão das principais características do universo medieval
 das cruzadas e produção de texto ficcional a partir do recurso da anedota.

 \paragraph{Objetivo} Proporcionar uma contextualização acerca da Idade Média e
 uma comparação com a atual realidade vivida no mundo através de uma produção 
 de texto ficcional.

 \paragraph{Justificativa} Por meio da leitura dos relatos de ``A cruzada das crianças''
 é possível acessar o panorama das cruzadas acontecidas durante a Idade Média europeia.
 Uma das fortes características deste período, junto às guerras, foram as epidemias de 
 doenças que devastaram populações. A ideia é que, a partir da leitura de ``Relato 
 de um leproso'', os alunos produzam um texto sob as premissas anedóticas acerca
 da atual pandemia de covid"-19 utilizando dados e personagens reais para criar uma 
 história ficcional.

 \paragraph{Metodologia}
 	\begin{enumerate}
 		\item
 		Na primeira parte da aula, é interessante que se aprofunde a discussão sobre 
 		os limites entre história e ficção. Leia, em sala se possível, os três
 		primeiros relatos do livro. No decorrer da leitura, estime um questionário
 		que deve ser respondido pelos alunos com os seguintes exemplos de perguntas: 

 		``Que histórias vocês conhecem sobre as cruzadas? Que personagens''?; ``Como a Igreja vê as cruzada?'';	``Quem eram os inimigos? Cristãos e muçulmanos realmente se conheciam?''; ``Como um goliardo enxergaria as cruzadas?'';	``Quem luta em uma guerra pela fé é sempre um fanático?'' 

 		Para mais informações acerca do tema, leia os tópicos apontados nesse manual, na seção ``Aprofundamentos''. 
 	
 		\item
 		Num segundo momento, à partir da leitura de ``Relato de um leproso'' e, retomando os 
 		conceitos de anedota vistos na aula anterior,
 		os alunos devem criar uma história ficcional com base na realidade vivida
 		no Brasil contemporâneo atingido pela pandemia mundial de covid"-19.
 		É importante ressaltar os cuidados que se deve ter ao se tratar de um tema
 		tão caro e sensível à vida humana. A boa produção, portanto, será aquela que 
 		trabalhe com a estrutura anedótica sem desrespeitar os valores básicos.

 	\end{enumerate}
 \paragraph{Tempo estimado} Três a quatro aulas de 50 minutos.

\Image{``A cruzada das crianças'', na ilustração de Gustave Dorée, represenda a mítica cruzada que inspirou Schwob na escrita do presente livro. (Domínio público.)}{PNLD0036-12}

\subsection{Pós"-Leitura}

 \paragraph{Tema} Relatando os relatos.

 \paragraph{Conteúdo} Criação de \textit{podcast} com narração das histórias do livro.

 \paragraph{Objetivo} Instigar a adaptação de uma obra de arte em outras linguagens, neste
 caso, a contação de histórias e criação de trilhas sonoras. 

 \paragraph{Justificativa} A difusão oral de histórias é uma característica de
 muitas sociedades no mundo inteiro. Na Europa medieval, na Antiguidade clássica,
 no passado do oeste africano, e mesmo no Nordeste brasileiro, narrativas eram, e em alguns
 casos ainda são difundidas acompanhadas de música. 
 É no encalço destas tradições que propomos uma leitura acompanhada de música
 das histórias presentes no livro com a criação final de um \textit{podcast} reunindo todas
 as leituras, de modo que a tradição e a inovação estejam ligadas.   

 \paragraph{Metodologia}
 	\begin{enumerate}
 		\item
 		Após a leitura do livro, os alunos devem dividir"-se em grupos 
 		e escolher um capítulo para fazer uma adaptação audiovisual que nada
 		mais será que uma narração da história com uma trilha sonora de fundo. 
 		Sugira capítulos onde haja muitos personagens para que as diferentes vozes 
 		do grupo possam ser utilizadas, como ``Relato de três criancinhas'', mas
 		lembre que não necessariamente todos precisam ficar responsáveis pelas
 		mesmas tarefas. Pelo contrário, a narração pode ser feita por uma
 		só pessoa enquanto as demais podem ficar responsáveis, por exemplo,
 		por: pesquisa de trilha sonora, edição do áudio etc. 
 		\item
 		Para instigar a criatividade dos grupos, o professor pode exibir, antes
 		da atividade, os seguintes exemplos, uns mais próximos da música e do lirismo,
 		outros mais focados no enredo das histórias:
 		- \href{https://www.youtube.com/watch?v=1M_VcqHoQ1M}{Lírica galaico"-portuguesa. Cantiga de amigo.}
 		- \href{https://www.youtube.com/watch?v=wLeI9KlGPq4}{(``O cabrito e o lobo''. Contadores de
 		História (TV Cultura)}
 		- \href{https://www.youtube.com/watch?v=ZrqRLzsp9tA&t=1429s}{Leitura dramática de ``A glória
 		e seu cortejo de horrores''. (Teatro Oficina)}
 		- \href{https://www.youtube.com/watch?v=MKLYfnAiIUE}{Vovó Cici conta o mito de Omolu.}
 		- \href{https://www.youtube.com/watch?v=4ldSot7OLeU}{``A chegada de Lampião no Inferno'', de José Pacheco.}
 		\item
 		Quando os materiais estiverem finalizados, deve ser criado um \textit{podcast} reunindo
 		todas as produções a ser divulgado nas redes sociais da escola e dos estudantes. 
 		É importante que, nesta etapa de finalização e divulgação, o professor propicie
 		uma conversa sobre a interação entre o passado e o presente tendo como motivo
 		a retomada de uma prática milenar num meio digital totalmente contemporâneo. 
 		Também poder ser levantadas questões sobre ficção e realidade, no sentido 
 		de que a cada vez que uma história é contada ela ganha nuances próprias
 		que não haviam antes graças àquele que a conta.
 	\end{enumerate}

 \paragraph{Tempo estimado} Três a quatro aulas de 50 minutos. 

\Image{Os trovadores, na Idade Média europeia, bem como os aedos na Grécia antiga, os griôs
no África ocidental e os repentistas no Nordeste brasileiros têm comum a arte de contar histórias acompanhados de música. (Autor desconhecido. Domínio público.)}{PNLD0036-13}

\section{Proposta de atividades II}


\subsection{Pré"-Leitura}

 \paragraph{Tema} As doenças na Idade Média.

 \paragraph{Conteúdo} Compreensão da realidade pandêmica vivida na Idade
 Média em decorrência da lepra e outras doenças contagiosas.

 \paragraph{Objetivo} Ambientar os alunos acerca do contexto histórico"-social 
 medieval por meio de uma pesquisa sobre as doenças mais comuns então.

 \paragraph{Justificativa} A ideia de uma pandemia era algo distante para
 muitos entre nós, ficando restrita a um passado quase remoto como a gripe
 espanhola no começo do último século ou as grandes pestes da Idade Média. 
 Agora, vivendo no contexto da pandemia global de covid"-19, retornar o
 olhar acerca destes outros exemplos na história do ocidente devem proporcionar 
 uma abordagem em perspectiva aos alunos sobre a realidade das personagens 
 do livro bem como as suas próprias.

 \paragraph{Metodologia}
 	\begin{enumerate}
 		\item
 		Com o auxílio de professores de Ciências Naturais e Ciências Humanas,
 		peça que os alunos, separados em grupos, realizem uma pesquisa em fontes confiáveis
 		e abertas a respeito das doenças mais mortíferas da Idade Média europeia. 
 		Hoje temos conhecimento de detalhes a respeito delas, como: origem, modo
 		de transmissão, taxa de mortalidade, prevenção etc., mas à época, estas
 		doenças eram vistas como sinais de castigos divinos, o que garantiu, dentre outros
 		motivos, a alcunha de ``Idade das Trevas'' a este período. Os alunos devem 
 		buscar responder perguntas científicas acerca destas doenças como : ``Por que
 		a taxa de mortalidade era tão alta?'', ``Como eram transmitidas?'', ``Os doentes
 		recebiam algum tipo de cuidado especial?''. Um grupo em específico deve
 		ficar responsável por responder às mesmas questões mas tratando da pandemia
 		de covid"-19, de modo que possam ficar claras as relações diretas de semelhança
 		e diferença entre os casos.
 		\item
 		Após coletados, os resultados devem ser organizados e apresentados em forma 
 		de \textbf{seminários} divididos por temas. Para as apresentações, podem usadas
 		projeções de imagens do período que representem médicos e doentes, por exemplo,
 		além de iconografias religiosas sobre o tema.
 	\end{enumerate}

 \paragraph{Tempo estimado} Três a quatro aulas de 50 minutos.

\Image{``O triunfo da morte'' (1562), de Pieter Bruegel, representa a devastação causada pelas
guerras e pela doença na Idade Média europeia. (Museu do Prado de Madri; Domínio Público)}{PNLD0036-11}.


\subsection{Leitura}

 \paragraph{Tema} Ciências e literatura de mãos dadas.

 \paragraph{Conteúdo} Leitura de ``Pocahontas princesa'' e produção de texto
 utilizando recursos das Ciências Naturais e Humanas para descrever cenários.

 \paragraph{Objetivo} Habilitar os alunos a articular conhecimentos de áreas
 distintas do conhecimento para construir uma história ficcional baseda
 em dados científicos.

 \paragraph{Justificativa} Bem como o autor fez em sua obra, aqui os alunos
 serão convidados a criar uma história partindo de dados da realidade que
 serão ficcionalizados por eles. Neste caso, estes dados, além de personagens
 históricas, devem ser tirados das Ciências Naturais e Humanas, como a descrição de
 relevos e paisagens, ou da fauna e flora local do espaço onde se passará a narrativa.

 \paragraph{Metodologia}
 	\begin{enumerate}
 		\item
 		Retomando a introdução acerca do recurso ficcional utilizado pelo
 		autor, que se vale de dados reais para produzir ficções, visto na história 
 		de Pocahontas lida pela turma, os alunos devem escolher uma personagem 
 		histórica para criar uma história. 
 		Para a criação desta história, deve ser feita uma pesquisa não só sobre
 		a personagem, em livros de História, mas também das características 
 		naturais da paisagem onde ela viveu e onde se desenrolará o enredo.
 		Neste caso, deve"-se usar fontes abertas e confiáveis para ambas as pesquisas.
 		O professor deve ressaltar a importância desta pesquisa visto que são os dados
 		recolhidos que garantirão o contato com a realidade da obra.
 		\item
 		Uma vez terminada a produção dos textos, estes devem ser publicados 
 		num jornal ou blog virtual da escola para a divulgação com a comunidade.
 	\end{enumerate}

 \paragraph{Tempo estimado} Três a quatro aulas de 50 minutos.


\subsection{Pós"-Leitura}

 \paragraph{Tema} Museu na escola. 

 \paragraph{Conteúdo} Criação de acervo cultural e iconográfico.

 \paragraph{Objetivo} Organizar um acervo dividido em eixos temáticos acerca da Idade Média.

 \paragraph{Justificativa} Obras de arte são formas de conhecimento acerca da realidade. 
 Não só a literatura, mas também as artes visuais desempenham este papel. Dito isto,
 sugerimos a atuação do professor de Artes no encabeçamento de uma curadoria por 
 parte dos alunos de obras que representem o período histórico da Idade Média a fim
 de construir uma galeria virtual e física de iconografias a este respeito. 

 \paragraph{Metodologia}
 	
 		\item
 		Com o auxílio do professor de Artes, a turma deve se dividir em grupos
 		que ficarão responsáveis por assuntos distintos, como: vida quotidiana, 
 		cruzadas, doenças, religiosidade etc. em obras de arte que representem a Idade Média. 
 		As pesquisas curatoriais podem ser feitas em fontes como a indicada na 
 		sessão \textbf{Referências complementares}, o \href{https://artsandculture.google.com/partner/national-museum-of-ancient-art}{Museu Nacional de Arte Antiga.} 
 		É interessante buscar itens de diferentes territórios (Península
		Ibérica, Ilhas Britânicas, França, Europa Centro"-Oriental, Bizâncio,
		Mundo Muçulmano, Escandinávia) e diversos períodos, pois isso tornará
		claro que o conceito do belo é mutante. Através da diversidade estética,
		perceber-se-á que o conceito do belo é construído e reconstruído ao
		longo dos tempos.
		Além da escolha de obras de artes antigas, o acesso a este universo 
		pode trazer à tona a percepção de elementos da vida quotidiana no
		mundo contemporâneo que fazem referência ao mundo medieval. Neste caso,
		pode ser criada uma sessão específica para isto na exposição.
		Os alunos podem utilizar as músicas selecionadas para a trilha sonora 
		do \textit{podcast} nas atividades anteriores para a ambientação do espaço da exposição. 

		Na ocasião da abertura da exposição, é interessante que o professor junte 
		os alunos em roda, e proponha a reflexão do porquê o acervo ter sido organizado 
		daquela ou dessa maneira. E se seria possível realizar outra organização. Isso visa mostrar que,
		mesmo em se trabalhando com objetos que remetem ao passado, o manuseio
		dos mesmos no presente influencia a forma como nós os compreendemos.
 
 \paragraph{Tempo estimado} Quatro aulas de 50 minutos.


\section{Aprofundamento}

Ao chegar ao Ensino Médio, é necessário que os estudantes se aprofundem
na compreensão das múltiplas linguagens e, sobretudo, da linguagem
literária. Em relação à literatura, a BNCC traz as seguintes
considerações:

\begin{quote}
``{[}...{]} a leitura do texto literário, que ocupa o centro do trabalho
no Ensino Fundamental, deve permanecer nuclear também no Ensino Médio.
Por força de certa simplificação didática, as biografias de autores, as
características de épocas, os resumos e outros gêneros artísticos
substitutivos, como o cinema e as HQs, têm relegado o texto literário a
um plano secundário do ensino. Assim, é importante não só (re)colocá-lo
como ponto de partida para o trabalho com a literatura, como
intensificar seu convívio com os estudantes. Como linguagem
artisticamente organizada, a literatura enriquece nossa percepção e
nossa visão de mundo. Mediante arranjos especiais das palavras, ela cria
um universo que nos permite aumentar nossa capacidade de ver e sentir.
Nesse sentido, a literatura possibilita uma ampliação da nossa visão do
mundo, ajuda-nos não só a ver mais, mas a colocar em questão muito do
que estamos vendo/vivenciando.'' (Brasil, 2018, p. 491)
\end{quote}

Nesta seção, desenvolvemos um trabalho de aprofundamento que, em diálogo
com a formação continuada de professores, oferece subsídios para a
abordagem do texto literário. A leitura em sala de aula de \emph{A
Cruzada das Crianças} e \emph{Vidas Imaginárias} pode ser enriquecida
pelo aprofundamento no universo literário em que a obra está inserida.

\subsection{A obra}

O livro reúne duas obras de Marcel Schwob: \emph{A Cruzada das Crianças}
e \emph{Vidas Imaginárias}, ambas de 1896, e refletem o brilhantismo de
seu autor.

A Cruzada das Crianças narra, por meio de oito relatos diferentes -- e
às vezes contraditórios -, a cruzada das crianças em direção a
Jerusalém, em 1212. Repleta de intertextualidades e mesclando fatos
reais com lendas decorrentes das narrativas orais, o autor cria uma obra
singular: uma espécie de memorial fictício, em que um goliardo, dois
Papas (Inocêncio III e Gregório IX), um leproso, um muçulmano, um
escrevente e algumas das próprias crianças contam suas versões da famosa
história das crianças que saíram da França e da Alemanha, das quais,
após terem embarcado em sete navios rumo à Terra Santa, nunca mais se
teve notícias.

Vidas Imaginárias apresenta biografias de personagens que, embora tenham
tido participação importante em vários episódios da História, foram dela
esquecidos. No prefácio da obra, Schwob avisa que a verdadeira biografia
não se deve restringir aos grandes feitos das pessoas, mas às suas
características pessoais, às suas ``anomalias'': ``As ideias dos grandes
homens são patrimônio da humanidade: cada um deles só possui de fato as
próprias esquisitices'' (p. 48). Assim, Schwob reconstrói a história de
vida desses personagens à margem da História, ressaltando em cada um
deles suas características particulares, humanizando-os. Dentre os
personagens biografados, destacam-se: o filósofo Crates, adepto de uma
corrente filosófica chamada cinismo, que abandonou uma vida de fartura
para viver nas ruas com seu cão; Nicolas Loyseleur, o juiz manipulador
que julgou Joana D'Arc; Petrônio, escritor romano, autor da obra
Satiricon, a quem dá um destino diferente do que teve com o Imperador
Nero; Cecco Angiolieri, poeta rancoroso, que, à semelhança de Antonio
Salieri (cuja obra foi ofuscada pela genialidade de seu contemporâneo,
Mozart) ficou à margem, ofuscado pelo brilho de Dante Alighieri.

\subsection{O autor}

\Image{Publicação de 1927 de Marcel Schwob, com seus escritos da juventude. (Gallica; Domínio Público)}{PNLD0036-10.png}


Nascido na França, em 23 de agosto 1867, Marcel Schwob ainda criança já revela
seus dotes literários, tendo, aos 11 anos, publicado seu primeiro
artigo: uma crítica à obra ``A Fifteen Year Old Captain'', de Júlio Verne,
publicada no jornal O Farol do Loire (Le Phare de la Loire), pertencente
a seu pai, George Schwob. Em sua adolescência, estuda latim, inglês e
alemão, o que lhe facilita o ingresso na Escola de Estudos Avançados em
Ciências Sociais (École des Hautes Etudes Sociales) e na Sorbonne, tendo
sido aluno de Ferdinand de Saussure (cujos estudos promoveram o
desenvolvimento da Linguística enquanto ciência). Doutora"-se em
Filologia Clássica e Línguas Orientais, mas, em 1889, desiste da
carreira acadêmica para se dedicar ao jornalismo e à literatura. A
partir daí publica várias obras, dentre as quais destacam"-se seis
coletâneas de contos: Cœur double (Coração Duplo, 1891), Le Roi au
masque d'or (O Rei na Máscara de Ouro, 1892), Mimes (1893), Le Livre de
Monelle (O Livro de Monelle, 1894), La Croisade des Enfants (A Cruzada
das Crianças, 1896) e Vies imaginaires (Vidas Imaginárias, 1896).




Além de exímio escritor, Schwob traduz algumas obras de peso, como Moll
Flanders, de Daniel Defoe (1895), Hamlet (1900) e Macbeth (1902), ambas
de Shakespeare.

É reconhecido por grandes escritores da época, como Paul Valéry (que lhe
dedica duas de suas obras: Introduction à la Méthode de Léonard de Vinci
e La Soirée avec M. Teste), Alfred Jarry (que lhe dedica a aclamada peça
Ubu Roi) e Oscar Wilde (que lhe dedica seu longo poema The Sphinx, em
``amizade e consideração''). Em 1936, o famoso escritor argentino Jorge
Luis Borges revela que seu livro Historia universal de la infâmia foi
inspirado nas Vidas Imaginárias de Schwob.


\Image{Retrato do autor desenhado por Theodore Simson em 1905, ano de seu falecimento. (Theodore Frederick Spicer Simson; Domínio Público)}{PNLD0036-04.png}


Além de dominar a arte literária, é considerado um erudito, e seu
profundo conhecimento, tanto da cultura greco-latina quanto da Idade
Média, permite-lhe criar obras que mesclam literatura e História.

De saúde bastante frágil, tendo passado por várias cirurgias, Marcel
Schwob morre prematuramente, aos 37 anos.

\subsection{Por que ler a obra?}

O talento de Marcel Schwob pode ser apreciado nas duas obras que
constituem esse volume: A Cruzada das Crianças, que transforma um fato
histórico em fato literário, e Vidas Imaginárias, que desnuda o traço
humano de personagens que ficaram à margem da História.

\subsection{A cruzada das crianças}


\subsubsection{O que a história diz sobre as cruzadas}

Não é nosso propósito aqui realizar um estudo sobre as cruzadas, de
forma que o assunto será abordado de forma rasa, a fim de que possamos
situar no tempo e no espaço a Cruzada a que o texto de Schwob se refere.

As cruzadas foram expedições militares, organizadas com a bênção da
Igreja Católica, durante parte da Idade Média (do final do século XI à
metade do século XIII), com o objetivo inicial de retomar Jerusalém, sob
domínio muçulmano. Evidentemente havia outros interesses por trás da
difusão do cristianismo, como a reabertura de rotas de comércio do Mar
Mediterrâneo, controlada pelos árabes.


\Image{Mapa da rota das Cruzadas, expedições militares organizadas com a benção da Igreja Católica. (Rowanwindwhistler; CC-BY-SA-3.0)}{PNLD0036-05.png}


Assim, embora As Cruzadas não tenham sido uma atividade exclusivamente
religiosa, tiveram como pano de fundo o espírito da cristandade, numa
época em que a fé se sobrepõe à razão e em que a Igreja dita as regras
de comportamento das pessoas.

A História registra a existência de várias Cruzadas, empreendidas tanto
por nobres como pelo povo, sendo a primeira a empreendida entre 1096 e
1099, sob o papado de Urbano II. Dentre todas elas, a chamada Cruzada
das Crianças mistura fatos reais com imaginários, decorrentes da ideia
fantasiosa de que o Santo Sepulcro só poderia ser reconquistado pelas
crianças, por serem elas isentas de pecado.


\Image{Manuscrito medieval de 1474 que buscou representar a Cruzada Popular de Pedro, O Eremita, de 1096 -- movimento extraoficial da Primeira Cruzada (Biblioteca Nacional da França; Domínio Público)}{PNLD0036-06.png}


Relatos de época (entre eles Roger Bacon e Vincent de Beauvais) citam
que, em 1212, um adolescente, de nome Estevão, esteve em presença do rei
da França, Felipe Augusto, para entregar-lhe uma carta em que contava
ter recebido a visita de Jesus Cristo, pedindo-lhe que organizasse uma
expedição só de crianças à Terra Santa, para expulsar de lá os
muçulmanos. O rei, após consultar-se com teólogos da Universidade de
Paris, convence-o a voltar para casa. Os cronistas assinalam também
outro movimento semelhante na Alemanha, encabeçado por um certo
Nicolau.

Sobre o destino desses movimentos, os relatos divergem bastante: o
cronista Alberic de Troisfontaines diz que as crianças partiram de Paris
e chegaram até Marselha, onde, conduzidos por Hugo Ferreus e Guillielmus
Porcus, embarcaram para Jerusalém em sete navios. Dois desses navios
naufragaram numa tempestade na ilha de Sint-Pierre, onde, mais tarde, o
Papa Gregório IX, mandou construir uma capela em homenagem aos afogados.
Os outros navios teriam chegado até Alexandria, onde as crianças
sobreviventes foram escravizadas. Outros cronistas dizem que a coluna
alemã atravessou os Alpes, chegando a Gênova. Sem terem conseguido
embarcar, alguns deles foram a Roma e outros tentaram regressar à
Alemanha, perecendo de fome e frio ou sendo atacados pelos moradores
locais.

Mesmo havendo tantas versões, citando números que chegam ao absurdo de
30 mil participantes (entre crianças e adolescentes), uma coisa é certa:
``a hostilidade do clero, das populações, e o espanto dos cronistas
vinham demonstrar o anacronismo dessas irrupções espontâneas nos quadros
de uma sociedade em vias de estabilização e enquadramento''. Cidades e
reinos não toleravam esses movimentos, como Gênova, que fez o grupo de
Nicolau abandonar a cidade, por medo de que essa multidão provocasse
desordens.

\subsubsection{A cruzada de Schwob}

A Cruzada das Crianças é uma história curta que, embora possa ser lida
``numa sentada só'', apresenta inúmeras referências, imagens e
intertextualidades, exigindo, no fundo, uma leitura atenta, a fim de que
não se percam esses elementos.

Importa lembrar que Marcel Schwob é filólogo e, por isso, para ele, as
fontes primárias (os manuscritos ou os documentos que garantem a
veracidade dos fatos), são o ponto de partida para qualquer trabalho de
cunho histórico. Em outras palavras, o que torna a obra As Cruzadas das
Crianças tão espetacular é a capacidade de seu autor dar, a fatos
históricos rigorosamente pesquisados, o caráter de obra literária, cujo
objetivo não é simplesmente nos informar um acontecimento, mas, a partir
dele, provocar emoção estética. Assim, Schwob entrelaça história e
ficção, num vaivém entre personagens reais e personagens fictícios.

A história é composta de relatos feitos por vários personagens: uns que
são observadores (um goliardo, um leproso, dois Papas, François
Longuejoue, escrevente, e um muçulmano), outros que são participantes da
expedição (Nicolas, Alain, Denis e a pequena Allys).

Em todos esses relatos, chama a atenção a presença comum e
constantemente reiterada de dois elementos: a ignorância (``nada sei'')
e a cor branca, ambos símbolos de pureza e inocência. Essa é a condição
para se chegar a Deus.

Outro fator importante de mencionar é o fato de os relatos serem feitos
no presente, conferindo à narrativa um caráter de veracidade. Os
personagens falam para o público leitor e para ninguém em particular. Na
verdade, é mais do que isso: os personagens falam para si mesmos, para
Deus, ou, como no caso do Papa Gregório IX, para o mar. É introspectivo
e externo, tem penetração psicológica, ao mesmo tempo em que vê
criticamente o fanatismo da sociedade medieval, por meio de personagens
diversos.

A seguir, apresenta-se a análise de alguns desses relatos, como forma de
convite a que se analisem os demais, deixando aqui algumas provocações:
como um goliardo enxerga esse evento, fruto de uma fé inabalável e de um
extremo fanatismo? Qual o recorte que o muçulmano faz de um evento que
se propõe à ``libertação'' da Terra Santa dos próprios muçulmanos? Qual
o efeito do contato do leproso com as crianças? O Papa Gregório IX tem o
mesmo posicionamento que seu antecessor, Inocêncio III? Como a Igreja vê
essa Cruzada? E, por fim, pode-se dizer que cada um dos
personagens-relatores tem uma visão diferente do que seja infância?

\subsubsection{O relato das três criancinhas}

Dentre os participantes, há um capítulo intitulado ``Relato de três
criancinhas'' (Nicolas, que é mudo, Alain e Denis) e outro, intitulado
``Relato da pequena Allys''.

No primeiro, ficamos sabendo que as três crianças, atendendo às ``vozes
brancas'' que os chamavam na noite, saíram ``pelas estradas rumo a
Jerusalém'' (p. 32). Eram vozes ``como as vozes dos pássaros mortos no
inverno'', e que chamavam todas as criancinhas. São, portanto, vozes que
não se ouvem, fruto de uma fé inabalável e de uma alucinação coletiva.
Por essa razão, embora digam que em seu caminho ``havia homens'' que os
``maldiziam'' e que as terras são ``perigosas para as crianças'',
havendo em toda parte ``densas florestas, e rios, e montanhas, e
espinhos'', ninguém faz mal a eles, pois são escoltados pelas vozes que
nunca os abandonam. Também ficamos sabendo que Nicolas é mudo e que há
na expedição um menino cego (Eustáquio), guiado por Allys.

O capítulo faz intertexto com outra história famosa, que circulava na
mesma época, a do flautista de Hamelin (o flautista que hipnotizava
ratos com sua flauta mágica e que, por não ter recebido o devido
pagamento por ter livrado a cidade de Hamelin de uma infestação desses
roedores, hipnotizou suas crianças, conduzindo-as para fora da cidade).
No conto medieval, somente três crianças sobrevivem, por não terem
podido acompanhar o flautista: ``uma cega, uma surda e uma manca''.


\Image{Ilustração do flautista de Hamelin, por Kate Greenaway. (Kate Greenaway; Domínio Público)}{PNLD0036-07.png}


Ainda hoje há, na cidade alemã de Hamelin, alguns indícios que fazem
referência a essa história:

\begin{itemize}
\item
  Uma residência privada, construída em 1602, em cuja fachada se
  encontra uma placa com os seguintes dizeres:
\end{itemize}

\emph{"Em 26 de junho de 1284, no dia de São João e São Paulo, 130
crianças nascidas em Hamelin foram retiradas da cidade por um flautista
vestido com roupas multicoloridas. Depois de passar pelo Calvário perto
de Koppenberg, desapareceram para sempre".}



\begin{itemize}
\item
  Há, nos registros da cidade, uma anotação datada de 1384, lamentando
  já terem se passado ``100 anos desde que nossas crianças partiram".
\end{itemize}

\begin{quote}

\end{quote}

\begin{itemize}
\item
  Descrições do vitral da Igreja de São Nicolau (destruída no século
  XVII), ilustram a figura de um flautista conduzindo várias crianças em
  transe, vestidas de branco.
\end{itemize}

\begin{itemize}
\item
  O manuscrito de Lüneburg (séc. XV) faz referência a um acontecimento
  envolvendo 130 crianças desaparecidas em 26 de junho de 1284, seguindo
  um flautista até Koppen.
\end{itemize}

\subsection{O relato de Inocêncio iii}

O Papa Inocêncio III começa seu relato dentro de uma sala
``desdourada'', longe da pompa e do luxo do Palácio Apostólico, ``pois o
Senhor não deve realmente ouvir a voz de seus padres em meio à pompa dos
mandamentos e bulas; e decerto nem são de seu agrado a púrpura, as
joias, ou as pinturas'' (p.28). Inocêncio sabe que o despojamento (mais
uma vez o branco aparece, nas vestes do Papa e nas paredes) é essencial
para que Deus o ouça e é a Ele a quem se dirige, confessando o fracasso
das Cruzadas militares e temendo a Cruzada das Crianças, que não é uma
obra pia: ``É de crer que o Maligno possui essas pobres criaturas'' (p.
29). Roga a Deus que o oriente, pois tanto ele quanto as crianças são
puros, ignorantes e inocentes.


\image{Representação do Papa Inocêncio III feita entre 1275 e 1230. (Grandes Chroniques de France,  Bibliothèque Sainte-Geneviève; CC-BY-SA-3.0)}{PNLD0036-08.png}


Mais uma vez aparece a intertextualidade com o flautista de Hamelin
(sendo o flautista o próprio Diabo), com uma passagem do Evangelho de
Marcos (cap. 5), em que o demônio (cujo nome é Legião, pois eram muitos)
entra no corpo de 2 mil porcos, precipitando-nos no mar, afogando-os, e
com o Evangelho de Mateus (cap. 2), em que Herodes mandou matar todos os
meninos menores de dois anos que havia em Belém e no seu entorno.

O relato de Inocêncio III é um dos relatos de maior beleza literária,
pelo tom confessional, pelo manejo com as palavras (Inocêncio e
inocente), pelas intertextualidades e pelas antíteses que se
apresentam:

-- a riqueza da Igreja e o despojamento da sala em que se encontra, como
explicitado acima;

-- a imponência do Sumo Pontífice que julga (``Há crimes muito grandes.
Podemos dar-lhes a absolvição. Há grandes heresias. Há heresias muito
grandes. Devemos puni-las impiedosamente'', p. 28), contrastando com o
homem que, diante do encontro solitário com Deus, revela sua fraqueza e
suas dúvidas (``Há crimes. Há crimes muito grandes. Há heresias. Há
heresias muito grandes. Minha cabeça oscila de fraqueza: talvez não se
deva punir, nem absolver'', p. 31).

Cabe aqui a observação de que, na vida real, Inocêncio III (cujo papado
se estabeleceu de 22/2/1198 a 16/7/1216) foi o idealizador da IV
Cruzada, a chamada ``cruzada da vergonha'', em que os cavaleiros
saquearam os cristãos. Assim, Schwob dá ao Papa Inocêncio III, por meio
da arte, a possibilidade de se redimir de sua culpa. Quando Inocêncio
diz Ecce homo (``Eis o homem'', frase dita por Pôncio Pilatos ao
apresentar Jesus - flagelado, atado e com a coroa de espinhos -- à
multidão hostil, que o condenou), coloca-se também flagelado diante de
Deus, penitenciando-se.


\image{O quadro de Manuel de La Cruz, de 1788, representa uma visão do Papa Inocêncio III, relacionada com São Francisco de Assis. (Manuel de la Cruz; Domínio Público)}{PNLD0036-09.png}


\subsection{O relato de François Longuejoue, escrevente}

O que mais chama atenção nesse relato é a ressignificação que Schwod dá
aos fatos históricos.

Na narrativa, François Longuejoue trabalha para Hugues Ferré, mercador
de Marselha. Relata que, aos 15 dias de setembro de 1212, várias
crianças foram à oficina de seu mestre, a fim de contratarem barcos que
os levassem à Terra Santa. Não tendo a quantia de barcos suficientes
para tão grande número de crianças, o mestre ordenou a Longuejoue que
solicitasse a outro mercador, Guillaume Porc, que inteirasse os barcos
restantes.

Mais uma vez constatamos a presença de personagens reais transmutados na
ficção de Schwob, que foi buscar num episódio histórico - a Noite do
Massacre de São Bartolomeu -  a figura de François Longuejoue:
huguenote salvo do massacre provocado pelos cristãos em Paris, no ano de
1572.

Também Hugues Ferré e Guillaume Porc são personagens reais: ambos
mercadores de Marselha, no século XIII.

Longuejoue prossegue, relatando que ``os senhores almotacéis, com razão
temendo a escassez, reuniram-se na câmara municipal onde, após
deliberar, mandaram chamar nossos ditos mestres a fim de exortá-los e
suplicar que expedissem as naves com grande diligência. O mar não se
encontra por ora muito favorável, por causa dos equinócios, mas há que
considerar que tal afluência poderia ser perigosa para nossa boa cidade,
tanto mais por essas crianças todas se encontrarem famintas'' (p. 34).

A passagem acima aproxima-se de uma preocupação real no início do século
XIII: a escassez de alimentos em regiões com grande trânsito de
camponeses sem terras. De fato, entre os séculos XI e XIII, houve um
considerável crescimento populacional na Europa Ocidental, não
condizente com o aprimoramento das técnicas de produção, o que vai
provocar, no século seguinte, a chamada Grande Fome.

Como diz a escritora e crítica literária Mariana Ianelli, ``Onde quer
que haja guerra, há uma cruzada de crianças''. A Cruzada das Crianças
de 1212, tão bem narrada na obra de Schwob, é só uma das inúmeras
cruzadas realizadas o tempo todo por crianças e jovens em várias partes
do mundo, seja porque foram convocadas pelas forças armadas de sua terra
natal, seja porque já não têm mais uma terra natal.

Finalmente, respondendo à pergunta Por que ler A Cruzada das Crianças?,
basta pensar que é preciso conhecer o passado para entender o presente e
planejar o futuro. Neste exato momento em que estudamos um acontecimento
ocorrido tantos séculos atrás -- a ida de crianças da Europa até
Jerusalém -- milhares de crianças estão partindo da Terra Santa e
cruzando o Mediterrâneo em direção à Europa. Infelizmente, muitas destas
crianças tiveram e terão o mesmo destino daquelas.

\subsection{Vidas imaginárias}

Mais uma vez temos aqui um escritor que transita com maestria entre
ficção e realidade, preenchendo, através da arte, as lacunas que a
realidade nos apresenta. São relatos biográficos curtos de personagens
reais, embora muitas vezes desconhecidos, apresentados em ordem
cronológica. Seu ponto de partida, assim como na outra obra que compõe o
mesmo volume, são documentos verídicos, sobre os quais vai
acrescentando, com esmero, elementos ficcionais.

Recusando-se à biografia histórica, Schwob inaugura uma biografia
ficcional ou bioficção. Segundo o autor, há uma diferença entre relatar
uma vida e relatar os acontecimentos de uma vida, o que faz com que ``a
arte do biógrafo'' seja ``dar igual valor à vida de um pobre ator e à
vida de Shakespeare'' (p. 55).

O poema ``Perguntas de um operário que lê'', de Bertold Brecht, traz os
seguintes versos iniciais:

\begin{verse}
Quem construiu Tebas, a das sete portas?\\
Nos livros vem o nome dos reis,\\
Mas foram os reis que transportaram as pedras?\\
\end{verse}

O poema questiona justamente a história oficial, fundada nos grandes
feitos de grandes personagens, deixando à margem milhões de pessoas que,
de fato, construíram ou colaboraram efetivamente para que esses grandes
feitos acontecessem.

A obra Vidas Imaginárias aborda exatamente esse assunto. Nela, uma
galeria de personagens esquecidas pela história aparece em seus feitos
particulares, numa prosa fluida e rica de conotações. Esta é a razão de
as histórias serem curtas: Schwob objetiva com isso acentuar fatos
concretos que individualizam o personagem.

Das vidas narradas, algumas pertencem a personalidades famosas; outras,
a pessoas a quem hoje desconhecemos: Empédocles (filósofo
pré"-socrático), Heróstrato (responsável pela destruição do templo de
Ártemis), Crates (filósofo grego), Septima encantatriz (escrava
africana), Lucrécio (poeta e filósofo romano), Clódia (filha de Claudius
Pulcher, político romano), Petrônio (escritor romano, autor da obra
Satiricon), Sufrah (mágico oriental), Frade Dolcino (frade condenado
pela Inquisição), Cecco Angiolieri (poeta), Paolo Uccello (pintor
renascentista), Nicolas Loyseleur (religioso, juiz), Katherine, a
rendeira (mulher amante), Alain (soldado do rei Carlos VIII), Gabriel
Spencer (ator elisabetano), Pocahontas (indígena da etnia powhatan que
se casou com o inglês John Rolfe), Cyril Tourneur (dramaturgo inglês de
fins do século XVI), William Phips (caça-tesouros inglês), Capitão Kid
(pirata escocês), Walter Kennedy (pirata inglês), Stede Bonnet (pirata
barbadiano), Sr. Burke e Sr. Hare (assassinos).

Para Schwob nada de novo poderia ser criado, pois tudo já tinha sido
dito e esquecido. Por isso se esmera nas formas, pois, segundo ele, o
estilo é a única coisa que um artista pode fazer para a arte. Assim, não
importa quem seja o biografado: qualquer indivíduo é digno de ser
personagem de suas histórias, pois a beleza da vida está na arte de
contá-la.

Como diz na vida imaginária que construiu para Paolo Ucello pintor,
``seu intento não estava na imitação, e sim na capacidade de desenvolver
soberanamente todas as coisas'' (p. 101).

Por essa razão, em seu texto há uma clara elaboração da escrita, que
pode ser observada no uso dos tempos verbais, no cruzamento do uso nas
pessoas do discurso e, sobretudo, na profusão de figuras de linguagem
(tão próprias da escola literária a que Marcel Schwob pertence, o
Simbolismo).

Na vida imaginária de Septima encantatriz, por exemplo, notamos a
confluência de marcas formais de 3ª e 1ª pessoa: a voz do narrador (o
biógrafo) e a voz de Septima (a biografada).

Ferida de amor, Septima invoca sua irmã morta, rogando-lhe que faça
Sextílio se apaixonar por ela. Nesse momento, o narrador dá voz à
Septima:

``--- Ó minha irmã --- disse ela ---, sai do teu sono para me escutar.''
(p.71)

Essa interferência da primeira pessoa em um gênero que é inteiramente
narrado em terceira pessoa permite uma interpretação com base na
linguística de Ferdinand de Saussure. A terceira pessoa não está no
mesmo plano da primeira e da segunda pessoas: enquanto essas são ``de
fato'' pessoas do discurso, participando ativamente dele, aquela está
ausente, distante. Narrar em terceira pessoa significa usar a língua;
narrar em primeira significa valer-se da fala. Para Schwob, biografia e
história se relacionam com a língua (que é pública e objetiva), enquanto
a vida e a arte se relacionariam com a fala (privada e subjetiva).

Ainda nessa vida imaginária, é de se notar como o autor cria uma
atmosfera fantástica e misteriosa, reforçada por momentos repulsivos em
que não poupa detalhes desagradáveis para descrever Foinissa, a irmã
morta de Septima:

``E Foinissa, morta, envolta em faixas fragrantes, sentou-se junto dele.
E ela não tinha cérebro nem vísceras; mas seu coração ressecado fora
recolocado em seu peito.'' (p. 73)

O trabalho com a linguagem também pode ser notado, como dito acima, nas
figuras de linguagem que se estendem por toda a obra:


\subsubsection{-- Metáforas e comparações}

\emph{``O universo pareceu se assemelhar aos floquinhos de lã que os
dedos da africana espalhavam pelas salas.''} (\emph{Lucrécio poeta}, p.
76)

``\emph{Era um lugar árido e pedregoso, entre quatro montanhas nuas
erguidas feito dedos para os quatro cantos do céu.'' (Sufrah geomante},
p. 82)

``\emph{Sua palavra era áspera qual vinho da montanha --- mas atraiu
Dolcino.'' (Frate Dolcino Herético}, p. 92)

``\emph{Naquela noite, saciou com fogo sua sede de rancor}.''
(\emph{Cecco Angiolieri poeta rancoroso}, p. 98)

``\emph{E Katherine vagava frente aos ateliês da luveira e da capuzeira,
e não raro se demorou a invejar o rosto sanguíneo da salsicheira, rindo
em meio a suas carnes de porco'' (Katherine, a rendeira mulher amante},
p. 112-3)


\subsubsection{-- Sinestesias:}

``\emph{o mar de cor profunda, a linha brumosa de Samos''.}
(\emph{Eróstrato incendiário}, p. 60)

\emph{``Rígidas pinturas ornavam as câmaras interiores''} (\emph{idem},
p.61)

\emph{``as pedras da casa em que Septima vivia eram de um rosa
trêmulo''} (\emph{Septima encantatriz}, p. 70)

\emph{``E quando o tremor purpúreo se apossou do ar da noite, ela saiu
pelo caminho que vai de Adrumeto até o mar''}. (\emph{idem}, p. 71)

\emph{``As canções para Bice (usava o seu nome vulgar) eram abstratas e
brancas; as suas eram cheias de força e de cor''.} (\emph{Cecco
Angiolieri poeta rancoroso}, p. 98)

\subsubsection{-- Prosopopeias:}

``\emph{Entretanto jorrava, da goela do Etna, uma coluna de negra fumaça
a lançar sua sombra sobre a Sicília}.'' (\emph{Empédocles, deus
presumido}, p. 57)

\emph{``A cratera do vulcão vomitava um feixe de chamas''} (\emph{idem},
p 60)

\emph{``O fogo enrolou-se nos capitéis das colunas, rastejou pelas
abóbadas}'' (\emph{Eróstrato incendiário,} p. 64)

\emph{``A pedra abraçava seus seios cingidos com tiras.''}
(\emph{Septima encantatriz,} p. 71)

``\emph{Ofereceu o colo à mordida salgada do vento marinho}``
(\emph{idem}, p 72)

\subsubsection{-- Humor e nas ironias:}

``\emph{Sua saúde era perturbada por contínuas flatulências que ele não
conseguia conter. Entrou em desespero e resolveu morrer. Crates soube de
seu tormento, e quis consolá-lo. Comeu uma medida de tremoços e foi se
encontrar com Metrocles. Perguntou-lhe se era a vergonha de sua
enfermidade que o afligia a tal ponto. Metrocles confessou que não podia
suportar esta desgraça. Então Crates, todo inchado de tremoços, soltou
gases na presença de seu discípulo, e afirmou-lhe que a natureza
submetia todos os homens ao mesmo mal. Censurou-o então por ter tido
vergonha dos outros e lhe ofereceu seu próprio exemplo. Então soltou
mais uns gases, tomou Metrocles pela mão e o levou consigo''. (Crates
cínico}, p. 69)

\emph{``Dominava perfeitamente a arte de torcer uma mecha em volta da
testa do prisioneiro até que os olhos lhe saltassem para fora, ou de
afagar-lhe o rosto com folhas de palmeira inflamadas.''} (\emph{Walter
Kennedy pirata iletrado}, p. 140)

\emph{``Admirável Teach, que soubeste governar sucessivamente quatorze
mulheres e livrar-se delas, e tiveste a ideia de entregar toda noite a
última (tem apenas dezesseis anos) a teus melhores companheiros (por
pura generosidade, grandeza de alma e ciência do mundo) na tua boa ilha
de Okerecok!''} (\emph{Major Stede Bonnet pirata por propensão}, p. 146)

\emph{``comprando uma velha chalupa de dez peças de canhão, equipou-a
com tudo quanto convinha à pirataria, tal como facões, arcabuzes,
escadas, pranchas, arpéus, machados, Bíblias (para prestar juramento),
pipas de rum, lanternas, fuligem para escurecer o rosto, pez, mechas
para queimar entre os dedos dos ricos mercadores e uma quantidade de
bandeiras pretas com caveira branca, dois fêmures cruzados e o nome da
nau: a Revanche.''} (\emph{idem}, p. 146)

\emph{``Naqueles primeiros anos do século, os médicos estudavam a
anatomia com paixão; mas, devido aos princípios da religião, tinham
muita dificuldade em conseguir sujeitos para dissecar. O sr. Burke,
espírito esclarecido que era, percebera aquela lacuna da ciência.''}
(\emph{Sr. Burke e Sr. Hare assassinos}, p. 154)

Tentar identificar o que é ficção e o que é realidade na obra Vidas
Imaginárias é um trabalho insano e uma perda de tempo: primeiro porque
as fontes pesquisadas por Schwob são de difícil acesso; segundo, e essa
talvez seja a razão principal, porque a beleza da obra está justamente
em acreditarmos que a vida desses personagens transcorreu como
imaginado.

Para que se tenha um exemplo, o livro se inicia com a vida imaginária
de Empédocles (Deus Presumido). Schwob certamente se valeu da leitura
de Diógenes Laércio (primeira metade do século III d.C.), historiador e
biógrafo de filósofos gregos, que escreveu Vidas e Doutrinas de
Filósofos Ilustres, uma obra em 10 volumes, organizados por escolas
filosóficas. Por meio dela, sabe-se que Empédocles era médico, orador e
conhecia a arte da magia, tendo também desempenhado um papel de destaque
nas questões públicas. Diógenes Laércio salienta o contraste entre sua
moderação na política e o excesso que emerge em seus escritos, nos quais
se apresenta como igual a um deus.

Figura de destaque, Empédocles despertou admiração por vários feitos:
para salvar uma safra, propôs bloquear o caminho aos ventos fortes por
meio de peles de burro, esticadas na passagem por onde corriam; para
erradicar uma epidemia de peste, reuniu as águas dos rios da região para
purificar o local. Laércio também conta que ele conseguiu trazer de
volta à vida uma mulher que há mais de trinta dias estava inerte por
conta de um ataque de catalepsia. Essa cura foi interpretada como uma
ressurreição, o que ampliou a fama de Empédocles aos olhos de seus
contemporâneos.

No entanto, ainda que Schwob tenha lido toda a obra de Laércio (que
também escreveu sobre a vida de Crato), não se pode ter certeza sobre
muitos dos acontecimentos sobre a vida de Empédocles, pois há
informações contraditórias e falta de detalhes. Desse modo, em sua vida
imaginária, Schwob preenche as lacunas do obscurantismo que envolve a
vida do poeta grego, construindo um personagem lendário.

Para os que defendem exclusivamente a biografia histórica, ou o relato
da vida ``como ela é'', é preciso lembrar que todas as vezes que nos
dispomos a escrever sobre qualquer assunto, nós assumimos um ponto de
vista e recortamos a realidade de acordo com esse ponto de vista. Em
outras palavras, do conjunto de fatos que constituem a vida de uma
pessoa, escolhemos só aqueles que, por interesses variados, nos
interessam numa dada época e/ou num dado lugar.

Desse modo, por que ler Vidas Imaginárias, de Marcel Schwob? Como o
próprio autor diz no prefácio de sua obra, ``A arte é contrária às
ideias universais, descreve apenas o individual, deseja apenas o único.
Não classifica; desclassifica'' (p 47). Temos aqui então duas obras em
uma: uma obra de vidas e uma obra de arte.


\section{Sugestões de referências complementares}\label{sugestoes}
% Atenção: Renier, aqui reunir bibliografia e pensar nas ref. 
\subsection{Filmes}
\begin{itemize}

	\item\textit{Cruzada}. Dirigido por Ridley Scott. (\textsc{eua}, Reino Unido, Espanha,
	Alemanha, 2005).

	Com a morte de seu pai, o jovem Balian herda terras e um título de
	nobreza em Jerusalém. Determinado a manter seu juramento de manter a paz
	na Terra Santa, ele permanece no local e serve como cavaleiro a um rei
	amaldiçoado.

	\item\textit{Loucuras na Idade Média}. Dirigido por Gil Junger. (\textsc{eua}, 2001).

	Nesta comédia, acompanhamos as aventuras de Jamal, um jovem trabalhador que, após receber
	uma pancada na cabeça, é transportado para a Inglaterra do século \textsc{xiv}.
	A ambientação (figurinos, cenários etc.) e os contextos sociais do período representados no filme
	podem ser úteis na contextualização do livro de Schwob.


	\item\textit{Meia"-noite em Paris}. Dirigido por Woody Allen. (Espanha e \textsc{eua}, 2011).

	Um jovem escritor estadunidense vai com sua noiva a Paris e acaba tendo acesso
	a uma carruagem que lhe permite voltar no tempo e conhecer personagens célebres da 
	história da cidade com quem conversa e cria relações. Por este motivo, é um filme 
	interessante para se observar alguns exemplos de ficção a partir da realidade.

	\item\textit{Pocahontas}. Dirigido por Mike Gabriel e Eric Goldberg. (\textsc{eua}, 1995)

	Animação dos estúdios Walt Disney, é baseada na história real da indígena do povo Powahtan
	que, sob condições escusas do contexto colonial norte"-americano, se casa com um inglês
	e vai com ele para a Inglaterra. É interessante se observar as diferenças e 
	semelhanças entre o enredo do filme da Disney e o capítulo do livro de Schwob,
	já que ambos partem da mesma história.

	\item\textit{O Sétimo Selo}. Dirigido por Ingmar Bergman. (Suécia, 1957).

	Ambientado na Idade Média, o filme conta a história de um cavaleiro que retorna 
	das Cruzadas e encontra o país devastado pela doença. Sua fé em Deus é sensivelmente 
	abalada e, enquanto ele reflete sobre o significado da vida, a Morte surge à sua 
	frente querendo levá"-lo, pois chegara sua hora. Objetivando ganhar tempo, o cavaleiro 
	convida"-a para um jogo de xadrez que decidirá se ele parte com a Morte ou não. 
	Tudo depende da sua vitória no jogo e a Morte concorda com o desafio, já que não perde nunca.

\end{itemize}

\subsection{Sites e museus}

	\href{https://artsandculture.google.com/partner/national-museum-of-ancient-art}{Museu Nacional de Arte Antiga.} 

	Em Lisboa mas com exposições virtuais abertas na plataforma Google Arts and Culture, o museu 
	é um dos maiores da Europa e o maior de Portugal em acervo acerca da Idade Média.  

	\href{https://www.news-medical.net/health/History-of-leprosy-(Portuguese).aspx#:~:text=A%20lepra%20originou%20em%20%C3%81frica,das%20trevas%2C%20at%C3%A9%20aproximadamente%201870.}{História da lepra.}

\section{Bibliografia comentada}

\textsc{dosse}, François. O desafio biográfico: escrever uma vida. São
Paulo: Edusp, 2016.

O autor disseca a história do gênero biográfico, observando uma
``libertação'' que se manifesta a partir do início dos anos de 1980,
redescobrindo as virtudes do gênero.

\textsc{dumas}, Alexandre. Robin Hood. Rio de Janeiro: Zahar, 2016.

Um dos maiores escritores de todos os tempos narra nestas páginas as
aventuras do lendário Robin Hood e de seu bando, que se arriscam para
promover justiça e igualdade na Inglaterra dos séculos \textsc{xii} e \textsc{xiii}.

\textsc{goff}, Jacques Le. A civilização do ocidente medieval. São Paulo:
Vozes, 2018.

Este livro consagrado é uma porta de entrada para a Idade Média e seu
universo de monges, clérigos, guerreiros, camponeses, artesãos e
comerciantes.

\textsc{riley"-smith}, Jonathan. As Cruzadas: uma história. Campinas:
Ecclesiae, 2019.

Autoridade no assunto, o autor explora os antecedentes das cruzadas,
seus principais agentes e acontecimentos, através de um texto amplo e
acessível.

\textsc{tavares}, Gonçalo. Histórias falsas. Rio de Janeiro: Casa Da
Palavra, 2008.

O escritor português mistura fantasia e realidade nas nove histórias
deste livro, cujos protagonistas têm relações diretas com personagens da
história da filosofia.

\textsc{voltaire}, François Marie Arouet. História das Cruzadas. São paulo: Madras, 2012.

Escrito pelo célebre filósofo francês, o livro conta, em linhas gerais,
a história das Cruzadas, desde a primeira até a tomada de Jerusalém.


\end{document}
