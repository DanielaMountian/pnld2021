\documentclass[12pt]{extarticle}
\usepackage{manualdoprofessor}
\usepackage{fichatecnica}
\usepackage{lipsum,media9,graficos}
\usepackage[justification=raggedright]{caption}
\usepackage[one]{bncc}
\usepackage[relicario]{../edlab}
\begin{document}


\newcommand{\AutorLivro}{William Wells Brown}
\newcommand{\TituloLivro}{Narrativa de William Wells Brown, escravo fugitivo}
\newcommand{\Tema}{Diálogos com a sociologia e com a antropologia}
\newcommand{\Genero}{Diário, biografia, autobiografia, relatos, memórias}
\newcommand{\imagemCapa}{./images/PNLD0038-01.png}
\newcommand{\issnppub}{---}
\newcommand{\issnepub}{---}
% \newcommand{\fichacatalografica}{PNLD0038-00.png}
\newcommand{\colaborador}{{Eduardo Modesto de Carvalho, Bruno Gradella e Vicente Castro}}


\title{\TituloLivro}
\author{\AutorLivro}
\def\authornotes{\colaborador}

\date{}
\maketitle

\begin{abstract}\addcontentsline{toc}{section}{Carta ao professor}
Este manual tem como objetivo fornecer subsídios para o trabalho com a
obra literária \emph{Narrativa de William Wells Brown}, escrita por ele
mesmo.

Nascido no Kentucky em 1814, William Brown foi um escritor estadunidense.
Exerceu, durante sua vida, diversas funções de trabalho doméstico e de serviços,
incluindo assistente de tipógrafo, auxiliar de alfaiate, lavrador,
barbeiro, cavalariço e cocheiro. Brown também foi assistente de um
traficante negreiro, responsável por preparar pessoas escravizadas
para a venda em Natchez, Mississípi, e Nova Orleans, Luisiana. Essa
experiência, que Brown vivenciou aos 16 anos, se destaca como um dos 
episódios mais emocionantes e aterrorizadores da sua \emph{Narrativa}. 

Acompanharemos, neste livro, as diversas histórias de William Brown se
esquivando das armadilhas do sistema escravista para conquistar sua liberdade.
Dentre essas, podemos destacar o episódio em que ele foi mandado, por seu dono,
com bilhete com a instrução de ser açoitado por um carcereiro na região 
do Mississípi. Temendo o conteúdo do bilhete, já que não sabia lê"-lo,
Brown, pensou rápido e pagou outro homem negro para levar o papel até 
a cadeia, onde seu substituto recebeu ``vinte chibatadas nas costas
nuas''.

Outra vez que se usou de artimanhas para garantir sua busca pela liberdade
foi quando simulou aceitar uma união com outra escravizada, encomendada
pelos seus ``donos" a fim de mantê-lo na propriedade, além de garantir
mais escravizados conforme os filhos do casal nascessem. Ciente dos motivos
ocultos, Brown dissimula e foge do ``combinado".

Publicada pela Sociedade Antiescravista de Massachusetts em julho de 1847, 
a \emph{Narrativa de William Wells Brown, escravo fugitivo} 
é um apelo à abolição da escravidão nos Estados Unidos e, por consequência, 
em todo o mundo. É um testemunho em primeira pessoa da escravidão norte"-americana 
e coloca os leitores cara a cara com o ambiente de violência social que impactava 
radicalmente a personalidade, a família e o desenvolvimento moral entre 
os escravizados. 

Esperamos que este material seja muito útil no trabalho do professor em sala de aula!

\end{abstract}

\tableofcontents


\Image{Gravura de William Wells Brown, 1847 (Autor Desconhecido; Domínio Público )}{PNLD0038-03.png}

\section{Propostas de Atividades I}

\subsection{Pré"-leitura}

\paragraph{Tema} O abolicionismo e os românticos

\paragraph{Conteúdo} Compreensão dos movimentos abolicionistas e da
participação dos poetas românticos em seus ideais.

\paragraph{Objetivo} Habilitar os alunos a identificar os ideais
abolicionistas em obras da terceira fase do Romantismo brasileiro.

\paragraph{Justificativa} Tendo em vista que uma das bases do presente
livro é a luta de William Brown por sua liberdade, o que resultou em diversas
fugas ao longo de seu período enquanto cativo, é importante, neste primeiro
contato, introduzirmos a discussão acerca dos movimentos abolicionistas.

Ainda que tenha variado em formatos de atuação, em
metodologias e políticas estratégicas, esse movimento, em geral, buscava
o fim do tráfico negreiro, a emancipação dos escravos e a integração da
população negra na sociedade. 

Uma das formas de atuação dos abolicionistas foi através
da literatura. No século \textsc{xix}, no Brasil ainda escravista,
estava em voga o movimento romântico, caracterizado sobretudo pela busca 
por uma identidade nacional. Em sua terceira geração, que é a que mais nos
interessa, a chamada
\emph{condoreira}, os poetas muito influenciados pelo poeta francês Victor Hugo
começaram a levar em conta também a parcela negra da população brasileira
neste anseio pela formação de uma identidade nacional. Daí a explosão 
de referências à liberdade e à abolição neste período, e a própria
alcunha que recebeu o momento --- ``condoreira'' faz referência a 
uma ave de visão ampla presente nos Andes. Uma das figuras proeminentes 
deste período, o baiano Castro Alves, conhecido como o ``Poeta 
dos Escravos'', escreveu obras como \emph{O Navio Negreiro} (1869) e 
\emph{Os Escravos} (1883).

Uma fonte interessante de consulta para essa atividade é a obra
\emph{As utopias românticas}, de Elias Thomé Saliba (São Paulo: Estação
Liberdade, 2008), que fornece elementos para analisar o Romantismo como
movimento sociocultural enraizado na história europeia, ligado às
revoluções burguesas, com ressonâncias em outros países e continentes.

\paragraph{Metodologia}
   \begin{enumerate}
    \item
    Para apresentar os alunos ao tema da escravidão, sugerimos que o 
    professor faça a leitura de um poema de Castro Alves presente
    em um dos livros citados na \textbf{Justificativa}. Após a leitura,
    o professor irá apresentar algumas características do movimento
    romântico e da geração condoreira.
    \BNCC{EM13LP46}

    \item
    Após a análise conjunta dos poemas, proponha uma pesquisa, em
    livros e \emph{sites}, a respeito de publicações abolicionistas, brasileiras e 
    estrangeiras, publicadas em diversos gêneros e formatos. É possível
  encontrar antigas reportagens e notícias, pertencentes ao campo
  jornalístico-midiático, ou mesmo panfletos abolicionistas do século
  \textsc{xix}.
  \BNCC{EM13LP30}

    \item
    Por fim, com o repertório construído sobre a poesia
  e o jornalismo oitocentistas, a partir da pesquisa de diferentes
  gêneros textuais, organize a produção de um manifesto
  abolicionista, colocando a turma, dividida em pequenos
  grupos, no papel de organizadores de um jornal abolicionista do
  século \textsc{xix}. Estimule a defesa de valores associados a liberdade,
  igualdade e respeito pelas diferenças. Com essa atividade, pretende-se
  também capacitar o aluno a realizar um cotejo entre diversas fontes de
  pesquisa que abordam uma mesma temática. Ao mesmo tempo, a turma
  construirá um repertório linguístico e literário que possibilitará
  mergulhar nos relatos contidos em \emph{Narrativa de William Wells
  Brown.}
  \BNCC{EM13LGG102}

   \end{enumerate}
\paragraph{Tempo estimado} Quatro aulas de 50 minutos.


\subsection{Leitura}

 \paragraph{Tema} Construindo um perfil biográfico

 \paragraph{Conteúdo} Leitura do primeiro capítulo do livro onde o
 narrador se apresenta, e trabalho com o gênero entrevista. 

 \paragraph{Objetivo e justificativa} \emph{Narrativa de William Wells Brown, escravo fugitivo}
 é uma obra que apresenta a visão de uma ex"-escravizado sobre si mesmo. 
 Um caso raro, como já dissemos, justamente devido ao fato de que a 
 escravidão impedia que pessoas na condição de escravos tivessem acesso
 à cultura letrada. Ainda assim, William Brown encontra um meio de aprender
 a escrever e, em fuga, traçar seu perfil autobiográfico ao qual hoje temos acesso
 neste livro.

 Levando em conta a importância que terá para o autor a possibilidade
 de se apresentar com suas próprias palavras, propomos que os alunos
 tenham a mesma experiência com uma atividade de criação de perfis 
 biográficos primeiro entre si, na sala de aula, e então com pessoas
 de seu entorno fora da escola.

 \paragraph{Metodologia}
   \begin{enumerate}
    \item
    No primeiro capítulo do livro William Brown faz uma apresentação
    de sua história pessoal e de sua família, além do contexto em que vivia --- o de escravizado. A partir da leitura deste capítulo que devido a sua curta 
    extensão pode ser feita em sala de aula, peça que os alunos levantem
    as principais características apresentadas pelo autor em sua auto"-apresentação
    na forma de um resumo.
\BNCC{EM13LP28}    

    \item
    Em seguida, o professor irá indicar a seguinte atividade:
    seguindo o exemplo de William Brown os alunos devem
    reunir"-se em grupos e fazer entrevistas a fim de formar um
    perfil biográfico de cada um. Eles devem formular 
    perguntas pré"-estabelecidas que deem atenção sobretudo à história 
    pessoal de cada um, sem deixar de lado seus gostos e interesses. 
    Realizada durante a aula, a coleta desses dados pode ser feita por 
    meio da escrita ou de uma gravação de áudio, neste caso, usando
    aparelhos celulares, caso todos possuam.
\BNCC{EM13LP19}

    \item 
    A partir da experiência em sala, os alunos serão estimulados, agora,
    a expandi"-la para fora da escola. Eles devem escolher uma pessoa 
    de sua família ou de sua rua, por exemplo, para entrevistar.
    O roteiro será o mesmo: devem preparar perguntas, só que desta vez mais 
    específicas para não correr o risco de deixar passar informações importantes, como:
    nome completo; idade e data de
    nascimento; local de nascimento; profissão. E, a partir daí, motivos que
    levaram a pessoa a exercer esse trabalho, principais desafios
    enfrentados e rotina profissional, etc. Podem continuar perguntando por frases e pensamentos, lidos ou
    ouvidos pela pessoa, que servem de orientação para a vida dela; opiniões
    sobre fatos da atualidade; pessoas próximas (amigos, parentes,
    vizinhos); episódios e momentos mais marcantes da vida; sonhos e planos
    para o futuro.
    Uma vez realizada a coleta dos dados, deve se passar para uma etapa de 
    transcrição do material. O texto corrido, com as perguntas e respostas,
    deve ser apresentado ao professor, que fará os comentários que achar necessários
    para melhor encaminhamento. 
    Por fim, os áudios devem ser tratados por meio de um programa de computador
    gratuito como o \emph{Audacity} e, em seguida, subidos para uma plataforma
    digital gratuita no formato de \emph{podcast}, de modo que as produções fiquem 
    à disposição de todos.
    \BNCC{EM13LP53}

   \end{enumerate}

 \paragraph{Tempo estimado} Quatro aulas de 50 minutos.


\Image{Representação do autor enquanto personagem principal do livro (Internet Archive; Domínio Público)}{PNLD0038-04.png}

\subsection{Pós"-leitura}

 \paragraph{Tema} Outros relatos de ex"-escravizados

 \paragraph{Conteúdo} Comparação do relato de William Brown com outras
 obras de caráter semelhante.

 \paragraph{Objetivo} Habilitar os alunos a fazer uma leitura comparativa
 entre o presente livro e a obra audiovisual \emph{Dez Anos de Escravidão},
 além de outras obras disponíveis e acessíveis ao professor. 

 \paragraph{Justificativa} A narrativa de William Brown, ainda que 
 rara --- um registro autobiográfico de um ex"-escravizado é algo raro
 já que o domínio da escrita era restrito a pessoas livres --- não 
 é a única. Temos também alguns outros exemplos contemporâneos 
 ao próprio autor como \emph{Incidentes da vida de uma escrava}, de Harriet
 Ann Jacobs, também nascida escrava, e \emph{Doze Anos de Escravidão}, de 
 Solomon Northup, nascido livre e sequestrado durante doze anos. Ambos 
 registros autobiográficos, o último ganhou uma adaptação ao cinema em 
 2013 do diretor Steve McQueen que, baseado no livro de Northup, ajudou
 a divulgar sua obra. O filme ganhou o Oscar de Melhor Filme em 2014. 

 Estimular o contato com outras obras similares à de William Brown
 é importante para ampliar o repertório dos alunos acerca do tema
 da escravidão e do abolicionismo, bem como para fazê"-los perceber
 que, mesmo que próximas, cada história é individual e contém 
 suas particularidades, afinal estamos tratando de seres humanos. 

 \paragraph{Metodologia}
   \begin{enumerate}
    \item 
    Durante a aula, a turma deve assistir ao filme \emph{Doze Anos de Escravidão},
    de Steve McQueen, baseado na narrativa autobiográfica de Solomon Northup. 
    Após o filme, o professor deve propor uma discussão sobre a obra que acabaram de 
    ver, trazendo elementos e passagens da narrativa de William Brown 
    para efeito de comparação.
    \BNCC{EM13LP13}

  \item
   Num segundo momento, recomenda-se a produção individual de uma
resenha crítica em que os
alunos poderão expor, por meio da argumentação, uma apreciação
pessoal da \emph{Narrativa de William Wells Brown}, enriquecida por
elementos oriundos do contraste com o filme \emph{Doze Anos de Escravidão}.
As produções podem ser compartilhadas em um jornal virtual criado 
pela própria turma para divulgação com o restante da escola.
   \end{enumerate}
\BNCC{EM13LP53}

 \paragraph{Tempo estimado} Três a quatro aulas de 50 minutos.


\section{Propostas de Atividades II}

\subsection{Pré"-leitura}

 \paragraph{Tema} A escravidão nas artes visuais

 \paragraph{Conteúdo} Pesquisa sobre obras de artes visuais que representem 
 o universo da escravidão: os escravizados, os senhores, os traficantes,
 a vida nas fazendas etc.

 \paragraph{Objetivo} Possibilitar o acesso dos alunos a um acervo
 visual acerca do tema da escravidão de modo que construam um referencial
 imagético antes do contato com a obra literária. 

 \paragraph{Justificativa} A \emph{Narrativa de William Wells Brown, escravo fugitivo}
 é um relato em primeira pessoa de um homem nascido sob o sistema escravista
 estadunidense que consegue fugir de seu senhor e alcançar a liberdade, que já 
 se pode observar no fato de ele próprio ter escrito o presente livro. O acesso
 à escrita era algo restrito a pessoas livres e, portanto, um demarcador
 social das diferenças. 

 O fato de ser uma narrativa autobiográfica é uma das características que torna
 a narrativa de William Brown tão especial. Afinal, é um dos poucos casos
 em que o próprio escravizado ou ex"-escravizado relata sua experiência 
 sob o sistema escravista. 

 Partindo desse pressuposto, propomos que os alunos tenham contato, antes 
 da leitura do livro, com obras de artes visuais que representem 
 a escravidão e suas personagens, como o escravizado, o senhor, o traficante,
 bem como os espaços onde estas pessoas viviam, e suas tradições culturais. 
 Com esta primeira apresentação, acreditamos que os leitores farão uma leitura
 mais aprofundada da obra a seguir. 

 Para isso, indicamos a visita virtual ao Museu de Arte de São Paulo (MASP)
 onde se encontra a exposição \emph{Histórias Afro"-Atlânticas}, que reúne
 justamente obras deste período, muitas produzidas por pessoas negras
 descendentes de escravizados. As obras bem como textos e vídeos informativos sobre a 
 exposição estão disponíveis para acesso gratuito no seguinte \emph{link}:
 \href{https://masp.org.br/exposicoes/historias-afro-atlanticas}{Histórias Afro"-Atlânticas}. 

 \paragraph{Metodologia}
   \begin{enumerate}
    \item
    O professor de Artes deve introduzir o tema da escravidão a partir das 
    artes visuais. Para isso, sugerimos que utilize os recursos da exposição
    \emph{Histórias Afro"-Atlânticas} do Museu de Arte de São Paulo disponíveis 
    gratuitamente em \href{https://masp.org.br/exposicoes/historias-afro-atlanticas}{Histórias Afro"-Atlânticas}.
    Lá o professor irá encontrar, além de um acervo virtual das obras da exposição,
    textos e vídeos expositivos e informativos.

    \item
    Após a introdução ao tema, os alunos
    devem se dividir em grupos para realizar uma visita virtual ao museu. 
    Cada um ficará responsável por pesquisar obras de arte de um recorte específico,
    como: tráfico negreiro, escravizados, negros livres, abolição, cultura negra etc.
    Todos os grupos devem se ater também aos textos e vídeos explicativos presentes 
    no \emph{site} do museu.
    \BNCC{EM13LGG604}

    \item
    A partir dos resultados da pesquisa, cada grupo deverá apresentar ao
    restante da turma as obras encontradas, 
    além de uma explicação acerca do contexto. Sugerimos a via apresentação de slides. Os resultados devem ser publicados
    num jornal ou revista virtual feito pelos próprios alunos para divulgação
    com o restante da escola e da comunidade.
  \end{enumerate}
 
\subsection{Leitura}

 \paragraph{Tema} A geografia dos Estados Unidos das Américas

 \paragraph{Conteúdo} Conhecimento das principais características
 da geografia estadunidense no sentido da rota de fuga dos ex"-escravizados,
 do Sul ao Norte. 

 \paragraph{Objetivo} Habilitar os alunos a identificar e apresentar 
 os aspectos geográficos do espaço onde se passa a narrativa de William Brown.

 \paragraph{Metodologia}
   \begin{enumerate}
    \item
    Chegada ao a leitura do livro, o professor de Geografia deve 
    pedir que os alunos tomem nota das principais indicações de lugares
    que apareçam. Sabendo que a rota dos escravizados estadunidenses seguia
    o fluxo Sul"-Norte, o professor deve apresentar o mapa dos Estados Unidos
    para a turma indicando quais eram, à época, os estados escravistas
    e os abolicionistas. Para isso, pode se valer \href{https://brasilescola.uol.com.br/historiag/guerra-secessao.htm}{este artigo} sobre Guerra Civil Americana,
    acerca dos aspectos geográficos no período da Guerra de Secessão. 
    \BNCC{EM13LP28}
    
    \item
    Depois da exposição do professor, os alunos, divididos em grupos,
    devem realizar uma pesquisa na Internet e em livros de Ciências Humanas
    sobre as especificidades do território
    estadunidense. As premissas da pesquisa devem ser definidas pelo
    professor, mas sugerimos por exemplo: formações geológicas,
    relevos, recursos hídricos, espaços industrializados ou agrários etc.
    \BNCC{EM13LP01}

    \item
    Uma vez realizada a pesquisa, os grupos devem realizar um seminário
    onde apresentarão os resultados --- além de um mapa, realizado em comum
    entre os grupos, onde se represente a trajetória de William Brown pelo
    território do país. Para a elaboração deste mapa, deve ser utilizada 
    a plataforma virtual gratuita do Google \href{https://www.google.com/intl/pt-BR/maps/about/mymaps/}{My Maps}.
       \end{enumerate}
       \BNCC{EM13LP02}


\subsection{Pós"-leitura} Quatro aulas de 50 minutos.

 \paragraph{Tema} Os impactos do sistema escravista no pós"-abolição.

 \paragraph{Conteúdo} Articulação do contexto escravista estadunidense presente
 no livro com outros contextos nas Américas por meio de pesquisa documental.

 \paragraph{Objetivo} Habilitar os alunos a traçar paralelos entre as experiências
 de países como Brasil, Estados Unidos, Haiti, Colômbia e Cuba, por exemplo,
 observando como estas nações lidaram com as questões sociais do pós"-escravismo.

 \paragraph{Justificativa} Ao longo da obra, foi possível observar aspectos envolvendo
a instituição do escravismo. Entretanto, mesmo após o fim da escravidão
nos Estados Unidos, algo comum a quase todos os países americanos com
passado escravista, nota"-se a falta de um planejamento para a integração
da população antigamente escravizada à sociedade. Faltaram políticas
reparadoras que visassem diminuir as consequências de anos de abuso.
Diante disso, a atividade final propõe, após o término da leitura, uma
reflexão sobre problemas contemporâneos (como racismo, segregação
espacial, desigualdade), partilhados por muitos países do continente americano com
passado escravista. 

 \paragraph{Metodologia}
   \begin{enumerate}
    \item
    Com a turma dividida em pequenos grupos, o professor deve orientar
    a escolha de um país americano por equipe. Cada uma deverá pesquisar, em livros e
  \emph{sites} de internet, a história da escravidão e as ressonâncias
  do passado escravista no país escolhido. Estimule a busca de vídeos,
  imagens, músicas, mapas e outros documentos que permitam traçar o
  histórico dessas nações, além das obras estudadas durante as atividades 
  didáticas: o próprio livro \emph{Narrativa de William Wells Brown, escravo fugitivo},
  o catálogo virtual da exposição \emph{Histórias Afro"-Atlânticas} e o filme
  \emph{Doze Anos de Escravidão}.
\BNCC{EM13LGG704}

    \item
    Em seguida, o professor deve promover, com o apoio das disciplinas das Ciências Humanas, um seminário em
  que cada um dos grupos apresente as informações encontradas, com
  auxílio da projeção de \emph{slides} informativos preparados com
  antecedência. Cada grupo deverá estabelecer conexões com trechos da
  obra trabalhada e constatar, por meio da análise comparativa,
  semelhanças entre as situações apresentadas no livro lido e nos dados
  obtidos por meio da pesquisa. Também é sugerida a produção de uma
  \emph{playlist} contendo músicas dos países pesquisados que tratem 
  da questão da escravidão e seus impactos na vida dos cidadãos.
  \BNCC{EM13LP34}

   \end{enumerate}

 \paragraph{Tempo estimado} Três a quatro aulas de 50 minutos.




\section{Aprofundamento}


Nesta seção, desenvolvemos um trabalho de aprofundamento que, em diálogo
com a formação continuada de professores, oferece subsídios para a
abordagem do texto literário. A leitura em sala de aula \emph{Narrativa
de William Wells Brown} pode ser enriquecida pelo aprofundamento no
universo literário em que a obra está inserida.

\subsection{Por que ler esta obra hoje?}

\SideImage{Na legenda da ilustração, o poema de Thomas Campbell \emph{United States! your banner wears/ Two emblems--one of fame:/ Alas! the other that it bears/ Reminds us of your shame:/ Your banner's constellation types/ White freedom with its stars, /But what's the meaning of the stripes?/ They mean your negroes' scars}. (Internet Archive; Domínio Público)}{PNLD0038-07.png}

É fundamental observar a obra como o \textbf{relato da vida} de um
ex-escravizado norte-americano que conseguiu fugir de um sistema 
cruel e estruturado. O relato autobiográfico de Brown permite
refletir sobre o universo da escravidão a partir do testemunho de uma
das vítimas do regime. No prefácio à primeira edição, \textsc{j.\,c.} Hathaway
assinalou que William Wells Brown vivenciou quase que todos os aspectos
do sistema escravagista norte-americano, revelando como as
instituições do país estavam voltadas para sustentar a submissão racial
e servil dos cativos.

Além disso, é uma obra que teve grande peso na luta abolicionista.
Deve-se lembrar que os estados do Sul chegaram a ter uma população de
negros escravizados orbitando entre 51 e 54\% do total populacional.
Esse número é extremamente alto e demonstra a forte estruturação do
sistema escravista. Além do mais, nesse período, o valor de um negro
escravizado nos Estados Unidos era tão elevado que perdia apenas para a
propriedade de terras.

Vale ressaltar também que Brown sempre acreditou nas tradições
anglo-americanas e cristãs protestantes, vindas de séculos anteriores,
enquanto buscava o imediato fim da escravidão e da subordinação racial
nos Estados Unidos. Por isso, o autor foi visto como um reformista e não
exatamente como um revolucionário, e essa classificação impactou a
difusão de suas ideias.

\subsection{Um testemunho da escravidão}

A metade do século \textsc{xix} foi um período em que a escravidão
norte-americana estava consolidada, anteriormente à Guerra Civil
(1860--1865), que romperia com a estrutura escravagista do país. Nesse
período, outras obras com a mesma temática foram publicadas, como as de
Frederick Douglass, que chegou a discutir publicamente com Brown, e
a de Harriet Ann Jacobs, autora de \emph{Incidentes da vida de uma escrava}
(1861). William Wells Brown foi um homem
atuante em seu tempo, que defendeu a liberdade dos negros e contribuiu
com o conjunto de publicações de autobiografias de ex-escravizados.

Assim como outros livros de escravizados fugidos do Sul para o Norte dos
Estados Unidos, publicados no século \textsc{xix}, a \emph{Narrativa de William
Wells Brown, escravo fugitivo} também diz respeito ao contexto social e
econômico norte"-americano do período.

Mas a mensagem principal é pela abolição da escravidão nos Estados Unidos
e no resto do mundo. William Brown foi, antes de qualquer outra coisa, um abolicionista --- que viveu na pele a terrível experiência da escravidão. Sua obra é também famosa por denunciar, em detalhes, o ambiente de violência
física e psicológica que o sistema escravista imprimia sobre suas vítimas.

% É interessante estabelecer relações entre a obra e o contexto em que ela
% foi produzida. Para isso, é recomendável estabelecer parcerias com
% professores da área de Ciências Humanas. Entre as estratégias de leitura
% indicadas na BNCC, estão aquelas que, ``por um lado, permitam a
% compreensão dos modos de produção, circulação e recepção das obras
% {[}...{]} e o desvelamento dos interesses e dos conflitos que permeiam
% suas condições de produção {[}...{]}''.

\subsection{Uma vida marcada pela história}

William Wells Brown foi um escritor norte"-americano, nascido em
Kentucky no ano de 1814. Ele veio ao mundo já na condição de
escravizado, uma vez que sua mãe era escrava. Aos 19 anos, Brown
consegue fugir de seu senhor e se estabelece em Ohio no ano de 1834.

Nascido escravo, um dos pontos mais marcantes da trajetória de William
Wells Brown ocorreu quando ele tinha apenas 16 anos. Com essa idade, foi
vendido por seu senhor para um traficante de escravos no Rio Mississípi
e acabou trabalhando com brancos escravizadores, participando do tráfico
de pessoas para o trabalho forçado. Esse fato acabou sendo crucial para
o desenvolvimento das experiências relatadas por Brown enquanto
escritor, por permitir compreender o funcionamento do sistema escravista
nos Estados Unidos, embora tenha sido polêmico para sua reputação.

Nove anos após sua fuga, William Wells Brown começou a dar palestras
para o público abolicionista dos estados do Norte, contando sua história
com uma boa retórica e já iniciando os preparativos para a escrita de seu
livro, publicado originalmente em 1847.

Logo em seguida, ele se fixou em Boston, local que, no século \textsc{xix}, era o
grande centro do abolicionismo estadunidense. Lá ele se engajou à causa
abolicionista, estudou e se tornou um grande escritor. Além da causa da
abolição, Brown também defendeu outras pautas sociais, como o sufrágio
feminino, a redução do consumo de álcool e a reforma penitenciária,
entre outras questões. Em meados de 1847, publica sua grande obra,
\emph{Narrativa de William Wells Brown, escravo fugitivo}.

Desse modo, o autor presenciou pessoalmente toda a crueldade praticada
contra os escravizados e a brutalidade do tráfico de pessoas.
Testemunhou também outra triste característica da escravidão
norte-americana, marcada, entre outras tragédias, pela indiferença com
que eram realizadas as separações de familiares. O próprio autor foi
separado de sua mãe, irmão e irmã.

Essas experiências tornam pungente o relato de Brown e revelam a
violência por trás do sistema escravista nos Estados Unidos do século
\textsc{xix}.

%\subsection{A escravidão nos Estados Unidos}

\subsection{A escravidão na literatura}

Além de autobiografia, o livro de Brown, assim como outros do mesmo
gênero, é uma excelente fonte de documentação histórica e revela a
relação opressiva e a subordinação racial e servil imposta pelos brancos
escravistas aos negros escravizados, no Sul dos Estados Unidos do século
\textsc{xix}.

Outro elemento que merece destaque é o fato de que a obra, assim como
outras autobiografias de negros escravizados, visavam combater a
literatura sulista, branca e escravocrata, a qual descrevia a vida nas
fazendas do Sul como algo idílico e bom, tanto para os brancos senhores
de terras quanto para os negros escravizados. Até mesmo um termo foi
criado para caracterizar esse movimento contrário à literatura
escravista: ``a escola literária do fugitivo heroico norte-americano''.
Sobre essa vertente literária, cabe ressaltar que a maioria dos
escritores a ela relacionados saíam dos estados escravistas mais
próximos ao Norte do país, ou seja, esses escritores fugidos da
escravidão não estavam nas fazendas de algodão do Extremo Sul dos
Estados Unidos. Esse dado geográfico promove um recorte histórico e
social na realidade retratada pela maior parte desses textos.

\subsection{Atividades para o aprofundamento da pesquisa}

\subsubsection{Escrever sobre o direito à liberdade}

A leitura de relatos acerca das vivências de ex"-escravizados nos
Estados Unidos do século \textsc{xix} convida à reflexão sobre temas ainda hoje
atuais, mas também possibilita o trabalho no campo das competências
socioemocionais. O registro de trajetórias marcadas pelo sofrimento e
pela superação das adversidades estimula o desenvolvimento da empatia
e da solidariedade em relação a dores alheias. Por isso, na esfera da
subjetividade e dos projetos de vida, os testemunhos de vida podem
servir como ponto de partida para debates e rodas de conversa sobre
temas relevantes para o universo dos adolescentes. Em parceria com
professores de Ciências Humanas, proponha a elaboração individual de
\textbf{crônicas argumentativas} com base nos temas do respeito às
diferenças e do direito à liberdade. 
\BNCC{EM13LGG103} % Escrita criativa

Verifique as visões da turma
sobre a noção de liberdade e observe que valores estão ligados, do
ponto de vista deles, à existência dessa condição. Em seguida,
relacione o estatuto do sujeito livre à necessidade de respeito e
tolerância das diferenças em todos os níveis da existência, mas,
sobretudo, no que tange a questões culturais e étnico"-raciais. Para
fundamentar a argumentação, estimule a pesquisa virtual a passagens da
Constituição que asseguram as liberdades individuais e coletivas.
Oriente, depois, o registro dos argumentos e contra"-argumentos no
caderno, para que seja possível articulá"-los em uma unidade
argumentativa coesa e coerente. Reserve um momento para a produção das
primeiras versões das crônicas, que devem ser preferencialmente
digitadas. Incentive a inserção de exemplos extraídos da atualidade, a
partir de novas consultas a \emph{sites} confiáveis de internet.
Partindo do cotidiano, é possível que a argumentação seja construída
com base em um olhar poético e sensível para a realidade vivida por
muitas pessoas em todo o mundo. O combate à discriminação deve ser
incentivado e argumentos que fujam do senso comum podem ser
construídos a partir da proposição de formas de intervenção concreta
sobre os problemas apontados. As produções podem ser compartilhadas
com os colegas e professores envolvidos na atividade, para que
comentários críticos e construtivos sejam feitos, de forma respeitosa
e democrática. Ao final, as versões definitivas podem ser publicadas
no \emph{site} do colégio, em redes sociais ou em um blog da turma,
após um processo de revisão e edição das produções textuais.

\Image{O traficante de escravos e o autor conduzindo uma grupo de escravos para o mercado do sul (Internet Archive; Domínio Público)}{PNLD0038-05.png}

\subsubsection{O preconceito racial no Brasil}

A luta antirracista é uma das principais pautas dos movimentos negros
em todo o mundo. O combate ao preconceito étnico"-racial deve ser
estimulado no cotidiano da sala de aula, sobretudo porque diversos
discursos veiculados socialmente perpetuam o chamado \emph{racismo
estrutural}, que muitas vezes não é notado por grande parte dos
falantes. 
\BNCC{EM13CHS502}

Por isso, a leitura de relatos autobiográficos, produzidos
no contexto da escravidão, pode fundamentar as discussões em sala de
aula. A partir das experiências de leitura de narrativas originárias
dos Estados Unidos, proponha a elaboração, individual ou em dupla, de
um \textbf{artigo de opinião} sobre o preconceito racial no Brasil.
Para isso, com auxílio de professores da área de Ciências Humanas, é
importante estabelecer paralelos comparativos sobre o regime
escravista nos dois países, bem como sobre as consequências
histórico"-sociais da escravidão. Atitudes segregacionistas, racismo
internalizado e estrutural, discriminação nas esferas pública e
privada, violência e desigualdade são aspectos que podem ser
explicados historicamente. Além disso, professores de Ciências da
Natureza podem apresentar argumentos científicos, sobretudo ligados ao
campo de estudo da Genética, para combater o preconceito e as teorias
eugenistas que ganharam força nos séculos \textsc{xix} e \textsc{xx}. A partir dos
debates interdisciplinares, os estudantes poderão articular os
argumentos sob a forma de artigos que discutam, de forma coesa e
coerente, as causas históricas, os impactos sociais e formas concretas
de atuação social e intervenção quanto ao problema do racismo na
sociedade brasileira. É importante incluir elementos trazidos das
leituras, mas também é recomendável estimular a consulta a livros e
\emph{sites} confiáveis sobre a luta antirracista. As primeiras
versões dos textos podem ser compartilhadas, em sala de aula, com os
colegas e professores envolvidos. Na sequência, as produções podem ser
revisadas, digitadas e editadas com auxílio do computador. Ao final,
os textos podem ser publicados no \emph{site} da escola, em redes
sociais ou no blog da turma destinado ao registro das experiências de
leitura.

\subsubsection{Novas interpretações de monumentos públicos}

A derrubada de estátuas de figuras históricas associadas, sobretudo,
ao racismo, foi uma expressão de protesto multiplicada em muitos
lugares do mundo, no ano de 2020. O movimento foi disparado pelo
assassinato de George Floyd, cidadão negro asfixiado por um oficial
branco, durante uma abordagem policial na cidade de Minnesota, nos
Estados Unidos. A partir desse fato, proponha a produção individual de
um \textbf{ensaio} escrito, por meio do qual os estudantes analisem
criticamente um aspecto da polêmica intervenção sobre estátuas e
monumentos públicos. 
\BNCC{EM13LGG305}


Para isso, estimule a reflexão sobre aspectos
ligados à questão, que podem servir de base para a formulação da tese
e da argumentação desenvolvida no texto.

Verifique se, na opinião dos alunos, a derrubada das estátuas compõe um
gesto: 

\begin{enumerate}
\item Agressivo, irresponsável e de violência gratuita contra o
patrimônio histórico \textbf{ou} necessário para a conscientização acerca
das injustiças sofridas pelas populações negras ao longo da História;
\item De vandalismo infundado, disfarçado em postura politicamente correta
\textbf{ou} aceitável para expressar a revolta e tornar público o racismo
presente na sociedade; 
\item De destruição da memória de uma nação e de
iconoclastia de objetos de arte \textbf{ou} de reparação junto às vítimas
silenciadas pelo opróbrio da escravidão; 
\item De repúdio à suposta
intocabilidade de figuras históricas, cúmplices e responsáveis pela
escravidão \textbf{ou} de grito de dor, vindo como resposta à segregação
racial, há tanto tempo legitimada e hoje implícita e explícita nas
relações sociais, marcadas pelo racismo estrutural.
\end{enumerate}

Esses tópicos tratam do tema sob duas perspectivas contrárias entre si.
Ao escolher um item para orientar a sua produção, o estudante tomará
partido de um lado ou de outro, mas não deverá desconsiderar os demais
pontos de vista: é preciso também leva"-los em conta ao formular os
argumentos e contra"-argumentos.

Para produzir um ensaio, é preciso mobilizar o repertório sociocultural
e fundamentar os argumentos com base em livros, filmes, letras de
música, obras de arte e textos históricos ou filosóficos, que podem ser
mencionados ao longo do texto. É interessante estimular a consulta a
\emph{sites} de periódicos confiáveis, ligados à divulgação de
atualidades.

O ensaísta é um autor que analisa diferentes aspectos das ideias,
observando cada um dos elementos como se fossem as faces de um prisma.
Lembre"-se, no entanto, de que é preciso defender uma tese central; ao
fazer um ensaio sobre o tema selecionado, é possível adotar diferentes
posturas para construir a argumentação: o aluno poderá acolher uma das
perspectivas, rejeitando a outra, em uma atitude com inclinação ao
formato \emph{ou\ldots{} ou}; também é possível escolher um lugar enunciativo
com inclinação ao formato \emph{e\ldots{} e}, apresentando, para tanto,
soluções alternativas.

Observe, em sala de aula, que as opiniões possíveis a respeito do tema
refletem as vozes às quais cada aluno responderá: elas são, de um lado,
aquelas vozes que comentaram, analisaram e discutiram esse tema na mídia
contemporânea; além disso, de outro lado, há também as vozes dos atores
sociais que, na vivência pretérita, presente e futura dos fatos ligados
à escravidão negra, constituem a História como arena de conflitos
operada discursivamente. Ao comentar, analisar e discutir o tema no
formato de um ensaio, o texto responderá por adesão a determinadas vozes
sociais e não a outras. Em parceria com professores de Ciências Humanas,
reflita profundamente com os alunos sobre o ponto de vista a ser
defendido, uma vez que ser responsivo no ato de criar um ensaio é ser
responsável socialmente.

Os alunos podem elaborar, no caderno, um projeto de texto para organizar
os principais argumentos sobre o tema. É interessante que elaborem mapas
de ideias, façam um \emph{brainstorm} a partir das múltiplas faces da
questão, listem tópicos que auxiliem a posterior elaboração dos
parágrafos e da orientação argumentativa coerente do texto. É importante
desenvolver uma reflexão sobre o aspecto selecionado, verificando se se
as intervenções sobre monumentos do patrimônio público constituem um
gesto agressivo, inconsequente e vão, ou se podem constituir uma
participação necessária, legítima e oportuna na História da humanidade.

O ensaio permite, ao longo do texto, marcas de 1ª pessoa que explicitem
a subjetividade do autor, por meio de expressões como ``Eu creio
que\ldots{}'', ``no meu ponto de vista, \ldots{}'', ``eu penso que\ldots{}'' etc. É
possível até mesmo estabelecer paralelos com experiências pessoais e
fazer breves relatos de situações pessoais que tenham ligação com o
tema. Ao mesmo tempo, esse gênero de texto acolhe modalizadores de
dúvida e incerteza, como ``talvez'', ``é provável que\ldots{}'', ``pode ser
que\ldots{}'', entre outros que captam o processo de construção do pensamento
autoral. Esse gênero textual não visa, portanto, como ocorre em outras
modalidades de argumentação, à construção de um efeito de objetividade,
uma vez que o ensaísta conduz o leitor por um passeio pelo universo das
ideias pessoais e, ao expor um ponto de vista, estimula a reflexão
independente e a formação de opiniões do público. O leitor, portanto,
deve desempenhar um papel ativo e acompanhar o raciocínio do autor,
investigando, completando a análise por meio da autorreflexão e
formulando conclusões consistentes.

A primeira versão escrita do ensaio pode ser compartilhada com os
colegas e com os professores envolvidos na atividade. É recomendável que
os alunos leiam as produções em voz alta e anotem os comentários, as
sugestões e as críticas construtivas. Os posicionamentos podem ser
rediscutidos, com respeito e liberdade de expressão, mas sempre
respeitando os direitos humanos. Após a apresentação, cada aluno fará as
alterações e reformulações necessárias. A segunda versão do texto será
digitada, utilizando o computador, de acordo com a norma padrão da
língua portuguesa. As versões finais podem ser publicadas nas redes
sociais, no \emph{site} da escola ou em um blog destinado à divulgação
dos trabalhos.

\subsubsection{Oficina de poesia sobre a tolerância}


O texto literário oferece possibilidades de reflexão e conhecimento de
situações da vida que podem se aproximar das experiências dos
estudantes ou colocá"-los em contato com realidades vividas por outras
pessoas. Com base na leitura de relatos autobiográficos de
ex"-escravizados norte"-americanos, os alunos podem conhecer
experiências de outros lugares e épocas, que permanecem atuais e
dialogam diretamente com o Brasil da contemporaneidade. Apesar de a
matéria básica das produções ser extraída da vida real, a elaboração
por meio da palavra confere uma dimensão literária às narrativas. Além
de documentos de época, revelam"-se olhares de sujeitos sensíveis,
marcados por situações históricas em contextos de crise. Para
aprofundar o trabalho com conflitos ligados à experiência juvenil,
proponha um mergulho poético nas noções de liberdade e tolerância:
desta vez, os estudantes produzirão \textbf{poemas líricos} para
expressar ideias e sentimentos ligados a esses temas. Como ponto de
partida, é possível retomar produções e discussões feitas ao longo da
leitura, bem como letras de música, pertencentes a gêneros variados,
que tratem igualmente desses valores. Utilizando uma \emph{playlist}
construída coletivamente pelos alunos, privilegiando o repertório do
\emph{hip hop}, reserve um momento para uma oficina de criação
poética. 
\BNCC{EM13LP47} %Sarau


Na sequência, proponha a leitura compartilhada das produções,
comentando"-as e incentivando eventuais reformulações. Por fim, os
textos podem ser digitados, de maneira a compor uma antologia poética
acerca da liberdade e da tolerância, que pode ser publicada no
\emph{site} da escola ou no blog da turma, destinado às atividades
feitas com base nas experiências de leitura.

\subsubsection{Reportagem sobre o tráfico negreiro}


As consequências da escravidão, em diferentes países do mundo,
geraram, ao longo do tempo, experiências semelhantes de preconceito e
discriminação. A leitura de relatos de ex"-escravizados estadunidenses
enriquecem as discussões sobre o tema e permitem traçar paralelos com
realidades diversas, incluindo a da História nacional. No Brasil
contemporâneo, as desigualdades sociais se perpetuam também em
decorrência do passado escravista. Em parceria com professores de
Ciências Humanas, retome as discussões sobre as raízes históricas da
escravidão, com ênfase no Brasil, e proponha uma pesquisa sobre o
tráfico negreiro, desta vez concentrada no contexto brasileiro. Por
meio da pesquisa em livros e \emph{sites} de internet, a turma,
dividida em pequenos grupos, buscará informações sobre números de
pessoas trazidas do continente africano para o Brasil, as
características do sistema social, político e econômico que
sustentavam a escravidão, mapas de rotas do tráfico de escravizados,
relatos acerca do cotidiano nos navios negreiros, as leis que
procuraram modificar esse cenário e poemas produzidos sobre essas
experiências, sobretudo pela vertente socialmente engajada do
Romantismo (conhecida como \emph{condoreira} e praticada, no Brasil,
sobretudo por Castro Alves). A partir dos dados coletados, oriente a
produção de uma \textbf{reportagem} escrita sobre o tráfico negreiro
no Brasil, em linguagem de divulgação, com o objetivo de promover
reflexões sobre questões atuais do país, tanto nas relações pessoais
quanto no mundo do trabalho. As versões finais dessas reportagens,
digitadas e editadas pelos grupos com auxílio do computador, podem ser
compartilhadas no \emph{site} da escola ou no blog da turma, destinado
à publicação de produções ligadas às experiências de leitura.
\BNCC{EM13CHS605} % Direitos Humanos

\section{Sugestões de referências complementares}\label{sugestoes}

\subsection{Filmes}

\begin{itemize}
\item
\textit{A cor púrpura}. Direção: Steven Spielberg (\textsc{eua}, 1985).

Após sofrer violências em casa, a jovem Celie enfrenta trajetórias de
dor e superação. Triste e solitária, ela escreve cartas para a irmã, até
que a chegada da amante do marido transforma seu destino.

\item
\textit{Besouro}. Direção: João Daniel Tikhomiroff (Brasil, 2009).

Na Bahia dos anos 20, o pequeno Manoel é apresentado à capoeira pelo
Mestre Alípio. Ao crescer, Besouro, como passa a ser chamado, recebe a
missão de defender seus semelhantes da opressão e do racismo.

\item
\textit{Django Livre}. Direção: Quentin Tarantino (\textsc{eua}, 2013).

O filme é construído em torno da improvável parceria entre Django, um
escravo liberto, e o Dr. Schultz, um caçador alemão de recompensas.
Juntos, irão atrás dos criminosos mais perigosos do sul dos EUA e
tentarão resgatar a esposa de Django.

\item
\textit{Doze anos de escravidão}. Direção: Steve McQueen (\textsc{eua}, 2013).

Vencedor de três estatuetas no Oscar de 2014, o filme conta a história
real de Solomon Northup, negro livre que foi escravizado por 12 anos no
sul dos EUA, ao cair na armadilha de uma oferta de emprego.

\item
\textit{Histórias cruzadas}. Direção: Tate Taylor (\textsc{eua}, Índia,
Emirados Árabes Unidos, 2012).

Adaptação do livro \emph{A Resposta}, este longa conta a história de
Skeeter, jovem que quer ser escritora. Ela começa a entrevistar mulheres
negras que deixaram suas vidas para trabalhar na criação dos filhos da
elite branca. Sua iniciativa irá desagradar muita gente.

\item
\textit{O sol é para todos}. Direção: Robert Mulligan (\textsc{eua}, 1963).

Baseado no romance homônimo, o filme conta a história de Tom Robinson,
jovem negro injustamente acusado de violentar uma mulher branca, e do
advogado Atticus Finch, que enfrentou a rejeição da cidade para defender
o réu.
\end{itemize}

\subsection{Lugares para visitar}

\begin{itemize}
\item\textbf{Museu Afro Brasil}


O \href{http://www.museuafrobrasil.org.br/}{Museu Afro Brasil}, localizado no Parque Ibirapuera, zona sul de São Paulo, reúne um
acervo contendo mais de cinco mil obras relacionadas à cultura africana e
afro-brasileira.
\end{itemize}

\section{Bibliografia comentada}

\begin{itemize}
\item
\textsc{adichie}, Chimamanda Ngozi. \textit{Americanah}. São Paulo: Companhia
das Letras, 2014.

Em busca de alternativas às universidades nigerianas, a jovem Ifemelu
emigra para os Estados Unidos. Enquanto se destaca no meio acadêmico,
ela depara com a questão racial e com as dificuldades da vida de mulher
negra e estrangeira.

\item
\textsc{adichie}, Chimamanda Ngozi. \textit{O perigo de uma história única}.
São Paulo: Companhia das Letras, 2019.

A escritora nigeriana defende que nosso conhecimento é construído pelas
histórias que escutamos e que, quanto mais diversas e numerosas forem
essas narrativas, mais completa será nossa compreensão sobre o mundo.

\item
\textsc{alexander}, Michelle. \textit{A nova segregação}. São Paulo: Boitempo,
2018.

Esta obra desafiou a opinião de que o governo Obama assinalava o advento
de uma nova era pós-racial. A autora analisa o sistema prisional dos EUA
e expõe como o racismo estrutural opera nas sociedades ocidentais.


\item
\textsc{angelou}, Maya. \textit{Eu sei por que o pássaro canta na gaiola}.
Bauru: Astral Cultural, 2018.

Neste romance emocionante, a autora conta a história de Marguerite Ann
Johnson, garota negra criada no sul dos EUA pela avó, e dá voz a jovens
que, como ela, enfrentam muitos preconceitos.

\item
\textsc{evaristo}, Conceição. \textit{Olhos d'água}. Rio de Janeiro: Pallas,
2014.

Com uma escrita lírica e sensível, a escritora mineira reúne breves
contos sobre o cotidiano, concentrando o foco de interesse sobre a vida
da população afro-brasileira.

\item
\textsc{horne}, Gerald. \textit{O sul mais distante}. São Paulo: Companhia das
  Letras, 2010.

O autor vê a escravidão em termos hemisféricos e defende que o sul
escravista dos EUA via, em uma aliança com o Brasil (o ``sul mais
distante''), uma forma de proteção contra um futuro embate com o norte
estadunidense, na Guerra de Secessão.

\item
\textsc{lee}, Harper. \textit{O sol é para todos}. São Paulo: José Olympio,
2006.

No sul dos EUA da década de 1930, uma garotinha esperta e observadora
relata a saga do pai, um advogado que arrisca tudo para defender um
homem negro, injustamente acusado de cometer um crime.

\item
\textsc{marquese}, Rafael; \textsc{salles}; Ricardo (org.). \textit{Escravidão e
  capitalismo histórico no século \textsc{xix}: Cuba, Brasil, Estados Unidos}.
  Rio de Janeiro: Civilização Brasileira, 2016.

O livro reúne ensaios de historiadores brasileiros e estrangeiros sobre
a escravização de negros nas Américas, ao longo do século \textsc{xix}.

\item
\textsc{mccullers}, Carson. \textit{A convidada do casamento}. São Paulo: Novo
  Século, 2008.

Frankie é uma menina cujos sonhos se chocam com sua rotina pacata,
compartilhada com seu primo e com Berenice, a cozinheira negra da casa.
Em um verão solitário nos EUA, suas incertezas aumentam quando recebe a
notícia do casamento do irmão.

\item
\textsc{morgan}, Edmund S. ``Escravidão e liberdade: o paradoxo americano''.
  In \textit{Estudos Avançados 14} (38), p. 21-50. São Paulo:
 \textsc{usp}, 2000. São Paulo, 2000. (Disponível em:
  \href{http://www.revistas.usp.br/eav/article/view/9507}{revistas.usp.br}.
  Acesso em: 8 de fev. de 2021.)

O autor busca compreender como o povo estadunidense pôde, desde o
princípio, desenvolver uma dedicação às ideias de liberdade e dignidade
humanas, e simultaneamente apoiar um sistema de trabalho que negava
diariamente esses valores.

\item
\textsc{northup}, Solomon. \textit{Doze anos de escravidão}. São Paulo:
Penguin, 2014.

O livro é o relato real e assombroso de um negro livre que, atraído por
uma proposta de trabalho, abandona a segurança do Norte dos EUA e acaba
sendo sequestrado e vendido como escravo no Sul.


\item
\textsc{porter}, Regina. \textit{Os viajantes}. São Paulo: Companhia das
  Letras, 2020.

Por meio das múltiplas perspectivas de personagens, a obra apresenta uma
trama que avança e volta no tempo. O livro traça um panorama da vida nos
Estados Unidos entre a década de 1950 e a eleição de Barack Obama para
presidente.

\item
\textsc{stockett}, Kathryn. \textit{A resposta}. Rio de Janeiro: Bertrand
Brasil, 2012.

A obra conta a história de Eugenia, jovem que deseja ser escritora. Ela
tem um plano brilhante, mas perigoso: escrever um livro em que
empregadas domésticas negras relatem seus relacionamentos com patroas
brancas do Mississípi, nos anos 1960.

\item
\textsc{stowe}, Harriet Beecher. \textit{A cabana do pai Tomás}. São Paulo:
Carambaia, 2018.

Um dos romances mais importantes da época da Guerra Civil Americana, a
obra conta a história do escravo Tom e influenciou intensamente as lutas
contra a escravidão.

\item
\textsc{walker}, Alice. \textit{A cor púrpura}. São Paulo: José Olympio, 2009.

Vencedora do Prêmio Pulitzer, a autora narra com sensibilidade a vida de
Celie, uma mulher negra no sul dos Estados Unidos, do começo do século
\textsc{xx}, que sofreu abusos do padastro e depois do marido.

\end{itemize}



\end{document}

