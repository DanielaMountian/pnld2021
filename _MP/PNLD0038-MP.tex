\documentclass{extrarticle}
\usepackage{manualdoprofessor}
\usepackage{fichatecnica}
\usepackage{lipsum,media9,graficos}
\usepackage[justification=raggedright]{caption}
\usepackage{bncc}
\usepackage[relicario]{logoedlab}
\begin{document}

\begin{document}


\newcommand{\AutorLivro}{William Wells Brown}
\newcommand{\TituloLivro}{Narrativa de William Wells Brown, escravo fugitivo}
\newcommand{\Tema}{Diálogos com a sociologia e com a antropologia}
\newcommand{\Genero}{Diário, biografia, autobiografia, relatos, memórias}
% \newcommand{\imagemCapa}{PNLD0038-01.png}
\newcommand{\issnppub}{---}
\newcommand{\issnepub}{---}
% \newcommand{\fichacatalografica}{PNLD0038-00.png}
\newcommand{\colaborador}{\textbf{Fulano de Tal} é uma pessoa incrível e vai fazer um bom serviço.}


\title{\TituloLivro}
\author{\AutorLivro}
\def\authornotes{\colaborador}

\date{}
\maketitle
\tableofcontents




\section{Atividades 1}

\BNCC{EM13LP26}

\subsection{Pré-leitura}
\subsection{Leitura}
\subsection{Pós-leitura}



\section{Atividades 2}

\subsection{Pré-leitura}
\subsection{Leitura}
\subsection{Pós-leitura}

\lipsum
\end{document}

