\documentclass[11pt]{extarticle}
\usepackage{manualdoprofessor}
\usepackage{fichatecnica}
\usepackage{lipsum,media9,graficos}
\usepackage[justification=raggedright]{caption}
\usepackage[one]{bncc}
\usepackage[relicario]{../edlab}
\begin{document}


\newcommand{\AutorLivro}{William Wells Brown}
\newcommand{\TituloLivro}{Narrativa de William Wells Brown, escravo fugitivo}
\newcommand{\Tema}{Diálogos com a sociologia e com a antropologia}
\newcommand{\Genero}{Diário, biografia, autobiografia, relatos, memórias}
\newcommand{\imagemCapa}{./images/PNLD0038-01.png}
\newcommand{\issnppub}{---}
\newcommand{\issnepub}{---}
% \newcommand{\fichacatalografica}{PNLD0038-00.png}
\newcommand{\colaborador}{\textbf{Eduardo Modesto de Carvalho, Bruno Gradella e Vicente Castro}}


\title{\TituloLivro}
\author{\AutorLivro}
\def\authornotes{\colaborador}

\date{}
\maketitle

\begin{abstract}
Este Manual tem como objetivo fornecer subsídios para o trabalho com a
obra literária \emph{Narrativa de William Wells Brown}, escrita por ele
mesmo.

Nascido no Kentucky em 1814, William Brown foi um escritor estadunidense.
Exerceu, durante sua vida, diversas funções de trabalho doméstico e de serviços,
incluindo assistente de tipógrafo, auxiliar de alfaiate, lavrador,
barbeiro, cavalariço e cocheiro. Brown também foi assistente de um
traficante negreiro, responsável por preparar pessoas escravizadas
para a venda em Natchez, Mississippi, e Nova Orleans, Luisiana. Essa
experiência, que Brown vivenciou aos 16 anos, se destaca como um dos 
episódios mais emocionantes e aterroradores da sua \emph{Narrativa}. 

Acompanharemos, neste livro, as diversas histórias de William Brown se
esquivando das armadilhas do sistema escravista para conquistar sua liberdade.
Dentre essas, podemos destacar o episódio em que ele foi mandado, por seu dono,
com bilhete com a instrução de ser açoitado por um carcereiro na região 
do Mississipi. Temendo o conteúdo do bilhete, já que não sabia lê"-lo,
Brown, pensou rápido e pagou outro homem negro para levar o papel até 
a cadeia, onde seu substituto recebeu ``vinte chibatadas nas costas
nuas''.

Outra vez que se usou de artimanhas para garantir sua busca pela liberdade
foi quando simulou aceitar uma união com outra escravizada, encomendada
pelos seus ``donos" a fim de mantê-lo na propriedade, além de garantir
mais escravizados conforme os filhos do casal nascessem. Ciente dos motivos
ocultos, Brown dissimula e foge do ``combinado".

Publicada pela Sociedade Antiescravista de Massachusetts em julho de 1847, 
a \emph{Narrativa de William Wells Brown, escravo fugitivo. Escrita por ele mesmo} 
é um apelo à abolição da escravidão nos Estados Unidos e, por consequência, 
em todo o mundo. É um testemunho em primeira pessoa da escravidão norte-americana 
e coloca os leitores cara a cara com o ambiente de violência social que impactava 
radicalmente a personalidade, a família e o desenvolvimento moral entre 
os escravizados. 

Esperamos que este material seja muito útil no trabalho do professor em sala de aula!



\end{abstract}

\tableofcontents


\Image{Gravura de William Wells Brown, 1847 (Autor Desconhecido; Domínio Público )}{PNLD0038-03.png}

\section{Atividades 1}

%\BNCC{EM13LP26}

\subsection{Pré-leitura}

%\BNCC{EM13LGG302}
%\BNCC{EM13LGG704}
%\BNCC{EM13LP10}
%\BNCC{EM13LP19}

% \paragraph{Tema}
% \paragraph{Conteúdo}
% \paragraph{Objetivo}
% \paragraph{Justificativa}
% \paragraph{Metodologia}
%   \begin{enumerate}
%   \end{enumerate}
% \paragraph{Tempo estimado}

O substantivo \emph{abolicionismo} nomeia um movimento
político que, em diferentes lugares do mundo, visava ao fim da
escravidão. Ainda que tenha variado em formatos de atuação, em
metodologias e políticas estratégicas, esse movimento, em geral, buscava
o fim do tráfico negreiro, a emancipação dos escravos e a integração da
população negra na sociedade. Nas aulas de Língua Portuguesa, discuta o
significado desse vocábulo no contexto dos estudos da significação. Toda
unidade lexical carrega consigo ideias e valores relacionados às
situações de interação social em que é empregada. Utilize poemas da 3ª
geração romântica no Brasil para mostrar, aos estudantes, o modo como o
movimento abolicionista foi divulgado pela literatura. Observe
características ligadas ao Romantismo, ajustadas aos ideias de
transformação social: essa literatura engajada cultivava ideais de
liberdade e de revisão dos valores da sociedade. Antes de iniciar a
leitura de \emph{Narrativa de William Wells Brown}, mostre como a
literatura brasileira esteve ligada às aspirações abolicionistas no
país. Uma fonte interessante de consulta para essa atividade é a obra
\emph{As utopias românticas}, de Elias Thomé Saliba (São Paulo: Estação
Liberdade, 2008), que fornece elementos para analisar o Romantismo como
movimento sociocultural enraizado na história europeia, ligado às
revoluções burguesas, com ressonâncias em outros países e continentes.
Ao explorar a temática da luta pelo fim da escravidão na literatura,
proponha também outras atividades:

\begin{itemize}
\item
  Após analisar coletivamente alguns poemas românticos da vertente
  condoreira -- representada, sobretudo, por Castro Alves --, promova um
  diálogo interdiscursivo com outras produções do século XIX, nacionais
  e estrangeiras. Com base na definição do substantivo
  \emph{abolicionismo}, proponha uma pesquisa, em livros e \emph{sites}
  de internet, a respeito de publicações abolicionistas, brasileiras e
  internacionais, publicadas em diversos gêneros e formatos. É possível
  encontrar antigas reportagens e notícias, pertencentes ao campo
  jornalístico-midiático, ou mesmo panfletos abolicionistas do século
  XIX. Nas aulas de leitura, com o repertório construído sobre a poesia
  e o jornalismo oitocentistas, a partir da pesquisa de diferentes
  gêneros textuais, organize a produção de um \textbf{manifesto}
  \textbf{abolicionista}, colocando a turma -- dividida em pequenos
  grupos -- no papel de organizadores de um jornal abolicionista do
  século XIX. Estimule a defesa de valores associados a liberdade,
  igualdade e respeito pelas diferenças. Com essa atividade, pretende-se
  também capacitar o aluno a realizar um cotejo entre diversas fontes de
  pesquisa que abordam uma mesma temática. Ao mesmo tempo, a turma
  construirá um repertório linguístico e literário que possibilitará
  mergulhar nos relatos contidos em \emph{Narrativa de William Wells
  Brown.}
\end{itemize}

Por fim, sugerimos a leitura conjunta, com a turma sentada
confortavelmente em círculo, da obra \emph{Narrativa de William Wells
Brown}. Ela pode ser realizada na sala de aula, na biblioteca ou na sala
de leitura, e diferentes alunos podem ler, destacar e comentar passagens
marcantes ou mesmo interpretar diferentes personagens. A forma do relato
autobiográfico favorece a oralidade e a imersão na história por meio do
ritmo e da sonoridade característicos dos depoimentos e confissões.
Ajuste o cronograma de leitura conforme o perfil do grupo e estimule
diálogos intertextuais e interdiscursivos, comentários e
compartilhamento de impressões, e construções coletivas de análises e
reflexões.

\subsection{Leitura}

%\BNCC{EM13LGG103}
%\BNCC{EM13LP02}
%\BNCC{EM13LP48}

% \paragraph{Tema}
% \paragraph{Conteúdo}
% \paragraph{Objetivo}
% \paragraph{Justificativa}
% \paragraph{Metodologia}
%   \begin{enumerate}
%   \end{enumerate}
% \paragraph{Tempo estimado}

A obra traz memórias e comentários, em primeira mão, de uma
pessoa que passou parte da vida reduzida à condição de escravizada.
Dessa forma, o leitor tem acesso a um universo que, muitas vezes, não é
abordado na escola, uma vez que o relato individual é comumente
sobreposto por documentos oficiais e textos produzidos por pessoas
brancas, por exemplo. Essa abordagem acaba, por vezes, construindo
apenas uma visão superficial ou parcial das narrativas, e é por isso que
nos últimos anos tem-se valorizado mais a utilização de diários,
memórias autobiográficas, entrevistas e depoimentos para reconstruir os
contextos de produção e circulação dos textos. Solicite aos alunos que
entrevistem pessoas mais velhas -- familiares, vizinhos ou membros da
comunidade escolar -- e coletem informações a respeito do cotidiano e
das relações sociais e de trabalho. Sugere-se que se peça para os
entrevistados fazerem uma comparação da fase atual com outros da vida. O
objetivo desta atividade é indicar ao aluno como a experiência
individual pode ser distinta, e por vezes mais rica, do que o simples
registro histórico de um fato. O material coletado pode ser inicialmente
organizado na forma de um quadro mural, antes de ter o formato de um
\textbf{perfil biográfico}. Para o desenvolvimento da atividade
simultaneamente à leitura da obra, siga as etapas propostas:

\Image{Representação do autor enquanto personagem principal do livro (Internet Archive; Domínio Público)}{PNLD0038-04.png}

\begin{itemize}
\item
  Esta proposta de produção textual complementa o trabalho de leitura da
  \emph{Narrativa de William Wells Brown}, juntamente com o
  desenvolvimento de habilidades ligadas ao processo de autoconhecimento
  e formação da identidade. As reflexões sobre a subjetividade podem ser
  exploradas segundo o eixo das relações estabelecidas entre o eu e os
  outros, uma vez que o sujeito se constitui a partir das relações com a
  alteridade. Para isso, os alunos são estimulados a escolher uma pessoa
  do convívio diário e praticar a escrita de um \textbf{perfil
  biográfico}. Será preciso desenvolver métodos de pesquisa de campo e,
  a partir de entrevistas -- registradas em anotações e gravações de
  áudio e vídeo, feitas com o uso de tecnologias digitais de informação
  e comunicação --, os estudantes produzirão versões do perfil e
  compartilharão as produções no formato de um álbum fotográfico digital
  e de um programa de \emph{podcasts}.
\end{itemize}

Em relação às competências e habilidades socioemocionais, o trabalho com
podcasts em sala de aula pode auxiliar a comunicação e as atividades
criativas.

Quanto ao gênero de texto proposto, diferentemente da biografia, que
procede a um levantamento exaustivo de dados sobre um indivíduo, o
perfil biográfico destaca alguns fatos relevantes da vida do
entrevistado e os organiza em uma sequência cronológica coerente, no
formato de um relato em 3ª pessoa.

Estabeleça uma comparação com a \emph{Narrativa de William Wells Brown}
e com passagens da autobiografia em que o sujeito apresenta depoimentos
sobre experiências passadas, opiniões sobre a escravidão e expectativas
em relação ao futuro.

\begin{itemize}
\item
  Para a criação do perfil biográfico, juntamente com a leitura da obra,
  o caráter documental pode se aliar a um olhar poético e sensível sobre
  a realidade retratada. Para aprofundamento acerca das questões
  referentes à escrita biográfica, sugere-se a consulta a um dos mais
  completos estudos sobre o tema, publicado em língua portuguesa:
  \emph{O desafio biográfico: escrever uma vida}, de François Dosse (São
  Paulo: EDUSP, 2009). Nessa obra, fundamental para as aulas de produção
  de texto, observa-se que o discurso biográfico permite refletir, de
  maneira aprofundada, sobre os procedimentos empregados para narrar os
  acontecimentos da vida de sujeitos com existência histórica
  comprovada. Dosse permite considerar a biografia e o perfil biográfico
  como gêneros textuais híbridos, ancorados sempre em duas dimensões
  complementares -- a histórica e a ficcional --, uma vez que a
  reprodução de eventos reais, vividos por alguém no passado, não é
  realizada apenas objetivamente, mas envolve o emprego da imaginação e
  da visão de mundo de um autor, que seleciona, filtra, organiza e
  interpreta os dados. Recomenda-se igualmente a consulta de: (a)
  BAKHTIN, Mikhail. O autor e o herói. In: \emph{Estética da criação
  verbal}. São Paulo: Martins Fontes. 1997, p. 23-220 e (b) VOLOSHINOV,
  V. N. Discourse in life and discourse in art. In \emph{Freudianism}.
  New York/ San Francisco/ London: Academic Press, 1976, p. 93-116.
\end{itemize}

\Image{Na legenda da ilustração: ``O traficante de escravos e o autor conduzindo uma grupo de escravos para o mercado do sul'' (Internet Archive; Domínio Público)}{PNLD0038-05.png}

Entre biógrafo e biografado, sobressai o que, em termos bakhtinianos,
pode-se chamar de \emph{olhar exotópico} (BAKHTIN, 1997: 23-42), ou
seja, o desdobramento da visão comprometida e ética de um sujeito sobre
outro -- presente, por exemplo, na relação literária entre o autor e o
herói --, de maneira a fazer com que, sob essa nova perspectiva, sejam
revelados aspectos do sujeito observado a que este jamais teria acesso,
sobretudo por falta de distanciamento ou de uma visão totalizante acerca
de si. Ainda sob a perspectiva bakhtiniana, Voloshinov (1976: 93-116),
ao tratar da distinção entre o "discurso da vida" e o "discurso da
arte", assinala a fusão ocorrida entre os discursos verbais e os eventos
da vida, sempre a partir do horizonte espacial e ideacional do sujeito
enunciador; para o pensador russo, integrante do Círculo de Bakhtin, o
discurso jamais reflete as situações extraverbais como um espelho faria
ao refletir um objeto, mas, devido à interferência de julgamentos
sociais e ideológicos, supõe uma conclusão avaliativa e analítica acerca
das situações observadas. O componente imaginativo também poderia ser
incluído entre os elementos que influenciam o tratamento discursivo da
realidade, sobretudo no âmbito do discurso biográfico.

De modo complementar à visão apresentada por Bakhtin/ Voloshinov, Dosse
aponta para um recurso fundamental, sempre empregado pelos autores
praticantes do gênero biográfico durante o trabalho de releitura e
reconstrução do passado, possibilitado pelo trabalho com a memória: "Não
apenas o biógrafo deve apelar para a imaginação em face do caráter
lacunar de seus documentos e dos lapsos temporais que procura preencher,
como a própria vida é um entretecido constante de memória e olvido"
(DOSSE: 2009, 55). Segundo o autor, a biografia -- esse "gênero espúrio,
fruto do casamento desnaturado da ficção com os fatos", como disse certa
vez Virginia Woolf (apud Dosse, 2009: 62) -- é composta por uma
combinação de erudição, criatividade artística e intuição empática em
relação ao indivíduo biografado.

O fazer biográfico compreende, ainda, alguns procedimentos canônicos
(DOSSE, 2009: 56-61), tais como: (a) a observância da ordem cronológica
dos fatos; (b) a centralização do biografado no decorrer da obra; (c) a
apreensão de detalhes singularizantes, capazes de individualizar o
biografado; (d) a consideração da relação imanente com a morte, sempre
presente no fazer literário e fundamental para textos que tratam da
existência humana, marcada pela finitude; (e) a adoção rigorosa do
"método histórico fundamental de comparação e confirmação de fontes
variadas" -- registros escritos e testemunhos orais --, uma vez que o
simples "estabelecimento de uma relação especular entre autor e herói"
conduz à perigosa e superficial identificação, já assinalada, entre obra
ficcional e dados biográficos; (f) a possibilidade de entrelaçamento
entre os fatos da vida e os pensamentos do biografado. Ao longo do
trabalho, os alunos terão oportunidade de converter o material obtido
nas pesquisas de campo em um texto capaz de recuperar a singularidade
do(a) entrevistado(a). No final, verifique de que maneira esse percurso
possibilitou uma reflexão e um conhecimento a respeito das próprias
identidades.

\begin{itemize}
\item
  Se conhecer a si mesmo envolve um processo infinito de autorreflexão,
  o sujeito só se define a partir de relações com os outros, no convívio
  interpessoal. A imagem que fazemos de nós mesmos é, em grande medida,
  influenciada por aquilo que desejamos que as pessoas ao redor
  percebam, pelo que tentamos ocultar, pelo modo com a sociedade nos vê
  e pelos valores que desejamos confirmar ou questionar. Em outras
  palavras, como percebemos durante a leitura da \emph{Narrativa de
  William Wells Brown}, a subjetividade só pode ser desenvolvida a
  partir da alteridade, do ato de nos colocarmos no lugar do outro e de
  estarmos dispostos a interagir com visões de mundo semelhantes ou
  diferentes das nossas.
\end{itemize}

Selecionada a pessoa sobre a qual os estudantes escreverão, eles podem
agendar um horário para entrevistá-la e colher o maior de números de
informações que puderem. Nesse processo, será preciso desenvolver a
habilidade de escuta, para permitir que o outro se expresse livremente e
para que você possa fazer a experiência de enxergar o mundo sob outras
perspectivas. Utilizando um aplicativo de celular para gravação de voz,
eles conversarão com o(a) entrevistado(a) a partir de um roteiro
previamente elaborado, mas permitindo-se também a deixar a conversa
seguir livremente e enveredar por tópicos não planejados previamente.

Também é possível fazer anotações no caderno, mas o mais importante
nesta etapa é ouvir o que o outro tem a dizer. Posteriormente, será
possível escutar novamente a conversa e transcrever os trechos
principais, anotar passagens interessantes e, a partir daí, esboçar a
primeira versão do perfil biográfico.

Verifique se as perguntas criadas pelos alunos permitem conhecer
diferentes aspectos sobre o(a) entrevistado(a). Algumas informações
básicas devem ser registradas: nome completo; idade e data de
nascimento; local de nascimento; profissão (e, a partir daí, motivos que
levaram a pessoa a exercer esse trabalho, principais desafios
enfrentados e rotina profissional); frases e pensamentos, lidos ou
ouvidos pela pessoa, que servem de orientação para a vida dela; opiniões
sobre fatos da atualidade; pessoas próximas (amigos, parentes,
vizinhos); episódios e momentos mais marcantes da vida; sonhos e planos
para o futuro.

\begin{itemize}
\item
  O estilo adotado no perfil pode ser documental, como forma de traçar o
  percurso de vida do(a) biografado(a). Você também pode explorar sua
  própria sensibilidade e, com olhar poético para os fenômenos do mundo,
  valorizar o registro por meio de figuras de linguagem ou fragmentos de
  poemas e canções.
\end{itemize}

Ao transcrever as falas, os grupos darão forma escrita às entrevistas
realizadas oralmente. É possível indicar as perguntas com o nome do
Entrevistador e as respostas com o nome do Entrevistado, para separar as
diferentes vozes presentes na interação. Como a transcrição consiste em
um registro escrito, oriente, na redação da primeira versão do perfil
biográfico, a substituição de algumas marcas de oralidade e expressões
típicas da linguagem informal por construções adequadas à variedade
linguística padrão.

Ao tratar as informações, lembre-se de indicar que sejam organizadas
cronologicamente, mesmo que, na entrevista, as lembranças tenham sido
apresentadas de maneira digressiva e pouco linear. Incentive o destaque
de fatos mais marcantes, curiosos, emocionantes ou inusitados.

\begin{itemize}
\item
  O perfil biográfico será redigido em prosa e deverá apresentar um
  relato capaz de comover, impressionar ou sensibilizar o público. O
  texto será divulgado no formato de um \emph{podcast}. Ao ser reunido
  aos demais perfis produzidos pela turma, você e seus colegas
  produzirão uma série de \emph{podcasts} sobre as trajetórias de vida
  das pessoas entrevistadas.
\end{itemize}

A primeira versão deverá ser digitada no computador, com o auxílio de um
programa para edição de textos, e compartilhada com a turma. Em seguida,
os estudantes retomam os textos produzidos e verificam se eles atendem
aos objetivos centrais, verificando se a entrevista foi adequadamente
realizada e permitiu respostas detalhadas para a posterior construção do
perfil biográfico. Posteriormente, releia a sua produção e faça os
ajustes e as reformulações que julgar necessárias.

A série de \emph{podcasts} divulgará os perfis biográficos, no formato
de arquivos de áudio que permitirão conhecer a trajetória de diferentes
integrantes da comunidade. As gravações serão transmitidas pela
internet, no formato de uma sequência de programas digitais de rádio. O
material poderá ser facilmente acessado pelo público, a partir de
celulares, \emph{tablets} ou computadores, e poderão ser ouvidos em
qualquer lugar e momento do dia. Os áudios deverão ser gravados e
publicados em uma plataforma de streaming. Incentive os alunos a criar
um nome para a série de programas produzida pela turma. Na produção dos
\emph{podcasts}, recomende atenção às marcas de oralidade.

Antes das gravações, feitas com o mesmo aplicativo de gravação de voz
utilizado para a entrevista com o(a) biografado(a), os alunos devem
ensaiar a leitura da versão definitiva do perfil e prestar atenção às
entonações necessárias. Oriente que escolham um ambiente silencioso para
gravar, com o menor número possível de ruídos e interferências externas.
É possível enriquecer as gravações, utilizando softwares de edição para
cortar, misturar sons e inserir trilhas sonoras ou fundos musicais.

Ao final, reserve um momento para apresentar a gravação em sala de aula
e avalie a qualidade do áudio, corrigindo pausas ou informações
desnecessárias, além de ruídos ou chiados. As fotografias dos indivíduos
biografados -- tanto os retratos feitos durante as conversas quanto as
fotos disponibilizadas -- formarão um mural na sala de aula ou no
corredor do colégio, e permitirão divulgar a série de \emph{podcasts}.
Durante a divulgação, os estudantes poderão organizar uma campanha
criativa pela escola e/ ou pelo bairro, assim como nas redes sociais,
com o objetivo de anunciar o lançamento da série. É interessante criar,
por exemplo, cartazes com QR-Codes de acesso aos \emph{podcasts} e, após
a estreia, anunciem a postagem de novos episódios. Procurem não lançar
todos os áudios de uma vez e criem uma periodicidade de publicação. Os
programas podem ser compartilhados e armazenados em páginas de redes
sociais, plataformas de streaming para distribuição digital de dados, ou
mesmo no \emph{site} ou em um blog da escola.

\subsection{Pós-leitura}

%\BNCC{EM13LGG102}
%\BNCC{EM13LGG303}
%\BNCC{EM13LGG402}
%\BNCC{EM13LGG703}
%\BNCC{EM13LP13}
%\BNCC{EM13LP14}
%\BNCC{EM13LP28}
%\BNCC{EM13LP29}
%\BNCC{EM13LP52}

% \paragraph{Tema}
% \paragraph{Conteúdo}
% \paragraph{Objetivo}
% \paragraph{Justificativa}
% \paragraph{Metodologia}
%   \begin{enumerate}
%   \end{enumerate}
% \paragraph{Tempo estimado}

Por fim, sugere-se um estudo comparativo da obra lida com
outros relatos de ex-escravos. Recomenda-se, para isso, a leitura de
obras como \emph{Nascidos na Escravidão}, organizado por Tâmis Parron, e
a exibição do filme \emph{Doze Anos de Escravidão}, de Steve McQueen,
baseado no relato de Solomon Northup, além de outras obras indicadas na
última sessão deste Manual. Desse modo, os estudantes perceberão que, de
certo modo, as narrativas guardam elementos muito individuais, enquanto
experiências das relações sociais que diferentes indivíduos vivenciaram.
Ainda assim, é possível encontrar elementos em comum em narrativas
diversas. Nesta atividade, além de reunir impressões sobre a obra lida,
será possível ampliar o repertório cultural e estabelecer comparações
entre as experiências relatadas. Recomenda-se a produção de uma
\textbf{resenha crítica}, produzida individualmente, em que os
estudantes poderão expor, por meio da argumentação, uma apreciação
pessoal da \emph{Narrativa de William Wells Brown}, enriquecida por
elementos oriundos do contraste com outras obras.

\Image{Na legenda da ilustração: ``O autor e a sua mãe presos e levados de volta à escravidão'' (Internet Archive; Domínio Público)}{PNLD0038-06.png}

\begin{itemize}
\item
  Orientando os alunos para que não deem \emph{spoilers} sobre
  acontecimentos centrais da obra livre, incentive a produção de uma
  resenha crítica para divulgar o livro para o público em geral. O
  objetivo do texto é despertar o interesse do leitor e instigar o
  desejo de ler a obra integralmente. Para isso, os estudantes farão uma
  breve sinopse inicial e, em seguida, destacarão os pontos positivos da
  obra lida. Podem ser ressaltadas informações sobre o contexto de
  produção e circulação, as inovações da voz narrativa, curiosidades
  sobre autoria, elementos relacionados ao contexto de época e demais
  motivações para a leitura. É sempre interessante estabelecer conexões
  com fatos e situações da atualidade, com comentários sobre a qualidade
  técnica e temática da obra. Paralelos com outras obras pertencentes ao
  repertório cultural dos alunos também são recomendáveis. As primeiras
  versões podem ser apresentadas nas aulas de produção de texto e, após
  os comentários do(a) professor(a) de Língua Portuguesa, os estudantes
  podem digitar a versão final da resenha crítica e publicá-la no
  \emph{site} da escola ou no blog da turma, destinado ao
  compartilhamento das experiências de leitura.
\end{itemize}


\section{Atividades 2}

%\BNCC{EM13CNT201}
%\BNCC{EM13CNT303}
%\BNCC{EM13CHS101}
%\BNCC{EM13CHS102}
%\BNCC{EM13CHS106}
%\BNCC{EM13CHS401}


A obra \emph{Narrativa de William Wells Brown} possibilita trabalhos
interdisciplinares e integradores de diferentes campos do saber e áreas
de conhecimento. A seguir, propomos algumas atividades que podem ser
desenvolvidas conjuntamente com professores de outras áreas. Além das
habilidades de Linguagens e suas Tecnologias e de Língua Portuguesa,
indicadas nas etapas da seção anterior e válidas também para esta,
listamos a seguir as habilidades de outras áreas, presentes na abordagem
interdisciplinar:

\subsection{Pré-leitura}

% \paragraph{Tema}
% \paragraph{Conteúdo}
% \paragraph{Objetivo}
% \paragraph{Justificativa}
% \paragraph{Metodologia}
%   \begin{enumerate}
%   \end{enumerate}
% \paragraph{Tempo estimado}

Durante a leitura do livro e realização das atividades com a
turma, professores da área de Ciências Humanas podem destacar aspectos
envolvendo a instituição do \textbf{escravismo nos Estados Unidos}. Se
forem discutidas questões sociais, políticas e econômicas relativas
àquele país, é também importante observar que o tráfico negreiro gerou
impactos no solo africano, de onde vinham os cativos. Nesse sentido, os
alunos podem realizar uma pesquisa em livros e \emph{sites} de internet,
sobre os \textbf{impactos gerados pelo tráfico negreiro em solo
africano}. É interessante abordar aspectos sociais, políticos e
econômicos, tanto durante a ocorrência do tráfico quanto no seu imediato
ocaso. Com as informações obtidas na pesquisa, a turma, reunida em
pequenos grupos, pode produzir uma breve reportagem em vídeo, em formato
de divulgação, para ser compartilhada no \emph{site} da escola ou no
blog da turma.


\begin{itemize}
\item
  Proponha aos alunos que digitem a primeira versão dos textos e
  compartilhem com a turma. Será uma oportunidade interessante para
  também ter contato com as versões iniciais das produções dos colegas.
  A partir da discussão sobre os temas, oriente eventuais reformulações
  nos arquivos e combinem a construção de um \emph{site} ou de uma
  página virtual para articular os textos no formato de um caderno ou
  banco de reportagens. Estabeleçam uma pauta de assuntos, com a
  sequência adequada de apresentação das produções. As versões
  definitivas deverão ser publicadas nesse endereço, em arquivos salvos
  em formato PDF e com uma configuração padronizada: os textos podem ser
  digitados em fonte Arial 12, com espaçamento de 1,5 cm entre as linhas
  e margens justificadas. Com isso, será garantida a identidade visual
  do conjunto.
\item
  O \emph{link} de acesso ao material e links individuais de consulta à
  reportagem de um grupo específico devem ser divulgados para a
  comunidade escolar e do bairro. A partir dessas versões, será
  elaborada uma reportagem audiovisual, em formato de vídeo, que será
  compartilhada no mesmo endereço eletrônico e/ ou em um canal da
  internet. Para elaborar essa segunda produção, reúnam as imagens
  coletadas, utilizem a versão escrita como base para a narração e
  editem os materiais audiovisuais obtidos na fase inicial da pesquisa.
\item
  Juntamente com os professores responsáveis, a turma programará um
  momento para exibir as versões finais das reportagens em formato
  audiovisual, que serão disponibilizadas na internet. Ao final, as
  produções poderão ser publicadas e compartilhadas com os demais
  membros da comunidade, no \emph{site} da escola ou no blog da turma.
\end{itemize}

\subsection{Leitura}

% \paragraph{Tema}
% \paragraph{Conteúdo}
% \paragraph{Objetivo}
% \paragraph{Justificativa}
% \paragraph{Metodologia}
%   \begin{enumerate}
%   \end{enumerate}
% \paragraph{Tempo estimado}

A escrita dos relatos autobiográficos, além de informações
sobre a história de vida de uma pessoa, também gera uma fonte histórica.
Autobiografias, diários e cartas, por vezes, foram gêneros desprezados
pela historiografia, devido à subjetividade explícita na superfície dos
textos. Entretanto, eles são materiais capazes de oferecer outras
perspectivas para a História oficial. No caso da obra lida, tem-se a
dimensão do que foi a escravidão dos Estados Unidos para a população
negra. Com auxílio do professor da área de Ciências Humanas, solicite
aos alunos que procurem -- em livros e \emph{sites} de internet --
biografias, relatos, depoimentos e cartas de pessoas que viveram em
outros tempos. Paralelamente à leitura da obra, peça para que cada um
dos grupos compartilhe, com a turma, os materiais encontrados. Ao fim,
cada grupo deve redigir uma \textbf{crônica} a partir dos materiais
levantados. Uma fonte relevante de consulta é o Museu da Pessoa
(\href{http://www.museudapessoa.org}{museudapessoa.org}), um museu
virtual e colaborativo de histórias de vida de pessoas comuns, com
acervo rico e variado. Outro banco de memórias que pode ser consultado
\emph{online} é o da página SP Invisível
(\href{http://www.spinvisivel.org}{spinvisivel.org}), mantida por
uma ONG que procura humanizar e resgatar histórias de vida dos moradores
de rua da cidade de São Paulo.

\begin{itemize}
\item
  Na crônica produzida pelos alunos, procure articular o olhar poético
  sobre a realidade e o registro de fatos ligados às memórias
  individuais e coletivas. Diferentemente do perfil biográfico proposto
  na Atividade 1, aqui as pessoas selecionadas não fazem parte do
  convívio direto dos alunos e podem ser trabalhadas de maneira ainda
  mais próxima ao registro literário, como personagens de ficção, embora
  sejam textos produzidos a partir de vivências de pessoas reais.
  Incentive a valorização de fatos emocionantes, engraçados ou
  comoventes. Com ajuda dos professores de Ciências Humanas, incentive a
  pesquisa complementar de fotografias de época, como forma de favorecer
  a visualização do período em questão. Por fim, as versões finais dos
  textos, após a revisão e correção dos professores envolvidos, poderá
  ser publicada no \emph{site} da escola ou no blog da turma.
\end{itemize}

\subsection{Pós-leitura}

% \paragraph{Tema}
% \paragraph{Conteúdo}
% \paragraph{Objetivo}
% \paragraph{Justificativa}
% \paragraph{Metodologia}
%   \begin{enumerate}
%   \end{enumerate}
% \paragraph{Tempo estimado}

Ao longo da obra, foi possível observar aspectos envolvendo
a instituição do escravismo. Entretanto, mesmo após o fim da escravidão
nos Estados Unidos, algo comum a quase todos os países americanos com
passado escravista, nota-se a falta de um planejamento para a integração
da população antigamente escravizada à sociedade. Faltaram políticas
reparadoras que visassem diminuir as consequências de anos de abuso.
Diante disso, a atividade final propõe, após o término da leitura, uma
reflexão sobre problemas contemporâneos (como racismo, segregação
espacial, desigualdade), partilhados por muitos países americanos com
passado escravista. O objetivo é fazer com que os alunos, com auxílio de
professores da área de ciências humanas, tracem paralelos entre as
experiências de países como Brasil, Estados Unidos, Haiti, Colômbia e
Cuba, por exemplo, observando como essas nações lidaram com as questões
sociais do pós-escravismo, e como esse passado deixa marcas nas
sociedades até os dias de hoje.

\begin{itemize}
\item
  Com a turma dividida em pequenos grupos, oriente a escolha de um país
  americano por equipe. Cada uma deverá pesquisar, em livros e
  \emph{sites} de internet, a história da escravidão e as ressonâncias
  do passado escravista no país escolhido. Estimule a busca de vídeos,
  imagens, mapas e outros documentos históricos que permitam traçar o
  histórico dessas nações. Em seguida, promova um \textbf{seminário} em
  que cada um dos grupos apresente as informações encontradas, com
  auxílio da projeção de \emph{slides} informativos preparados com
  antecedência. Cada grupo deverá estabelecer conexões com trechos da
  obra trabalhada e constatar, por meio da análise comparativa,
  semelhanças entre as situações apresentadas no livro lido e nos dados
  obtidos por meio da pesquisa.
\end{itemize}


\section{Aprofundamento}

Ao chegar ao Ensino Médio, é necessário que os estudantes se aprofundem
na compreensão das múltiplas linguagens e, sobretudo, da linguagem
literária. Em relação à literatura, a BNCC traz as seguintes
considerações:

\begin{quote}
{[}...{]} a leitura do texto literário, que ocupa o centro do trabalho
no Ensino Fundamental, deve permanecer nuclear também no Ensino Médio.
Por força de certa simplificação didática, as biografias de autores, as
características de épocas, os resumos e outros gêneros artísticos
substitutivos, como o cinema e as HQs, têm relegado o texto literário a
um plano secundário do ensino. Assim, é importante não só (re)colocá-lo
como ponto de partida para o trabalho com a literatura, como
intensificar seu convívio com os estudantes. Como linguagem
artisticamente organizada, a literatura enriquece nossa percepção e
nossa visão de mundo. Mediante arranjos especiais das palavras, ela cria
um universo que nos permite aumentar nossa capacidade de ver e sentir.
Nesse sentido, a literatura possibilita uma ampliação da nossa visão do
mundo, ajuda-nos não só a ver mais, mas a colocar em questão muito do
que estamos vendo/vivenciando. (Brasil, 2018, p. 491)
\end{quote}

Nesta seção, desenvolvemos um trabalho de aprofundamento que, em diálogo
com a formação continuada de professores, oferece subsídios para a
abordagem do texto literário. A leitura em sala de aula \emph{Narrativa
de William Wells Brown} pode ser enriquecida pelo aprofundamento no
universo literário em que a obra está inserida.

\subsection{Um testemunho da escravidão nos Estados Unidos}

Assim como outros livros de escravizados fugidos do Sul para o Norte dos
Estados Unidos, publicados no século XIX, a \emph{Narrativa de William
Wells Brown -- Escravo fugitivo} também se situa no contexto social e
econômico norte-americano do período.

A obra contém um \textbf{apelo ao fim da escravidão} nos Estados Unidos
e no resto do mundo. Também nos é mostrado todo o ambiente de violência
física e psicológica que o sistema escravista imprimia sobre as pessoas
por ele vitimadas.

É interessante estabelecer relações entre a obra e o contexto em que ela
foi produzida. Para isso, é recomendável estabelecer parcerias com
professores da área de Ciências Humanas. Entre as estratégias de leitura
indicadas na BNCC, estão aquelas que, ``por um lado, permitam a
compreensão dos modos de produção, circulação e recepção das obras
{[}...{]} e o desvelamento dos interesses e dos conflitos que permeiam
suas condições de produção {[}...{]}''.

\subsection{Uma vida marcada pelo tempo}

William Wells Brown foi um escritor norte-americano, nascido em
Kentucky no ano de 1814. Ele veio ao mundo já na condição de
escravizado, uma vez que sua mãe era escrava. Aos 19 anos, Brown
consegue fugir de seu senhor e se estabelece em Ohio no ano de 1834.

Nascido escravo, um dos pontos mais marcantes da trajetória de William
Wells Brown ocorreu quando ele tinha apenas 16 anos. Com essa idade, foi
vendido por seu senhor para um traficante de escravos no Rio Mississipi
e acabou trabalhando com brancos escravizadores, participando do tráfico
de pessoas para o trabalho forçado. Esse fato acabou sendo crucial para
o desenvolvimento das experiências relatadas por Brown enquanto
escritor, por permitir compreender o funcionamento do sistema escravista
nos Estados Unidos, embora tenha sido polêmico para sua reputação.

Nove anos após sua fuga, William Wells Brown começou a dar palestras
para o público abolicionista dos estados do Norte, contando sua história
com uma boa retórica e já iniciando os preparativos para a escrita deste
livro, publicado originalmente em 1847.

Logo em seguida, ele se fixou em Boston, local que, no século XIX, era o
grande centro do abolicionismo estadunidense. Lá ele se engajou à causa
abolicionista, estudou e se tornou um grande escritor. Além da causa da
abolição, Brown também defendeu outras pautas sociais, como o sufrágio
feminino, a redução do consumo de álcool e a reforma penitenciária,
entre outras questões. Em meados de 1847, publica sua grande obra,
\emph{Narrativa de William Wells Brown -- escravo fugitivo}.

Desse modo, o autor presenciou pessoalmente toda a crueldade praticada
contra os escravizados e a brutalidade do tráfico de pessoas.
Testemunhou também outra triste característica da escravidão
norte-americana, marcada, entre outras tragédias, pela indiferença com
que eram realizadas as separações de familiares. O próprio autor foi
separado de sua mãe, irmão e irmã.

Essas experiências tornam pungente o relato de Brown e revelam a
violência por trás do sistema escravista nos Estados Unidos do século
XIX.

\subsection{Por que ler esta obra hoje?}

É fundamental observar a obra como o \textbf{relato da vida} de um
ex-escravizado norte-americano que conseguiu fugir de um sistema
escravista cruel e estruturado. O relato autobiográfico de Brown permite
refletir sobre o universo da escravidão a partir do testemunho de uma
das vítimas do regime. No prefácio à primeira edição, J.C. Hathaway
assinalou que William Wells Brown vivenciou quase que todos os aspectos
do sistema escravagista norte-americano, revelando como todas as
instituições do país estavam voltadas para sustentar a submissão racial
e servil dos cativos.

Além disso, é uma obra que teve grande peso na luta abolicionista.
Deve-se lembrar que os Estados do Sul chegaram a ter uma população de
negros escravizados orbitando entre 51 e 54\% do total populacional.
Esse número é extremamente alto e demonstra a forte estruturação do
sistema escravista. Além do mais, nesse período, o valor de um negro
escravizado nos Estados Unidos era tão elevado que perdia apenas para a
propriedade de terras.

Vale ressaltar também que Brown sempre acreditou nas tradições
anglo-americanas e cristãs protestantes, vindas de séculos anteriores,
enquanto buscava o imediato fim da escravidão e da subordinação racial
nos Estados Unidos. Por isso, o autor foi visto como um reformista e não
exatamente como um revolucionário, e essa classificação impactou a
difusão de suas ideias.

\Image{Na legenda da ilustração, o poema de Thomas Campbell ``United States! your banner wears/ Two emblems--one of fame:/ Alas! the other that it bears/ Reminds us of your shame:/ Your banner's constellation types/ White freedom with its stars, /But what's the meaning of the stripes?/ They mean your negroes' scars.'' (Internet Archive; Domínio Público)}{PNLD0038-07.png}

\subsection{A escravidão nos Estados Unidos}

A metade do século XIX foi um período em que a escravidão
norte-americana estava consolidada, anteriormente à Guerra Civil
(1860-1865) que romperia com a estrutura escravagista do país. Nesse
período, outras obras com a mesma temática foram publicadas, como as de
Frederick Douglass -- que chegou a discutir publicamente com Brown -- e
a de Harriet Ann Jacobs -- autora de Incidentes da vida de uma escrava
--, em 1861. Portanto, vemos que William Wells Brown foi um homem
atuante em seu tempo, defendendo a liberdade dos negros e contribuindo
com o conjunto de publicações de autobiografias de ex-escravizados.
Brown faleceu no ano de 1884, em Chelsea, Massachusetts.

\subsection{A escravidão na literatura}

Além de autobiografia, o livro de Brown, assim como outros do mesmo
gênero, é uma excelente fonte de documentação histórica e revela a
relação opressiva e a subordinação racial e servil imposta pelos brancos
escravistas aos negros escravizados, no Sul dos Estados Unidos do século
XIX.

Outro elemento que merece destaque é o fato de que a obra, assim como
outras autobiografias de negros escravizados, visavam combater a
literatura sulista, branca e escravocrata, a qual descrevia a vida nas
fazendas do Sul como algo idílico e bom, tanto para os brancos senhores
de terras quanto para os negros escravizados. Até mesmo um termo foi
criado para caracterizar esse movimento contrário à literatura
escravista: ``a escola literária do fugitivo heroico norte-americano''.
Sobre essa vertente literária, cabe ressaltar que a maioria dos
escritores a ela relacionados saíam dos estados escravistas mais
próximos ao Norte do país, ou seja, esses escritores fugidos da
escravidão não estavam nas fazendas de algodão do Extremo Sul dos
Estados Unidos. Esse dado geográfico promove um recorte histórico e
social na realidade retratada pela maior parte desses textos.

\subsection{Sugestões de atividades complementares: relações dialógicas e
intertextuais}

%\BNCC{EM13LP03}
%\BNCC{EM13LP04}
%\BNCC{EM13LP49}
%\BNCC{EM13LP51}

No Ensino Médio, da mesma forma que no Ensino Fundamental, a \textsc{bncc}
organiza o trabalho com as práticas de linguagem em cinco \textbf{campos
de atuação social}. São eles: campo da vida pessoal, campo da vida
pública, campo jornalístico"-midiático, campo artístico"-literário e campo
das práticas de estudo e pesquisa.

De acordo com essa divisão, propomos na sequência um trabalho
interdiscursivo e intertextual com a obra \emph{Narrativa de William
Wells Brown.}

\subsubsection{Campo da vida pessoal}

\begin{quote}
O campo da vida pessoal pretende funcionar como espaço de articulações
e sínteses das aprendizagens de outros campos postas a serviço dos
projetos de vida dos estudantes. As práticas de linguagem privilegiadas
nesse campo relacionam"-se com a ampliação do saber sobre si, tendo em
vista as condições que cercam a vida contemporânea e as condições
juvenis no Brasil e no mundo.

Está em questão também possibilitar vivências significativas de práticas
colaborativas em situações de interação presenciais ou em ambientes
digitais e aprender, na articulação com outras áreas, campos e com os
projetos e escolhas pessoais dos jovens, procedimentos de levantamento,
tratamento e divulgação de dados e informações e o uso desses dados em
produções diversas e na proposição de ações e projetos de natureza
variada, para fomentar o protagonismo juvenil de forma
contextualizada. (\textsc{bncc}, p. 494)
\end{quote}

A leitura de relatos acerca das vivências de ex-escravizados nos
Estados Unidos do século XIX convida à reflexão sobre temas ainda hoje
atuais, mas também possibilita o trabalho no campo das competências
socioemocionais. O registro de trajetórias marcadas pelo sofrimento e
pela superação das adversidades estimula o desenvolvimento da empatia
e da solidariedade em relação a dores alheias. Por isso, na esfera da
subjetividade e dos projetos de vida, os testemunhos de vida podem
servir como ponto de partida para debates e rodas de conversa sobre
temas relevantes para o universo dos adolescentes. Em parceria com
professores de Ciências Humanas, proponha a elaboração individual de
\textbf{crônicas argumentativas} com base nos temas do respeito às
diferenças e do direito à liberdade. Verifique as visões da turma
sobre a noção de liberdade e observe que valores estão ligados, do
ponto de vista deles, à existência dessa condição. Em seguida,
relacione o estatuto do sujeito livre à necessidade de respeito e
tolerância das diferenças em todos os níveis da existência, mas,
sobretudo, no que tange a questões culturais e étnico-raciais. Para
fundamentar a argumentação, estimule a pesquisa virtual a passagens da
Constituição que asseguram as liberdades individuais e coletivas.
Oriente, depois, o registro dos argumentos e contra-argumentos no
caderno, para que seja possível articulá-los em uma unidade
argumentativa coesa e coerente. Reserve um momento para a produção das
primeiras versões das crônicas, que devem ser preferencialmente
digitadas. Incentive a inserção de exemplos extraídos da atualidade, a
partir de novas consultas a \emph{sites} confiáveis de internet.
Partindo do cotidiano, é possível que a argumentação seja construída
com base em um olhar poético e sensível para a realidade vivida por
muitas pessoas em todo o mundo. O combate à discriminação deve ser
incentivado e argumentos que fujam do senso comum podem ser
construídos a partir da proposição de formas de intervenção concreta
sobre os problemas apontados. As produções podem ser compartilhadas
com os colegas e professores envolvidos na atividade, para que
comentários críticos e construtivos sejam feitos, de forma respeitosa
e democrática. Ao final, as versões definitivas podem ser publicadas
no \emph{site} do colégio, em redes sociais ou em um blog da turma,
após um processo de revisão e edição das produções textuais.

\subsubsection{Campo de atuação na vida pública}

\begin{quote}
No cerne do campo de atuação na vida pública estão a ampliação da
participação em diferentes instâncias da vida pública, a defesa dos
direitos, o domínio básico de textos legais e a discussão e o debate de
ideias, propostas e projetos. {[}\ldots{}{]}

Ainda no domínio das ênfases, indica"-se um conjunto de habilidades que
se relacionam com a análise, discussão, elaboração e desenvolvimento de
propostas de ação e de projetos culturais e de intervenção social.
(\textsc{bncc}, p. 494)
\end{quote}

A luta antirracista é uma das principais pautas dos movimentos negros
em todo o mundo. O combate ao preconceito étnico-racial deve ser
estimulado no cotidiano da sala de aula, sobretudo porque diversos
discursos veiculados socialmente perpetuam o chamado \emph{racismo
estrutural}, que muitas vezes não é notado por grande parte dos
falantes. Por isso, a leitura de relatos autobiográficos, produzidos
no contexto da escravidão, pode fundamentar as discussões em sala de
aula. A partir das experiências de leitura de narrativas originárias
dos Estados Unidos, proponha a elaboração, individual ou em dupla, de
um \textbf{artigo de opinião} sobre o preconceito racial no Brasil.
Para isso, com auxílio de professores da área de Ciências Humanas, é
importante estabelecer paralelos comparativos sobre o regime
escravista nos dois países, bem como sobre as consequências
histórico-sociais da escravidão. Atitudes segregacionistas, racismo
internalizado e estrutural, discriminação nas esferas pública e
privada, violência e desigualdade são aspectos que podem ser
explicados historicamente. Além disso, professores de Ciências da
Natureza podem apresentar argumentos científicos, sobretudo ligados ao
campo de estudo da Genética, para combater o preconceito e as teorias
eugenistas que ganharam força nos séculos XIX e XX. A partir dos
debates interdisciplinares, os estudantes poderão articular os
argumentos sob a forma de artigos que discutam, de forma coesa e
coerente, as causas históricas, os impactos sociais e formas concretas
de atuação social e intervenção quanto ao problema do racismo na
sociedade brasileira. É importante incluir elementos trazidos das
leituras, mas também é recomendável estimular a consulta a livros e
\emph{sites} confiáveis sobre a luta antirracista. As primeiras
versões dos textos podem ser compartilhadas, em sala de aula, com os
colegas e professores envolvidos. Na sequência, as produções podem ser
revisadas, digitadas e editadas com auxílio do computador. Ao final,
os textos podem ser publicados no \emph{site} da escola, em redes
sociais ou no blog da turma destinado ao registro das experiências de
leitura.


\subsubsection{Campo jornalístico"-midiático}

\begin{quote}
Em relação ao campo jornalístico"-midiático, espera"-se que os jovens
que chegam ao Ensino Médio sejam capazes de: compreender os fatos e
circunstâncias principais relatados; perceber a impossibilidade de
neutralidade absoluta no relato de fatos; adotar procedimentos básicos
de checagem de veracidade de informação; identificar diferentes pontos
de vista diante de questões polêmicas de relevância social; avaliar
argumentos utilizados e posicionar"-se em relação a eles de forma ética;
identificar e denunciar discursos de ódio e que envolvam desrespeito aos
Direitos Humanos; e produzir textos jornalísticos variados, tendo em
vista seus contextos de produção e características dos gêneros. Eles
também devem ter condições de analisar estratégias
linguístico"-discursivas utilizadas pelos textos publicitários e de
refletir sobre necessidades e condições de consumo.

No Ensino Médio, os jovens precisam aprofundar a análise dos interesses
que movem o campo jornalístico midiático, da relação entre informação e
opinião, com destaque para o fenômeno da pós"-verdade, consolidar o
desenvolvimento de habilidades, apropriar"-se de mais procedimentos
envolvidos na curadoria de informações, ampliar o contato com projetos
editoriais independentes e tomar consciência de que uma mídia
independente e plural é condição indispensável para a democracia.

Como já destacado, as práticas que têm lugar nas redes sociais têm
tratamento ampliado. (\textsc{bncc}, p. 494-495)
\end{quote}

A derrubada de estátuas de figuras históricas associadas, sobretudo,
ao racismo, foi uma expressão de protesto multiplicada em muitos
lugares do mundo, no ano de 2020. O movimento foi disparado pelo
assassinato de George Floyd, cidadão negro asfixiado por um oficial
branco, durante uma abordagem policial na cidade de Minnesota, nos
Estados Unidos. A partir desse fato, proponha a produção individual de
um \textbf{ensaio} escrito, por meio do qual os estudantes analisem
criticamente um aspecto da polêmica intervenção sobre estátuas e
monumentos públicos. Para isso, estimule a reflexão sobre aspectos
ligados à questão, que podem servir de base para a formulação da tese
e da argumentação desenvolvida no texto.

Verifique se, na opinião dos alunos, a derrubada das estátuas compõe um
gesto:

\begin{enumerate}
\item agressivo, irresponsável e de violência gratuita contra o
patrimônio histórico; ou 
\item necessário para a conscientização acerca
das injustiças sofridas pelas populações negras ao longo da História;
\item de vandalismo infundado, disfarçado em postura politicamente correta
ou 
\item aceitável para expressar a revolta e tornar público o racismo
presente na sociedade; 
\item de destruição da memória de uma nação e de
iconoclastia de objetos de arte ou 
\item de reparação junto às vítimas
silenciadas pelo opróbrio da escravidão; 
\item de repúdio à suposta
intocabilidade de figuras históricas, cúmplices e responsáveis pela
escravidão ou 
\item de grito de dor, vindo como resposta à segregação
racial, há tanto tempo legitimada e hoje implícita e explícita nas
relações sociais, marcadas pelo racismo estrutural.
\end{enumerate}

Esses tópicos tratam do tema sob duas perspectivas contrárias entre si.
Ao escolher um item para orientar a sua produção, o estudante tomará
partido de um lado ou de outro, mas não deverá desconsiderar os demais
pontos de vista: é preciso também leva-los em conta ao formular os
argumentos e contra-argumentos.

\begin{quote}
Para produzir um ensaio, é preciso mobilizar o repertório sociocultural
e fundamentar os argumentos com base em livros, filmes, letras de
música, obras de arte e textos históricos ou filosóficos, que podem ser
mencionados ao longo do texto. É interessante estimular a consulta a
\emph{sites} de periódicos confiáveis, ligados à divulgação de
atualidades.

O ensaísta é um autor que analisa diferentes aspectos das ideias,
observando cada um dos elementos como se fossem as faces de um prisma.
Lembre-se, no entanto, de que é preciso defender uma tese central; ao
fazer um ensaio sobre o tema selecionado, é possível adotar diferentes
posturas para construir a argumentação: o aluno poderá acolher uma das
perspectivas, rejeitando a outra, em uma atitude com inclinação ao
formato \emph{ou... ou}; também é possível escolher um lugar enunciativo
com inclinação ao formato \emph{e... e}, apresentando, para tanto,
soluções alternativas.

Observe, em sala de aula, que as opiniões possíveis a respeito do tema
refletem as vozes às quais cada aluno responderá: elas são, de um lado,
aquelas vozes que comentaram, analisaram e discutiram esse tema na mídia
contemporânea; além disso, de outro lado, há também as vozes dos atores
sociais que, na vivência pretérita, presente e futura dos fatos ligados
à escravidão negra, constituem a História como arena de conflitos
operada discursivamente. Ao comentar, analisar e discutir o tema no
formato de um ensaio, o texto responderá por adesão a determinadas vozes
sociais e não a outras. Em parceria com professores de Ciências Humanas,
reflita profundamente com os alunos sobre o ponto de vista a ser
defendido, uma vez que ser responsivo no ato de criar um ensaio é ser
responsável socialmente.

Os alunos podem elaborar, no caderno, um projeto de texto para organizar
os principais argumentos sobre o tema. É interessante que elaborem mapas
de ideias, façam um \emph{brainstorming} a partir das múltiplas faces da
questão, listem tópicos que auxiliem a posterior elaboração dos
parágrafos e da orientação argumentativa coerente do texto. É importante
desenvolver uma reflexão sobre o aspecto selecionado, verificando se se
as intervenções sobre monumentos do patrimônio público constituem um
gesto agressivo, inconsequente e vão, ou se podem constituir uma
participação necessária, legítima e oportuna na História da humanidade.

O ensaio permite, ao longo do texto, marcas de 1ª pessoa que explicitem
a subjetividade do autor, por meio de expressões como ``Eu creio
que...'', ``no meu ponto de vista, ...'', ``eu penso que...'' etc. É
possível até mesmo estabelecer paralelos com experiências pessoais e
fazer breves relatos de situações pessoais que tenham ligação com o
tema. Ao mesmo tempo, esse gênero de texto acolhe modalizadores de
dúvida e incerteza, como ``talvez'', ``é provável que...'', ``pode ser
que...'', entre outros que captam o processo de construção do pensamento
autoral. Esse gênero textual não visa, portanto, como ocorre em outras
modalidades de argumentação, à construção de um efeito de objetividade,
uma vez que o ensaísta conduz o leitor por um passeio pelo universo das
ideias pessoais e, ao expor um ponto de vista, estimula a reflexão
independente e a formação de opiniões do público. O leitor, portanto,
deve desempenhar um papel ativo e acompanhar o raciocínio do autor,
investigando, completando a análise por meio da autorreflexão e
formulando conclusões consistentes.

A primeira versão escrita do ensaio pode ser compartilhada com os
colegas e com os professores envolvidos na atividade. É recomendável que
os alunos leiam as produções em voz alta e anotem os comentários, as
sugestões e as críticas construtivas. Os posicionamentos podem ser
rediscutidos, com respeito e liberdade de expressão, mas sempre
respeitando os direitos humanos. Após a apresentação, cada aluno fará as
alterações e reformulações necessárias. A segunda versão do texto será
digitada, utilizando o computador , de acordo com a norma padrão da
língua portuguesa. As versões finais podem ser publicadas nas redes
sociais, no \emph{site} da escola ou em um blog destinado à divulgação
dos trabalhos.
\end{quote}

\subsubsection{Campo artístico"-literário}

\begin{quote}
No campo artístico"-literário busca"-se a ampliação do contato e a
análise mais fundamentada de manifestações culturais e artísticas em
geral. Está em jogo a continuidade da formação do leitor literário e do
desenvolvimento da fruição. A análise contextualizada de produções
artísticas e dos textos literários, com destaque para os clássicos,
intensifica"-se no Ensino Médio. Gêneros e formas diversas de produções
vinculadas à apreciação de obras artísticas e produções culturais
(resenhas, vlogs e podcasts literários, culturais etc.) ou a formas de
apropriação do texto literário, de produções cinematográficas e teatrais
e de outras manifestações artísticas (remidiações, paródias,
estilizações, videominutos, fanfics etc.) continuam a ser considerados
associados a habilidades técnicas e estéticas mais refinadas.

A escrita literária, por sua vez, ainda que não seja o foco central do
componente de Língua Portuguesa, também se mostra rica em possibilidades
expressivas. (\textsc{bncc}, p. 495-496).
\end{quote}

O texto literário oferece possibilidades de reflexão e conhecimento de
situações da vida que podem se aproximar das experiências dos
estudantes ou colocá-los em contato com realidades vividas por outras
pessoas. Com base na leitura de relatos autobiográficos de
ex-escravizados norte-americanos, os alunos podem conhecer
experiências de outros lugares e épocas, que permanecem atuais e
dialogam diretamente com o Brasil da contemporaneidade. Apesar de a
matéria básica das produções ser extraída da vida real, a elaboração
por meio da palavra confere uma dimensão literária às narrativas. Além
de documentos de época, revelam-se olhares de sujeitos sensíveis,
marcados por situações históricas em contextos de crise. Para
aprofundar o trabalho com conflitos ligados à experiência juvenil,
proponha um mergulho poético nas noções de liberdade e tolerância:
desta vez, os estudantes produzirão \textbf{poemas líricos} para
expressar ideias e sentimentos ligados a esses temas. Como ponto de
partida, é possível retomar produções e discussões feitas ao longo da
leitura, bem como letras de música, pertencentes a gêneros variados,
que tratem igualmente desses valores. Utilizando uma \emph{playlist}
construída coletivamente pelos alunos, privilegiando o repertório do
\emph{hip hop}, reserve um momento para uma oficina de criação
poética. Na sequência, proponha a leitura compartilhada das produções,
comentando-as e incentivando eventuais reformulações. Por fim, os
textos podem ser digitados, de maneira a compor uma antologia poética
acerca da liberdade e da tolerância, que pode ser publicada no
\emph{site} da escola ou no blog da turma, destinado às atividades
feitas com base nas experiências de leitura.

\subsubsection{Campo das práticas de estudo e pesquisa}

\begin{quote}
O campo das práticas de estudo e pesquisa mantém destaque para os
gêneros e habilidades envolvidos na leitura/escuta e produção de textos
de diferentes áreas do conhecimento e para as habilidades e
procedimentos envolvidos no estudo. Ganham realce também as habilidades
relacionadas à análise, síntese, reflexão, problematização e pesquisa:
estabelecimento de recorte da questão ou problema; seleção de
informações; estabelecimento das condições de coleta de dados para a
realização de levantamentos; realização de pesquisas de diferentes
tipos; tratamento dos dados e informações; e formas de uso e
socialização dos resultados e análises.

Além de fazer uso competente da língua e das outras semioses, os
estudantes devem ter uma atitude investigativa e criativa em relação a
elas e compreender princípios e procedimentos metodológicos que orientam
a produção do conhecimento sobre a língua e as linguagens e a formulação
de regras. (\textsc{bncc}, p. 495-496)
\end{quote}

As consequências da escravidão, em diferentes países do mundo,
geraram, ao longo do tempo, experiências semelhantes de preconceito e
discriminação. A leitura de relatos de ex-escravizados estadunidenses
enriquecem as discussões sobre o tema e permitem traçar paralelos com
realidades diversas, incluindo a da História nacional. No Brasil
contemporâneo, as desigualdades sociais se perpetuam também em
decorrência do passado escravista. Em parceria com professores de
Ciências Humanas, retome as discussões sobre as raízes históricas da
escravidão, com ênfase no Brasil, e proponha uma pesquisa sobre o
tráfico negreiro, desta vez concentrada no contexto brasileiro. Por
meio da pesquisa em livros e \emph{sites} de internet, a turma,
dividida em pequenos grupos, buscará informações sobre números de
pessoas trazidas do continente africano para o Brasil, as
características do sistema social, político e econômico que
sustentavam a escravidão, mapas de rotas do tráfico de escravizados,
relatos acerca do cotidiano nos navios negreiros, as leis que
procuraram modificar esse cenário e poemas produzidos sobre essas
experiências, sobretudo pela vertente socialmente engajada do
Romantismo (conhecida como \emph{condoreira} e praticada, no Brasil,
sobretudo por Castro Alves). A partir dos dados coletados, oriente a
produção de uma \textbf{reportagem} escrita sobre o tráfico negreiro
no Brasil, em linguagem de divulgação, com o objetivo de promover
reflexões sobre questões atuais do país, tanto nas relações pessoais
quanto no mundo do trabalho. As versões finais dessas reportagens,
digitadas e editadas pelos grupos com auxílio do computador, podem ser
compartilhadas no \emph{site} da escola ou no blog da turma, destinado
à publicação de produções ligadas às experiências de leitura.

\section{Referências complementares}

\subsection{Livros}

\begin{itemize}
\item
ADICHIE, Chimamanda Ngozi. \textbf{Americanah}. São Paulo: Companhia
das Letras, 2014.


Em busca de alternativas às universidades nigerianas, a jovem Ifemelu
emigra para os Estados Unidos. Enquanto se destaca no meio acadêmico,
ela depara com a questão racial e com as dificuldades da vida de mulher
negra e estrangeira.

\item
ANGELOU, Maya. \textbf{Eu sei por que o pássaro canta na gaiola}.
Bauru: Astral Cultural, 2018.

Neste romance emocionante, a autora conta a história de Marguerite Ann
Johnson, garota negra criada no sul dos EUA pela avó, e dá voz a jovens
que, como ela, enfrentam muitos preconceitos.

\item
EVARISTO, Conceição. \textbf{Olhos d'água}. Rio de Janeiro: Pallas,
2014.

Com uma escrita lírica e sensível, a escritora mineira reúne breves
contos sobre o cotidiano, concentrando o foco de interesse sobre a vida
da população afro-brasileira.

\item
LEE, Harper. \textbf{O sol é para todos}. São Paulo: José Olympio,
2006.

No sul dos EUA da década de 1930, uma garotinha esperta e observadora
relata a saga do pai, um advogado que arrisca tudo para defender um
homem negro, injustamente acusado de cometer um crime.

\item
NORTHUP, Solomon. \textbf{Doze anos de escravidão}. São Paulo:
Penguin, 2014.
\end{itemize}

O livro é o relato real e assombroso de um negro livre que, atraído por
uma proposta de trabalho, abandona a segurança do Norte dos EUA e acaba
sendo sequestrado e vendido como escravo no Sul.

\item
STOCKETT, Kathryn. \textbf{A resposta}. Rio de Janeiro: Bertrand
Brasil, 2012.

A obra conta a história de Eugenia, jovem que deseja ser escritora. Ela
tem um plano brilhante, mas perigoso: escrever um livro em que
empregadas domésticas negras relatem seus relacionamentos com patroas
brancas do Mississipi, nos anos 1960.

\item
STOWE, Harriet Beecher. \textbf{A cabana do pai Tomás}. São Paulo:
Carambaia, 2018.

Um dos romances mais importantes da época da Guerra Civil Americana, a
obra conta a história do escravo Tom e influenciou intensamente as lutas
contra a escravidão.

\item
WALKER, Alice. \textbf{A cor púrpura}. São Paulo: José Olympio, 2009.

Vencedora do Prêmio Pulitzer, a autora narra com sensibilidade a vida de
Celie, uma mulher negra no sul dos Estados Unidos, do começo do século
XX, que sofreu abusos do padastro e depois do marido.
\end{itemize}

\subsection{Filmes}

\begin{itemize}
\item
\textbf{A cor púrpura}. Direção: Steven Spielberg (EUA, 1985).

Após sofrer violências em casa, a jovem Celie enfrenta trajetórias de
dor e superação. Triste e solitária, ela escreve cartas para a irmã, até
que a chegada da amante do marido transforma seu destino.

\item
\textbf{Besouro}. Direção: João Daniel Tikhomiroff (Brasil, 2009).

Na Bahia dos anos 20, o pequeno Manoel é apresentado à capoeira pelo
Mestre Alípio. Ao crescer, Besouro, como passa a ser chamado, recebe a
missão de defender seus semelhantes da opressão e do racismo.

\item
\textbf{Django Livre}. Direção: Quentin Tarantino (EUA, 2013).

O filme é construído em torno da improvável parceria entre Django, um
escravo liberto, e o Dr. Schultz, um caçador alemão de recompensas.
Juntos, irão atrás dos criminosos mais perigosos do sul dos EUA e
tentarão resgatar a esposa de Django.

\item
\textbf{Doze anos de escravidão}. Direção: Steve McQueen (EUA, 2013).

Vencedor de três estatuetas no Oscar de 2014, o filme conta a história
real de Solomon Northup, negro livre que foi escravizado por 12 anos no
sul dos EUA, ao cair na armadilha de uma oferta de emprego.

\item
\textbf{Histórias cruzadas}. Direção: Tate Taylor (EUA, Índia,
Emirados Árabes Unidos, 2012).

Adaptação do livro \emph{A Resposta}, este longa conta a história de
Skeeter, jovem que quer ser escritora. Ela começa a entrevistar mulheres
negras que deixaram suas vidas para trabalhar na criação dos filhos da
elite branca. Sua iniciativa irá desagradar muita gente.

\item
\textbf{O sol é para todos}. Direção: Robert Mulligan (EUA, 1963).

Baseado no romance homônimo, o filme conta a história de Tom Robinson,
jovem negro injustamente acusado de violentar uma mulher branca, e do
advogado Atticus Finch, que enfrentou a rejeição da cidade para defender
o réu.
\end{itemize}

\subsection{Lugar para visitar}

\begin{itemize}
\item\textbf{Museu Afro Brasil}
(\href{http://www.museuafrobrasil.org.br/}{museuafrobrasil.org.br}).
\end{itemize}

O museu, localizado na Vila Mariana, zona sul de São Paulo, reúne um
acervo contendo mais de 5 mil obras relacionadas à cultura africana e
afro-brasileira.

\end{itemize}

\section{Bibliografia comentada}

\begin{itemize}
\item
ADICHIE, Chimamanda Ngozi. \textbf{O perigo de uma história única}.
São Paulo: Companhia das Letras, 2019.

A escritora nigeriana defende que nosso conhecimento é construído pelas
histórias que escutamos e que, quanto mais diversas e numerosas forem
essas narrativas, mais completa será nossa compreensão sobre o mundo.

\item
ALEXANDER, Michelle. \textbf{A nova segregação}. São Paulo: Boitempo,
2018.

Esta obra desafiou a opinião de que o governo Obama assinalava o advento
de uma nova era pós-racial. A autora analisa o sistema prisional dos EUA
e expõe como o racismo estrutural opera nas sociedades ocidentais.

\item
  HORNE, Gerald. \textbf{O sul mais distante}. São Paulo: Companhia das
  Letras, 2010.

O autor vê a escravidão em termos hemisféricos e defende que o sul
escravista dos EUA via, em uma aliança com o Brasil (o ``sul mais
distante''), uma forma de proteção contra um futuro embate com o norte
estadunidense, na Guerra de Secessão.

\item
  MARQUESE, Rafael; SALLES; Ricardo (org.). \textbf{Escravidão e
  capitalismo histórico no século XIX: Cuba, Brasil, Estados Unidos}.
  Rio de Janeiro: Civilização Brasileira, 2016.

O livro reúne ensaios de historiadores brasileiros e estrangeiros sobre
a escravização de negros nas Américas, ao longo do século XIX.

\item
  McCULLERS, Carson. \textbf{A convidada do casamento}. São Paulo: Novo
  Século, 2008.

Frankie é uma menina cujos sonhos se chocam com sua rotina pacata,
compartilhada com seu primo e com Berenice, a cozinheira negra da casa.
Em um verão solitário nos EUA, suas incertezas aumentam quando recebe a
notícia do casamento do irmão.

\item
  MORGAN, Edmund S. ``Escravidão e liberdade: o paradoxo americano''.
  \emph{In} \textbf{Estudos Avançados} 14 (38), p. 21-50. São Paulo:
  Universidade de São Paulo, 2000. São Paulo, 2000. (Disponível em:
  \href{http://www.revistas.usp.br/eav/article/view/9507}{revistas.usp.br}.
  Acesso em: 8 de fev. de 2021.)

O autor busca compreender como o povo estadunidense pôde, desde o
princípio, desenvolver uma dedicação às ideias de liberdade e dignidade
humanas, e simultaneamente apoiar um sistema de trabalho que negava
diariamente esses valores.

\item
  PORTER, Regina. \textbf{Os viajantes}. São Paulo: Companhia das
  Letras, 2020.

Por meio das múltiplas perspectivas de personagens, a obra apresenta uma
trama que avança e volta no tempo. O livro traça um panorama da vida nos
Estados Unidos entre a década de 1950 e a eleição de Barack Obama para
presidente.

\end{itemize}



\end{document}

