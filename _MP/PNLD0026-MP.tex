\documentclass[11pt]{extarticle}
\usepackage{manualdoprofessor}
\usepackage{fichatecnica}
\usepackage{lipsum,media9,graficos}
\usepackage[justification=raggedright]{caption}
\usepackage[one]{bncc}
\usepackage[lamparina]{../edlab}

\begin{document}


\newcommand{\AutorLivro}{Machado de Assis}
\newcommand{\TituloLivro}{Pai contra mãe e outros contos}
\newcommand{\Tema}{Ficção, mistério e fantasia}
\newcommand{\Genero}{Conto, crônica e novela}
\newcommand{\imagemCapa}{./images/PNLD0026-01.png}
\newcommand{\issnppub}{---}
\newcommand{\issnepub}{---}
% \newcommand{\fichacatalografica}{PNLD0026-00.png}
\newcommand{\colaborador}{{Carlos Rogério Duarte Barreiros}}


\title{\TituloLivro}
\author{\AutorLivro}
\def\authornotes{\colaborador}

\date{}
\maketitle


\begin{abstract}\addcontentsline{toc}{section}{Carta ao professor}
É com muita alegria que lhe apresentamos a obra \emph{Pai contra Mãe e
outros contos de Machado de Assis}, uma coletânea de mais de trinta
contos de nosso maior escritor. Cada um desses textos tem um interesse
específico e, reunidos, eles compõem um conjunto bastante fértil para
análise em sala de aula. Mas, antes de falar de cada um deles
especificamente, é necessário avaliar a importância de ler Machado de
Assis no Ensino Médio e justificar, de forma geral, nossa escolha pelos
textos desse autor consagrado.

Primeiramente, acreditamos que nosso maior clássico pode ser um ótimo
ponto de partida para os estudantes conhecerem outros escritores.
Acreditamos também que tudo depende da escolha dos textos: para ser
porta de entrada para outras obras, o Machado que vamos ler tem de ser
do interesse de nossos alunos -- por isso é que selecionamos os contos
desta antologia: em todos eles podem ser identificados temas atuais, que
merecem discussão. Esse é um dos maiores prazeres da leitura, que
queremos despertar em nossos estudantes: um texto puxa o outro, um tema
contido aqui se repete ali e alimenta a curiosidade, o interesse, a
pesquisa. Antes de existirem os mecanismos de busca da internet, os
melhores livros já continham em si, potencialmente, a dinâmica do link,
que leva a outro, e a mais outro, quase infinitamente.

Mas não é só isso. Machado de Assis não se tornou nosso maior clássico
por acaso. Entre as muitas habilidades que tinha como escritor, uma se
destaca para nós, professores: a de conseguir transformar em forma
literária concreta, que podemos saborear na leitura, a dinâmica
específica de nosso país. Dizendo de forma simples, Machado sabia
transformar a experiência de ser brasileiro em texto literário, não
apenas por meio de descrições de lugares, pessoas e práticas sociais,
mas também por criar uma \emph{forma literária} que nos caracteriza e
com a qual nos identificamos. Somente grandes escritores têm essa
habilidade -- e é nossa função mostrá-la aos estudantes, para que eles
possam avaliar melhor os escritores contemporâneos à luz dessa
experiência acumulada. Não precisamos nos restringir à leitura de
Machado de Assis nas nossas salas de aula, mas ele é sempre um bom farol
para iluminar as avaliações que fazemos da produção literária atual,
porque poucos escritores entenderam nossa sociedade de forma tão
aprofundada e precisa.

Um exemplo simples dessa habilidade de Machado de Assis está na
linguagem de seu texto. Se compararmos as obras de Machado com as de
autores da mesma época, como Raul Pompeia, Eça de Queirós ou Júlia Lopes
de Almeida -- todos os três exímios prosadores --, perceberemos como
Machado se destaca, não porque seja necessariamente \emph{melhor} do que
eles (cada um deles é grande escritor à sua maneira), mas por soar, de
certa forma, mais próximo do leitor do presente. A capacidade de manter
uma prosa que tem um pé nos registros mais formais e outro na
informalidade quase coloquial permite que o jovem leitor do século XXI
se aproxime com mais facilidade de Machado de Assis. Nesse sentido,
talvez só Lima Barreto tenha conseguido uma linguagem tão acessível.
Esses dois escritores inscrevem no vocabulário, na sintaxe, na dicção e
no ritmo de seus textos um registro fluente, sem abandonar a linguagem
típica da época, mais estilizada e menos acessível aos nossos
estudantes, mas ultrapassando-a e renovando-a. Machado de Assis se
torna, assim, um caso quase único de \emph{clássico} de leitura
relativamente fácil. E esse é também um ótimo motivo para continuar a
lê-lo na sala de aula.

Com Machado de Assis é sempre assim: cada geração descobre em seus
textos aspectos que não tinham a atenção dos leitores do passado,
renovando a obra de nosso grande escritor. E é por isso que convidamos
você, professora, professor, a mergulhar mais uma vez na obra dele. Nas
próximas páginas, você encontrará:

\begin{itemize}
\item
  a \textbf{Proposta de Atividades I}, voltada aos professores de Língua
  Portuguesa, formulada nos termos da Base Nacional Comum Curricular,
  dividida em atividades pré-leitura, atividades de leitura e atividades
  pós-leitura;
\item
  a \textbf{Proposta de Atividades II}, com os mesmos objetivos e
  estrutura, mas voltada especificamente a professores de outras
  disciplinas, para que eles possam usar também os textos de nossa
  antologia;
\item
  o \textbf{Aprofundamento}, também voltado aos professores de Língua
  Portuguesa, com orientações que lhes permitam compreender melhor a
  obra, seu gênero e sua linguagem, por meio do exercício da leitura
  crítica, criativa e propositiva, articulando-a com outras, literárias
  e não literárias;
\item
  as \textbf{Sugestões de referências complementares}, com diversas
  fontes de análise, para que você possa fazer os contos analisados
  dialogarem com outras obras, e a \textbf{bibliografia comentada}, com
  referências bibliográficas para que você possa preparar sua aula com
  profundidade.
\end{itemize}

\end{abstract}

\tableofcontents

\Image{Machado de Assis (de óculos), aos 67 anos, com Joaquim Nabuco, 1906 (Augusto Malta; Domínio Público)}{PNLD0026-03.png}

\section{Atividades 1}
%\BNCC{EM13LP26}

\subsection{Pré-leitura}

\paragraph{Tema} Vida e obra de Machado de Assis

\paragraph{Conteúdo} Análise da vida e da obra de Machado de Assis.

\paragraph{Objetivos}
\begin{enumerate}
\item
Introduzir conhecimentos gerais a respeito da
vida e da obra de Machado de Assis; 

\item
Contextualizar os textos da
coletânea no conjunto dessa obra e em seu momento histórico; 

\item
Debater os critérios devido aos quais Machado de Assis é considerado o maior
escritor da Literatura Brasileira; 

\item
Avaliar a atualidade da obra de
Machado de Assis e seu alcance entre os leitores do século XXI.
\end{enumerate}

\paragraph{Justificativa} Contextualizar a leitura é motivar os alunos a
efetuá-la, daí a proposta de apresentar o autor e o conjunto da sua obra
antes de partir para os textos da antologia. No caso específico de
Machado de Assis, essa atividade talvez seja ainda mais necessária,
devido ao caráter canônico da obra desse escritor. De forma simples,
trata-se de responder à pergunta ``por que Machado de Assis é
considerado o maior escritor da Literatura Brasileira?'' e ``por que é
necessário estudá-lo ainda hoje, cerca de 140 anos depois de sua
atividade literária mais marcante?''.

\paragraph{Metodologia}
\begin{enumerate}
\item
Dividir os estudantes em grupos de pesquisa que analisarão diferentes
passagens da vida e da obra machadianas e as apresentarão aos colegas e
ao professor. Sugerimos a divisão a seguir:
\BNCC{EM13LP30}

\begin{enumerate}
\item A infância de Machado de Assis: nascimento no Morro do Livramento;
classe social da família de Machado de Assis; o contexto econômico,
social e político do Rio de Janeiro na virada da década de 1830 para a
de 1840;

\Image{Casa em que morou Machado de Assis no Cosme Velho (Acervo Digital da Biblioteca Nacional; Domínio Público)}{PNLD0026-04.png}

\item O jovem Machado de Assis: as primeiras amizades literárias do autor e
a ``Sociedade Petalógica''; a importância da personalidade de Paula
Brito para a formação de Machado de Assis; a importância do jornalismo
na vida de Machado de Assis;

\SideImage{Desenho do escritor brasileiro e primeiro editor de Machado de Assis, Francisco de Paula Brito  (Acervo Digital da Biblioteca Nacional; Domínio Público)}{PNLD0026-10.png}

\item Primeiros romances de Machado de Assis: importância e recepção dos
romances de Machado de Assis (\emph{Ressurreição}, 1872; \emph{A Mão e a
Luva}, 1874; \emph{Helena}, 1876; \emph{Iaiá Garcia}, 1878); por que
esses não são considerados os \emph{grandes romances} do autor?

\item Maturidade da obra de Machado de Assis: importância e recepção dos
chamados romances da maturidade (\emph{Memórias Póstumas de Brás Cubas},
1881; \emph{Quincas Borba}, 1892; \emph{Dom Casmurro}, 1900; \emph{Esaú
e Jacó}, 1904; \emph{Memorial de Aires}, 1908); por que esses romances
foram considerados \emph{grandes}?

\item A importância da Academia Brasileira de Letras e as críticas que se
podem fazer a ela: O que é a Academia Brasileira de Letras? Qual a
importância dessa academia na vida de Machado de Assis? O que ela
significa na vida cultural brasileira? Quais críticas podem ser feitas à
Academia Brasileira de Letras? Quais são os episódios recentes que
revelam a importância e os limites dessa instituição? Lembrar, aqui, a
polêmica referente à candidatura da escritora Conceição Evaristo à
cadeira de número 7 da ABL, em 2018;

\item A polêmica a respeito da cor da pele de Machado de Assis: O que é o
chamado ``branqueamento'' de Machado de Assis? Quais são os exemplos
concretos desse processo? Qual é o contexto em que esse processo ocorre?
Por que esse processo ocorre? Existem outras personalidades públicas do
Brasil que também foram branqueadas? Em que medida o chamado racismo
estrutural brasileiro contribuiu para esse processo?

\item Machado de Assis fora do Brasil: recentemente, Machado de Assis foi
publicado com sucesso nos Estados Unidos. Quais são os textos de Machado
de Assis publicados no exterior? Quais estudiosos estrangeiros
analisaram a obra de Machado de Assis fora do Brasil? Por que a obra de
Machado de Assis teve sucesso fora do Brasil?

\Image{Machado de Assis aos 25 anos (Acervo Digital da Biblioteca Nacional; Domínio Público)}{PNLD0026-05.png}

\end{enumerate}

\item
Cada um dos grupos deve pesquisar os temas apresentados em arquivos
da internet. Cabe ao professor orientar esse processo de pesquisa. Quais
são as fontes mais confiáveis? Quais as menos confiáveis? Por quê?
\BNCC{EM13LP12}

\item
Depois de completo o processo de pesquisa, os estudantes podem
apresentar os resultados obtidos de uma das seguintes maneiras:

\begin{enumerate}
\item podcast de até 15 minutos, com entrega de roteiro na forma de texto;
\BNCC{EM13LP34}

\item vídeo publicado no YouTube, no formato de documentário, em até 15
minutos, com entrega de roteiro na forma de texto;

\item peça teatral de até 15 minutos, com entrega de roteiro na forma de
texto;
\BNCC{EM13LP53}

\item leitura em voz alta de folheto de cordel, de até 15 minutos, com
entrega desse folheto ao professor;

\item apresentação formal tradicional, de até 15 minutos, com entrega de
relatório de pesquisa;

\item declamação de poema no formato de slam poesia, de até 15 minutos, com
entrega do poema escrito;

\item canção de até 15 minutos, que pode ser de composição original dos
estudantes, ou paródia de canção já existente, com entrega da letra e da
cifra.
\end{enumerate}

\item
É necessário destacar que cada uma das formas de apresentação dos
resultados requer dos estudantes a pesquisa sobre o gênero específico
dos textos;

\item
Em qualquer um dos formatos de relatório, é fundamental que, além de
apresentar fatos objetivos a respeito da vida e da obra de Machado de
Assis, os estudantes argumentem em defesa de um ponto de vista;

\item
Depois das apresentações, os alunos deverão apresentar aos colegas as
dificuldades enfrentadas nos processos de pesquisa e preparação dos
relatórios específicos. Deverão também apresentar o percurso por meio do
qual chegaram às conclusões e pontos de vista propostos nos relatórios a
respeito das polêmicas da vida e da obra de Machado de Assis.
\end{enumerate}

\paragraph{Tempo estimado}

\subsection{Leitura}


\paragraph{Tema} Os contos na obra de Machado de Assis

\paragraph{Conteúdo} 
\begin{enumerate}
\item
Leitura, compreensão e análise das obras da
antologia; 
\item
Características do conto; 
\item
Elementos gerais da
narrativa: foco narrativo, enredo, personagens, tempo e espaço.
\end{enumerate}

\paragraph{Objetivos:} 
\begin{enumerate}
\item
Estimular a leitura das obras da antologia, em grupo ou individualmente; 
\item
Promover a reflexão sobre os elementos que compõem a narrativa (foco narrativo, enredo, personagens, tempo e espaço);
\item
Ler, compreender, analisar e comparar os textos da antologia.
\end{enumerate}

\paragraph{Justificativa} As edições de 2011, 2015 e 2019 da pesquisa
Retratos do Brasil revelam um dado inegável: a influência do professor
na formação dos leitores. Para despertar o gosto pela leitura entre
nossos alunos, precisamos ser \emph{mediadores} entre a obra, sua
linguagem, suas estruturas e o estudante, de preferência estabelecendo
uma relação fundamentada no prazer, na identificação e na liberdade de
interpretação. Eis o nosso desafio: \emph{ler com os alunos},
apresentando-lhes as passagens decisivas de um texto -- porque são
engraçadas, assustadoras, emocionantes etc. --, explicando-lhes
\emph{por que} elas chamam a atenção, estabelecendo \emph{conexões}
entre elas e nossa vida presente, revelando as intuições do autor quanto
às práticas sociais atuais, seus conflitos, dilemas, conquistas, ouvindo
as impressões dos estudantes a respeito de tudo isso e propondo-lhes a
análise do que eles gostam -- canções, séries de TV ou de plataformas de
\emph{streaming}, novelas, filmes e outros produtos culturais -- a
partir dessa comparação. Dizendo de maneira simples, o processo de
formação de leitor deve estar baseado no repertório de nossos próprios
estudantes, a partir do qual podemos estabelecer os pontos de contato
com as obras que lhes apresentamos, de forma a ampliar esse repertório.

\paragraph{Metodologia}
\begin{enumerate}
\item
A primeira atividade imprescindível é a leitura, compreensão e
análise dos textos da antologia. O ideal é que alguns textos sejam lidos
\emph{integralmente} na sala de aula. Conscientes das dificuldades dessa
atividade no contexto escolar, sugerimos que esse exercício de leitura
seja preparado previamente: os alunos podem preparar-se para uma
\emph{leitura interpretativa}, que não seja monótona, com a orientação
prévia do professor a respeito do enredo e das personagens do texto
escolhido;

\item
O exercício de compreensão do texto se baseia na análise dialogada de
elementos fundamentais de alguns textos da antologia -- que permitirão,
por sua vez, reconhecer as características do conto, como gênero. É
fundamental sugerir aqui que essa análise não seja feita de forma
teórica, anterior à leitura, mas que \emph{seja depreendida a partir da
leitura do texto}. Sabemos que esse é um exercício desafiador, mas nos
parece ainda o melhor caminho para habilitar os estudantes à compreensão
e à análise de textos literários;
\BNCC{EM13LP46}

\item
Os elementos fundamentais dos textos -- foco narrativo, enredo,
personagens, espaço e tempo -- revelarão as características gerais do
conto e as especificidades de cada um deles. Na teoria tradicional, o
conto se caracteriza pela unidade da ação, em torno de um só conflito,
com restrição do espaço, do tempo e evidentemente das personagens
envolvidas. ``A Cartomante'' corresponde, de forma geral, a essa
proposição geral das feições que o conto deve ter, de maneira que esse
conto pode servir de ponto de partida para outros. Por outro lado, a
análise de ``Pai contra Mãe'' revelará um conto cujos primeiros
parágrafos aproximam-se da narrativa histórica, com a descrição de
práticas sociais, hábitos e instrumentos ligados à escravidão -- e cabe
ao professor a análise dos efeitos dessa particularidade;

\item
A comparação entre textos da antologia é, portanto, fundamental para
a compreensão, não apenas dos elementos internos de cada um deles, mas
também do gênero a que pertencem, de maneira geral;
\BNCC{EM13LP49}

\item
Depois de lidos, compreendidos e analisados, os textos devem ser
interpretados pelos estudantes com a mediação do professor. Esse
processo é aprofundado na atividade de pós-leitura.
\end{enumerate}

\paragraph{Tempo estimado}

\subsection{Pós-leitura}

\paragraph{Tema} Quatro gêneros da linguagem jornalística.

\paragraph{Conteúdo} Criação coletiva de um jornal virtual com dois
textos, um podcast e um vídeo: uma reportagem noticiando a captura e o
aborto de Arminda, a mulher escravizada de ``Pai contra Mãe''; uma
entrevista com Pestana, protagonista de ``Um homem célebre''; um
editorial analítico-argumentativo, na forma de podcast, a respeito do
assassinato de Rita e Camilo, relacionando-o ao feminicídio e ao
machismo e criticando a abordagem sensacionalista de um vídeo, também
feito pelos alunos, reproduzindo a linguagem sensacionalista dos
programas policiais de televisão.
\BNCC{EM13LGG303}
\BNCC{EM13LGG304}

\paragraph{Objetivos} 
\begin{enumerate}
\item
Aprofundamento da leitura dos textos da antologia
por meio da adaptação a outras linguagens; 
\item
Valorização da organização
coletiva do trabalho, de modo a alcançar um resultado final em que as
diferentes partes estejam coerentemente organizadas; 
\item
Produção de
conteúdos críticos a respeito de assuntos polêmicos, antecedida de
debates; 
\item
Debates a respeito das diferentes linguagens e dos pontos de
contato entre elas; 
\item
Criação de página da internet, de modo a revelar
o conhecimento da linguagem e comunicação digital.
\end{enumerate}

\paragraph{Justificativa} Enquanto os exercícios de leitura, compreensão e
análise marcam-se, de forma geral, pelas primeiras aproximações do
texto, seguidas de atividades de descrição de suas características, as
práticas interpretativas abrem espaço para que os estudantes avaliem,
opinem e expressem juízos de valor quanto às leituras. Os professores
devemos estimular essa dinâmica, na exata medida em que ela serve ao
exercício do espírito crítico e abre canais para que possamos fazer
mediações entre a realidade dos textos analisados e a de nossos
estudantes.

Acreditamos que a criação de um jornal virtual permite:
\begin{enumerate}
\item
Que os alunos
aprofundem a leitura dos textos da antologia para adaptá-los a outras
linguagens; 
\item
Que o professor avalie uma ampla gama de competências e
habilidades dos estudantes, de acordo com as potencialidades de cada um;
\item
Que os alunos produzam coletivamente conteúdos críticos, em formatos
diversos; 
\item
Que professor e alunos debatam as diferentes linguagens e
os pontos de contato entre elas; 
\item
Que a produção dos alunos ganhe
visibilidade no meio digital, de forma a contar com a avaliação não
apenas no contexto escolar, mas também no ambiente das redes, de acordo
com as interações alcançadas; 
\item
Que assuntos polêmicos do cotidiano
sejam debatidos pelos alunos, de modo a estimular o debate, a
argumentação, o espírito crítico e a negociação.
\end{enumerate}

\paragraph{Metodologia}
Para criar um jornal virtual, o professor deverá dividir os estudantes
em grupos cujos objetivos correspondam às habilidades com as quais os
alunos se identifiquem e/ou a respeito das quais eles pretendam
conhecer. Sugerimos a seguinte divisão:
\BNCC{EM13LP18}

\begin{enumerate}
\item grupo dos alunos de redação da reportagem a respeito da captura e
aborto de Arminda, relatando o enredo de ``Pai contra Mãe'' na forma
jornalística, exercitando a paráfrase. O primeiro desafio deste grupo é
a redação de um texto repleto de marcadores típicos da impessoalidade
característica das reportagens, em oposição aos textos opinativos,
adaptando os eventos da novela de Machado de Assis para o contexto e a
linguagem do século XXI. Além disso, esse grupo também deve ser capaz de
produzir uma paráfrase sucinta, mas completa, dos eventos relatados no
conto. Note-se, ainda, que o processo de redação, aqui, é coletivo:
deve-se estimular que os alunos se dividam em grupos de \emph{redatores}
(os que efetivamente escrevem o texto), \emph{revisores} (os que fazem a
revisão gramatical do texto) e \emph{editores} (responsáveis pela
coerência do texto como um todo, já que ele terá sido escrito a muitas
mãos);

\Image{Manuscrito do capítulo ``A moeda de Vespasiano'' do livro ``Memórias Póstumas de Brás Cubas'' de Machado de Assis, 1884 (Acervo Digital da Biblioteca Nacional; Domínio Público)}{PNLD0026-07.png}

\item grupo dos alunos de redação da entrevista com Pestana, protagonista
do conto ``Um homem célebre'', em que ele relata, em primeira pessoa, a
trajetória para o sucesso, a fama na cidade do Rio de Janeiro, a morte
da esposa e o desejo de compor música erudita, em vez de polcas. O
grande desafio deste grupo é transpor a linguagem do conto de Machado de
Assis para a oralidade típica das entrevistas, adaptando os eventos para
o contexto e a linguagem do século XXI. Também deve estar disponível, ao
final da página da transcrição da entrevista, um arquivo virtual com um
trecho dessa entrevista, representado por dois alunos, um no papel de
entrevistado, outro no de entrevistador;

\item grupo dos alunos de redação do editorial crítico ao assassinato de
Rita, na forma de podcast, enfatizando as questões polêmicas sugeridas
no texto de Machado de Assis: especialmente o feminicídio e o machismo.
Do ponto de vista do gênero textual, o desafio deste grupo é redigir um
texto analítico-argumentativo, na linguagem de podcast, atualizando o
conto de Machado de Assis para o contexto do Brasil atual, partindo do
princípio de que o enredo de ``A Cartomante'' aconteceu no século XXI,
fazendo as adaptações necessárias. Além disso, a equipe de redatores
deve pesquisar os temas do feminicídio e do machismo, analisar-lhes as
implicações um no outro, verificar-lhes a atualidade e \emph{tomar
partido};

\item grupo de alunos que redigirão o roteiro do programa de TV
sensacionalista, em que será apresentada uma reportagem a respeito do
assassinato de Rita e Camilo. O desafio desses alunos é reproduzir a
linguagem violenta desses programas no limite da paródia, de modo a
evidenciar a crítica que se pode fazer a eles;

\item grupo dos alunos que trabalharão com as imagens, seu tratamento e sua
adequação aos textos. Deve-se estimular a criação de imagens próprias,
de autoria dos alunos, baseadas na iconografia tradicional a respeito de
Machado de Assis. Quanto mais autorais forem as ilustrações da página,
mais valorizadas elas serão. Para a reportagem, sugere-se ilustração com
representações de Arminda e Cândido Neves, além de outras que os
estudantes julgarem significativas; para a entrevista, sugere-se a
representação de Pestana, além de imagens que ilustrem as festas citadas
no conto, em que as polcas do protagonista são executadas com sucesso;
finalmente, para o editorial, sugere-se a criação de gráficos que
ilustrem os casos de feminicídio no Brasil;

\item grupo dos alunos que organizarão a página da internet, suas fontes,
disposição, legibilidade. A tarefa mais desafiadora deste grupo é o
diálogo com os outros grupos. A identidade visual final da página da
internet deve corresponder tanto à iconografia criada pelo grupo das
imagens quanto ao conteúdo dos textos criados pelos grupos de redação;

\item grupo dos alunos que formularão estratégias de divulgação da página
da internet, bem como apresentarão os resultados dessa divulgação e
serão responsáveis pela interação com os leitores. O desafio deste grupo
é conhecer a fundo o conteúdo formulado pelos grupos de redação, de modo
a produzir respostas coerentes na interação com leitores. A qualidade do
trabalho desse grupo também será avaliada de acordo com o número de
leitores alcançado, de preferência extrapolando as ``bolhas'' virtuais
dos próprios alunos da sala;

\item grupo de alunos envolvidos na produção do vídeo. O desafio desses
alunos está na criação de uma ambiência convincente, baseada em cenário,
figurino, iluminação e sonoplastia que reproduzam a linguagem violenta
dos programas policiais. Da mesma maneira, é preciso escolher os alunos
que farão o papel de atores, de acordo com o texto preparado pelos
grupos de roteiristas, e os que se responsabilizarão pelas atividades
técnicas: direção, edição e publicação do vídeo;

\item grupo dos alunos envolvidos na produção do podcast. O primeiro
desafio desses alunos é encontrar uma plataforma gratuita para a
divulgação do material, que deve ser precedido de vinheta atrativa para
o público, com locução adequada ao formato.
\end{enumerate}

Evidentemente, a proposta apresentada acima pode ser adaptada a
quaisquer contos da antologia, de acordo com a abordagem do professor.

\paragraph{Tempo estimado}

\section{Atividades 2}

\subsection{Pré-leitura}


\paragraph{Tema} Relações entre história e literatura

\paragraph{Conteúdo} Comparação entre textos e discursos da história e da
literatura.

\paragraph{Objetivo} 
\begin{enumerate}
\item
Analisar visões de mundo, conflitos de interesse e
ideologias presentes nos discursos históricos e ficcionais; 
\item
Analisar semelhanças e diferenças entre esses discursos; 
\item
Analisar diferentes
graus de parcialidade e imparcialidade nesses discursos, levando em
consideração os recursos de linguagem utilizados para obter os efeitos
pretendidos.
\end{enumerate}

\paragraph{Justificativa} Os limites entre o discurso histórico e o
discurso ficcional têm motivado, de um lado, pesquisas que consideram
que o texto literário é documento histórico precioso e, de outro, obras
literárias que, para além dos romances históricos, questionam a validade
das narrativas históricas oficiais, recontando, de maneira crítica,
irônica ou paródica, eventos do passado e suspeitando de suas versões
consagradas. Entendemos que, em ``Pai contra Mãe'', por exemplo, Machado
de Assis antevia, de certa forma, os limites difusos entre história e
ficção, de modo que propomos aqui a reflexão sobre esse conto.

\paragraph{Metodologia}

\begin{enumerate}
\item Em diálogo com os alunos, o professor deve apresentar as
especificidades do discurso histórico, seus alcances e limites, sem
preocupar-se com exposição extensiva de textos teóricos. Um exemplo
bastante interessante, para dar início ao debate, é a distinção
estabelecida por Aristóteles, na \emph{Poética}, proposta a seguir de
forma geral: a famosa afirmação de que o historiador conta \emph{o que
aconteceu}, enquanto o poeta conta \emph{o que poderia ter acontecido};

\item Depois desse debate inicial, o professor pode dar exemplos da
parcialidade dos discursos históricos a partir de pontos de vista
específicos. É importante fazê-lo a partir de exemplos concretos que se
inscrevam na linguagem, como as diferentes designações de um mesmo
evento: na perspectiva dos povos indígenas, a \emph{invasão} de suas
terras em 1500; na perspectiva dos portugueses, o \emph{descobrimento}
do Brasil. Da mesma maneira, em 1964, o \emph{Golpe Civil-Militar}, do
ponto de vista do campo democrático, e a \emph{Revolução de 1964}, dos
militares. Trata-se, fundamentalmente, de preparar os alunos para o
discurso que encontrarão na leitura dos primeiros parágrafos de ``Pai
contra Mãe'';

\item Estimular entre os estudantes o debate contemporâneo a respeito da
disputa de narrativas e da pós-verdade. De maneira geral, propor a eles
uma reflexão adensada sobre eventos recentes interpretados de maneiras
distintas de acordo com pressupostos diversos;
\BNCC{EM13LP40}

\item Embora esta seja uma atividade de pré-leitura, sugerimos a análise
dos primeiros parágrafos do conto ``Pai contra Mãe'', sem a preocupação
da leitura integral desse texto. Mostre aos alunos como os primeiros
parágrafos desse conto se aproximam do discurso historiográfico. É o
caso de levantar hipóteses a respeito das intenções do autor ao
redigi-los;

\item Proponha aos alunos uma pesquisa a respeito dos castigos impostos às
pessoas escravizadas, a partir do exemplo da máscara de folha de
flandres descrita no texto. Expanda a pesquisa para além dos costumes e
proponha também uma análise da violência presente na sociedade
brasileira do século XIX, fosse ela mais ou menos explícita. O que
interessa aqui é que os alunos percebam como a escravidão e a
brutalidade que lhe é inerente pautaram nossas formas de convívio,
nossas instituições e nossas ideias;
\BNCC{EM13LP12}

\item Proponha a seguir as formas de cultura e resistência das pessoas
escravizadas no Brasil e verifique de que maneira essas formas também
são matrizes de nossa cultura;

\item O ponto de chegada desses debates deve ser a análise das culturas
africanas que chegaram ao Brasil. Peça aos alunos que façam um
levantamento dos povos, idiomas, visões de mundo, hábitos, religiões,
danças e canções dos africanos que foram trazidos a nosso país. Quais
são os traços dessas culturas que estão presentes naquilo que chamamos
de culturas brasileiras? Qual o legado dessas culturas?

\item Apenas a título de exemplo: um ponto de partida muito interessante é
a análise da história da capoeira, que era forma de luta e resistência
dos povos africanos no Brasil e que se revestiu da feição de dança, para
proteger seus praticantes de perseguições. Como se sabe, treinos de
lutas eram proibidos e reprimidos, mas danças e cantos eram
informalmente permitidos aos escravos, para evitar que fossem acometidos
do \emph{banzo}, isto é, da depressão pela vida que levavam; os treinos
do combate eram, portanto, disfarçados por meio da dança, que guarda até
hoje um duplo caráter, entre o desfrute lúdico do corpo e a finalidade
marcial, combativa;

\item Sugerimos, finalmente, que você proponha aos estudantes a criação
colaborativa de um blog informativo a respeito das culturas africanas no
Brasil. O blog deve conter textos, vídeos e podcasts;
\BNCC{EM13LP19}

\item Sugere-se a divisão dos alunos nos seguintes grupos:

\begin{enumerate}
\item grupo dos alunos de \emph{redação, revisão e edição} dos textos e dos
roteiros dos vídeos e podcasts. O primeiro desafio deste grupo é a
redação de textos repletos de marcadores típicos da impessoalidade
característica dos textos informativos. Além disso, esse grupo também
deve ser capaz de produzir paráfrases sucintas das informações
pesquisadas. Note-se, ainda, que o processo de redação, aqui, é
coletivo: deve-se estimular que os alunos se dividam em grupos de
\emph{redatores} (os que efetivamente escrevem o texto),
\emph{revisores} (os que fazem a revisão gramatical do texto) e
\emph{editores} (responsáveis pela coerência do texto como um todo, já
que ele terá sido escrito a muitas mãos);

\item grupo dos alunos que trabalharão com as imagens ilustrativas, seu
tratamento e sua adequação aos textos produzidos pelo grupo anterior.
Deve-se estimular a criação de imagens próprias, de autoria dos alunos.
Quanto mais autorais forem as ilustrações do blog, mais valorizadas elas
serão;

\item grupo dos alunos que organizarão o blog na internet, suas fontes,
disposição, legibilidade. A tarefa mais desafiadora deste grupo é o
diálogo com os outros grupos. A identidade visual final do blog deve
corresponder tanto à iconografia criada pelo grupo das imagens quanto ao
conteúdo dos textos criados pelos grupos de redação;

\item grupo dos alunos que formularão estratégias de divulgação do blog,
bem como apresentarão os resultados dessa divulgação e serão
responsáveis pela interação com os leitores. O desafio deste grupo é
conhecer a fundo o conteúdo formulado pelo grupo de redação, de modo a
produzir respostas coerentes na interação com leitores. A qualidade do
trabalho desse grupo também será avaliada de acordo com o número de
leitores alcançado, de preferência extrapolando as ``bolhas'' virtuais
dos próprios alunos da sala.
\end{enumerate}

\item Uma maneira de envolver os estudantes na pesquisa é permitir a
criatividade na apresentação dos resultados. Dando continuidade ao
exemplo do item 8, a respeito da capoeira: uma forma de apresentação
dessa pesquisa é a própria dança. A expressão corporal e a dança devem
ser bem-vindas à sala de aula, como se prevê na Base Nacional Comum
Curricular. Assim, também são adequadas, como formas de apresentação dos
resultados de pesquisa, quaisquer expressões corporais, que podem ser
registradas em vídeo ou apresentadas ao vivo, para a turma.
\end{enumerate}

\subsection{Leitura}

\paragraph{Tema} Relações entre história e literatura.

\paragraph{Conteúdo:} Análise:
\begin{enumerate}
\item
do pano de fundo histórico do conto ``Um Homem Célebre''; 
\item
das relações entre arte, entretenimento e política; 
\item
das matrizes da música popular brasileira.
\end{enumerate}

\paragraph{Objetivo}
\begin{enumerate}
\item
Analisar visões de mundo, conflitos de interesse e
ideologias presentes nos discursos históricos e ficcionais; 
\item
Analisar
semelhanças e diferenças entre esses discursos; 
\item
Analisar a importância de conteúdos históricos na estruturação da obra literária;
\item
Analisar as relações entre arte e sociedade; 
\item
Avaliar a importância da canção popular brasileira na cultura brasileira.
\end{enumerate}

\paragraph{Justificativa} É conhecida a importância dos eventos históricos
na estruturação da obra de Machado de Assis, especialmente desde a
publicação da pesquisa de John Gledson, no clássico estudo \emph{Machado
de Assis: Ficção e História} e na obra mais recente \emph{Por um novo
Machado de Assis}. A abordagem histórica contribui, portanto, não só
para a compreensão do desfecho do conto, mas sobretudo para o
entendimento do projeto literário machadiano.
\BNCC{EM13LP01}

\Image{Estátua de bronze de Machado de Assis na fachada do prédio da Academia Brasileira de Letras (Escultura de Humberto Cozzo; Domínio Público)}{PNLD0026-08.png}

Uma consequência direta dessa análise é o debate a respeito da crítica
do autor ao mercado musical de sua época que, em termos gerais, pode ser
bastante fértil na sala de aula. Sob a influência do editor do
protagonista do conto, os títulos das polcas são associados a eventos
marcantes das disputas políticas do tempo, com o objetivo de popularizar
as canções e vendê-las mais. Essa relação entre política, arte e
entretenimento continua atual e presente, especialmente no que se refere
à veiculação das causas por meio da canção, do cinema e das séries
disponíveis nas plataformas de \emph{streaming}.

Finalmente, cabe avaliar com os alunos a importância da canção popular
brasileira no conjunto da nossa cultura, já que o Brasil se destaca
nessa forma de arte. Lembremos que nosso país, desde a década de 1990, é
dos poucos do planeta que consomem mais sua própria produção de canção
do que a norte-americana.

\paragraph{Metodologia}

\begin{enumerate}
\item Sugerimos que o professor leia com os alunos, integralmente, o conto
``Um homem célebre'', inventariando com eles os trechos em que episódios
históricos são importantes. Geralmente, os leitores mais inexperientes
não se importam com as datas, mas a compreensão do desfecho do conto --
a ``única pilhéria'' pronunciada por Pestana em toda a vida -- requer a
compreensão da alternância de conservadores e liberais na composição dos
gabinetes de ministros, desde 1871, quando Pestana compõe a primeira
polca, até 1885, quando falece. Evidentemente, o professor de História
identificará nesses anos a Crise do Segundo Império, o pano de fundo do
conto;

\item Identificadas as datas, os acontecimentos e o processo pelo qual
passava a sociedade brasileira naqueles anos, é o momento de propor aos
alunos o debate a respeito das relações entre arte, política e
entretenimento, orientado pelas seguintes perguntas: existe
entretenimento sem implicações políticas? O que distingue a arte
engajada politicamente da arte de entretenimento, supostamente sem
compromisso político? É possível mobilizar politicamente o público, por
meio da arte? É legítimo fazer isso?

\item Feito esse debate, sugira aos alunos que se dividam em grupos, que
vão pesquisar diferentes aspectos da história da canção popular
brasileira. Alguns temas bastante interessantes podem ser os seguintes:

\begin{enumerate}
\item as matrizes populares (as chamadas ``Casas das Tias'') e as matrizes
industriais (os primórdios da indústria fonográfica) da canção no
Brasil;
\BNCC{EM13LP50}

\item os primeiros grandes compositores e a importância do carnaval e da
Era do Rádio;
\BNCC{EM13LP13}

\item a importância da Bossa-Nova;

\item a importância da MPB engajada, nos festivais da canção;

\item a importância do Tropicalismo;

\item a importância da Jovem Guarda;

\item a importância de compositores de difícil classificação, como Tim Maia
e Raul Seixas;

\item a importância do rock dos anos 80;

\item a importância da música sertaneja;

\item a importância do pagode dos anos 80 e 90;

\item a importância da axé music dos anos 90;

\item a importância do funk;

\item a importância do rap.
\end{enumerate}

Evidentemente, os temas acima são apenas sugestões.

\item O resultado da pesquisa acima pode ser apresentado na forma de
podcast. Para criá-lo, os estudantes devem encontrar uma plataforma
gratuita para a divulgação do material, que deve ser precedido de
vinheta atrativa para o público, com locução adequada ao formato. Eles
também devem criar um roteiro para o podcast, de maneira a prender a
atenção do ouvinte.
\end{enumerate}

\paragraph{Tempo estimado}

\subsection{Pós-leitura}

\paragraph{Tema} Violência na sociedade brasileira.

\paragraph{Conteúdo} Análise e reflexão a respeito do conto ``O Enfermeiro''.

\paragraph{Objetivo}
\begin{enumerate} 
\item
Promover debate a respeito da violência na
sociedade brasileira; 
\item
Analisar visões de mundo, conflitos de
interesse e ideologias presentes no discurso ficcional; 
\item
Analisar
diferentes graus de parcialidade e imparcialidade nesse discurso,
levando em consideração os recursos de linguagem utilizados para obter
os efeitos pretendidos.
\end{enumerate}

\paragraph{Justificativa} A violência está presente na literatura e na
realidade em que ela é produzida. Dessa forma, cabe ao professor avaliar
com os estudantes, em primeiro lugar, a produção e reprodução de
processos sociais violentos e suas manifestações no discurso literário.
\BNCC{EM13LGG304}
\BNCC{EM13LGG305}

\paragraph{Metodologia}
\begin{enumerate}
\item Leia, compreenda e analise, com os alunos, o conto ``O Enfermeiro''.
Nesse processo, identifique, especialmente, as passagens em que se
manifesta a brutalidade de Coronel Felisberto e sua consequência mais
violenta: a vingança de Procópio, o narrador, que acaba por assassinar o
Coronel;

\item Proponha um debate a respeito do tipo social do Coronel: um grande
proprietário de terras, escravista e violento. Qual é a estrutura social
que ampara e justifica a existência de homens como esse? Para preparar
sua aula, leia o famoso texto de Roberto Schwarz, ``As ideias fora do
lugar'', em que ele apresenta o \emph{favor} como nexo entre os homens
livres e dependentes da sociedade brasileira (como é o caso de Procópio)
e os poderosos (como o Coronel). Tente refletir com os alunos a respeito
da \emph{mentalidade escravista} do Coronel e sua lógica, de que o abuso
é parte integrante;

\item Depois de identificar com os alunos os tipos sociais encarnados nas
personagens machadianas, proponha as seguintes perguntas, que devem
orientar a reflexão cujos resultados eles apresentarão na forma de
argumentações orais (descritas a seguir, nos itens de XX a XX): quais
são os limites da legítima defesa? Na noite em que foi atacado pelo
Coronel Felisberto, Procópio passou dos limites? Levando em consideração
a posição social das personagens, caso o crime se tornasse público e
fosse a julgamento, é possível atenuar ou aumentar uma eventual pena a
cumprir?
\BNCC{EM13LP25}

\item Agora, proponha aos alunos a seguinte situação: o assassinato
cometido por Procópio foi descoberto, e ele será julgado por júri
popular. Você, professora ou professor, será o juiz que organizará o
julgamento encenado na sala. Os alunos devem dividir-se nos seguintes
grupos:
\BNCC{EM13LP27}

\begin{enumerate}
\item grupo dos \emph{advogados de defesa}, que devem articular a defesa de
Procópio, por escrito e oralmente;

\item grupo da \emph{promotoria}, que deve articular a acusação de
Procópio, por escrito e oralmente;

\item grupo do \emph{júri popular}, que deve avaliar os argumentos
apresentados e formular uma decisão tomada coletivamente, em maioria
absoluta, por escrito e oralmente.
\end{enumerate}

Evidentemente, os alunos devem utilizar, na argumentação, não apenas os
conhecimentos a respeito da sociedade brasileira, debatidos na primeira
parte da atividade, como também outras informações a respeito da justiça
brasileira, pesquisadas antes da apresentação. A encenação dos papéis de
defesa, promotoria e júri pode ser bastante envolvente.
\end{enumerate}

\section{Aprofundamento}

Dizendo de maneira simples, a leitura de Machado de Assis sempre exigirá
atenção redobrada do professor e dos alunos. Os contos desse autor,
especialmente, pela concisão, guardam pistas de análise que, muitas
vezes, não são notadas na primeira leitura e guardam chaves
analítico-interpretativas preciosas. A título de exemplo, proporemos
aqui uma breve análise do conto ``Um Homem Célebre'', cujos elementos
gerais podem servir de referência para o estudo de outros contos da
antologia. Esta análise ampara a Atividade 2 de Leitura, acima
apresentada.

Imaginemos que o conto tem ``camadas'' de leitura e análise; nossa
finalidade, como professores, é ir tão fundo quanto possível com os
alunos. A camada mais superficial da leitura de ``Um Homem Célebre'' é o
encadeamento de eventos do enredo, nos quais se destaca o Pestana, nosso
protagonista, talentoso compositor de polcas, com alguma fama na cidade
do Rio de Janeiro, mas insatisfeito pela feição popular dessas
composições. Pestana ambiciona ser um compositor de música erudita, e
seus artistas preferidos nos são apresentados pelo narrador do texto:
Mozart, Beethoven Bach e Schumann, para citar os mais famosos. É esse o
primeiro dado do texto que chama a atenção e que pode ser um ponto de
partida para a reflexão mais aprofundada do conto, que sugeriremos a
seguir; antes, concluamos esta breve paráfrase do enredo. Pestana decide
casar-se com uma viúva de vinte e sete anos, ``boa cantora e tísica'',
isto é, acometida de tuberculose. O músico espera que o matrimônio e a
esposa o conduzam pelo caminho da criação de uma música da qualidade dos
mestres, mas é tudo em vão: mesmo depois da morte dela, ele não consegue
sequer compor um \emph{réquiem}, uma missa em homenagem à mulher
falecida. Desde então, Pestana vai perdendo o sentido de viver, sempre
compondo polcas para sustentar-se até a morte, em 1885, quatorze anos
depois da primeira composição que o celebrizou. Chama a atenção, aqui, a
figura do editor que vendia as polcas e lhes atribuía títulos de acordo
com os acontecimentos políticos do tempo, para popularizar ainda mais as
obras. No leito de morte, pouco antes de falecer, Pestana promete não
uma, mas duas polcas ao editor: uma para a ocasião da subida dos
conservadores ao poder, uma para a dos liberais. É a única pilhéria de
sua vida, cujo fim coincide com o fim do conto.

\Image{Panorama em 3 partes da Enseada do Botafogo, entre 1860 e 1879 (Acervo Digital da Biblioteca Nacional; Domínio Público)}{PNLD0026-09.png}

Inicialmente, há dois elementos a aprofundar: o primeiro, o oportunismo
do editor, que atribui às polcas títulos de ocasião, de acordo com o
sabor das mudanças políticas. O segundo é o contexto político em que se
passam as ações do conto. Analisaremos cada um deles, a seguir.

As relações entre arte, política e entretenimento podem render um debate
bastante interessante com nossos estudantes. As implicações do
engajamento político da obra de arte podem ser introduzidas a partir do
repertório dos alunos, isto é, das canções populares que ouvem e das
séries ou filmes a que assistem. A título de exemplo: os estudantes que
são fãs de \emph{rap} e da cultura \emph{Hip hop} como um todo
certamente terão bastante a contribuir no que se refere às relações
entre arte e política; aqui, os produtos culturais são vistos não apenas
como objetos de desfrute estético, mas também como veículos de crítica
aos discursos dominantes. Por outro lado, outros gêneros são
aparentemente menos afeitos ao enfrentamento político.

É nesse momento do debate que a professora ou professor pode intervir,
adicionando ao debate uma nova problematização: a de que o caráter de
\emph{produto} das obras de arte, em alguma medida, pode relativizar sua
\emph{perspectiva crítica}. O conto de Machado de Assis contribui
sensivelmente para esse debate: as polcas de Pestana são pequenas peças
de entretenimento, sem compromisso político; é, aliás, esse caráter
descomprometido que frustra o artista. Para ele, o entretenimento se
opõe à grande arte dos músicos que ele admira. Acrescente-se aí a
associação dos títulos aos episódios políticos do momento -- e teremos a
chance de perceber em que medida a lógica de mercado absorve outras
esferas da vida humana, como a arte e a política.

As perguntas derivadas da análise acima são muitas. Apresentamos a
seguir algumas delas:

\begin{enumerate}
\item Em que medida o caráter de produto consumível relativiza a
perspectiva crítica contida em uma obra de arte?

\item Em que medida o engajamento político relativiza o valor estético de
uma obra de arte?

\item Quais são as consequências da redução da arte e da política ao
caráter de produtos?
\end{enumerate}

Na Bibliografia deste material, sugeriremos a leitura do famoso texto
``A obra de arte na época de sua reprodutibilidade técnica'', do
filósofo alemão Walter Benjamin, para qualificar o debate proposto
acima. Nesse ensaio penetrante, o autor reflete a respeito do
\emph{valor de culto} que tinham as obras de arte no período
pré-capitalista; paulatinamente, conforme a perspectiva de mercado se
consolida, culminando com a Revolução Industrial do século XIX, as obras
perdem o valor de culto e ganham \emph{valor de exibição}, de acordo com
sua circulação. Trata-se de uma reflexão estética densa, que pode trazer
à sala de aula bons debates.

O contexto político brasileiro dos vinte anos que precedem a Proclamação
da República pode iluminar a discussão acima. Como se sabe, desde a
Guerra do Paraguai, a monarquia brasileira entrou em crise, que se
agravou com a organização de setores mais progressistas, como os
republicanos e os abolicionistas. A alternância dos gabinetes de
ministros conservadores e liberais reflete, em alguma medida, os debates
da época, principalmente no que concerne à libertação das pessoas
escravizadas. A personagem do editor de Pestana, aqui, é bastante
esclarecedora: trata-se de uma personagem que encarna, no conjunto do
conto, o espírito mercantil, que converte arte e política em produto
rentável, como se Machado de Assis quisesse mostrar ao leitor que, para
além das aparências, há uma lógica subjacente, de caráter mercadológico,
que orienta, em boa medida, as ações dos homens e seus efeitos. Dizendo
de maneira simples e objetiva, Machado nos alerta: as polcas e seus
títulos, os conflitos políticos e seus desdobramentos podem ser
entendidos como \emph{espetáculos}, cujo brilho ofusca estruturas menos
visíveis e muito mais profundas.

\Image{Machado de Assis acompanhado de colegas como Pereira Passos, Joaquim Nabuco, José Américo dos Santos, Aloysio de Carvalho, Lafayette Rodrigues Pereira e Gastão da Cunha. (Augusto Malta; Domínio Público)}{PNLD0026-06.png}

\section{Sugestões de referências complementares}\label{sugestoes}

\subsection{Filmes:}

\begin{itemize}


\item \textit{A Cartomante} Direção: Marcos Faria (Brasil, 1974).

A adaptação para o cinema do conto de Machado de Assis, que leva o mesmo título 
do filme, narra a história, em 1871 na cidade do Rio de Janeiro, de uma mulher casada que, 
por ter um caso extraconjugal com o amigo do marido, resolve consultar uma cartomante para 
prever seu destino.

\item \textit{A Cartomante} Direção: Wagner Assis e Pablo Uranga (Brasil, 2004).

Esse filme trata"-se de uma refilmagem da versão de 1974, do mesmo conto 
de Machado de Assis, traduzida para um contexto mais contemporâneo.

\item \textit{Quanto vale ou é por quilo} Direção: Sérgio Bianchi (Brasil, 2005).

Nesse filme, o diretor estabelece uma comparação perturbadora entre o
tráfico de escravizados e a capitalização da miséria por meio do
marketing social, no século \textsc{xxi}. Trata"-se de uma livre adaptação do
conto ``Pai contra Mãe''.

\item \textit{Um homem célebre} Direção: Miguel Faria Júnior (Brasil, 1974).

Nessa outra adaptação do conto de Machado de Assis, ``Um homem célebre'', narra"-se 
a história de um jovem músico que almeja sucesso e fama, mas uma série de eventos 
passam a acontecer em sua vida, desestabilizando"-o e o frustrando.

\end{itemize}

\subsection{\emph{Sites}:}

\begin{itemize}

\item Aquarelas de Jean"-Baptiste Debret:
\href{https://artsandculture.google.com/entity/jean-baptiste-debret/m049vrh}{Google Arts \& Culture}

Para conhecer o Rio de Janeiro do século \textsc{xix}, você pode observar as
aquarelas de \textbf{Jean-Baptiste Debret}. Muitas delas podem ser
encontradas na internet, especialmente na página desse artista nesse site da Google Arts \& Culture.

Debret viveu no Rio de Janeiro nas décadas de 1820 e 1830 e
retratou passagens do cotidiano. Embora esse período seja anterior ao da
vida de Machado de Assis, a obra de Debret é um ótimo recurso visual para
conhecer o cotidiano carioca, no qual a escravidão chama a atenção.

\item Rio de Janeiro Antigo: \href{http://brasilianafotografica.bn.br}{Brasiliana Fotográfica}

Consulte também o \emph{site} da Brasiliana Fotográfica para acessar fotos do Rio de
Janeiro dos últimos anos do século \textsc{xix}.

\end{itemize}


\section{Bibliografia comentada}

\begin{itemize}


\subsection{Panoramas gerais de vida e obra de Machado de Assis}

\item \textsc{bosi}, Alfredo \emph{et al}. \textit{Machado de Assis}. São Paulo: Ática,
1982.

Embora seja de difícil acesso, porque não foi reeditado, esse livro
contém uma das mais alentadas coletâneas de textos de Machado de Assis,
por isso vale a pena indicá-lo. Desde uma biografia inicial, escrita por
Valentim Facioli, da Universidade de São Paulo, até textos de diversos
pesquisadores e especialistas, passando por uma coletânea significativa
dos textos de Machado de Assis, em todos os gêneros nos quais ele se
destacou: crítica, crônica, conto, romance e poesia. Trata-se de
exemplar fundamental em qualquer coleção de estudiosos de Machado de
Assis, apesar da aludida dificuldade de encontrá-lo.

\item \textsc{bosi}, Alfredo. ``Machado de Assis''. In: \_\_\_\_\_\_. \textit{História
Concisa da Literatura Brasileira}. 50.ed. São Paulo: Cultrix, 2015.
p.184-194.

Apresentação geral da obra de Machado de Assis por Alfredo Bosi, um dos
maiores pesquisadores brasileiros da obra desse escritor.

\item \textsc{candido}, Antonio. ``Esquema de Machado de Assis''. In: \_\_\_\_\_\_.
\textit{Vários Escritos}. 3.ed.rev.ampl. São Paulo: Duas Cidades, 1995.
p. 17-40.

Apresentação geral da obra de Machado de Assis por Antonio Candido. Vale
especialmente pelo levantamento dos grandes temas da obra do escritor.

\item \textsc{magalhães jr}., R. \textit{Vida e obra de Machado de Assis}. v.1.
Aprendizado; v.2. Ascensão; v.3. Maturidade; v.4. Apogeu. Rio de
Janeiro: Ed. Record, 2008.

A biografia mais extensa e minuciosa de Machado de Assis.

\item \textsc{merquior}, José Guilherme. ``Machado de Assis e a prosa impressionista''.
In: \_\_\_\_\_\_. \textit{De Anchieta a Euclides: breve história da
literatura brasileira}. São Paulo: É Realizações, 2014. p. 243-321.

Apresentação geral da obra de Machado de Assis por José Guilherme
Merquior, um dos maiores estudiosos desse escritor. Vale especialmente
pelo panorama da prosa realista da segunda metade do século XIX.

\item \textsc{pereira}, Lúcia Miguel. \textit{Prosa de ficção (de 1870 a 1920):
História da Literatura Brasileira.} 2ª ed. Rio de Janeiro: José Olympio,
1957.

Apresentação geral da obra de Machado de Assis por Lúcia Miguel Pereira,
uma das maiores estudiosas dos contos desse escritor.

\subsection{Sobre os contos desta antologia e sobre assuntos a eles relacionados}

\item \textsc{benjamin}, Walter. \textit{A obra de arte na época de sua
reprodutibilidade técnica}. São Paulo: L\&PM, 2018.

Este ensaio clássico de Walter Benjamin é fundamental para estudar a
mercantilização da obra de arte no contexto do século XIX. No conjunto
do texto, o autor se refere, sobretudo, ao cinema, mas o texto pode ser
ponto de partida para reflexões sobre a arte como um todo.

\item \textsc{gledson}, John. \textit{Machado de Assis: ficção e história}. Rio de
Janeiro: Paz e Terra, 2007.

O crítico literário John Gledson mergulhou fundo nas relações entre
história e ficção na obra de Machado de Assis. Embora não contenha a
análise específica dos contos de nossa antologia, este livro serve de
esteio não apenas para a compreensão da economia, da sociedade, da
política e da cultura do tempo de Machado de Assis, mas também para a
análise da obra desse autor.

\_\_\_\_\_\_. \textit{Por um novo Machado de Assis}. São Paulo:
Companhia das Letras, 2006.

Neste livro, o autor de \emph{Machado de Assis: ficção e história}
recolheu ensaios menores, alguns deles de análise a respeito de contos
de nossa antologia. O primeiro é ``O machete e o violoncelo: introdução
a uma antologia dos contos de Machado de Assis'', uma apresentação geral
dos contos de nosso autor. O segundo, ``História do Brasil em
\emph{Papéis Avulsos}, de Machado de Assis'', explica as relações entre
ficção e história nos contos de nosso autor. A seguir, o ensaio ``Conto
de Escola: uma lição de história'' contém uma análise penetrante desse
conto. Finalmente, em ``Machado de Assis e o Rio de Janeiro em vários
tempos'', você pode encontrar a importância da cidade do Rio de Janeiro
na obra de nosso autor.

\item \textsc{gomes}, Laurentino. \textit{Escravidão: do primeiro leilão de cativos em
Portugal à morte de Zumbi dos Palmares}. Vol.1. Rio de Janeiro: Globo
Livros, 2019.

O escritor Laurentino Gomes promete três volumes de \emph{Escravidão}. O
primeiro deles, que cobre a História da Escravidão no Brasil do século
XVI ao XVIII, foi publicado em 2019 -- e mesmo que não se refira
especificamente ao século em que viveu Machado de Assis, é referência
fundamental para refletir sobre as bases do racismo e da violência em
nossa sociedade, em prosa fluente.

\item \textsc{hooks}, bell. \textit{Ensinando pensamento crítico: sabedoria prática}.
Trad. Bhuvi Libanio. São Paulo: Elefante, 2020.

Como as atividades sugeridas neste manual sempre envolvem debates e
trabalhos em grupo, julgamos adequado sugerir essa impactante obra da
militante norte-americana bell hooks, pseudônimo de Gloria Jean Watkins,
em homenagem à bisavó, Bell Hooks, ``uma mulher de língua afiada, que
falava o que vinha à cabeça, que não tinha medo de erguer a voz''. Nessa
obra, como está evidente no título, a autora ensina, entre outras
coisas, a estimular o pensamento e a atitude crítica em sala de aula,
por meio da partilha de experiências pessoais, que resulta no que ela
chama de ``comunidade de aprendizagem''.

\item \textsc{nascimento}, Abdias. \textit{O Genocídio do Negro Brasileiro: processo de
um racismo mascarado}. São Paulo: Perspectiva, 2016.

Abdias Nascimento é um dos mais importantes pesquisadores e ativistas
pelos direitos dos afrodescendentes no Brasil. Esse histórico ensaio
apresenta uma extensa análise de mitos e realidades a respeito do
racismo no Brasil, especialmente o que o autor chama de ``estratégias de
genocídio'' dos negros. Leitura indispensável para a compreensão do
racismo em nosso país.

\item \textsc{parrine}, Raquel. ``Aspectos de Teoria do Conto em Machado de Assis''.
\textit{Eutomia: Revista Online de Literatura e Linguística.} v.1, n.3,
2009. Disponível em:

\href{https://periodicos.ufpe.br/revistas/EUTOMIA/article/view/1902/1489}{periodicos.ufpe.br}.

Análise da consolidação do conto, no Brasil, por meio da obra de Machado
de Assis. Leitura útil para conhecer a teoria do conto de Edgar Allan
Poe, na qual a autora fundamenta sua análise dos textos machadianos, e
para compreender-lhes a arquitetura literária. Atenção à análise da
matriz jornalística do conto do século XIX, do hibridismo de muitos
contos de Machado de Assis e do interesse do autor pelo conto
fantástico.

\item \textsc{severiano}, Jairo. \textit{Uma história da música popular brasileira: das
origens à Modernidade}. São Paulo: Editora 34, 2008.

Nessa obra, o projeto do autor é contar os mais de duzentos anos de
história da música popular brasileira. Para a análise de ``Um Homem
Célebre'', interessam os capítulos de ``formação'', que cobrem o período
que vai de 1770 a 1928.

\end{itemize}



\end{document}
