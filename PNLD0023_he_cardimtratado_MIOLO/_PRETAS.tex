

\textbf{Fernão Cardim} (Viana de Alvito, Alentejo, 1548?--Abrantes, Salvador, 1625) foi uma das mais eminentes figuras da 
Companhia de Jesus, onde ingressou em 1556. Permaneceu cerca de 42 anos no território brasílico, onde chegou em 1583, como secretário do padre visitador Cristóvão de Gouveia. Reitor dos Colégios da Bahia e do Rio de Janeiro, 
foi eleito na congregação provincial para Procurador da Província do Brasil, em Roma, onde viveu entre 1598 e 1601. 
Na viagem de regresso ao Brasil, foi 
capturado por corsários ingleses, despojado dos seus textos e levado para Inglaterra onde esteve prisioneiro 
durante cerca de dois anos. Já no Brasil foi nomeado Provincial da Companhia de Jesus, cargo que 
desempenhou entre 1604 e 1609. 

\textbf{Tratado da terra e gente do Brasil} foram escritos, entre 1583 e 1601, 
pelo Padre Jesuíta Fernão Cardim, nos anos seguintes à sua chegada ao Brasil, 
quando desempenhou o cargo de secretário do padre visitador Cristóvão de Gouveia. O livro manteve"-se inédito 
em língua portuguesa até 1847, embora tenha sido publicado parcialmente em inglês, em 1625, com atribuição
a outro autor. 
Os \textit{Tratados} de Cardim permitem"-nos ter um conhecimento da terra brasileira de Quinhentos e dos povos ameríndios, assim como do papel dos Jesuítas nessa região e dos hábitos da vida nos engenhos. 
A obra é de interesse não apenas pela descrição da terra e do clima, já que aborda ainda a fauna e a flora, e os seus 
habitantes, procurando salientar a importância que esta terra poderia vir a ter no futuro, pois já se evidenciava como 
sendo ``outro Portugal''. 

\textbf{Ana Maria de Azevedo} é licenciada em História pela Faculdade de Letras da Universidade de Lisboa, 
onde obteve o grau de mestre em História e Cultura do Brasil, com dissertação sobre a vida e a obra 
do Padre Fernão Cardim, e onde lecionou durante 15 anos, como convidada, a disciplina de História do Brasil. 
%Jorge: Foi preciso encurtar!
%As suas principais áreas de investigação têm-se centrado na história da presença portuguesa no Brasil 
%e no encontro de povos e culturas nos séculos \textsc{xvi} e \textsc{xvii}. 
%Proferiu conferências e comunicações sobre temas da História do Brasil e da Didática da História, 
%em universidades e instituições culturais portuguesas, brasileiras, timorenses, norte-americanas, 
%espanholas, alemãs, entre outras. 
Publicou uma edição crítica à \textit{Carta de Pêro Vaz de Caminha 
a El"-Rei D. Manuel sobre o Achamento do Brasil} (2000); \textit{Portugal e as Novas Fronteiras Quinhentistas} 
(2001), entre outros livros e artigos. Foi Comissária Científica do site temático \textit{Viagem dos Portugueses}, 
dedicado ao Brasil, da Biblioteca Nacional de Portugal (2001). Realizou a revisão científica do volume ``A América Pré-Colombiana'', da \textit{História da Humanidade}, ed. portuguesa do Círculo de Leitores, 2007.


