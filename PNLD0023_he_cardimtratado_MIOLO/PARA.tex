\part{\textsc{paratexto}} 

\hyphenation{An-chie-ta}
\chapter[Fernão Cardim e o Brasil quinhentista]{Fernão Cardim e o Brasil quinhentista}
\hedramarkboth{Fernão Cardim e o Brasil\ldots{}}{Ana Maria de Azevedo}

\begin{flushright}
\textsc{ana maria de azevedo}
\end{flushright}

\section{Sobre o autor}

É pouco clara a data do nascimento de Fernão Cardim, considerando a
maior parte dos historiadores que terá nascido ao redor de 
1548--1549, em Viana de Alvito, no
Alentejo, pertencente ao Arcebispado de Évora, numa família de ``estirpe
antiga e importante''. Era filho de Gaspar
Clemente e de D.\,Inês Cardim. Teve três irmãos, sendo o mais velho o
Doutor Jorge Cardim Fróis, que ocupou cargos importantes na
administração da Justiça, chegando mesmo a ser Desembargador dos
Agravos da Casa da Suplicação, na corte de Lisboa. Os outros dois
irmãos, Lourenço Cardim e Diogo Fróis, também foram membros da
Companhia de Jesus, tendo o primeiro morrido num ataque à armada onde
se encontrava a caminho do Brasil e a quem Cardim se refere na sua
\textit{Narrativa epistolar}, e tendo o segundo sido lente de Teologia
Moral, no Colégio de Évora.

Cardim ingressou na Companhia em 1556 e veio a falecer no
Brasil, na aldeia de Abrantes, nos subúrbios de Salvador, em 1625,
depois de uma vida intensa de permanência nesse território,\footnote{ Vide 
o necrológio escrito pelo Pe. Antônio Vieira, resumindo"-lhe a
vida como a de ``um varão verdadeiramente religioso e de vida
inculpável, mui afável e benigno, e em especial com seus súbditos'', in
\textit{Annua da provincia do Brasil dos annos de 1624 e 1625}, 
publicada nos \textit{Anais da Biblioteca Nacional do Rio de Janeiro}, 
1827, \textsc{xix}, p. 187.} durante 42 anos, interrompidos
apenas por uma viagem como Procurador da Província de Jesus a Roma,
entre 1598 a 1601, e durante o período em que esteve prisioneiro na
Inglaterra, entre 1601 e 1603. 

 Da vida do Padre Fernão Cardim, antes da ida para o Brasil, sabe"-se
que fez estudos em Artes e Teologia, no Colégio de Évora. Foi depois
ministro do mesmo Colégio e adjunto do mestre de noviços em Coimbra. Já
com este cargo e professo de quatro votos, Cardim foi nomeado, em 1582,
secretário do visitador Cristóvão de Gouveia,\footnote{\textls[-20]{Cristóvão
de Gouveia nasceu a 8 de janeiro de 1542, na cidade do Porto, e entrou
na Companhia de Jesus, em Coimbra, a 10 de janeiro de 1556.}} 
seguindo para o Brasil a 5 de março do ano seguinte, na nau
\textit{Chagas de São Francisco}, onde ia também o governador Manuel
Teles Barreto,\footnote{ O governador Manuel Teles Barreto
desempenhou esse cargo entre 1583 e 1587, sendo o seu período o mais
difícil das relações institucionais com a Companhia de Jesus nesse
território. Foi o primeiro governador a ser nomeado durante o período
filipino, vindo a ter problemas com os padres jesuítas e, nomeadamente,
com o visitador Cristóvão de Gouveia de quem Cardim era secretário.} 
funcionários régios e vários padres jesuítas.\footnote{ Cf.
Joaquim Veríssimo Serrão, \textit{Do Brasil Filipino ao Brasil de 1640}. 
São Paulo: Companhia Editora Nacional, 1968, p. 27.} Depois de cerca
de 66 dias de viagem, com dez de paragem na ilha da
Madeira, chegaram à Baía de Todos os Santos, a 9 de maio de 1583. Entre
essa data e 1590, podemos acompanhar Cardim por terras das capitanias
de Bahia, Ilhéus, Porto Seguro, Pernambuco, Espírito Santo, Rio de
Janeiro e São Vicente, mais tarde São Paulo, descrevendo as paisagens e
os fatos que observava e dando"-nos uma ideia precisa e interessante do
Brasil dessa época, assim como as ordens proferidas pelo visitador para
bom funcionamento dos colégios e das residências existentes naquelas partes.

Além de reitor dos colégios da Bahia e Rio de Janeiro,\footnote{ O Pe. 
Fernão Cardim desempenhou o cargo de reitor dos colégios da
Bahia, entre 1590 e 1593, e posteriormente a partir de 1607 até a data
da sua morte, em 1625, e do colégio de São Sebastião do Rio de Janeiro,
entre 1596 e 1598.} o Padre Fernão Cardim foi eleito na congregação
provincial Procurador da Província do Brasil, em Roma, em 1598,
onde permaneceu cerca de três anos.\footnote{ Sobre a sua presença
em Roma, entre 1598 e 1601, pouco conseguimos descobrir, o mesmo
acontecendo em relação à sua presença em Bruxelas, antes de 1603.}
Quando regressava ao Brasil, Cardim e os seus companheiros de viagem
foram capturados, ainda na barra de Lisboa, por corsários ingleses,
tendo sido levado para a Inglaterra, onde esteve prisioneiro durante cerca
de três anos, enclausurado na prisão de Gatehouse, que Cardim denominou
de \textit{Gatus}, em Londres. Temos conhecimento deste período através
das cartas escritas do cárcere, por Cardim, em que procura diligenciar
a sua libertação, assim como a dos seus companheiros, e em que pretende
acima de tudo que lhe fossem devolvidos os seus manuscritos, que
considera como a maior riqueza da sua propriedade,
objetivo que não conseguiu alcançar pois nunca mais teve contato com
os mesmos. Enquanto esteve enclausurado, entre dezembro de 1601 e março
de 1603, Cardim negociou a sua libertação, através de pedidos de
clemência dirigidos à rainha Elizabeth \textsc{i}, ou a \textit{Sir} Robert Cecil, conde de
Salisbury, figura de destaque da corte e, ainda, através de
diligências, por si feitas, para a libertação de cavaleiros ingleses
prisioneiros na Espanha e em Flandres, que acabaram por ter
êxito.\footnote{ Sobre a prisão do Pe. Fernão Cardim em Londres e
as diligências desencadeadas para a sua libertação, veja"-se o artigo de
W. H. Grattan Flood, ``Portuguese Jesuits in England in Penal Times'', in
\textit{The Month}, nº143, 1924, pp. 157--159.}

Após a sua libertação da Inglaterra e ao regressar ao Brasil, o Padre
Fernão Cardim foi nomeado Provincial do Brasil da Companhia de Jesus,
cargo que exerceu até 1609. Entre as medidas importantes tomadas por
Cardim no desempenho dessas funções, destaca"-se a encomenda da
hagiografia, do Padre José de Anchieta,\footnote{ Vide
``Carta do Padre Provincial Fernão Cardim, para o nosso
Reverendo Padre Geral Aquaviva'', escrita da Bahia, a 8 de maio de
1606, in \textit{Annaes da Bibliotheca Nacional do Rio de Janeiro}, 
1907, vol. \textsc{xxix}, Rio de Janeiro, Officinas de Artes Graphicas da
Biblioteca Nacional, 1909, pp. 183--184. O texto desta carta foi
publicado na íntegra na nossa dissertação já citada, no anexo
documental \textit{doc. 10}, p. 16.} e a preocupação em organizar
missões de conversão dos indígenas, nomeadamente os índios
Carijós,\footnote{ Os índios \textit{Carijós}, também denominados
de \textit{Guaranis}, pertenciam à família tupi"-guarani. Viviam na
região da lagoa dos Patos e no litoral do Paraná e Santa Catarina,
ocupando uma ampla faixa que compreendia parte do litoral desses dois
estados e o sertão até Assunção, no Paraguai.} Guaranis, na região da
Lagoa dos Patos, no sul do território.

Quanto à sua formação, sabemos que Fernão Cardim, após a primeira
fase de noviciado, continuou os seus estudos, que compreendiam cursos
de latim, com desenvolvimento de conhecimentos em gramática,
humanidades e retórica, grego, filosofia e teologia. Realizou, assim, a
formação em humanidades, artes liberais e teologia, estando, por
conseguinte, preparado para a principal missão dos jesuítas, que era o
ensino e a conversão dos índios à fé cristã. A sua formação jurídica
manifesta"-se em particular nas questões subjacentes ao Brasil
nascente e aos problemas inerentes a esse nascimento, sobretudo no que
concerne aos povos ameríndios. Esta sua formação é notória nas cartas,
nas informações e nos artigos que escreveu ao rei,
apresentando"-lhe as questões que este deveria ter em atenção sobre o
Brasil da época.
A par do Direito, Cardim mostra conhecimentos e
preocupações do âmbito da teologia moral, instrumento absolutamente
necessário para a sua formação e missão perante os povos ameríndios e
os portugueses, com quem contatou nas terras brasílicas.

 Da sua preparação cultural conhecemos pouco, mas o estudo dos seus
textos permitem"-nos sentir a influência de alguns teólogos e filósofos,
como Santo Agostinho e São Tomás de Aquino, além, evidentemente, da
grande marca das Epístolas e dos Evangelhos. Como um homem culto da
época, Cardim tinha uma formação privilegiada, e como um humanista
quinhentista era conhecedor dos autores clássicos. Plínio devia mesmo
constar das suas leituras, pois aparecem várias vezes termos que fazem
a relação entre a obra clássica \textit{Naturalis Historiae Libri} e
a obra cardiniana. Relevante é ainda a influência das
\textit{Etimologias} de Santo Isidoro de Sevilha. A própria organização
dos assuntos ao longo dos seus textos incluídos nos \textit{Tratados da
terra e da gente do Brasil} mostra a preocupação em apresentá"-los
divididos, no caso dos animais não voadores quadrúpedes, em selvagens,
domésticos e exóticos, além do volucrário, dedicado às aves, o
ictuário, dedicado aos peixes e, por fim, o ofidário, para os répteis.
Era o modelo seguido pelos bestiários dos escritores antigos, a par do
herbário e do lapidário. 

 Cardim mostra ainda ter conhecimento dos seus
contemporâneos, como Nicolau Monardes, médico e naturalista espanhol
que escreveu várias obras sobre os produtos oriundos do continente
americano e que criou, inclusivamente, um Museu de História Natural em
Sevilha, em 1574. É ainda evidente a influência de textos em que se
forneciam informações sobre os produtos do Oriente, como os de João de
Barros, Garcia da Orta, Cristóvão da Costa e Duarte Barbosa.
Apercebemo"-nos, ao longo da obra cardiniana, que o seu autor não
recolheu informações de relance ou ao sabor do acaso, mas que, pelo
contrário, procurou mais dados que lhe permitissem dar a conhecer as
plantas e os animais do espaço brasílico, comparando"-os com os já
conhecidos, quer fossem do Oriente, de Portugal ou muitas vezes da Espanha. 

Cardim mostra também conhecimento das línguas ameríndias, sobretudo
do Tupi"-Guarani,\footnote{\textls[-10]{ O Tupi foi a língua usada pelos
jesuítas em suas catequeses desde o Maranhão até São Vicente. Não era
língua própria de uma tribo, mas uma ``língua geral'', resultante de uma
uniformização léxica racional de vários dialetos, que veio a ser
fixada pela gramática do Padre José de Anchieta e pelo vocabulário
jesuítico. O Guarani é um ramo do Tupi"-Guarani e foi falado
desde São Vicente até o Paraguai, onde ainda o é hoje, ainda que muito
influenciado pelo castelhano. Cf. Francisco da Silveira Bueno,
\textit{Vocabulário Tupi"-Guarani/Português}. São Paulo: Brasilivros, 
5ª ed., 1987.}} pois ao longo dos seus textos inclui muitas 
vezes vocábulos e até mesmo frases nesse idioma, em que procura
fornecer a tradução dos mesmos. Mas surgem também nomes em
abanheenga\footnote{ Entenda"-se por abanheenga ou
abanheém a língua tupi antiga.} e em quechua ou 
quíchua.\footnote{Língua falada pelos povos quíchuas, que na época 
da colonização espanhola habitavam a região que se estende ao
norte e oeste de Cuzc. Cardim possivelmente toma conhecimento da língua quechua na obra de Fr. Domingo de Santo
Tomás, \textit{Lexicon, o Vocabulario de la lengua general del Peru}.}


\section{Sobre a obra}

A obra de padre Fernão é essencialmente formada
por cartas destinadas ao Provincial da Companhia, em Lisboa, e outros
escritos, nos quais dá conta do que observa e das suas opiniões sobre a
terra e as gentes. Informante, naturalista, etnógrafo, botânico,
geólogo, zoólogo e cronista, Fernão Cardim foi, ainda, a testemunha que
viveu os acontecimentos que descreve de uma forma muito cativante,
contribuindo para o estudo da história social da fase inicial da
colonização do ``seu Brasil''. Um ``outro Portugal'', na sua opinião.

Os seus escritos mantiveram"-se desconhecidos durante séculos, só
vindo, em parte, a ser divulgados em língua portuguesa e atribuídos a
este jesuíta no século \textsc{xix}, mais precisamente em 1847. Na sua
totalidade os textos cardinianos continuam sem ser editados até ao
presente em Portugal. Durante longo tempo permaneceram inéditos, tendo
alguns sido publicados em inglês, mas atribuídos a outro autor.
Situação interessante e que mostra bem as vicissitudes por que, em
Quinhentos, passavam, muitas vezes, os manuscritos.

Dos textos de Cardim, o que foi divulgado em primeiro lugar em língua
portuguesa foi a \textit{Narrativa epistolar de uma viagem e missão
jesuítica pela Bahia, Ilheos, Porto Seguro, Pernambuco, Espírito Santo,
Rio de Janeiro, S.~Vicente (S.~Paulo), etc, desde o anno de 1583 ao de
1590, indo por visitador o Padre Christóvão de Gouvêa}, trabalho
editado em 1847 por Francisco Adolfo Varnhagen.
Trata"-se de informações recolhidas por Cardim, quando desempenhava o
cargo de secretário do visitador Cristóvão de Gouveia e compiladas em
duas cartas, dirigidas ao provincial da Assistência de Portugal. 

Rica de informações, essa \textit{Narrativa epistolar} voltaria a ser publicada no Rio de Janeiro, em 1925, data do centenário da morte do autor, incluída numa obra
dedicada a Cardim, em conjunto com os outros textos deste jesuíta,
intitulada \textit{Tratados da terra e gente do Brasil}, com Introdução
e notas de Baptista Caetano, Capistrano de Abreu e Rodolfo Garcia.
Nesta obra foram incluídos os restantes textos que Fernão Cardim
escreveu sobre a terra e gentes do Brasil, com os títulos \textit{Do
princípio e origem dos índios do Brasil e de seus costumes, adoração e
cerimônias} e \textit{Do clima e terra do Brasil e de algumas cousas
notáveis que se acham assim na terra como no mar}, que tinham sido
publicados pela primeira vez em inglês, no ano de 1625, em Londres, na
coleção \textit{Purchas his Pilgrimes}, sob o título \textit{A
Treatise of Brasil written by a portuguese wich had long live there.}\footnote{ Cf. 
Samuel Purchas, \textit{Purchas his Pilgrimes}, 
London, 1625, ``The Seaventh Booke'' -- ``Voyages to and about the
Southern America, with many Marine Observations and Discourses of Those
Seas and Lands by English"-men and others'', onde estão publicados os
textos de Fernão Cardim, vol. \textsc{iv}, pp. 1289--1320.} 

Samuel Purchas, que adquirira estes manuscritos por bom preço, depois
dos mesmos terem sido confiscados ao padre Fernão Cardim, após ter sido
capturado por corsários ingleses e expropriado dos seus bens,
considerou"-os de grande qualidade e os mais completos que já tinha
visto sobre o Brasil, atribuindo"-os a um ``frade ou jesuíta português'',
de quem o corsário inglês Francis Cook, de Dartmouth, se tinha
apoderado, em uma viagem ao Brasil, em 1601, e que os tinha vendido por
vinte xelins a um certo mestre Hackett. Apresenta assim esses textos:

\begin{hedraquote}
Leitor, apresento aqui o mais exato Tratado do Brasil que
já vi escrito por alguém, especialmente na História das múltiplas e
diversificadas nações e costumes dos homens; assim como na história
natural dos animais, serpentes, aves, peixes, árvores, plantas, com
espécies de assinalável raridade dessas regiões. Foi escrito (segundo
parece) por um padre (ou jesuíta) português que viveu trinta anos
nessas partes.\footnote{ Cf. Samuel Purchas, \textit{op. cit.}, 
ed. de Glasgow, 1906, pp. 417--418. Tradução da autora.}
\end{hedraquote}

No entanto, como nas últimas páginas dos manuscritos viessem
incluídas umas receitas medicinais assinadas pelo irmão Manuel
Tristão,\footnote{ Manuel Tristão (1546--1631?), natural dos Açores,
das ilhas do Faial ou Santa Maria. Entrou na Companhia de Jesus já com
22 anos de idade, a 19 de maio de 1568, onde foi enfermeiro durante
muitos anos no Colégio dessa cidade; passou a Pernambuco e residia, em
1606, no Colégio de Olinda e, em 1607, na aldeia de Sto. André de
Goiana. Entre 1613 e 1617 esteve na aldeia de N. Sra. da Escada. Vivia já
velho no Colégio de Olinda, em 1621. Deixou uma breve 
\textit{Coleção de receitas medicinais}, conhecida de Samuel Purchas, em 1625. Cf.
Samuel Purchas, \textit{op. cit.}, p. 417.} contemporâneo de Cardim,
enfermeiro do Colégio da Bahia, que também esteve no Colégio de Olinda
e em várias aldeias, Purchas considerou"-o como autor dos
\textit{Tratados}, afirmando em nota lateral: ``Encontrei 
no fim do livro algumas receitas medicinais e o nome subscrito
Ir. Manuel Tristão, enfermeiro do Colégio da Bahia: o qual imagino
tenha sido o autor deste Tratado.''\footnote{ Cf. Samuel
Purchas, \textit{op. cit.}, p. 417 (nota lateral). Tradução da autora.}

Coube a Capistrano de Abreu o mérito de reivindicar para o
Padre Fernão Cardim a autoria dos referidos manuscritos, publicando, em
1881, o tratado referente aos índios.\footnote{ Fernão Cardim,
\textit{Do princípio e origem dos índios do Brasil}, Rio de Janeiro.
Typographia da ``Gazeta de Notícias'', 1881, 121 páginas e na
\textit{Revista do Instituto Histórico e Geográfico Brasileiro}, 57,
1ª p., em 1894, pp. 183--212.} Nesta edição, este especialista cardiniano
demonstra como os manuscritos devem ser atribuídos a Cardim, não apenas
pela coincidência dos textos terem sido roubados em 1601 pelo mesmo
corsário inglês, que também na mesma altura aprisionara a nau onde
viajava o padre Fernão Cardim e o espoliara dos seus manuscritos, mas
também porque, pela colação desses autógrafos com o texto da
\textit{Narrativa epistolar}, já antes atribuída a Cardim, é bem
patente o mesmo estilo de escrita e até alguma semelhança de assunto.
Demonstra ainda que pelo exame do texto se verifica que o opúsculo
foi escrito em 1584, data em que Cardim já se encontrava no Brasil,
tendo chegado a 9 de maio de 1583, como se sabe pela \textit{Narrativa epistolar}, 
o que pode esclarecer a autoria do referido manuscrito. Até
porque a primeira carta deste texto é de 16 de outubro, o que aproxima
muito a data de redação das duas versões e a conformidade de ideias e
de forma, ainda que os objetivos e os destinatários da mesma sejam diferentes.
O próprio confronto da assinatura de Cardim, desta primeira carta, com as das 
cartas enviadas do cárcere londrino permite"-nos identificá"-lo como o autor desse texto.   

O mesmo historiador brasileiro considera também que um enfermeiro,
apesar dos seus conhecimentos específicos, não era um erudito e que,
quanto às receitas medicinais que aparecem incluídas no final dos
manuscritos, poderiam ter sido ensinadas pelo dito enfermeiro ao
missionário.\footnote{ Cf. Rodolfo Garcia, ``Introdução'', in
Fernão Cardim, \textit{op. cit.}, pp. 11--22.} O próprio Fernão Cardim
mostra ao longo dos seus textos ter conhecimentos de algumas receitas e
hábitos medicinais, preocupando"-se em descrever em pormenor as doenças
e respectivos tratamentos que foram ministrados ao longo da viagem,
assim como as várias ervas que os índios utilizavam para as mezinhas e
as que eram medicinais, comparando"-as com as existentes ou conhecidas em Portugal. 

No próprio ano de 1881, já parte do texto do tratado \textit{Do clima
e terra do Brasil} tinha começado a ser publicado pelo Dr.\,Fernando
Mendes de Almeida, na \textit{Revista Mensal da Secção da Sociedade de
Geographia de Lisboa no Rio de Janeiro}, mas apenas os dois primeiros
capítulos e sem o nome do autor.\footnote{ O texto está publicado no
tomo \textsc{i}, números 1 e 2 .} No terceiro número dessa revista foi
incluída uma carta de Capistrano de Abreu, em que explicava a origem do
manuscrito que servia para impressão e atribuindo a sua autoria a
Fernão Cardim. Mais tarde, em 1885, no tomo \textsc{iii} da mesma
publicação, o referido historiador publicou a versão integral
do referido tratado e um estudo biobibliográfico sobre o seu autor. 

No que concerne ao segundo tratado, dedicado aos índios,
também publicado em 1625 por Purchas, identificada que estava a
autoria do primeiro texto, e uma vez que o estilo é o mesmo e os
conhecimentos do Brasil apresentados nos dois textos são idênticos,
Capistrano de Abreu considerou que a autoria é do Padre Fernão Cardim.

Estes dois textos de Cardim e as duas cartas que formavam a
\textit{Narrativa epistolar} foram compilados, finalmente, numa obra
única, com o título de \textit{Tratados da terra e gente do Brasil}, em
1925, e publicados, no Rio de Janeiro, na comemoração do centenário da
sua morte.\footnote{ Cf. Fernão Cardim, \textit{Tratados da terra
e gente do Brasil}, Rio de Janeiro, Editores J. Leite e Cia., 1925.} O
título de conjunto foi justificado por Afrânio Peixoto, na altura
presidente da Academia Brasileira de Letras, da seguinte forma:

\begin{hedraquote}
Pela primeira vez reúnem"-se, num só tomo, com o seguimento
que parece lógico, o aparelho de notas elucidativas e o título a que
têm direito, os tratados do Padre Fernão Cardim sobre o Brasil. [\ldots{}]
Portanto, aos três tratados do Padre Fernão Cardim parece exato o
título que lhes damos: \textit{Tratados da terra e gente do Brasil} 
que são agora não só a homenagem a um grande missionário que
amou, observou, sofreu e tratou o Brasil primitivo, com contribuição do
nosso reconhecimento a essas missões jesuíticas, que educaram os
primeiros brasileiros, e para os de todos os tempos deixaram memórias
passadas nos seus escritos, cartas e narrativas.\footnote{ Cf. Afrânio Peixoto, 
nota introdutória à obra de Fernão Cardim, \textit{op. cit.}, pp. 11--12.} 
\end{hedraquote}

 Os textos de Fernão Cardim foram publicados, posteriormente,
em 1933, com as mesmas notas e textos introdutórios de Baptista
Caetano, Capistrano de Abreu e Rodolfo Garcia, todos importantes
historiadores brasileiros e que se dedicaram ao estudo da obra
cardiniana.\footnote{ Cf. Fernão Cardim, \textit{Tratados da terra
e gente do Brasil}. Rio de Janeiro/São Paulo: J. Leite, Companhia
Editora Nacional, 1933.} Mais recentemente, em 1980, foi publicada uma nova
edição destes tratados cardinianos, mas que mantém o mesmo
texto, notas e estudos introdutórios da edição de 1925, uma vez que se
trata de um \textit{fac"-símile}.\footnote{ Cf. Fernão Cardim, \textit{Tratados
da terra e gente do Brasil}. Belo Horizonte/São Paulo: Itatiaia/Ed\textsc{usp}, 1980.} 
Em Portugal os textos de Cardim nunca tinham sido publicados até então.

\section{Sobre o gênero}

O texto de Fernão Cardim pode ser classificado no gênero do relato histórico que, em linhas fundamentais, define"-se como a narrativa em que um sujeito, inscrito em um determinado tempo histórico, debruça"-se sobre fatos, descrições e interpretações desse momento histórico no qual vive. Para o historiador francês Paul Veyne, o relato histórico segue uma forma similar à forma tradicional de escrever história, seguindo um \textit{continuum} espaço"-temporal.

Apesar dessa relativa unidade, o relato, considerado como uma forma de fazer história,
é parcial e subjetivo, pois não consegue apreender a globalidade dos acontecimentos, apenas
aquilo que está ao alcance do narrador e, mesmo isso, não de uma forma pura, mas filtrado pela sua subjetividade e pelos objetivos de seu relato.
Para Veyne, estaríamos assim quase próximos do romance:

\begin{quote}
A história é uma narrativa de eventos: todo o resto resulta disso. Já que é, de fato, uma narrativa, ela não faz reviver esses eventos, assim como tampouco o faz o romance; o vivido, tal como ressai das mãos do historiador, não é o dos atores; é uma narração,
o que permite evitar alguns falsos problemas. Como o romance, a
história seleciona, simplifica, organiza, faz com que um século
caiba numa página.\footnote{\textsc{veyne}, Paul. \textit{Como se escreve a história}. Brasília: Editora Universidade de Brasília, 1999, p.\,18.}
\end{quote}

Seguindo nessa linha de pensamento, podemos observar, com o historiador francês Marc Bloch, que o relato é apenas um ``vestígio'' da história, um pequeno pedaço do factual que, pela pena de um narrador, pôde"-se cristalizar no tempo e ser transmitido a gerações posteriores, sendo apenas uma das infinitas possibilidades de apreensão e compreensão de determinados fenômenos:

\begin{quote}
Quer se trate das ossadas
emparedadas nas muralhas da Síria, de uma palavra cuja forma ou emprego revele um
costume, de um relato escrito pela testemunha de uma cena antiga [ou recente], o que
entendemos efetivamente por documentos senão um ``vestígio'' quer dizer, a marca,
perceptível aos sentidos, deixada por um fenômeno em si mesmo impossível de captar?\footnote{\textsc{bloch}, Marc. \textit{Apologia da história}. Rio de Janeiro: Zahar, 2002, p.\,73.}
\end{quote}

O texto de Cardim se inscreve nesse gênero pois, ao intentar relatar à Coroa portuguesa a situação de sua recém"-descoberta colônia, ele descreveu e detalhou as paisagens, animais, frutos, árvores, peixes, pássaros, ervas, plantas, alimentos, costumes indígenas etc. da terra brasileira. Por meio de seus relatos, conseguimos vislumbrar, pela ótica de um padre quinhentista, o que era o Brasil naquela época.

É com significativa admiração que Fernão Cardim
apresenta, por exemplo, a cidade de São Sebastião do Rio de Janeiro, aonde chegara nos
finais do ano de 1584:

\begin{hedraquote}
A cidade está situada em um monte de boa vista para o mar,
e dentro da barra tem uma baía que bem parece que a pintou o supremo
pintor e arquiteto do mundo novo Deus Nosso Senhor, e assim é cousa
formosíssima e a mais aprazível que há em todo o Brasil, nem lhe chega
a vista do Mondego e Tejo\ldots{}\footnote{ Cf. Fernão Cardim,
\textit{op. cit.}, pp. 267--268.}
\end{hedraquote}
 
Homem sensível que se deixou seduzir pela
formosura natural dessa cidade, ainda hoje considerada como uma das
mais belas do mundo. Não era apenas a visão edênica do Novo Mundo,
usual por parte dos escritores quinhentistas, mas também uma
apreciação mais apurada do território brasileiro e das suas
potencialidades. De fato, o cenário do continente americano
oferecia"-se aos primeiros descobridores e mais tarde aos
colonizadores: ``[\ldots{}] armado pela expectativa de um \textit{non plus ultra} de maravilha, 
encantamento e bem"-aventurança, sempre a inundá"-lo em sua luz mágica [\ldots{}]'',
como afirma Sérgio Buarque de Holanda, naquela que
é considerada como um clássico da historiografia brasileira, a
\textit{Visão do Paraíso.}\footnote{ Cf. Sérgio Buarque da Holanda,
\textit{Visão do Paraíso}. Rio de Janeiro: José Olympio, 1959, p. 268.}

As descrições dos autores quinhentistas e seiscentistas
correspondiam, quase, ao tradicional tema dos hortos de delícias.
Trata"-se nela da bondade dos ares, da sanidade da terra, da feliz
temperança do clima, da abundância e variedade do mantimento,
principalmente das frutas, da amenidade e beleza da vegetação,
sugerindo a imagem dos formosos jardins e hortos do Éden.

A nostalgia do jardim do Éden e a convicção de que se aproximavam os
tempos escatológicos, a par da vontade de estender a religião cristã a
terras novas e o desejo de encontrar em abundância o ``tão desejado
ouro e pedras preciosas'' e os outros produtos raros, conjugaram"-se
para considerar aquelas terras tão maravilhosas como o verdadeiro
Paraíso Terrestre. A cultura na qual participavam e os sonhos que ela
veiculava levaram os viajantes do Quinhentos a reencontrar nas regiões
insólitas da América as características das terras abençoadas que
assediavam desde a Antiguidade a imaginação dos ocidentais.\footnote{ Cf. Jean Delumeau, 
\textit{Uma História do Paraíso}. Lisboa: Terramar, 1994, p. 134.} 

 O próprio Cristóvão Colombo estava convencido de que as 
Índias se encontravam na vizinhança do Paraíso Terrestre, pois estava
fortemente impressionado pela beleza da Hispaniola (São
Domingos/Haiti), que considerou ser esta ilha única no mundo, porque se
achava coberta de toda a espécie de árvores que pareciam tocar o céu e
não perdiam nunca as suas folhas.\footnote{ Cf. Cristóvão
Colombo. \textit{\OE uvres}. Paris: Galimard, 1961, p. 181.} A
propósito de uma outra paisagem insular, a do Cabo Hermoso, escreveu
Colombo: ``Ao chegar à altura desse cabo, vem da terra um
odor de flores e de árvores tão bom e tão suave, que era a coisa mais
doce do mundo.''\footnote{ \textit{Idem, ibidem}, p. 181.} 

Idêntica sensação transmite Fernão Cardim nos seus textos ao
descrever as paisagens das terras brasílicas. Tocado pelas cores,
odores, sabores e paisagens tão diversificados, descreve"-os de uma
forma minuciosa, elogiando as suas qualidades e recorrendo,
frequentemente, a analogias para possibilitar ao leitor uma melhor
visualização daquilo que ele tinha conseguido observar, sentir, cheirar
ou mesmo saborear. Mas o que não nos deixa dúvida é que este autor
procurava também, através dos seus textos, dar a conhecer aquelas
terras com interesse para que fossem aproveitadas, já que tinham um clima
mais ameno, onde havia menos doenças e as terras eram mais férteis e,
em muitos aspectos, semelhantes a Portugal. Mais de uma vez não deixa
de afirmar nos seus textos que:

\begin{hedraquote} 
Este Brasil é outro Portugal, e não falando no clima que é
muito mais temperado, e sadio, sem calmas grandes, nem frios, e donde
os homens vivem muito com poucas doenças, como de cólica, fígado,
cabeça, peitos, sarna, nem outras enfermidades de Portugal; nem falando
do mar que tem muito pescado, e sadio; nem das cousas da terra que Deus
cá deu a esta nação.\footnote{ Cf. Fernão Cardim, \textit{op. cit.}, p. 157.} 
\end{hedraquote}

 A preocupação pela descrição rigorosa do território
brasileiro permite"-nos também um conhecimento precioso do cotidiano
nos engenhos e nas fazendas, onde Cardim pernoitou, quando da sua
viagem como secretário do visitador. São informações de significativo
valor, pois além de mostrarem o papel dos jesuítas nessa região,
apresentam"-nos alguns dos hábitos de vida nos engenhos.




