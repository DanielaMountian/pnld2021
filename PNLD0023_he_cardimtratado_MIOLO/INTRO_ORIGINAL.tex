\hyphenation{An-chie-ta}
\chapter[Introdução, \emph{por Ana Maria de Azevedo}]{Introdução}
\hedramarkboth{Introdução}{Ana Maria de Azevedo}

\begin{flushright}
\textsc{ana maria de azevedo}
\end{flushright}

\noindent{}``He outro Portugal nasceu!'' Frase profética de um homem quinhentista que mostrou conhecer o
Brasil que acabara de nascer. Um autor que, apesar de não ter escrito
para o prelo, deixou"-nos um conjunto de textos muito importante para o
conhecimento da \textit{terra e gente} do Brasil quinhentista. O Padre
Fernão Cardim partiu para o Brasil em 1583, onde permaneceu cerca de
cinquenta anos, desempenhando vários cargos no seio da Companhia de
Jesus, de que era membro. Percorrendo o território brasileiro, ao mesmo
tempo que descrevia, como era hábito da época, aquilo que observava e
sentia sobre a terra e gente daquele território pelo qual foi
verdadeiramente atraído. 

Como bom observador, Cardim registra nos seus textos os hábitos e
costumes não só dos indígenas, como também dos próprios portugueses
com quem ia contatando. Entramos, assim, através dos seus escritos, em
contato com os povos ameríndios, ao mesmo tempo que conhecemos os
gostos, os pecados, as fortunas dos colonos e, até mesmo, as iguarias,
as vaidades das damas e o esplendor das festas. Botânico, etnólogo,
geólogo, zoólogo e cronista, Fernão Cardim aparece"-nos como um
naturalista, não esquecendo o seu papel de figura da Igreja, como
missionário e membro da Companhia de Jesus. Espírito do Quinhentos,
Cardim é, no entanto, mais do que um simples humanista ao descrever a
terra e as gentes do Brasil. A sua visão do território brasileiro e dos
seus habitantes é a de um homem que conseguia visualizar \textit{um
novo mundo}, para os europeus, onde coabitavam novos povos,
animais e plantas e onde um \textit{novo Portugal} nascia, utilizando a
sua própria expressão.

Nesta introdução, é nossa intenção dar a conhecer um pouco da vida
deste jesuíta que viveu quase meio século no Brasil, assim como da sua
obra, que deve ser integrada no conjunto dos textos renascentistas que
procuravam exatamente descrever as terras e gentes encontradas.
Procuramos, para tal, selecionar os aspectos mais relevantes da sua
biografia, que nos permitem apresentá"-lo como o autor dos
\textit{Tratados da terra e gentes do Brasil.} Esperamos que, com esta
divulgação do Padre Fernão Cardim, fique o interesse pela leitura dos
seus textos, já que não é usualmente muito citado nas obras portuguesas
referentes ao Brasil da época, apesar de o ser de forma significativa
na historiografia brasileira, onde é mesmo considerado como um dos
principais autores do período colonial. A ele se refere o historiador
José Honório Rodrigues nestes termos:

\begin{hedraquote} 
Fernão Cardim foi uma das maiores figuras da
Companhia de Jesus no Brasil e o conjunto da sua obra, tanto sobre o
clima e a terra, como sobre o princípio e origem dos índios, [\ldots{}]
tanto o poderia colocar na historiografia religiosa jesuítica, como na
historiografia geral, entre os primeiros cronistas.\footnote{ Cf. 
José Honório Rodrigues, \textit{História da História do Brasil}, 
1ª parte, ``Historiografia colonial'', São Paulo: Companhia Editora
Nacional, 1979, p. 265.}
\end{hedraquote}

Ou ainda a opinião de um dos seus maiores estudiosos, Rodolfo
Garcia, que afirma na Introdução à obra de Cardim:

\begin{hedraquote} 
Quantos estudem o passado brasileiro hão"-de reconhecer que
no acervo dos serviços prestados às nossas letras históricas existe em
aberto grande dívida de gratidão para com esse meritório jesuíta. De
fato, entre os que em fins do século \textsc{xvi} trataram das coisas do
Brasil, foi Fernão Cardim dos mais crédulos informantes, em depoimentos
admiráveis, que muita luz trouxeram à compreensão do fenômeno na
primeira colonização do país. Foi dos precursores da nossa História,
quando ainda o Brasil, por assim dizer, não tinha história. [\ldots{}] Seus
depoimentos são os de testemunha presencial, e valem ainda mais pela
espontaneidade e pela sinceridade com que singelamente os prestou.\footnote{ Cf. 
Rodolfo Garcia, ``Introdução'', in Fernão Cardim, \textit{Tratados da terra e 
gente do Brasil}. Belo Horizonte/São Paulo/Itatiaia: Ed\textsc{usp}, 1980, p. 13.} 
\end{hedraquote}

Testemunhos significativos que mostram o valor da obra do
jesuíta e a sua importância no contexto da produção portuguesa sobre o
Brasil do Quinhentos. Uma obra rica, que merecia uma maior divulgação em
Portugal, a par de outras, como a do seu companheiro José de Anchieta,
com quem contatou nas suas viagens pelo território brasileiro e cujos
textos se chegam a confundir com os de Cardim. De fato, o Padre Fernão
Cardim, pelas circunstâncias da sua vida, ficou entre este e outro
jesuíta, Antônio Vieira, formando uma tríade de apóstolos, missionários
que educaram os primeiros brasileiros e que defenderam os ameríndios da
escravidão. 

\section*{Padre Fernão Cardim, o homem}

É pouco clara a data do nascimento de Fernão Cardim considerando a
maior parte dos historiadores que terá nascido ao redor de 
1548--1549,\footnote{ Sobre esta questão da data de nascimento de Fernão
Cardim vide a dissertação de mestrado de Ana Maria de Azevedo, intitulada
\textit{O Padre Fernão Cardim (1548--1625), contribuição para o estudo
da sua vida e obra}, apresentada na Faculdade de Letras da Universidade
de Lisboa, em 1995, pp. 4--10.} em Viana de Alvito, no
Alentejo, pertencente ao Arcebispado de Évora, numa família de ``estirpe
antiga e importante''.\footnote{ Fernão Cardim era filho de Gaspar
Clemente e de D. Inês Cardim. Teve três irmãos, sendo o mais velho o
Doutor Jorge Cardim Fróis, que ocupou cargos importantes na
administração da Justiça, chegando mesmo a ser Desembargador dos
Agravos da Casa da Suplicação, na corte de Lisboa. Os outros dois
irmãos, Lourenço Cardim e Diogo Fróis, também foram membros da
Companhia de Jesus, tendo o primeiro morrido num ataque à armada onde
se encontrava a caminho do Brasil e a quem Cardim se refere na sua
\textit{Narrativa epistolar}, e tendo o segundo sido lente de Teologia
Moral, no Colégio de Évora. Vide a obra do Pe. Sebastião de Abreu,
\textit{Vida e virtudes do admirável Padre Joam Cardim da Companhia de
Jesus, portuguez, natural de Vianna do Alentejo, Évora}, Évora, na
Oficina desta Universidade, 1659, dedicada a um dos dez sobrinhos do
Pe. Fernão Cardim também membro da Companhia de Jesus.}

Cardim ingressou na Companhia em 1556 e veio a falecer no
Brasil, na aldeia de Abrantes, nos subúrbios de Salvador, em 1625,
depois de uma vida intensa de permanência nesse território,\footnote{ Vide 
o necrológio escrito pelo Pe. Antônio Vieira, resumindo"-lhe a
vida como a de ``um varão verdadeiramente religioso e de vida
inculpável, mui afável e benigno, e em especial com seus súbditos'', in
\textit{Annua da provincia do Brasil dos annos de 1624 e 1625}, 
publicada nos \textit{Anais da Biblioteca Nacional do Rio de Janeiro}, 
1827, \textsc{xix}, p. 187.} durante 42 anos, interrompidos
apenas por uma viagem como Procurador da Província de Jesus a Roma,
entre 1598 a 1601, e durante o período em que esteve prisioneiro na
Inglaterra, entre 1601 e 1603. 

 Da vida do Padre Fernão Cardim, antes da ida para o Brasil, sabe"-se
que fez estudos em Artes e Teologia, no Colégio de Évora. Foi depois
ministro do mesmo Colégio e adjunto do mestre de noviços em Coimbra. Já
com este cargo e professo de quatro votos, Cardim foi nomeado, em 1582,
secretário do visitador Cristóvão de Gouveia,\footnote{ Cristóvão
de Gouveia nasceu a 8 de janeiro de 1542, na cidade do Porto, e entrou
na Companhia de Jesus, em Coimbra, a 10 de janeiro de 1556.} 
seguindo para o Brasil a 5 de março do ano seguinte, na nau
\textit{Chagas de São Francisco}, onde ia também o governador Manuel
Teles Barreto,\footnote{ O governador Manuel Teles Barreto
desempenhou esse cargo entre 1583 e 1587, sendo o seu período o mais
difícil das relações institucionais com a Companhia de Jesus nesse
território. Foi o primeiro governador a ser nomeado durante o período
filipino, vindo a ter problemas com os padres jesuítas e, nomeadamente,
com o visitador Cristóvão de Gouveia de quem Cardim era secretário.} 
funcionários régios e vários padres jesuítas.\footnote{ Cf.
Joaquim Veríssimo Serrão, \textit{Do Brasil Filipino ao Brasil de 1640}. 
São Paulo: Companhia Editora Nacional, 1968, p. 27.} Depois de cerca
de 66 dias de viagem, com dez de paragem na ilha da
Madeira, chegaram à Baía de Todos os Santos, a 9 de maio de 1583. Entre
essa data e 1590, podemos acompanhar Cardim por terras das capitanias
de Bahia, Ilhéus, Porto Seguro, Pernambuco, Espírito Santo, Rio de
Janeiro e São Vicente, mais tarde São Paulo, descrevendo as paisagens e
os fatos que observava e dando"-nos uma ideia precisa e interessante do
Brasil dessa época, assim como as ordens proferidas pelo visitador para
bom funcionamento dos colégios e das residências existentes naquelas partes.

Além de reitor dos colégios da Bahia e Rio de Janeiro,\footnote{ O Pe. 
Fernão Cardim desempenhou o cargo de reitor dos colégios da
Bahia, entre 1590 e 1593, e posteriormente a partir de 1607 até a data
da sua morte, em 1625, e do colégio de São Sebastião do Rio de Janeiro,
entre 1596 e 1598.} o Padre Fernão Cardim foi eleito na congregação
provincial Procurador da Província do Brasil, em Roma, em 1598,
onde permaneceu cerca de três anos.\footnote{ Sobre a sua presença
em Roma, entre 1598 e 1601, pouco conseguimos descobrir, o mesmo
acontecendo em relação à sua presença em Bruxelas, antes de 1603.}
Quando regressava ao Brasil, Cardim e os seus companheiros de viagem
foram capturados, ainda na barra de Lisboa, por corsários ingleses,
tendo sido levado para a Inglaterra, onde esteve prisioneiro durante cerca
de três anos, enclausurado na prisão de Gatehouse, que Cardim denominou
de \textit{Gatus}, em Londres. Temos conhecimento deste período através
das cartas escritas do cárcere, por Cardim, em que procura diligenciar
a sua libertação, assim como a dos seus companheiros, e em que pretende
acima de tudo que lhe fossem devolvidos os seus manuscritos, que
considera como a maior riqueza da sua propriedade,\footnote{ As
cartas escritas de Londres pelo Pe. Fernão Cardim têm"-se mantido
inéditas, tendo sido referidas pelo Pe. Serafim Leite na sua obra,
\textit{História da Companhia de Jesus no Brasil}, vol. \textsc{viii}, Rio de
Janeiro, Instituto Nacional do Livro, 1949, pp. 132--137. Estas cartas
encontram"-se mencionadas e transcritas em inglês na Biblioteca do
Museu Britânico, in \textit{Hatfield Papers}, in \textit{Historical
Manuscripts Commission Reports}, Londres, 1910, e os manuscritos
originais escritos e assinados por Fernão Cardim, em português,
castelhano e latim, encontram"-se na Biblioteca de Hatfield
House, residência dos marqueses de Salisbury, sucessores do
conde de Salisbury, a quem eram dirigidas a maioria das cartas escritas
por Cardim quando se encontrava prisioneiro. Foram divulgadas
e transcritas na nossa dissertação de mestrado, vol. \textsc{ii}, pp. 3--15.}
objetivo que não conseguiu alcançar pois nunca mais teve contato com
os mesmos. Enquanto esteve enclausurado, entre dezembro de 1601 e março
de 1603, Cardim negociou a sua libertação, através de pedidos de
clemência dirigidos à rainha Elizabeth \textsc{i}, ou a \textit{Sir} Robert Cecil, conde de
Salisbury, figura de destaque da corte e, ainda, através de
diligências, por si feitas, para a libertação de cavaleiros ingleses
prisioneiros na Espanha e em Flandres, que acabaram por ter
êxito.\footnote{ Sobre a prisão do Pe. Fernão Cardim em Londres e
as diligências desencadeadas para a sua libertação, veja"-se o artigo de
W. H. Grattan Flood, ``Portuguese Jesuits in England in Penal Times'', in
\textit{The Month}, nº143, 1924, pp. 157--159.}

Após a sua libertação da Inglaterra e ao regressar ao Brasil, o Padre
Fernão Cardim foi nomeado Provincial do Brasil da Companhia de Jesus,
cargo que exerceu até 1609. Entre as medidas importantes tomadas por
Cardim no desempenho dessas funções, destaca"-se a encomenda da
hagiografia, do Padre José de Anchieta,\footnote{ Vide
``Carta do Padre Provincial Fernão Cardim, para o nosso
Reverendo Padre Geral Aquaviva'', escrita da Bahia, a 8 de maio de
1606, in \textit{Annaes da Bibliotheca Nacional do Rio de Janeiro}, 
1907, vol. \textsc{xxix}, Rio de Janeiro, Officinas de Artes Graphicas da
Biblioteca Nacional, 1909, pp. 183--184. O texto desta carta foi
publicado na íntegra na nossa dissertação já citada, no anexo
documental \textit{doc. 10}, p. 16.} e a preocupação em organizar
missões de conversão dos indígenas, nomeadamente os índios
Carijós,\footnote{ Os índios \textit{Carijós}, também denominados
de \textit{Guaranis}, pertenciam à família tupi"-guarani. Viviam na
região da lagoa dos Patos e no litoral do Paraná e Santa Catarina,
ocupando uma ampla faixa que compreendia parte do litoral desses dois
estados e o sertão até Assunção, no Paraguai.} Guaranis, na região da
Lagoa dos Patos, no sul do território.

Quanto à sua formação, sabemos que Fernão Cardim, após a primeira
fase de noviciado, continuou os seus estudos, que compreendiam cursos
de latim, com desenvolvimento de conhecimentos em gramática,
humanidades e retórica, grego, filosofia e teologia. Realizou, assim, a
formação em humanidades, artes liberais e teologia, estando, por
conseguinte, preparado para a principal missão dos jesuítas, que era o
ensino e a conversão dos índios à fé cristã. A sua formação jurídica
manifesta"-se em particular nas questões subjacentes ao Brasil
nascente e aos problemas inerentes a esse nascimento, sobretudo no que
concerne aos povos ameríndios. Esta sua formação é notória nas cartas,
nas informações e nos artigos que escreveu ao rei,
apresentando"-lhe as questões que este deveria ter em atenção sobre o
Brasil da época.\footnote{ Estes artigos que se têm mantido
inéditos, foram incluídos por Samuel Purchas na sua obra
\textit{Purchas, his Pilgrimes}, publicada em Londres, em 1625, logo a
seguir ao texto dos \textit{Tratados} atribuídos mais tarde ao Padre
Fernão Cardim, vol. \textsc{iv}, pp. 1320--1325. Foram traduzidos da edição
moderna de \textit{Hakluytus Posthumus or Purchas His Pilgrimes}, vol. \textsc{xvi}, 
Glasgow, Leon and Sons, 1906, intitulados de ``Artigos referentes
ao dever da Majestade de El"-Rei Nosso Senhor e ao bem comum de todo o
Estado do Brasil'' e publicados no artigo de Maria Odília Dias Curly,
``Um texto de Cardim inédito em português?'', in \textit{Revista de
História}, São Paulo, 1964, nº 58, vol. \textsc{xxviii}, Ano 15, abril"-junho,
pp. 455--482, incluídos e debatidos na nossa dissertação de Tese de
Mestrado, pp. 161--181.} A par do Direito, Cardim mostra conhecimentos e
preocupações do âmbito da teologia moral, instrumento absolutamente
necessário para a sua formação e missão perante os povos ameríndios e
os portugueses, com quem contatou nas terras brasílicas.

 Da sua preparação cultural conhecemos pouco, mas o estudo dos seus
textos permitem"-nos sentir a influência de alguns teólogos e filósofos,
como Santo Agostinho e São Tomás de Aquino, além, evidentemente, da
grande marca das Epístolas e dos Evangelhos. Como um homem culto da
época, Cardim tinha uma formação privilegiada, e como um humanista
quinhentista era conhecedor dos autores clássicos. Plínio devia mesmo
constar das suas leituras, pois aparecem várias vezes termos que fazem
a relação entre a obra clássica \textit{Naturalis Historiae Libri} e
a obra cardiniana. Relevante é ainda a influência das
\textit{Etimologias} de Santo Isidoro de Sevilha. A própria organização
dos assuntos ao longo dos seus textos incluídos nos \textit{Tratados da
terra e da gente do Brasil} mostra a preocupação em apresentá"-los
divididos, no caso dos animais não voadores quadrúpedes, em selvagens,
domésticos e exóticos, além do volucrário, dedicado às aves, o
ictuário, dedicado aos peixes e, por fim, o ofidário, para os répteis.
Era o modelo seguido pelos bestiários dos escritores antigos, a par do
herbário e do lapidário. 

 Cardim mostra ainda ter conhecimento dos seus
contemporâneos, como Nicolau Monardes, médico e naturalista espanhol
que escreveu várias obras sobre os produtos oriundos do continente
americano e que criou, inclusivamente, um Museu de História Natural em
Sevilha, em 1574. É ainda evidente a influência de textos em que se
forneciam informações sobre os produtos do Oriente, como os de João de
Barros, Garcia da Orta, Cristóvão da Costa e Duarte Barbosa.
Apercebemo"-nos, ao longo da obra cardiniana, que o seu autor não
recolheu informações de relance ou ao sabor do acaso, mas que, pelo
contrário, procurou mais dados que lhe permitissem dar a conhecer as
plantas e os animais do espaço brasílico, comparando"-os com os já
conhecidos, quer fossem do Oriente, de Portugal ou muitas vezes da Espanha. 

Cardim mostra também conhecimento das línguas ameríndias, sobretudo
do Tupi"-Guarani,\footnote{ O Tupi foi a língua usada pelos
jesuítas em suas catequeses desde o Maranhão até São Vicente. Não era
língua própria de uma tribo, mas uma ``língua geral'', resultante de uma
uniformização léxica racional de vários dialetos, que veio a ser
fixada pela gramática do Padre José de Anchieta e pelo vocabulário
jesuítico. O Guarani é um ramo do Tupi"-Guarani e foi falado
desde São Vicente até o Paraguai, onde ainda o é hoje, ainda que muito
influenciado pelo castelhano. Cf. Francisco da Silveira Bueno,
\textit{Vocabulário Tupi"-Guarani/Português}. São Paulo: Brasilivros, 
5ª ed., 1987.} pois ao longo dos seus textos inclui muitas 
vezes vocábulos e até mesmo frases nesse idioma, em que procura
fornecer a tradução dos mesmos. Mas surgem também nomes em
abanheenga\footnote{ Entenda"-se por abanheenga ou
abanheém a língua tupi antiga.} e em quechua ou 
quíchua.\footnote{ Este conhecimento da língua quechua ou
quéchua baseia"-se, possivelmente, na obra de Fr. Domingo de Santo
Tomás, \textit{Lexicon, o Vocabulario de la lengua general del Peru,
cõpuesto por el Maestro F. Domingo de S. Thomas}, Valladolid, por
Francisco Fernandez de Cordoua, (al fin 1560), in \textit{Catálogo de
obras impresas en Los Siglos \textsc{xvi} al \textsc{xviii} existentes en las
bibliotecas españolas}, sección \textsc{i}, siglo \textsc{xvi}, letra C, Madrid, 1972.
Veja"-se sobre esta obra quinhentista, José Maria Vargas, \textit{La
primera gramática Quichua, escrita por Fr. Domingo de Santo Tomás en
Valladolid en 1560}, Quito, Instituto Histórico Dominicano, 1947; e
ainda sobre esta língua falada pelos povos quíchuas, que na época 
da colonização espanhola habitavam a região que se estende ao
norte e oeste de Cuzco, o artigo de George Kubler, ``The Quechua in the
Colonial World'', in \textit{Handbook of South America Indians.}
\textit{The Andean Civilizations}, Edited by Julian H. Steward,
Washington, D.C., Smithsonian Institution Press, 1946, vol. 2, pp. 331--410.}

\section*{O contexto histórico do Brasil\break na época de Cardim} 

Fernão Cardim chegava ao Brasil em 1583, como secretário do visitador
Cristóvão de Gouveia, em pleno período filipino e quando esse território 
começou a assumir uma importância crescente no contexto da economia portuguesa. 

Com a Monarquia Dual em Portugal, entre 1580 e 1640, o Brasil não
pôde deixar de aderir à causa dos Filipes, quer concordasse ou não com
essa situação. Era ``um outro Portugal'', na profética expressão
de Cardim, que se estava a formar no território brasileiro. A marca
portuguesa e ``regional'' deslocava"-se, nesse período, para um outro
Brasil de concepção ``atlântica'', graças ao papel da administração,
quase sempre eficaz, e ao esforço, sem dúvida, criador do homem
português, que se prendeu à terra, a par do trabalho dos escravos
negros, que, desde muito cedo, foram importados em grande número da África.

 A inadaptação cultural dos índios ao trabalho permanente na terra,
assim como a preparação mais adequada dos africanos, no que diz
respeito ao domínio das técnicas ligadas à criação de gado, agricultura
ou até metalurgia, levou muitos capitães"-donatários a procurar
substituí"-los por africanos. Até porque estes mostravam adaptar"-se
melhor ao clima e às doenças, ao mesmo tempo que evidenciavam uma
capacidade de trabalho que os ameríndios não dispunham.\footnote{ Cf. 
Jorge Couto, \textit{A Construção do Brasil. Ameríndios,
Portugueses e Africanos, do início do povoamento a finais de
Quinhentos}. Lisboa: Cosmos, 1995, pp. 303--306.}

 Fundavam"-se povoados, abriam"-se linhas de comércio, criavam"-se
cargos públicos e aumentavam os interesses privados, cristianizava"-se o
gentio, descobriam"-se novas riquezas e defendia"-se a terra, de forma a
que nenhuma das suas parcelas fosse ocupada por estrangeiros.\footnote{ Cf. Joaquim 
Veríssimo Serrão, \textit{op. cit.}, pp. 1--2.}

Afastados do reino e dos problemas que aí decorriam com a entrega do
poder aos reis espanhóis, os portugueses do Brasil não intervieram na
questão dinástica. O apego do homem português à pátria distante já era
mais político do que afetivo. A distância assim o exigia, bem como as
dificuldades da vida. Perante isto aceitaram o monarca espanhol sem
contestação, a não ser alguma de caráter esporádico, mas também sem
grandes euforias. O mesmo se passaria quando da Restauração da
monarquia portuguesa, em 1640. 

Foi, de fato, durante o período filipino que se deu a fase de
apogeu da economia do açúcar brasileiro, marcando"-se a data de 1570
para o fim do sistema de administração estatal para o Comércio do
Oceano Índico, incrementando"-se o desenvolvimento da cultura da cana
sacarina.\footnote{ Cf. Frédéric Mauro, \textit{Portugal, o
Brasil e o Atlântico (1570--1670)}, trad. port., vol. \textsc{i}. Lisboa:
Estampa, 1989, p. 29.} Mas foi ainda notório nesse período o
incremento da expansão provocada pelo gado, com a penetração e a
conquista do Nordeste, com a expulsão dos franceses, com o impulso que
irá provocar o bandeirismo. Foi um período de expansão no interior e
conquista no litoral, desde a Paraíba, em 1585, até a ocupação do Pará,
em 1616, em plena época da presença de Cardim no Brasil. 
Desenvolvimento e expansão do Brasil que são visíveis não só no número
de engenhos de açúcar que são construídos na época em que Cardim
visitou o território, mas também no número de expedições feitas para o
interior, com as entradas e bandeiras,\footnote{ As bandeiras eram
expedições sertanejas empreendidas pelos paulistas, também denominados
de ``bandeirantes'', que penetravam pelos sertões em busca de cativos
indígenas e de pedras ou metais preciosos, sobretudo durante o século
\textsc{xvii}. Eram conhecidas na época como ``entradas'', ``tropas'' ou ``armações''.
Cf. John M. Monteiro, ``Bandeiras'', in \textit{Dicionário da História
da Colonização Portuguesa no Brasil}, coord. de Maria Beatriz Nizza da
Silva. Lisboa: Verbo, 1994, pp. 96--98.} a fundação de novas aldeias e
cidades e o consequente alargamento das fronteiras. É a penetração para
o interior e ao longo dos rios, desbravando os territórios, descobrindo
novas terras e riquezas, ao mesmo tempo que entravam em contato com tribos indígenas. 

Numa região com uma população estimada, no ano de 1583, em cerca de
57000 habitantes, dos quais 25000 eram brancos, 18000 ameríndios
convertidos e 14000 escravos negros, segundo o testemunho dos autores
da época, como o próprio Cardim e Anchieta, Olinda e Salvador eram as
povoações mais populosas e com maior desenvolvimento econômico,
mantendo"-se ainda incipiente São Sebastião do Rio de Janeiro.

A cidade de Salvador, capital da Capitania Real da Bahia, tinha na
época da presença de Cardim uma população estimada em cerca de 15000
almas, sendo uns 3000 portugueses, cerca de 4000 escravos e 8000
índios. Funcionava como cabeça do Estado do Brasil, nela residindo o
governador"-geral. Por seu lado, na cidade de São Sebastião do Rio de
Janeiro residiam apenas cerca de 150 vizinhos, em 1584.\footnote{ Cf. Fernão Cardim, 
\textit{Tratados da terra e gente do Brasil}, transcrição do texto, introdução e notas de 
Ana Maria de Azevedo, Lisboa, Comissão Nacional para as Comemorações dos Descobrimentos Portugueses, 
1997, p. 157. Utilizamos esta edição, referida a partir de agora como op. cit., 
para as referências em notas da presente edição.}

Atacada a Bahia pelos holandeses, em 1624, a grande maioria da
população refugiou"-se no interior. Cardim, que era na altura o reitor
do colégio e vice"-provincial, viu o seu colégio ser transformado em
armazém de vinhos, segundo o testemunho dos cronistas, e os próprios
mercadores aí se fixarem. Expulso e perseguido, refugiou"-se na aldeia
do Espírito Santo, mais tarde Abrantes, onde veio a falecer, em 1625,
depois de dirigir os destinos da Companhia nesse território.\footnote{ O Pe. 
Fernão Cardim teve que dirigir os destinos da Companhia no
território brasileiro, apesar de doente e já idoso, porque o Pe. Antônio 
de Matos, que chegara para substituir o provincial na
administração da província e mais doze padres, foi capturado e
conduzido para a Holanda, onde estiveram prisioneiros nos cárceres
públicos de Amsterdã, mais de vinte meses, até que foram resgatados
por diligências do Geral da Companhia.} 

\section*{Os Tratados de Fernão Cardim}

\begin{hedraquote}
Este Brasil é outro Portugal, e não falando no clima,
que é muito mais temperado, e sadio, sem calmas grandes, nem frios, e
donde os homens vivem muito em poucas doenças; [\ldots{}] nem falando do mar
que tem muito pescado, e sadio; nem das cousas da terra que Deus cá deu
a esta nação.\footnote{ Cf. Fernão Cardim, \textit{op. cit.}, p. 157.}
\end{hedraquote}

É esta visão do Brasil como \textit{um outro Portugal} que o Padre
Fernão Cardim procura transmitir ao longo dos textos. No entanto, não
escreveu para o prelo, e o conjunto da sua obra é essencialmente formado
por cartas destinadas ao Provincial da Companhia, em Lisboa, e outros
escritos, nos quais dá conta do que observa e das suas opiniões sobre a
terra e as gentes. Informante, naturalista, etnógrafo, botânico,
geólogo, zoólogo e cronista, Fernão Cardim foi, ainda, a testemunha que
viveu os acontecimentos que descreve de uma forma muito cativante,
contribuindo para o estudo da história social da fase inicial da
colonização do ``seu Brasil''. Um ``outro Portugal'', na sua opinião.

Os seus escritos mantiveram"-se desconhecidos durante séculos, só
vindo, em parte, a ser divulgados em língua portuguesa e atribuídos a
este jesuíta no século \textsc{xix}, mais precisamente em 1847. Na sua
totalidade os textos cardinianos continuam sem ser editados até ao
presente em Portugal. Durante longo tempo permaneceram inéditos, tendo
alguns sido publicados em inglês, mas atribuídos a outro autor.
Situação interessante e que mostra bem as vicissitudes por que, em
Quinhentos, passavam, muitas vezes, os manuscritos.

Dos textos de Cardim, o que foi divulgado em primeiro lugar em língua
portuguesa foi a \textit{Narrativa epistolar de uma viagem e missão
jesuítica pela Bahia, Ilheos, Porto Seguro, Pernambuco, Espírito Santo,
Rio de Janeiro, S.~Vicente (S.~Paulo), etc, desde o anno de 1583 ao de
1590, indo por visitador o Padre Christóvão de Gouvêa}, trabalho
editado em 1847 por Francisco Adolfo Varnhagen,\footnote{ Lisboa,
Imprensa Nacional, 1847.} que o dedicou à memória do cônego Januário da
Cunha Barbosa, fundador do Instituto Histórico e Geográfico Brasileiro.
Não é, no entanto, com este título que ocorre no \textit{Catálogo dos
Manuscritos da Bibliotheca Eborense}, ordenado por Joaquim Heliodoro da
Cunha Rivara, onde se inscreve: \textit{Enformação da Missão do Padre
Christóvão Gouvêa às partes do Brasil no anno de 83} (duas
cartas).\footnote{ Cf. Joaquim Heliodoro da Cunha Rivara.\label{cunharivara}
\textit{Catálogo dos Manuscritos da Bibliotheca Eborense}. Lisboa:
1950, tomo \textsc{i}, p. 19.} O manuscrito, referenciado nesse catálogo e na
edição do Visconde de Porto Seguro, encontra"-se na Biblioteca Pública e
Arquivo Distrital de Évora, incluído num conjunto de textos de vários
autores, com o título de \textit{Miscelânea, Cousas do Brasil}, com um
tipo de letra quinhentista e com a assinatura de Fernão Cardim igual à
das cartas enviadas do cárcere londrino.\footnote{ Cf. Biblioteca
Pública e Arquivo Distrital de Évora, \textit{Códice} \textsc{cxvi}, 1--33; 
entre as páginas 73 e 94v, encontra"-se a primeira carta, e entre
as páginas 95 e 98v. a segunda carta.}

Trata"-se de informações recolhidas por Cardim, quando desempenhava o
cargo de secretário do visitador Cristóvão de Gouveia e compiladas em
duas cartas, dirigidas ao provincial da Assistência de Portugal. A
primeira, escrita a 16 de outubro de 1585, era dirigida ao padre
Sebastião de Morais, que foi o nono provincial, entre 1580 e 1588, e a
segunda carta, em 1 de maio de 1590, também datada do Colégio da Bahia,
era dirigida ao Padre João Correia, que desempenhou as funções de
décimo provincial. É um relato da situação do Brasil em finais de
Quinhentos e da ação da Companhia de Jesus nesse território. Uma
descrição do que observava: como eram recebidos os missionários nas
capitanias, colégios e aldeias indígenas; dos hábitos dos ameríndios,
mas também dos colonos; dos interesses econômicos das várias regiões e,
até, do grau de religiosidade que envolvia esses indígenas, muito
superior ao de alguns portugueses. 

É com um espírito crítico que Cardim vai fornecendo ou transmitindo
informações nas cartas que remeteu ao seu superior em Portugal. Se se
mostra benevolente para com os índios, não reage da mesma forma com os
portugueses, que não sentiam a religiosidade dos ofícios e que viviam
mais para o luxo e vaidades. Considera mesmo que ``em
Pernambuco se acha mais vaidade que em Lisboa!''. Mas refere"-se aos
ofícios e cultos religiosos que celebravam ao longo das várias etapas
da viagem e também descreve, com um rigor impressionante, os rituais
antropofágicos praticados pelos ameríndios.

Rica de informações, esta \textit{Narrativa epistolar} é uma obra
fundamental para o conhecimento das atividades desenvolvidas no
território visitado pelo padre visitador e por Cardim. São aí
descritos os engenhos de açúcar, incluindo o seu funcionamento, os
principais produtos explorados em cada capitania e quem a dirigia, a
par dos hábitos indígenas e dos seus instrumentos, adornos, armas,
danças e cantos. De forma constante, o olhar de Cardim parece que
acompanhava a paisagem, os animais e as plantas com que se cruzava, ao
mesmo tempo que se interessava pela situação dos aldeamentos e pelos
problemas dos indígenas. No entanto, não é apenas o humanista curioso
que vai colocando no papel as suas observações, é também o sacerdote, o
missionário que se alegra com o número de batismos e casamentos que
vão celebrando. Mas é também o teólogo que não deixa de questionar o
seu valor e viabilidade perante as situações a que assistia na prática
diária das tabas indígenas.

Questões que não deixam de se captar ao longo destas informações que
são, sem dúvida, mais do que simples relatos e que suplantam, por isso
mesmo, outras escritas na mesma época. 

Esta \textit{Narrativa epistolar} foi publicada, de novo, no Rio de
Janeiro, por A. J. de Melo Morais, no seu texto integral, com o título
de \textit{Missões do Pe. Fernão Cardim}, na \textit{Chorographia
histórica},\footnote{ Cf. A. J. de Melo Morais,
\textit{Chorographia histórica}. Rio de Janeiro: 1860, tomo \textsc{iv}, 
pp. 417--457.} que correspondem à \textit{História dos jesuítas}\footnote{ Cf. \textit{História dos jesuítas}, Rio de Janeiro, 1872, 
tomo \textsc{ii}, pp. 417--457.} do mesmo autor. 

O texto cardiniano foi ainda reproduzido parcialmente, no que diz
respeito às descrições da cidade do Rio de Janeiro, pela revista
\textit{Guanabara}, em 1851\footnote{ Cf. revista
\textit{Guanabara}. Rio de Janeiro: 1851, vol. \textsc{ii}, pp. 115--122.} e, em
relação aos episódios que se referem à região de Pernambuco, foi
publicado em 1893, pela \textit{Revista do Instituto Arqueológico,
Histórico e Geográfico Pernambucano}, com algumas anotações de F.A.
Pereira da Costa,\footnote{ Cf. \textit{Revista do Instituto
Arqueológico, Histórico e Geográfico Pernambucano}. Recife: 1893, 43,
pp. 189--206.} e ainda nos \textit{Anais pernambucanos}, com
notas do mesmo autor.\footnote{ Cf. \textit{Anais pernambucanos}, 
vol. \textsc{i}. Recife: Arquivo Público Estadual, 1951.} As
descrições referentes à Bahia foram inseridas por Brás do Amaral, em
nota às \textit{Memórias históricas e políticas}, da autoria de
Accioli, em 1919.\footnote{ Cf. Hildebrando Accioli, 
\textit{Memórias Históricas e Políticas}. Bahia: 1919, vol. \textsc{i}, pp. 465--472.}

Mais tarde, os estudiosos de Cardim, sobretudo Capistrano de Abreu,
que recorreu a Lino Assunção para confronto dos textos publicados com
os manuscritos existentes em Évora,\footnote{ Cf. \textit{Fontes
da História do Brasil}, ``Cartas de Capistrano de Abreu a Lino
Assunção'', Lisboa, s. ed., 1946, pp. 17, 36, 48 e
``Correspondência'', \textsc{ii}, p. 462.} verificaram que o texto de
Varnhagen, e que serviu para as reproduções seguintes, apresentava
algumas incorreções e omissões, o que só foi possível detectar através
da colação com o apógrafo eborense. 

A \textit{Narrativa epistolar} voltaria a ser publicada no Rio de
Janeiro, em 1925, data do centenário da morte do autor, incluída numa obra
dedicada a Cardim, em conjunto com os outros textos deste jesuíta,
intitulada \textit{Tratados da terra e gente do Brasil}, com Introdução
e notas de Baptista Caetano, Capistrano de Abreu e Rodolfo Garcia.
Nesta obra foram incluídos os restantes textos que Fernão Cardim
escreveu sobre a terra e gentes do Brasil, com os títulos \textit{Do
princípio e origem dos índios do Brasil e de seus costumes, adoração e
cerimônias} e \textit{Do clima e terra do Brasil e de algumas cousas
notáveis que se acham assim na terra como no mar}, que tinham sido
publicados pela primeira vez em inglês, no ano de 1625, em Londres, na
coleção \textit{Purchas his Pilgrimes}, sob o título \textit{A
Treatise of Brasil written by a portuguese wich had long live there.}\footnote{ Cf. 
Samuel Purchas, \textit{Purchas his Pilgrimes}, 
London, 1625, ``The Seaventh Booke'' -- ``Voyages to and about the
Southern America, with many Marine Observations and Discourses of Those
Seas and Lands by English"-men and others'', onde estão publicados os
textos de Fernão Cardim, vol. \textsc{iv}, pp. 1289--1320.} 

Samuel Purchas, que adquirira estes manuscritos por bom preço, depois
dos mesmos terem sido confiscados ao padre Fernão Cardim, após ter sido
capturado por corsários ingleses e expropriado dos seus bens,
considerou"-os de grande qualidade e os mais completos que já tinha
visto sobre o Brasil, atribuindo"-os a um ``frade ou jesuíta português'',
de quem o corsário inglês Francis Cook, de Dartmouth, se tinha
apoderado, em uma viagem ao Brasil, em 1601, e que os tinha vendido por
vinte xelins a um certo mestre Hackett. Apresenta assim esses textos:

\begin{hedraquote}
Leitor, apresento aqui o mais exato Tratado do Brasil que
já vi escrito por alguém, especialmente na História das múltiplas e
diversificadas nações e costumes dos homens; assim como na história
natural dos animais, serpentes, aves, peixes, árvores, plantas, com
espécies de assinalável raridade dessas regiões. Foi escrito (segundo
parece) por um padre (ou jesuíta) português que viveu trinta anos
nessas partes.\footnote{ Cf. Samuel Purchas, \textit{op. cit.}, 
ed. de Glasgow, 1906, pp. 417--418. Tradução da autora.}
\end{hedraquote}

No entanto, como nas últimas páginas dos manuscritos viessem
incluídas umas receitas medicinais assinadas pelo irmão Manuel
Tristão,\footnote{ Manuel Tristão (1546--1631?), natural dos Açores,
das ilhas do Faial ou Santa Maria. Entrou na Companhia de Jesus já com
22 anos de idade, a 19 de maio de 1568, onde foi enfermeiro durante
muitos anos no Colégio dessa cidade; passou a Pernambuco e residia, em
1606, no Colégio de Olinda e, em 1607, na aldeia de Sto. André de
Goiana. Entre 1613 e 1617 esteve na aldeia de N. Sra. da Escada. Vivia já
velho no Colégio de Olinda, em 1621. Deixou uma breve 
\textit{Coleção de receitas medicinais}, conhecida de Samuel Purchas, em 1625. Cf.
Samuel Purchas, \textit{op. cit.}, p. 417.} contemporâneo de Cardim,
enfermeiro do Colégio da Bahia, que também esteve no Colégio de Olinda
e em várias aldeias, Purchas considerou"-o como autor dos
\textit{Tratados}, afirmando em nota lateral: ``Encontrei 
no fim do livro algumas receitas medicinais e o nome subscrito
Ir. Manuel Tristão, enfermeiro do Colégio da Bahia: o qual imagino
tenha sido o autor deste Tratado.''\footnote{ Cf. Samuel
Purchas, \textit{op. cit.}, p. 417 (nota lateral). Tradução da autora.}

Coube a Capistrano de Abreu o mérito de reivindicar para o
Padre Fernão Cardim a autoria dos referidos manuscritos, publicando, em
1881, o tratado referente aos índios.\footnote{ Fernão Cardim,
\textit{Do princípio e origem dos índios do Brasil}, Rio de Janeiro.
Typographia da ``Gazeta de Notícias'', 1881, 121 páginas e na
\textit{Revista do Instituto Histórico e Geográfico Brasileiro}, 57,
1ª p., em 1894, pp. 183--212.} Nesta edição, este especialista cardiniano
demonstra como os manuscritos devem ser atribuídos a Cardim, não apenas
pela coincidência dos textos terem sido roubados em 1601 pelo mesmo
corsário inglês, que também na mesma altura aprisionara a nau onde
viajava o padre Fernão Cardim e o espoliara dos seus manuscritos, mas
também porque, pela colação desses autógrafos com o texto da
\textit{Narrativa epistolar}, já antes atribuída a Cardim, é bem
patente o mesmo estilo de escrita e até alguma semelhança de assunto.
Demonstra ainda que pelo exame do texto se verifica que o opúsculo
foi escrito em 1584, data em que Cardim já se encontrava no Brasil,
tendo chegado a 9 de maio de 1583, como se sabe pela \textit{Narrativa epistolar}, 
o que pode esclarecer a autoria do referido manuscrito. Até
porque a primeira carta deste texto é de 16 de outubro, o que aproxima
muito a data de redação das duas versões e a conformidade de ideias e
de forma, ainda que os objetivos e os destinatários da mesma sejam diferentes.
O próprio confronto da assinatura de Cardim, desta primeira carta, com as das 
cartas enviadas do cárcere londrino permite"-nos identificá"-lo como o autor desse texto.   

O mesmo historiador brasileiro considera também que um enfermeiro,
apesar dos seus conhecimentos específicos, não era um erudito e que,
quanto às receitas medicinais que aparecem incluídas no final dos
manuscritos, poderiam ter sido ensinadas pelo dito enfermeiro ao
missionário.\footnote{ Cf. Rodolfo Garcia, ``Introdução'', in
Fernão Cardim, \textit{op. cit.}, pp. 11--22.} O próprio Fernão Cardim
mostra ao longo dos seus textos ter conhecimentos de algumas receitas e
hábitos medicinais, preocupando"-se em descrever em pormenor as doenças
e respectivos tratamentos que foram ministrados ao longo da viagem,
assim como as várias ervas que os índios utilizavam para as mezinhas e
as que eram medicinais, comparando"-as com as existentes ou conhecidas em Portugal. 

No próprio ano de 1881, já parte do texto do tratado \textit{Do clima
e terra do Brasil} tinha começado a ser publicado pelo Dr. Fernando
Mendes de Almeida, na \textit{Revista Mensal da Secção da Sociedade de
Geographia de Lisboa no Rio de Janeiro}, mas apenas os dois primeiros
capítulos e sem o nome do autor.\footnote{ O texto está publicado no
tomo \textsc{i}, números 1 e 2 .} No terceiro número dessa revista foi
incluída uma carta de Capistrano de Abreu, em que explicava a origem do
manuscrito que servia para impressão e atribuindo a sua autoria a
Fernão Cardim. Mais tarde, em 1885, no tomo \textsc{iii} da mesma
publicação, o referido historiador publicou a versão integral
do referido tratado e um estudo biobibliográfico sobre o seu autor. 

No que concerne ao segundo tratado, dedicado aos índios,
também publicado em 1625 por Purchas, identificada que estava a
autoria do primeiro texto, e uma vez que o estilo é o mesmo e os
conhecimentos do Brasil apresentados nos dois textos são idênticos,
Capistrano de Abreu considerou que a autoria é do Padre Fernão Cardim.

Estes dois textos de Cardim e as duas cartas que formavam a
\textit{Narrativa epistolar} foram compilados, finalmente, numa obra
única, com o título de \textit{Tratados da terra e gente do Brasil}, em
1925, e publicados, no Rio de Janeiro, na comemoração do centenário da
sua morte.\footnote{ Cf. Fernão Cardim, \textit{Tratados da terra
e gente do Brasil}, Rio de Janeiro, Editores J. Leite e Cia., 1925.} O
título de conjunto foi justificado por Afrânio Peixoto, na altura
presidente da Academia Brasileira de Letras, da seguinte forma:

\begin{hedraquote}
Pela primeira vez reúnem"-se, num só tomo, com o seguimento
que parece lógico, o aparelho de notas elucidativas e o título a que
têm direito, os tratados do Padre Fernão Cardim sobre o Brasil. [\ldots{}]
Portanto, aos três tratados do Padre Fernão Cardim parece exato o
título que lhes damos: \textit{Tratados da terra e gente do Brasil} 
que são agora não só a homenagem a um grande missionário que
amou, observou, sofreu e tratou o Brasil primitivo, com contribuição do
nosso reconhecimento a essas missões jesuíticas, que educaram os
primeiros brasileiros, e para os de todos os tempos deixaram memórias
passadas nos seus escritos, cartas e narrativas.\footnote{ Cf. Afrânio Peixoto, 
nota introdutória à obra de Fernão Cardim, \textit{op. cit.}, pp. 11--12.} 
\end{hedraquote}

 Os textos de Fernão Cardim foram publicados, posteriormente,
em 1933, com as mesmas notas e textos introdutórios de Baptista
Caetano, Capistrano de Abreu e Rodolfo Garcia, todos importantes
historiadores brasileiros e que se dedicaram ao estudo da obra
cardiniana.\footnote{ Cf. Fernão Cardim, \textit{Tratados da terra
e gente do Brasil}. Rio de Janeiro/São Paulo: J. Leite, Companhia
Editora Nacional, 1933.} Mais recentemente, em 1980, foi publicada uma nova
edição destes tratados cardinianos, mas que mantém o mesmo
texto, notas e estudos introdutórios da edição de 1925, uma vez que se
trata de um \textit{fac"-símile}.\footnote{ Cf. Fernão Cardim, \textit{Tratados
da terra e gente do Brasil}. Belo Horizonte/São Paulo: Itatiaia/Ed\textsc{usp}, 1980.} 
Em Portugal os textos de Cardim nunca tinham sido publicados até então.

\section*{O Brasil quinhentista visto\break pelo Pe. Fernão Cardim}

\begin{hedraquote}
A cidade está situada em um monte de boa vista para o mar,
e dentro da barra tem uma baía que bem parece que a pintou o supremo
pintor e arquiteto do mundo novo Deus Nosso Senhor, e assim é cousa
formosíssima e a mais aprazível que há em todo o Brasil, nem lhe chega
a vista do Mondego e Tejo\ldots{}\footnote{ Cf. Fernão Cardim,
\textit{op. cit.}, pp. 267--268.}
\end{hedraquote}
 
É com esta significativa admiração que Fernão Cardim
apresenta a cidade de São Sebastião do Rio de Janeiro, aonde chegara nos
finais do ano de 1584. Homem sensível que se deixou seduzir pela
formosura natural dessa cidade, ainda hoje considerada como uma das
mais belas do mundo. Não era apenas a visão edênica do Novo Mundo,
usual por parte dos escritores quinhentistas, mas também uma
apreciação mais apurada do território brasileiro e das suas
potencialidades. De fato, o cenário do continente americano
oferecia"-se aos primeiros descobridores e mais tarde aos
colonizadores: ``[\ldots{}] armado pela expectativa de um \textit{non plus ultra} de maravilha, 
encantamento e bem"-aventurança, sempre a inundá"-lo em sua luz mágica [\ldots{}]'',
como afirma Sérgio Buarque de Holanda, naquela que
é considerada como um clássico da historiografia brasileira, a
\textit{Visão do Paraíso.}\footnote{ Cf. Sérgio Buarque da Holanda,
\textit{Visão do Paraíso}. Rio de Janeiro: José Olympio, 1959, p. 268.}

As descrições dos autores quinhentistas e seiscentistas
correspondiam, quase, ao tradicional tema dos hortos de delícias.
Trata"-se nela da bondade dos ares, da sanidade da terra, da feliz
temperança do clima, da abundância e variedade do mantimento,
principalmente das frutas, da amenidade e beleza da vegetação,
sugerindo a imagem dos formosos jardins e hortos do Éden.

A nostalgia do jardim do Éden e a convicção de que se aproximavam os
tempos escatológicos, a par da vontade de estender a religião cristã a
terras novas e o desejo de encontrar em abundância o ``tão desejado
ouro e pedras preciosas'' e os outros produtos raros, conjugaram"-se
para considerar aquelas terras tão maravilhosas como o verdadeiro
Paraíso Terrestre. A cultura na qual participavam e os sonhos que ela
veiculava levaram os viajantes do Quinhentos a reencontrar nas regiões
insólitas da América as características das terras abençoadas que
assediavam desde a Antiguidade a imaginação dos ocidentais.\footnote{ Cf. Jean Delumeau, 
\textit{Uma História do Paraíso}. Lisboa: Terramar, 1994, p. 134.} 

 O próprio Cristóvão Colombo estava convencido de que as 
Índias se encontravam na vizinhança do Paraíso Terrestre, pois estava
fortemente impressionado pela beleza da Hispaniola (São
Domingos/Haiti), que considerou ser esta ilha única no mundo, porque se
achava coberta de toda a espécie de árvores que pareciam tocar o céu e
não perdiam nunca as suas folhas.\footnote{ Cf. Cristóvão
Colombo. \textit{\OE uvres}. Paris: Galimard, 1961, p. 181.} A
propósito de uma outra paisagem insular, a do Cabo Hermoso, escreveu
Colombo: ``Ao chegar à altura desse cabo, vem da terra um
odor de flores e de árvores tão bom e tão suave, que era a coisa mais
doce do mundo.''\footnote{ \textit{Idem, ibidem}, p. 181.} 

Idêntica sensação transmite Fernão Cardim nos seus textos ao
descrever as paisagens das terras brasílicas. Tocado pelas cores,
odores, sabores e paisagens tão diversificados, descreve"-os de uma
forma minuciosa, elogiando as suas qualidades e recorrendo,
frequentemente, a analogias para possibilitar ao leitor uma melhor
visualização daquilo que ele tinha conseguido observar, sentir, cheirar
ou mesmo saborear. Mas o que não nos deixa dúvida é que este autor
procurava também, através dos seus textos, dar a conhecer aquelas
terras com interesse para que fossem aproveitadas, já que tinham um clima
mais ameno, onde havia menos doenças e as terras eram mais férteis e,
em muitos aspectos, semelhantes a Portugal. Mais de uma vez não deixa
de afirmar nos seus textos que:

\begin{hedraquote} 
Este Brasil é outro Portugal, e não falando no clima que é
muito mais temperado, e sadio, sem calmas grandes, nem frios, e donde
os homens vivem muito com poucas doenças, como de cólica, fígado,
cabeça, peitos, sarna, nem outras enfermidades de Portugal; nem falando
do mar que tem muito pescado, e sadio; nem das cousas da terra que Deus
cá deu a esta nação.\footnote{ Cf. Fernão Cardim, \textit{op. cit.}, p. 157.} 
\end{hedraquote}

 A preocupação pela descrição rigorosa do território
brasileiro permite"-nos também um conhecimento precioso do cotidiano
nos engenhos e nas fazendas, onde Cardim pernoitou, quando da sua
viagem como secretário do visitador. São informações de significativo
valor, pois além de mostrarem o papel dos jesuítas nessa região,
apresentam"-nos alguns dos hábitos de vida nos engenhos.

\section*{O «olhar» cardiniano da Terra brasílica}

 Os \textit{Tratados} de Fernão Cardim permitem"-nos ter um
conhecimento aproximado da terra brasileira do Quinhentos, já que são
descrições em primeira mão. É, de fato, o fruto da experiência vivida,
que é descrito por este autor, a par de um significativo poder de
observação. Assim, o seu ``olhar'' pelas terras brasileiras não vai
ficar, apenas, pelas condições climatéricas ou do solo, já que aborda
ainda a fauna e a flora. Menciona, de forma cuidada, os diversos
animais terrestres, marinhos e aéreos, assim como os arbustos, as
plantas e árvores, procurando apresentar sempre a sua utilidade ou os seus perigos.

A influência dos conhecimentos dos índios é evidente em Cardim, em
muitos dos textos, nomeadamente na utilização dos próprios nomes em
tupi dos animais e plantas, por eles atribuídos às espécies. Refere
também a sua utilidade, quer para a alimentação, quer para outros
fins, esclarecendo da economia de subsistência dos índios, que
aproveitavam todos os recursos que a natureza lhes proporcionava. Esta
situação é corrente por parte dos naturalistas portugueses que, tal
como Cardim, estudaram a fauna brasílica e que foram, sobretudo,
tributários da tradição utilitária medieval, em parte, talvez, por
falta de uma formação específica que contribuísse para valorizar as
novidades científicas com que depararam. Por outro lado, porque essa
motivação prática constituía um dos anseios da época, um dos interesses
inerentes aos Descobrimentos e à sua motivação econômica, era natural
que, ao descreverem os seres vivos, os apreciassem não só pela sua
beleza ou originalidade, como também pelo seu valor como
recursos.\footnote{ Cf. Carlos Almaça, ``Os portugueses do Brasil
e a zoologia pré"-lineana'', in \textit{A Universidade e os
Descobrimentos}. Lisboa: \textsc{in"-cm}, 1993, p. 192.} 

No que diz respeito aos animais, Cardim menciona 108 de
várias espécies de mamíferos, répteis, aves, peixes, moluscos e
crustáceos, procurando sempre alertar para os perigos que alguns podiam
trazer, ou para a beleza e originalidade de outros. Assim acontece, por
exemplo, com a preguiça, animal que surpreendeu a maioria dos cronistas
e viajantes da época, como, por exemplo, Gabriel Soares de Sousa, na
obra \textit{Notícia do Brasil.}\footnote{ Cf. Gabriel Soares de
Sousa. \textit{Notícia do Brasil}. Lisboa: Alfa, 1989, p. 180.}

 Outros animais, como as cobras, despertaram em Cardim um
olhar de respeito e ao mesmo tempo de temor. Ao descrevê"-las procura
alertar para os seus perigos e para o procedimento a adotar no caso de
serem mordidos. Uma delas é, por exemplo, a \textit{jararaca}, que em
etimologia tupi significa ``que envenena quem agarra'', e cujo veneno
mata, de fato, em menos de vinte e quatro horas, e de que Cardim dá
testemunho das várias espécies. Interessante é a analogia que Cardim
tira da existência da grande quantidade de ofídios e o tipo de clima,
alegando que:

\begin{hedraquote}
Parece que este clima influi peçonha, assim pelas infinitas
cobras que há, como pelos muitos Alacrás [lacraus], aranhas e outros
animais imundos, e as lagartixas são tantas que cobrem as paredes das
casas, e agulheiros delas.\footnote{ \textit{Idem, ibidem}, p. 84.}
\end{hedraquote}

Mas Cardim não descreve, minuciosamente, apenas os animais
mortíferos e horrendos que observou. Sentiu"-se igualmente atraído pela
beleza das cores e dos cantos das muitas aves que viu e ouviu no
Brasil. É ele próprio que afirma:

\begin{hedraquote}
Assim como este clima influi peçonha, assim parece influir formosuras nos pássaros, e assim
como toda a terra é cheia de bosques, e arvoredos, assim o é de formosíssimos pássaros de 
todo o gênero de cores.\footnote{  \textit{Idem, ibidem}, p. 84.}
\end{hedraquote}

 Os peixes, muito abundantes nas águas brasileiras, foram
outras das espécies a que Fernão Cardim dedicou muita atenção,
distinguindo os que habitavam em água salgada e doce, os que eram
peçonhentos, além de outras espécies da fauna marítima, como os
caranguejos, lagartos de água e os lobos"-do"-mar. O seu gosto pelos
peixes é notório, talvez devido à dieta alimentar, que permitia maior
consumo deste tipo de alimento do que de carne. Compara"-os aos
existentes em Portugal, mostrando as semelhanças, ao mesmo tempo que
enaltece a qualidade dos peixes das águas brasileiras. Mas, a par das
descrições, marcadamente rigorosas, da fauna marinha, que poderemos
afirmar quase científicas, já que graças aos pormenores nos permitem
identificar perfeitamente as espécies descritas e até visualizá"-las e
confrontá"-las com as que ainda hoje existem no atual território
brasileiro, encontramos a descrição dos homens marinhos e monstros do
mar, como o \textit{Ipupiara}. É a mesma lenda dos tritões,
sereias e outros seres fantásticos que são manifestados nos textos dos
autores que trataram o Brasil quinhentista. A eles se referem Pêro de
Magalhães de Gândavo, Gabriel Soares de Sousa, Frei Vicente de
Salvador, entre outros, que os descrevem à semelhança de Cardim. Esta
antropologia fantasista já tinha tradição na Península Ibérica, quer
através dos textos de Solinus e Plínio, quer através das
\textit{Etimologias}, de Santo Isidoro de Sevilha, muito frequentes nas
bibliotecas da época medieval. São deste último autor as seguintes palavras:

\begin{hedraquote}
Assim como em cada nação há certos monstros de homens,
assim no Universo há certos povos de monstros, tais o dos gigantes, os
cinocéfalos, os cíclopes etc. Os gigantes [\ldots{}] dizem que foi a terra,
com sua mole imensa, que os gerou\ldots{} Os cinocéfalos são assim chamados
porque têm cabeça de cão e no seu próprio latir manifestam ter mais de
besta que de homens: são naturais da Índia.\footnote{ Cf. Santo Isidoro 
de Sevilha, \textit{Etimologias}, trad. castelhana, Madrid, \textsc{b.a.c.}, 1951, 
pp. 280--281, cit. in J. S. da Silva Dias, \textit{Os descobrimentos portugueses 
e a problemática cultural do século \textsc{xvi}}. Coimbra. Universidade de Coimbra: 1973, p. 195.}
\end{hedraquote}

Fernão Cardim não deixa de mencionar, com significativo interesse, os
outros peixes que encontrou semelhantes aos de Portugal e Espanha,
considerando"-os mais saborosos do que os da Europa. É o caso das
tainhas, garoupas, chicharros, pargos, sargos, gorazes,
dourados, peixes"-agulha, pescadas, sardinhas, raias, toninhas, linguados
e salmonetes. Atribui"-lhes qualidades diferentes ao afirmar:

\begin{hedraquote}
Todo este peixe é sadio cá nestas partes que se come sobre
leite, e sobre carne, e toda uma quaresma, e de ordinário sem azeite
nem vinagre, e não causa sarna, nem outras enfermidades como na Europa,
antes se dá aos enfermos de cama.\footnote{ Cf. Fernão Cardim, op.cit., p. 132.} 
\end{hedraquote}

A descrição da terra brasílica não ficava completa se Cardim não
apresentasse também as árvores e ervas que aí existiam e a sua
utilidade, quer para a alimentação, quer para a farmacopeia. Assim,
Fernão Cardim assume as funções de botânico ao examinar e descrever
cerca de 64 espécies da flora brasileira, enumerando as
árvores de fruto indígenas, as que servem para extrair madeira, as que
têm água, as que se criam na água salgada, as ervas que são fruto e que
se comem, ao mesmo tempo que é o boticário que procura conhecer as
``árvores que servem para medicinas'' e as ``ervas que
servem para mezinhas'', ainda as ``que são cheirosas'', 
realçando o contributo dos índios para o conhecimento dessas plantas,
além de evidenciar, também, conhecimento de obras de autores
contemporâneos que a elas se referem.\footnote{ O Padre Fernão
Cardim cita no seu texto Nicolau Monardes, médico e naturalista
espanhol, que nasceu em Sevilha em 1493, aí falecendo em 1588.
Contemporâneo de Cardim, apesar de não ter estado na América,
dedicou"-se ao estudo das produções naturais desse continente através
dos testemunhos dos viajantes, conseguindo formar um pequeno museu de
história natural, que foi um dos mais antigos da Europa, existindo
desde 1554. A sua principal obra foi \textit{Primeira y segunda y
tercera partes de la historia medicinal de las cosas que se traen de
nostras Indias Occidentales que sirven en medicina, etc.}, publicada em
Sevilha, em 1574, onde se encontram diversos tratados anteriormente
dados à estampa.} É significativo o interesse gastronômico que Cardim
coloca nas suas descrições dos animais e das plantas. Uma das que lhe
despertou mais interesse, como a outros viajantes, foi a
erva"-santa, também conhecida por \textit{fumo, pétum, ou
petigma}, atendendo à sua variada utilização no tratamento das doenças
e pelo estado de embriaguez em que deixava as pessoas. Afirma Cardim:

\begin{hedraquote}
Esta erva"-santa serve muito para várias enfermidades, como
ferimentos, catarros etc., e principalmente serve para doentes da
cabeça, estômago e asmáticos\ldots{}\footnote{ Cf. Fernão Cardim, \textit{op. cit.}, pp. 123--124.} 
\end{hedraquote}

 Cardim refere"-se ao tabaco, planta que atraiu muito os
viajantes da época pela sonolência que provocava. Já Damião de Góis, na
\textit{Crónica de D. Manuel}, em 1556--1567, a mencionara: ``E
a que chamamos erva do Brasil, do fumo e eu chamaria erva"-santa, a que
dizem que eles (\textit{os índios}) chamam betun. Esta erva
trouxe primeiramente a Portugal Luís de Góis''.\footnote{ Cf. Damião 
de Góis, \textit{Crônica do Felicíssimo Rei Dom Manuel}, 
Lisboa, 1566--1567, p.1, Cap. 56, fl. 52. Luís de Góis era irmão de
Pero de Góis, donatário de São Tomé, com quem foi para o Brasil. Foi um
dos companheiros de Martim Afonso de Sousa na fundação de São Vicente,
ingressou posteriormente na Companhia de Jesus. Terá sido ele a
introduzir o hábito de fumar em Portugal.}

 Fernão Cardim interessou"-se também pelas árvores
e plantas das quais se extraíam óleos que os índios utilizavam para
untar o corpo e o cabelo, adorno que o surpreendeu bastante, tal como a
outros viajantes quinhentistas e de todos os tempos, pois presentemente
etnólogos e antropólogos têm feito estudos sobre estas técnicas. São de
referir os trabalhos de Florestan Fernandes sobre os Tupinambás, de
Manuel Diegues Júnior sobre as diversas etnias e culturas brasileiras,
e, entre tantos outros, de Claude Lévi"-Strauss, que na sua obra já
clássica, \textit{Tristes trópicos}, procurou mesmo explicar os vários
desenhos e pinturas corporais dos Bororo.\footnote{ Cf. Claude
Lévi"-Strauss, \textit{Tristes trópicos}. Lisboa: Edições
70, 1986, pp. 171--191.}
 
 De todas as espécies de flora da terra brasílica que muito
agradou a Cardim, como a quase todos os autores da época, foram
``as ervas que são fruto e se comem'', de entre as quais mereceu
maior destaque a \textit{mandioca}, considerado o mantimento que
usualmente servia para fazer pão nas terras brasileiras. São"-lhe
atribuídas outras faculdades, além de ingrediente essencial para o
fabrico da farinha para o pão, biscoito, mingau,\footnote{ O \textit{mingau} 
é um caldo ou papa rala, que ainda hoje está incluído
nos hábitos alimentares do povo brasileiro, assim como o
\textit{pirão}, que é uma papa grossa, que os índios faziam com
diversas farinhas de mandioca e caroços de algodão, milho ou
amendoim.} bolos, queijadinhas de açúcar, tortas, empadas, para o
fabrico de bebidas fermentadas e até de um remédio para o tratamento
do fígado. Principal planta alimentar para os ameríndios da floresta
tropical, tornou"-se um dos principais, senão mesmo o principal
alimento do povo brasileiro e depois dos povos africanos, por ter sido
transplantado para a África.

 O valor alimentar desta planta justifica o testemunho que a
ela dedica Cardim e outros viajantes da época que, ao fazê"-lo, estão a
referir"-se àquilo que hoje se denomina de ``mandioqueira amarga'', cujas
raízes são tóxicas devido à presença de ácido cianídrico, exigindo
complexos tratamentos para se tornarem comestíveis. A que Cardim chama
de aipim é a ``mandioqueira doce'', cujas raízes são isentas de
ácido cianídrico e são consumidas e utilizadas na confecção de diversos
produtos alimentares. A bebida fermentada, o cauim, que os
índios confeccionavam com essa planta ou outras, como o milho,
batata"-doce, cará e amendoim, ou ainda com mel, seiva de palmeiras e
de frutos como o ananás ou o caju, e a que Cardim se refere,
descrevendo o processo de fabrico, era obtida pela fervura destas
plantas e pela adição de saliva (com a ptialina), conseguindo
assim, através desta, o processo de fermentação. Esta operação era
realizada pelas jovens da aldeia, de preferência virgens, ou, na falta
destas, por mulheres casadas que se abstinham por alguns dias da
prática de relações sexuais de forma a conceder"-lhe ``que
é o que dizem que lhe põe a virtude segundo a sua 
gentilidade''.\footnote{ Cf. G. S. de Sousa, \textit{op. cit.}, pp. 110--111. O
texto deste autor é muito rigoroso e completo sobre esta planta, a
mandioca, a sua utilidade e a diversidade de aproveitamento,
indo ao encontro do testemunho do Padre Fernão Cardim.} 

 As bebidas fermentadas, de acordo com o testemunho de Fernão
Cardim e de outros autores seus contemporâneos, desempenhavam uma
importante função no padrão de vida indígena, sobretudo como
fornecedoras de vitaminas, compensando parcialmente as deficiências da
dieta alimentar. O seu consumo variava consoante o grupo tribal, já que
se uns as consumiam regularmente, outros apenas o faziam em grandes
celebrações coletivas, especialmente na época da piracema\footnote{ A \textit{piracema} 
era, neste sentido, o nascimento de mais um
elemento da tribo. O termo tupi designava a saída dos peixes para a
desova e por analogia designava também o nascimento das crianças.
Ocorre pela primeira vez num texto em latim, com o primeiro sentido, em
1560, numa carta de José de Anchieta e num texto português,
apenas em 1895, com Joaquim Veríssimo, em \textit{A pesca na
Amazônia.}} ou por ocasião do sacrifício ritual de 
prisioneiros.\footnote{ Cf. Jorge Couto, \textit{op. cit.}, pp. 85--86.}

 Não é apenas a mandioca que é referida por Cardim dentro do
conjunto vegetal das ``ervas que são fruto e se comem''. São
ainda mencionadas a naná,\footnote{ Esta planta que Cardim
denomina de \textit{naná}, do tupi \textit{na"-nã} = ``cheira"-cheira'', é
o ananaseiro.} e o maracujá\footnote{ Cardim
refere"-se ao maracujá ou murucujá, que em tupi
significa ``fruto que faz vaso'', ``que dá vasilha''.}. À
primeira, o ananás, atribui muitas qualidades, além da fruta que
considera muito gostosa, comparando o sumo ao sabor do melão; anota que
servia para curar os doentes ``da pedra'' e de febres e até para tirar
nódoas da roupa. Acrescenta que os índios também faziam vinho desta
fruta, que servia para desenjoar no mar. Por sua vez, o maracujá é uma
espécie vegetal alimentícia muito elogiada por Cardim não só pelo
sabor, como pela beleza da planta, que trepa pelas paredes e árvores como se fosse hera. 

 É notório, ao longo dos \textit{Tratados} cardinianos, a
apresentação da terra brasílica, das suas qualidades climáticas, dos
rios que aí existiam, ao mesmo tempo que são mencionados os animais e
as plantas que Cardim foi vendo ao longo das suas viagens. A observação
foi, de fato, uma prática do conhecimento nos viajantes e cronistas
portugueses e estrangeiros que entraram em contato com o Novo
Mundo. Fato estimulado pelo ineditismo dos ambientes e pela riqueza
da fauna e flora descobertas nessas paragens americanas. Experimentaram
as novas plantas, elogiadas com agrado, mostrando o interesse pelos
hábitos alimentares dos indígenas. De fato, estes aproveitavam
avidamente as reduzidas fontes vegetais ricas em proteínas e gorduras
que os vários ecossistemas lhes forneciam, incluindo caules, fungos,
folhas e raízes comestíveis, além dos frutos.

Mas Fernão Cardim não se esquece de mencionar as espécies botânicas
e zoológicas que os portugueses transplantaram para o território
brasileiro. Dentro do hábito de difusão das espécies e dos alimentos
entre continentes, os portugueses transportaram para o Brasil animais e
plantas, estudando o seu comportamento, desde as hortaliças e árvores
frutíferas, até as culturas de grande extensão ou rendimento, cuja
exploração nem sempre obteve sucesso, como a dos cereais, mas que
outras vezes trouxe significativos resultados, como a da cana sacarina.

A colonização do Brasil provocou um dos mais amplos processos de
cruzamento intercontinental de espécies vegetais. Do reino e das ilhas,
os colonos portugueses levaram, além da cana sacarina e da videira,
árvores de frutos (figueiras, romãzeiras, laranjeiras, limoeiros,
cidreiras) e hortaliças (couves, alfaces, nabos, rabanetes, pepinos,
coentros, funcho, salsa, alhos, agrião, cenouras, berinjelas,
espinafres, entre outras). 

Mas difundiram"-se ainda plantas e árvores de origem asiática, como o
coqueiro, uma variedade de arroz (\textit{Orzya sativa)} e o melão, e de
origem africana, como a melancia, malagueta, inhame, tamareira,
feijão"-congo e quiabo.

A par da flora introduziram"-se ainda no território brasileiro
numerosos animais, que se salientaram pela sua importância econômica,
nomeadamente bois, vacas, cavalos, éguas, burros, porcos, carneiros,
ovelhas, cabras, galinhas, patos e perus, bem como o cão, por quem o
índio passou a ter grande estima, passando a ser de grande valia na
caça, já que era utilizado para perseguir os animais e forçá"-los a
abandonar os esconderijos.

Era outro Portugal que se estava a construir no Brasil, utilizando a
expressão de Fernão Cardim: ``Este Brasil é já outro Portugal''. 
É sem dúvida uma perspectiva de previsão do que viria a ser o Brasil.
Já Pêro Vaz de Caminha procurara transmitir ao rei D. Manuel, na sua
carta, a admiração pela nova terra descoberta: ``mas a
terra em si é de muito bons ares, [\ldots{}] em tal maneira é graciosa que,
querendo"-a aproveitar, dar"-se"-á nela tudo, por bem das águas que
tem''.\footnote{ Cf. Pêro Vaz de Caminha. \textit{Carta de
Achamento ao Rei Dom Manuel}, ed. modernizada por Luís de Albuquerque,
in \textit{O reconhecimento do Brasil}. Lisboa: Alfa, 1989, p. 25.} 

\section*{O «olhar» cardiniano dos povos ameríndios} 

 No segundo \textit{Tratado} com o título \textit{Do
princípio e origem dos índios do Brasil e de seus costumes, adoração e
cerimônias},\footnote{ Este texto que passaremos a designar apenas
com o título \textit{Do princípio e origem dos índios do Brasil}, 
encontra"-se incluído na obra já referida de Fernão Cardim,
\textit{Tratados da terra e gente do Brasil}, pp. 163--207.} e mesmo na
\textit{Narrativa epistolar}, o Padre Fernão Cardim procura dar a
conhecer os povos indígenas, especificando a variedade de nações e
línguas, os seus costumes, cerimônias, hábitos alimentares, habitação,
adornos, armas, instrumentos, além dos seus costumes, da forma
de receber os hóspedes, de educar os filhos, de tratar as mulheres, de
comunicar com os espíritos e do hábito de matar os prisioneiros e de
comer carne humana. Fá"-lo como um etnógrafo, quase que
cientificamente, sem emitir juízos de valor, sem criticar os indígenas
pelos seus hábitos, ainda que, por vezes, fossem tão diferentes dos
europeus. Detectamos, mesmo, ao longo das suas palavras, uma certa
sensibilidade, admiração e até estima, ao descrever as suas
características físicas, hábitos sociais e culturais.

Tema que, ainda hoje, é abordado na bibliografia sobre o Brasil
colonial, quer através do confronto com os povos que ainda hoje existem,
sobretudo no interior do Brasil, onde conseguiram refugiar"-se após a
chegada dos europeus, e que mantêm muitas das características
fisiológicas, socioeconômicas e culturais, quer através das informações
que chegaram até nós, como as do Padre Fernão Cardim, entre outros. No
entanto, estes testemunhos são muito diversos e por vezes divergentes
nos seus dados, escritos por homens de formação diferente, de índole
diversa e de atividades profissionais muito variadas. Trata"-se de
relatos de viagens e de correspondência entre missionários, cujos
autores escreveram de posições religiosas e perspectivas até
divergentes: há jesuítas ibéricos, como Manuel da Nóbrega, José de
Anchieta, Fernão Cardim e Francisco Soares, franciscanos e capuchinhos
franceses, como André Thevet, Claude d'Abbeville e Yves d'Évreux, o
huguenote Jean de Léry, o colono Gabriel Soares de Sousa, o artilheiro
alemão Hans Staden, o humanista Pêro de Magalhães de Gândavo, entre outros,
além das \textit{Cartas Jesuíticas.}

As abordagens descritivas sobre a terra e as gentes do Brasil
constituíram um rápido acréscimo dos dados humanos e científicos
anteriores, contribuindo para um melhor conhecimento das populações
indígenas\footnote{ Cf. Eugénio dos Santos, ``Índios e
Missionários no Brasil Quinhentista: do confronto à cooperação'', in
\textit{Revista da Faculdade de Letras}, \textsc{ii} série, Vol. \textsc{ix}, Porto,
1992, pp. 107--118.} ameríndias que, sendo ágrafas, não nos deixaram
testemunhos escritos sobre elas próprias e sobre a opinião que formaram
dos portugueses e dos outros europeus, homens tão diferentes pelo
vestuário, instrumentos, armas, alimentos, organização social, religião
e até aspecto físico. Além disso, os senhores vindos do mar
apresentavam"-se como donos da terra, queriam ocupar as regiões mais
férteis e saudáveis, exigiam que os servissem, mesmo usando a força e
sem guerra declarada, exibiam hábitos e costumes diferentes.

Ameríndios ou ``gentios'', como a eles se refere Cardim, eram
populações diferentes e com estágios civilizacionais diversificados e
que, mais tarde, vão ser protagonistas do mito do ``bom selvagem''.
Trata"-se de culturas que evoluíram à margem das chamadas ``grandes
civilizações'', quer sejam ocidentais, orientais, ou mesoamericanas.
Sociedades que desenvolveram, no entanto, um modelo cultural adequado
às características do ecossistema da floresta tropical, com técnicas 
de caça e pesca, e até agrícolas, ainda que continuem coletores, 
além de um diversificado conjunto de instrumentos, de uma
organização social e política muito específica e de um sistema de crenças.

É esta posição de compreensão da diferença, da diversidade cultural
do gênero humano e, consequentemente, respeito pelos povos ameríndios
que, de fato, encontramos nos textos do Padre Fernão Cardim. Este
começa, no seu tratado sobre os índios, por fazer uma
apresentação daqueles que denomina de ``gentio'', termo
usualmente utilizado pelos cronistas e viajantes da época, ao se
referirem aos povos que não eram nem cristãos, nem sarracenos ou
turcos, os tradicionais ``infiéis'', com quem os Seguidores da Cruz
lutaram para dilatar a sua fé. Cabia ao mouro, até então, o papel de
adversário maior porque era o único verdadeiramente concorrencial. Os
oponentes Cristão"-Mouros e Cristão"-Gentios filtram"-se
através dos interesses da civilização material mas, sobretudo, através
da civilização espiritual do cristianismo entendida como ideologia do
Império qual religião nacional que assinala a Portugal um
destino universal.\footnote{ Cf. Luís Filipe Barreto,
\textit{Descobrimentos e Renascimento}. 2ª ed. Lisboa: \textsc{in"-cm}, 1983, pp. 239--242.} 

 O ``outro'', sem ser o ``gentio'', era encarado de forma muito
radical pelo Homem de inícios do Quinhentos, antes do movimento
de Expansão Marítima, pondo"-se em causa se seriam humanos ou bestas,
como é o caso do Cardeal d'Ailly, na sua famosa \textit{Imago Mundi}, 
redigida por volta de 1410, arrimado à autoridade da Escritura, que 
considera que os trogloditas ``fazem as habitações nas cavernas, se
alimentam de serpentes, e não têm uso da fala''.\footnote{ Cf. Pierre 
d'Ailly, \textit{Imago Mundi}. Paris: 1930, vol 2º, pp. 362--363, cit. in J.S. da Silva Dias,
\textit{op. cit.}, p. 193.} Não deixou, ainda, de aludir à
existência, divulgada por Ptolomeu e vários autores antigos, de

\begin{hedraquote}
[\ldots{}] homens selvagens antropófagos, com feição disforme e horrível,
nas duas regiões extremas da Terra [\ldots{}], trata"-se de seres acerca dos
quais é difícil precisar se são homens ou bestas.\footnote{ \textit{Idem, ibidem}, 
vol 1, pp. 240--241, cit. \textit{Idem, ibidem}, p. 193.} 
\end{hedraquote}

Mas estes ``gentios'' evidenciam, no entanto, uma
abertura que indicia, por comparação com mouros e judeus, a maior
facilidade da sua conversão.\footnote{ Cf. José da Silva Horta, 
``A imagem do africano pelos portugueses antes dos contatos'', in
\textit{O confronto do olhar. O encontro de povos na época das
navegações portuguesas. Séculos \textsc{xv} e \textsc{xvi}}, coord. Antonio Luís Ferronha,
Lisboa, Caminho, 1991, pp. 43--64.} O encontro do português com o índio
dá"-se no quadro das regras históricas de distribuição do 
poder.\footnote{ Cf. Luís Filipe Barreto, \textit{op. cit.}, p. 178.}

O Padre Fernão Cardim apresenta"-nos o índio brasileiro como ágrafo,
isto é, sem conhecimento de escrita, o que nos impede de conhecermos as
suas ideias, os seus pensamentos. Mostra"-o ainda sem conhecimentos da
origem do Mundo. Não é de estranhar que Cardim, sendo um homem de
Quinhentos, considere que os índios não tinham, de acordo com os parâmetros
da época, uma religião. Mas através da sua descrição testemunha"-se a
existência de crenças em uma religião natural, ao afirmar que sabem que
têm alma e que esta não morre. Por oposição ao espírito mau,
Curupira, Cardim menciona um espírito bom, Tupã, que
significa em tupi ``pai que está no alto'', a quem os Tupis atribuíam a
origem dos trovões e dos relâmpagos, e, mais tarde, os missionários
estendem esta denominação ao conceito de Deus:

\begin{hedraquote}
Não têm nome próprio com que expliquem a Deus, mas dizem que
Tupã é o que faz os trovões e relâmpagos, e que este é o que lhes deu
as enxadas, e mantimentos, e por não terem outro nome mais próprio e
natural, chamam a Deus Tupã.\footnote{ \textit{Idem, ibidem}, p. 88.} 
\end{hedraquote}

Já o Padre Manuel da Nóbrega atribuíra a Tupã as mesmas qualidades,
ao afirmar: ``gentilidade a nenhuma coisa adora, nem
conhece a Deus, somente aos trovões chama Tupã, que é como quem diz
coisa divina''.\footnote{ Cf. Manuel da Nóbrega, ``Informação das
Terras do Brasil'', in \textit{Cartas do Brasil e mais escritos}, org.
de Serafim Leite, Coimbra, Ed. da Universidade de Coimbra, 1955, p.
62.} Desde a carta de Pêro Vaz de Caminha que esta era a
visão do ameríndio:

\begin{hedraquote}
E segundo o que a mim e a todos
pareceu, esta gente não lhes falece outra coisa para ser cristã que
entenderem"-nos, porque assim tornavam aquilo que nos viam fazer como a
nós mesmos, por onde pareceu a todos que nenhuma idolatria nem adoração
têm.\footnote{ Cf. Pêro Vaz de Caminha, \textit{op. cit.}, p. 24.}
\end{hedraquote}

Mais tarde Pêro de Magalhães de Gândavo assume uma posição mais radical,
ao escrever na sua \textit{História da província de Santa Cruz}:

\begin{hedraquote} 
[\ldots{}] A língua que usam, carece de três letras, convém
a saber, não se acha nela f, nem l, nem r, coisa digna de espanto, 
porque assim não têm fé, nem lei, nem rei [\ldots{}]. 
Não adoram a coisa alguma, nem têm para si que há depois da
morte glória para os bons e pena para os maus. E o que sentem da
imortalidade da alma não é mais que terem para si que seus defuntos
andam na outra vida feridos, despedaçados, ou de qualquer maneira que
acabaram nesta.\footnote{ Cf. Pêro de Magalhães de Gândavo,
\textit{História da Província de Santa Cruz a quem vulgarmente chamamos
Brasil}, comentário final de Jorge Couto, in \textit{O Reconhecimento do Brasil}, 
dir.~de Luís de Albuquerque, Lisboa, Alfa, 1989, p. 102.} 
\end{hedraquote}

 Mas os testemunhos de Nóbrega e Cardim sobre o conceito de
Tupã e as suas atribuições não colhem o acordo de muitos
etnólogos modernos, como Alfred Métraux, que veem nele não a figura
principal da mitologia indígena, mas uma personagem secundária,
atribuindo essa primazia a \textit{Monan.}\footnote{ Cf. Alfred
Métraux, \textit{A Religião dos Tupinambás e suas relações com as
demais tribos tupi"-guaranis}, trad. de Estevão Pinto, 2ª ed., São
Paulo, Companhia Editora Nacional, 1950, pp. 45--55.} Segundo uma
nova reinterpretação, esta questão mitológica foi de novo alterada,
considerando que os povos do tronco Tupi não privilegiavam nenhuma
destas duas divindades, reconhecendo apenas a Tupã a função de
destruição do mundo, pelos trovões e relâmpagos, enquanto que a Monan
destinavam a função de criação.\footnote{ Cf. Hélène Clastres,
\textit{La terre sans mal. Le Prophétisme Tupi"-Guarani}, Paris, Seuil,
1975, pp. 27--34.} 

No entanto, Cardim torna"-se mais crítico quanto à atuação dos pajés
e caraíbas e das práticas mágicas, o que se entende perante o seu papel
de homem religioso e com todo o interesse em converter os indígenas,
sendo"-lhe difícil aceitar algumas das suas práticas. Estas personagens,
consideradas predestinadas para exercer atividades mágico"-religiosas,
desempenhavam um papel fulcral nas sociedades ameríndias. 

 Os pajés ou xamãs eram os principais detentores dos meios de controlar
os fenômenos sobrenaturais, incertos e perigosos. Eram respeitados e
temidos enquanto vivos e, em alguns grupos tribais, sobretudo Guaranis,
venerados depois de mortos. Desempenhavam tarefas vitais para a sua
vida material e espiritual, como praticar esconjuros, purificar os
alimentos, garantir a segurança dos caçadores, interpretar os sonhos e
dirigir cerimônias de caráter coletivo. Exerciam ainda a função de
curandeiros e presidiam às danças rituais. A essa tarefa de curandeiros
se refere Cardim: 

\begin{hedraquote}
[\ldots{}] Usam de alguns feitiços e feiticeiros, não porque
creiam neles, nem os adorem, mas somente se dão a chupar em suas
enfermidades parecendo"-lhes que receberão saúde, mas não por lhes
parecer que há neles divindades, e mais o fazem por receber saúde que
por outro algum respeito.\footnote{ Cf. Fernão Cardim, \textit{op. cit.}, p. 166.}
\end{hedraquote}

 No entanto, a tarefa primordial desse personagem consistia em
comunicar com os espíritos através do canto e da dança, assegurando,
desse modo, a mediação entre a sociedade e o mundo sobrenatural. Devido
a estas funções, o pajé dispunha de uma posição privilegiada no seio da
comunidade, mantendo, fora do exercício das suas práticas, apenas
contato com os principais, que o tratavam com reverência, cultivando
uma imagem de gravidade e inacessibilidade. Vivia mesmo isolado numa
cabana, afastada das tabas, não ousando ninguém lá penetrar.

Por sua vez, caraíbas parece ser um termo originário dos
idiomas caribe e aruaque, em que significa ``homem valente'', que foi
introduzido na língua tupi com o sentido de ``homem sagrado'', ``homem
santo'', como testemunha Cardim: ``Entre eles se
alevantaram algumas vezes alguns feiticeiros, a que chamam \textit{Caraíba},
\textit{Santo} ou \textit{Santidade}, e é de ordinário algum índio de ruim 
vida''\footnote{ \textit{Idem, ibidem}, pp. 166--167 .}. Eram profetas
itinerantes que gozavam de grande prestígio e despertavam algum temor.
Visitavam as aldeias cada três a quatro anos e as comunidades
procuravam agradar"-lhes e evitar, a todo o custo, a sua inimizade,
recebendo"-os com grandes festividades, dando"-lhes os melhores alimentos
ou até cedendo"-lhes as suas mulheres. Eram, no entanto, figuras
ambíguas, pois tanto podiam trazer saúde, curando as enfermidades ou
fornecendo alimentos e cativos, como espalhar a doença, a fome, a
derrota, a desgraça e a morte.\footnote{ Sobre o papel
desempenhado por estes personagens, os \textit{pajés} ou \textit{xamãs}
e os \textit{caraíbas}, nas sociedades ameríndias, veja"-se Egon
Schaden, \textit{A Mitologia Heroica de Tribos Indígenas do Brasil.
Ensaio Etno"-sociológico}, 3ª ed., São Paulo, 1989, pp. 117--119; Alfred
Métraux, \textit{A Religião dos Tupinambás e suas Relações com a das
demais tribos tupi"-guaranis}, prefácio, tradução e notas de Estevão
Pinto e apresentação de Egon Schaden, 2ªed., São Paulo, 1979 (1928),
pp. 65--79; Claude Lévi"-Strauss, \textit{Tristes Trópicos}, trad. port.,
Lisboa, Edições 70, 1981, p. 224 e Jorge Couto, \textit{op. cit.}, pp. 109--117.} 

 Os mais conceituados caraíbas eram considerados pelos
Tupi"-guarani como reencarnações dos seus heróis míticos e afirmavam
ter poderes para transformar homens em pássaros ou em outros animais,
para ressuscitar os mortos, fazer nascer plantas, visitar as Terras
dos mortos, ou mesmo entrar em contato com os espíritos. Por vezes,
estes ``homens"-deuses'', na expressão de Alfred Métraux, despertavam
surtos de profetismo, incitando as populações a segui"-los em busca de
uma mítica Terra sem Mal que se podia alcançar em vida. Seria uma
espécie de paraíso terrestre, um lugar de abundância, de ausência de
trabalho, de recusa dos princípios fundamentais da organização social,
de juventude perpétua e de imortalidade. Esta procura da Terra sem Mal
pretenderia, essencialmente, superar as dificuldades de toda a ordem
por que passavam, em determinadas conjunturas, as comunidades indígenas,
não tendo por finalidade negar a condição humana ou abolir a
organização social.\footnote{ Esta questão da procura da Terra sem
Mal, em guarani \textit{yuy marane'y}, pelas populações tupi"-guarani, 
tem sido sujeita a várias interpretações. Veja"-se sobre este tema:
Hélène Clastres, \textit{La Terre sans Mal. Le Prophétisme
Tupi"-Guarani}, Paris, 1975, pp. 61--18 e 141--145; Bartolomeu Melià,
\textit{Ñande Reko, nuestro modo de ser y bibliografia geral comentada}, 
La Paz, 1988, pp. 47--51; \textit{Idem}, ``La Tierra"-Sin"-Mal de los Guarani:
Economia y Profecia'', in \textit{América Indígena} (Cidade do México),
\textsc{xlix} (3), 1989, pp. 491--507 e Jorge Couto, \textit{op. cit.}, pp. 109--117.} 

 Por outro lado, Cardim não questiona a inexistência de um
chefe ou autoridade, e várias vezes nos seus textos aparecem
referências ao principal, que denomina de Morubixaba,
considerando que era aquele que tinha mais poder dentro da aldeia:
``Nesta casa mora um principal, ou mais, a que todos
obedecem''.\footnote{ Cf. Fernão Cardim, \textit{op. cit.}, p. 171--172.}

Mas o nosso autor mostra mais interesse em fazer uma apresentação
cuidada dos índios, dos seus hábitos, costumes e cerimônias, ao mesmo
tempo que se preocupa em apresentar a sua cultura material, ou seja,
instrumentos, adornos, habitação e confecção de alimentos. Vai ainda
pondo em causa as suas ligações familiares, como o casamento, a
família, filhos, atendendo à falta de conversão e dos respectivos
sacramentos. Preocupa"-se, assim, com o casamento entre os índios, que
confirma existir, mas que questiona atendendo à poligamia e à
facilidade com que deixavam as mulheres.

A questão da poliginia, ou seja, o casamento de um homem com
várias mulheres, a que Cardim se refere, não era, no entanto, uma
situação costumeira entre os grupos tribais indígenas. Apenas um
reduzido número de membros, os mais poderosos, como o chefe, o pajé e
os grandes guerreiros, é que tinham possibilidade de possuir mais de uma
mulher. O número de mulheres marcava o prestígio e importância do
marido polígamo, principalmente sendo muitas vezes as próprias mulheres
que procuravam mulheres mais jovens para os maridos, chamando"-lhes
``filhas'' ou ``sobrinhas'', ao mesmo tempo que mostrava a existência dos
casamentos monogâmicos, ao qual se acrescentam relações de natureza
extraconjugal. Os condicionalismos sociais tornavam a monogamia mais
usual do que a poligamia, sendo as separações usuais, como refere
Cardim.\footnote{ Cf. Florestan Fernandes, \textit{A organização
social dos tupinambás}, 2ª ed., São Paulo/Brasília, 1989, p. 203.}

Apesar desta possibilidade de prática de poliginia, os índios tinham
por hábito tratar bem as mulheres, segundo o testemunho deste autor,
com uma certa admiração e respeito pela forma como os homens as
protegiam. Por exemplo, quando estavam fora da aldeia, as mulheres
caminhavam atrás para que, se caíssem numa cilada, pudessem fugir,
enquanto eles lutavam; no regresso à taba, a estratégia era inversa,
porque se fossem atacados a mulher poderia fugir e pedir auxílio.
Dentro da aldeia ou em terra segura a mulher caminhava sempre à frente
do marido para que este a pudesse ver, pois como afirma Cardim:
``são ciosos e querem sempre ver a 
mulher''.\footnote{ Cf. Fernão Cardim, \textit{op. cit.}, p. 178.} 

Outro dos aspectos que este jesuíta admira nos hábitos dos povos
ameríndios é a forma como estimavam e educavam os filhos. Afirma mesmo
que: ``amam os filhos extraordinariamente. [\ldots{}] Estimam
mais fazerem bem aos filhos que a si próprios''.\footnote{ \textit{Idem, ibidem}, 
p. 173.} Situação que não era muito usual na
Europa Quinhentista, onde a criança vivia integrada no mundo dos
adultos, sendo mais comum o afastamento afetivo dos pais em relação à
criança, entregue às amas"-de"-leite, enquanto que as índias amamentavam
os seus filhos cerca de ano a ano e meio, e andavam sempre com eles,
amarrados às costas ou ao quadril. Esta prática funcionava também como
forma de controle de natalidade. Cardim atribui mesmo a esta
afetividade entre as mães e os filhos o temperamento dos meninos
índios, os \textit{curumis}, que eram alegres e amigos uns dos outros,
não pelejando, nem pronunciando nomes ofensivos entre eles, nem em
relação a seus pais. Esta afeição dos meninos índios pelos seus pais
transpôs"-se para os jesuítas, a quem estimavam muito, na opinião de
Cardim: ``estimam muito e amam os padres, porque lhos
criam e ensinam a ler, escrever, contar e tanger, cousas que eles muito
estimam.''\footnote{ \textit{Idem, ibidem}, pp. 171--172.} 

 O Padre Fernão Cardim dedica ainda parte do seu Tratado 
à habitação e à maneira de dormir. Descreve as \textit{ocas}, que diz
serem ``casas de madeira cobertas de folhas'', onde
viviam cerca de duzentas e mais pessoas. Eram grandes habitações
comunitárias com duas ou três aberturas muito baixas e pequenas, sem
janelas. Dentro das ocas era um labirinto, com vários lanços, 
onde colocavam as redes de dormir, objetos pessoais (adornos, armas, amuletos), além dos
utensílios familiares (cestos, potes, alfaias agrícolas, entre outros)
e as reservas alimentares, mantendo sempre o fogo aceso dia e noite
para se agasalharem e se protegerem dos animais, principalmente cobras,
e ainda para afugentarem os espíritos. 

 Na apresentação do ``gentio'', descreve ``o modo
que têm de se vestir'', ou melhor de ``não se vestirem'', já que
tinham por hábito andarem nus. Um dos aspectos que deve ter
impressionado mais os homens do Quinhentos, humanistas, viajantes ou
não, foi o contato com a nudez dos ameríndios, atendendo aos valores
morais da época. Elogia"-lhes ainda os cabelos, que são lisos, negros e
bem tratados. Para o europeu, o corpo humano era o instrumento direto
do pecado e, como tal, devia de ser o mais coberto possível, castigado
e escondido, até mesmo os cabelos para não atraírem sexualmente os
homens.\footnote{ Cf. \textit{História da Vida Privada}, vol. \textsc{ii},
dir.~Philippe Ariès. Lisboa: Afrontamento, 
1990, pp. 362--363.} Enquanto que, para o ameríndio, ele servia de uma forma de
contato direto com a natureza, uma forma de comunhão com os outros
seres vivos e até de vivência em comunidade. Outro fator que
evidentemente permitia essa nudez eram as condições climáticas. O que
transparece do texto cardiniano e nos da maioria dos escritores
contemporâneos não é de crítica, mas sim de espanto por este hábito.

 Desde a carta de Pêro Vaz de Caminha que esta
admiração era sentida, ao afirmar:

\begin{hedraquote}
A feição deles é serem
pardos (à) maneira de avermelhados, de bons rostos e bons narizes
bem feitos; andam nus sem nenhuma cobertura, nem estimam nenhuma coisa 
cobrir nem mostrar suas vergonhas, e estão cerca disso em tanta inocência 
como estão em mostrar o rosto.\footnote{ Cf. Pêro Vaz de Caminha, \textit{op. cit.}, p. 11.} 
\end{hedraquote}

 Ainda que seja idêntica a visão dos índios, medeiam entre estes dois
testemunhos cerca de oitenta anos. De fato, o Padre Fernão Cardim
contatou com grupos de povos ameríndios, Macro"-Tupi,\footnote{ Macro"-Tupi é 
um dos troncos ameríndios constituído por sete famílias
(Tupi"-Guarani, Mundurucu, Juruna, Ariquém, Tupari, Ramarama e Mondé)
que se dividem em vários grupos ou línguas e subgrupos ou
dialetos.} que já contavam com cerca de trinta anos de
catequese sistemática, iniciada no governo de Tomé de Sousa, pelo Padre
Manuel da Nóbrega. Índios a quem os jesuítas tinham procurado
transmitir alguns hábitos mais moderados, sobretudo no que concerne à
antropofagia, poligamia e consumo de bebidas fermentadas. 

O indígena ``cardiniano'' é apresentado, na maior parte das vezes, de
uma forma simpática e respeitosa. É descrito como alegre, dócil,
afável, melancólico por vezes, com gosto pela música, dotado de ritmo
e com capacidades de bom dançarino e cantor. Vaidoso nos seus adornos e
armas, gostava de se enfeitar e de utilizar cores garridas. Apreciava
as mulheres, mas respeitava"-as e procurava protegê"-las. Mostra"-se
admirado com as suas capacidades físicas, apodando"-os de corajosos,
resistentes à dor e às intempéries, ágeis e valentes, com boa visão e
audição. Qualidades que faziam dos índios bons caçadores, nadadores e pescadores.

A capacidade de sociabilidade é também apreciada por Cardim, que
elogia a forma como recebiam os hóspedes, distribuindo alimentos entre
eles e saudando"-os com um choro profundo (saudação lacrimosa) e o
respeito que mostravam pelos objetos pessoais e mulheres alheios. Foi
exatamente com a saudação de \textit{Ereiupe} (Vieste?), que este
jesuíta foi recebido nas aldeias, ao longo da sua viagem, como
secretário do padre visitador Cristóvão de Gouveia. Capacidades a que
Cardim acrescenta ainda, com admiração, a de amizade e respeito pelos
portugueses, por quem eram capazes de combater contra outros índios:

\begin{hedraquote}
[\ldots{}] estes foram e são os amigos dos portugueses, com ajuda e
armas, conquistaram esta terra, pelejando contra seus próprios parentes
e outras diversas nações.\footnote{ Cf. Fernão Cardim, \textit{op. cit.}, p. 192.}
\end{hedraquote}

 Mesmo quando o Padre Fernão Cardim descreve os seus
hábitos que podem ser considerados menos adequados, ou mesmo
inadequados, fá"-lo de uma forma não cáustica. É o caso do consumo
exagerado de bebidas fermentadas (o cauim), que tomavam nas
suas celebrações e noutras ocasiões festivas, embriagando"-se, e nos
rituais de antropofagia. Descreve"-os não de um modo depreciativo ou
mesmo crítico, mas sim de uma forma objetiva, quase desprovida de
juízos valorativos, chegando mesmo a adotar uma posição quase irônica.
Talvez porque Cardim nunca tenha posto em causa, e isso é também
evidente ao longo dos seus textos, a condição humana dos ameríndios,
considerando"-os apenas como homens diferentes nos seus hábitos. Não é
o olhar do ``civilizado'' perante o ``selvagem''.\footnote{ Entenda"-se, 
como \textit{selvagem}, o conceito etnológico do habitante
da selva, sem qualquer conotação depreciativa do seu estádio de
desenvolvimento e sem ainda o conceito ideológico do ``Bom Selvagem'',
o homem mítico, em estado de natureza incontaminada, ainda senhor dos
seus instrumentos, como vai ser considerado a partir do século \textsc{xvi} e
sobretudo pelos iluministas do Setecentos.} Apresenta, assim, a 
prática antropofágica como um gosto, um cerimonial que lhes dá prestígio, que é a
homenagem dos conquistadores, dos melhores guerreiros, dos vencedores,
perante os que foram vencidos. Mas mesmo para estes é uma morte
respeitada, mais valiosa do que qualquer outra, e que será vingada
pelos seus semelhantes. Era uma honra morrer como um guerreiro, era uma
passagem para o Além de uma forma gloriosa e evidenciava uma
personalidade forte por parte do indígena que sabia que ia morrer como um valente.

 A antropofagia, ou seja, o hábito de comer carne humana sob várias
modalidades, verificou"-se entre quase todos os povos ameríndios, com
maior destaque para os tupis. Esta prática revestia"-se de caráter
exclusivamente ritual, ainda que recentemente seja também encarada de
outra forma pela historiografia.\footnote{ Vide a tese de
doutoramento de Mário Maestri, \textit{A Terra dos Males sem Fim.
Agonia Tupinambá no Litoral Brasileiro, Século \textsc{xvi}}, Porto
Alegre"-Bruxelas, 1990--1991.} As notícias fornecidas pelos textos
quinhentistas e seiscentistas relatam a sua importância na organização
social indígena, como fator indispensável aos ritos de nominação e
iniciação. Os autores quinhentistas e seiscentistas, jesuítas ou não,
procuraram distinguir nos seus textos o canibalismo alimentar, em que
era evidente o gosto pelo consumo de carne humana, praticado sobretudo
por grupos tribais caribes, aruaques, jês e outros, e a antropofagia
ritual, nobre, movida exclusivamente pelo desejo de vingança que
ocorria entre os povos tupis.\footnote{ Cf. Manuela Carneiro da
Cunha, ``Imagens dos índios do Brasil: o Século \textsc{xvi}'', in
\textit{Estudos Avançados}-\textsc{usp}, São Paulo, 4 (10), 1990, p. 108.} 

Para os etnólogos e antropólogos antigos e atuais, a antropofagia
praticada pelos tupis estava marcada por uma função exclusivamente
ritual destinada a comemorar os ancestrais míticos e os antepassados
memoráveis, bem como a vingar os membros da aldeia recentemente mortos
em combate, auxiliando os seus espíritos a alcançar o 
sobrenatural.\footnote{ Cf. Florestan Fernandes, \textit{A função social da
guerra na sociedade tupinambá}, São Paulo, Museu Paulista, 1952, pp. 316--349.} 
Por sua vez, acreditam ainda que a questão do prestígio que a
captura e o sacrifício do prisioneiro traziam para o guerreiro era
muito importante dentro da hierarquia social, já que lhes permitia
adquirir mais nomes e, consequentemente, mais prestígio. A morte no
terreiro era também muito importante para o cativo, que a preferiria à
morte por doença, a apodrecer e a ser comido pelos bichos. Era uma
morte digna e gloriosa, não por ser heroica, pois tinha sido capturado,
mas preferível à morte natural. O exocanibalismo tupi dependia de um
conjunto de crenças escatológicas, nomeadamente o horror ao
enterramento do corpo e à podridão.\footnote{ Cf. Eduardo
Viveiros de Castro, \textit{Arawetê: os deuses canibais}, Rio de
Janeiro, Jorge Zahar Editor, 1986, pp. 596--696.}

Esta interpretação do ritual da antropofagia, que se baseia
essencialmente em concepções metafísicas, realçando a morte heroica em
idade ativa e desvalorizando o envelhecimento e a morte natural, tem
sido posta em causa por outros antropólogos e etnólogos, sobretudo de
influência materialista, que consideram que existe uma certa
contradição, já que os velhos, sendo os mais importantes e de maior
prestígio em um grupo tribal, morriam de morte natural e não era por
esse motivo que perdiam a sua importância.

Ainda uma outra corrente, de arqueólogos e historiadores, defende que
a ingestão de carne humana pelos tupis teria a ver com a necessidade de
comer carne, abastecendo"-se de proteínas de que tinham necessidade
sobretudo quando estavam em viagem, não necessitando assim de caçar ou
pescar.\footnote{ Cf. José Proenza Brochado, ``A expansão dos tupi e
da cerâmica de tradição policrômica amazônica'', in \textit{Dédalo}, 27,
São Paulo, 1989, p. 59.} Outro estudo recente conclui que a excessiva
concentração no litoral terá provocado esgotamento da caça, provocando
carências alimentares e de proteínas, que geraram o canibalismo, como
uma necessidade de suplemento alimentar.\footnote{ Cf. Mário
Maestri, \textit{op. cit.}, pp. 44--55.}

Dificilmente podemos pôr de parte estas teorias de que os ameríndios
ao consumirem carne humana revestiam"-lhe uma função alimentar, que
vinha em parte substituir a escassez de proteínas e até de gorduras e a
falta de animais de grande porte, até pelos testemunhos dos escritores
antigos que assistiram a esses rituais. No entanto, não se pode também
ignorar, nem muito menos menosprezar, a ideia de vingança e da
superioridade dos guerreiros vencedores sobre o grupo tribal derrotado.

De fato, perante os testemunhos de Cardim e de outros autores
contemporâneos compreendemos que as sociedades indígenas eram
estruturadas em função da guerra, os grupos tribais desenvolveram uma
escala de estratificação social em que o valor e importância se baseava
fundamentalmente na capacidade de perseguir e matar o maior número
possível de inimigos. O próprio Fernão Cardim menciona este fato, no
início do texto ``Do modo que este gentio tem
acerca de matar e comer carne humana'', em que afirma:

\begin{hedraquote}
De todas as honras e gostos da vida, nenhum é tamanho para este gentio
como matar e tomar nomes nas cabeças de seu contrário.\footnote{ Cf. Fernão 
Cardim, \textit{op. cit.}, p. 182.} 
\end{hedraquote}

Através do texto cardiniano podemos acompanhar ``passo a passo'' todo
o ritual da preparação e morte do prisioneiro, capturado e da prática
da antropofagia. O pormenor e o rigor da descrição levam"-nos a
considerar que este deve ter assistido a um destes cerimoniais, e que a
sua descrição não é apenas fruto de informações que lhe transmitiram.
Descrições densas, brilhantes, marcadas por uma grande compreensão,
dão"-nos uma visão viva, profunda e até impressionante do comportamento
e da mentalidade dos rituais de antropofagia dos indígenas. Semelhante
descrição é feita pelo artilheiro alemão Hans Staden, que esteve
prisioneiro dos tupinambás, durante cerca de nove meses, em 1554, e que
esteve prestes a ser devorado.\footnote{ Vide Hans Staden,
\textit{Duas viagens ao Brasil}, Belo Horizonte/São Paulo, Itatiaia/Ed\textsc{usp}, 1974.} 

Questão pertinente à prática da antropofagia, a par de outras, como a
poligamia, a falta de autoridade política (sem rei), a falta de uma
religião orgânica (sem deus a quem adorar), a sua rudeza mental,
``com um anzol os converto, com outros os desconverto; a tudo
dizem sim''\footnote{ Cf. Manuel da Nóbrega, \textit{Cartas do
Brasil e mais escritos}, com introdução e notas históricas e críticas
de Serafim Leite, Coimbra, Universidade de Coimbra, 1955, pp. 215--218.}
e a atração pelo mato, que influía na estabilidade dos índios e na
possibilidade e validade da sua conversão. 

Se, em princípio, a situação de \textit{tanquam tabula rasa},\footnote{ Esta 
expressão é referida no manuscrito existente
na Biblioteca Pública e Arquivo Distrital de Évora, de 31 de dezembro
de 1583, \textit{ms.} nº \textsc{cxvi}, fl. 44, incluído numa Miscelânea de
manuscritos, com o título genérico de \textit{Cousas do Brasil}, que
tem sido atribuído ao Padre José de Anchieta, mas que tem sido
levantada a possibilidade de ser do Padre Fernão Cardim. Vide a nossa
dissertação de Mestrado já citada, pp. 161--181.} atribuída
aos povos ameríndios, parecia favorável à sua catequização, ela veio a
revelar"-se, pelo contrário, negativa, pois se por acaso existisse um
rei, e este se convertesse, a população seguiria o seu exemplo. Se
houvesse uma religião para confrontar, seria de o fazer, para mostrar a
força do cristianismo. Assim, este condicionalismo veio marcar uma das
características da conversão do gentio, que devia ser individual, índio
a índio, ou melhor, como logo perceberam os jesuítas, de menino índio
a menino índio, e destes aos índios adultos. A aculturação tinha que
ser conseguida individualmente, procurando destruir em cada um o seu
pendor multissecular da sua própria cultura adepta da antropofagia,
poligamia, consumo de bebidas em excesso, e ainda adepta de um
nomadismo intermitente. Situação muito diferente da encontrada no
Oriente, nomeadamente na Índia, na China ou no Japão, onde os
portugueses se confrontaram com povos com estruturas
político"-administrativas e religiões milenares. 

Fernão Cardim, como cristão e evidentemente como membro do clero,
apresenta através dos seus textos esta situação, sobretudo quando
refere a facilidade que os índios tinham em se converter e em
solicitarem o seu batismo e casamento segundo os ritos cristãos.
Casamentos, segundo a Igreja Católica, mas com grande dificuldade
cumpridos nos preceitos dos Mandamentos, na medida em que o casamento e
a forma de o encarar eram muito diferentes segundo as tribos ameríndias.
Poligamia e adultério eram duas constantes da vida dos índios. Situação
que se complicava depois que tomavam o batismo, já que o choque de
mentalidades era evidente, atendendo às condições de vida e aos hábitos
milenares. Casamento pela lei da natureza, por razões sociais, ou
casamento pela lei da graça? Questão pertinente para homens de
Quinhentos ou Seiscentos que sentiam esta enorme contradição, perante
a mentalidade e os dogmas da Igreja.

Para solucionar esta questão e a dificuldade dos índios de se
tornarem monogâmicos, na sua totalidade, os padres jesuítas começaram a
encarar o casamento de outra forma: um homem e uma mulher, vivendo em
comunhão com os seus filhos, há vários anos, e tendo eles mais de trinta anos, 
poderiam ser considerados como ``vivendo em matrimônio''.\footnote{ Cf. \textit{Informação
dos casamentos dos índios}, pelo Pe. Franciso Pinto, in Códice
\textsc{cxvi}/1--33, fl. 131--134, in \textit{Cousas do Brasil}, Biblioteca
Pública e Arquivo Distrital de Évora.} 

Por outro lado, o exemplo que era dado por alguns cristãos, até
mesmo membros do clero secular, que se amancebavam com mulheres
indígenas, não permitia aos índios compreender o objetivo do
sacramento do matrimônio. Papel importante, digamos mesmo fundamental,
dos padres da Companhia de Jesus e outros missionários, que tinham de
superar estas contradições morais e sociais. Cardim lastima esta
situação, entre outras que se viviam no território brasileiro de
finais do Quinhentos, e alerta mesmo o monarca português para a
inexistência e falta de preparação dos poucos curas que havia,
afirmando:

\begin{hedraquote}
Toda a costa do Brasil está carecida e destituída de padres
que pudessem ser curas para as almas, [\ldots{}] Não existem padres que
saibam cumprir o seu ofício, e seus costumes e vidas são muito
difamados, porque vieram de Portugal suspensos de ordens por crimes
graves e também por incorrigíveis; e aqui como faltam sacerdotes são
logo feitos curas de almas. [\ldots{}] São muito ignorantes. [\ldots{}] E suas
vidas são dignas de misericórdia, porque enfim o cuidado deles é
encher"-se de dinheiro, alimento, bebida e seguir os vícios da carne,
com grande escândalo dos leigos.\footnote{ Cf. Fernão Cardim,
\textit{Artigos}, \textit{op. cit.}, pp. 455--482.}
\end{hedraquote}

 De forma diferente de atuar, como em relação aos povos
ameríndios, a quem Cardim mostrava compreensão, aqui a sua crítica
pode mesmo ser considerada ``dura'', perante a atuação do clero
secular, aproveitando para elogiar o comportamento e o papel dos
jesuítas, que não era de simples missionários ou professores. A sua
crítica insere"-se no contexto de contestação e de alteração que a
própria Igreja Católica e o clero secular estavam a ser sujeitos.

Mas o olhar do Padre Fernão Cardim não fica apenas pela questão do
valor e da moralidade do casamento dos índios. Preocupa"-se, ainda, com
a situação que viviam alguns, que eram utilizados pelos senhores dos
engenhos como mão"-de"-obra escrava, dificultando a atuação dos padres
da Companhia e de outros membros do clero.\footnote{ \textit{Idem, ibidem}, 
pp. 455--482.} Situação de conflito que já se vinha a
sentir desde o tempo do Padre Manuel da Nóbrega, já que os jesuítas
procuravam diligenciar no sentido de conseguir a liberdade dos índios e
o seu reconhecimento como ``homens livres''. Atitude esta, dos inacianos,
que granjeou entre os índios um grande prestígio, sentindo que existia
entre os portugueses quem os defendesse. 

Durante a permanência do visitador Cristóvão de Gouveia, de quem
Cardim foi secretário, a questão da conversão e escravidão dos povos
ameríndios foi analisada e tornou"-se polêmica. O padre visitador,
depois de muita insistência para Lisboa, conseguiu que, por uma lei de
24 de fevereiro de 1587, Filipe \textsc{ii} restringisse o critério de guerra
justa, e organizasse as entradas com a presença dos inacianos, e com
autorização do governador, e ainda regulamentasse a repartição de
índios, ``persuadidos'' a virem à costa para trabalhar nos engenhos e
fazendas.\footnote{ Cf. Arquivo Histórico Ultramarino,
\textit{Registos}, \textsc{i}, 45--47v., cit. in Serafim Leite, \textit{História
da Companhia de Jesus no Brasil}, tomo \textsc{ii}, liv. \textsc{ii}, p. 211.}

O interesse de Cardim pelos povos ameríndios com quem contatou logo
no início da sua chegada a terras brasileiras e que vai descrevendo, ao
mesmo tempo que vai desenvolvendo uma atuação política pela sua
defesa, vai mesmo ao ponto de apresentar, no \textit{Tratado} dedicado
aos índios, uma exaustiva enumeração de várias nações, cerca de 104, 
algumas delas não mencionadas em outros textos quinhentistas e
seiscentistas. Agrupa"-as em dois grandes grupos: os tupis e os
tapuias, com base na língua utilizada, o que levava os
primeiros a considerar os segundos como ``bárbaros'', já que os tupis
falavam a denominada ``língua geral'', e os tapuias uma língua
completamente diferente, de grande dificuldade de
compreensão.\footnote{ Cf. Manuel Diegues Júnior, \textit{Etnias e
Culturas do Brasil}, 3ª ed., Rio de Janeiro, Ed. Letras e Artes, 1963, p. 16.}

A questão da língua é uma das preocupações de Cardim, assim
como de outros membros da Companhia de Jesus, para se poderem entender
e fazer"-se compreender pelos índios, diligenciando na sua conversão, 
uma das questões com que os jesuítas se debateram no Brasil, e muito
concretamente Fernão Cardim. É que este conseguia vislumbrar nos
índios, para além do corpo, uma alma, igual à de todos, que era preciso
catequizar e remir. Considerava"-os como homens, que era possível
aperfeiçoar e educar, mas que estavam prontos a aceitar essa nova
religião, procedendo como verdadeiros cristãos, assistindo à missa com
autêntico fervor, por vezes, na sua opinião crítica, superior ao de
muitos colonos cristãos.

 Este jesuíta evidencia ao longo dos seus textos uma visão diferente
em alguns aspectos da de outros autores contemporâneos. Notam"-se, de
fato, duas opiniões diferentes nos textos que fazem uma descrição dos
povos ameríndios. Para alguns, o índio é um ser inócuo, inocente e
pacífico por natureza, que só quando exacerbado pelas provocações dos
europeus se tornava agressivo, cruel e vingativo. Mas, para outros, ele
é cruel e perigoso por natureza e por isso defendem a posição dos
europeus de lhe moverem guerras. Posições, em alguns casos, extremistas já que os índios possuíam,
como todos os outros homens, qualidades boas e más. Vivendo errantes e
em estado quase que ``primitivo'',\footnote{ Considere"-se como
\textit{primitivo}, no conceito da antropologia cultural, os membros 
das etno"-sociedades, dos povos estudados pelos etnógrafos, mas
sem qualquer cunho depreciativo, como aconteceu durante muitos anos, já
que eles não devem ser considerados como ``selvagens'', nem como povos
sem cultura ou sobreviventes da pré"-história. Um povo primitivo não é
um povo retrógrado ou atrasado, e também não é um povo sem história,
segundo Claude Lévy"-Strauss, ainda que ignorem a escrita, que sejam
ágrafos, as formas sociais e as técnicas das sociedades atuais, eles têm as
suas próprias técnicas, os seus hábitos e cultura.} deixavam"-se guiar
mais pelo direito da força, que pela força do direito. Não se pode, por
isso, incriminar apenas os europeus pela existência de guerras, pois é
certo que muitos grupos, mais belicistas, viviam em contínuos
conflitos e vinganças entre si, que terminavam quase sempre em rituais
de antropofagia.\footnote{ Cf. Arlindo Rubert, \textit{A Igreja
no Brasil}, vol. \textsc{i}. Santa Maria, Pallotti: 1981, pp. 135--137.} 

\section*{Algumas considerações finais}

``He outro Portugal nasceu!'' Iniciamos o nosso texto introdutório com esta mesma frase,
que fomos citando várias vezes ao longo do mesmo, pois consideramos que
era a grande ``mensagem'' que o Padre Fernão Cardim procurava transmitir
nos seus textos. Era o ``olhar'' de um homem do Quinhentos, mas um ``olhar'' 
aberto, de quem está receptivo a tudo o que observa e experimenta. 

É a visão do homem que permaneceu no Brasil durante cerca de
cinquenta anos, percorrendo grande parte do seu território, ainda que
basicamente no litoral e nas aldeias fundadas pelos membros da
Companhia, e não a visão de um simples viajante que passou pelas
terras brasileiras, ou que foi hóspede ou prisioneiro dos índios, como
Jean de Léry ou Hans Staden, entre tantos outros testemunhos. É a visão 
do Homem que chega ao Brasil, vindo de Portugal, já em idade
madura, com 35 anos, e uma formação cultural, humanista já
consolidada. Note"-se que os textos cardinianos foram redigidos logo à
sua chegada, quando desempenhava o cargo de secretário do padre
visitador Cristóvão de Gouveia, durante a sua viagem por terras
brasileiras, procurando enviar para o padre provincial em Portugal as
informações sobre o estado das aldeias e do contato com os povos
ameríndios. São dessa época a primeira carta, que compõe a denominada
\textit{Narrativa epistolar de uma viagem e missão
jesuítica pela Bahia, Ilheos, Porto Seguro, Pernambuco, Espírito Santo,
Rio de Janeiro, S.~Vicente (S.~Paulo), etc, desde o anno de 1583 ao de
1590, indo por visitador o Padre Christóvão de Gouvêa},\footnote{ Este Texto também
é por vezes denominado de \textit{Informação da Missão do Padre
Cristóvão de Gouveia às partes do Brasil, ano de 83}, encontrando"-se
publicado na obra conjunta dos textos cardinianos na nossa edição de 1997, 
pp. 211--283.} e uma segunda carta, já de 1590, incluída
também na mesma Informação, escrita já após a partida do padre
visitador e quando Cardim já era reitor do colégio da Bahia.

Os restantes textos atribuídos ao padre Fernão Cardim, que compõem os
\textit{Tratados da terra e gente do Brasil}, foram redigidos na mesma
época, entre a data de chegada, 1583 a 1590, quando muito até 1601,
data em que os seus manuscritos foram apreendidos pelos corsários
ingleses, vindo a ser mais tarde publicados em Londres, em 1625, por
Samuel Purchas. Não é, pois, o testemunho de quem está de passagem, mas
sim de um homem que chegou ao território brasileiro e que ficou, para
aí vir a construir uma vida dedicada ao desenvolvimento dos ideais
cristãos e que procurou proteger os povos ameríndios, ao mesmo tempo
que diligenciava na sua conversão. 
É o humanista a procurar entender aqueles povos diferentes com quem
entrava em contato. Mas um humanista que evidencia uma grande
capacidade analítica e tolerante na interpretação dos atos dos índios.
Espírito de curiosidade e de aceitação, na maioria dos casos, do
``outro'', na visão humanista de aceitar o índio como um ser humano. Um
homem novo, um encontro de culturas, num mundo novo. Um mundo com bons
ares e onde tudo se podia cultivar. Não ainda a visão do ``bom
selvagem'', que virá a despontar com Montaigne (1580) e acentuar"-se com
as Luzes, mas para a qual contribuíram os textos dos humanistas e, neste
caso concreto, de Fernão Cardim.

O ``índio'' deste jesuíta e de outros escritores contemporâneos é um
ser humano que vive no seu habitat, com os seus defeitos, mas também
com as suas qualidades. É um ser que come os seus prisioneiros, mas que
estima os seus filhos; que esquarteja os corpos desses cativos, mas que
antes os respeita e considera; que tem várias mulheres, mas que honra a
primeira, procurando sempre protegê"-la; que come e bebe demasiado nas
cerimônias rituais, mas que tem consideração pelos seus hóspedes.

A obra de Fernão Cardim integra"-se no contexto dos textos
renascentistas que procuravam descrever as terras e gentes descobertas
pelos portugueses, dando a conhecer o \textit{novo mundo}, onde
coabitavam novos povos, animais e plantas. O grande contributo dos
portugueses para o Renascimento é, sem dúvida, essa nova visão do homem
e do mundo, que veio alterar os modelos tradicionais.

Um dos aspectos que transparece de todos os textos sobre o Brasil e
as suas gente é a situação do seu autor, se permaneceu muito tempo
nesses territórios ou se estava de ``passagem'', assim como o momento em
que os textos são redigidos, se é o do impacto ou já depois de uma
certa permanência. Vivências diferentes que se transpõem para a forma
como encaram o índio. Se os escritos dos exploradores e viajantes
possuem, como característica comum a todos eles, uma visão parcelar do
índio, muito exterior e imediata, nos aspectos que mais rapidamente
ressaltam de um primeiro contato, por outro lado entre os
colonizadores há como que uma maneira mais interior e mais completa de
encarar os costumes dos indígenas, procurando interpretá"-los, enquanto
que os primeiros apenas se limitam a descrevê"-los.\footnote{ Cf.
M.C. Osório Dias, \textit{O índio do Brasil na literatura portuguesa
dos séculos \textsc{xvi}, \textsc{xvii} e \textsc{xviii}}, Tese de licenciatura, Faculdade de
Letras, Universidade de Coimbra, 1961, pp. 41--51.} Entre
essas fontes situam"-se os roteiros de navegação, de teor essencialmente
técnico e os relatos de viagem, estruturados em forma de diário; as
cartas narrativas, onde a personalidade do autor emerge de maneira mais
evidente, assim como os valores socioculturais; os regimentos, alvarás,
requerimentos e outros textos político"-administrativos e por vezes
econômicos; as inquirições, acerca dos testemunhos dos moradores e do
seu contato com os povos ameríndios; os tratados científicos, contendo
informações de ordem etnológica, zoológica, botânica, entre tantos
outros aspectos dignos de realce; e finalmente as gramáticas, que
permitem o estudo das línguas indígenas.\footnote{ Cf. Marina
Massimi, ``Visões do homem e aspectos psicológicos no encontro entre a
cultura portuguesa e as culturas indígenas do Brasil, no século \textsc{xvi}.
Documentos e perspectivas de análise'', in \textit{Actas do Congresso da
Missionação Portuguesa e Encontro de Culturas}, Lisboa, 1992, pp. 609--627.}

Mas a maioria dos testemunhos são de membros da Companhia de Jesus,
que através das suas cartas, como estas do Padre Fernão Cardim,
transmitem informações importantes da vida, hábitos e costumes dos
povos com quem contataram. Textos de uma enorme extensão e variedade,
que incluem desde os apontamentos pontuais transmitidos, em geral, por
via epistolar, aos tratados e aos documentos oficiais.\footnote{ Cf. Luís 
Filipe Barreto, ``O Brasil e o índio nos textos jesuítas do
século \textsc{xvi}'', in \textit{Actas do Congresso de Missionação Portuguesa e
Encontro de Culturas}, Lisboa, 1992, pp. 607--608.} Textos que permitem
ter um conhecimento da terra e gente brasílica. 
Obras que nos permitem sentir a vivência dos seus autores, cronistas,
viajantes, missionários, juristas e até colonos, ao contatarem com os
povos ameríndios. Intensa produção textual que evidencia a importância
e o interesse que o Brasil despertava nos autores europeus da época
quinhentista e nos meios ultramarinos portugueses, como novo espaço de
exploração e colonização.\footnote{ Cf. Rui Loureiro, ``A visão
do índio brasileiro nos tratados portugueses de finais do século \textsc{xvi}'',
in \textit{O confronto do olhar. O encontro dos povos na época das
navegações portuguesas}, coord. de Antonio Ferronha, Lisboa, Caminho,
1990, pp. 259--283.} Interesse notório na curiosidade evidenciada pela cultura portuguesa
quinhentista, que aborda pontos de observação psicológica, com
surpreendente agudeza analítica. O contato com esses remotos povos
significou uma grande evolução na capacidade de interpretação dos atos
humanos.\footnote{ Cf. Jorge Borges de Macedo, ``Livros impressos
em Portugal no século \textsc{xvi}. Interesses e formas de mentalidades'', in
\textit{Arquivos do Centro Cultural Português}, 1975, pp. 181--221.}
Espírito de curiosidade e de aceitação, na maioria dos casos do
``outro'', na visão humanista de aceitar o índio como um ser humano. Um
\textit{homem novo}, o ameríndio aparece como o \textit{outro
civilizacional}, que tinha estado ausente dos quadros tradicionais do
saber europeu até a descoberta do continente americano.

O ``Brasil'' entra no Renascimento português numa época em que o
homem do século \textsc{xvi} procura fundamentar a sua visão do mundo num elogio
ao presente e ao seu valor civilizacional.\footnote{ Cf. Luís
Filipe Barreto, \textit{Descobrimentos e Renascimento. Formas de ser e
pensar nos séculos \textsc{xv} e \textsc{xvi}}, 2ª ed., Lisboa, Imprensa Nacional"-Casa da
Moeda, 1983, pp. 12--50.} Mas a realidade do homem quinhentista é 
muito diversa da imagem que nos representa o seu
universo mental. O ser quinhentista é um fenômeno de extrema
complexidade ao mesmo tempo contínuo e descontínuo frente à
medievalidade. Por outro lado, um ser que consegue encarar o
\textit{Novo Mundo} e o \textit{Novo Homem} com um espírito tolerante,
na sua maioria, e que procura compreendê"-lo. Voltaire considerava
mesmo que o século \textsc{xvi} é o ``belo século'', que nos apresenta os maiores
espetáculos que o mundo jamais forneceu.\footnote{ Cf. Voltaire,
\textit{Essai sur les m\oe urs et l'ésprit des nations et sur
les principaux Faits de l'Histoire depuis Charlemagne jusqu'à Louis
\textsc{xiii}}, Paris, ed. R. Pomeau, vol. \textsc{i}, cap. \textsc{lxxxii}, p. 757, cit. in Luís
Filipe Barreto, \textit{op. cit.}, p. 24.} Século que marca a
reabilitação de toda a Europa, o seu arranque civilizacional a nível
intelectual e material. 
Arranque para o qual muito contribuíram os descobrimentos
portugueses. A ressonância das viagens marítimas e da conquista dos
continentes pelos portugueses na consciência cultural do europeu era
significativa. Porque os portugueses ao avançarem para o Sul e ao
transporem a linha do Equador tinham desvendado outro hemisfério, o
meridional, desconhecido até então, ao mesmo tempo que davam a
conhecer tudo o que estava abaixo da zona tórrida, desde as estrelas à
fauna, flora e populações, sob a abóbada dos antípodas.\footnote{ Cf. Alexandre 
Geraldini, \textit{Itinerarium ad regiones sub
aequinoctiali plaga constitutas\ldots{}}, Roma, 1631, pp. 204--205, cit. in
W.G.L. Randles, \textit{Da Terra Plana ao Globo Terrestre. Uma Rápida
Mutação Epistemológica, 1480--1520}, trad. port., Lisboa,
Gradiva, 1990, p. 39.} 

O choque psicológico vinha a sentir"-se desde a viagem de Diogo Cão ao
Golfo da Guiné, por 1484--1486, já que se registou, pela primeira vez, a
existência de outro mundo, ou de um novo mundo, e o erro descompassado
em que tinham incorrido e persistido os geógrafos antigos e modernos. A
descoberta da comunicabilidade entre o mar Oceano e o Índico, ou seja,
da passagem de sueste, efetuada pela expedição de 1487--1488, comandada
por Bartolomeu Dias, confirmou a viabilidade de navegar entre os dois
oceanos e de conseguir alcançar a Índia pelo Atlântico Sul. 

 Mas a atitude de ruptura só verdadeiramente se catalisou com a
chegada de Cristóvão Colombo às Antilhas e a de Pedro Álvares Cabral
ao Brasil. Era, de fato, um \textit{Novo Homem} e um \textit{Novo
Mundo}, que os povos ibéricos encontraram no continente americano e,
através destes, o resto da Europa.

Estes ensinamentos da nova geografia propagaram"-se por toda a Europa,
ainda que com alguma reação em relação ao novo continente descoberto,
o americano, que alguns defenderam como sendo autônomo, exumando temas
esquecidos da cultura antiga e desafiando eventuais censuras da
Igreja, denominando"-o de ``Novo Mundo''.\footnote{ Cf. W.G. Randles,
\textit{op. cit.}, pp. 117--122.} Depois que os europeus alargam o
espaço geográfico, eles são portadores de inquietudes diante do
desconhecido, mas também portadores de respostas. As terras, as
plantas, os animais, mas sobretudo os homens não são idênticos aos
conhecidos depois de séculos através da Europa, e é necessário que se
tenha conhecimento desse fato. E o discurso português é o fundador
desta nova visão.\footnote{ Cf. Alfredo Margarido, ``La Vision de
L'Autre (Africain et Indien d'Amérique) dans la Renaissance
Portuguaise'', in \textit{L'Humanisme Portuguais et l'Europe}, Actes
du \textsc{xxi} Colloque Internacional d'Études Humanistes, Tours, 3--13
Juillet, 1978, pp. 507--555.} 

O contato com a nova terra e gentes do território brasílico era um
dos temas preferidos pelos autores portugueses e outros europeus,
sobretudo franceses. Considera"-se, mesmo, que o contributo dos
portugueses para o ``mito do bom selvagem'' foi significativo e que
desde a \textit{Carta de Achamento} de Pêro Vaz de Caminha que a
imagem do silvícola se tem mantido como a do ``bom selvagem''. Os
ameríndios pareciam almas puras e eram portadores de uma bondade
natural que fascinou Cabral, os seus companheiros e muitos que os
seguiram, como Fernão Cardim.\footnote{ Cf. Joaquim Veríssimo
Serrão, ``Da Terra de Vera Cruz à América Portuguesa'', in
\textit{Actas dos 1º Cursos Internacionais de Verão de Cascais}, 
Cascais, Câmara Municipal de Cascais, 1994, pp. 303--313.} 

É notório neste autor o Homem completo, que procura captar o maior
número de conhecimentos, observando tudo o que o rodeia, um humanista
que procura um saber em harmonia com o viver e ainda um saber em
harmonia com um novo mundo. Mas sempre um saber global, total, que
consiga transmitir o maior número de informações aos seus superiores.
Nele encontramos o geógrafo, que estuda a terra, o seu clima e a sua
habitabilidade; o etnógrafo, que descreve os povos indígenas, seus usos
e costumes, com respeito e coerência; o zoólogo e o botânico, que
observa com rigor a fauna e flora desconhecidas, descrevendo"-as de uma
forma quase visual; o cronista que traça os hábitos das populações, até
mesmo os gastronômicos, e que menciona as missões dos jesuítas, os seus
colégios e residências, o estado das capitanias, os seus habitantes e
suas produções, o progresso ou a decadência da Colônia e as suas
causas, assim como os problemas que tinham de enfrentar diariamente,
alertando mesmo o poder para as questões a resolver.

Fernão Cardim não se mostra, de uma forma geral, espantado pela terra
e pelos homens que encontra. Preocupa"-se em descrevê"-los, o que faz
minuciosamente, salientando aspectos físicos, de temperamento e de
comportamento. O seu estado de alma estava imbuído de um otimismo
quase ingênuo, face à terra e aos homens. Até mesmo os animais mais
estranhos não são caracterizados de uma forma desagradável e
indiferente, procurando integrá"-los no seu habitat. O mesmo acontece
com a flora brasileira, tão rica em aspecto, gosto e sabores e que
Cardim descreve mostrando conhecê"-la profundamente. Logo na seleção
dos temas observados nos povos ameríndios, deu prioridade aos assuntos
de caráter mais humano, que mostrassem melhor as características
físicas e psicológicas desses povos, com quem tinha contatado,
nomeadamente a forma como os índios tratavam as mulheres, a educação
das crianças, a questão dos casamentos, a saudação lacrimosa, os
incitamentos dos principais, além, evidentemente, da questão da
antropofagia, dos rituais mágico"-religiosos. 
Informações e descrições diferentes que terão que ver com a sua
formação e a própria vivência no território brasileiro, além do momento
em que foram escritos e, concomitantemente, com a personalidade e
maturidade do próprio autor. Em Cardim encontramos não só as descrições
impessoais que existem nas outras obras, mas ainda a projeção de
sentimentos de alegria e tristeza, de admiração e de orgulho.
Sentimos ainda, de certa forma, as cores, os odores, os sabores, como
se estivéssemos a viver o acontecimento a par de Cardim.

Seguindo a sua Carta ao padre provincial, em que dá conhecimento da
viagem do padre visitador Cristóvão de Gouveia, encontramos o texto de
um jesuíta que vê o seu superior debater"-se, como o fará mais tarde o
próprio Cardim, sobre a validade do batismo e do casamento dos índios,
e que no final da confissão não deixa de dizer, na própria língua
indígena: \textit{Xê rair tupã de hiruamo}, ou seja, a expressão: ``Filho, Deus vá contigo!''. 

Riqueza de testemunhos que nos permitem afirmar que, entre os
cronistas e viajantes que estiveram no Brasil no Quinhentos, Fernão
Cardim mostrou um conhecimento excepcional do território, das gentes,
da fauna e flora. Valor que não passou despercebido a Ferdinand Denis,
que louvou a excelência do texto das cartas da \textit{Narrativa
epistolar}, ao afirmar:

\begin{hedraquote}
um pequeno livro escrito num estilo cativante e que se deve a um missionário
até agora desconhecido, [\ldots{}] o Padre Fernão Cardim, [\ldots{}] dotado de um
sentimento poético, duma rara delicadeza e que se revela como sem seu
conhecimento, em algumas das cartas confidenciais que escreveu a um
superior.\footnote{ Cf. Ferdinand Denis, \textit{Une Fête Brésilienne 
Célébrée à Roeun en 1550}, publicada em Paris, em 1851,
cit. in ``Introdução'' de Rodolfo Garcia, in Fernão Cardim, \textit{op. cit.}, 
pp. 17--18. Tradução da autora.} 
\end{hedraquote}

Este autor utiliza mesmo o texto cardiniano para descrever as festas e os cânticos dos índios. 
O Padre Fernão Cardim foi, sem dúvida, uma figura importante da
produção quinhentista portuguesa, pelos seus depoimentos, que são
testemunho presencial, e feitos de uma forma sincera, rigorosa e sem
juízos de valor. Sujeitos a várias leituras e interpretações, os textos
cardinianos não poderão deixar de ocupar esse lugar que lhes é devido
no contexto humanista português, tal como o têm sido no brasileiro, a
par de outros autores quinhentistas e seiscentistas que procuraram
transmitir esse significativo \textit{encontro de culturas}. 

\section*{A edição do texto}

 Ao preparar"-se a edição da obra do Padre Fernão Cardim
procuramos, essencialmente, dar a conhecer os textos escritos por esse
jesuíta entre 1583 e 1590, ou possivelmente até 1601, data em que os
manuscritos foram confiscados. Tratando"-se de escritos de finais de
Quinhentos, houve a preocupação em manter o mesmo estilo de composição,
não realizando significativas alterações. Estruturamos os dois textos
que compõem a obra cardiniana de acordo com a primeira edição conjunta,
de 1925, no Brasil, porque nos pareceu adequada ao melhor conhecimento
da terra e gente do Brasil.

Assim, transcreveram"-se os dois tratados 
referentes ao clima e terra do Brasil e aos índios.
Ao longo do texto mantivemos os termos e expressões tupis, em
itálico, utilizadas pelo autor, procurando dar"-lhes uma adequada
significação e explicando, sempre que possível, a sua etimologia e a
primeira vez que foram introduzidas na língua portuguesa. Foram também
mantidas as expressões em latim. Utilizamos para esse fim uma
bibliografia lexical específica, além das notas de Rodolfo Garcia,
Capistrano de Abreu e Baptista Caetano, incluídas na primeira edição
conjunta dos textos cardinianos, de 1925.

Nas notas de rodapé, remetemos aos aspectos que consideramos que
poderiam esclarecer o leitor sobre os diversificados assuntos
apresentados por Cardim. Procuramos que este aparato crítico não fosse
muito denso, o que nem sempre foi possível, dado o interesse que alguns
temas mereciam, até, por vezes, com uma mais desenvolvida abordagem.
Optamos por não utilizar um glossário no fim da obra com os termos
desusados ou em tupi, ou outra língua indígena, considerando que não
facilitava a leitura dos textos, incluindo, como tal, esses termos nas
notas de fim de página. 

 No que concerne às normas de transcrição ortográfica, procuramos
manter o mais possível o texto cardiniano, sem provocar alterações
significativas, mas tornando"-o acessível e atraente para o maior
número de leitores. Assim, nesta transcrição:

\begin{itemize}
\item atualizou"-se o emprego das vogais, dos ditongos nasais em final de
sílaba e plurais e da nasalação expressa por til; retiraram"-se as vogais
germinadas, com ou sem acento de acordo com as regras atuais de
ortografia; uniformizou"-se a atualização do ``h'' e do ``g'' oclusivos;

\item retiraram"-se as consoantes germinadas e atualizaram"-se os empregos de
``s/z/c/ç'' e do ``r'', eliminando"-se os grupos de consoantes e digramas
``ch, ph, th, qu'', hoje já em desuso;

\item desenvolveram"-se as abreviaturas sempre que possível ou, em casos
duvidosos, remeteram"-se para nota de rodapé; substituiu"-se o sinal \& por
``e'', ou por ``etc.'', se estava mais de acordo com o texto;

\item regularizou"-se o uso de maiúsculas segundo a forma atual, ainda
que se mantivessem algumas de acordo com o espírito do texto da época;

\item e mantiveram"-se algumas oscilações gráficas e formas arcaicas.
\end{itemize}

 No que diz respeito à acentuação procurou"-se deixá"-la da forma como
estava, alterando"-se sobretudo nas situações em que a ausência da mesma
podia afetar a boa compreensão do texto. A mesma regra foi seguida em
relação à pontuação, até porque a maioria destes textos quinhentistas
destinava"-se a ser ouvida, ou seja, lida em voz alta, tendo a
pontuação, talvez por isso, a cadência do ritmo respiratório. Para
tornar o texto menos denso foram abertos vários parágrafos não
contemplados nos manuscritos. Manteve"-se, no entanto, a mesma
divisão em capítulos estruturada por Cardim.

\begin{bibliohedra} 
\item A presente edição da obra de Fernão Cardim encontra"-se fundamentada
numa bibliografia diversificada, atendendo à variedade de temas
tratados nos textos dos \textit{Tratados da terra e gente do Brasil}, 
além do estudo do próprio autor, do Brasil Quinhentista e das
sociedades ameríndias, assim como da fauna e flora brasílicas. Optamos
por selecionar apenas algumas das obras que consideramos mais relevantes.

\vspace*{2ex}
\scriptsize\textbf{Fontes impressas}

\tit{ABBEVILLE}, Claude d'. \textit{História da Missão dos Padres
Capuchinhos na Ilha do Maranhão e terras Circunvizinhas}, trad. port.
Sérgio Milliet, São Paulo, Liv. Martins Editora, 1945.

\tit{ABREU}, Sebastião de. \textit{Vida e Virtudes do Admirável Padre João
Cardim da Companhia de Jesus}, Évora, Universidade de Évora, 1659.

\tit{AYLLY}, Pierre d'. \textit{Ymago Mundi}, texto modernizado e notas de
Edmond Buran, 3º vol., Paris, 1930.

\tit{ANCHIETA}, José de. \textit{Cartas, Informações, Fragmentos Históricos
e Sermões}, Belo Horizonte/São Paulo, Itatiaia/Ed\textsc{usp}, 1988.

\tit{ANTONIL}, André João. \textit{Cultura e Opulência do Brasil por suas
drogas e minas}, Lisboa, Pub.~Alfa, 1989.

\tit{BARROS}, João de. \textit{Décadas da Ásia}, 4 vols., selecção, notas e
introdução de Antonio Baião, Lisboa, Livraria Sá da Costa, 1945.

\tit{BARBOSA}, Duarte. \textit{Livro do que viu e ouviu no Oriente Duarte
Barbosa}, ed.~de Luís de Albuquerque, Lisboa, Pub.~Alfa, 1989.

\tit{CAMINHA}, Pêro Vaz de. \textit{Carta do Achamento do Brasil ao rei D.
Manuel}, ed.~de Manuel Viegas Guerreiro e de Eduardo Borges Nunes,
Lisboa, Imprensa Nacional"- Casa da Moeda, 1974.

\tit{CARDIM}, Fernão. \textit{Tratados da Terra e Gente do Brasil}, Transcrição do texto, introdução e notas de Ana Maria de Azevedo, Lisboa, Comissão Nacional para as Comemorações dos Descobrimentos Portugueses (\versal{CNCDP}), 1997.

\tit{Cartas jesuíticas, cartas avulsas, (1550--1568)}, Belo Horizonte/São~Paulo, Itatiaia/Ed\textsc{usp}, 1988.

\tit{COLOMBO}, Cristóvão. \textit{La Découverte de l'Amérique}, Paris, Maspero, 1979.

\tit{COLOMBO}, Cristóvão. \textit{\OE uvres}, Paris, Galimard, 1961.

\tit{COSTA}, Cristóvão da. \textit{Tratado das Drogas e Medicinas das Índias
Orientais}, ed.~de Walter Jaime, Junta de Investigações Científicas do Ultramar, 1964.

\tit{DENIS}, Ferdinand. \textit{Brasil}, trad port., Belo
Horizonte/São Paulo, Itatiaia/Ed\textsc{usp}, 1980.

\tit{ÉVREUX}, Yves d'. \textit{Viagem ao Norte do Brasil}, int. e notas de
Ferdinand Denis, trad. port. de César Augusto Marques, Maranhão, s. ed., 1874.

\tit{GÂNDAVO}, Pêro de Magalhães de. \textit{História da Província de Santa
Cruz}, texto modernizado por Maria da Graça Pericão e comentário de
Jorge Couto, Lisboa, Pub.~Alfa, 1989.

\tit{KNIVET}, Anthony. \textit{Narração da Viagem que nos anos de 1591 e
seguintes fez da Inglaterra ao Mar do Sul, em Companhia de Thomas
Cavendish}, trad. de J.H. Duarte Pereira, in \textit{Revista do
Instituto Histórico, Geográfico e Etnográfico do Brasil}, tomo \textsc{xl}, Rio de Janeiro, 1947.

\tit{LAFITAU}, Joseph"-François. \textit{M\oe urs des sauvages américains,
comparées aux m\oe urs des premiers temps}, Paris, ed. La Découverte, 1994.

\tit{LAS CASAS}, Bartolomeu de. \textit{Brevíssima Relação da Destruição das
Índias}, trad. de Júlio Henriques, Lisboa, Ed. Antígona, 1990.

\tit{LÉRY}, Jean de. \textit{Viagem à Terra do Brasil}, trad. port., Belo
Horizonte/São Paulo, Itatiaia/Ed\textsc{usp}, 1980.

\tit{LISBOA}, Frei Cristóvão de. \textit{História dos Animais e Árvores do
Maranhão}, ed. facsimilada, Lisboa, Arquivo Histórico Ultramarino e
Centro de Estudos Históricos Ultramarinos, 1967.

\tit{MARCGRAVE}, G. \textit{História Natural do Brasil}, trad. port., São
Paulo, Ed. do Museu Paulista, 1942.

\tit{MONTAIGNE}, Michel de. \textit{Essais}, 2 vols., Paris, Liv. Générale Française, 1972.

\tit{NÓBREGA}, Manuel da. \textit{Cartas do Brasil e mais escritos}, introd.~e 
notas históricas e críticas de Serafim Leite, Coimbra, Acta Universitatis Conimbricensis, 1955.

\titidem. \textit{Diálogo sobre a Conversão do Gentio}, ed.
de Serafim Leite, Lisboa, Comissão do \textsc{iv} Centenário da Fundação de São Paulo, 1954.

\tit{Novas Cartas Jesuíticas, (de Nóbrega a Vieira)}, coligidas por
Serafim Leite, São Paulo, Companhia Editora Nacional, 1940.

\tit{ORTA}, Garcia da. \textit{Colóquios dos Simples e Drogas da Índia}, ed.
do Conde de Ficalho (Lisboa, 1891), edição facsimilada, 2
vols., Lisboa, Imprensa Nacional"-Casa da Moeda, 1987.

\tit{PLÍNIO}. \textit{Naturalis Historiae Libri}, \textsc{xxxvii}, ed. J. Harduin, 5
tomos, Paris, 1685.

\tit{SALVADOR}, Frei Vicente do. \textit{História do Brasil, (1500--1627)}, 
Belo Horizonte/São Paulo, Itatiaia/Ed\textsc{usp}, 1982.

\tit{SEVILHA}, Santo Isidoro de. \textit{Etimologias}, trad. castelhana,
Madrid, \textsc{b.a.c}, 1951.

\tit{SOARES}, Francisco. \textit{Cousas Notáveis do Brasil}, texto
modernizado por Maria Graça Pericão e comentário de Luísa Black,
Lisboa, Pub.~Alfa, 1989.

\tit{SOUSA}, Gabriel Soares de. \textit{Notícia do Brasil}, comentário de
Luís de Albuquerque, Lisboa, Pub.~Alfa, 1989.

\tit{SOUSA}, Pêro Lopes de. \textit{Relação da navegação (1530--1532)}, texto
modernizado por Luís de Albuquerque e comentário de Maria do Anjo
Ramos, Lisboa, Pub.~Alfa, 1989.

\tit{STADEN}, Hans. \textit{Duas Viagens ao Brasil}, Belo Horizonte/São
Paulo, Itatiaia/Ed\textsc{usp}, 1974.

\tit{THEVET}, André. \textit{Les Singularités de la France Antarctique
(1557)}, ed. apresentada e anotada por Frank Lestringant, Paris, Ed.~Chandeigne, 1997.

\tit{VIEIRA}, Antonio. \textit{Cartas de Antônio Vieira}, org. de J.L. de
Azevedo, Coimbra, Imprensa Nacional, 1925.

\vspace*{2ex}
\scriptsize\textbf{Bibliografia geral}


\tit{ABREU}, João Capistrano de. \textit{Capítulos de História Colonial
(1500--1800)}, 4ª ed., Rio de Janeiro, Soc. Capistrano de Abreu/Liv.
Briguet, 1954.

\tit{ALMAÇA}, Carlos, ``Os Portugueses do Brasil e a Zoologia Pré"-Lineana'', in
\textit{A Universidade e os Descobrimentos}, Lisboa, Imprensa
Nacional"-Casa da Moeda, 1993, pp. 175--194.

\tit{BARBALHO}, Nelson. \textit{Dicionário do Açúcar}, Recife, Editora
Massangana, 1984.

\tit{BARRETO}, Luís Filipe. \textit{Descobrimentos Portugueses e
Renascimento. Formas de Ser e de Pensar nos Séculos \textsc{xv} e \textsc{xvi}}, 2ª ed.,
Lisboa, Imprensa Nacional"-Casa da Moeda, 1983.

\titidem. \textit{Os Descobrimentos e a ordem de saber --
uma análise sócio"-cultural}, Lisboa, Gradiva, 1987.

\tit{BOSI}, Alfredo. \textit{História Concisa da Literatura Brasileira}, São
Paulo, Ed. Cultrix, s.d.

\tit{BOXER}, Charles Ralph. \textit{O Império Colonial Português, 1415--1825}, 
trad. port., Lisboa, Ed. 70, 1975.

\tit{Brasil A/Z, Larousse Cultural}, Enciclopédia Alfabética, São
Paulo, Editora Universo, 1988.

\tit{CALMON}, Pedro. \textit{História da Fundação da Bahia}, Baía, Pub. do
Museu do Estado, 1949.

\tit{CARVALHO}, Joaquim Barradas de, ``O Descobrimento do Brasil através dos
Textos'', in \textit{Separata da Revista de História}, nºs 80 e 82, São
Paulo, 1969--1970.

\tit{CASTELLO}, José Aderaldo, \textit{Manifestações Literárias do Período
Colonial (1500--1808)}, São Paulo, Ed. Cultrix, s.d.

\tit{CORTESÃO}, Jaime. \textit{A Colonização do Brasil}, Lisboa, Portugália,
1969.

\titidem. \textit{A Fundação de São Paulo capital geográfica do
Brasil}, Rio de Janeiro, Livros de Portugal, 1955.

\titidem. \textit{Introdução à História das Bandeiras}, 2
vols., Lisboa, Livros Horizonte, 1975.

\tit{COUTINHO}, Afrânio, dir.~\textit{A Literatura no Brasil}, 2ª ed., Rio
de Janeiro, Editorial Sul Americana, 1968.

\tit{COUTO}, Jorge. \textit{A Construção do Brasil. Ameríndios, Portugueses
e Africanos do início do povoamento a finais de Quinhentos}, Lisboa,
Ed. Cosmos, 1995.

\tit{Cultura Portuguesa na Terra de Santa Cruz}, coord. de Maria
Beatriz Nizza da Silva, Lisboa, Ed. Estampa, 1995.

\tit{DELUMEAU}, Jean. \textit{Uma História do Paraíso. O Jardim das
Delícias}, Lisboa, Terramar, 1994.

\tit{DIAS}, J. Sebastião da Silva. \textit{Os Descobrimentos e a
Problemática Cultural do Século \textsc{xvi}}, Coimbra, Universidade de Coimbra, 1973.

\tit{Dicionário da História da Colonização Portuguesa no Brasil}, 
coord. de Maria Beatriz Nizza da Silva, Lisboa, Ed. Verbo, 1994.

\tit{Dicionário de História de Portugal}, 6 vols., dir.~de Joel
Serrão, Lisboa, Iniciativas Editoriais, 1975.

\tit{Dicionário de Literatura Portuguesa}, 5 vols., dir.~de Jacinto
Prado Coelho, 3ª ed., Porto, Figueirinhas, 1978.

\tit{Dicionário de Folclore Brasileiro}, org. de Luís da Câmara
Cascudo, Rio de Janeiro, Ediouro, 3ªed., 1972. 

\tit{FEBVRE}, Lucien. \textit{O problema da descrença no século \textsc{xvi}}, trad.
port., Lisboa, Editorial Início, s.d. 

\tit{FERRÃO}, José E. Mendes. \textit{A Aventura das Plantas e os
Descobrimentos Portugueses}, Lisboa, Instituto de Investigação
Científica Tropical, Comissão Nacional para as Comemorações dos
Descobrimentos Portugueses e Fundação Berardo, 1992.

\tit{FREYRE}, Gilberto. \textit{Casa Grande e Senzala. Formação da Família
Brasileira sob o Regime de Economia Patriarcal}, Lisboa, Ed. Livros do
Brasil, s.d.

\titidem. \textit{Em torno de um novo conceito de
Tropicalismo}, Coimbra, Tipografia Coimbra Ed., 1952.

\titidem. \textit{Interpretação do Brasil. Aspectos da
formação social brasileira como processo de amalgamento de raças e
culturas}, Lisboa, Ed. Livros do Brasil, s.d.

\titidem. \textit{Problemas Brasileiros de Antropologia}, 4ª
ed., Rio de Janeiro, José Olympio, 1973.

\tit{GODINHO}, Vitorino Magalhães. \textit{Os Descobrimentos e a Economia
Mundial}, 4 vols., Lisboa, Ed. Presença, 1981--1983.

\titidem. \textit{Mito e mercadoria, utopia e
prática de navegar, séculos \textsc{xii}-\textsc{xviii}}, Lisboa, Difel, 1990.

\tit{GUERREIRO}, Manuel Viegas. \textit{A Carta de Pêro Vaz de Caminha lida
por um etnógrafo}, Lisboa, Edições Cosmo, 1992.

\tit{HENRIQUES}, Isabel Castro e \textsc{margarido}, Alfredo. \textit{Plantas e
Conhecimento do Mundo nos Séculos \textsc{xv} e \textsc{xvi}}, Lisboa, Pub.~Alfa, 1989.

\tit{História da Colonização Portuguesa do Brasil}, dir.~de Carlos
Malheiro Dias, 3 vols., Porto, Litografia Nacional, 1921--1924.

\tit{História da Expansão Portuguesa no Mundo}, dir.~de Antonio
Baião, Hernâni Cidade e Manuel Múrias, 3 vols., Lisboa, Ed. Ática, 1937.

\tit{História Geral da Civilização Brasileira}, dir.~de Sérgio
Buarque de Holanda, São Paulo, Difel"-Difusão Editorial, 1985.

\tit{História de Portugal}, 15 vols., dir.~de João
Medina, Lisboa, Ediclube, 1992.

\tit{HOLANDA}, Sérgio Buarque de. \textit{Raízes do Brasil}, 10ª ed., Rio de
Janeiro, Liv. José Olympio, 1976.

\titidem. \textit{Visão do Paraíso. Os Motivos
Edênicos no Descobrimento e Colonização do Brasil}, Rio de Janeiro,
Liv. José Olympio, 1959.

\tit{LEITE}, Serafim. \textit{Novas Páginas da História do Brasil}, Lisboa,
Academia Portuguesa de História, 1962.

\tit{MARGARIDO}, Alfredo. \textit{As surpresas da Flora no tempo dos
Descobrimentos}, Lisboa, Ed. Elo, 1994.

\tit{MAURO}, Frédéric. \textit{O Império Luso"-Brasileiro (1620--1750)}, 
Lisboa, Ed. Estampa, 1991.

\titidem. \textit{Portugal, o Brasil e o Atlântico (1570--1670)}, 
Lisboa, Imprensa Universitária, Ed. Estampa, 1989.

\tit{MOISÉS}, Massaud. \textit{História da Literatura Brasileira}, vol. \textsc{i},
``Origens, Barroco, Arcadismo'', 3ª ed., São Paulo, Cultrix, 1990.

\tit{MONTEIRO}, John M. e \textsc{moscoso}, Francisco. \textit{América Latina
Colonial}, São Paulo, Cela, 1990.

\tit{MORALES PADRON}, Francisco. \textit{Historia del Descubrimiento y
Conquista de América}, 3ª ed., Madrid, Ed. Nacional, 1973.

\tit{Nas Vésperas do Mundo Moderno -- Brasil}, Lisboa, Comissão
Nacional para as Comemorações dos Descobrimentos Portugueses, 1991.

\tit{Nova História da Expansão Portuguesa}, dir.~de Joel Serrão e
A.H. de Oliveira Marques, Lisboa, Ed. Estampa, 1992.

\tit{O Confronto do Olhar}. \textsc{o encontro dos povos na época das
navegações portuguesas séculos xv e xvi}, coord. de Antonio Luís
Ferronha, Lisboa, Ed. Caminho, 1991.

\tit{O Império Luso"-Brasileiro (1500--1620)}, coord. de Harold
Johnson e Maria Beatriz Nizza da Silva, vol. \textsc{vi} da \textit{Nova
História da Expansão Portuguesa}, dir.~de Joel Serrão e A.H. de Oliveira
Marques, Lisboa, Ed. Estampa, 1992.

\tit{PINTO}, João Rocha. \textit{A Viagem, Memória e Espaço}, Lisboa, Liv.
Sá da Costa, 1989.

\tit{POMBO}, Rocha. \textit{História do Brasil}, 11ª ed., Lisboa, Ed.
Melhoramentos, 1963.

\tit{PRADO JÚNIOR}, Caio. \textit{História Econômica do Brasil}, 38ª ed.,
São Paulo, Brasiliense, 1990.

\tit{PRADO}, J. F. de Almeida. \textit{Formação Histórica da Nacionalidade
Brasileira. Os Primeiros povoadores do Brasil}, São Paulo, Companhia
Editora Nacional, 1935.

\tit{RAMOS}, Arthur. \textit{Introdução à Antropologia Brasileira}, Rio de
Janeiro, Liv. Editora da Casa do Estudante do Brasil, 1961.

\tit{RANDLES}, W. G. L. \textit{Da Terra Plana ao Globo Terrestre. Uma
Rápida Mutação Epistemológica, 1480--1520}, trad. port., Lisboa, Gradiva, 1990.

\tit{RODRIGUES}, José Honório. \textit{Teoria da História do Brasil}, 5ª
ed., São Paulo, Companhia Editora Nacional, 1978.

\titidem. \textit{A Pesquisa Histórica no Brasil}, 4ª
ed., São Paulo, Companhia Editora Nacional, 1982.

\titidem. \textit{História da História do Brasil}, 1ª
parte -- ``Historiografia Colonial'', São Paulo, Companhia Editora
Nacional, 1979.

\tit{RUBERT}, Arlindo. \textit{A Igreja no Brasil, Origem e Desenvolvimento,
Século \textsc{xvi}}, vol. \textsc{i}, Santa Maria"- R.S., Livr. Ed. Pallotti, 1981.

\tit{SERRÃO}, Joaquim Veríssimo. \textit{A Historiografia Portuguesa,
Doutrina e Crítica}, vol. \textsc{i}, Séculos \textsc{xii}-\textsc{xvi}, Lisboa, Ed. Verbo, 1972.

\titidem. \textit{Do Brasil Filipino ao Brasil de
1640}, São Paulo, Companhia Editora Nacional, 1968.

\titidem. \textit{História de Portugal}, 2ª ed., 12
vols., Lisboa, Ed. Verbo, 1980--1993.

\titidem. \textit{O Rio de Janeiro no Século \textsc{xvi}}, 2
vols., Lisboa, Comissão Nacional das Comemorações do \textsc{iv} Centenário do
Rio de Janeiro, 1965. 

\titidem. \textit{O Tempo dos Filipes em Portugal e no
Brasil (1580--1640)}, Lisboa, Ed. Colibri, 1994.

\tit{SILVA}, Maria Beatriz Nizza da. \textit{Guia da História Colonial}, 
Porto, Universidade Portucalense, 1992.

\tit{SIMONSEN}, Roberto C. \textit{História Económica do Brasil
(1500--1820)}, São Paulo, Companhia Editora Nacional, 1962.

\tit{Suma Etnológica Brasileira}, Comissão ed. Darcy Ribeiro, Berta
G. Ribeiro e Carlos de Araújo Moreira Neto, Petrópolis, Vozes, 1986.

\tit{VARNHAGEN}, Francisco Adolfo de. \textit{História Geral do Brasil antes
da sua separação e independência de Portugal, } 4ª ed., São Paulo, Ed.
Melhoramentos, 1948.

\tit{Viagens e Viajantes no Atlântico Quinhentista}, coord. de
Maria da Graça M. Ventura, Lisboa, Ed. Colibri, 1996.

\tit{VIANA}, Hélio. \textit{A Formação Brasileira}, Rio de Janeiro, José
Olympio, 1935.

\titidem. \textit{História do Brasil}, 9ª ed., São Paulo, Ed.
Melhoramentos, 1972.

\tit{WEHLING}, Arno e \textsc{wehling}, Maria José C. de. \textit{Formação do Brasil
Colonial}, Rio de Janeiro, Nova Fronteira, 1994. 

\vspace*{2ex}
\scriptsize\textbf{Estudos sobre Missionação}

\tit{BEOZZO}, José Oscar. \textit{Leis e Regimentos das Missões: política
indigenista no Brasil}, São Paulo, Edições Loyola, 1983.

\tit{CABRAL}, Luís Gonzaga. \textit{Jesuítas no Brasil, (século \textsc{xvi})}, São
Paulo, Melhoramentos, s.d.

\tit{COUTO}, Jorge. \textit{O Colégio dos Jesuítas do Recife e o destino do
seu Património, (1759--1777)}, 2 vols., dissertação de mestrado
apresentada à Faculdade de Letras da Universidade de Lisboa, 1990, (policopiada).

\tit{HAUBERT}, Maxime. \textit{Índios e Jesuítas no Tempo das Missões,
(Séculos \textsc{xvii}-\textsc{xviii})}, trad. port., São Paulo, Companhia das
Letras"-Círculo do Livro, 1990.

\tit{LEAL}, Antonio Henriques. \textit{Apontamentos para a História dos
Jesuítas no Brasil}, tomo \textsc{i}, Maranhão, Liv. Popular Magalhães e Cª,
1874.

\tit{LEITE}, Serafim. \textit{Artes e Ofícios dos Jesuítas no Brasil
(1549--1760)}, Lisboa, Ed. Brotéria, 1953.

\titidem. \textit{História da Companhia de Jesus no Brasil}, 10
vols., Lisboa"-Rio de Janeiro, Liv. Portugália"-Instituto Nacional do
Livro, 1938--1950.

\titidem. \textit{Monumenta Brasiliae (1528--1568)}, 4vols.,
Roma, Monumenta Histórica Societatis Iessu, 1956--1960.

\titidem. \textit{Nóbrega e a Fundação de São Paulo}, Lisboa,
Instituto de Intercâmbio Luso"-Brasileiro, 1953.

\titidem. \textit{Breve Itinerário para a biografia do Padre
Manuel da Nóbrega}, Lisboa"-Rio de Janeiro, s. ed., 1955. 

\titidem. \textit{Novas Cartas Jesuíticas (De Nóbrega a Vieira)}, 
Brasiliana, vol. 194, 5ª série, São Paulo, Companhia Editora Nacional, 1940.

\titidem. \textit{Os Governadores Gerais do Brasil e os Jesuítas
no Século \textsc{xvi}}, apresentado no \textsc{i} \textit{Congresso de História da
Expansão Portuguesa no Mundo}, Lisboa, Soc.~Nacional de Tipografia, 1937.

\titidem. \textit{Suma Histórica da Companhia de Jesus no Brasil
(1549--1760), (Assistência de Portugal)}, Lisboa, Junta de Investigação
do Ultramar, 1965.

\tit{RODRIGUES}, Francisco. \textit{A Companhia de Jesus em Portugal e nas
Missões}, Porto, Edição do Apostolado de Imprensa, 1935.

\titidem. \textit{A Formação Intelectual do jesuíta}, 
Porto, Liv. Magalhães e Moniz ed., 1917.

\tit{SANTOS}, Eugénio dos, ``Índios e Missionários no Brasil Quinhentista: do
confronto à cooperação'', in \textit{Revista da Faculdade de Letras do
Porto}, \textsc{ii} série, vol. \textsc{ix}, pp. 107--118.

\vspace*{2ex}
\scriptsize\textbf{Estudos sobre o Padre Fernão Cardim}

\tit{AZEVEDO}, Ana Maria de. \textit{O Padre Fernão Cardim (1548--1625).
Contribuição para o Estudo da sua Vida e Obra}, 2 vols., dissertação de
mestrado apresentada à Faculdade de Letras da Universidade de Lisboa, 1996, (policopiada).

\tit{ABREU}, Capistrano de. artigo apenso de \textit{O Jornal}, de 27 de
Janeiro de 1925, in Fernão Cardim. \textit{Tratados da terra e gente do
Brasil}, Belo Horizonte/São Paulo, Itatiaia/Ed\textsc{usp}, 1980, pp. 199--206.

\tit{CURLY}, Maria Odília Dias, ``Um texto de Cardim inédito em Português?'',
in \textit{Revista de História}, São Paulo, 1964, vol. \textsc{xxviii}, Ano 15,
Abril"--Junho nº58, pp. 455--482.

\tit{ESPINOSA}, J. Manuel, ``Fernão Cardim: Jesuit Humanist of Colonial
Brazil'', in \textit{Mid"-America: ``An Historical Review''}, vol. 24, New
Series \textsc{xiii}, nº4, Chicago, Outubro 1942, pp. 252--271.

\tit{FERNANDES}, Eunícia Barros Barcelos. \textit{Cardim e a colonialidade}, 
dissertação de mestrado apresentada no departamento de História da
PUC/RJ, Rio de Janeiro,1995, (policopiada).

\tit{FLOOD}, W.H. Grattan, ``Portuguese Jesuits in England in Penal Times'',
in \textit{The Month}, nº143, 1924, pp. 157--159.

\tit{LEITE}, Serafim, ``Fernão Cardim, autor da Informação da Província do
Brasil para Nosso Padre, de 31 de Dezembro de 1583'', in \textit{Jornal
do Commercio}, Rio de Janeiro, 30 de Dezembro de 1945.

\tit{PIRES}, Maria Antonieta Neves. \textit{A Narrativa epistolar de Fernão
Cardim. A ``Carta ânua'' ou o outro lado da Narrativa}, dissertação de
mestrado apresentada à Faculdade de Letras da Universidade de Lisboa,
1996, (policopiada).

\vspace*{2ex}
\scriptsize\textbf{Estudos sobre as Sociedades Ameríndias e a Alteridade}

\tit{ABREU}, Aurélio M.G. de. \textit{Culturas Indígenas do Brasil}, São Paulo, Traco, s.d.

\tit{AVEVEDO}, Eliane. \textit{Raça: conceito e preconceito}, São Paulo, Ática, 1987.

\tit{Barbares \& Sauvages. Images et Reflets dans la Culture
Occidentale}, Actes du Colloque de Caen, 26--27 Fevereiro 1993, coord.
de Jean"-Louis Chevalier, Mariella Colin e Ann Thomson, Paris, Presses
Universitaires de Caen, 1994.

\tit{BELTRÃO}, Luiz. \textit{O índio, um mito brasileiro}, Petrópolis, Vozes, 1977.

\tit{BERGMANN}, Michel. \textit{Nasce um povo: estudo antropológico da
população brasileira; como surgiu, composição racial, evolução futura}, 
Petrópolis, Vozes, 1978.

\tit{BUENO}, Francisco da Silveira. \textit{Vocabulário
Tupi"-Guarani/Português}, 5ª ed., São Paulo, Brasillivros, 1982.

\tit{CASTRO}, Eduardo Viveiros de. \textit{Arawete. Os Deuses Canibais}, Rio
de Janeiro, Jorge Zahar Ed., 986.

\tit{CLASTRES}, Hélène. \textit{Terra Sem Mal. O Profetismo Tupi"-Guarani}, 
trad. port., São Paulo, Brasiliense, 1978.

\tit{CUNHA}, Antonio Geraldo da. \textit{Dicionário Histórico das Palavras
Portuguesas de Origem Tupi}, 3ªed., São Paulo, Melhoramentos, 1989

\titidem. \textit{Antropologia do Brasil: mito,
história, etnicidade}, São Paulo, Brasiliense, 1986.

\titidem, ``Imagens de índios do Brasil: o Século
\textsc{xvi}'', in \textit{Estudos Avançados da} \textsc{usp}, São Paulo, 1990.

\titidem (dir.) \textit{História dos índios do
Brasil}, São Paulo, Companhia das Letras, 1992.

\tit{Destins Croisés. Cinq siècles de rencontres avec les
Amérindiens}, Paris, Albin Michel/Unesco, 1992. 

\tit{DIAS}, Maria da Conceição Osório. \textit{O índio do Brasil na
Literatura Portuguesa dos Séculos \textsc{xvi}, \textsc{xvii} e \textsc{xviii}}, tese de
licenciatura apresentada à Faculdade de Letras da Universidade de Coimbra, 1961.

\tit{DIEGUES JÚNIOR}, Manuel. \textit{Etnias e Culturas do Brasil}, Rio de
Janeiro, Ed. Letras e Artes, 1963.

\tit{DOURADO}, Mecenas. \textit{A Conversão do Gentio}, Rio de Janeiro, Liv.
São José, 1958.

\tit{DUCHET}, Michèle (dir.) \textit{L'Amérique de Théodore de Bry.}
\textit{Une collection de voyages protestante du \textsc{xvi} siècle.}
\textit{Quatre Études d'iconographie}, Paris, Centre National de la
recherche cientifique, s.d.

\titidem. \textit{Anthropologie et histoire au siècle des
Lumières}, Paris, Albin Michel, 1971.

\tit{ELLIOT}, J. H. \textit{O Velho e o Novo Mundo}, Lisboa, Querco, 1984. 

\tit{FERNANDES}, Florestan. \textit{A Função Social da Guerra na Sociedade
Tupinambá}, São Paulo, Ed. Museu Paulista, 1952.

\titidem. \textit{A Investigação Etnológica e outros
ensaios}, Petrópolis, Vozes, 1975.

\titidem. \textit{Organização Social dos Tupinambás}, 
2ªed., São Paulo, Difusão Europeia do Livro, 1963.

\tit{FIGUEIREDO}, Lima. \textit{Índios do Brasil}, São Paulo, Companhia Ed.
Nacional, 1939.

\tit{FRANCO}, Affonso Arinos de Mello. \textit{O índio Brasileiro e a
Revolução Francesa}, Rio de Janeiro, Liv. José Olympio, 1933.

\tit{GAMBINI}, Roberto. \textit{O espelho índio e a destruição da alma
indígena}, Rio de Janeiro, Espaço e Tempo, 1988.

\tit{GOMES}, Mércio Pereira. \textit{Os índios e o Brasil}, Petrópolis, Vozes, 1988.

\tit{HEMMING}, John. \textit{Red Gold: The Conquest of the Brazilian
Indians, 1500--1760}, Cambridge, Harvard University Press, 1978.

\tit{Humanismo y Visión del Otro en la España Moderna: cuatro
estudios}, coord. Berta Aires, Jesus Bustamante, Francisco
Castilla e Fermin del Pino, Madrid, Consejo Superior de
Investigaciones Científicas, 1992.

\tit{JOURDIN}, M. Mollat du, ``L'Alterité, découverte des découvertes'', in
\textit{Voyager à la Renaissance}, Actas do Colóquio de Tours, Paris,
Maisoneuve et Larose, 1987.

\tit{LESTRINGANT}, Frank. \textit{Le Huguenot et le Sauvage}, Paris, Aux
Amateurs de Livres, 1990.

\titidem. \textit{Le cannibale}, Paris, Perrin, 1994.

\tit{LÉVY"-STRAUSS}, Claude. \textit{Tristes Trópicos}, Lisboa, Edições 70,
1979.

\tit{MAESTRI}, Mário. \textit{A Terra dos Males sem Fim. Agonia Tupinambá no
Litoral Brasileiro, Século \textsc{xvi}}, Porto Alegre/Bruxelas, 1990/1991.

\tit{MARCHANT}, Alexander. \textit{Do Escambo à escravidão: as relações
económicas de Portugueses e índios na colonização do Brasil, 1500/1580}, 
São Paulo, Ed.~Nacional, 1980.

\tit{MELATTI}, Júlio Cezar. \textit{Los índios del Brasil}, México, Sep
Setentas, 1973.

\tit{METRAUX}, Álfred. \textit{A Religião dos Tupinambás}, trad. port. de
Estevão Pinto, São Paulo, Companhia Ed. Nacional, 1950.

\titidem. \textit{La Civilization matérialle des tribus
Tupi"-Guarani}, Paris, Paul Geuthner, 1928.

\titidem. \textit{Religions et magies indiennes de l'Amérique
du Sud}, Paris, Gallimard, 1989.

\tit{MONTEIRO}, John Manuel, ``Brasil indígena no século \textsc{xvi}: dinâmica
histórica tupi e as origens da sociedade colonial'', in \textit{Ler
História}, nº 19, Lisboa, 1990, pp. 91--103.

\titidem. \textit{Negros da Terra, índios e Bandeirantes
nas origens de São Paulo}, São Paulo, Companhia das Letras, 1994.

\tit{MARGARIDO}, Alfred, ``La Vision de l'Autre (Africains et Indiens
d'Amérique) dans la Renaissance Portugaise'', in \textit{Actes du
Coloque Internacional d'Études Humanistes}, Tours, 1978, Paris, Centro
Cultural Português, 1984, pp. 507--555. 

\tit{PINTO}, Estevão. \textit{Os Indígenas do Nordeste}, 2 vols.,
São Paulo, s. ed., 1935,

\tit{PROUS}, André. \textit{Arqueologia Brasileira}, Brasília, s.~ed., 1992.

\tit{RAMINELLI}, Ronald. \textit{Imagens da Colonização. A representação do
índio de Caminha a Vieira}, Rio de Janeiro, Jorge Zahar Ed.,
1996. 

\tit{RAMOS}, Alcida Rita. \textit{Sociedades Indígenas}, São Paulo, Editora
Ática, 1986.

\tit{RIBEIRO}, Berta G. \textit{O índio na Cultura Brasileira}, 2ª ed., Rio
de Janeiro, Revan, 1991.

\titidem. \textit{O índio na História do Brasil}, 6ª ed., São
Paulo, Global, 1989.

\tit{RIBEIRO}, Darcy. \textit{Os índios e a Civilização}, 6ª ed.,
Petrópolis, Vozes, 1993.

\tit{RODRIGUES}, Aryon Dall'Igna. \textit{Línguas Brasileiras. Para o
Conhecimento das Línguas Indígenas}, São Paulo, Loyola, 1987.

\tit{SCHADEN}, Egon. \textit{A mitologia heroica das tribos indígenas do
Brasil. Ensaio etnossociológico}, São Paulo, Ed. da Universidade, 1989.

\tit{SOUZA}, Laura de Mello. \textit{O diabo e a Terra de Santa Cruz:
feitiçaria e religiosidade popular no Brasil colonial}, São Paulo,
Companhia das Letras, 1986.

\titidem. \textit{Inferno Atlântico; demologia e
colonização, séculos \textsc{xvi}-\textsc{xviii}}, São Paulo, Companhia das Letras, 1993.

\tit{THOMAS}, Georg. \textit{A Política Indigenista no Brasil 1500--1640}, 
São Paulo, Loyola, 1982.

\tit{TODOROV}, Tzetan. \textit{A Conquista da América: a questão do outro}, 
Lisboa, Litoral Edições, 1990.

\tit{VAINFAS}, Ronald. \textit{Ideologia e Escravidão}, Petrópolis, Vozes,
1986.

\titidem, ``Idolatrias luso"-brasileiras: as santidades
indígenas'', in \textit{América em tempo de conquista}, Rio de Janeiro,
Jorge Zahar Ed., 1992.

\titidem. \textit{A heresia do trópico, santidades ameríndias
no Brasil colonial}, tese de professor titular apresentada ao CEG"-ICHF,
Universidade Federal Fluminense, Niterói, 1993.
\end{bibliohedra}
 



