\chapterspecial{O romance nacional sueco}{}{Leon Rabelo}
\hedramarkboth{O romance nacional sueco}{leon rabelo}



\epigraph{Strindberg tem me perseguido a vida inteira: já o amei, já o odiei, 
joguei seus livros contra a parede; mas livrar-me dele, jamais.}{\textsc{ingmar bergman}}

\section{Sobre o autor}


\noindent\textsc{Todo povo} possui imagens icônicas de si mesmo, 
malgrado a consciência de que elas às vezes escondem as nuances e 
reduzem a complexidade de sua história cultural a um punhado de elementos. 
O que significa pertencer a um determinado povo, quais são seus traços fundamentais? 
Respostas imediatas a esse tipo de pergunta têm contraindicações óbvias, e
mesmo assim as damos mais frequentemente do que gostaríamos de admitir.
Seja por hábito inconsciente, seja por preguiça intelectual, acabamos
recorrendo a um repertório estabelecido de referências que se impõe --- ainda que
momentaneamente --- sobre as controvérsias.  Pois afinal: como de outra
forma unir sob uma mesma bandeira esse nó de disparidades e
incoerências chamado “nação”?  Para a Suécia e os suecos, 
\textit{Gente de Hemsö} é uma das obras que conseguem essa proeza. 

A ironia --- que não raramente acompanha tais casos --- é que seu autor,
August Strindberg, jamais se prestou a esse papel, nem muito menos era
esse o intento original do seu livro. Strindberg jamais foi simplista,
nunca deixou de duvidar de tudo e principalmente de si mesmo. Como ele
próprio se descreve, numa de suas citações mais conhecidas: ``\textit{Eu
sou um homem maldito, o que sei são muitas artes}''. Em relação ao seu
país de origem, a Suécia, foi tão meticuloso quanto pôde na exposição
das suas contradições, abraçando a polêmica sempre que esta se lhe
avistava. Poucos rejeitaram, como Strindberg, os modismos e as
convenções. Fustigava sem piedade os seus conterrâneos, tornando"-se
problemático aos olhos de muitos destes e incômodo para o gosto
convencional. Portanto, é desconcertante --- e paradoxalmente explicável
– que esse mesmo autor tenha criado algumas das imagens mais
persistentes sobre a natureza física e humana de seu país natal, 
firmando"-se como poucos no cânone cultural de seu povo. 

Alguns anos antes da publicação de \textit{Gente de Hemsö}, em 1887, Strindberg já experimentara
sua primeira consagração literária. Obras como
\textit{Röda rummet} (Quarto vermelho) e a peça teatral \textit{Mäster Olof} (Mestre Olof) 
conquistaram logo o público pelo domínio da linguagem, pela forma
precisa em que os motivos interiores de seus personagens eram
retratados e pela naturalidade de suas ações dramáticas. Na
literatura sueca, Strindberg está entre os primeiros que
verdadeiramente levam o texto além do legado novecentista
e lhe abre inauditos horizontes. Nesse sentido, suas
escolhas estilísticas já se anunciavam: ele buscava a clareza, o tom
direto e o despojamento de meios. Ele se lamentava: 

\begin{hedraquote}
A maioria acredita ser profundo aquilo que apenas é rebuscado. Isso é
errado. O rebuscado é apenas o mal"-realizado; o obscuro costuma ser
falso. A sabedoria mais alta é simples, clara e atravessa o cérebro até
o coração.
\end{hedraquote}

%\section{O crítico polêmico}

Outra marca distintiva de Strindberg, bastante conhecida: ele
nasce, vive e morre como um autor crítico. Para ele, a crítica será uma
verdadeira vocação, uma sina e, muitas vezes, sua maior desgraça pessoal. 

A relação problemática de Strindberg com seu contexto intelectual e
literário é um exemplo disso. Pertencente à geração “dos anos 1880”, 
num grupo em que figuravam autores como Ola Hansson, Tor Hedberg,
Oscar Levertin e Axel Lundegård, Strindberg envolvia"-se muitas vezes em
brigas terríveis com seus contemporâneos. É famosa, por exemplo, sua
contenda com o nascente movimento feminista sueco. Muitas de suas
integrantes eram escritoras importantes, como Anne Charlotte
Leffler e Victoria Benedictsson. 

Strindberg atacava as feministas com
uma veemência que só pode ser classificada de irracional e que o torna alvo de 
críticas até hoje, principalmente em seu próprio país. Repare"-se a contradição: apesar de
Strindberg, já em 1884, ter se expressado a favor do sufrágio feminino, e de ter sempre 
se envolvido e convivido com mulheres independentes e emancipadas, 
sua postura intelectual com relação às mulheres é
hoje chamada por alguns de misógina. Sem querer,
em absoluto, defender todas as posições de Strindberg, talvez se pudesse fazer a
ressalva de que muitos dos seus argumentos se revestem de uma aguda inteligência,
com observações ao menos merecedoras de reflexão e réplica.

Além da questão feminista, Strindberg também exercitava seu ímpeto 
iconoclasta em diversas incursões pela ensaística, metendo"-se em assuntos
históricos suecos, discutindo identidades, valores e processos de
formação nacional. Acabou se vendo jogado no furacão de um estrondoso debate
público, atraindo a ira da parte conservadora de seu público leitor. Em
tom acrimonioso, a propósito da publicação de sua obra histórica
\textit{O povo sueco}, ele observa numa carta a seu amigo, o
desenhista Carl Larsson: “Aqui as balas estão zunindo nos
ouvidos, mas eu apenas levanto minha perna e urino sobre a coleira”. 

\subsection{Strindberg e Bergman}

Outro ícone sueco, Ingmar Bergman, sempre teve
Strindberg em sua obra. Entre o autor e o cineasta, cujas vidas
artísticas estão separadas por menos de meio século, há permanente,
tensa e produtiva relação de influência. Bergman dirigiu várias das
peças teatrais de Strindberg e foi marcado, sobretudo, por suas fases
criativas posteriores, onde ele entraria em densos e mais sombrios
territórios da psique humana e das aporias dos relacionamentos
afetivos. Ambos compartilham de uma série de paixões e vicissitudes: a
sofisticação nos traços psicológicos, o esmero técnico, nunca
retroceder diante de uma contradição, a relação difícil que tiveram com
seu entorno cultural e pessoal, a crítica implacável aos moralismos e
às convenções, a universalidade de seus questionamentos quanto à
condição humana. Sobretudo, eram dois artistas que sofriam da mesma
inescapável necessidade de \textit{dizer tudo}, não importando o preço
a pagar. E não se pode esquecer que eles também compartilhavam
o mesmo humor momentâneo, a leveza inesperada, o lirismo e a esperança
final de redenção. 

Em filmes como \textit{Monika e o desejo}, \textit{O sorriso de
uma noite de amor}, \textit{Sétimo selo} e \textit{Persona} podemos ver
como Bergman nos apresenta o mesmo retrato da natureza sueca que
Strindberg descreve em seus textos, uma natureza que mantém com os
personagens uma relação a um só tempo silenciosa e devastadora. Nessas
imagens, bem como em muitos outros aspectos, certamente Bergman deve
muito a Strindberg.  Talvez não seja por acaso que Bergman tenha
terminado seus dias afastado do mundo, isolado também ele numa ilha do
litoral sueco. 


%\medskip
%\section{Um livro escrito no exílio}
\section{Sobre a obra}

Toda essa contenda acaba por cobrar seu preço: em 1883 Strindberg parte rumo 
à França e à Suíça para um exílio voluntário que duraria seis anos. E foi assim, numa
situação de afastamento de seu país e com o intuito de reconquistar
seus leitores, que ele concebe \textit{Gente de Hemsö}. Talvez pela saudade,
talvez por vingança, ele se propôs nessa obra a recriar o ambiente natural e
humano dos arquipélagos da Suécia, lugar de nascimento de sua cultura
há milênios. Estendendo"-se ao longo de quase todo o litoral do
país, foi nesse mundo de águas e ilhas que a natureza e o clima se
entrelaçaram na própria identidade do povo sueco. Basta lembrar que a
palavra “enseada”, em sueco, “\textit{vik}”, está na raiz da palavra “viking”,
que por sua vez significa “habitante de enseada”. 

Strindberg conhecia bem esse universo, tendo passado vários verões na
ilha de Kymmendö, no arquipélago ao sul de Estocolmo, e visitando"-a
pela primeira vez em 1871. É essa ilha que ele usará como modelo para a
sua fictícia ilha de Hemsö. Está claro que a cultura dos arquipélagos suecos,
na época de Strindberg, não se assemelhava muito --- ao menos exteriormente ---
ao passado mítico dos nórdicos, tendo já passado por mil anos de cristianismo
e quatrocentos de modernidade.  Ainda assim, encontrava"-se lá
preservada uma quase intocada relação originária das pessoas com o seu
entorno natural, com as condições climáticas e com as tradições milenares. Esses
elementos serão a espinha dorsal da nova obra. 

Nas próprias palavras de Strindberg para o editor Pehr Staaf:

\begin{hedraquote}
Trata"-se de um retrato em forma de romance sobre a vida rural
nos arquipélagos suecos, o primeiro romance, genuíno, que escrevi.
Repleto de paisagens suecas, camponeses, homens ilustrados, pastores,
capelães etc. Coisas belas, coisas feias, coisas tristes, alegres,
cômicas, vívidas, todas elas publicáveis. E se algo estiver forte além
do costume, pode ser suprimido ou amenizado.
\end{hedraquote}

Mais adiante, ele escreve:

\begin{hedraquote}
Se de fato meu nome estiver tão impopular, tão odiado, como o
velho ladrão Albert Bonnier\footnote{ Albert Bonnier (1820---1900), um 
dos principais editores suecos, fundador da Bonnier.} por anos tenta me fazer crer, então não restará outro
remédio que não escrevê"-lo popular novamente, o que talvez não seja  %"escrevê-lo popular" é estranho. Inscrevê-lo? Torná-lo?
impossível, já que retornei, de corpo e alma, após uma série de
experimentos, para a literatura artística, usando a psicologia
moderna como ferramenta de ajuda.
\end{hedraquote}

Estão aí os ingredientes principais de \textit{Gente de Hemsö},
apenas que debaixo da simplicidade da descrição não aparece a
maestria com que eles foram trabalhados. Pois longe de uma mera
descrição bucólica da natureza, longe de colocá"-la apenas como pano de
fundo para sua narrativa, Strindberg dá a ela uma radical identidade
própria, ora terrível, ora imensamente sedutora. A natureza, em
Strindberg, se impõe sem clemência aos personagens, malhando"-os
incessantemente, formando"-os e, às vezes, devorando"-os. Percebe"-se a
grande alegria com que Strindberg lhe dá voz própria e escuta seus
rumores, como ele nos mostra o mar e a terra, o sol e a neve, descrevendo
seu imemorial pêndulo de adversidade e bonança. 

Quanto aos aspectos humanos desse universo, quanto aos seus personagens
e à tal “ferramenta da psicologia moderna” com que Strindberg
os descreverá, é evidente que não há aqui nada de esquemático. Apenas,
talvez, quanto aos encadeamentos de certos aspectos na narrativa,
notamos um certo artificialismo quando Strindberg procura conferir ao enredo 
um tom exemplar. Mas os personagens de Strindberg, mesmo os
menores, não estão ali como fantoches para quaisquer teses. Ou melhor:
embora Strindberg certamente tivesse uma opinião formada sobre o
microcosmo que ele retratava, encenando"-a em seus personagens, ele
nunca a impõe de maneira direta ou grosseira. Como autor diferenciado
que era, Strindberg escondia muito bem as costuras de sua criação e nos
dá criaturas que, mais do que expressão de uma visão particular, são
dotados de vida autônoma. Seja na expressão dos dilemas e agruras da
vida nos arquipélagos, seja na forma que essa vida se prestava a ser
plataforma literária de Strindberg no ataque que ele impetrava contra
os moralismos das classes urbanas leitoras da época, temos aqui um
autor que sabia desaparecer atrás de sua obra e nos brindar com um
texto vivo e fluído. 

O ataque contra todas as posições de moralismo, aliás, é uma constante
em Strindberg, que ele cultivava de forma obsessiva e que
muitas vezes era sua grande diversão. Em \textit{Gente de Hemsö}, a
rusticidade da vida rural é pintada em tintas fortes e é evidente como
o autor a apresenta como uma alternativa à moralidade burguesa da
cidade. Tomemos, por exemplo, a maneira com que Strindberg retrata a
magnífica figura do pastor Nordström. Esse homem, oriundo da cidade, 
de uma cultura refinada, e desde a juventude sabedor de latim e
dogmática, vai se transformando ao longo dos muitos anos vividos
no arquipélago, tornando"-se finalmente quase indistinguível dos rudes
pescadores locais. Plenamente imerso nos costumes e no
linguajar destes, ele já não tem papas na língua e fala para sua
congregação nos próprios termos dela. O anticlerical Strindberg não deixa
passar essa oportunidade para descer o malho no que ele achava ser os
exageros do luteranismo sueco da época, e isso de uma maneira bastante
ousada para seus contemporâneos. 

Portanto, e apesar de Strindberg ter prometido aos seus editores 
amenizar as passagens mais fortes, houve uns tantos trechos da obra
original que foram censurados nas primeiras edições. Em especial, foram
cortados os trechos onde o pastor dá vazão ao seu pesado humor e seu 
apreço às meninas, o que foi considerado inapropriado para as partes 
mais delicadas do público. A essa altura, esquecido da própria promessa, 
Strindberg explode contra ``as imbecis tentativas de castração por parte 
do editor, apenas pela suposição de que o público elegante não poderia
suportar algumas passagens apenas um pouco mais realistas''.\footnote{ Essas passagens
censuradas foram incorporadas em edições a partir dos anos 1970, com base 
nos manuscritos originais de Strindberg, e estão presentes nesta tradução.}

De qualquer modo, a estrutura de \textit{Gente de Hemsö} ficará intacta, não
deixando de ser, também, uma obra cheia de alegria vital e com
passagens de impagável humor rústico. 

\section{Sobre o gênero}

O romance de Strindberg aproxima"-se do naturalismo
francês, ligação evidente na maneira escancarada em que os elementos
populares são apresentados, e não nos parece exagero
especular se Strindberg aqui não antecipa um tom quase joyceano. 
Não tanto quanto à técnica modernista, mas na forma despudorada e saborosa em
que ele retrata os aspectos vulgares da existência humana.
Vemos como em \textit{Gente de Hemsö} um autor profundamente intelectual 
e urbano, como era Strindberg, se deixa levar para longe das convenções 
da boa sociedade e se apaixonar pela escatologia e a rudeza da vida,
tentando plasmá"-la em seu texto. Tendo depois \textit{Gente de Hemsö} 
atravessado o longo século \textsc{xx}, é impressionante como essas
passagens funcionam ainda hoje e dão força à obra. 

Essa aproximação com o naturalismo pode ser observada nas muitas passagens em que
Strindberg faz as vezes de um naturalista ou botânico, colecionando com
avidez e prazer um infindável número de espécies animais e vegetais dos
arquipélagos e da natureza escandinava. Peixes, pássaros, árvores,
flores vêm em profusão sobre nós. Em outras horas, Strindberg assume o
papel de um antropólogo e elenca meticulosamente todo um arsenal de
ferramentas de navegação, pesca, caça e trabalho na terra. Nota"-se que
ele reuniu com grande afinco todos esses elementos para a composição de
seu retrato, dando"-lhe, além da qualidade literária, um rico valor documental --- o que torna, sem dúvida, uma empreitada com muitas dificuldades traduzir 
\textit{Gente de Hemsö} para o português de nossos dias.

O mais importante, no entanto, é perceber que a verdadeira intenção de
Strindberg nessas passagens é causar um estranhamento no leitor em
relação ao mundo que ele está pintando, mesmo no leitor sueco de sua
época, habituado à cidade e desacostumado com os arquipélagos. É como
se ele quisesse nos dar a sensação que teve Carlsson, o personagem
principal da obra, ao chegar como intruso na ilha de Hemsö e na vida de
seus habitantes. Carlsson tem a impressão de ter “chegado a uma
terra estranha”. E assim nos sentimos nós, sendo essa sensação
multiplicada ainda mais pela distância geográfica, temporal e cultural
que há entre o mundo de Hemsö e o nosso.

Apesar dessa distância, Strindberg continua a ser muito importante e atual
para toda uma cultura literária e visual dos nossos dias: ele permanece
moderno. Lembremos, ainda, que Strindberg era também um inventivo fotógrafo amador
e um excelente pintor. Na sua obra literária, percebe"-se a proximidade com 
essas outras linguagens que, sem serem subjugadas, harmonizam"-se num todo
artístico fascinante e único. Não é por outra razão, como testemunho do enorme sucesso
que \textit{Gente de Hemsö} obteve na Suécia, que a obra foi levada às telas nada mais
que quatro vezes.  A primeira, numa versão muda, já em 1919. As outras se
seguiram em 1944, 1955, e a última, uma série para \textsc{tv}, em 1966. 
Os suecos simplesmente não conseguiram esquecer essas imagens que
Strindberg lhes legara, tendo de reescrevê"-las incessantemente e 
plasmando"-as em novos espelhos de si mesmos. 



\asterisc

Gostaríamos de agradecer ao diretor do \textit{Projekt Strindberg}, professor
Per Stam, pela valorosa contribuição à presente tradução. O Projeto
Strindberg é mantido pela Universidade de Estocolmo e irá re"-editar,
até o ano do centenário da morte de Strindberg, em 2012, a íntegra de
sua obra escrita --- incluídas as cartas e diversos textos avulsos. 




