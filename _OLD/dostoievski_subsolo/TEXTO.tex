\chapter*{}
\thispagestyle{empty}

\vfill

O autor das memórias e as próprias \emph{Memórias} são, evidentemente, fictícios. Não obstante, pessoas como o responsável pelas memórias não apenas podem como devem existir
em nossa sociedade, levando em consideração as circunstâncias gerais
de sua formação. Quis apresentar ao público, de modo mais evidente que
o habitual, um personagem de uma época recente. Trata"-se de um
representante de uma geração que ainda persiste. Nesse trecho,
intitulado \emph{O Subsolo}, o personagem apresenta a si mesmo, sua
visão do mundo, e tenta esclarecer os motivos pelos quais surgiu e
tinha que surgir em nosso meio. No trecho seguinte, já aparecem as
``memórias'' propriamente ditas do personagem sobre alguns
acontecimentos de sua vida.

\medskip

\hfill{}Fiódor Dostoiévski

\chapter{O subsolo}

\section{I}

Sou um homem doente\ldots{} Sou um homem perverso. Sou um homem nada
atraente. Acho que tenho dor no fígado. Aliás, não entendo patavina da
minha doença e não sei de verdade o que é que me dói. Não me trato, nem
nunca me tratei, embora respeite a medicina e os doutores. Ademais, sou
supersticioso ao extremo; o suficiente para respeitar a medicina. (Sou
instruído o suficiente para não ser supersticioso, mas sou
supersticioso). Sim senhores, não quero me tratar de raiva. Isso os
senhores provavelmente não vão querer compreender. Pois bem, mas eu
entendo. Eu, evidentemente, não vou conseguir explicar quem nesse caso
estou atormentando com a minha raiva; sei muito bem que não tenho como
``prejudicar'' os médicos com os quais não me trato; sei ainda melhor
que, com tudo isso, só prejudico a mim mesmo, ninguém mais. Mas mesmo
assim, se eu não me trato, é de raiva. O fígado está doendo, então que
doa ainda mais!

Já faz muito tempo que vivo desse jeito: vinte anos. Agora estou com
quarenta. Antes eu trabalhava, agora não mais. Eu era um funcionário
público perverso. Era rude, e encontrava nisso satisfação. Já que eu não
levava propina, essa era minha recompensa. (Piada ruim; mas não vou
apagar. Escrevi achando que seria muito espirituosa; agora que vi que
era só desejo de exibicionismo torpe é que não apago, de propósito!)
Quando alguém se aproximava da minha mesa pedindo informação, eu rangia
os dentes e sentia um deleite insaciável ao conseguir causar aflição.
Quase sempre conseguia. A maior parte deles era gente tímida, como se
sabe que são os requerentes. Porém, dentre os almofadinhas, havia um
oficial que eu, particularmente, não podia suportar. Não queria se
submeter de jeito nenhum, e tilitnava o sabre de forma nojenta. Travei
uma guerra de um ano e meio com ele por causa desse sabre. Por fim,
venci. Ele parou de tilintar. A propósito, isso sucedeu ainda na minha
juventude. Mas os senhores sabem no que consistia o ponto principal da
minha raiva? Pois a coisa toda consistia, e aí residia a maior torpeza,
em que eu, a cada instante, na mesma hora da descarga mais forte de
bílis, reconhecia para mim mesmo, envergonhado, que não apenas não era
um homem mau, como nem sequer exaltado, mas que só ficava assustando os
pardais à toa e me divertindo com isso. Minha boca está espumando, mas
basta me trazerem uma boneca e me darem chá com açúçar e eu
provavelmente sossego. Posso até me comover, embora depois possivelmente
vá ficar ragendo os dentes e padecendo de insônia durante meses, de
vergonha. Esse é o meu hábito.

Menti agora há pouco ao dizer que era um funcionário perverso. Menti de
raiva. Só aprontava umas traquinagens com os requerentes e com o oficial
mas, na essência, numca consegui fazer o mal. Reconhecia em mim a cada
instante muitíssimos elementos que se opunham a isso. Sentia esses
elementos que se opunham fervendo dentro de mim. Sabia que ferviam em
meu interior a vida toda, pedindo para sair, só que eu jamais permiti,
não permiti que saíssem de propósito. Atormentaram"-me até a vergonha;
levaram"-me a convulsões e, por fim, me fartaram, como me fartaram! Por
acaso os senhores acham que agora estou me arrependendo, que estou a
lhes pedir perdão por algo?.. Estou certo de que essa é a sua
impressão\ldots{} Aliás, asseguro"-lhes que, para mim, tanto faz o que os
senhores acham\ldots{}

Não apenas não consegui ser mau, como nada mais: nem mau, nem bom, nem
canalha, nem honrado, nem herói, nem inseto. Agora vivo no meu canto,
provocando"-me com o consolo raivoso e sem serventia nenhum de que um
homem inteligente não pode se tornar nada a sério, e que só o estúpido
vira alguma coisa. Sim senhores, o homem inteligente do século \textsc{xix} deve
e tem a obrigação moral de ser uma criatura preponderantemente sem
caráter; uma pessoa de caráter, de ação, é uma criatura
preponderantemente limitada. Essa é a minha convicção de quarenta anos.
Agora estou com quarenta anos, e quarenta anos são toda uma vida; é a
mais provecta velhice. Viver mais do que quarenta anos é indecente,
vulgar, imoral! Respondam de forma franca e honrada: quem vive mais do
que quarenta anos? Eu lhes digo quem vive: os estúpidos e patifes. Digo
isso na cara de todos os velhos, de todos esses velhos respeitáveis, de
todos esses velhos de cabelo prateado e perfumados! Digo isso na cara do
mundo inteiro! Tenho direito de falar assim, porque vou viver até os
sessenta. Vou viver até os setenta! Vou viver até os oitenta!.. Um
minuto! Deixem"-me tomar fôlego\ldots{}

Talvez os senhores achem que desejo diverti"-los. Também estão enganados
a esse respeito. Não sou de jeito nenhum uma pessoa tão alegre quanto
pareço, ou como talvez lhes pareça; aliás, caso, irritados com toda essa
tagarelice (e já estou sentindo que se estão se irritando), inventarem
de perguntar quem exatamente eu sou, responderei: sou
assessor"-colegial\footnote{Oitavo dos catorze graus da tabela de
  patentes que regulamentava o serviço civil na Rússia até a Revolução
  de 1917. {[}\textsc{n.\,t.}{]}}. Trabalhava para ter o que comer (mas só para
isso) e quando, no ano passado, um parente distante me legou seis mil
rublos em testamento, aposentei"-me de imediato e me instalei no meu
canto. Antes eu morava nesse canto, mas agora me instalei nele. Meu
quarto é sujo e nojento, na periferia da cidade. Minha criada é uma
mulher do campo, velha, malvada por estupidez, exalando sempre um fedor
nojento. Dizem que o clima de São Petersburgo me é nocivo e que, com
meus meios insignificantes, é muito caro morar aqui. Eu sei disso tudo,
sei melhor do que todos esses sábios e experientes conselheiros e
indicadores. Só que vou ficar em São Petersburgo; não vou sair de São
Petersburgo\ldots{} Não saio porque\ldots{} Ah! Afinal, dá absolutamente na mesma
se saio ou se não saio.

A propósito: do que uma pessoa direita pode falar com a maior
satisfação?

Resposta: de si mesma. Então agora vou falar de mim.

\section{II}

Agora, senhores, desejo lhes narrar, queiram ou não queiram escutar,
porque não consegui me tornar nem sequer um inseto. Digo"-lhes de forma
solene que muitas vezes desejei me tornar um inseto. Mas não fui digno
nem disso. Juro"-lhes, senhores, que ser consciente demais é uma doença,
uma verdadeira e completa doença. Para uso cotidiano, seria mais do que
suficiente a consciência humana comum, ou seja, a metade, um quarto a
menos do que a porção que cabe à pessoa de nosso infeliz século \textsc{xix} que,
ainda por cima, tem a excepcional infelicidade de habitar em São
Petersburgo, a cidade mais abstrata e premeditada de todo o globo
terrestre. (Há cidades premeditadas e não premeditadas). Teria sido
totalmente suficiente, por exemplo, a consciência com que vivem as assim
chamadas pessoas espontâneas e de ação. Aposto que os senhores acham que
escrevo tudo isso por gabolice, para zombar das pessoas de ação e que,
sempre por gabolice, fico tilintando o sabre de um jeito besta, que nem
meu oficial. Porém, senhores, quem poderia se vangloriar e ainda fazer
gabolices com suas próprias doenças?

Aliás, o que é que estou dizendo? Todos fazem isso, vangloriam"-se de
suas doenças, e talvez eu mais do que todos. Não discutamos; minha
objeção é insensata. Porém, mesmo assim estou fortemente convicto de que
não apenas muita consciência, como qualquer consciência é uma doença.
Insisto nisso. Deixemos isso de lado por um minuto. Digam"-me o seguinte:
por que acontecia que, como se fosse de propósito, naqueles mesmos, sim,
naqueles mesmos instantes em que eu estaria mais propício a reconhecer
toda a fineza de ``tudo que é belo e sublime''\footnote{A união dos
  conceitos de ``belo e sublime'' remonta a tratados estéticos do século
  \textsc{xviii} (por exemplo, ``Uma Investigação Filosófica acerca da Origem das
  nossas Ideias do Sublime e do Belo, de Edmund Burke (1756);
  ``Observações sobre o Sentimento do Belo e do Sublime'', de Kant
  (1764); e outros). Depois da revisão da estética da arte ``pura'' nos
  anos 1840--1860, essa expressão adquiriu matiz irônico. {[}\textsc{n.\,t.}{]}},
como se dizia entre nós certa época, ocorria"-me de não apenas pensar
como cometer feitos tão indecorosos que\ldots{} bem, em suma, que todos
talvez cometam, mas os quais, como que de propósito, ocorreram"-me
exatamente quando eu mais tinha consciência de que não devia cometê"-los?
Quanto mais eu pensava no bem e em todo esse ``belo e sublime'', mais
fundo baixava no lodo, e mais apto ficava a me atolar nele por completo.
O mais importante, porém, era que nada disso me parecia casual, mas sim
que era como tinha que ser. Como se essa fosse minha condição mais
normal, e de jeito nenhum uma doença ou uma deterioração, tanto que, por
fim, passou o desejo de lutar contra ela. Quase acabei acreditando (ou
talvez tenha mesmo acreditado) que talvez essa fosse minha condição
normal. Mas primeiramente, no começo, quantos tormentos suportei nessa
luta! Não acreditava que o mesmo acontecesse com os outros e, por isso,
guardei o segredo comigo a vida inteira. Tinha vergonha (e talvez tenha
até hoje); cheguei ao ponto de sentir um prazer secreto, anormal,
canalha ao regressar para meu canto em uma noite abjeta de São
Petersburgo e admitir com vigor que voltara a cometer uma torpeza, que
ela era irreversível e, lá dentro, em segredo, roía, roía, serrava e
chupava até que, por fim, a amargura se convertia em uma doçura infame e
maldita e, finalmente, em um prazer decidido e sério! Sim, um prazer, um
prazer! Insisto nisso. Comecei a falar disso porque quero realmente
saber: os outros também têm esse tipo de prazer? Explico"-me: esse prazer
residia exatamente na consciência bastante clara de minha humilhação; de
sentir ter chegado ao grau mais baixo; de que era imundo, e não tinha
como ser diferente; de que não tinha escapatória, de que jamais seria
uma pessoa diferente; de que, ainda que me restasse tempo e fé para me
transformar, certamente não desejaria tal transformação; e de que, ainda
que desejasse, nem assim faria alguma coisa, pois talvez não houvesse em
que me transformar. Mas o mais importante e o ponto final é que tudo
isso acontece de acordo com as leis normais e fundamentais da
consciência reforçada, pela inércia que emana diretamente dessas leis e,
como consequência, você não apenas não se transforma, como não faz
simplesmente nada. Resulta, por exemplo, como consequência da
consciência reforçada: é verdade que você é um canalha, como se fosse um
consolo para o canalha sentir que ele é mesmo um canalha. Mas chega\ldots{}
Ai, falei um monte de besteira, mas expliquei o quê?.. Como explicar
esse prazer? Mas vou explicar! Vou levar até o fim! Peguei a pena para
isso\ldots{}

Por exemplo, tenho um amor"-próprio terrível. Sou desconfiado e
suscetível como um corcunda ou um anão mas, na verdade, ocorreram"-me uns
instantes nos quais, se tivessem me dado uma bofetada, talvez eu tivesse
ficado até feliz. Estou falando sério: eu provavelmente teria conseguido
encontrar mesmo aí algum tipo de prazer, obviamente o prazer do
desespero, só que o desespero também tem os prazeres mais ardentes, em
especial quando você reconhece com muita força o caráter inescapável de
sua posição. Quanto à bofetada, com ela você é esmagado pela consciência
da papa a que foi reduzido. O principal é que, por mais que rumine, o
resultado é que sempre sou o primeiro dentre os culpados e, ainda mais
ultrajante, culpado sem culpa e, por assim dizer, de acordo com as leis
da natureza. Em primeiro lugar, sou culpado por ser mais inteligente do
que todos que me rodeiam. (Sempre me achei mais inteligente do que todos
que me rodeavam e, às vezes, acreditem, até me envergonhava disso. Pelo
menos, passei a vida olhando meio de lado, e jamais consegui olhar as
pessoas nos olhos). Por fim, sou culpado porque, se houvesse em mim
magnanimidade, haveria também tormento maior, devido à consciência de
sua total inutilidade. Afinal, eu provavelmente não saberia o que fazer
com minha magnanimidade: nem perdoar, já que o ofensor poderia ter me
golpeado de acordo com as leis a natureza, e as leis da natureza não
podem ser perdoadas nem esquecidas, pois, ainda que sejam leis da
natureza, sempre é ofensivo. Por fim, ainda que eu não tivesse vontade
alguma de ser magnânimo e, pelo contrário, desejasse me vingar do
ofensor, não conseguiria vingança nenhuma, já que provavelmente não me
decidiria a fazer nada, mesmo que pudesse. Por que não me decidiria? A
respeito disso em particular, quero dizer umas duas palavras.

\section{III}

Pois as pessoas que sabem se vingar e, em geral, se defender, como elas
fazem, por exemplo? Afinal, quando são tomadas, digamos, pelo sentimento
de desforra, nessa hora nada mais há ou sobra em seu ser além desse
sentimento. Esse senhor se lança direto contra seu objetivo, como um
touro furioso, com os chifres para baixo, e só um muro pode detê"-lo.
(Aliás: diante de um muro, esses senhores, ou seja, os homens
espontâneos e de ação, se dão francamente por vencidos. Para eles, o
muro não é um desvio, como por exemplo para nós, pessoas reflexivas que,
em consequência, nada fazem; não é um pretexto para dar para trás, um
pretexto no qual normalmente não acreditamos, mas que sempre nos deixa
muito felizes. Não, eles se dão por vencidos com toda franqueza. Para
eles, o muro significa algo tranquilizador, uma solução moral e final,
talvez até algo mística\ldots{} Mas depois falamos do muro). Pois bem, esse
homem espontâneo eu considero o homem verdadeiro e normal, como a terna
mãe natureza gostaria de vê"-lo, ao engendrá"-lo na terra, com amor.
Invejo esse homem com o máximo da bílis. É estúpido, não vou discutir,
mas talvez o homem normal tenha que ser estúpido, como saber? Talvez
isso seja até muito belo. Fico ainda mais convicto dessa, digamos,
suspeita quando, por exemplo, tomamos a antítese do homem normal, ou
seja, o homem de consciência reforçada, saído, naturalmente, não do seio
da natureza, mas da retorta (isso já é quase misticismo, senhores, mas
também suspeito disso), ou seja, o homem da retorta às vezes se dá tão
por vencido por sua antítese que ele mesmo, com toda sua consciência
reforçada, considera"-se escrupulosamente um camundongo, e não um homem.
Talvez um camundongo de consciência reforçada, mas mesmo assim um
camundongo, enquanto o outro é um homem e, consequentemente\ldots{} e assim
por diante. O principal é que ele, ele mesmo se considera um camundongo;
ninguém lhe pediu isso; esse é um ponto importante. Contemplemos agora
esse camundongo em ação. Suponhamos, por exemplo, que também esteja
ofendido (e ele quase sempre está ofendido), e também deseje vingança.
Nele talvez ainda haja mais raiva acumulada que no \emph{l'homme de la
nature et de la vérité}\footnote{O homem da natureza e da verdade, em
  francês no original. Alusão ao pensador suíço Jean"-Jacques Rousseau
  (1712--1778). {[}\textsc{n.\,t.}{]}}. O desejo abjeto e baixo de retribuir
o mal ao ofensor talvez o arranhe de forma ainda mais abjeta que
\emph{l'homme de la nature et de la vérité,} pois \emph{l'homme de la
nature et de la vérité,} em sua estupidez inata, julga sua desforra pura
e simples justiça; já o camundongo, em consequência da consciência
reforçada, desmente tal justiça. Chegamos enfim à ação em si, ao ato de
desforra. O infeliz camundongo, para além da vileza inicial, conseguiu
amontoar ao seu redor, na forma de perguntas e dúvidas, outras tantas
vilezas; uma pergunta levou a tantas perguntas insolúveis que, sem
querer, ao seu redor se acumulou uma lavagem tão funesta, uma sujeira
tão fétida, constituída por suas dúvidas, agitações e, por fim, pelas
cuspidas das pessoas espontâneas que ficam ao seu redor, solenes, sob a
forma de juízes e ditadores, rindo dele a não mais poder, com toda a
saúde de suas gargantas. Obviamente, o que lhe resta é dar adeus a tudo
com sua pata e, com um sorriso de desprezo afetado, no qual nem ele
acredita, esgueirar"-se envergonhado para sua fenda. Lá, em seu subsolo
infame e fétido, nosso camundongo ofendido, batido e ridicularizado
mergulha sem demora em uma raiva fria, venenosa e, principalmente,
perene. Por quarenta anos seguidos vai recordar sua ofensa nos mínimos e
mais vergonhosos pormenores e, além disso, a cada vez acrescentará, por
si, promenores ainda mais vergonhosos, provocando e irritando a si
mesmo, raivosamente, com sua própria fantasia. Vai se envergonhar de sua
fantasia mas, mesmo assim, recordar tudo, rever tudo, imaginar histórias
fantásticas sob o pretexto de que também podiam ter acontecido, e nada
perdoará. Talvez até comece a se vingar, só que de forma irregular, em
miudezas, por debaixo dos panos, incógnito, sem acreditar em seu direito
de se vingar, nem no êxito da vingaça, e sabendo de antemão que todas
essas tentativas de desforra o farão sofrer cem vezes mais do que aquele
de que se vinga, o qual talvez não sinta nem sequer uma coceira. No
leito de morte, voltará a se lembrar de tudo, com os juros acumulados
esse tempo todo, e\ldots{} Mas é exatamente nesse semidesespero frio e
asqueroso, nessa semicrença, nesse sepultamento consciente de si mesmo
vivo, por pesar, no subsolo, por quarenta anos, nessa consciência
reforçada e mesmo assim parcialmente duvidosa do caráter inescapável de
sua posição, em todo esse veneno dos desejos insatisfeitos que estão lá
dentro, em toda essa febre de hesitações, nas decisões tomadas para
sempre e arrependimentos que voltam a aparecer um minuto depois, aí é
que reside o sumo daquele estranho prazer a que me referi. É tão sutil a
ponto de, às vezes, não se dar a conhecer, e as pessoas levemente
limitadas, ou simplesmente de nervos fortes, não conseguem entendê"-lo
nem um pouco. ``É possível que também não entendam --- os senhores vão
acrescentar, com um sorriso largo --- aqueles que nunca levaram uma
bofetada'' --- fazendo"-me dessa forma notar, com educação, que, em minha
vida, eu talvez também tenha experimentado a bofetada e, portanto, fale
como conhecedor. Aposto que os senhores acham isso. Mas acalmem"-se,
senhores, não levei bofetadas, embora para mim dê absolutamente na mesma
que essa seja a sua opinião. É possível que eu até lamente ter
distribuído poucas bofetadas na vida. Mas chega, nenhuma palavra mais a
respeito desse tema que lhes interessa de forma extraordinária.

Prossigo com calma, falando das pessoas de nervos fortes que não
compreendem um determinado prazer refinado. Esses senhores, em alguns
casos, por exemplo, embora mujam como touros a plena voz, e embora isso,
digamos, confira"-lhes a mais elevada honra, como já disse, diante de uma
impossibilidade, sossegam porém de imediato. Impossibilidade é um muro
de pedra? Que muro de pedra? Bem, obviamente as leis da natureza, as
conclusões das ciências naturais, a matemática. Se, por exemplo,
demonstrarem que você veio do macaco, não adianta franzir o cenho, tem
que aceitar como é\footnote{De acordo com a edição russa, o interesse
  sobre as origens do homem no país se aguçou no começo de 1864, devido
  à publicação, em São Petersburgo, de uma tradução russa do livro
  ``Evidências sobre o lugar do homem na natureza'', de Thomas Henry
  Huley (1825--1895), conhecido como ``o buldogue de Darwin'' por sua
  veemência na defesa da teoria da evolução. É possível que a frase de
  Dostoiévski seja uma alusão a uma polêmica surgida na imprensa da
  época. {[}\textsc{n.\,t.}{]}}. Se demonstrarem que, basicamente, uma gotinha da
sua gordura deve lhe ser mais cara do que cem mil dos seus semelhantes,
e que nesse resultado se resolvem finalmente as assim chamadas virtudes,
obrigações e demais maluquices e preconceitos, você tem que aceitar, não
há o que fazer a respeito, já que dois e dois é a matemática. Tente
retrucar.

``Perdão --- vão lhes gritar ---, não dá para se insurgir; dois e dois são
quatro! A natureza não lhe pede permissão; ela não tem nada a ver com
seus desejos, nem se suas leis lhe agradam ou não. O senhor tem
obrigação de aceitá"-la como é e, em consequência, todos seus resultados.
Quer dizer, é um muro, o muro existe\ldots{} etc, etc''. Senhor Deus, o que
tenho a ver com as leis da natureza e da aritmética quando, por algum
motivo, essas leis e dois e dois não me agradam? Obviamente não vou dar
com a testa nesse muro, já que não reúno forças para tanto, mas não vou
me resignar só porque tem um muro de pedra e minhas forças não bastaram.

Como se um muro de pedra desses realmente contivesse o sossego e
realmente encerrasse em si alguma palavra para o mundo, apenas porque
dois e dois são quatro. Oh, absurdo dos absurdos! Muito melhor
compreender tudo, reconhecer tudo, todas as impossibilidades e muros de
pedra; não se resignar diante de nenhuma dessas impossibilidades e muros
de pedra, se a resignação repugna; chegar, pelo caminho das combinações
lógicas mais inescapáveis, às conclusões mais repulsivas sobre o tema
eterno se é possível ter culpa diante do muro de pedra, embora novamente
se faça ver com clareza que não há culpa alguma e, em consequência
disso, rangendo os dentes em silêncio e, impotente, paralisar"-se com
volúpia na inércia, sonhando que não há nem contra quem se irritar; que
não há objeto e, talvez, jamais se encontre, que há aí uma fraude, um
engodo, uma trapaça, simplesmente uma lavagem; não se sabe o quê, não se
sabe quem mas, apesar de todo esse desconhecimento e engodo, sente"-se
dor de verdade e, quanto menos se sabe, maior a dor!

\section{IV}

--- Ha"-ha"-ha! Depois disso, o senhor vai descobrir prazer até em dor de
dente! --- os senhores gritarão, entre risos.

--- Mas e daí? Mesmo na dor de dente há prazer --- respondo. --- Meus dentes
doeram um mês inteiro; sei o que é isso. Daí é claro que não é o caso de
se irritar em silêncio, mas de gemer; só que esses gemidos não são
sinceros, são gemidos com malícia, e toda a coisa está nessa malícia.
Nesses gemidos também se expressa o prazer do sofredor; se não
encontrasse prazer nisso, não se poria a gemer. É um ótimo exemplo,
senhores, e vou desenvolvê"-lo. Nesses gemidos se exprime, em primeiro
lugar, toda inutilidade da dor, que é humilhante para nossa consciência;
toda a legitimidade da natureza, que os senhores obviamente desprezam,
mas com a qual sofrem assim mesmo, e ela não. Expressa"-se também a
consciência de que não existe um inimigo, mas a dor sim, a consciência
de que os senhores, com todos os Wagenheim\footnote{De acordo com a
  edição russa, havia oito dentistas com o nome Wagenheim na lista de
  endereços de São Petersburgo, em meados da década de 1860, espalhados
  por toda a cidade. {[}\textsc{n.\,t.}{]}} possíveis, são totalmente escravos dos
seus dentes; de que, se alguém quiser, seus dentes param de doer e, se
não quiser, vão ficar doendo por mais três meses; e de que, por fim, se
os senhores ainda não estiverem de acordo e protestarem assim mesmo, o
único consolo que resta é espancar a si mesmos ou bater com o punho no
muro da forma mais dolorosa; mais do que isso, decididamente, não há.
Pois bem, por essas ofensas sangrentas, por essas zombarias por parte de
alguém desconhecido é que começa, finalmente, o prazer, que às vezes
chega à mais alta voluptuosidade. Peço"-lhes, senhores, que apurem alguma
vez os ouvidos aos gemidos do homem instruído do século \textsc{xix} que padece
dos dentes, no segundo ou terceiro dia da dor, quando já começa não mais
a gemer como no primeiro dia, ou seja, apenas porque os dentes doem; não
como um mujique rude, mas como um homem tocado pelo desenvolvimento e
pela civilização europeia, como um homem ``que renunciou ao solo e aos
princípios populares''\footnote{Expressão característica das revistas
  \emph{Tempo} e \emph{Época,} editadas por Dostoiévski. Presente, por
  exemplo, em anúncios de 1861, 1862 e 1863 de \emph{Tempo,} ou no
  artigo do escritor ``Sobre os novos órgãos literários e novas
  teorias'' (1863). {[}\textsc{n.\,e.}{]}}, como agora dizem. Tais gemidos se
tornam algo nojentos, de uma raiva obscena, prolongando"-se por dias e
noites inteiros. E ele bem sabe que os gemidos não lhe trazem proveito
algum; sabe melhor do que todos que apenas está a dilacerar e irritar os
outros inutilmente; sabe que até o público perante o qual sofre e toda
sua família ouvem"-no com asco, não botam um tostão de fé nele e entendem
que poderia gemer de outro jeito, mais simples, sem trilos nem floreios,
e que só está brincando, por raiva e malícia. Bem, pois é em todos esses
atos conscientes e ignomínias que se encerra a voluptuosidade. ``Eu os
incomodo, machuco o coração, não deixo ninguém dormir em casa. Então não
durmam, sintam a cada instante que meus dentes doem. Agora para vocês
não sou o herói que antes quis parecer, mas apenas uma pessoa abjeta, um
\emph{chenapan}\footnote{Patife, em francês russificado no original. {[}\textsc{n.\,t.}{]}}. Pois seja! Estou muito feliz por ter sido decifrado.
Sentem"-se mal ao ouvir meus gemidos canalhas? Pois que se sintam; vou
dar mais um trilo nocivo\ldots{}'' Nem agora estão entendendo, senhores? Não,
pelo visto é necessário um profundo desenvolvimento e uma profunda
consciência para compreender todos os meandros dessa voluptuosidade!
Estão rindo? Os senhores me deixam muito contente. Senhores, é claro que
minhas piadas são de mau gosto, irregulares, atabalhoadas, sem
confiança. Mas isso é porque não tenho respeito por mim mesmo. Por acaso
uma pessoa consciente pode ter algum respeito por si mesma?

\section{V}

Mas seria possível, seria possível uma pessoa que tentou buscar prazer
na própria humilhação ter algum respeito por si mesma? Não digo isso
agora amolecido por algum tipo de arrependimento. Em geral, não poderia
suportar dizer: ``Perdão, papai, não vou mais fazer'', não por não ser
capaz de dizê"-lo mas, pelo contrário, talvez justamente por ser capaz
demais disso, não é? Como que de propósito, acontecia"-me de fazê"-lo nos
casos em que não tinha nem sombra de culpa. Isso já era muito mais vil.
Ao mesmo tempo, voltava a me enternecer de todo coração, arrependia"-me,
as lágrimas corriam e é claro que eu ludibriava a mim mesmo, embora não
estivesse fingindo de jeito nenhum. Já era o coração aprontando alguma
sujeira\ldots{} Daí não dava para culpar nem as leis da natureza, embora
mesmo assim essas leis tenham sido a maior causa de ofensa da minha
vida. Faz mal lembrar isso tudo, e nessa época já fazia. Afinal, em um
minuto já me acontecia de perceber que tudo aquilo era uma mentira, uma
mentira afetada, ou seja, todos esses arrependimentos, toda essa
comoção, todos esses votos de regeneração. Perguntarão: por que eu
ficava me estropiando e atormentando? Resposta: porque era muito chato
ficar sentado de braços cruzados; então me lançava a afetações. Verdade,
era isso. Observem melhor a si mesmos, senhores, e compreenderão que é
isso. Inventei"-me umas aventuras e criei uma vida para viver de algum
jeito. Quantas vezes me aconteceu, bem, por exemplo, de me ofender por
nada, de propósito; sabia muito bem que estava me ofendendo por nada,
ludibriava a mim mesmo, mas a coisa era levada a tal ponto que, no fim,
virava verdade, e me ofendia de fato. De certa forma, fui atraído a vida
inteira por pregar essas peças, tanto que, no fim, perdi o controle.
Outra vez quis me apaixonar com força, até duas vezes. Sofri mesmo,
senhores, asseguro"-lhes. No fundo do coração não acreditava que estava
sofrendo, era uma zombaria, mas sofria assim mesmo, de forma real e
autêntica: ficava com ciúmes, saía de mim\ldots{} E tudo por tédio, senhores,
tudo por tédio; a inércia me esmagava. Pois o fruto direto, legítimo,
espontâneo da consciência é a inércia, ou seja, ficar sentado de braços
cruzados conscientemente. Já mencionei isso acima. Repito, repito com
ênfase: todas as pessoas espontâneas e de ação são de ação por serem
estúpidas e limitadas. Como explicar? Assim: em consequência de sua
limitação, tomam as causas mais próximas e secundárias pelas principais,
assegurando"-se dessa forma, mais rápido e mais fácil, de que encontraram
o fundamento indiscutível de sua ação, e daí se tranquilizam; isso é o
mais importante. Afinal, para começar a agir, antes é preciso estar
totalmente tranquilo, para que não reste dúvida alguma. E como eu, por
exemplo, me tranquilizo? Cadê as causas principais nas quais me apoio,
cadê os fundamentos? Onde vou buscá"-los? Exercito"-me no pensamento e, em
consequência, cada causa principal arrasta uma outra, ainda mais
principal, e assim por diante, sem fim. Essa é exatamente a essência de
toda consciência e pensamento. Talvez sejam de novo as leis da natureza.
E qual é o resultado final? Aquele mesmo. Lembrem"-se: falei de vingança
há pouco. (Os senhores provavelmente não se detiveram muito nisso). Foi
dito: uma pessoa se vinga por ver justiça nisso. Quer dizer que achou a
causa principal, achou o fundamento principal: exatamente a justiça.
Deve estar tranquila de todos os lados e, em consequência, vinga"-se com
calma e sucesso, convicta de que comete uma ação honrada e justa. Só que
eu não vejo essa justiça, tampouco encontro qualquer virtude e, em
consequência, se resolver me vingar, será só de raiva. Claro que a raiva
pode sobrepujar tudo, todas as minhas dúvidas, e talvez tenha sucesso
absoluto ao servir de causa principal, exatamente por não ser uma causa.
Em consequência dessas malditas leis da consciência, meu rancor volta a
ser submetido a uma decomposição química. Você olha e o objeto se
evapora, as razões se desfazem, o culpado não é achado, a ofensa não é
mais ofensa, mas o fado, algo do gênero de uma dor de dente da qual
ninguém é culpado e, em consequência, volta a restar aquela mesma saída,
ou seja, bater no muro com mais força. Daí você dá um aceno de despedida
por não ter encontrado a causa principal. Mas experimente se deixar
arrebatar de foma cega pelo sentimento, sem raciocinar, sem causa
principal, expulsando então a consciência: odiar ou amar, só para não
ficar de braços cruzados. No mais tardar depois de amanhã vai começar a
se desprezar, por ter ludibriado a si mesmo de forma intencional. O
resultado: uma bola de sabão e a inércia. Oh, senhores, é possível que
eu só me considere uma pessoa inteligente por, durante a vida inteira,
não ter conseguido nem começar, nem terminar nada. Que eu seja, que seja
um tagarela, um tagarela inofensivo e enfadonho como todos nós. Mas o
que fazer, se o destino patente e único de todo homem inteligente é a
tagarelice, ou seja, a transferência intencional do deserto para o
vazio?

\section{VI}

Oh, se eu não fizesse nada só de preguiça. Senhores, como então eu me
respeitaria. Eu me respeitaria justamente porque pelo menos a preguiça
eu estaria em condições de ter; existiria em mim pelo menos uma
característica afirmativa, da qual eu estaria seguro. Pergunta: quem é
ele? Resposta: um preguiçoso; seria agradabilíssimo ouvir isso a meu
respeito. Quer dizer, uma determinação afirmativa, quer dizer, há o que
se dizer a meu respeito. ``Preguiçoso!'' --- afinal, isso é um título e
uma nomeação, isso é uma carreira, senhores. Não brinquem, é assim. Eu
seria então, por direito, membro do primeiro entre os clubes, e minha
única ocupação seria me respeitar sem cessar. Conheci um senhor que, a
vida inteira, orgulhava"-se em ser um perito em vinho Lafite. Considerava
isso sua qualidade afirmativa, e jamais duvidava de si. Morreu com a
consciência não apenas tranquila, mas triunfante, e tinha absoluta
razão. E se eu então escolhesse uma carreira, seria um preguiçoso e um
glutão, só que não simples mas, por exemplo, simpatizante de todo o belo
e sublime. Que lhes parece? Venho pensando nisso há muito tempo. Esse
``belo e sublime'' apertou minha nuca com força aos quarenta anos; isso
foi aos quarenta anos mas, naquela época, ah, então teria sido outra
coisa! Teria encontrado de imediato a atividade correspondente, justo a
de beber à saúde de tudo que é belo e sublime. Eu me agarraria a
qualquer ocasião de, primeiro, verter uma lágrima em minha taça e,
depois, beber a todo esse belo e sublime. Teria então convertido tudo no
mundo em belo e sublime; buscaria o belo e o sublime na imundície mais
indiscutível e abjeta. Seria lacrimejante como uma esponja molhada. Um
pintor, por exemplo, pinta um quadro de Gue. Imediatamente bebo à saúde
do artista que pintou o quadro de Gue, pois amo tudo que é belo e
sublime. Um autor escreve ``como apraz a cada um''; bebo imediatamente à
saude ``de cada um'', pois amo tudo que é ``belo e sublime''\footnote{Aqui
  há um ataque polêmico contra Saltykov"-Schedrin, que escreveu em
  \emph{O Contemporâneo,} em 1863, um comentário favorável sobre o
  quadro \emph{A Última Ceia}'', de N. N. Gue (1831--1894). Esse quadro,
  exibido pela primeira vez na exposição de outono da Academia de Arte,
  em 1863, suscitou pontos de vista contraditórios. Posteriormente,
  Dostoiévski censuraria Gue pela mistura premeditada da realidade
  ``histórica e atual'' que, em sua opinião, ``resultava falso e uma
  ideia preconcebida, e tudo de falso é uma mentira e já não tem nada de
  realismo'' (1873, \emph{Diário de um Escritor,} A Respeito de Uma
  Exposição). ``Como apraz a cada um'' foi um artigo de
  Saltykov"-Schedrin publicado em \emph{O Contemporâneo} em 1863. {[}\textsc{n.\,e.}{]}}. Exigiria respeito por isso, perseguiria quem não me demonstrasse
respeito. Morreria tranquilo, morreria solene: um encanto, um encanto
absoluto''. Deixaria crescer então uma tal barriga, armaria um tamanho
queixo triplo, fabricaria um tal nariz de sândalo\footnote{Nariz de
  bêbado. {[}\textsc{n.\,e.}{]}}que todo passante diria, ao olhar para mim: ``Que
máximo! Esse sim é verdadeiro e afirmativo!'' Seja como quiserem, é
agradabilíssimo ouvir esse tipo de opinião em nossa era negativa,
senhores.

\section{VII}

Mas tudo isso são sonhos dourados. Oh, digam"-me, quem foi o primeiro a
afirmar, o primeiro a proclamar, que uma pessoa só faz obscenidades por
não conhecer seus reais interesses; e que, se for esclarecida, tiver os
olhos abertos para seus reais e normais interesses, imediatamente para
de fazer obscenidades, imediatamente começa a ser boa e nobre, pois, uma
vez esclarecida e compreendendo seu real proveito, verá que ele consiste
na bondade e, sabendo"-se que uma pessoa não pode agir contra seu próprio
proveito de forma deliberada, em consequência, ela inevitavelmente
passaria a fazer o bem? Oh, que bebê! Oh, que criança pura e inocente!
Pois quando aconteceu, em primeiro lugar, ao longo de todos esses
milênios, de o homem agir apenas em seu próprio proveito? O que fazer
com os milhões de fatos que testemunham que as pessoas \emph{de forma
deliberada,} ou seja, plenamente conscientes de seus reais proveitos,
deixaram"-nos em segundo plano e se lançaram em outro caminho, no risco,
no acaso, sem serem coagidas por nada nem ninguém, como se apenas não
desejassem justo o caminho indicado e, de forma voluntária e obstinada,
passaram a um outro, difícil, disparatado, buscando"-o quase nas trevas?
Então quer dizer que essa obstinação e voluntarismo eram mais agradáveis
do que qualquer proveito\ldots{} Proveito! O que é o proveito? Os senhores
assumiriam a tarefa de definir de modo completamente preciso no que
exatamente consiste o proveito humano? E se acontecer de \emph{alguma
vez} o proveito humano não apenas puder, mas até dever consistir
exatamente em desejar o mau, e não o vantajoso? E se for assim, se esse
caso for apenas admissível, então toda a regra se reduz a pó. O que
acham, um caso desses pode acontecer? Estão rindo; riam, meus senhores,
mas apenas respondam: o proveito humano pode ser calculado com total
exatidão? Não há aqueles que não apenas não se enquadram como não podem
se enquadrar em qualquer classificação? Afinal, senhores, até onde sei,
toda sua lista de proveitos humanos foi elaborada a partir dos dados das
cifras estatísticas e das fórmulas científico"-econômicas. Todo seu
proveito é prosperidade, riqueza, liberdade, tranquilidade, etc, etc;
assim, uma pessoa que fosse, por exemplo, contra toda essa lista de
forma clara e deliberada seria, na opinião dos senhores, e claro que até
mesmo na minha, um obscurantista ou um louco completo, como não? Mas
vejam o que é espantoso: por que acontece de todos esses estatísticos,
sábios e amantes do gênero humano, no cômputo dos proveitos humanos,
deixarem um de fora o tempo todo? Nem o levam em conta como deveriam, e
todo o cálculo depende disso. Não seria um grande problema pegar esse
proveito e incluí"-lo na lista. Mas o nefasto é que esse sábio proveito
não cabe em classificação alguma, não se enquadra em nenhuma lista.
Tenho, por exemplo, um amigo\ldots{} Ei, senhores! Ele também é seu amigo;
sim, de quem ele não é amigo? Ao se preparar para a ação, esse senhor
imediatamente relata, com eloquência e clareza, como terá que proceder
exatamente de acordo com as leis da razão e da verdade. Mais ainda:
agitado e apaixonado, vai falar dos interesses humanos reais e normais;
com ironia, reprovará a miopia dos estúpidos que não compreendem seu
próprio proveito, nem o real significado da virtude; e --- dentro de um
quarto de hora, sem qualquer pretexto repentino vindo de fora,
justamente devido a algo interior que é mais forte do que seus
interesses --- apronta outra, ou seja, vai claramente contra o que estava
dizendo; contra as leis da razão, contra o proveito próprio, bem, em
suma, contra tudo\ldots{} Advirto que meu amigo é uma pessoa coletiva e, por
isso, é difícil culpá"-lo sozinho por algo. Ou seja, não existiria de
fato alguma coisa, meus senhores, que seria mais cara a qualquer pessoa
do que os melhores desses proveitos, ou (para não quebrar a lógica) um
proveito mais proveitoso (exatamente o que deixaram de fora, como acabei
de falar), que é mais importante e proveitoso que todos os outros
proveitos, e pelo qual a pessoa, caso necessário, estaria pronta para ir
contra todas as leis, ou seja, contra a razão, a honra, a tranquilidade,
a prosperidade, em suma, contra todas essas coisas maravilhosas e
vantajosas apenas para obter esse proveito primordial e mais proveitoso,
que lhe é mais caro do que todos?

--- Bem, mas mesmo assim é um proveito --- os senhores vão me interromper.
--- Perdão, senhores, ainda estou me explicando, não é uma questão de jogo
de palavras, mas de que esse proveito é justamente tão notável que
destroi todas nossas classificações e os sistemas elaborados pelos
amantes do gênero humano para a felicidade do gênero humano,
arrebenta"-os o tempo todo. Em suma, atrapalha tudo. Mas antes de dizer o
nome desse proveito, desejo me comprometer pessoalmente e, por isso,
declaro com audácia que todos esses maravilhosos sistemas, todas essas
teorias que elucidam à humanidade quais são seus interesses reais e
normais para que ela se precipite sem falta para obtê"-los, tornando"-se
de imediato bondosa e nobre, entrementes são, na minha opinião, mera
logística. Sim senhores, logística! Pois sustentar essa teoria de
renovação de todo o gênero humano por meio do sistema do proveito
próprio é, para mim, quase o mesmo que\ldots{} apoiar, por exemplo, de acordo
com Buckle, que a civilização suaviza o homem, o qual, como
consequência, se torna menos sangrento e menos propício à
guerra\footnote{Em ``História da Civilização na Inglaterra'', obra em
  dois tomos do historiador e sociólogo inglês Henry Thomas Buckle
  (1821--1862), está expressa a ideia de que o desenvolvimento da
  civilização leva à interrupção das guerras entre os povos. {[}\textsc{n.\,e.}{]}}.
Chega a isso, ao que parece, pela lógica. Mas o homem é tão apaixonado
por sistemas e pela dedução abstrata que está prestes a deformar a
verdade de propósito, sem ver nem ouvir nada, só para justificar sua
lógica. Por isso tomo esse exemplo bastante convincente. Basta olhar ao
redor: o sangue corre aos rios, da forma mais alegre, como champanhe.
Vejam nosso século \textsc{xix}, em que Buckle viveu. Vejam Napoleão --- o grande
e o atual. Vejam a América do Norte, a união eterna. Vejam, por fim, o
caricatural Schleswig"-Holstein\footnote{Referências a Napoleão \textsc{i},
  Napoleão \textsc{iii}, à Guerra de Secessão dos \textsc{eua} (1861--5) e à guerra de
  Prússia e Áustria contra a Dinamarca (1864) pela posse de
  Schleswig"-Holstein. {[}\textsc{n.\,t.}{]}}\ldots{} O que a civilização suavizou em
nós? A civilização só elabora no homem a multiplicidade de sensações
e\ldots{} decididamente nada mais. E através do desenvolvimento dessa
multiplicidade, o homem talvez até chegue a encontrar prazer no sangue.
Afinal, isso já ocorreu. Os senhores já notaram que os sanguinários mais
refinados, aos pés dos quais não chegam Átila ou Stenka Rázin\footnote{Stiepan
  Timofiêievitch Rázin (1630--1671) foi o líder de uma rebelião de
  cossacos na Rússia do século \textsc{xvii}. {[}\textsc{n.\,t.}{]}}, quase sempre são os
cavalheiros mais civilizados, e se chamam menos a atenção do que Átila e
Stenka Rázin é justamente porque os encontramos com muito mais
frequência, são muito mais comuns, tornaram"-se familiares. Pelo menos,
se o homem não se tornou mais sanguinário com a civilização, ficou
provavelmente pior, um sanguinário mais vil do que antes. Antes ele via
justiça em ser sanguinário, exterminando quem devia com a consciência
tranquila; agora, embora consideremos o derramamento de sangue uma
vileza, ocupamo"-nos dessa vileza ainda mais do que antes. Quem é pior?
Podem decidir. Dizem que Cleópatra (desculpem"-me pelo exemplo da
história romana) gostava de cravar alfinetes de ouro no peito de suas
escravas, encontrando prazer em seus gritos e convulsões. Os senhores
dirão que isso que foi em uma época relativamente bárbara; que hoje a
época também é bárbara, pois (também relativamente) hoje também se
cravam alfinetes; que hoje porém o homem aprendeu a ver, às vezes com
mais clareza que na época bárbara, mas que ainda está longe de
\emph{aprender} a se comportar como é indicado pela razão e pela
ciência. Mas mesmo assim os senhores estão completamente certos de que é
inevitável que ele aprenda, quando esses velhos e maus hábitos tiverem
caído e quando o bom senso e a ciência educarão por inteiro e orientarão
para a normalidade a natureza humana. Os senhores têm certeza de que
então o homem vai parar de se enganar \emph{de forma deliberada} e, por
assim dizer, não terá vontade de separar sua vontade de seus interesses
normais. Ainda mais: os senhores dirão que, daí, a própria ciência
ensinará o homem (embora, para mim, isso já seja um luxo), que na
verdade ele não tem nem vontade, nem capricho, que nunca os teve, e que
ele não é mais do que uma tecla de piano\footnote{Alusão à frase do
  pensador francẽs Denis Diderot (1713--1784) no ``Diálogo entre
  D'Alembert e Diderot'' (1769): ``somos instrumentos dotados de
  sensibilidade e memória. Nossos sentidos são como teclas tocadas pela
  natureza que nos envolve e que frequentemente se tocam a si mesmas''.
  {[}\textsc{n.\,e.}{]}} ou um registro de órgão; e que, acima de tudo, ainda há
leis da natureza; assim, tudo que ele faz não acontece de acordo com seu
desejo, mas por si só, de acordo com as leis da natureza.
Consequentemente, basta apenas descobrir essas leis da natureza e o
homem já não responderá por sua conduta, e sua vida se tornará
excepcionalmente fácil. Todas as condutas humanas vão ser obviamente
então calculadas de acordo com essas leis, com a matemática, como uma
tábua de logaritmos, até o 108.000, e inscritas em calendários; ou,
melhor ainda, aparecerão umas edições bem pensantes, como os atuais
dicionários enciclopédicos, nas quais tudo será calculado e designado
com tamanha exatidão que não haverá mais ações nem aventuras.

Então --- sempre são os senhores a falar --- serão estabelecidas novas
relações econômicas, tão prontas e também calculadas com precisão
matemática que em um instante vão desaparecer todas as questões
possíveis, justamente porque haverá todas as respostas possíveis. Então
será erigido um palácio de cristal\footnote{Alusão polêmica ao
  \emph{Quarto sonho de Vera Pávlovna}, no romance \emph{O que fazer}?,
  de Tchernychévski. Lá se descreve um palacete ``de ferro e cristal''
  no qual, como também imaginado por Charles Fourier (na \emph{Teoria da
  Unidade Universal}) viveriam as pessoas da sociedade socialista. O
  modelo para esse edifício foi o Palácio de Cristal de Londres,
  descrito por Dostoiévski nas \emph{Notas de inverno sobre impressões
  de verão}. {[}\textsc{n.\,e.}{]}}. Então\ldots{} Bem, em suma, então chegará o pássaro
Kagan\footnote{Pela tradição popular, o pássaro que traz felicidade às
  pessoas. {[}\textsc{n.\,e.}{]}}. Claro que não há como garantir (daí já sou eu
falando) que então não será, por exemplo, terrivelmente chato (pois o
que fazer quando tudo estiver calculado numa tabela), embora tudo seja
extraordinariamente sensato! Claro, o que não se inventa por tédio!
Afinal, vão enfiar alfinetes de ouro por tédio, mas isso não há de ser
nada. O mau (de novo sou eu falando) é que pode ser que as pessoas
também se alegrem com os alfinetes de ouro. Pois o homem é estúpido, de
uma estupidez fenomenal. Ou seja, talvez não seja completamente estúpido
mas, em compensação, é de uma ingratidão que não se encontra similar.
Pois eu, por exemplo, não me espantaria nem um pouco se de repente, em
meio ao bom senso generalizado do futuro, aparecesse um \emph{gentleman}
com uma fisionomia vil ou, melhor dizendo, retrógrada e ridícula,
colocasse as mãos nas ancas e dissesse a todos: pois bem, senhores,
vamos acabar com todo esse bom senso de uma vez, com um chute, com o
único objetivo de mandar para o diabo todos esses logaritmos e voltar a
viver de acordo com nossa vontade estúpida! Isso ainda não seria nada,
mas o ultrajante é que ele inevitavelmente teria seguidores: assim é o
homem. E tudo isso devido ao mais oco dos motivos, que aparentemente não
é digno sequer de menção; justamente que o homem, sempre e por toda
parte, seja quem for, gosta de agir como quer, e de jeito nenhum como
determinam a razão e o proveito; pode querer ir contra o proveito
próprio, e às vezes \emph{deve, de forma deliberada} (já é uma ideia
minha). Seu desejo próprio, livre e voluntário, seu próprio capricho,
por mais extravagante, sua fantasia, ainda que exasperada até a loucura,
é tudo, é aquilo que ficou de fora, o mais proveitoso dos proveitos, que
não se enquadra em classificação alguma e devido ao qual todos os
sistemas e teorias sempre se espatifam no inferno. E de onde todos esses
sábios tiraram que o homem necessita de uma vontade normal e virtuosa?
De onde imaginaram impreterivelmente que o homem necessita
impreterivelmente de uma vontade proveitosa e virtuosa? O homem só
precisa de uma vontade \emph{independente,} custe o que custar e leve a
que levar essa independência. O diabo é que sabe o que é essa vontade\ldots{}

\section{VIII}

--- Ha"-ha"-ha! Mas veja que, como queira, em essência, a vontade também não
existe! --- os senhores vão me interromper com uma gargalhada. --- A ciência
conseguiu fazer uma análise anatômica do homem a tal ponto que agora
sabemos que a vontade e o assim chamado livre arbítrio não são nada além
de\ldots{}

--- Perdão, senhores, eu mesmo queria começar assim. Admito que cheguei a
me assustar. Estava prestes a bradar que o diabo sabe de que depende e o
que é a vontade, e que isso, talvez, seja uma bênção, mas daí me lembrei
da ciência e\ldots{} refreei"-me. Daí os senhores falaram. Pois, de fato, se
descobrirem de verdade alguma fórmula para todas nossas vontades e
caprichos, ou seja, de que dependem, de onde vêm, como exatamente se
propagam, para onde se dirigem nesse e naquele caso, etc, etc, ou seja,
uma autêntica fórmula matemática, daí o homem talvez pare imediatamente
de desejar, ou melhor, vai mesmo parar. Afinal, quem vai querer desejar
por tabela? Mais ainda: de homem, ele imediatamente vai se converter em
um registro de órgão ou algo do gênero, pois o que é um homem sem
desejo, sem arbítrio e sem vontade, senão uma alavanca entre os
registros de um órgão? Que lhes parece? Calculemos as probabilidades:
isso pode ocorrer ou não?

--- Hum\ldots{} --- decidirão os senhores ---, nossas vontades se equivocam em
grande parte devido à visão equivocada de nossos proveitos. Desejamos às
vezes um puro absurdo por ver nesse absurdo, em nossa estupidez, o
caminho mais fácil para a obtenção de algum proveito previamente
determinado. Porém, quando tudo isso for bem explicado, calculado no
papel (o que é muito possível, já que é torpe e insensato crer com
antecipação que o homem jamais conhecerá algumas leis da natureza), daí
obviamente não existirá o assim chamado desejo. Pois se a vontade alguma
vez estiver de perfeito acordo com a razão, daí vamos raciocionar, e não
querer, pois não será possível, por exemplo, conservando a razão,
\emph{querer} algo insensato e, dessa forma, ir conscientemente contra a
razão e desejar algo nocivo contra si\ldots{} E como todas as vontades e
raciocínios realmente poderão ser calculados, já que em algum momento
serão descobertas as leis do nosso assim chamado livre arbítrio, então,
sem brincadeira, poderá ser estabelecida algo como uma tabela, de modo
que vamos de fato desejar de acordo com ela. Pois se, por exemplo,
alguma vez calcularem e me demonstrarem que, se fiz uma figa\footnote{Gesto
  ofensivo na Rússia. {[}\textsc{n.\,t.}{]}} a alguém, foi exatamente porque não
podia deixar de ter feito, e que devia infalivelmente colocar os dedos
daquela forma, o que me resta então de \emph{liberdade,} especialmente
se eu for instruído e tiver concluído algum curso de ciências? Pois
então poderei calcular toda minha vida com trinta anos de antecedência;
em suma, se as coisas se arranjarem assim, nada nos restará a fazer;
dará na mesma, teremos que aceitar. E, em geral, sem nos cansar, teremos
que repetir para nós mesmos que impreterivelmente, em dado instante e em
dadas circunstâncias, a natureza não nos consulta; que é necessário
aceitá"-la como é, e não como a fantasiamos, e se nos dirigirmos de fato
à tabela e ao calendário, bem, e\ldots{} talvez até à retorta, que fazer, é
preciso aceitar também a retorta! Senão ela se fará aceitar sem vocês\ldots{}

--- Sim, senhores, mas para mim aí é que está o embaraço! Senhores,
perdoem"-me por ter me posto a filosofar; são quarenta anos de subsolo!
Permitam"-me fantasiar. Vejam: a razão, senhores, é uma coisa boa, isso
não se discute, mas a razão é apenas razão, satisfazendo apenas as
faculdades racionais do homem, enquanto a vontade é uma manifestação de
toda a vida, ou seja, de toda a vida humana, com a razão e com todas as
coceiras. E embora nossa vida, nessa manifestação, resulte amiúde em
porcaria, mesmo assim é vida, e não apenas extração de raiz quadrada.
Afinal eu, por exemplo, de forma absolutamente natural, quero viver para
satisfazer todas as minhas faculdades, e não para satisfazer apenas
minhas faculdades racionais, ou seja, um vigésimo delas. O que a razão
sabe? A razão só sabe o que conseguiu aprender (algumas coisas talvez
jamais saiba aprender; embora isso não seja consolo, por que não
dizê"-lo?), e a natureza humana age em toda sua gama, com tudo que nela
existe, consciente e inconsciente e, mesmo desafinando, vive. Suspeito,
senhores, que me contemplem com pena; vão me repetir que um homem
instruído e evoluído, em suma, como será o homem do futuro, não pode
desejar conscientemente algo que lhe traz desvantagem, que isso é
matemática. Só que eu lhes repito pela centésima vez que há apenas um
caso, apenas um, em que a pessoa pode, de forma deliberada e consciente,
desejar o que lhe é prejudicial, estúpido, até estupidíssimo, justamente
para \emph{ter o direito} de desejar o que é estupidíssimo e não ser
obrigada a querer só o que é sábio. Afinal, isso é estúpido, é um
capricho e, mesmo assim, senhores, pode ser que isso seja o há de mais
proveitoso para nós sobre a terra, especialmente em certos casos. E em
particular pode ser mais proveitoso do que todos os proveitos, mesmo no
caso em que nos provoca danos claros, contradizendo as conclusões mais
razoáveis de nosso raciocínio a respeito do proveito, já que, de toda
forma, conserva o que nos é mais importante e caro, ou seja, nossa
personalidade e nossa individualidade. Alguns até afirmam que isso é de
fato o que o homem tem de mais caro; claro que a vontade pode, como
quiserem, também coincidir com a razão, em especial se não houver abuso,
e o uso for moderado; isso é vantajoso e, por vezes, até louvável. Só
que, com muita frequência, e até na maior parte das vezes, a vontade
está em completo e obstinado desacordo com a razão e\ldots{} e\ldots{} e sabiam
que isso também é vantajoso e, por vezes, até louvável? Senhores, vamos
supor que o homem não é estúpido. (De fato, não há como dizer isso a seu
respeito, nem que seja apenas porque, se ele for estúpido, então quem
será inteligente?) Mas, se não é estúpido, mesmo assim é de uma
ingratidão monstruosa! Uma ingratidão fenomenal. Chego a pensar que a
melhor definição do homem é: criatura de duas pernas e ingrata. Mas isso
ainda não é tudo; não é seu defeito principal; o principal defeito é a
perpétua ingratidão, perpétua, começada no Dilúvio Universal, até o
período Schleswig"-Holstein do destino humano. Imoralidade e,
consequentemente, imprudência, pois há muito tempo é sabido que a
imprudência não vem de outro lugar, senão da imoralidade. Experimentem
lançar um olhar à história da humanidade; pois bem, o que vêem?
Grandiosidade? Talvez também grandiosidade; pois o Colosso de Rodes, por
si só, quer dizer algo! Não é à toa que Anaiévski\footnote{A. E.
  Anaiévski (1788--1866), autor de falsificações literárias que foram
  objeto de constante zombaria no jornalismo das décadas 1840--1860
  escreveu, na brochura \emph{Guia dos Curiosos} (1854): ``Alguns
  autores creem que o Colosso de Rodes foi criado por Semíramis,
  enquanto outros afirmam que não foi erigido por mão humana, mas pela
  natureza''. {[}\textsc{n.\,e.}{]}} atesta que uns dizem que ele é obra da mão
humana, enquanto outros afirmam que foi construído pela própria
natureza. Colorido? Talvez também colorido: basta tomar apenas os
uniformes de gala militares e civis de todos os séculos e todos os
povos, isso quer dizer alguma coisa, e os uniformes de serviço também
deixam de queixo caído --- nenhum historiador resiste. Monotonia? Certo,
talvez também monotonia: lutas e lutas, luta"-se hoje, lutou"-se ontem,
vai se lutar depois; hão de convir que é até monótono demais. Em suma,
pode"-se dizer tudo da história universal, tudo que puder vir à cabeça da
imaginação mais perturbada. Só não dá para dizer uma coisa: que é
sensata. Os senhores vão se engasgar na primeira palavra. E tem ainda
aquela coisa que se encontra a todo momento: aparecem constantemente na
vida pessoas sensatas e de bem, sábios amantes do gênero humano, que
colocam como o objetivo de toda sua vida comportar"-se da forma mais
bondosa e sensata possível para, por assim dizer, iluminar os próximos e
demonstrar"-lhes propriamente que é possível de fato viver de forma boa e
sensata. E então? Como se sabe, mais cedo ou mais tarde, no fim da vida,
esses amantes se traem, dando margem a anedotas, algumas das quais
bastante indecentes. Agora lhes pergunto: o que se pode esperar do
homem, essa criatura dotada de qualidades tão estranhas? Cubram"-no de
todos os bens terrestres, afoguem"-no totalmente em felicidade, até a
cabeça, de modo que só se vejam bolhas na superfície, como dentro da
água; dêem"-lhe uma tal abundância econômica para que não lhe reste nada
mais a fazer a não ser dormir, comer pão doce e cuidar da continuação da
história universal, e daí o homem, por pura ingratidão, por pura
pasquinada, vai cometer uma torpeza. Chegará a arriscar o pão doce e
desejará de propósito o absurdo mais nefasto, a insensatez mais
antieconômica, apenas para misturar a essa sensatez positiva seu
elemento de fantasia nefasta. Quer reter unicamente seus sonhos
fantásticos e estupidez vulgar justamente para confirmar para si (já que
isso é absolutamente indispensável) que as pessoas são sempre pessoas, e
não teclas de piano que as leis da natureza tocam e ameaçam tocar até
que não haja nada além do calendário e não se possa desejar nada. E tem
mais: mesmo no caso que ele se revelasse de fato uma tecla de piano, se
isso lhe fosse demonstrado pelas leis naturais e da matemática, nem
assim criaria juízo, e faria de propósito alguma coisa contra,
unicamente por ingratidão; apenas para insistir no que é seu. E no caso
de os meios lhe faltarem, inventaria a destruição e os caos, inventaria
diversos sofrimentos e insistira no que é seu! Lançaria uma maldição ao
mundo e, como apenas o homem pode amaldiçoar (é seu privilégio, a
principal diferença entre ele e os outros animais), talvez apenas com a
maldição conseguiria o que é seu, ou seja, assegurar"-se de fato de que é
um homem, e não uma tecla de piano! Caso os senhores digam que isso
também pode ser calculado em uma tabela --- o caos, as trevas, a
maldição, de modo que a mera possibilidade de cálculo antecipado
paralisa tudo, e a razão vence ---, dai o homem vai ficar louco de
propósito, para não ter razão e insistir no que é seu! Creio nisso,
respondo por isso, pois toda questão da humanidade, na aparência e na
realidade, consiste apenas em que o homem demonstre para si que é um
homem, e não um registro! Com as ancas, mas demonstra; como um
troglodita, mas demonstra. Depois disso, como não pecar, como não louvar
que isso ainda não exista e que a vontade entrementes ainda dependa de
sabe o diabo o quê\ldots{}

Os senhores me gritarão (caso ainda me agraciem com seus gritos) que
ninguém está me privando de meu arbítrio; que apenas estão cuidando de
estabelecer de alguma forma que meu arbítrio, minha própria vontade,
coincida com meus interesses normais, com as leis da natureza e da
aritmética.

--- Ah, senhores, cadê a vontade própria quando a coisa chega a tabelas e
aritméticas, quando está em marcha apenas dois e dois são quatro? Dois e
dois serão quatro mesmo sem minha vontade. Como se vontade própria fosse
isso!

\section{IX}

Senhores, claro que estou fazendo piada, e sei que minha piada é um
fiasco, só que não dá para levar tudo como piada. Pode ser que eu faça
piada rangendo os dentes. Senhores, umas questões me atormentam;
solucionem"-nas para mim. Pois os senhores, por exemplo, desejam desmamar
o homem dos velhos hábitos e corrigir seu arbítrio conforme as
exigências da ciência e do senso comum. Mas como os senhores sabem que o
homem não apenas pode como \emph{deve} ser reformado desse jeito? De
onde deduziram que a vontade humana \emph{tem} que ser corrigida de modo
tão indispensável? Em suma, como sabem que essa correção de fato trará
proveito ao homem? E, se for para dizer tudo, como estão convictos com
tamanha \emph{certeza} de que não ir contra as vantagens reais e normais
garantidas pelos argumentos da razão e da aritmética é de fato o que há
de mais proveitoso para o homem, consistindo em lei para toda a
humanindade? Afinal, isso não é, entrementes, mais do que uma hipótese.
Supondo que seja uma lei da lógica, talvez não seja de jeito nenhum da
humanidade. Os senhores acham talvez que estou louco? Permitam"-me uma
ressalva. Concordo: o homem é um animal predominantemente criador,
condenado a se lançar de forma consciente a um objetivo e a se ocupar da
arte da engenharia, ou seja, eterna e incessantemente construir estradas
\emph{para onde quer que seja.} Mas talvez ele tenha vontade de às vezes
se desviar justamente por estar \emph{condenado} a abrir essa estrada e,
mais ainda, por mais estúpido que o homem espontâneo e de ação seja,
mesmo assim às vezes vem"-lhe à mente que uma estrada quase sempre leva
\emph{para algum lugar,} e que o principal não é para onde, mas sim que
ela apenas leve, e que a criança comportada não desdenhe da arte da
engenharia e não se entregue ao ócio ruinoso que, como se sabe, é a mãe
de todos os vícios. O homem adora criar e abrir estradas, isso é
indiscutível. Mas por que também ama até a paixão a destruição e o caos?
Digam"-me! A esse respeito, porém, desejo proferir duas palavras em
particular. Será que ele não ama tanto a destruição e o caos (pois é
indiscutível que às vezes ama muito) por um medo instintivo de alçancar
os objetivos e completar o edifício que criou? Como saber, pode ser que
ele só goste desse edifício de longe, jamais de perto; pode ser que
goste apenas de criá"-lo, mas não de habitá"-lo, cedendo"-o depois
\emph{aux animaux domestiques}\footnote{Aos animais domésticos, em
  francês no original. {[}\textsc{n.\,t.}{]}}, como formigas, carneiros,
etc, etc. Afinal, as formigas são dotadas de um gosto totalmente
diverso. Possuem um edifício notável do gênero, indestrutível para
sempre: o formigueiro.

Com o formigueiro as veneráveis formigas começaram, com o formigueiro
provavelmente terminarão, o que confere grande honra à sua perseverança
e assertividade. O homem, porém, é uma criatura leviana e desonesta, e
talvez, a exemplo do jogador de xadrez, aprecie apenas o processo de
obtenção do objetivo, mas não o objetivo em si. E, quem sabe (não dá
para garantir), pode ser que todo objetivo que existe, para o qual a
humanidade se precipita, consista apenas nesse processo ininterrupto de
obtenção, que, para dizer de outra forma, é a própria vida, e não de
fato no objetivo, que, evidentemente, não deve ser diferente de dois e
dois são quatro, ou seja, uma fórmula, só que dois e dois são quatro já
não é a vida, meus senhores, mas o começo da morte. Pelo menos, o homem
sempre temeu de alguma forma esses dois e dois são quatro, como eu agora
temo. Suponhamos que o homem só faça buscar esses dois e dois são
quatro, atravesse oceanos, sacrifique a vida nessa busca, mas achar,
encontrar de verdade, meu Deus, como ele teme. Pois sente que, assim que
encontrar, não haverá mais nada a buscar. Terminado o trabalho, os
trabalhadores pelo menos recebem dinheiro, vão ao botequim, depois
acabam na delegacia --- aí tem ocupação para uma semana. E o homem, para
onde vai? A cada vez nota"-se nele pelo menos um certo embaraço ao obter
tais objetivos. Gosta do processo, mas não gosta nem um pouco de
obtê"-los, e claro que isso é terrivelmente ridículo. Em suma, o homem é
de constituição cômica; em tudo isso, obviamente, há um trocadilho. Só
que dois e dois são quatro é uma coisa de fato insuportável. Dois e dois
são quatro é, na minha opinião, mero atrevimento. Dois e dois são quatro
lança um olhar de peralta, fica de través no caminho, com a mão na
cintura, e cospe. Concordo que dois e dois são quatro é uma coisa
magnífica; porém, se for para louvar tudo, então dois e dois são cinco
às vezes é uma coisinha muito graciosa.

E por que os senhores têm essa certeza tão firme e solene de que só o
normal e afirmativo, em suma, só a prosperidade traz proveito ao homem?
A razão não estaria enganada quanto ao proveito? Afinal, não pode ser
que o homem não goste apenas da prosperidade? Não pode ser que ele goste
do sofrimento na mesma medida? Não pode ser que ele ache o sofrimento
tão proveitoso quanto a prosperidade? O homem às vezes tem um amor
terrível pelo sofrimento, até paixão, e isso é fato. Nem é preciso
recorrer à história universal; perguntem a si mesmos, desde que sejam
homens e tenham vivido um pouco. No que tange à minha opinião pessoal,
gostar só da prosperidade chega a ser algo indecoroso. Pode ser bom,
pode ser mau, só que quebrar algo também é às vezes muito agradável.
Pois no fundo não sou nem pelo sofrimento, nem pela prosperidade. Sou\ldots{}
por meu capricho, e por que ele me seja garantido quando necessário.
Sei, por exemplo, que o sofrimento não é permitido nos
\emph{vaudevilles.} No palácio de cristal também é inconcebível:
sofrimento é dúvida, é negação, e que palácio de cristal é esse no qual
é possível duvidar? Contudo, estou certo de que o homem jamais se
recusará ao verdadeiro sofrimento, ou seja, à destruição e ao caos.
Afinal, o sofrimento é a única causa da consciência. Apesar de ter
afirmado no começo que a consciência, na minha opinião, é a maior
desgraça do homem, sei que o homem a ama e não vai trocá"-la por
satisfação alguma. A consciência, por exemplo, está infinitamente acima
de dois e dois. Após dois e dois, obviamente, não sobra nada não apenas
a fazer, mas mesmo a conhecer. Tudo que então poderemos fazer é fechar
os cinco sentidos e submergir em contemplação. Bem, com a consciência o
resultado é idêntico, ou seja, tampouco haverá o que fazer, mas, pelo
menos, será possível se chicotear às vezes, o que no fim dá uma animada.
Embora retrógrado, ainda é melhor do que nada.

\section{X}

Os senhores acreditam no edifício de cristal, indestrutível para sempre,
ou seja, ao qual jamais será possível mostrar furtivamente a língua, nem
fazer figa com a mão no bolso. Bem, talvez por isso eu tema esse
edifício, por ser de cristal, indestrutível para sempre, e porque nem de
forma furtiva seja possível lhe mostrar a língua.

Pois vejam: se, em vez de palácio, fosse um galinheiro e chovesse, pode
ser que eu entrasse ali para não me molhar, mas nem assim a gratidão por
ele ter me protegido da chuva me levaria a tomar o galinheiro por um
palácio. Os senhores vão rir, chegarão a dizer que, nesse caso, um
galinheiro e um palacete são a mesma coisa. Sim, responderei, se
tivéssemos que viver só para não nos molharmos.

Que fazer, porém, se eu meti na cabeça que não vivo só para isso e que,
se tiver que viver, que seja em um palacete? É a minha vontade, é o meu
desejo. Os senhores só vão erradicá"-la quando mudarem o meu desejo. Bem,
mudem"-no, seduzam"-me com outro, dêem"-me outro ideal. Entrementes, não
vou tomar um galinheiro por um palácio. Pode se dar que o edifício de
cristal seja uma lorota, inadmissível pelas leis da natureza, e que eu o
tenha imaginado apenas em consequência de minha própria estupidez, em
consequência de hábitos estranhos e irracionais de nossa geração. Mas o
que tenho a ver com não ser admissível? Não dá na mesma se ele existir
em meus desejos ou, melhor dizendo, se existir enquanto existirem meus
desejos? Os senhores talvez voltarão a rir? Podem rir; aceito todas as
troças, mas nem assim vou dizer que estou saciado quando quero comer;
sei mesmo assim que, por compromisso, não vou sossegar com o zero
periódico incessante apenas porque ele existe de acordo com as leis da
natureza e existe \emph{de verdade.} Não aceitarei, como coroação de
meus desejos, um prédio de aluguel com apartamentos para inquilinos
pobres e contratos a mil por ano e, por via das dúvidas, uma placa do
dentista Wagenheim. Aniquilem meus desejos, apaguem meus ideais,
mostrem"-me algo melhor e os seguirei. Os senhores possivelmente dirão
que não vale a pena se meter; mas eu, nesse caso, posso responder"-lhes
da mesma forma. Estamos falando a sério; caso não desejem me agraciar
com sua atenção, não vou me humilhar. Tenho o subsolo.

Entrementes sigo vivendo e desejando, e que meus braços sequem se eu
carregar um tijolinho que seja desse prédio de aluguel\footnote{Alusão
  polêmica a uma expressão encontrada nos livros de Victor Considerant
  (1808--1893), socialista utópico francês, discípulo de Fourier:
  ``Carrego minha pedra para o edifício da sociedade futura''. Tais
  palavras, polemizando com os socialistas utópicos, são repetidas por
  Razumikhin e Raskólnikov no romance de Dostoiévski \emph{Crime e
  Castigo.} {[}\textsc{n.\,e.}{]}}! Não reparem no fato de há pouco eu mesmo ter
repudiado o edifício de cristal apenas pelo motivo de ser impossível
provocá"-lo com a língua. Não o disse por gostar tanto de mostrar a
língua. Pode ser que tenha me irritado apenas por não ter encontrado até
agora nenhum edifício dentre os seus ao qual não mostraria a língua.
Pelo contrário, deixaria me cortarem a língua, por pura gratidão, se as
coisas se arranjassem de forma que nunca mais tivesse vontade de
mostrá"-la. O que tenho a ver se tal arranjo é impossível, e se for
preciso se satisfazer com os apartamentos? Por que fui constituído com
esses desejos? Será que fui constituído assim para chegar à conclusão de
que toda minha constituição é pura balela? Será que todo o objetivo é
esse? Não creio.

E, a propósito, fiquem sabendo: estou convicto de que essa gente do
subsolo precisa ser mantida a rédea curta. Talvez possa ficar quarenta
anos calada no subsolo mas, quando irrompe à luz, sai falando, falando,
falando\ldots{}

\section{XI}

No final das contas, senhores, o melhor é não fazer nada! O melhor é a
inércia consciente! Assim, viva o subsolo! Embora eu tenha dito invejar
o homem normal até a última bílis, nessas condições em que o vejo não
gostaria de ser ele (embora mesmo assim não deixe de invejá"-lo. Não,
não, o subsolo é em todo caso mais proveitoso!) Lá pelo menos é
possível\ldots{} Ei! Olha como estou mentindo! Estou mentindo porque sei,
como dois e dois, que o melhor não é o subsolo, mas algo diferente,
completamente diferente, pelo que anseio, mas que não encontro de jeito
nenhum! Para o diabo com o subsolo!

Vejam o que seria ainda melhor: se eu acreditasse em alguma coisa de
tudo que escrevi agora. Juro"-lhes, senhores, que não acredito em
nenhuma, nenhuma das palavrinhas que redigi agora! Ou seja, talvez eu
acredite, mas, ao mesmo tempo, não sei por que, sinto e suspeito estar
mentindo como um sapateiro.

--- Então por que escreveu isso tudo? --- os senhores dirão.

--- E se eu deixasse os senhores encerrados por quarenta anos sem qualquer
ocupação e os visitasse quarenta anos depois, no subsolo, para saber a
que ponto chegaram? Seria possível largar uma pessoa por quarenta anos
sozinha, sem fazer nada?

--- Isso é uma vergonha, isso é uma humilhação! --- os senhores talvez me
diriam, balançando as cabeças com desprezo. --- O senhor anseia pela vida
e resolve as questões vitais com barafundas de lógica. Como são
impertinentes, como são insolentes as suas saídas e, ao mesmo tempo,
como tem medo! O senhor profere disparates e está satisfeito com eles; o
senhor profere insolências, teme"-as o tempo todo e fica pedindo
desculpas. Assegura que não tem medo de nada e, ao mesmo tempo, bajula
nossa opinião. Assegura que range os dentes e, ao mesmo tempo, graceja
para nos fazer rir. Sabe que seus gracejos não são espirituosos mas,
pelo visto, está muito satisfeito com sua dignidade literária. Pode ser
que tenha sofrido de fato, mas não tem nenhum respeito pelo próprio
sofrimento. Há verdade no senhor, porém nenhum pudor; a mais mesquinha
vaidade o leva a arrastar essa verdade à exibição, à vergonha, ao
mercado\ldots{} O senhor de fato deseja dizer algo, porém escamoteia por
temor a última palavra, pois não possui a firmeza para exprimi"-la,
apenas o atrevimento covarde. Jacta"-se de sua conscisência, mas fica
vacilando, pois, embora sua inteligência funcione, seu coração é turvado
pela depravação, e sem um coração puro não haverá consciência plena e
justa. E quanta impertinência há no senhor, como fica insistindo, como
faz caretas! Mentira, mentira e mentira!

Obviamente, todas essas palavras dos senhores fui eu que agora redigi.
Isso também é do subsolo. Fiquei lá por quarenta anos seguidos, ouvindo
essas palavras por uma pequena fresta. Eu as inventei, pois só havia
isso para inventar. Não é de se estranhar que as tenha aprendido de cor
e assumido forma literária\ldots{}

Mas será, será que os senhores são mesmo tão crédulos a ponto de
imaginar que vou imprimir isso tudo e ainda lhes dar para ler? E eis
ainda outro problema: por que, de fato, eu os chamo de ``senhores'', por
que me dirijo aos senhores como se fossem meus leitores de verdade? Esse
tipo de confissão que tenciono começar a relatar não se imprime, nem se
oferece à leitura alheia. Pelo menos não tenho tamanha firmeza, nem
considero necessário ter. Mas vejam: veio"-me à mente uma fantasia, que
desejo realizar a todo custo. A questão é a seguinte.

Nas lembranças de qualquer pessoa, há coisas que ela não revela a todos,
apenas aos amigos. Há outras que não revela nem aos amigos, apenas a si
mesma e, ainda assim, em segredo. Mas há, finalmente, aquelas que tem
medo de revelar até a si mesma, coisas que toda pessoa honrada acumula
bastante. Chega a ser assim: quanto mais honrada a pessoa, mais coisas
dessas ela possui. Pelo menos, eu mesmo só decidi há pouco tempo
recordar algumas de minhas antigas aventuras, que até então havia sempre
contornado, até com algum desassossego. Agora que não apenas recordo,
como até resolvi anotá"-las, desejo colocar à prova: é possível ser
completamente franco consigo mesmo e não ter medo da verdade por
inteiro? Observo a propósito: Heine afirma que autobiografias verídicas
são quase impossíveis, e que a pessoa com certeza mente a seu próprio
respeito\footnote{No segundo tomo do livro \emph{Sobre a Alemanha,}
  publicado na França, nas \emph{Confissões} (1853--1854), Heine
  escreveu: ``A redação das próprias caracerísticas teria sido um
  trabalho não apenas incômodo, mas simplesmente impossível
  {[}\ldots{}{]} apesar de todo desejo de ser franco, ninguém consegue
  dizer a verdade sobre si mesmo''. Daí ele afirma que Rousseau, em suas
  \emph{Confissões, ``}faz afirmações mentirosas para esconder sua
  verdadeira conduta'', ou por vaidade. {[}\textsc{n.\,e.}{]}}. Em sua opinião,
Rousseau, por exemplo, mentiu de forma constante e premeditada a seu
próprio respeito em suas confissões, por vaidade. Tenho certeza de que
Heine está certo: lembro muito bem como, às vezes, por pura vaidade, é
possível atribuir a si mesmo verdadeiros crimes, e chego a conceber
muito bem de que tipo pode ser essa vaidade. Heine, porém, falava de uma
pessoa que se confessa em público. Já eu escrevo apenas para mim mesmo,
e declaro de uma vez por todas que, se escrevo como se me dirigisse a
leitores, é só por exibição, porque assim é mais fácil escrever. É a
forma, apenas a forma vazia, jamais terei leitores. Já declarei isso\ldots{}

Não desejo ter nenhuma restrição na redação de minhas memórias. Não vou
estabelecer nenhuma ordem ou sistema. Vou escrever o que lembrar.

Pois bem, os senhores, por exemplo, podiam me levar ao pé da letra e
perguntar: se o senhor de fato não conta com leitores, então por que faz
esses acordos consigo mesmo e até põe no papel que não vai estabelecer
ordem ou sistema, que vai escrever o que lembrar, etc, etc? A quem está
se explicando? Com quem está se desculpando?

--- Vejam só --- respondo.

Aqui, a propósito, há toda uma psicologia. Talvez eu seja apenas um
covarde. E pode ser que eu imagine um público deliberadamente, para me
comportar de forma mais decente quando for escrever. Os motivos podem
ser milhares.

Mas ainda: para que, por que exatamente quero escrever? Se não for para
o público, por que não posso recordar tudo mentalmente, sem passar para
o papel?

Isso mesmo; só que no papel tudo fica um pouco mais solene. Tem algo de
mais inspirador, serei mais judicioso e o estilo vai melhorar. Além
disso, pode ser que escrever realmente me propicie alívio. Pois agora,
por exemplo, aflige"-me particularmente uma recordação ridícula. Veio"-me
à lembrança com clareza há alguns dias e ficou comigo até agora, como um
motivo musical enfadonho que não quer ir embora. Contudo, tenho que me
livrar dela. Possuo centenas dessas lembranças; de tempos em tempos,
porém uma dessas centenas se separa de alguma forma e me aflige. Por
algum motivo creio que, se escrever, ela vai embora. Por que não tentar?

Por fim, estou entediado, e fico constantemente sem fazer nada. Escrever
seria de fato como um trabalho. Dizem que o trabalho torna o homem bom e
honrado. Eis, pelo menos, uma chance.

Agora temos uma neve quase úmida, amarela, turva. Ontem também nevou,
vem nevando há dias. Tenho a impressão de que foi a propósito da neve
úmida que me lembrei dessa anedota de que agora desejo me livrar. Pois
bem, então será uma novela a propósito da neve úmida.

\chapter{A propósito da neve úmida\footnote[*]{O crítico e memorialista
  P. V. Ánnenkov, no artigo \emph{Notas sobre a literatura russa} (1849)
  já notava que ``a chuvinha molhada e a neve úmida'' apareciam como
  elementos infalíveis da paisagem de São Petersburgo nas novelas dos
  escritores da ``escola natural'' e seus imitadores. {[}\textsc{n.\,e.}{]}}}

\epigraph{Quando das trevas do erro\\
Com ardente palavra de persuasão\\
Retirei a alma caída,\\
E, plena de profundo tormento,\\
Amaldiçoaste, torcendo os braços,\\
O vício que te enredava;\\
Quando a consciência esquecida\\
Puniste com a lembrança,\\
E me narraste o conto\\
De tudo que houve antes de mim,\\
E, de repente, tapando o rosto com as mãos,\\
Plena de vergonha e horror,\\
Resultaste em pranto,\\
Revoltada, comovida\ldots{}\\
etc, etc, etc.}{\textsc{de uma poesia de n. a. nekrássov}\footnotemark}\footnotetext{Nikolai Aleksêievitch
  Nekrássov (1821--1878), poeta contemporâneo de Dostoiévski. {[}\textsc{n.\,t.}{]}}

\section{I}

Naquele tempo eu tinha apenas 24 anos. Minha vida já era então sombria,
desregrada e solitária ao ponto da selvageria. Não me dava com ninguém,
evitava até falar e cada vez mais me enfurnava em meu canto. No serviço,
na chancelaria, até me esforçava por não olhar para ninguém, e notava
muito bem que meus colegas não apenas me consideravam esquisito, como ---
sempre tive essa impressão --- me olhavam com certo asco. Veio"-me à
mente: por que ninguém, além de mim, tem a impressão de ser fitado com
asco? Um dos colegas de chancelaria tinha um rosto repugnante e picado
de varíola, e até parecia um bandido. Tenho a impressão de que não
conseguiria nem olhar para ninguém se tivesse um rosto tão indecente.
Outro tinha um uniforme tão ensebado que fedia ao seu redor. Contudo,
nenhum desses senhores se perturbava, nem pelo traje, nem pelo rosto,
nem pela moral. Nem um nem outro imaginavam ser fitados com asco; e, se
imaginassem, não faria diferença, desde que o olhar não partisse da
chefia. Agora está completamente claro para mim que, em consequência de
minha vaidade irrestrita e, portanto, da exigência para comigo mesmo, eu
me olhava com muita frequência com insatisfação violenta, que chegava ao
asco e, por isso, atribuía mentalmente meu olhar aos outros. Por
exemplo, odiava meu rosto, achava"-o torpe, chegava a suspeitar que tinha
uma expressão vulgar e, por isso, cada vez que aparecia no serviço,
esforçava"-me penosamente por me manter o mais independente possível,
para que não desconfiassem de minha vulgaridade, e o rosto exprimisse a
maior nobreza possível. ``Que seja um rosto feio --- pensava ---, mas, em
compensação, que seja nobre, expressivo e, principalmente, de uma
inteligência \emph{extraordinária}''. Só que eu sabia, de forma doída e
certa, que meu rosto jamais exprimiria tais qualidades. Mas o mais
horrível é que eu o achava decididamente estúpido. E eu teria me
conformado completamente com a inteligência. Eu teria concordado até
mesmo com uma expressão vulgar, desde que ao mesmo tempo achassem meu
rosto terrivelmente inteligente.

Obviamente, eu odiava todos os colegas, do primeiro ao último,
desprezava todos e, ao mesmo tempo, era como se os temesse. Ocorreu de
eu de repente colocá"-los acima de mim. Acontecia de modo algo repentino:
ora desprezava"-os, ora os colocava acima de mim. Uma pessoa evoluída e
honesta não pode ser vaidosa sem uma exigência irrestrida para consigo
mesmo e sem desprezar"-se, em alguns momentos, até o ódio. Porém,
desprezando ou colocando a cima de mim, baixava os olhos a quase todo
encontro. Cheguei a fazer experiências, se aguentaria o olhar daquele
ali, e sempre era o primeiro a baixar. Isso me atormentava até dar
raiva. Também tinha um medo doentio de ser ridículo e, por isso,
idolatrava servilmente a rotina em tudo que era externo; entregava"-me
com amor ao caminho trilhado por todos, temendo com toda a alma qualquer
excentricidade. Mas como poderia resistir? Eu era evoluído de forma
doentia, como deve ser evoluído um homem de nosso tempo. Já eles eram
todos obtusos e parecidos uns com os outros, como carneiros no rebanho.
Pode ser que apenas eu, em toda a chancelaria, me achasse o tempo todo
um covarde e um escravo, e o achasse exatamente por ser evoluído. Mas
não era só impressão, era isso mesmo: eu era um covarde e um escravo.
Digo sem nenhum acanhamento. Todo homem honesto de nosso tempo é e deve
ser um covarde e um escravo. Essa é sua condição normal. Estou
profundamente convencido disso. Foi feito assim e ajustado para isso. E
não no presente, devido a circustâncias casuais, mas em todos os tempos
o homem honesto tem que ser um covarde e um escravo. É a lei da natureza
para todas as pessoas honestas sobre a terra. Caso lhe ocorra de se
fazer de valente perante alguém, que não se alegre nem se arrebate com
isso: há de se acovardar perante outro. Esse é o resultado único e
eterno. Só os asnos e seus híbridos são valentes e, mesmo assim, até
determinado muro. Não se deve nem prestar atenção neles, pois não querem
dizer absolutamente nada.

Uma outra circunstância ainda me atormentava: justamente que ninguém
parecia comigo, e eu não me parecia com ninguém. ``Sou sozinho e eles
são \emph{todos''}, pensava, e caía em melancolia.

Só isso deixa evidente que eu ainda era bem criança.

Acontecia também o oposto. Pois às vezes tinha repulsa de ir à
chancelaria: cheguei ao ponto de muitas vezes voltar do serviço doente.
Mas de repente, sem mais nem menos, vinha uma fase de ceticismo e
indiferença (sempre tive fases), e eu ria de minha intolerância e
aversão, recriminando"-me por meu \emph{romantismo.} Ora não queria falar
com ninguém, ora chegava não apenas a puxar conversa, como a imaginar
fazer amizade. Toda aversão de repente desaparecia num instante, sem
mais nem menos. Quem sabe, pode ser que jamais tenha existido, será que
era falsa, livresca? Não resolvi essa questão até agora. Certa vez
fiquei mesmo amigo deles, passei a frequentar suas casas, jogar
\emph{préférence}\footnote{Jogo de cartas. {[}\textsc{n.\,t.}{]}}, tomar
vodca, discorrer sobre a indústria\ldots{} Mas aqui me permitam fazer uma
digressão.

Nós, russos, falando em geral, nunca tivemos os estúpidos românticos
alemães e especialmente franceses, pairando acima das estrelas, sobre os
quais nada age; mesmo que a terra rache sob seus pés, mesmo que toda a
França pereça nas barricadas, continuarão os mesmos, não mudarão nem por
decoro, e ficarão o tempo todo cantando suas canções das estrelas, por
assim dizer, até o fim da vida, pois são uns imbecis. Na terra russa não
há imbecis; isso é sabido; isso nos distingue das terras alemãs. Como
consequência, naturezas siderais não se encontram entre nós em condição
pura. Foram nossos publicistas e críticos ``afirmativos'' de então, à
caça dos Kostanjoglos e dos tios Piotr Ivánovitch\footnote{O
  proprietário de terras exemplar Kostanjoglo aparece no segundo tomo
  das \emph{Almas Mortas,} de Gógol. Piótr Ivánovitch Adúiev --- do
  romance \emph{Uma história comum} (1847)\emph{,} de I. A. Gontcharov
  --- é a encarnação do pensamento sensato e do empreendedorismo prático.
  {[}\textsc{n.\,e.}{]}}, e tomando"-os bestamente como nosso
ideal, que imaginaram nossos românticos, calculando"-os tão no espaço
sideral como na Alemanha ou na França. Pelo contrário, a característica
do nosso romântico está em total e direta oposição ao europeu sideral, e
nenhuma medida europeia serve aqui. (Perdoem"-me por empregar a palavra
``romântico'', palavrinha antiga, venerável, emérita e conhecida de
todos.) A característica do nosso romântico é tudo compreender,
\emph{tudo ver e ver com frequência de forma incomparavelmente mais
clara do que as nossas inteligências mais afirmativas;} não se conformar
com nada nem com ninguém mas, ao mesmo tempo, não desdenhar nada; tudo
contornar, ceder a tudo, ser político com todos; nunca perder de vista o
objetivo proveitoso, prático (uns apartamentozinhos do Estado, umas
aposentadoriazinhas, umas estrelinhas), contemplar esse objetivo através
de todo entusiasmo e volumezinhos de versos líricos e, ao mesmo tempo,
conservar indestrutível, até o fim da vida, ``o belo e o sublime'', e
conservar a si mesmo por completo, como uma pequena joia em um algodão,
ainda que, por exemplo, em proveito daquele ``belo e sublime''. Nossos
romântico é uma pessoa ampla e o primeiro velhaco de todos nossos
velhacos, asseguro"-lhes\ldots{} até por experiência. Tudo isso, obviamente,
se o romântico for inteligente. Mas o que estou dizendo? O romântico
sempre é inteligente, apenas quis observar que, embora tenha havido
imbecis românticos entre nós, eles não contam apenas porque, no
desabrochar de suas forças, transfiguraram"-se definitivamente em alemães
e, para guardar melhor sua pequena joia, instalaram"-se por lá, a maioria
em Weimar ou na Floresta Negra. Eu, por exemplo, desprezava francamente
meu emprego, e não cuspia nele só por ser indispensável, porque estava
lá e recebia dinheiro por isso. Como resultado, reparem, eu não cuspia.
Nosso romântico antes enlouquece (o que, por sinal, sucede muito
raramente), mas não vai cuspir se não tiver outra carreira em vista, não
será expulso aos empurrões, talvez seja levado ao manicômio como ``rei
da Espanha''\footnote{O ensandecido Popríschin se acha o rei da Espanha
  na novela \emph{Diário de um louco,} de Gógol (1835). {[}\textsc{n.\,e.}{]}}, e
mesmo assim só se ficar muito louco. Mas só os frouxos e loiros ficam
loucos. Um número incontável de românticos chega, em consequência, a
cargos importantes. Uma rara versatilidade! E que capacidade para os
sentimentos mais contraditórios! Então eu já me consolava com isso, e
continuo da mesma opinião. Por isso temos tantas ``naturezas amplas'',
que não perdem o ideal mesmo após a derradeira queda; ainda que não
mexam um dedo por esse ideal, ainda que sejam rematados bandidos e
ladrões, estimam o ideal original até as lágrimas, com uma
extraordinária honra na alma. Sim senhores, apenas entre nós o mais
rematado canalha pode ter a mais completa e elevada honra na alma sem,
ao mesmo tempo, em nada deixar de ser um canalha. Repito, é tão comum
que nossos românticos às vezes saiam tão velhacos (emprego a palavra
``velhacos'' com amor) no trabalho, demonstrem tamanho faro para a
realidade e conhecimento positivo que a chefia estupefacta e o público
só fazem estalar a língua de pasmo.

A versatilidade é realmente de pasmar, e sabe Deus no que ela vai se
converter e que forma adquirir nas mais recentes circunstâncias, e o que
nos pressagia para o futuro. Mas esse material não é nada mau, senhores!
Não o digo por uma patriotada ridícula! A propósito, estou certo de que
os senhores pensam que estou tirando sarro. Quem sabe, pode ser o
contrário, ou seja, estão seguros de que eu penso mesmo assim. Em todo
caso, senhores, considerarei ambas as suas opiniões uma honra, com
especial satisfação. E perdoem minha digressão.

Obviamente não mantive a amizade com meus camaradas, separei"-me
rapidamente e, em consequência da inexperiência da juventude, parei até
de cumprimentá"-los, rompendo com eles. Isso, aliás, me aconteceu apenas
uma vez. Em geral, eu estava sempre só.

Em casa, em primeiro lugar, lia cada vez mais. Tinha vontade de que as
sensações exteriores abafassem tudo que fervilhava incessantemente
dentro de mim. E minha única possibilidade de sensações exteriores era a
leitura. Claro que a leitura ajudava muito: emocionava, deleitava,
atormentava. Porém, às vezes dava um tédio terrível. Apesar de tudo,
tinha vontade de me movimentar e, de repente, mergulhava no escuro, no
subterrâneo, no abjeto: não na perversão, mas na perversãozinha. Minhas
paixõezinhas eram agudas, ardentes devido ao caráter sempre doentio de
minha irritação. Os arrebatamentos eram histéricos, com lágrimas e
convulsões. Além da leitura, não havia para onde ir, ou seja, não havia
nada ao meu redor que eu pudesse respeitar e que me atraísse. Ainda por
cima, acumulava"-se a angústia: surgiu uma sede histérica de contradição,
de contraste, e eu me joguei na perversão. Mas não foi de jeito nenhum
para me justificar que comecei a falar tanto agora\ldots{} Aliás, não! É
mentira! Queria justamente me justificar. Esse lembrete faço agora para
mim, senhores. Não quero mentir. Dei minha palavra.

Caía na perversão solitário, à noite, em segredo, temeroso, sujo, com
uma vergonha que não me deixava nos instantes mais repugnantes, quando
chegava à maldição. Já então eu trazia o subsolo na alma. Tinha um medo
terrível de que alguém me visse, encontrasse, reconhecesse. Naquela
época, eu frequentava vários lugares bastante tenebrosos.

Certa vez, passando à noite por uma pequena taverna, vi pela janela
iluminada uns senhores se atracando com tacos de bilhar, com um deles
sendo jogado pela janela. Em outra ocasião, teria sentido muito asco;
porém, naquele instante, meu estado era tal que invejei o senhor que
fora jogado a ponto de ingressar no bilhar da taverna: ``Quem sabe eu
entro na briga e também sou jogado pela janela''.

Não estava bêbado, mas o que os senhores querem que eu faça, a angústia
pode nos deixar tão histéricos! Só que não deu em nada. Revelou"-se que
eu não conseguia nem pular pela janela, e saí sem brigar.

Ao primeiro passo, um oficial me enquadrou.

Eu estava de pé perto da mesa de bilhar e, por ignorância, bloqueava a
passagem; ele me pegou pelos ombros e, em silêncio --- sem advertência
nem explicação ---, mudou"-me de lugar e passou por mim como se não me
notasse. Eu teria perdoado uma surra, mas não tinha como perdoar ser
mudado de lugar sem ser notado de forma tão definitiva.

Sabe o diabo o que eu teria dado então por uma briga de verdade, mais
correta, mais decente, mais, por assim dizer, \emph{literária}! Fui
tratado como uma mosca. O oficial tinha uns dez
\emph{verchóks}\footnote{Antiga medida russa. O oficial teria cerca de
  1,86m. {[}\textsc{n.\,e.}{]}} de altura; já eu sou baixinho e mirrado. A briga,
por sinal, estava em minhas mãos: bastava protestar, e é claro que seria
jogado pela janela. Mas mudei de ideia e preferi\ldots{} me escafeder,
exaltado.

Saí da taverna perturbado e alvoroçado, direto para casa e, no dia
seguinte, continuei minha perversão mais tímido, retraído e triste do
que antes, como se tivesse lágrimas nos olhos, mas continuei assim
mesmo. Não achem, aliás, que me intimidei com o oficial por covardia:
nunca fui covarde na alma, embora na prática tenha me acovardado
ininterruptamente, porém segurem o riso, há uma explicação para isso;
tenho explicação para tudo, estejam certos.

Oh, se esse oficial fosse daqueles que concordam em duelar! Só que não,
tratava"-se justamente de um daqueles senhores (que pena! Que
desapareceram há tempos) que preferiam agir com tacos ou, como o tenente
Pirogov, de Gógol, com a chefia\footnote{Na novela \emph{Avenida Névski}
  (1835), de Gógol, depois de ser espancado por um marido ofendido, um
  artesão alemão, o tenente Pirogov queria se queixar ao general e, ao
  mesmo tempo, ``apresentar uma petição por escrito ao Estado"-Maior''.
  {[}\textsc{n.\,e.}{]}}. Não se batiam em duelo e, com os tipinhos à paisana,
achavam que o duelo seria em todo caso indecente e, em geral,
consideravam o duelo inconcebível, coisa de livre"-pensador, de francês,
embora saíssem ofendendo à vontade, especiamente no caso de terem dez
\emph{verchóks} de altura.

Não me intimidei por covardia, mas por vaidade ilimitada. Não tive medo
dos dez \emph{verchóks} de altura, nem de apanhar doído e ser jogado
pela janela; minha coragem física era suficiente, de verdade; mas a
coragem moral faltou. Tive medo de que todos os presentes --- do marcador
insolente ao funcionarizinho putrefato e espinhento que circulava por
ali com o colarinho engordurado --- não entendessem e caíssem no riso
quando eu protestasse e me dirigisse a eles em linguagem literária. Pois
sobre o ponto de honra, ou seja, não sobre a honra, mas sobre o ponto de
honra (\emph{point d'honneur}) não temos até agora outro jeito de falar,
senão em linguagem literária. Em linguagem corrente, o ``ponto de
honra'' não se faz compreender. Estava plenamente convicto (um faro para
a realidade, apesar de todo o romantismo!) de que todos simplesmente
rebentariam de rir, e o oficial não iria simplesmente, ou seja,
inofensivamente, me espancar, como não deixaria de me dar joelhadas,
fazendo"-me rodear dessa forma a mesa de bilhar, e depois talvez
concedesse a graça de me jogar pela janela. Obviamente, essa mísera
história não podia terminar, para mim, só com isso. Depois, encontrei
com frequência esse oficial na rua, e prestei bastante atenção nele. Só
não sei se ele me reconheceu. Não deve ter reconhecido; concluo por
alguns indícios. Mas, eu, eu o fitava com ódio e raiva, e assim
seguiu\ldots{} por alguns anos! Minha raiva chegou a se fortalecer e aumentar
com os anos. No começo, devagarinho, pus"-me a me informar a respeito
desse oficial. Foi difícil, pois eu não conhecia ninguém. Certa vez,
porém, alguém gritou seu sobrenome na rua, quando eu o seguia à
distância, como se estivesse atado a ele, e assim fiquei sabendo como se
chamava. Outra vez, segui"-o até seu apartamento e, por uma moeda de dez
copeques, fiquei sabendo pelo zelador onde morava, em que andar, se
sozinho ou com mais alguém, etc --- em suma, tudo que dá para apurar via
zelador. Certa manhã, embora jamais tivesse praticado a literatura,
veio"-me de repente a ideia de descrever esse oficial de forma
acusatória, uma caricatura em uma novela. Escrevi a novela com prazer.
Acusei, cheguei a difamar; no início, deformei o sobrenome de modo que
fosse possível reconhecê"-lo imediatamente mas, depois de uma reflexão
madura, troquei e mandei para os ``Anais da Pátria''\footnote{Revista
  literária publicada entre 1818 e 1884. {[}\textsc{n.\,t.}{]}}. Só que naquela
época as acusações não estavam em moda, e minha novela não foi
publicada. Isso me deu muito desgosto. A raiva às vezes simplesmente me
sufocava. Por fim, decidi desafiar meu oponente para um duelo. Redigi
uma carta maravilhosa e atraente, implorando que se desculpasse comigo;
em caso de recusa, desafiava para um duelo com muita firmeza. A carta
estava redigida de modo que o oficial, caso entendesse um pouquinho do
``belo e sublime'', sairia então correndo até mim sem falta, para se
atirar no meu pescoço e me propor amizade. E como isso teria sido bom!
Teríamos vivido tão bem! Tão bem! Ele me defenderia com sua imponência;
eu o enobreceria com minha sofisticação, bem, e\ldots{} com as ideias, e
muitas outras coisas poderiam acontecer! Imaginem, já haviam passado
dois anos desde sua ofensa, e meu desafio era o mais indecoroso
anacronismo, apesar de toda habilidade de minha carta, que explicava e
ocultava o anacronismo. Porém, graças a Deus (até hoje agradeço ao
Altíssimo com lágrimas) não mandei minha carta. Um frio me percorre a
pele quando me lembro do que poderia ter sido se tivesse mandado. E de
repente\ldots{} e de repente me vinguei da forma mais simples e mais genial!
De repente me ocorreu uma ideia luminosa. Às vezes, nos feriados, eu
frequentava a Avenida Névski às três horas, passeando do lado
ensolarado. Quer dizer, não tinha nada de passeio, eu experimentava
incontáveis tormentos, humilhações e derrames de bílis; mas era disso
mesmo que precisava. Esgueirava"-me entre os passantes como um dojô, da
forma mais feia, cedendo passo o tempo todo ora a generais, ora a
oficiais de cavalaria e dos hussardos, ora a senhoras; nessa hora,
sentia dores convulsivas no coração e um calor na espinha só de imaginar
a \emph{misère}\footnote{Miséria, em francês russificado no original.
  {[}\textsc{n.\,t.}{]}} de meu terno, a \emph{misère} e a vulgaridade de minha
figurinha a se esgueirar. Era o tormento dos tormentos, uma humilhação
incessante e insuportável do pensamento, que vinha da sensação
incessante e imediata de que eu era uma mosca para todo aquele mundo,
uma mosca abjeta e indecente, mais inteligente do que todos, mais
evoluída do que todos, mais nobre do que todos, é evidente, mas
incessantemente uma mosca que cedia passo a todos, humilhada por todos e
ofendida por todos. Para que acumulava esse sofrimento, por que ia até a
Névski, não sei, mas simplesmente era \emph{arrastado} para lá a cada
oportunidade.

Já então comecei a experimentar aqueles acessos de deleite de que falei
no primeiro capítulo. Depois da história com o oficial, comecei a ser
atraído para lá com ainda mais força: era na Névski que o encontrava
mais, era lá que o admirava. Ele também ia para lá com maior frequência
nos feriados. Embora também cedesse passagem aos generais e pessoas
importantes, e também serpenteasse entre eles como um dojô, os que eram
como eu, ou piores, ele simplesmente esmagava; ia direto para cima
deles, como se tivesse espaço aberto diante de si, e não saía do caminho
em hipótese alguma. Eu me embriagava de raiva ao contemplá"-lo, e\ldots{}
exacerbado, cedia passagem toda vez. Atormentava"-me que nem na rua
conseguisse estar em pé de igualdade com ele. ``Por que você é
infalivelmente o primeiro a sair da frente? --- eu me importunava, em
histeria furiosa, acordando às vezes às duas da manhã. --- Por que justo
você, e não ele? Afinal isso não é uma lei, não está escrito em lugar
nenhum. Que seja igual, como costuma acontecer quando duas pessoas
delicadas se encontram: ele se desvia metade, você outra metade, e ambos
passam, respeitando"-se mutuamente''. Mas não era assim, eu acabava
cedendo passagem, e ele nem reparava no meu desvio. Então, uma ideia
espantosa de repente me ocorreu. ``E se --- pensei ---, ao encontrá"-lo,
eu\ldots{} não ceder passagem? Não ceder passagem de propósito, talvez até
chegar a empurrá"-lo: ah, como vai ser?'' Essa ideia ousada aos poucos se
apossou de mim, a ponto de não me dar sossego. Sonhava com isso o tempo
todo, de forma terrível, ia mais à Névski, com o propósito de imaginar
com maior frequência como o faria, quando o fizesse. Estava em êxtase.
Cada vez mais, tal intento me parecia provável e possível. ``Obviamente,
não propriamente empurrar --- eu pensava, enternecendo"-me antecipadamente
de alegria ---, mas assim, simplesmente não ceder passagem, dar uma
batida, não para doer muito, mas ombro a ombro, o quanto o decoro
determinar; ou seja, quanto ele bater em mim eu bato nele''. Por fim,
estava totalmente decidido. Só que os preparativos tomaram tempo demais.
Em primeiro lugar, na hora da realização eu tinha que estar com o mais
decente dos aspectos e cuidar do terno. ``Em todo caso, se, por exemplo,
virar uma história pública (e o público lá é \emph{superflu}\footnote{Em
  francês russificado no original. Aqui, a palavra quer dizer
  requintado; Nozdriov a emprega nesse sentido nas \emph{Almas Mortas,}
  de Gógol. {[}\textsc{n.\,e.}{]}}: a condessa passeia, o príncipe D. passeia, toda
a literatura passeia), é preciso estar bem trajado; isso é convincente,
e nos coloca direto, de alguma forma, em pé de igualdade aos olhos da
alta sociedade''. Com esse objetivo, pedi um adiantamento e comprei
luvas negras e um chapéu decente na loja de Tchúrkin. As luvas negras me
pareciam mais sólidas e mais de \emph{bon ton} do que as cor de limão
que inicialmente almejara. ``É uma cor muito chamativa, como se a pessoa
quisesse chamar muito a atenção'', e não levei as cor de limão. Já tinha
preparado uma boa camisa, com abotoaduras brancas de osso, há muito
tempo; o capote, porém, me atrasou demais. Meu capote em si não era nada
mau, esquentava; mas era de algodão, e o colarinho era de pele de
guaxinim, o que o fazia extremamente lacaio. Era preciso a todo custo
trocar o colarinho por um de castor, como o dos oficiais. Para isso,
pus"-me a passear pelo Gostíny Dvor\footnote{Grande e tradicional galeria
  na Avenida Névski. {[}\textsc{n.\,t.}{]}} e, depois de algumas tentativas,
coloquei minha mira em um castor alemão barato. Embora esses castores
alemães gastem muito rápido e fiquem com um aspecto miserável, no
começo, quando novos, parecem até bastante decentes; e eu só precisava
para uma vez. Perguntei o preço: mesmo assim era caro. Depois de um
raciocínio ponderado, decidi vender meu colarinho de guaxinim. A quantia
restante, que para mim era muito significativa, resolvi pedir emprestada
a Anton Antônytch Siétotchkin, meu chefe de sessão, um homem humilde,
porém sério e positivo, que não emprestava dinheiro a ninguém, mas ao
qual, ao entrar no emprego, eu fora especialmente recomendado pela
pessoa importante que me encaminhara ao serviço. Sofri de forma
horrenda. Pedir dinheiro a Anton Antônytch parecia"-me uma monstruosidade
e uma vergonha. Cheguei a ficar duas, três noites sem dormir e, no
geral, dormia pouco, estava febril; meu coração ficava vagamente gelado
ou de repente começava a pular, pular, pular!.. Anton Antônytch primeiro
ficou espantado, depois franziu o cenho, depois refletiu e assim mesmo
emprestou o dinheiro, pegando uma autorização por escrito para retirar a
quantia devida do meu salário em duas semanas. Dessa forma, tudo estava
finalmente pronto; o belo castor subiu ao trono do guaxinim asqueroso, e
dei início à ação aos poucos. Não dava para decidir tudo na primeira
ocasião, a esmo; era preciso operar com jeito, justamente de modo
gradual. Mas admito que, depois de reiteradas tentativas, até comecei a
me desesperar: não havia como trombar nele, simplesmente isso! Por mais
que eu me preparasse, por mais que tencionasse --- parece que agora mesmo
vamos trombar, fico de olho ---, novamente eu abria caminho, e ele passava
sem reparar em mim. Ao me aproximar dele, cheguei a rezar para que Deus
me concedesse a determinação. Certa vez, eu estava plenamente decidido,
mas acabou que, quando ele estava a meu alcance, no derradeiro insante,
à distância de uns dois \emph{verchóks,} não tive ânimo. Ele passou por
mim com a maior tranquilidade e eu voei para o lado, como uma bola.
Nessa noite, voltei a padecer de febre, e delirei. E, de repente, tudo
terminou de forma que não poderia ser melhor. Na noite anterior,
decidira em definitivo não realizar meu intento nefasto e deixar tudo
para trás, e me encaminhei para a Névski pela última vez, com o objetivo
de apenas ver como eu deixaria tudo para trás. De repente, a três passos
de meu inimigo, decidi"-me de forma inesperada, semicerrei os olhos e
trombamos com tudo, ombro a ombro! Não me afastei nem um \emph{verchók}
e passei por ele, absolutamente de igual para igual! Ele nem olhou, e
fez cara de não ter reparado; mas estava só fazendo pose, tenho certeza.
Tenho certeza disso até hoje. Obviamente, doeu mais em mim; ele era mais
forte, mas a questão não era essa. A questão era que eu alcancei o
objetivo, mantive a dignidade, não cedi passagem e me coloquei em
público em pé de igualdade com ele, do ponto de vista social. Voltei
para casa completamente vingado por tudo. Estava em êxtase. Triunfara, e
cantava árias italianas. Obviamente, não vou lhes descrever o que me
ocorreu nos três dias seguintes; caso tenham lido meu primeiro capítulo,
``O Subsolo'', adivinharão por si sós. Depois transferiram o oficial
para algum lugar; faz catorze anos que não o vejo. O que é feito agora
do meu querido? Está esmagando quem?

\section{II}

Minha fase de perversão terminou, porém, e fiquei tremendamente enojado.
Veio um arrependimento, que expulsei: o nojo era grande demais. Aos
poucos, contudo, acostumei"-me a isso também. Eu me acostumava a tudo, ou
seja, não exatamente acostumava, mas como que concordava de bom grado
suportar. Mas eu tinha uma saída para reconciliar tudo, que era
refugiar"-me em ``tudo que é belo e sublime'' --- claro que em sonho.
Sonhei terrivelmente, sonhei por três meses seguidos, esquecido em meu
canto e, creiam ou não, nesses momentos não me parecia com aquele senhor
que, na desordem de seu coração de galinha, pregara um castor alemão no
colarinho de seu capote. De repente, virei herói. Não teria nem sequer
aceitado a visita do meu tenente de dez \emph{verchóks} de altura.
Então, não conseguia nem imaginá"-lo. Como eram meus sonhos e como pude
me satisfazer com eles, é difícil dizer agora, mas então eu estava
satisfeito. A propósito, mesmo agora estou em parte satisfeito com eles.
Os sonhos eram especialmente mais suaves e mais intensos depois da
perversão, vinham junto com remorso e lágrimas, com maldição e êxtase.
Havia momentos de enlevo tão assertivo, de tamanha felicidade, que não
sentia em meu interior nem a menor zombaria, meu Deus. Havia fé,
esperança, amor. É que então eu acreditava cegamente que algum milagre,
alguma circustância exterior de repente alargaria, ampliaria aquilo
tudo; de repente, apresentar"-se"-ia o horizonte da atividade
correspondente, benfazeja, maravilhosa e, o principal,
\emph{completamente pronta} (nunca soube exatamente qual, mas o
principal era que estivesse completamente pronta), e então eu entraria
de repente no mundo de Deus, quase em um cavalo branco, e de coroa de
louros. Não podia compreender um papel secundário, e justamente por
isso, na realidade, desempenhava com muita tranquilidade o menor deles.
Ou herói ou lixo, não havia meio"-termo. Foi isso que me arruinou, pois
no lixo eu me consolava por ter sido herói em outra época, e o herói
encobria o lixo, dizendo: é uma vergonha para um homem comum chafurdar
no lixo, mas o herói está alto demais para se sujar; em consequência,
pode chafurdar no lixo. É notável que esses acessos de ``todo belo e
sublime'' me ocorressem na hora da perversão, exatamente quando me
encontrava no mais fundo, em lampejos isolados, como que me lembrando de
mim mesmo, sem exterminar, contudo, a perversão com seu surgimento; pelo
contário, era como se a avivassem por contraste, e ocorressem na
proporção exata para dar um bom molho. Esse molho consistia em
contradição e sofrimento, em torturante análise interna, e todos esses
tormentos e tormentozinhos conferiam algo de picante, e até sentido à
minha devassidão: em suma, cumpriam por completo a função de um bom
molho. Tudo isso não carecia de certa profundidade. Mas poderia eu
concordar com uma devassidão simples, vulgar, espontânea, de escrivão, e
suportar todo esse lixo? O que nela então poderia me cativar e me fazer
sair à noite para a rua? Não senhores, eu tinha uma escapatória nobre
para tudo\ldots{}

Mas quanto amor, senhores, quanto amor vivenciei nesses meus sonhos,
nessa ``salvação em tudo que é belo e sublime'': ainda que um amor
fantástico, ainda que um amor jamais aplicado de fato a nada humano, era
tão grande que posteriormente, de fato, não sentia nem necessidade de
aplicá"-lo: seria um luxo supérfluo. Tudo, aliás, sempre terminava do
modo mais próspero, com uma passagem preguiçosa e arrebatada à arte, ou
seja, às formas maravilhosas do ser, completamente prontas, fortemente
roubadas de poetas e romancistas e adaptadas a todos os usos e
exigências possíveis. Eu, por exemplo, triunfo sobre todos; obviamente,
todos são reduzidos a pó e coagidos a reconhecer todas minhas
perfeições, e eu perdoo a todos. Na qualidade de poeta célebre e
camarista, apaixono"-me; recebo incontáveis milhões e imediatamente os
dôo ao gênero humano\footnote{Esse sonho do herói do ``subsolo''
  renasceria posteriormente na ideia ``rothschildiana'' do protagonista
  do romance \emph{O Adolescente,} que também queria, depois de acumular
  riquezas imensas e se deleitar com o poder, doar milhões às pessoas.
  {[}\textsc{n.\,e.}{]}}, confessando de súbito, diante de todo o povo, minhas
ignomínias que, obviamente, não são simplesmente ignomínias, encerrando
em si uma quantidade extraordinária de ``belo e sublime'', algo de
manfrediano\footnote{Aqui: algo de altivo, elevado. Manfred é o herói do
  poema dramático homônimo de Byron (1817), no qual se viu refletida a
  filosofia da ``dor do mundo''. {[}\textsc{n.\,e.}{]}}. Todos choram e me beijam
(de outra forma seriam uns patetas), e eu vou ensinar novas ideias,
descalço e faminto, e desbaratar os retrógados em Austerlitz\footnote{Aqui
  e adiante, o herói se imagina no papel de Napoleão \textsc{i}. No caso dado,
  tem em vista a vitória de Napoleão \textsc{i} em Austerlitz, em 2 de dezembro
  de 1805, contra as tropas russas e austríacas. Os pesquisadores
  descobriram ligações entre os sonhos do herói e o romance
  social"-utópico \emph{Viagens a Icária} (1840), de Étienne Cabet. Na
  utopia de Cabet, o filantropo"-reformador também derrota uma coalizão
  de reis retrógrados na batalha de Austerlitz. {[}\textsc{n.\,e.}{]}}. Depois
toca"-se uma marcha, é concedida anistia, o papa concorda em se mudar de
Roma para o Brasil\footnote{O conflito entre Napoleão \textsc{i} e o papa Pio
  \textsc{vii}, em resultado do qual o imperador francês foi excomungado em 1809,
  e o pontífice transformou"-se em prisioneiro de fato de Napoleão \textsc{i} por
  cinco anos, terminou com o regresso de Pio \textsc{vii} a Roma, em 1814. {[}\textsc{n.\,e.}{]}}; depois um baile para toda a Itália em Villa Borghese, às margens
do Lago de Como\footnote{Evidentemente, tem"-se em vista aqui a
  celebração, em 1806, da fundação do Império Francês, marcada para 15
  de agosto, aniversário de Napoleão \textsc{i}. A Villa Borghese, em Roma,
  criada na primeira metade do século \textsc{xviii} e enfeitada com prédios
  elegantes, fontes e estátuas, pertencia nessa época a Camillo
  Borghese, com o qual se casou a irmã de Napoleão \textsc{i}, Pauline. O Lago de
  Como fica nos Alpes italianos. {[}\textsc{n.\,e.}{]}}, já que o Lago de Como é
transferido para Roma com esse propósito; depois uma cena nos arbustos,
etc, etc, sabiam? Os senhores dirão que é vil e vulgar trazer agora isso
tudo a público, depois de tantos enlevos e lágrimas que eu mesmo admiti.
Mas por que vil? Por acaso pensam que me envergonho disso tudo, e que
tudo isso é mais estúpido do que aconteceu nas vidas dos senhores? Além
disso, creiam que não arranjei isso de todo mal\ldots{} Nem tudo aconteceu no
Lago de Como. Aliás, os senhores têm razão; de fato, é vil e vulgar. E o
mais vil de tudo é que comecei a me justificar perante os senhores. E
ainda mais vil ter feito agora essa observação. Mas agora chega, senão
não vou acabar nunca: sempre haverá algo mais vil\ldots{}

Não estava em condições de sonhar por mais do que três meses
consecutivos, e comecei a sentir uma necessidade insuperável de me
lançar na sociedade. Lançar"-me na sociedade, para mim, significava
visitar meu chefe de seção, Anton Antônytch Siétotchkin. Foi o único
conhecido constante de toda minha vida, e agora eu mesmo me espanto com
essa circunstância. Mas eu só ia à casa dele quando estava nessa fase em
que meus sonhos haviam chegado a tamanha felicidade que era
indispensável abraçar sem falta as pessoas e toda a humanidade; para
isso, era necessária pelo menos uma pessoa que existisse de fato. Aliás,
só se podia aparecer na casa de Anton Antônytch nas terças"-feiras (era o
seu dia) e, em consequência, eu tinha que adiar minha necessidade de
abraçar toda a humanidade até terça. Esse Anton Antônytch morava nas
Cinco Esquinas\footnote{Lugar de São Petersburgo em que se encontravam a
  Avenida Zágorodny, a travessa Tchernychov (atual rua Lomonóssov), a
  rua Raziêzjaia e a rua Tróitskaia (atual rua Rubinstein). {[}\textsc{n.\,e.}{]}},
no quarto andar, e em quatro aposentos, um menor do que o outro, com
aparência muito econômica e amarela. Morava com duas filhas e a tia
delas, que servia o chá. Das filhas, uma tinha treze, e a outra catorze
anos, ambas de nariz arrebitado, e eu sempre ficava terrivelmente
embaraçado, pois elas ficavam o tempo todo cochichando entre si e dando
risadinhas. O anfitrião normalmente ficava em seu gabinete, em um sofá
de couro, na frente de uma mesa, com algum convidado grisalho, um de
nossos funcionários ou até de outro departamento. Nunca vi mais do que
dois ou três visitantes, sempre os mesmos. Discorriam sobre o imposto
sobre a bebida alcoólica, sobre as negociações no Senado, sobre os
vencimentos, sobre a indústria, sobre Sua Excelência, sobre os meios de
agradar, e assim por diante, e assim por diante. Tinha a paciência de
ficar como um bobo perto dessas pessoas, ouvindo"-as por umas quatro
horas, sem ousar nem saber o que falar. Ficava atoleimado, suava algumas
vezes, era acometido de paralisia; mas tudo isso era bom e proveitoso.
De volta para casa, adiava por algum tempo meu desejo de abraçar toda a
humanidade.

A propósito, eu tinha um outro conhecido, Símonov, ex"-colega de escola.
Talvez houvesse muitos colegas de escola meus em São Petersburgo, mas eu
não me dava com eles, e até parei de cumprimentá"-los na rua. Pode ser
que tenha ido trabalhar em outro departamento para não estar com eles e
romper de vez com toda minha odiosa infância. Maldita seja aquela
escola, aqueles horrendos anos de galés! Em suma, apartei"-me dos colegas
imediatamente, assim que fiquei livre. Sobravam duas ou três pessoas que
eu ainda cumprimentava ao encontrar. Dentre elas estava Símonov, que em
nada se distinguia na escola, era tímido e silencioso, mas no qual eu
distinguia certa independência de caráter, e até mesmo honestidade. Nem
chego a pensar que fosse muito limitado. Passei com ele alguns instantes
bastante luminosos, mas duraram pouco e, de alguma forma, cobriram"-se de
névoas. Pelo visto, ele se incomodava com as lembranças, temendo,
aparentemente, que eu voltasse a incorrer no tom de antes. Suspeito que
eu lhe causasse muita repugnância, mas mesmo assim ia à sua casa, pois
não estava realmente seguro disso.

Pois certa quinta"-feira, sem suportar a solidão, e sabendo que às
quintas a porta de Anton Antônytch estava fechada, lembrei"-me de
Símonov. Ao subir ao quarto andar, pensava justamente que esse senhor se
incomodava comigo, e que era inútil visitá"-lo. Mas acabou que tais
razões, como que de propósito, incitaram"-me ainda mais a uma situação
ambígua, e eu entrei. Fazia quase um ano que vira Símonov pela última
vez.

\section{III}

Encontrei ali mais dois de meus colegas de escola. Pelo visto, falavam
de algo importante. À minha chegada, nenhum deles prestou quase atenção
nenhuma em mim, o que era até estranho, dado que não os via há anos.
Visivelmente me consideravam algo como a mais comum das moscas. Não me
tratavam assim nem na escola, embora todos por lá me odiassem. Claro que
entendi que deviam me desprezar agora devido ao insucesso de minha
carreira profissional, por ter decaído muito, andar em maus trajes e
assim por diante, o que, a seus olhos, constituía a fachada de minha
incapacidade e importância diminuta. De qualquer forma, eu não esperava
tamanho grau de desprezo. Símonov chegou a se espantar com minha
chegada. Mesmo antes, já parecia espantado com minhas visitas. Tudo isso
me desconcertou; sentei"-me, algo angustiado, e me pus a escutar sua
conversa.

Era um papo sério e até intenso sobre o jantar de despedida que aqueles
senhores queriam organizar, no dia seguinte, para um camarada que estava
de partida para uma província longínqua, um oficial chamado Zverkov.
\emph{Monsieur} Zverkov fora meu colega durante todo o tempo de escola.
Passei a odiá"-lo particularmente nas séries mais avançadas. Nas
primeiras séries era apenas um menino bonitinho e travesso, de que todo
mundo gostava. Aliás, eu já o odiava nas primeiras séries, exatamente
por ser um menino boitinho e travesso. Sempre fora mau aluno, piorando a
cada ano; contudo, conseguiu concluir a escola com êxito, por ter
proteção. Em seu último ano de escola, recebeu como herança duzentas
almas\footnote{Servos. {[}\textsc{n.\,t.}{]}}e, como quase todos nós éramos pobres,
passou a se fanfarronear. Era vulgar até o último grau, porém um bom
sujeito, mesmo quando fanfarroneava. Apesar das formas exteriores
fantásticas e farsescas de honra e glória, todos, à exceção de muito
poucos, mais bajulavam Zverkov quanto mais ele fanfarroneava. E não
bajulavam para obter vantagem, mas por ele ser uma pessoa favorecida
pelos dons da natureza. Ademais, considerava"-se Zverkov um especialista
em habilidade e boas maneiras. Isso me deixava especialmente furioso.
Odiava o som agudo e autoconfiante de sua voz, a adoração por suas
próprias piadas, que eram terrivelmente estúpidas, embora ele tivesse o
dom da palavra; odiava seu rosto belo, porém estúpido (pelo qual, aliás,
trocaria com gosto o meu, \emph{inteligente}), e seus modos
desembaraçados de oficial da década de 1840. Odiava que ele contasse
seus futuros sucessos com as mulheres (não se decidira a começar com as
mulheres por ainda não possuir dragonas de oficial, pelas quais ansiava
com impaciência), e como se bateria em duelo a cada instante. Lembro"-me
de como, sempre calado, de repente me engalfinhei com Zverkov, certa
vez, quando ele, ao falar com os colegas, no recreio, de suas futuras
indecências, e brincando por fim como um cachorrinho ao sol, afirmou de
repente que nenhuma moça camponesa de sua aldeia seria deixada sem
atenção, que isso era \emph{droit de seigneur}\footnote{Direito de
  senhor, em francês no original. Costume feudal medieval, chamado de
  direito à primeira noite, pelo qual a camponesa de uma propriedade
  devia passar a primeira noite nupcial com seu senhor. {[}\textsc{n.\,e.}{]}},
que os mujiques que ousassem protestar seriam todos açoitados, e que
dobraria a corveia daqueles canalhas barbudos. Nossos brutamontes
aplaudiram, e eu não me engalfinhei por pena das moças e seus pais, mas
simplesmente porque aquele escaravelho estava sendo tão aplaudido. Saí
então vencedor, mas Zverkov, embora estúpido, era alegre e insolente, e
depois saiu rindo, de modo que, em verdade, não venci por completo: o
riso ficava do seu lado. Depois ele me venceu algumas vezes, mas sem
raiva, como que brincando, de passagem, rindo. De raiva e desprezo, eu
não respondia. Na época da formatura, ele deu um passo em minha direção;
não me opus muito, pois aquilo me lisonjeava; mas logo nos afastamos, de
forma rápida e natural. Depois ouvi sobre seus êxitos de caserna, de
tenente, sobre como ele \emph{farreava.} Depois vieram outros boatos,
sobre como tinha \emph{sucesso} no trabalho. Não me cumprimentava mais
na rua, e eu desconfiava que temia se comprometer ao saudar alguém tão
insignificante como eu. Também o avistei certa vez no teatro, no
terceiro balcão, já de alamares. Cortejava e se arqueava diante das
filhas de um vetusto general. Em três anos, decaíra muito, embora ainda
fosse bastante belo e hábil, como antes; dera uma inchada, começara a
engordar; era evidente que, pelos trinta anos, estaria completamente
obeso. Esse era o Zverkov que estava finalmente de partida, e ao qual
nossos colegas queriam dar um jantar. Eles mantiveram três anos de
relações ininterruptas com o tenente, embora, no fundo, tenho certeza de
que não se consideravam em pé de igualdade com ele.

Dos dois visitantes de Símonov, um era Fierfítchkin, russo de sangue
alemão, de baixa estatura, cara de macaco, um estúpido que era o riso de
todos, meu pior inimigo desde as primeiras séries, um fanfarrãozinho
vulgar e insolente, que afetava o mais melindroso amor"-próprio embora,
obviamente, fosse no fundo um covarde. Estava dentre os admiradores de
Zverkov que o bajulavam às claras e viviam tomando dinheiro emprestado
dele. O outro visitante de Símonov, Trudoliúbov, era uma pessoa que não
se destacava em nada, um militar alto, de fisionomia fria, bastante
honrado, porém que se curvava a qualquer sucesso, e que só sabia
raciocinar em termos de promoção. Era uma espécie de parente distante de
Zverkov, e isso, por mais tolo que seja dizer, conferia"-lhe alguma
importância entre nós. Sempre me considerou insignificante; tratava"-me,
embora não exatamente com cortesia, de forma suportável.

--- Pois bem, se forem sete por cabeça --- disse Trudoliúbov ---, como
estamos em três, saem vinte e um \emph{rublos}; dá para jantar bem.
Claro que Zverkov não paga.

--- É evidente, já que o estamos convidando --- decidiu Símonov.

--- Por acaso vocês acham --- interveio Fierfítchkin, arrogante e
impetuoso, como um lacaio a se jactar das estrelas de seu patrão general
---, por acaso vocês acham que Zverkov vai nos deixar pagar sozinhos? Vai
aceitar por delicadeza, mas vai pedir uma \emph{meia dúzia} por conta
própria.

--- Bem, o que nós quatro vamos fazer com meia dúzia? --- observou
Trudoliúbov, prestando atenção apenas na meia dúzia.

--- Certo, somos três, com Zverkov quatro, vinte e um rublos no
\emph{Hôtel de Paris}, amanhã, às cinco horas --- concluiu, por fim,
Símonov, que fora escolhido como o responsável.

--- Como assim vinte e um? --- eu disse, com algum nervosismo e até, pelo
visto, ofendido. --- Se me contarem, não serão vinte e um, mas vinte e
oito rublos.

Tive a impressão de que me convidar de forma tão súbita e inesperada
seria até muito belo, de que todos ficariam imediatamente vencidos, e me
olhariam com respeito.

--- Então o senhor também quer? --- notou Símonov, com insatisfação, como
que evitando me olhar. Ele me conhecia muito bem.

Enfurecia"-me que ele me conhecesse tão bem.

--- Por que não, senhores? Afinal, ao que parece, também sou um colega, e
admito que fiquei até ofendido por ter sido excluído --- voltei a
borbulhar.

--- E onde deveríamos buscá"-lo? --- interveio Fierfítchkin, rude.

--- O senhor nunca se deu bem com Zverkov --- acrescentou Trudoliúbov,
franzindo o cenho. Só que eu me aferrava e não largava.

--- Parece"-me que isso ninguém tem o direito de julgar"-- retruquei, com a
voz trêmula, como se tivesse acontecido sabe Deus o quê. --- Talvez eu
queira ir agora justamente porque antes não me dava bem.

--- Bem, quem poderá entendê"-lo\ldots{} essas coisas elevadas\ldots{} --- riu"-se
Trudoliúbov.

--- Será incluído --- decidiu Símonov, dirigindo"-se para mim. --- Amanhã, às
cinco horas, no \emph{Hôtel de Paris}; não se engane.

--- O dinheiro! --- quis começar Fierfítchkin, a meia"-voz, apontando"-me para
Símonov, mas se interrompeu, pois até Símonov ficou embaraçado.

--- Chega --- disse Trudoliúbov, levantando"-se. --- Se ele está com tanta
vontade, que venha.

--- É que temos nosso próprio círculo de amigos --- irritou"-se
Fierfítchkin, também pegando o chapéu. --- Não é uma reunião oficial. Pode
ser que nem desejemos sua presença\ldots{}

Saíram; à partida, Fierfítchkin não me fez nenhuma saudação, Trudoliúbov
mal acenou, sem olhar. Símonov, com o qual fiquei frente a frente, caíra
em em um pasmo enfadado, e me fitava com estranheza. Não se sentou, nem
me convidou a fazê"-lo.

--- Hum\ldots{} sim\ldots{} então é amanhã. Vai dar o dinheiro agora? É para saber
com certeza --- balbuciou, embaraçado.

Corei e, ao fazê"-lo, lembrei"-me de que, em tempos imemoriais, devera
quinze rublos a Símonov que, aliás, jamais esquecera, e que eu jamais
pagara.

--- Convenha, Símonov, que eu não podia saber que, ao vir para cá\ldots{} muito
me desagrada ter esquecido\ldots{}

--- Tudo bem, tudo bem, dá na mesma. Pague amanhã ao jantar. Era só para
saber\ldots{} O senhor, por favor\ldots{}

Interrompeu"-se e passou a andar pelo cômodo com ainda mais enfado. Ao
caminhar, apoiava"-se nos saltos e os batia com força.

--- Eu o estou retardando? --- perguntei, após dois minutos de silêncio.

--- Oh não! --- agitou"-se, de repente. --- Ou seja, na verdade, está. Veja,
ainda preciso ir\ldots{} Lá não é longe\ldots{} --- acrescentou, com voz de desculpa
e, em parte, envergonhado.

--- Ai, meu Deus! Por que não me dis"-se? --- gritei, pegando o boné, com um
ar surpreendentemente desenvolto, que sabe Deus de onde veio.

--- Mas não fica longe\ldots{} A dois passos daqui\ldots{} --- repetiu Símonov,
levando"-me à antessala com um ar de azáfama que não combinava com ele de
jeito nenhum. --- Então amanhã, às cinco horas em ponto! --- gritou"-me, na
escada: estava muito satisfeito com minha partida. Já eu me encontrava
em fúria.

--- Afinal, por que diabos, por que diabos fui me meter? --- rangia os
dentes, ao caminhar pela rua --- E por causa daquele canalha, por causa
daquele leitão do Zverkov! Obviamente que não preciso ir; obviamente
tenho que cuspir; sou obrigado ou o quê? Amanhã mesmo aviso Símonov pelo
correio municipal\ldots{}

Mas eu me enfurecia por saber que iria com certeza; que iria de caso
pensado; e quanto mais inconveniente e indecorosa fosse minha presença,
mais rápido eu iria.

Havia até um obstáculo determinante à minha ida: não tinha dinheiro. Ao
todo, tinha nove rublos. Porém, no dia seguinte, como ordenado mensal,
tinha que dar sete a Apollon, meu criado, que morava na minha casa por
essa quantia, comendo por conta própria.

Não pagar era impossível, a julgar pelo caráter de Apollon. Mas a
respeito desse canalha, dessa minha chaga, falarei em algum momento
posterior.

Contudo, eu sabia bem que não pagaria assim mesmo, e iria sem falta.

Nessa noite, tive o mais hediondo dos sonhos. Não é de se estranhar: era
esmagado o tempo todo por lembranças de meus anos de galé da vida
escolar, das quais não consegui me liberar. Fui enfiado naquela escola
por parentes distantes dos quais dependia, e de que desde então jamais
tive notícia, como um órfão, já amedrontado pelas broncas, já
meditabundo, taciturno e olhando assustado para tudo ao redor. Os
colegas me receberam com animosidade e zombarias impiedosas, por eu não
ser parecido com nenhum deles. Mas eu não podia suportar a zombaria; não
podia me acostumar tão fácil, como eles se acostumavam um com os outros.
Odiei"-os de imediato, encerrando"-me em um orgulho assustado, ferido e
desmedido. Sua grosseria me revoltava. Riam com cinismo da minha cara,
da minha figura desajeitada; contudo, como suas próprias caras eram
estúpidas! Na nossa escola, as expressões dos rostos degeneraram em
especial estupidez. Quantas crianças maravilhosas ingressavam nela.
Depois de alguns anos, tornava"-se repugnante encará"-las. Ainda aos
dezesseis anos, contemplava"-as sombrio; já naquela época, espantava"-me
com a miudeza de sua mentalidade, a estupidez de suas ocupações, jogos,
conversas. Não entendiam tantas coisas indispensáveis, não se
interessavam por temas tão inspiradores e surpreendentes que a
contragosto passei a considerá"-los inferiores a mim. Não era a vaidade
ultrajada que me levava a isso e, pelo amor de Deus, não me venham com
aquelas objeções estereotipadas que me dão náusea: ``que eu apenas
sonhava, enquanto eles já entendiam a realidade da vida''. Eles não
entendiam nada, nenhuma realidade da vida, e juro que isso era o que
mais me revoltava deles. Pelo contrário, a realidade mais evidente, que
saltava aos olhos, eles apreendiam com uma estupidez fantástica, e já
então estavam habituados a se curvar apenas ao êxito. Riam com crueldade
e desprezo de tudo que era justo, porém humilhado e oprimido. Tomavam
cargo por inteligência; aos dezesseis anos já discorriam sobre
sinecuras. Claro que muito ali vinha da estupidez, dos maus exemplos que
os rodeavam na infância e na adolescência. Eram depravados até a
monstruosidade. Obviamente, aí também há muita influência externa, muito
cinismo afetado; obviamente, a juventude e algum frescor cintilavam
neles mesmo por detrás da depravação; mas mesmo o frescor era neles nada
atraente, e se manifestava como uma certa perversão. Eu os odiava
terrivelmente, embora talvez fosse pior do que eles. Pagavam"-me na mesma
moeda, sem ocultar a repulsa por mim. Porém, eu não desejava mais o seu
amor; pelo contrário, ansiava o tempo todo por sua humilhação. Para me
livrar das zombarias, comecei deliberadamente a estudar o melhor
possível, chegando a ficar entre os melhores. Isso impressionou"-os.
Ademais, todos começaram a compreender aos poucos que eu lia livros que
eles não conseguiam ler, e entendia coisas (que não entravam no conteúdo
de nosso curso especial) das quais nem tinham ouvido falar. Encaravam
isso tudo com ferocidade e zombaria, mas se submetiam moralmente, ainda
mais que até os professores prestavam atenção em mim por causa disso. As
zombarias se interromperam, mas a hostilidade permaneceu,
estabelecendo"-se relações frias e tensas. Por fim, eu mesmo não
aguentei: com os anos, desenvolveu"-se uma necessidade de gente, de
amigos. Tentei começar a me aproximar deles; mas essa aproximação sempre
saía artificial, e terminava por si mesma. Certa vez, tive de alguma
forma um amigo. Só que eu era um déspota de coração; queria ter um
domínio desmedido de sua alma; queria suscitar nele desprezo por todos
que o rodeavam; exigia"-lhe um rompimento altivo e definitivo com esse
meio. Assustei"-o com minha amizade apaixonada; levei"-o às lágrimas, ao
suor; era ingênuo e entregou a alma; porém, quando se entregou de todo,
odiei"-o imediatamente, afastando"-o de mim, como se precisasse dele
apenas para derrotá"-lo, apenas para submetê"-lo. Mas eu não podia ganhar
de todos; meu amigo também não se parecia com nenhum deles, constituindo
a mais rara exceção. Ao sair da escola, a primeira coisa que fiz foi
deixar o emprego especial ao qual fora designado, para romper todas as
ligações, amaldiçoar o passado e cobri"-lo de pó\ldots{} Sabe o diabo porque
depois disso eu fui me arrastar até aquele Símonov!..

De manhã, pulei cedo da cama, dando um salto nervoso, como se tudo fosse
começar a acontecer naquele instante. Eu acreditava, porém, que
aconteceria, e seria infalivelmente naquele dia, uma reviravolta radical
na minha vida. Por falta de hábito, sei lá, durante toda a minha vida, a
qualquer acontecimento externo, por menor que fosse, sempre tive a
impressão de que naquele dia aconteceria uma reviravolta radical. Aliás,
fui para o serviço como de hábito, mas escapei para casa duas horas mais
cedo, para me preparar. O principal, pensava, é não ser o primeiro a
chegar, senão vão achar que fiquei muito alegre. Só que as coisas
principais desse gênero eram milhares, e me alvoroçavam até a exaustão.
Limpei minhas botas mais uma vez, eu mesmo; Apollon não as limparia duas
vezes no dia por nada no mundo, achando que era a desordem. Limpei"-as
depois de roubar a escova da antessala, para que ele não reparasse e
depois não passasse a me desprezar. Depois examinei detalhadamente meu
traje e achei que tudo era velho, surrado, gasto. Tinha me desleixado
muito. O uniforme talvez estivesse em ordem, mas não dava para sair para
jantar de uniforme. O principal era que nas calças havia uma enorme
mancha amarela, na altura do joelho. Eu pressentia que só essa mancha já
tiraria nove décimos da minha dignidade. Também sabia que era bem baixo
pensar assim. ``Mas agora não é para pensar; chegou a hora da
realidade'' --- pensava, e perdia o alento. Sabia também muito bem, mesmo
então, que estava exagerando monstruosamente tais fatos; mas o que
fazer: não conseguia mais me controlar, e tremia de febre. Imaginava com
desespero com que superioridade e frieza aquele ``canalha'' do Zverkov
me encontraria; com que desprezo obtuso e insuperável o imbecil do
Trudoliúbov me encararia; que riso mau e insolente o escaravelho do
Fierfítchkin me reservaria, para ser servil a Zverkov; com que perfeição
Símonov entenderia tudo, e como me desprezaria pela baixeza de minha
vaidade e pusilanimidade; e, principalmente, como tudo seria mísero, não
\emph{literário,} ultrajante. Claro que o melhor de tudo seria não ir.
Mas isso era o mais impossível de tudo: quando uma coisa começa a me
puxar, eu me enfio por inteiro, de cabeça. Depois, eu passaria a vida
inteira me provocando: ``E aí, acovardou"-se, acovardou"-se diante da
\emph{realidade,} acovardou"-se!''. Pelo contrário, tinha um desejo
apaixonado de mostrar a toda aquela ``corja'' que não era o covarde que
eu mesmo imaginava. Mais ainda: no mais alto paroxismo da febre de
covardia, sonhava em superá"-los, triunfar, arrebatar, obrigá"-los a me
amar, ainda que ``pela elevação das ideias e indiscutível
originalidade''. Eles largariam de Zverkov, que ficaria sentado à parte,
calado e envergonhado, enquanto eu o esmagaria. Depois talvez fizesse as
pazes com ele, bebesse e o tratasse por \emph{você,} mas o pior e mais
ofensivo para mim é que já então eu sabia, com absoluta certeza que, em
essência, não precisava de nada daquilo, não desejava de jeito nenhum
esmagá"-los, submetê"-los, atraí"-los, e que eu seria o primeiro a não dar
um tostão por aquele resultado, caso o atingisse. Oh, como rezei a Deus
para que aquele dia passasse rápido. Em angústia inexprimível eu ia até
a janela, abria o postigo e observava a bruma turva da neve úmida a cair
espessa\ldots{}

Finalmente, meu vetusto relógio de parede assobiou as cinco. Peguei o
chapéu e, tentando não olhar para Apollon --- que desde de manhã esperava
que eu lhe entregasse o salário mas, por orgulho, não queria ser o
primeiro a falar ---, deslizei por ele e pela porta e, com uma carruagem
de luxo, que alugara para isso com a última moeda de cinquenta copeques,
cheguei como um nobre ao \emph{Hôtel de Paris}.

\section{IV}

Já na véspera eu sabia que seria o primeiro a chegar. Mas não se tratava
mais de uma questão de primazia.

Não apenas não havia nenhum deles, como quase não achei o nosso
reservado. A mesa ainda não estava posta. O que aquilo queria dizer?
Depois de muitas indagações, finalmente fiquei sabendo do garçom que o
jantar estava encomendado para as seis horas, e não para as cinco.
Confirmei isso também no bufê. Dava até vergonha perguntar. Ainda eram
apenas cinco e vinte e cinco. Se tinham mudado a hora, em todo caso
deviam me informar; o correio municipal era para isso, em vez de me
sujeitar à desonra perante mim mesmo e\ldots{} até perante os garçons.
Sentei"-me; o garçom começou a por a mesa; em sua presença, o ultraje era
ainda maior. Pelas seis horas, além das lâmpadas acesas, levaram velas
ao aposento. O garçom, porém, nem pensara em levá"-las quando eu cheguei.
No cômodo vizinho jantavam, em mesas separadas, dois clientes sombrios,
taciturnos e de ar zangado. Um dos reservados distantes era muito
barulhento; chegavam a gritar; ouviam"-se as gargalhadas de um bando de
gente; ouviam"-se uns guinchos indecorosos em francês; era um jantar com
damas. Em suma, algo muito nojento. Raramente passei momentos mais
desagradáveis, de modo que quando eles, às seis em ponto, apareceram
todos de uma vez, eu, ao primeiro instante, alegrei"-me como se fossem
libertadores, e quase esqueci que devia demonstrar estar ofendido.

Zverkov entrou na frente de todos, evidentemente no comando. Ria, e com
ele riam todos; porém, ao me avistar, Zverkov se aprumou, caminhando
pausadamente, dobrou um pouco a cintura, algo sedutor, e me deu a mão,
afável mas não muito, com uma polidez cautelosa, quase de general, como
se, ao dar a mão, se defendesse de algo. Eu imaginava, pelo contrário,
que, logo ao entrar, ele soltaria sua gargalhada de antes, fininha e
esganiçada, e que, às primeiras palavras, começariam suas piadas e
tiradas sem graça. Eu me preparara para isso desde à véspera, mas não
esperava de jeito nenhum tamanha arrogância, um cumprimento tão altivo.
Será que agora ele se considerava acima de mim de modo tão
incomensurável, com relação a tudo? Se ele apenas quisesse me ofender
com essa pose de general, então eu achava que tudo bem; cuspiria naquilo
de alguma forma. Mas e se ele realmente não tivesse nenhum desejo de
ofender e, em seu crânio de carneiro, tivesse penetrado a sério a
ideiazinha de que estava acima de mim de modo incomensurável, e não
podia olhar para mim com ar que não fosse protetor? Essa mera suposição
já me fazia arquejar.

--- Fiquei sabendo com surpresa do seu desejo de participar --- começou,
ciciando, sussurrando e arrastando a voz, coisa que não fazia antes. ---
Nunca mais nos encontramos. O senhor nos evitava. Sem razão. Não somos
tão terríveis como o senhor pensa. Pois bem, em todo caso, estou feliz
em re"-co"-me"-çar\ldots{}

E se virou, negligente, para pendurar o chapéu na janela.

--- Está esperando há muito tempo? --- perguntou Trudoliúbov.

--- Cheguei às cinco em ponto, como foi estabelecido ontem --- respondi, em
voz alta e zangada, que prometia para logo uma explosão.

--- Mas você não o informou da mudança de horário? --- Trudoliúbov voltou"-se
para Símonov.

--- Não. Esqueci --- respondeu aquele, desprovido de qualquer
arrependimento e, sem se desculpar comigo, foi pedir uns petiscos.

--- Então o senhor está aqui há uma hora, ah, coitado! --- gritou Zverkov,
zombeteiro, já que no seu entendimento, isso devia ser de fato
terrivelmente engraçado. Atrás dele, com voz canalhinha e sonora, de
cachorro, o pulha do Fierfítchkin caiu na gargalhada. Também achava
minha situação muito engraçada e embaraçosa.

--- Não tem nenhuma graça! --- gritei para Fierfítchkin, ficando cada vez
mais zangado. --- Os culpados são os outros, não eu. Não fizeram caso de
me avisar. Isso, isso isso\ldots{} é simplesmente absurdo.

--- Não só absurdo, mas ainda outra coisa. --- resmungou Trudoliúbov,
defendendo"-me, ingênuo. --- O senhor é brando demais. Claro que não foi de
propósito. E como é que Símonov\ldots{} hum!

--- Se me aprontassem uma dessas --- notou Fierfítchkin ---, eu\ldots{}

--- Mandaria que lhe servissem alguma coisa --- interrompeu Zverkov ---, ou
simplesmente pediria o jantar, sem esperar.

--- Convenham que poderia ter feito isso sem pedir licença --- atalhei. ---
Se esperei, é\ldots{}

--- Sentemo"-nos, senhores --- gritou Símonov, retornando ---, está tudo
pronto; respondo pelo champanhe, está estupidamente gelado\ldots{} Como não
conheço o seu apartamento, onde poderia encontrá"-lo? --- voltou"-se para
mim, mas novamente sem me encarar. Pelo visto, tinha algo contra mim.
Quer dizer que tinha reconsiderado depois do dia anterior

Todos se sentaram, e eu também. A mesa era redonda. Trudoliúbov estava à
minha esquerda, Símonov à direita. Zverkov estava na minha frente;
Fierfíctchkin do lado, entre ele e Trudoliúbov.

--- Di"-ga, o senhor está\ldots{} no departamento? --- Zverkov continuou a se
ocupar de mim. Vendo que eu estava embaraçado, imaginou seriamente que
eu devia receber atenção e, por assim dizer, ânimo. ``O que é isso, será
que ele quer que eu lhe dê uma garrafada?''- pensava eu, irado.
Irritara"-me por falta de hábito, com uma rapidez antinatural.

--- Na chancelaria número\ldots{} --- respondi com voz entrecortada, olhando para
o prato.

--- E\ldots{} é vvantajoso para o ssenhor? Di"-iga, o que o leevou a deixar o
emprego anterior?

--- O que me le"-e"-evou foi minha vontade de deixar o emprego anterior ---
arrastei a voz duas vezes mais do que ele, já quase sem me controlar.
Fierfítchkin bufou. Símonov me fitou com ironia; Trudoliúbov parou de
comer e passou a me examinar com curiosidade.

Zverkov ficou chocado, mas não quis deixar transparecer.

---Be"-e"-em, e quais são os mantimentos?

--- Que mantimentos?

--- Quer dizer, os vencimentos?

--- Por que está me submetendo a exame?

Apesar disso, declarei imediatamente quanto recebia de vencimentos.
Fiquei terrivelmente corado.

--- Modesto --- observou Zverkov, com ares de importância.

--- Sim senhor, não dá para jantar em café"-restaurante! --- interrompeu
Fierfítchkin, descarado.

--- Na minha opinião, chega a ser simplesmente pobre --- notou Trudoliúbov,
sério.

--- E como o senhor emagreceu, como mudou\ldots{} desde aquela época\ldots{} ---
acrescentou Zverkov, já não sem veneno, ao examinar"-me e a meu traje com
compaixão insolente.

--- Chega de embaraçá"-lo --- gritou Fierfítchkin, entre risinhos.

--- Prezado senhor, fique sabendo que não estou embaraçado --- estourei,
por fim ---, escute"-me! Estou jantando aaqui, ``no café"-restaurante'', com
meu próprio dinheiro, o meu, e não o alheio, repare nisso,
\emph{monsieur} Fierfítchkin.

--- Co"-o"-mo? Quem aqui não está jantando com seu próprio dinheiro? O
senhor parece\ldots{} --- insistiu Fierfítchkin, vermelho como um lagostim,
fitando"-me nos olhos com furor.

--- As"-sim --- respondi, sentindo que tinha
ido longe demais ---, e creio que seria melhor nos ocuparmos de assuntos
mais inteligentes.

--- O senhor aparentemente tenciona nos exibir sua inteligência?

--- Não se preocupe, aqui isso seria completamente supérfluo.

--- O que é isso, meu senhor, que está cacarejando, hein? O senhor não
ficou ruim da cabeça naquele seu \emph{lepartamento}?

--- Chega, senhores, chega! --- gritou Zverkov, onipotente.

--- Como isso é estúpido! --- resmungou Símonov.

--- De fato, é estúpido, nós nos reunimos em um grupo de amigos para
despedir"-nos de um bom camarada que parte em \emph{voyage}\footnote{Viagem,
  em francês russificado no original. {[}\textsc{n.\,t.}{]}}, e o senhor
fica acertando as contas --- pôs"-se a falar Trudoliúbov, dirigindo"-se com
rudeza apenas a mim. --- O senhor mesmo se convidou ontem, então não
perturbe a harmonia geral\ldots{}

--- Chega, chega --- gritou Zverkov. --- Parem, senhores, isso não vai bem.
Melhor eu lhes contar como há três dias não me casei por pouco\ldots{}

Daí começou uma pasquinada sobre como aquele senhor por pouco não se
casara há três dias. Das bodas, por sinal, não houve palavra, mas em seu
conto cintilavam generais, coronéis e até fidalgos da câmara, dentre os
quais Zverkov estava quase na liderança. Desencadearam"-se risos de
aprovação; Fierfítchkin chegou a ganir.

Todos me largaram, e fiquei esmagado e aniquilado.

``Senhor, isso é companhia para mim? --- pensei. --- E que idiota me mostrei
diante deles! Aliás, permiti a Fierfítchkin muita coisa. Esses toscos
acham que me concedem uma honra ao me dar um lugar em sua mesa, mas não
entendem que sou eu, eu quem lhes concede a honra, e não eles!
``Emagreceu! O terno!'' Oh, malditas calças! Zverkov há pouco reparou na
mancha amarela do joelho\ldots{} Que mais? Agora mesmo, nesse minuto vou me
levantar da mesa, pegar o chapéu e simplesmente partir, sem proferir
palavra\ldots{} De desprezo! E amanhã, talvez, um duelo. Canalhas. Não
lamento os sete rublos. Talvez achem\ldots{} O diabo que os carregue! Não
lamento os sete rublos! Vou partir nesse minuto!\ldots{}''

Obviamente fiquei.

Por pesar, vertia copos de Lafit\emph{e} e xerez. Por falta de hábito,
embriaguei"-me rápido e, com a embriaguez, cresceu também o enfado. De
repente, veio"-me a vontade de ofendê"-los do modo mais atrevido e depois
partir. Aproveitar o momento e me exibir, depois eles que digam: apesar
de ridículo, ele é inteligente\ldots{} e\ldots{} e\ldots{} em suma, que vão para o
diabo!

Lancei"-lhes um olhar descarado, com os olhos embaciados. Mas eles já
tinham se esquecido completamente de mim. Dentre \emph{eles} havia
barulhos, gritos, alegria. Zverkov falava o tempo todo. Passei a
escutar. Zverkov contava de uma dama exuberante, que ele por fim levara
a se declarar (obviamente mentia como um cavalo), assunto no qual fora
especialmente ajudado por seu amigo íntimo, um princepezinho, o hussardo
Kólia\footnote{Apelido de Nikolai. {[}\textsc{n.\,t.}{]}}, que possuía três mil
almas.

--- No entanto, esse Kólia, que possui três mil almas, não está aqui para
se despedir do senhor --- intrometi"-me de repente na conversa. Todos se
calaram na hora.

--- O senhor já está bêbado --- dignou"-se finalmente a reparar em mim
Trudoliúbov, com um olhar de esguelha, de desprezo, na minha direção.
Zerkov me examinava em silêncio, como um inseto. Baixei os olhos.
Símonov rapidamente começou a servir o champanhe.

Trudoliúbov ergueu a taça, seguido por todos, menos eu.

--- À sua saúde e boa viagem! --- gritou para Zverkov. --- Aos velhos tempos,
senhores, ao nosso futuro, hurra!

Todos beberam e se puseram a beijar Zverkov. Não me mexi; a taça estava
na minha frente, cheia e intacta.

--- Por acaso o senhor não vai beber? --- berrou Trudoliúbov, perdendo a
paciência e se dirigindo a mim em tom ameaçador.

--- Quero fazer um brinde de minha parte, especial\ldots{} daí vou beber,
senhor Trudoliúbov.

--- Hidrófobo nojento! --- resmungou Símonov.

Aprumei"-me na cadeira e peguei a taça, febril, aprontando"-me para algo
raro e sem saber exatamente o que iria dizer.

--- \emph{Silence}\footnote{Silêncio, em francês no original. (N do T.)}!
--- gritou Fierfítchkin. --- Lá vem a inteligência! --- Zverkov aguardava,
muito sério, entendendo do que se tratava.

--- Senhor tenente Zverkov --- comecei ---, fique sabendo que eu odeio
frases, frasistas e cinturas apertadas\ldots{} Esse é o primeiro ponto, e
depois dele vem um segundo.

Todos se agitaram fortemente.

--- O segundo ponto: odeio a devassidão e os devassos. Especialmente os
devassos!

--- Terceiro ponto: amo a verdade, a franqueza e a honra --- prosseguia,
quase maquinalmente, pois começara a gelar de terror, não entendendo
como falava daquele jeito\ldots{} --- Amo o pensamento, \emph{monsieur}
Zverkov; amo a camaradagem verdadeira, em pé de igualdade, e não\ldots{}
hum\ldots{} Amo\ldots{} Aliás, para quê? Bebo à sua saúde, \emph{monsieur}
Zverkov. Seduza as circassianas, atire nos inimigos da pátria e\ldots{} e.. À
sua saúde, \emph{monsieur} Zverkov!

Zverkov levantou"-se da cadeira, fez"-me uma reverência e disse:

--- Sou"-lhe muito grato.

Estava terrivelmente ofendido, e até pálido.

--- Vá para o diabo --- bramiu Trudoliúbov, batendo com o punho na mesa.

--- Não senhor, há que se quebrar a cara dele por isso! --- guinchou
Fierfítchkin.

--- Temos que expulsá"-lo! --- resmungou Símonov.

--- Nenhuma palavra, senhores, nenhum gesto! --- gritou Zverkov, solene,
detendo a indignação geral. --- Agradeço a todos, mas eu mesmo saberei
demonstrar o valor que dou às palavras dele.

--- Senhor Fierfítchkin, amanhã mesmo o senhor há de me dar satisfação por
suas palavras de hoje! --- eu disse, alto, dirigindo"-me a Fierfítchkin com
ares de impotância.

--- Ou seja, um duelo? Pois não --- ele respondeu mas, na verdade, eu
estava tão ridículo em meu desafio, e aquilo combinava tão pouco com
minha figura, que todos, inclusive Fierfítchkin, praticamente deitaram
de tanto rir.

--- Sim, claro, vamos largá"-lo! Afinal, já está totalmente bêbado! ---
afirmou Trudoliúbov, com asco.

--- Jamais me perdoarei por tê"-lo incluído! --- voltou a resmungar Símonov.

``E agora vou dar uma garrafada em todos'' --- pensei, peguei a garrafa
e\ldots{} enchi meu copo.

``\ldots{}Não, é melhor ficar sentado até o fim! --- continuei a pensar. --- Os
senhores ficariam felizes se eu partisse. Por nada. Vou ficar sentado de
propósito e beber até o fim, como sinal de que não lhes dou a menor
importância. Vou ficar sentado e beber, pois aqui é um botequim e eu
paguei a entrada. Vou ficar sentado e beber, pois os considero ums
fantoches, uns fantoches inexistentes. Vou ficar sentado e beber\ldots{} e
cantar, se me der vontade, sim senhores, e cantar, pois tenho esse
direito\ldots{} de cantar\ldots{} hum''.

Só que não cantei. Tentava apenas não olhar para nenhum deles; adotei a
pose mais independente, aguardando com impaciência que eles fossem os
\emph{primeiros} a falar. Mas, puxa, eles não falaram. E como desejei,
como desejei fazer as pazes com eles naquele instante. Deu nove horas,
por fim dez. Passaram da mesa ao sofá. Zverkov esparramou"-se em uma
poltrona, apoiando uma perna em uma mesinha redonda. Levaram o vinho
para lá. De fato, pediu três garrafas por sua conta. Obviamente, não fui
convidado. Todos sentaram ao seu redor, no sofá. Ouviam"-no quase com
devoção. Era visível que o amavam. ``Por quê? Por quê?'' --- pensava com
meus botões. Por vezes, chegavam ao êxtase da embriaguez e se beijavam.
Falaram do Cáucaso, da paixão verdadeira, do \emph{gálbik}\footnote{Jogo
  de cartas de azar. {[}\textsc{n.\,e.}{]}}, de postos de trabalho
vantajosos; dos rendimentos do hussardo Podkharjévski, que nenhum deles
conhecia pessoalmente, alegrando"-se que ganhasse bem; da graça e beleza
extraordinária da princesa D., que nenhum deles tampouco tinha visto;
por fim, chegou"-se à imortalidade de Shakespeare.

Sorri com desprezo e caminhei pelo outro lado do aposento, em frente ao
sofá, ao longo da parede, da mesa até a estufa e vice"-versa. Queria
demonstrar com todas as forças que podia passar bem sem eles; contudo,
apoiado nos saltos, fazia as botas baterem de propósito. Mas era tudo em
vão. \emph{Aqueles ali} não prestavam atenção. Tive paciência para ficar
assim, na frente deles, das oito às onze horas, sempre sozinho e no
mesmo lugar, da mesa à estufa e da estufa de volta à mesa. ``Ando assim,
e ninguém pode me proibir''. O garçom me olhava algumas vezes ao entrar
no reservado; as idas e vindas frequentes faziam minha cabeça rodar; por
instantes, eu tinha a impressão de delirar. Ao longo dessas três horas,
suei e fiquei seco por três vezes. De vez em quando, uma ideia se
cravava no meu coração com dor profunda e venenosa: passariam dez,
vinte, quarenta anos e mesmo assim, dali a quarenta anos, eu recordaria
com repulsa e huminhação esses que eram os minutos mais imundos,
ridículos e terríveis de toda a minha vida. Não era possível humihar a
mim mesmo de modo mais desonesto e voluntário, eu entendia isso
completamente, completamente, e mesmo assim continuava a ir da mesa para
a estufa e vice"-versa. ``Oh, se os senhores apenas soubessem de que
sentimentos e ideias sou capaz, e como sou evoluído!'' --- pensava por
instantes, dirigindo"-me mentalmente ao sofá em que estavam sentados meus
inimigos. Só que meus inimigos se comportavam como se eu não estivesse
no aposento. Uma vez, apenas uma vez se viraram para mim, justamente
quando Zverkov se pôs a falar de Shakespeare, e eu de repente soltei uma
gargalhada de desprezo. Dei uma bufada tão falsa e sórdida que eles
interromperam a conversa para observar em silêncio por dois minutos,
sérios, sem rir, como eu ia pela parede, da mesa à estufa, e como
\emph{não prestava atenção alguma neles.} Porém, não deu em nada: eles
não disseram nada e, dois minutos depois, voltarm a me largar. Deu onze
horas.

--- Senhores --- gritou Zverkov, levantando"-se do sofá ---, agora vamos todos
\emph{para lá.}

--- Claro, claro! --- disseram os outros.

Virei"-me bruscamente para Zverkov. Encontrava"-me tão extenuado, tão
alquebrado, que até me mataria para que aquilo acabasse! Estava com
febre; molhados de suor, os cabelos haviam grudado na testa e nas
têmporas.

--- Zverkov! Peço"-lhe perdão --- disse, abrupto e decidido ---, ao senhor
também, Fierfítchkin, e a todos, a todos, ofendi a todos!

--- A"-ha! Duelo não é com você! --- sibilou Fierfítchkin, venenoso.

Meu coração se dilacerava, dolorosamente.

--- Não, não tenho medo de duelo, Fierfítchkin! Estou pronto para me bater
com o senhor amanhã, mesmo depois de fazer as pazes. Até insisto nisso,
e o senhor não pode se recusar. Quero demonstrar"-lhe que não tenho medo
de duelo. O senhor será o primeiro a atirar, e eu atirarei para o ar.

--- Está se divertindo --- observou Símonov.

--- Simplesmente um delírio! --- replicou Trudoliúbov.

--- Mas nos deixe passar, o senhor está no meio do caminho!\ldots{} O que
deseja? --- respondeu Zverkov, com desprezo. Todos estavam vermelhos; seus
olhos brilhavam; tinham bebido muito.

--- Peço sua amizade, Zverkov, eu o ofendi, porém\ldots{}

--- Ofendeu? O se"-nhor? A m"-mim? Fique sabendo, prezado senhor, que
jamais, em quaisquer circunstâncias, poderá \emph{me} ofender!

--- Chega do senhor, fora! --- reforçou Trudoliúbov. --- Vamos.

--- A Olímpia é minha, senhores, é o acordo! --- gritou Zverkov.

--- Não discutimos! Não discutimos! --- responderam, rindo.

Fiquei coberto de cuspe. O bando saiu do aposento fazendo barulho,
Trudoliúbov puxou uma canção estúpida. Símonov se deteve por um
minutinho, para dar gorjeta aos garçons. De repente, aproximei"-me dele:

--- Símonov! Dê"-me seis rublos! --- disse, decidido e desesperado.

Ele me fitou com assombro extraaordinário, com os olhos embotados.
Também estava bêbado.

--- Por acaso quer ir \emph{para lá} conosco?

--- Sim!

--- Não tenho dinheiro! --- atalhou, com um riso de desprezo, e saiu do
reservado.

Peguei"-o pelo capote. Era um pesadelo.

--- Símonov! Eu vi o seu dinheiro, por que está recusando? Por acaso sou
um canalha? Cuidado ao recusar: se o senhor soubesse, se soubesse para
que estou pedindo! Disso depende tudo, todo meu futuro, todos meus
planos\ldots{}

Símonov sacou o dinheiro e praticamente o atirou em mim.

--- Pegue, já que é tão sem vergonha! --- disse, impiedoso, e correu para
alcançá"-los.

Fiquei sozinho por um instante. Desordem, restos de comida, um cálice
quebrado no chão, vinho derramado, bitucas de \emph{papirossa}\footnote{Cigarro
  com boquilha de cartão. {[}\textsc{n.\,t.}{]}}, embriaguez e delírio na cabeça,
uma angústia torturante no coração e, por fim, um lacaio, que tinha
visto e ouvido tudo e me fitava nos olhos, curioso.

--- \emph{Para lá}! --- gritei. --- Ou todos eles suplicam minha amizade de
joelhos, abraçando minhas pernas, ou\ldots{} ou vou dar uma bofetada em
Zverkov!

\section{V}

--- Ei"-lo então, ei"-lo então, finalmente, o choque com a realidade --- eu
murmurava, correndo escada abaixo a toda pressa. --- Quer dizer que isso
não é mais o papa saindo de Roma e partindo para o Brasil; quer dizer
que isso não é mais o baile no Lago de Como!

``Você é um canalha --- passou"-me pela cabeça --- se for rir disso
agora!''

--- Que seja! --- gritei, respondendo a mim mesmo. --- Agora já está tudo
perdido mesmo!

Perdera os traços deles, mas tudo bem: eu sabia para onde iam.

Na entrada estava um boleeiro solitário, noturno, de burel, todo
polvilhado da neve úmida que continuava caindo e parecia quente. O tempo
estava molhado e abafado. O cavalinho pequeno, peludo e malhado também
estava todo polvilhado e tossia; também me lembro disso. Joguei"-me no
trenó de tábuas mas, bastou levantar o pé para subir, a lembrança de
como Símonov tinha acabado de me dar os seis rublos me abateu de tal
modo que desabei como um saco.

--- Não! É preciso fazer muito para resgatar isso tudo --- berrei ---, mas
nessa noite, ou eu resgato tudo, ou morro ali mesmo. Vamos!

Partimos. Todo um turbilhão rodava na minha cabeça.

``Não vão implorar de joelhos por minha amizade. Isso é uma miragem, uma
miragem vulgar, repugnante, romântica e fantástica; assim como o baile
no Lago de Como. Por isso, \emph{tenho} que dar uma bofetada em Zverkov!
É minha obrigação. Assim, está decidido: agora vou voando lhe dar uma
bofetada''.

--- Corra!

O boleeiro puxava as rédeas.

``Assim que chegar, bato. Antes da bofetada, preciso dizer algumas
palavras, como preâmbulo? Não! Simplesmente chego e bato. Todos estarão
sentados no salão, e ele em um sofá, com Olímpia. Maldita Olímpia! Certa
vez, riu na minha cara e me rejeitou. Vou arrastar Olímpia pelos cabelos
e Zverkov pelas orelhas! Não, melhor pegar por uma orelha e assim
conduzi"-lo por todo o quarto. Pode ser que eles comecem a me bater e
empurrar. É o mais provável. Pois seja! Mesmo assim, a primeira bofetada
foi minha: a iniciativa foi minha e, pelas leis da honra, isso é tudo;
ele já estará estigmatizado, e não se limpará dessa bofetada com surra
nenhuma, apenas com um duelo. Terá que se bater. Bem, então eles que me
espanquem agora. Pois seja, seus ingratos! Trudoliúbov é quem vai bater
mais, é tão forte; Fierfítchkin provavelmente vai se atracar de lado e,
sem dúvida, agarrar os cabelos. Mas que seja, que seja! Vim para isso.
Seus crânios de carneiro finalmente serão obrigadas a decifrar o trágico
disso tudo! Quando estiverem me arrastando até a porta, gritarei que, na
verdade, eles não valem o meu mindinho!.

--- Corra, cocheiro, corra! --- eu gritava para o boleeiro.

Ele chegou a tremular e agitar o cnute. Pois meu grito era muito feroz.

``Vamos nos bater à alvorada, já está decidido. O departamento acabou.
Em vez de departamento, Fierfítchkin acaba de dizer \emph{lepartamento.}
Mas onde conseguir as pistolas? Bobagem! Pego um adiantamento e compro.
E a pólvora, e as balas? Isso é com o padrinho. E como conseguir isso
tudo até a alvorada? E onde arrumar um padrinho? Não tenho
conhecidos\ldots{}''

--- Bobagem! --- gritei, chacoalhando ainda mais. --- Bobagem!

``A primeira pessoa que eu encontrar e abordar na rua estará obrigada a
ser meu padrinho, como se fosse tirar um afogado da água. As suposições
mais excêntricas devem ser consideradas. Se amanhã eu convidar o próprio
diretor para ser meu padrinho, ele tem que concordar e guardar segredo
por mero sentimento cavalheiresco! Anton Antônytch\ldots{}''

A questão é que nesse mesmo instante fez"-se para mim mais claro e mais
vivo do que tudo que existia no mundo todo o absurdo ignóbil de minhas
hipóteses e o reverso da medalha, porém\ldots{}

--- Corra, cocheiro, corra, tratante, corra!

--- Ei, patrão! --- disse a força da terra.

De repente, um frio se apoderou de mim.

``Mas não seria melhor\ldots{} mas não seria melhor\ldots{} ir agora direto para
casa? Oh, meu Deus! Por que, por que fui me convidar ontem para esse
jantar? Mas não, não é possível! E o passeio de três horas da mesa à
estufa? Não, eles, eles e mais ninguém devem me pagar por esse passeio!
Eles têm que pagar pela desonra!

--- Corra!

``E se me mandarem para a delegacia? Não ousarão! Têm medo do escândalo.
E se Zverkov se recusar a duelar por desprezo? Isso é bem provável: mas
daí eu lhe mostro\ldots{} Vou me precipitar à estação de posta, amanhã,
quando ele estiver de partida, agarrá"-lo pela perna, arrancar"-lhe o
capote quando estiver prestes a subir no veículo. Vou fincar os dentes
em sua mão, vou mordê"-lo. ``Vejam todos a que ponto pode chegar um homem
desesperado!'' Que me bata na cabeça, como todos atrás dele. Gritarei a
todo o público: ``Vejam o jovem fedelho que vai seduzir circassianas com
o meu cuspe na cara!''

É óbvio que, depois disso, tudo estará acabado! O departamento
desaparecerá da face da terra. Serei detido, julgado, expulso do
emprego, mandado para a cadeia, deportado para a Sibéria. Não tem
problema! Quinze anos mais tarde, vou me arrastar atrás dele,
esfarrapado, um mendigo, quando me soltarem da prisão. Vou encontrá"-lo
em alguma cidade de província. Estará casado e feliz. Terá uma filha
adulta\ldots{} Direi: ``veja, seu verdugo, veja minhas faces cavadas e meus
farrapos! Perdi tudo: a carreira, a felicidade, a arte, a ciência,
\emph{a mulher amada,} e tudo por sua causa. Aqui estão as pistolas. Vim
para descarregar minha pistola, e\ldots{} e o perdôo''. Daí darei um tiro
para o ar, e depois nem sinal de mim\ldots{}''

Cheguei até a chorar, embora soubesse com total exatidão, naquele mesmo
instante, que tudo aquilo vinha de Sílvio e da \emph{Mascarada,} de
Liérmontov\footnote{Sílvio, protagonista da novela \emph{O Tiro,} de
  Púchkin; \emph{Mascarada,} drama de Liérmontov. {[}\textsc{n.\,t.}{]}}. De
repente, passei uma vergonha terrível, a ponto de parar o cavalo, descer
do trenó e ficar parado na neve, no meio da rua. O cocheiro me fitava,
perplexo e suspirando.

O que fazer? Não dava para ir para lá, era um absurdo; e não dava para
deixar o caso porque, se assim fosse\ldots{} Senhores! Como podia deixar?
Depois de tamanhas ofensas!

= Não! --- exclamei, voltando a pular no trenó. --- Está predestinado, é o
destino! Corra, corra para lá!

Impaciente, dei com o punho no pescoço do boleeiro.

--- Mas o que é isso, para que brigar? --- gritou o pequeno mujique,
fustigando, entretanto, o rocim, de modo que este até começou a
escoicear com as patas traseiras.

A neve úmida caía aos flocos; tirei o casaco, não estava para aquilo.
Esquecera"-me de todo o resto, pois me resolvera definitivamente pela
bofetada e sentia, com horror, que seria \emph{infalivelmente agora},
aconteceria naquele instante, e que \emph{já não havia forças que
pudessem impedir.} Lampiões desertos cintilavam lúgubres na bruma
nevada, como tochas em um enterro. A neve se acumulara debaixo do meu
capote, da sobrecasaca e da gravata, e ali derreteu; não me cobri, pois
tudo já estava mesmo perdido! Por fim, chegamos. Apeei quase
inconsciente, saí correndo pelos degraus e me pus a bater na porta com
as mãos e com os pés. Minhas pernas estavam especialmente enfraquecidas
na altura dos joelhos. Abriram rápido; era como se soubessem de minha
chegada. (De fato, Símonov preverira"-os de que, talvez, viesse mais um,
pois lá era necessário prevenir e, em geral, precaver"-se. Tratava"-se de
uma daquelas ``lojas de moda'' que a polícia exterminou há tempos. De
dia era mesmo uma loja; à noite, porém, com uma recomendação, era
possível visitar). Com passos rápidos, passei pela loja escura, chegando
ao salão conhecido, onde ardia uma vela, e fiquei perplexo: não havia
ninguém.

--- Mas onde estão eles? --- perguntei a alguém.

Mas eles, obviamente, já tinham conseguido se dispersar\ldots{}

Diante de mim havia uma pessoa de sorriso estúpido, a proprietária, que
eu conhecia em parte. Um minuto mais tarde, a porta se abriu, e entrou
outra pessoa.

Sem prestar atenção em nada, caminhei pelo aposento falando sozinho, ao
que parece. Era como se tivesse sido salvo da morte e, com todo meu ser,
pressentia"-o com alegria; pois eu teria dado a bofetada, infalivelmente
teria dado a bofetada! Mas agora eles não estavam e\ldots{} tudo
desaparecera, tudo mudara!.. Olhei ao redor. Ainda não podia
compreender. Lancei um olhar maquinal à garota que entrara: na minha
frente, reluzia um rosto fresco, jovem, algo pálido, de sobrancelhas
retas e escuras e um olhar sério e algo surpreso. Isso me agradou de
cara; eu a teria odiado se ela sorrisse. Passei a olhar com mais
atenção, como que me esforçando: ainda não reunira as ideias. Havia algo
de cândido e bom naquele rosto, porém uma estranheza séria. Tenho
certeza de que isso a prejudicava ali, e de que, por causa disso, nenhum
daqueles tolos jamais reparara nela. A propósito, não podia ser chamada
de bela, embora fosse de alta estatura, forte e bem constituída. Sua
roupa era de uma simplicidade extraordinária. Algo de abjeto me mordeu;
fui até ela\ldots{}

Lancei um olhar casual ao espelho. Meu rosto perturbado pareceu"-me
extremamente repulsivo: pálido, mau, vulgar, de cabelos desgrenhados.
``Pois seja, fico feliz --- pensei --- exatamente por me mostrar a ela
repulsivo; isso me agrada\ldots{}''

\section{VI}

\ldots{}Em algum lutar atrás do tabique, como que submetido a forte pressão,
como se alguém o sufocasse, o relógio roncou. Ao ronco artificialmente
prolongaado seguiu"-se uma batida de relógio fininha, ruinzinha e algo
inesperada, como se alguém tivesse dado um pulo repentino para a frente.
Deram duas horas. Acordei, embora não estivesse dormindo, apenas
deitado, semiconsciente.

O quarto estreito, apertado e baixo, atravancado por um enorme
guarda"-roupa e caixas de papelão, trapos e todo tipo de traste de
vestuário jogado estava escuro, quase por completo. O toco de vela em
cima da mesa, no fim do quarto, apagava"-se por inteiro, lançando luzes
tênues e intermitentes. Em alguns minutos, as trevas seriam absolutas.

Voltei a mim rápido; as lembranças vieram todas imediatamente, de uma
vez, sem esforço, como se estivessem me vigiando para cair em cima de
mim. Mesmo em minha letargia, ficou na memória o tempo todo algo como um
ponto, que não se esquecia de nada, em torno do qual meus sonhos
caminhavam pesadamente. Mas era estranho: tudo que me ocorrera naquele
dia agora me parecia, ao despertar, um passado longinquo, como se o
tivesse vivido longinquamente.

Tinha um torpor na cabeça. Algo parecia correr acima de mim, tocando"-me,
excitando"-me e perturbando"-me. A angústia e a bílis voltavam a se
acumular e buscar saída. De repente, ao meu lado, avistei dois olhos
abertos, que me examinavam com curiosidade e obstinação. O olhar era de
uma indiferença fria, lúgubre, como algo completamente alheio; dava uma
impressão penosa.

Um pensamento lúgubre nasceu no meu cérebro, percorrendo todo meu corpo
com uma sensação desagradável, parecida com a da entrada em um subsolo
úmido e bolorento. Soava algo artificial que justamente só agora esses
dois olhos tivessem inventado de me examinar. Lembro"-me ainda de que, ao
longo de duas horas, não dissera uma palavra àquela criatura, não o
considerando absolutamente necessário; há pouco, isso até me agradara,
por algum motivo. Só que agora, de repente, surgia"-me com força a ideia
disparatada, repugnante como uma aranha, da perversão, que, sem amor,
rude e desavergonhada, começa justo onde o verdadeiro amor é coroado.
Ficamos nos encarando desse jeito por muito tempo, mas ela não baixava
os olhos diante dos meus, nem alterava o olhar, de modo que, por fim,
fiquei mal.

--- Qual o seu nome? --- perguntei, com voz entrecortada, para acabar
rápido.

--- Liza --- ela respondeu, quase cochichando, mas de cara feia, e
afastando o olhar.

Fiquei em silêncio.

--- Hoje o tempo\ldots{} neve\ldots{} ruim --- disse, quase para mim, colocando a
mão atrás da cabeça, com angústia, e olhando para o teto.

Ela não respondeu. Tudo aquilo era indecente.

--- Você é daqui? --- perguntei, um minuto depois, quase estourando de
raiva, virando de leve a cabeça para ela.

--- Não.

--- De onde?

--- De Riga --- ela disse, de má vontade.

--- Alemã?

--- Russa.

--- Está aqui faz tempo?

--- Aonde?

--- Na casa.

--- Duas semanas. --- Ia falando de forma cada vez mais entrecortada. A vela
apagou de vez; eu não conseguia mais distinguir seu rosto.

--- Tem pai e mãe?

--- Sim\ldots{} não\ldots{} tenho.

--- Onde estão?

--- Lá\ldots{} em Riga.

--- Quem são?

--- Assim\ldots{}

--- Como assim? Quem são, de que classe?

--- Pequenos burgueses.

--- Você sempre morou com eles?

--- Sim.

--- Quantos anos tem?

--- Vinte.

--- Por que fugiu deles?

--- Assim\ldots{}

Esse \emph{assim} queria dizer: não me encha, está me enjoando.
Calamo"-nos.

Sabe Deus por que não saí. Estava ficando cada vez mais enjoado e
angustiado. Imagens de tudo que ocorrera naquele dia, sem que eu
quisesse, passaram a acorrer à minha memória por si só, de modo
desordenado. De repente me lembrei de uma cena que tinha visto na rua,
de manhã, trotando apressado para o serviço.

--- Hoje estavam carregando um caixão e quase deixaram cair --- disse, de
repente, em voz alta, sem desejar começar uma conversa de jeito nenhum,
mas assim, quase sem querer.

--- Um caixão?

--- Sim, na rua Siênnaia; estavam tirando de um porão.

--- De um porão?

--- Não de um porão, mas de um andar inferior\ldots{} sabe\ldots{} lá de baixo\ldots{} de
uma casa de má fama\ldots{} Tinha tanta sujeira em volta\ldots{} Cascas, lixos\ldots{}
fedia\ldots{} era detestável.

Silêncio.

--- É ruim enterrar hoje! --- recomecei, apenas para não ficar calado.

--- Por que ruim?

--- A neve, a umidade\ldots{} (Bocejei).

--- Dá na mesma --- ela disse, de repente, depois de algum silêncio.

--- Não, é sórdido\ldots{} (Voltei a bocejar). Os coveiros com certeza estavam
xingando porque a neve encharcava. E na sepultura com certeza tinha
água.

--- Por que tinha água na sepultura? --- perguntou, com alguma curiosidade,
porém com a fala mais rude e entrecortada do que antes. Algo começou a
me incitar de repente.

--- Como assim, no fundo, de água, tinha uns seis \emph{vierchóks.} Não dá
para abrir uma cova seca lá em Vólkovo.

--- Por quê?

--- Como por quê? O lugar é cheio de água. Aqui tem pântano por todo lado.
Daí colocam na água. Eu mesmo vi\ldots{} muitas vezes\ldots{}

(Não tinha visto nenhuma vez, nem jamais estivera em Vólkovo, só ouvira
dizer).

--- Por acaso para você dá tudo na mesma, morrer?

--- Mas por que eu vou morrer? --- ela respondeu, como se defendendo.

--- Algum dia você mai morrer, e vai morrer como a defunta de hoje\ldots{} Ela
era\ldots{} também uma moça sozinha\ldots{} Morreu de tísica.

--- A garota devia ter morrido no hospital\ldots{} (Ela já sabia do fato,
pensei; disse garota, e não moça).

--- Ela devia à proprietária --- repliquei, cada vez mais incitado pela
discussão ---, e trabalhou quase até o fim, embora estivesse tísica. Os
cocheiros das redondezas falaram com os soldados e contaram isso. Com
certeza eram seus conhecidos. Estavam rindo. Reuniram"-se no botequim em
sua memória. (Aqui eu também contava muitas lorotas).

Silêncio, profundo silêncio. Ela nem se mexia.

--- Mas não seria melhor morrer no hospital?

--- E não dá na mesma?\ldots{} Por que eu vou morrer? --- ela acrescentou,
zangada.

--- Agora não, mas e depois?

--- Mesmo depois\ldots{}

--- Nada disso! Agora você é jovem, bela, viçosa, tem valor por causa
disso. Só que, depois de um ano nessa vida, não será mais assim, vai
definhar.

--- Daqui a um ano?

--- Em todo caso, daqui a um ano vai ter menos valor --- prossegui, com
perversa alegria. --- Daqui você vai passar para um lugar inferior, para
outra casa. Daqui a outro ano, para uma terceira casa, cada vez mais
baixa e, em sete anos, também estará na Siênnaia, no porão. Isso ainda
vai ser bom. A desgraça será se, além disso, aparecer \emph{alguma}
doença, bem, uma fraqueza no peito\ldots{} ou se ficar resfriada, ou outra
coisa. Nessa vida, é difícil a doença passar. Se pegar, é capaz de não
largar mais. E daí você morre.

--- Pois bem, eu morro --- ela respondeu, já totalmente irada, e
remexendo"-se rapidamente.

--- Mas dá pena.

--- De quem?

--- Da vida.

Silêncio.

--- Você tinha noivo? Hein?

--- Por que quer saber?

--- Ah, não a estou interrogando. Não quero saber. Por que ficou brava?
Claro que você deve ter suas contrariedades. Que me importa? Mas dá
pena.

--- De quem?

--- De você.

--- Não é nada\ldots{} --- ela sussurrou, quase inaudível, e voltou a se remexer.

Isso me agastou na hora. Como? Eu tinha sido tão dócil, e ela\ldots{}

--- Mas o que você acha? Está num caminho bom, né?

--- Não acho nada.

--- Pior se não acha. Acorde, ainda há tempo. Há tempo, sim. Você ainda é
jovem, é bonita; pode se apaixonar, se casar, ser feliz\ldots{}

--- Nem todas as casadas são felizes --- atalhou, com a fala rude de antes.

--- Claro que nem todas, mas mesmo assim é muito melhor do que aqui.
Incrivelmente melhor. Dá para viver com amor e sem felicidade. Mesmo na
desgraça a vida é boa, é bom viver, não importa como. Mas aqui o que há
além de\ldots{} fedor? Argh!

Virei"-me com repulsa; não estava mais argumentando com frieza. Passei a
sentir que me inflamava ao falar. Já ansiava por expor as
\emph{ideiazinhas} íntimas, que tinha incubado em um canto. De repente,
algo me incendiou, ``aparecera'' um objetivo.

--- Não repare que estou aqui, não sou exemplo para você. Talvez eu seja
até pior do que você. Aliás, vim para cá bêbado --- apressei"-me, mesmo
assim, a me justificar. --- Além disso, o homem jamais é exemplo par a
mulher. São casos distintos: ainda que eu me emporcalhe e me suje, não
sou escravo de ninguém; cheguei, fui embora e não estou mais. Sacudo"-me
e já era. Já você é uma escrava desde o início. Sim, escrava! Você
entrega tudo, toda sua vontade. Depois, vai querer romper essas
correntes, e não terá como: estarão apertadas com cada vez mais força.
Malditas correntes. Eu as conheço. Nem vou falar do resto, você
possivelmente não entenderia, mas me diga: você por acaso está devendo
para a patroa? Ah, mas veja! --- acrescentei, embora ela não respondesse,
e ficasse apenas me escutando calada, com todo seu ser; --- Veja a sua
corrente! Nunca vai comprar sua liberdade. Eles fazem assim. É como
vender a alma ao diabo\ldots{} \ldots{}Enquanto isso, eu\ldots{} pode ser que
também seja tão infeliz, quer saber, e me enfie na lama de propósito,
também de angustia. Afinal, há quem beba de desgosto: bem, eu estou aqui
de desgosto. Mas diga o que tem de bom aqui: eu fui\ldots{} com você\ldots{} há
pouco, e não proferimos palavra, depois você ficou me examinando como
uma selvagem; e eu fiz o mesmo. Assim é que se ama? Assim é que uma
pessoa deve se relacionar com outra? Uma pouca vergonha, isso é que é!

--- Sim! --- ela ecoou, brusca e apressada. Fiquei até espantado com a
pressa desse \emph{sim.} Poderia significar que essa mesma ideia
percorrera sua cabeça ao me examinar, há pouco? Significaria qe ela
também era capaz de ideias?.. ``Que diabo, é curioso, isso é
\emph{afinidade --} pensei, quase esfregando as mãos. --- E como não
dominar uma alma jovem dessas?..''

O que mais me cativava era o jogo.

Virou a cabeça para mais perto de mim e, no escuro, tive a impressão de
que a apoiou na mão. Pode ser que me examinasse. Como lamentei não poder
divisar seus olhos. Ouvia seu respirar profundo.

--- Por que você veio para cá? --- comecei, já com alguma autoridade.

--- Assim\ldots{}

--- Afinal, como é bom viver na casa dos pais! É quente, é livre; o
próprio ninho.

--- E se for pior que isso?

``Preciso encontrar o tom --- ocorreu"-me ---, o sentimentalismo talvez não
dê muito resultado''.

Aliás, isso só me passou pela cabeça momentaneamente. Juro que ela me
interessava de verdade. Ademais, eu estava relaxado e bem disposto. E a
trapaça convive facilmente com o sentimento.

--- Como não? --- apressei"-me a responder. --- Acontece de tudo. Afinal, estou
seguro de que alguém a ofendeu, e de que é ele o culpado, e não
\emph{você.} Embora não saiba nada da sua história, uma moça como você
com certeza não veio parar aqui por vontade própria\ldots{}

--- Que tipo de moça eu sou? --- sussurrou, quase inaudível, mas eu ouvi.

``Que diabo, estou bajulando. Isso é torpe. Mas talvez também seja
bom\ldots{}'' Ela se calava.

--- Veja, Liza, vou falar de mim! Se eu tivesse tido uma família desde a
infância, não seria como sou hoje. Penso nisso com frequência. Pois, por
pior que seja a família, sempre são o pai e a mãe, não inimigos, não
gente de fora. Pelo menos uma vez por ano vão lhe demonstrar amor.
Apesar de tudo você sabe que está em casa. Só que eu cresci sem família;
certamente foi por isso que saí tão\ldots{} insensível.

Voltei a esperar.

``Talvez não entenda --- pensei ---, e é mesmo ridículo: a moral''.

--- Se eu fosse pai, e tivesse uma filha, acho que a amaria mais do que os
filhos, de verdade --- comecei, ladeando, como se falasse de outra coisa,
para distraí"-la. Confesso que enrubesci.

--- Por que isso? --- perguntou.

Ah, então está escutando!

--- Assim; não sei, Liza. Veja: conheço um pai que era um homem rígido e
severo, mas ficava de joelhos diante da filha, beijava as mãos e os pés,
não se cansava de admirá"-la; verdade. Quando a filha dançava, nas
noitadas, ele ficava cinco horas no mesmo lugar, sem tirar os olhos
dela. Era louco por ela; eu entendo. Ela ficava cansada e adormecia, à
noite, e ele acordava para ir beijá"-la no sono e fazer o sinal da cruz.
Andava com um casaquinho ensebado, era avarento com todos mas, para ela,
comprava sempre o mais novo, dava presentes ricos, e que alegria quando
o presente agradava. O pai sempre deve amar a filha mais do que a mãe.
Algumas moças são felizes em casa! Acho que nem deixaria minha filha se
casar.

--- Mas como assim? --- perguntou, sorrindo de leve.

--- Teria ciúmes, meu Deus. Bem, como ela iria se por a beijar um outro?
Amar um outro mais do que o pai? É duro até de imaginar. Claro que tudo
isso é bobagem; claro que, no fim, todo mundo cria juízo. Acho que eu,
antes de casá"-la, iria me atormentar de preocupação: rejeitaria todos os
noivos. Mas, assim mesmo, terminaria por casá"-la com o que ela amasse.
Pois o pai sempre acha o que a filha ama o pior de todos. É assim. Por
causa disso, muita coisa ruim acontece nas famílias.

--- Outros são mais felizes vendendo a filha do que casando"-a com honra ---
ela disse, de repente.

Ah! Então é isso!

--- Isso, Liza, é nas malditas famílias em que não há Deus nem amor ---
secundei, com ardor --- e, onde não há amor, não há também juízo. Verdade
que há famílias assim, mas não é delas que estou falando. Pelo visto,
você não encontrou bondade na sua família, para falar assim. Você é
verdadeiramente infeliz. Hum\ldots{} Tudo isso vem mais da pobreza.

--- Por acaso entre os nobres é melhor? Mesmo na pobreza, gente honrada
vive bem.

--- Hum\ldots{} sim. Pode ser. Também tem o seguinte, Liza: a pessoa só gosta
de contabilizar sua desgraça, mas não contabiliza a felicidade. Se
contabilizasse direito, veria então que sempre tem o seu quinhão
assegurado. Bem, mas se tudo dá certo na família, Deus abençoa, o marido
se revela bom, ama você, mima você, não sai de perto de você! Essa
família é boa! Mesmo com alguma desgraça ainda é algo bom; pois onde não
há desgraça? Talvez se case, daí \emph{saberá por si mesma.} Tomemos
pelo menos os primeiros tempos de casada com aquele que você ama: é a
felicidade, quanta felicidade se tem às vezes! A três por dois. Nesses
primeiros tempos, até as discussões com o marido vão acabar bem. Tinha
uma que, quanto mais amava, mais discussões com o marido arranjava. É
verdade, eu a conheci: ``Pois bem, eu amo muito, e o atormento por amor,
para que você sinta''. Você sabia que é possível atormentar uma pessoa
de propósito, por amor? Especialmente as mulheres. E pensam consigo
mesmas: ``Em compensação, depois vou amá"-lo e acariciá"-lo tanto que não
é pecado atormentá"-lo agora''. Em casa todos se alegram com você, que
está bem, alegre, calma, honrada\ldots{} Também há outras que são ciumentas.
Ele sai, e conheço uma que não aguenta e sai correndo naquela mesma
noite, devagarinho, para espiar: não estaria ele ali, naquela casa, com
aquela? Isso já é ruim. Ela sabe que é ruim, seu coração para e se
mortifica, mas ama; tudo isso é por amor. E como é bom fazer as pazes
depois da briga, reconhecer a própria culpa ou perdoar! E como é bom
para os dois, como fica bom de repente, como se tivessem voltado a se
conhecer, a se casar, como se o amor começasse de novo. E ninguém,
ninguém deve saber o que se passa enre marido e mulher quando eles se
amam. E, seja qual for a briga, não devem chamar nem a própria mãe para
julgar, nem contar nada um do outro. Julgue por si mesma. O amor é um
segredo divino, que deve ser escondido dos olhos de todos os outros,
haja o que houver. Isso o faz mais santo, melhor. Um respeita mais o
outro, e o respeito é a base de muita coisa. E uma vez que houve amor,
que se casaram por amor, por que haveria o amor de passar? Por que não
seria possível conservá"-lo? É raro o caso em que não é possível
conservá"-lo. Bem, se o marido lograr ser bom e honesto, como então o
amor vai passar? Verdade que o primeiro amor conjugal passa, mas daí vem
um amor ainda melhor. Daí as almas se encontram, todos os assuntos são
resolvidos em conjunto; um não tem segredo para o outro. Virão os
filhos, e então todos os tempos, mesmo os mais difíceis, parecerão
felizes: só é preciso amar e ter coragem. Daí até o trabalho será
alegre, daí terá de recusar o pão algumas vezes em prol dos filhos, e
isso também será alegre. Afinal, depois eles vão amá"-la por isso: você
estará acumulando para si mesma. Os filhos crescem, você sente que é um
exemplo, um apoio para eles; que, quando morrer, eles levarão consigo,
por toda vida, seus sentimentos e ideias, como receberam de você, à sua
imagem e semelhança. Quer dizer que se trata de um grande dever. Assim,
como pai e mãe não ficariam mais próximos? Dizem que é duro ter filhos.
Quem diz isso? É uma felicidade celestial! Você gosta de crianças
pequenas, Liza? Eu amo terrivelmente. Sabe, um menino rosado, sugando
seu seio: o coração de qualquer marido se volta para esposa quando ele a
vê sentada com seu bebê! Um bebezinho rosadinho, rechonchudinho, abre os
braços, espreguiçando"-se; pezinhos e mãozinhas suculentas, unhinhas
limpinhas, pequeninas, tão pequeninas que parecem ridículas, os olhinhos
como se já entendesse tudo. E suga, mexe no seu seio com a mãozinha,
brinca. O pai chega, ele se solta do peito, se inclina todo para trás,
olha para o pai, ri --- como se Deus soubesse o que tem de tão engraçado
--- e de novo, de novo se põe a sugar. Daí pega e dá uma mordida no seio
da mãe, se os dentes já saíram, fitando"-a de esguelha com seus olhinhos:
``Está vendo, mordi!'' Será que então não é tudo felicidade quando estão
os três, marido mulher e bebê, juntos? É possível perdoar muita coisa
por esses instantes. Não, Liza, primeiro é preciso aprender a viver por
si só, para depois acusar os outros!

``Quadrinhos, esses são os quadrinhos de que você precisa! --- pensei
comigo mesmo, embora, por Deus, tenha falado com sentimento, e enrubesci
de repente. --- E se ela de repente cair na gargalhada, onde é que vou me
enfiar então?'' --- Essa ideia me deixava furioso. No fim do discurso,
fiquei realmente inflamado, e agora o amor"-próprio me fazia sofrer. O
silêncio se prolongou. Tive até vontade de empurrá"-la.

--- O que o senhor\ldots{} --- começou, de repente, e parou.

Mas eu já tinha entendido tudo: havia um outro tremor em sua voz, nem
brusco, nem rude, nem teimoso como há pouco, mas algo suave e acanhado,
tão acanhado que, de repente, me senti envergolhando e culpado perante
ela.

--- O quê? --- perguntei, com terna curiosidade.

--- Mas o senhor\ldots{}

--- O quê?

--- Algo no senhor\ldots{} é como um livro --- disse, e algo de zombeteiro
voltou a se ouvir em sua voz.

Essa observação foi um beliscão dolorido. Eu não esperava.

Não compreendi que ela se mascarava na zombaria de propósito, que esse é
o último subterfúgio habitual das pessoas acanhadas e de coração casto,
cuja alma tentam penetrar de modo rude e inoportuno, e que, até o último
instante, não se rendem por orgulho, temendo exprimir seus sentimentos.
Já pela timidez com que se aproximara, em algumas tentativas, de sua
zombaria, antes de por fim se decidir a exprimi"-la, eu teveria ter
adivinhado. Mas não adivinhei, e um sentimento perverso se apoderou de
mim.

``Espere aí'', pensei.

\section{VII}

--- Ei, basta, Liza, que livro é esse quando me sinto sórdido, como alguém
de fora? Só que não sou de fora. Tudo isso me despertou na alma\ldots{} Será
possível, será possível que você não se sinta sórdida aqui? Não, pelo
visto o hábito é muito significativo! Sabe o diabo o que o hábito pode
fazer com uma pessoa. Mas será que você pensa a sério que nunca vai
envelhecer, será eternamente bela e vão mantê"-la aqui eternamente? Nem
vou falar da obscenidade daqui\ldots{} Aliás, quero lhe falar da sua vida
atual: embora agora você seja jovem, linda, boa, com alma, com
sentimento; bem, saiba que eu, assim que acordei, há pouco,
imediatamente me senti sórdido por estar aqui com você! Só foi possível
vir parar aqui bêbado. Porém, se você vivesse em outro lugar, como vivem
as pessoas boas, pode ser que não apenas a cortejasse, como simplesmente
me enamorasse de você, ficasse contente com um olhar ou uma palavra sua;
espreitaria você no portão, ficaria de joelhos na sua frente; eu a
olharia como noiva, considerando isso uma honra. Não ousaria pensar algo
impuro a seu respeito. Aqui, porém, eu sei que basta assobiar e você,
queira ou não queira, virá atrás de mim, e não sou eu que devo perguntar
a sua vontade, mas você a minha. O último dos mujiques, ao fazer um
contrato de trabalho, nem assim se escraviza por inteiro, e sabe que
aquilo tem um prazo. Mas cadê o seu prazo? Pense apenas: o que você está
entregando? O que escraviza? A alma, a alma, sobre a qual não tem poder,
está sendo escravizada junto com o corpo! Você expõe seu amor à
profanação de qualquer bêbado! O amor! Só que isso é tudo, só que isso é
um diamante, um tesouro virginal, o amor! Para serem dignos desse amor,
alguns estão prestes a entregar a alma, a acorrer à morte. Mas o que
vale o seu amor aqui? Você é comprada por inteiro, e para que vão
solicitar o amor, quando por aqui tudo é possível sem amor? Não pode
haver ofensa maior para uma moça, compreende? Bem, ouvi dizer que dão
alegria a vocês, bobas, deixando que tenham amantes aqui. Pois isso é
uma complacência, um engano, um riso na cara de vocês, e vocês
acreditam. O quê, esse amante a ama de fato? Não creio. Como vai amar,
sabendo que nessa hora mesmo podem chamá"-la para longe dele? Depois
disso, ele é um nojento! Por acaso respeita uma gota sua que seja? O que
ele tem em comum com você? Ele ri de você e ainda a rouba; eis todo o
seu amor! Será bom se não bater. Mas pode também bater. Pergunte"-lhe, se
você tiver um desses: ele vai se casar com você? Vai rir na sua cara, se
não cuspir ou espancar, e ele mesmo pode não valer dois tostões furados.
Pense; e para que você arruinou sua vida aqui? Pelo café que lhe dão e
pelas refeições fartas? E para que a alimentam? Uma outra pessoa,
honrada, não deixaria esse quinhão passar pela garganta ao saber para
que é alimentada. Aqui você é devedora, vai ser sempre devedora, até o
fim, até quando os fregueses começarem a ter aversão de você. E isso há
de ser logo, não tenha esperança na juventude. Aqui tudo é rápido como o
correio. Você será expulsa. E não vão simplesmente expulsar, como bem
antes começarão a criticar, começarão a xingar, como se você não tivesse
entregue sua saúde e juventude à patroa, como se não tivesse destruído a
alma por ela, e sim a arruinado, reduzido à miséria, roubado. E não
espere apoio: as outras amigas também vão lhe cair em cima, por
servilismo, pois aqui estão todas escravizadas, e perderam a consciência
e a piedade há muito tempo. Acanalharam, e não há xingamentos mais
sórdidos, vulgares e ultrajantes do que os delas. E você vai deixar tudo
aqui, tudo, sem ressalvas, a saúde, a juventude, a beleza, as esperanças
e, aos vinte e dois anos, terá a aparência de trinta, e será bom se não
adoecer, ore a Deus por isso. Afinal, certamente você acha que agora não
está trabalhando, está no ócio! Só que não há nem jamais houve trabalho
mais pesado e forçado no mundo. Parece que o coração deveria se debulhar
em lágrimas. E você não vai ousar dizer palavra, nem meia palavra,
quando for expulsa daqui, irá embora como uma culpada. Vai se mudar para
outro lugar, depois para um terceiro, depois ainda para um outro e, por
fim, chegará à Siênnaia. Lá também vão começar a espancá"-la; é a
amabilidade local; o cliente de lá não sabe acariciar se não bater. Não
acredita que lá seja tão repugnante? Vá, dê uma olhada, poderá observar
com seus próprios olhos. Avistei uma delas no Ano Novo, junto à porta.
Fora expulsa por zombaria, para esfriar um pouco, pois andava
choramingando muito, e fecharam a porta para ela. Às nove da manhã já
estava totalmente bêbada, desgrenhada, seminua, toda batida. Toda
empoada, com manchas negras nos olhos; escorria sangue do nariz e dos
dentes: algum cocheiro tinha acabado de dar um jeito nela. Estava
sentada em uma escadinha de pedra, com um peixe salgado na mão;
choramingava, pranteando algo sobre sua ``disgraça'', martelando os
degraus da escada com o peixe. Na entrada, apinharam"-se ums cocheiros e
soldados bêbados, mexendo com ela. Você não acredita que também vai
ficar assim? Eu também não gostaria de acreditar, mas quem sabe, pode
ser que dez, oito anos atrás, essa mesma do peixe salgado tenha chegado
de algum lugar fresca, inocente como um querubinzinho, purinha; não
conhecia o mal, corava a cada palavra. Talvez fosse como você,
orgulhosa, suscetível, diferente de todas, com um olhar de rainha e
sabedora de que a felicidade completa aguardava aquele que a amasse e
que ela amasse. Viu como acabou? E se, naquele mesmo instante em que
martelava com o peixe os degraus sujos, bêbada e desgrenhada, ela se
recordasse de todo seu passado, dos anos de pureza na casa paterna,
quando ainda ia à escola, e o filho do vizinho a espreitava no caminho,
assegurando que a amaria por toda a vida, que lhe confiaria seu destino,
quando eles afirmaram amar um ao outro para sempre e se casar quando
estivessem mais crescidos? Não, Liza, você seria feliz, feliz se
morresse logo, de tísica, em algum lugar, em um canto, no porão, como a
de hoje. No hospital, diz você? Muito bem, vão levar, mas e se ainda
dever à patroa? A tísica é uma doença assim; não é uma febre. Até os
últimos minutos a pessoa tem esperança, e diz que está bem. Ela se
consola. E isso é proveitoso para a patroa. Não se preocupe, é assim;
significa que vendeu a alma e, ainda por cima, deve dinheiro, significa
que não se atreve a dar nem um fio. Vai morrer, e todos vão largá"-la,
vão lhe dar as costas, pois o que ainda há para tirar de você? Ainda vão
recriminá"-la por ocupar um lugar à toa, por não morrer logo. Você vai
rogar água, que lhe darão com insultos: ``Quando é que vai bater as
botas, sua canalha? Fica atrapalhando o sono, gemendo, os clientes ficam
com nojo''. É verdade; eu mesmo ouvi essas palavras. Moribunda, será
enfiada no canto mais fedorento do porão, escuro, úmido; deitada,
sozinha, o que você vai repensar, então? Morta, será preparada rápido,
por mãos alheias, entre resmungos, com impaciência, sem ninguém para
abençoá"-la, sem ninguém para suspirar por você, apenas vão querer
tirá"-la dos ombros o quanto antes. Comprarão um ataúde, vão carregá"-la
como carregaram a pobre de hoje, irão ao \emph{botequim} em sua memória.
No túmulo, lama, borras, neve úmida; vão fazer cerimônia por sua causa?
``Desça"-a, Vaniukha; que ``disgraça'', até aqui está de pernas para
cima, essa aí. Encurte as cordas, seu velhaco''. --- ``Está bem assim''. ---
``Como bem? Está deitada de lado. Também era uma pessoa, ou não? Certo,
está bem, cubra''. Nem ficar muito tempo se xingando por sua causa vão
querer. Vão cobrir o quanto antes com barro úmido e azul e irão para o
botequim\ldots{} Daí será o fim de sua memoria na terra; aos túmulos de
outros filhos irão pais, maridos, mas o seu não terá nem lágrima, nem
suspiro, nem lembrança, e ninguém, ninguém, jamais, no mundo inteiro,
irá visitá"-la; seu nome desaparecerá da face da terra assim, como se
você nunca tivesse existido nem nascido! A lama, o pântano, ainda que, à
noite, quando os mortos se levantam, você bata na tampa do caixão:
``Gente boa, deixe"-me viver no mundo! Vivi, mas não vi a vida, minha
vida passou em um instante; foi bebida em um botequim na Siênnaia; gente
boa, deixe"-me viver mais uma vez!..''

Cheguei a tamanha exaltação que um espasmo se armou em minha garganta,
e\ldots{} de repente, eu me detive, ergui"-me assustado e, inclinando a
cabeça, amedrontado, apurei o ouvido, com o coração palpitante. Havia
motivo para perturbação.

Já pressentira há muito tempo que lhe revirava a alma e partia o coração
e, quanto mais me certificava disso, com maior rapidez, e a maior força
possível, desejava atingir meu objetivo. O jogo, o jogo me atraía;
aliás, não era só o jogo\ldots{}

Sabia que estava falando de modo tenso, laborioso, quase livresco, em
suma, não sabia fazer diferente de ``como um livro''. Isso, porém, não
me perturbava; afinal, eu sabia, pressentia, que seria compreendido, e
esse mesmo caráter livresco podia ser de ainda mais valia. Só que agora,
atingido o efeito, fiquei de repente atemorizado. Não, jamais, eu jamais
fora testemunha de tamanho desespero! Ela estava deitada de bruços,
enfiando a cara no travesseiro com força e agarrando"-o com ambas as
mãos. Seu peito se dilacerava. O corpo jovem estremecia por inteiro,
como se tivesse convulsões. Os soluços oprimidos no peito pressionavam,
rasgavam e de repente prorrompiam para fora em berros e gritos. Então se
aferrava ainda mais ao travesseiro: não desejava que ninguém mais ali,
nem uma alma soubesse de seus tormentos e lágrimas. Mordia o
travesseiro, mordeu a mão até sangrar (isso eu vi depois) ou, agarrando
com os dedos as tranças desfeitas, paralisava de esforço, prendendo a
respiração e apertando os dentes. Comecei a lhe dizer algo, pedir que se
acalmasse, mas sentia que não conseguiria e, de repente, com um
calafrio, quase aterrorizado, comecei a me preparar, às apalpadelas,
para ir embora o quanto antes. Estava escuro: por mais que tentasse, não
conseguia acabar rápido. De repente, apalpei uma caixa de fósforos e um
castiçal com uma vela inteira, intacta. Bastou a luz iluminar o quarto e
Liza se ergueu de um salto, sentou"-se e me contemplou com a cara torta e
um rosto meio doido, quase insano. Sentei"-me a seu lado e a tomei pelas
mãos; ela voltou a si e se lançou em minha direção, querendo me abraçar,
mas não conseguiu e inclinou a cabeça na minha direção, em silêncio.

--- Liza, minha amiga, estou errado\ldots{} perdoe"-me --- comecei, mas ela
fincou seus dedos no meu braço com tamanha força que eu entendi que
falava mal, e parei.

--- Esse é meu endereço, Liza, venha para a minha casa.

--- Irei\ldots{} --- ela sussurrou, decidida, sem jamais erguer a cabeça.

--- Agora vou embora, adeus\ldots{} até logo.

Levantei"-me, ela também se levantou e, de repente, corou por inteiro,
estremeceu, pegou um lenço que estava na cadeira e cobriu dos ombros até
o queixo. Após fazê"-lo, voltou a dar um sorriso algo doentio, enrubesceu
e me lançou um olhar estranho. Sentia"-me doente; apressei"-me para sair,
escafeder"-me.

--- Espere --- ela disse de repente, já no saguão, à porta, detendo"-me com
a mão no capote; colocou a vela no chão, precipitadamente, e saiu
correndo --- pelo visto, lembrara"-se de algo, que queria trazer para me
mostrar. Ao correr, ficou toda corada, seus olhos brilhavam, um sorriso
se mostrava nos lábios: o que era aquilo? Esperei a contragosto; ela
regressou em um minuto, com um olhar que parecia pedir perdão por alguma
coisa. No geral, não se tratava mais do mesmo rosto nem do mesmo olhar
de há pouco, lúgubre, desconfiado e obstinado. Agora seu olhar era
suplicante, suave e, além disso, crédulo, carinhoso, timido. Assim as
crianças olham para aqueles que amam muito ou a que pedem algo. Seus
olhos eram castanho"-claros, olhos maravilhosos, vivos, capazes de
refletir o amor e o ódio soturno.

Sem me explicar nada --- como se eu, na qualidade de criatura superior,
devesse saber de tudo sem explicações ---, ela me entregou um papelzinho.
Nesse momento, todo seu rosto se irradiou com a mais ingênua solenidade,
quase infantil. Abri. Era uma carta para ela, de um estudante de
medicina ou algo do gênero: uma declaração de amor bastante empolada,
floreada, mas excepcionalmente respeitosa. Não me recordo agora das
expressões, mas me lembro muito bem que, por detrás do estilo elevado,
transparecia um sentimento verdadeiro, que não dá para falsificar. Ao
acabar de ler, encontrei um olhar ardente, curioso e de impaciência
infantil. Ela cravara os olhos no meu rosto e aguardava com
impaciência:: o que eu diria? Em algumas palavras rápidas, porém com
certa alegria, e como que se orgulhando, explicou"-me que estivera em uma
festa dançante em uma casa de família, de ``gente muito, muito boa,
\emph{gente de família,} e onde \emph{ainda não sabiam de nada,}
absolutamente de nada'', pois tinha acabado de chegar e, mesmo assim\ldots{}
não se decidira ainda a ficar e partiria sem falta, assim que tivesse
pagado a dívida\ldots{} ``Bem, lá também estava esse estudante, que passara a
noite inteira dançando com ela, falando com ela, e se revelou que ainda
em Riga, ainda em criança a conhecera, tinham brincado juntos, só que
fazia muito tempo, e até conhecia seus pais, só que \emph{daquilo} ele
não sabia nada"-nada"-nada, nem desconfiava! E eis que no dia seguinte às
danças (três dias atrás), ele lhe enviara, através de uma amiga com a
qual ela fora à festa, aquela carta\ldots{} e\ldots{} é tudo''.

Baixou os olhos cintilantes de forma algo envergonhada ao fim do relato.

Pobrezinha, guardava a carta daquele estudante como um tesouro e fora
correndo até seu único tesouro por não querer que eu partisse sem saber
que era amada de forma honrada e verdadeira, e que lhe falavam com
respeito. Provavelmente aquela carta estava destinada a jazer em um
porta"-jóias, sem consequência. Mas dava na mesma; tenho certeza de que
ela a guardaria a vida inteira como um tesouro, como seu orgulho e
justificativa, e naquele instante se lembrara dela e a trouxera para
ingenuamente se orgulhar diante de mim, apresentá"-la a meus olhos para
que eu a visse, para que eu a elogiasse. Não disse nada, apertei sua mão
e saí. Tinha tanta vontade de ir embora\ldots{} Percorri o caminho inteiro a
pé, apesar da neve úmida continuar caindo aos flocos. Estava extenuado,
esmagado, perplexo. A verdade, porém, já reluzia por detrás da
perplexidade. Uma verdade torpe!

\section{VIII}

Aliás, não concordei logo em admitir essa verdade. Ao acordar pela
manhã, depois de algumas horas de sono profundo e de chumbo, e
compreendendo de imediado tudo que ocorrera na véspera, cheguei a me
espantar com meu sentimentalismo com Liza, com todos aqueles ``horrores
e piedades de ontem''. ``Afinal tive um ataque feminino de nervos, arre!
--- decidi. --- E por que fui lhe impingir meu endereço? E se ela vier? A
propósito, pois seja, que venha; não é nada\ldots{}'' Porém,
\emph{evidentemente,} a questão principal e mais importante de agora não
era essa: era preciso me apressar e, a qualquer custo, salvar o quanto
antes minha reputação aos olhos de Zverkov e Símonov. Isso era o
principal. De Lisa, naquela manhã, cheguei a me esquecer completamente,
de tão atarefado.

Antes de tudo, era preciso quitar sem demora a divida da véspera com
Símonov. Decidi"-me por um meio desesperado: tomar todos os quinze rublos
emprestados de Anton Antônovitch\footnote{Antônytch, como vinha sendo
  escrito até então, é a forma abreviada de Antônovitch. {[}\textsc{n.\,t.}{]}}.
Como que de prepósito, naquela manhã ele estava de humor maravilhoso,
emprestando de imediato, ao primeiro pedido. Isso me deixou tão feliz
que, ao redigir o recibo, com ar temerário, informei"-lhe \emph{com
desdém} que, na véspera, ``farreara com uns amigos no \emph{Hôtel de
Paris:} era a despedida de um colega, até se poderia dizer que um amigo
de infância e, sabe, é um grande pândego, um mimado, mas, obviamente, de
boa família, posição importante, carreira brilhante, espirituoso,
encantador, tem intrigas com umas damas, entende: tomamos uma ``meia
dúzia'' a mais e\ldots{}'' E mais nada; afirmei isso tudo com muita leveza,
desembaraço e autossuficiência.

Ao chegar em casa, escrevi sem demora a Símonov.

Até agora me admiro ao recordar o tom franco de \emph{gentleman,}
bonachão e aberto, de minha carta. Hábil, nobre e, principalmente, sem
quaisquer palavras supérfluas, assumi a culpa por tudo. Justifiquei"-me,
``se ainda for lícito me justificar'', por minha completa falta de
costume etílico, tendo me embebedado ao primeiro cálice, (que teria)
bebido antes deles, quando os aguardava no \emph{Hôtel de Paris,} das
cinco às seis horas. Pedia desculpas principalmente a Símonov; pedia"-lhe
que transmitisse minhas explicações a todos os outros, especialmente a
Zverkov, que, ``pelo que me lembro, como que através de um sonho'', eu
parecia ter insultado. Acrescentava que teria ido à casa de todos, mas
estava com dor de cabeça, e o pior de tudo era a vergonha. Fiquei
especialmente satisfeito com ``certa leveza'', quase desdém (aliás,
totalmente a propósito) que de repente emanava de minha caneta e, melhor
que todas as razões possíveis, dava"-lhes logo a entender que eu encarava
``toda aquela nojeira da véspera'' de forma bastante independente; não
fiquei de jeito nem maneira arrasado, como os senhores provavelmente
achavam mas, pelo contrário, encarava aquilo como devia encarar um
\emph{gentleman} com respeito sereno por si mesmo. Como se dissesse: um
valente não leva bronca.

--- E não tem até uma coisa lúdica, de marquês? --- admirava"-me, relendo a
nota. --- Tudo isso porque sou uma pessoa evoluída e instruída! Um outro,
no meu lugar, não saberia como sair do apuro, mas eu escapei e vou
farrear de novo, tudo isso porque sou ``uma pessoa instruída e evoluída
de nosso tempo''. E, na verdade, talvez tudo isso tenha acontecido ontem
por causa do álcool. Hum\ldots{} mas não, não foi o álcool. Não tomei vodca
alguma das cinco às seis, enquanto os aguardava. Mentira a Símonov;
mentira sem pudor; e nem assim tinha vergonha\ldots{}

Aliás, não ligo! O principal é que que me safei.

Coloquei seis rublos na carta, lacrei e pedi a Apollon que a levasse a
Símonov. Sabendo que havia dinheiro na carta, Apollon ficou mais
respeitoso e concordou em ir. À noite, saí para passear. Minha cabeça
ainda doía e rodava por causa da véspera. Porem, quanto mais a noite
avançava e o crepúsculo se adensava, mais mudavam e se embaralhavam
minhas impressões e, com elas, as ideias. Algo não morria no meu
interior, no fundo do coração e da consciência, não queria morrer e se
exprimia por meio de uma angústia cruciante. Vagueava mais pelas ruas
mais apinhadas e comerciais, pela Mieschânskaia, pela Sadôvaia, perto do
Jardim de Iussúpov. Sempre gostei particularmente de perambular por
essas ruas ao crepúsculo, justo quando se adensa a multidão de
passantes, comerciantes e artesãos, de rostos preocupados ao ponto da
cólera, voltando para casa depois da jornada de trabalho. Agradava"-me
justamente esse rebuliço mesquinho, esse prosaico insolente. Dessa vez,
todo esse aperto na rua me irritou ainda mais. Não conseguia me
controlar, achar um fim. Algo se erguia, se erguia, sem cessar,
dolorosamente, e não queria sossegar. Voltei para casa totalmente
transtornado. Como se levasse algum crime na alma.

Atormentava"-me o tempo todo a ideia de que Liza viria. Estranhava que,
de todas as lembranças da véspera, a lembrança dela fosse a que
especialmente me atormentava. De todo o resto eu já tinha conseguido me
esquecer à noite, com um abano de mão, e ficara totalmente satisfeito
com minha carta para Símonov. Só com isso eu não estava satisfeito. Como
se Liza fosse meu único tormento. ``Então, e se ela vier? --- pensava, sem
cessar. --- E daí, não é nada, pois que venha. Hum. Só vai ser ruim ela
ver, por exemplo, como eu vivo. Ontem me mostrei na frente dela\ldots{} como
um herói\ldots{} e agora, hum! Aliás, é muito ruim ter me desleixado tanto.
Esse apartamento é simplesmente uma indigência. E fui decidir sair para
jantar ontem com aquela roupa! E esse meu sofá de oleado, com o
enchimento saltando para fora! E o meu roupão, que não cobre nada! Que
farrapos\ldots{} E ela vai ver tudo, e também vai ver Apollon. Esse animal
provavelmente vai insultá"-la. Vai destratá"-la para me fazer uma
grosseria. E eu, como de hábito, vou me acovardar, começar a trotar na
sua frente, a me cobrir com as abas do roupão, a rir, a mentir. Ui, que
indecência! E essa nem é a principal indecência. Há ainda uma maior,
mais torpe, mais vulgar! Sim, mais vulgar! Novamente, sim, novamente
colocar essa infame máscara de mentira!..''

Ao chegar a essa ideia, explodi:

``Por que infame? Infame no quê? Ontem falei com franqueza. Lembro"-me de
que havia um sentimento verdadeiro em mim. Queria justamente despertar
nela sentimentos nobres\ldots{} se ela chorou, foi bom, terá um efeito
benéfico\ldots{}''

Só que nem assim havia jeito de eu me acalmar.

Por toda essa noite, quando eu já estava de volta para casa, já depois
das nove horas, quando, pelas minhas contas, Liza não tinha como vir,
ela me aparecia assim mesmo e, principalmente, vinha à minha lembrança
sempre na mesma posição. De tudo que acontecera na véspera, um momento
em particular me ocorria com especial intensidade: quando iluminei o
quarto com o fósforo e avistei sua cara pálida e torta, com olhar de
mártir. Que riso penoso, artificial e torto ela tinha naquele minuto! Só
que eu ainda não sabia que, quinze anos depois, ainda continuaria a
imaginar Liza justo com o sorriso torto e desnecessário que tinha
naquele minuto.

No dia seguinte, já estava de novo pronto para considerar tudo isso uma
bobagem, uma estrepolia dos nervos e, principalmente, \emph{um exagero.}
Sempre considerei esse o meu ponto fraco, por vezes temendo"-o bastante:
``exagero tudo, e aí é que eu me complico'', repetia para mim mesmo, a
toda hora. Mas, em compensação, ``em compensação Liza deve vir assim
mesmo'': esse era o refrão que concluía todos meus raciocínios de então.
Ficava tão intranquilo que chegava às vezes à fúria. ``Vem! Vem sem
falta --- exclamava, correndo pelo quarto --- não hoje, talvez amanhã, mas
vai aparecer! E aquele maldito romantismo de todos esses \emph{corações
puros}! Oh torpeza, oh estupidez, oh mediocridade dessas ``intragáveis
almas sentimentais''! Bem, como não entender, como seria possível,
parece"-me, não entender?\ldots{}'' Mas daí eu me detinha, até em grande
aflição.

``E como são poucas, poucas --- pensava, de passagem --- as palavras
necessárias, o idílio necessário (e um idílio ainda por cima afetado,
livresco, montado) para imediatamente revirar toda uma alma humana a meu
bel"-prazer. Isso é que é solo fresco!

Por vezes me vinha a ideia de ir até ela, ``contar"-lhe tudo'' e pedir
que não venha à minha casa. Porém, a essa ideia, erguia"-se em mim
tamanho ódio que tinha a impressão de que, se essa ``maldita'' Liza de
repente aparecesse perto de mim, eu a esmagaria, insultaria, cuspiria,
expulsaria, bateria!

Passou, entretanto, um dia, dois, três, ela não veio, e comecei a me
acalmar. Ficava especialmente animado e alegre depois das nove horas, e
de vez em quando até começava a ter sonhos bastante doces: ``Por
exemplo, salvo Liza justamente por ela vir aqui e eu lhe falar\ldots{}
Desenvolvo"-a, educo"-a. Por fim, noto que ela me ama, ama"-me com paixão.
Finjo não compreender (aliás, não sei por que finjo; bem, provavelmente
para embelezar). Por fim, toda aflita, maravilhosa, tremendo e
soluçando, lança"-se a meus pés e diz que sou seu salvador e que me ama
acima de tudo no mundo. Fico atônito, porém\ldots{} ``Liza --- digo ---, será
que você pensa que eu não reparei no seu amor? Vi tudo, adivinhei, mas
não ousava atentar contra o seu coração, primeiro, por ter influência
sobre você e temer que, por gratidão, você se obrigasse a responder ao
meu amor, forçando"-se a suscitar um sentimento que talvez não existisse,
e eu não queria isso, pois se trata de\ldots{} despotismo\ldots{} Isso é
indelicado (bem, em suma, eu me embrulhava em alguma fineza europeia, à
la George Sand, de inefável nobreza\ldots{}) Mas agora, agora você me
pertence, você é minha criação, você é pura, maravilhosa, você é minha
maravilhosa esposa.

``Na minha casa, ousada e livre,

Entra como a absoluta senhora!''\footnote{Últimos versos do poema de
  Nekrássov que é citado no começo da segunda parte deste livro. {[}\textsc{n.\,t.}{]}}

Daí começamos a levar nossa vida, vamos para o exterior, etc, etc''. Em
suma, ficava vil até para mim, e eu terminava mostrando a língua para
mim mesmo.

``Mas nem vão deixar a 'patife' sair! --- pensava. --- Parece que não a
deixam passear muito, ainda mais à noite (por algum motivo, eu achava
que ela viria sem falta à noite, às sete em ponto). Aliás, ela disse que
não fora escravizada por completo, que tinha direitos especiais; quer
dizer, hum! Que diabo, ela vem, vem sem falta!''

Foi bom, ainda, que Apollon me distraísse nessa época com suas
grosserias. Estava acabando com o que me restava de paciência! Era a
minha chaga, um flagelo enviado pela providência. Vínhamos em constante
duelo verbal, por anos seguidos, e eu o odiava. Meu Deus, como eu o
odiava! Acho que nunca odiei tanto alguém na vida como ele,
especialmente em alguns momentos. Era um homem de meia"-idade,
sobranceiro, alfaiate por meio período. Porém, por motivo desconhecido,
desprezava"-me para além dos limites, e me olhava com uma altivez
insuportável. Aliás, olhava para todos com altivez. Bastava ver aquela
cabeça loira, bem penteada, aquele topete armado sobre a testa e untado
com azeite, aquela boca sólida, sempre em forma de
\emph{íjitza}\footnote{Triangular (a \emph{íjitza} era a última letra do
  velho alfabeto eclesiástico russo, e tinha o som de i). {[}\textsc{n.\,e.}{]}},
e os senhores já se sentiriam diante de uma criatura que não duvidava de
si em nada. Era pedante no mais alto grau, o maior pedante que conheci;
além disso, de um amor"-próprio que talvez só fosse digno de Alexandre da
Macedônia. Era apaixonado por cada um de seus botões, por cada uma de
suas unhas, infalivelmente enamorado, assim parecia! Tratava"-me com
absoluto despotismo, dirigia"-me a palavra extraordinariamente pouco e,
se calhava de me dirigir o olhar, esse olhar era duro, de uma
autoconfiança majestosa e constantemente zombeteiro, levando"-me às vezes
à ira. Cumpria seus deveres com o ar de que estava me fazendo o maior
dos favores. Aliás, quase não fazia nada por mim, não se considerando
com obrigação nenhuma de fazê"-lo. Não podia haver dúvida de que me
achava o maior tolo do mundo e, se ``me mantinha junto a si'', era
unicamente porque devia receber de mim, a cada mês, um salário.
Concordava em ``não fazer nada'' por mim por sete rublos por mês. Por
causa dele, muitos de meus pecados serão perdoados. O ódio por vezes
chegava a um ponto que apenas seu caminhar estava prestes a me dar
convulsões. Mas o que me deixava especialmente enojado era o seu falar
chiado. Tinha a língua um pouco mais comprida do que o devido, ou algo
do gênero, que o fazia ciciar e chiar o tempo todo e, ao que parece,
orgulhava"-se terrivelmente disso, imaginando que lhe conferia uma
dignidade extraordinária. Falava baixo, comedido, com as mãos para trás
e os olhos no chão. Deixava"-me particularmente irritado quando se punha
a ler o Livro dos Salmos para si, por detrás do tabique. Suportei muitas
batalhas por causa dessa leitura. Só que ele gostava terrivelmente de
ler à noite, em voz baixa e regular, arrastando as palavras, como oração
fúnebre. Curioso que tenha acabado assim: hoje se ocupa de ler o Livro
dos Salmos para os mortos e, além disso, extermina ratazanas e faz graxa
para sapatos. Naquela época, porém, eu não podia expulsá"-lo, como se ele
estivesse quimicamente unido à minha existência. Além disso, ele não
concordaria em sair da minha casa por nada. Eu não precisava morar em
\emph{chambres"-garnies}\footnote{Quartos mobiliados, em francês
  russificado no original. {[}\textsc{n.\,t.}{]}}: meu apartamento era minha
mansão, minha casca, o estojo em que me escondia de toda a humanidade, e
Apollon, sabe o Diabo por que, parecia"-me pertencer àquele apartamento,
e por sete anos inteiros não pude expulsá"-lo.

Segurar, por exemplo, seu salário, ainda que por dois ou três dias, era
impossível. Ele aprontaria uma história tal que eu não saberia onde me
meter. Naqueles dias, porém, encontrava"-me tão exasperado com todos que
resolvi, por algum motivo, \emph{castigar} Apollon e não lhe pagar o
salário por duas semanas. Já me preparava para fazer isso há tempos, uns
dois anos, unicamente para demonstrar"-lhe que não devia ousar se fazer
de tão importante comigo e que, caso não quisessee, eu sempre poderia
não lhe pagar. Decidira não tocar no assunto com ele e até ficar calado
de próposito, para derrotar seu orgulho e forçá"-lo a ser o primeiro a
falar do salário. Então tiraria da gaveta sete rublos, mostraria que os
tinha e estavam separados de caso pensado, só que ``não quero, não
quero, simplesmente não quero lhe pagar o salário, não quero porque é
\emph{meu desejo}'', porque essa é ``minha vontade de senhor'', por ele
ser desrespeitoso, por ele ser grosso; porém, caso ele pedisse com
respeito, talvez eu me abrandasse e pagasse; se assim não fosse,
esperaria duas semanas, três, um mês inteiro\ldots{}

Porém, por maior que fosse meu desagrado, ele é que venceu. No quarto
dia eu não aguentava mais. Ele começou do jeito que sempre começava
nesse tipo de caso, pois esse tipo de caso já tinha ocorrido, justamente
me pondo à prova (e devo observar que eu já sabia disso tudo com
antecedência, conhecia de cor sua tática infame): começava por me
dirigir um olhar excepcionalmente severo, que não desviava por alguns
minutos, especialmente ao me encontrar ou se despedir de mim. Se, por
exemplo, eu resistia e fazia de conta que não reparava nesses olhares,
ele, tão calado quanto antes, procedia aos suplícios subsequentes. De
repente, sem como nem por que, andava suave e em silêncio até o meu
quarto, quando eu estava caminhando ou lendo, parava perto da porta,
colocava uma mão para trás, afastava um pé e me dirigia seu olhar, que
não não era severo, mas de desprezo absoluto. Caso eu de repente lhe
perguntasse o que queria, ele não responderia nada, seguindo a me fitar
obstinadamente por alguns segundos, depois, apertando os lábios de certo
modo, com ar significativo, virava"-se devagar e partia devagar para o
seu quarto. Duas horas depois, voltava a entrar e a surgir na minha
frente do mesmo jeito. Acontecia que, em meu furor, eu já não lhe
perguntava o que queria, simplesmente erguendo a cabeça de forma brusca
e imperiosa, e também passava a fitá"-lo obstinadamente. Ficávamos nos
encarando desse jeito por dois minutos; por fim, ele se virava, devagar
e altivo, e voltava a sair por duas horas.

Se nem com isso tudo eu me persuadisse e continuasse minha rebeldia, ele
de repente começaria a suspirar na minha cara, suspirar de forma longa e
profunda, como se medisse com cada um desses suspiros a profundidade de
minha queda moral e, obviamente, sempre terminava por triunfar por
completo: eu me encolerizava, gritava mas assim mesmo era coagido a
cumprir o que estava em questão.

Dessa vez, mal começaram as habituais manobras dos ``olhares severos'',
eu saí de mim imediatamente e caí em cima dele, irado. Afinal, já estava
bastante irritado por outras razões.

--- Pare! --- gritei, frenético, quando ele se virou devagar e em silêncio,
com uma mão para trás, encaminhando"-se para o seu quarto. --- Pare! Volte,
volte, estou falando com você! --- eu devo ter rugido de forma bem pouco
natural, pois ele se virou e passou a me olhar até com certo espanto.
Seguiu, a propósito, sem proferir palavra, o que me agastava.

--- Como você ousa vir até mim sem ser chamado e me olhar desse jeito?
Responda!

Porém, fitando"-me tranquilo na penumbra, ele voltou a se virar.

--- Pare! --- rugi e corri atrás dele. --- Não saia do lugar! Assim. Agora
responda: você veio olhar o quê?

--- Se nesse instante o senhor tiver alguma ordem, é meu dever cumpri"-la
--- respondeu, ciciando em voz baixa e comedida, erguendo as
sobrancelhas, passando a cabeça tranquilamente de um ombro a outro e
voltando a se calar: tudo isso com tranquilidade horripilante.

--- Não é isso, não é isso que estou perguntando, seu carrasco! --- gritei,
tremendo de raiva. --- Eu mesmo vou lhe dizer, seu carrasco, para que você
veio aqui: está vendo que eu não lhe pago o salário, por orgulho não
quer se rebaixar e pedir e, por causa disso, vem me castigar com esses
olhares estúpidos, me atormentar, e não de"-e"-esconfia, seu carrasco, que
isso é estúpido, estúpido, estúpido, estúpido, estúpido!

Estava para se virar de novo, em silêncio, mas eu o detive.

--- Escute --- gritei. --- O dinheiro está aqui, veja; aqui! (tirei da
mesinha) Os sete rublos, só que você não vai receber, não vai
re"-e"-e"-ceber até que venha, com respeito, de cabeça baixa, me pedir
perdão. Ouviu?

--- Isso não pode ser! --- respondeu, com uma autoconfiança artificial.

--- Será! --- gritei. --- Dou"-lhe minha palavra de honra de que será!

--- Não tenho porque lhe pedir perdão --- ele prosseguiu, como se não
tivesse notado de jeito nenhum meus gritos ---, pois foi o senhor que me
xingou de ``carrasco'', ofensa pela qual sempre posso ir à polícia do
quarteirão prestar queixa.

--- Vá! Preste! --- rugi. --- Vá agora, nesse minuto, nesse segundo! E mesmo
assim você é um carrasco! Carrasco! Carrasco! --- Ele, porém, só fez me
olhar para depois se virar e, sem mais ouvir meus gritos de apelo,
encaminhou"-se suavemente para seu quarto, sem se voltar.

``Se não fosse a Lisa, não haveria nada disso! --- decidi comigo mesmo.
Depois de um minuto parado, dirigi"-me a ele, atrás do tabique, altivo e
solene, mas devagar, e com o coração batendo forte.

--- Apollon! --- falei baixo e pausado, porém arquejante. --- Vá agora mesmo,
sem demora, atrás do inspetor de quarteirão!

Enquanto isso, ele já se sentara à mesa, pusera os óculos e pegara algo
para costurar. Porém, ao ouvir minha ordem, bufou e riu.

--- Agora, nesse minuto, vá! Vá, vá, ou você não imagina o que vai ser!

--- O senhor efetivamente não está no seu juízo --- notou, sem sequer
levantar a cabeça, ciciando devagar e prosseguindo a enfiar a linha. ---
Onde é que já se viu uma pessoa ir à delegacia dar queixa de si mesma?
No que tange ao medo, o senhor está se esgoelando à toa, pois nada
acontecerá.

--- Vá! --- eu gania, agarrando"-o pelo ombro. Tinha ganas de bater nele.

Só que eu não tinha ouvido que, naquele instante, a porta do saguão se
abrira devagar e sem ruído e um vulto entrara, detivera"-se e passara a
nos fitar, perplexo. Olhei, morri de vergonha e me precipitei para o meu
quarto. Lá, agarrando os cabelos com ambas as mãos, encostei a cabeça na
parede e fiquei paralisado nessa posição.

Dois minutos depois, ressoaram os passos lentos de Apollon.

--- Tem \emph{uma aí} perguntando pelo senhor --- disse, olhando"-me com
especial severidade, para depois se apartar e dar passagem a Liza. Não
queria ir embora, examinando"-nos com ar zombeteiro.

--- Fora! Fora! --- ordenei, perdendo o controle. Nesse instante, meu
relógio fez um esforço, chiou e bateu as sete.

\section{IX}

\epigraph{Na minha casa, ousada e livre,\\
Entra como a absoluta senhora!}{\textsc{da poesia que já citei}}

Estava diante dela mortificado, envergonhado, asquerosamente embaraçado
e, ao que parece, rindo, tentando com todas as forças me cobrir com as
abas de meu roupãozinho felpudo de algodão, exatamente como ainda há
pouco imaginara, em depressão. Deois de ficar dois minutos na nossa
frente, Apollon se foi, o que não aliviou nada para mim. O pior de tudo
é que ela de repente também ficou embaraçada, a um ponto que eu não
esperava. Obviamente, olhava para mim.

--- Sente"-se --- eu disse, maquinalmente, empurrando"-lhe a cadeira ao lado
da mesa, enquanto me sentava no sofá. Obediente, ela logo se sentou,
fitando"-me com os olhos bem abertos e, pelo visto, esperando algo de
mim. Essa ingenuidade também me levava ao furor, mas eu me contive.

Deveria tentar não reparar em nada, como se tudo estivesse normal, só
que ela\ldots{} Tinha uma sensação vaga de que ela me pagaria caro \emph{por
tudo aquilo.}

\emph{-} Você me surpreendeu em uma situação estranha, Liza --- comecei,
gaguejando e sabendo que esse era justo o jeito que não devia começar.

--- Não, não, não ache nada! --- gritei, ao ver que ela tinha corado de
repente. --- Não me envergonho da minha pobreza. Pelo contrário, encaro"-a
com orgulho. Sou pobre, porém tenho nobreza\ldots{} É possível ser pobre e
ter nobreza --- balbuciei. --- A propósito\ldots{} quer chá?

--- Não\ldots{} --- ela quis começar.

--- Espere!

Levantei"-me de um salto e fui correndo até Apollon. Afinal, precisava
sumir em algum lugar.

--- Apollon --- sussurrei, em fala rápida e febril, jogando na sua frente
os sete rublos que conservara em meu punho o tempo todo ---, esse é o seu
salário; veja, estou pagando; mas, por isso, você tem que me salvar:
traga sem demora da taverna chá e dez torradas. Se não quiser ir, vai
fazer a infelicidade de uma pessoa! Você não sabem como é essa mulher\ldots{}
Isso é tudo! Pode ser que você ache alguma coisa\ldots{} Mas vocẽ não sabe
como é essa mulher!..

Apollon, já sentado, ao trabalho, e novamente de óculos, incialmente,
sem largar a agulha, olhou de viés para o dinheiro; depois, sem prestar
a menor atenção em mim nem me responder, continuou a se dedicar à linha,
que ainda enfiava na agulha. Esperei três minutos, parado, na sua
frente, com os braços \emph{à la Napoléon}\footnote{À Napoleão, em
  francês no original. {[}\textsc{n.\,t.}{]}}. Minhas têmporas estavam
molhadas de suor; eu estava pálido, sentia"-o. Porém, graças a Deus, ao
olhar para mim, ele com certeza ficou com pena. Ao terminar com a linha,
soergueu"-se devagar, afastou a cadeira devagar, tirou os óculos devagar,
contou o dinheiro devagar e, por fim, perguntou"-me por cima do ombro:
devo pegar meia porção? Saiu do quarto devagar. Ao regressar para Liza,
outra coisa me ocorreu: não devia sair correndo daquele jeito, de
roupãozinho, para onde desse na telha, e depois seja o que for?

Voltei a me sentar. Ela me fitava, intranquila. Ficamos em silêncio por
alguns minutos.

--- Eu o mato! --- gritei, de repente, dando na mesa com o punho com tamanha
força que a tinta do tinteiro derramou.

--- Ah, o que é isso? --- ela gritou, trêmula.

--- Eu o mato, mato! --- eu gania, batendo na mesa, em total furor e, ao
mesmo tempo, total compreensão de como esse furor era estúpido.

--- Você não sabe, Liza, o que esse carrasco faz comigo. É o meu
carrasco\ldots{} Agora foi atrás de torrada; ele\ldots{}

E de repente me debulhei em lágrimas. Era um ataque. Como senti vergonha
em meio aos soluços; mas não conseguia me segurar. Ela se assustou.

--- O que o senhor tem? O que o senhor tem? --- gritava, agitando"-se perto
de mim.

--- Água, dê"-me água, está lá! --- balbuciei, com voz débil, reconhecendo
para mim mesmo, a propósito, que poderia passar muito bem sem água e sem
balbuciar com voz débil. Só que eu, como direi, \emph{representava} para
salvar o decoro, embora o ataque fosse real.

Ela me deu água, fitando"-me como que desnorteada. Nesse instante,
Apollon trouxe o chá. Tive de repente a impressão de que aquele chá
corriqueiro e prosaico era terrivelmente indecente e mísero depois de
tudo que houvera, e enrubesci. Liza olhou para Apollon até assustada.
Ele saiu sem nos fitar.

--- Liza, você me despreza? --- eu disse, encarando"-a obstinadamente,
tremendo de impaciência para saber sua opinião.

Ela ficou embaraçada e não ousou responder.

--- Tome chá! --- afirmei, com raiva. Estava zangado comigo mesmo mas,
obviamente, tinha que sobrar para ela. De repente, uma raiva contra ela
fervilhava em meu coração; tinha a impressão de que a mataria. Para me
vingar, jurei mentalmente não lhe dirigir a palavra. ``Pois ela é a
causa disso tudo'' --- pensei.

O silêncio já se prolongava por cinco minutos. O chá jazia na mesa; não
tocamos nele; cheguei a ponto de não querer começar a beber de
propósito, para constrangê"-la ainda mais; ela estava sem jeito de ser a
primeira a beber. Fitou"-me algumas vezes, com triste perplexidade. Eu
prosseguia em silêncio obstinado. Claro que o principal mártir era eu
mesmo, pois reconhecia toda a baixeza asquerosa da minha estupidez irada
e, ao mesmo tempo, não tinha como me conter.

--- Quero\ldots{} ir embora\ldots{} dali\ldots{} de uma vez por todas --- começou, para
romper o silêncio de alguma forma, mas, coitada! Exatamente com isso não
deveria ter começado a falar naquela hora, que já era tão estúpida, a
uma pessoa que já era tão estúpida como eu. Até o meu coração doeu de
pena de sua franqueza inábil e desnecessária. Porém, algo de indecoroso
imediatamente esmagou toda a pena; até me espicaçou mais: o mundo que se
acabe! Passaram mais cinco minutos.

--- Por acaso estou incomodando? --- começou, tímida, quase inaudível, e se
pôs a se levantar.

Mas bastou eu avistar esse primeiro arroubo de dignidade ofendida para
chegar a tremer de raiva, explodindo de imediato.

--- Diga"-me, por favor, para que você veio aqui? --- comecei, arquejante e
até sem levar em conta a ordem lógica das minhas palavras. Queria dizer
tudo de uma vez, de um jorro; nem cuidava por onde começaria.

--- Por que você veio? Responda! Responda! --- eu gritava, quase fora de
mim. --- Vou lhe dizer, mãezinha, por que veio. Você veio porque então eu
lhe disse \emph{palavras piedosas}\footnote{Citação do romance
  \emph{Oblômov} (1859), de Gonctharov: assim Zakhar se refere aos
  sermões edificantes de seu patrão (capítulo 8 da parte 1, capítulo 7
  da pate 2). {[}\textsc{n.\,e.}{]}}. Pois bem, você se enterneceu e ficou
com vontade de mais ``palavras piedosas''. Pois fique sabendo que então
eu estava rindo de você. E estou rindo de novo. Por que treme? Sim,
rindo! Antes daquilo fui ofendido, no jantar, por aqueles que chegaram
antes de mim. Fui até lá para dar uma sova em um deles, um oficial; mas
não consegui, não o encontrei; tinha que descarregar a ofensa em alguém,
ser recompensado, você apareceu, despejei minha raiva em cima de você e
ri. Humilharam"-me, então eu também queria humilhar; reduziram"-me a
trapos, então eu queria demonstrar poder\ldots{} Foi isso que aconteceu, e
você achando que eu fui até lá especialmente para salvá"-la, hein? Você
achava isso? Você achava isso?

Sabia que ela talvez estivesse confusa e não recordaria os detalhes; mas
também sabia que se lembrava perfeitamente do essencial. E assim foi.
Ficou pálida como um lenço, quis dizer alguma coisa, seus lábios se
retorceram, doentios; porém caiu na cadeira, como se tivesse sido
cortada por um machado. Depois, durante o tempo todo em que me escutava,
temia de um pavor horrendo, de boca e olhos abertos. O cinismo, o
cinismo de minhas palavras a esmagava\ldots{}

--- Salvar! --- prossegui, saltando da cadeira e correndo diante dela, para
a frente e para trás do aposento. --- Salvar de quê? E eu ainda posso ser
pior do que você. Por que então você não me jogou no focinho, quando eu
estava lhe fazendo o sermão: ``E você, diga, veio para cá para quê? Por
acaso para pregar moral?'' Poder, poder é o que necessitava então,
necessitava do jogo, necessitava conseguir as suas lágrimas, a
humilhação, a sua histeria, é disso que eu então necessitava! Eu mesmo
não me aguentei, porque sou sujo, assustei"-me e sabe"-se lá por que diabo
lhe dei meu endereço, de bobeira. Logo depois, antes ainda de chegar em
casa, eu a estava xingando de tudo que há no mundo por causa desse
endereço. Já a odiava por ter lhe mentido. Pois só queria brincar com as
palavras, sonhar em minha cabeça, e, na realidade, sabe do que preciso?
Que vocês se acabem, é isso. Preciso de sossego. E para que não me
perturbem, venderia agora mesmo o mundo inteiro por um copeque. O mundo
deve acabar ou devo tomar meu chá? Digo que o mundo deve acabar, para
que eu sempre tome meu chá. Você sabia disso ou não? Bem, mas eu sei que
sou um pulha, um canalha, um egoísta, um preguiçoso. Assim, fiquei três
dias tremendo de medo de que você viesse. Sabe o que mais me preocupou
nesses três dias? Que então eu surgi como herói na sua frente, e que
agora você de repente me veria no meu roupãozinho rasgado, miserável,
nojento. Há pouco lhe disse que não me envergonho da minha pobreza; pois
fique sabendo que me envergonho, envergonho"-me acima de tudo, é o meu
pior temor, pior do que se roubasse, pois sou vaidoso como se tivessem
me arrancado a pele, e o ar sozinho me causasse dor. Mas será que nem
agora você adivinhou que jamais vou perdoá"-la por ter me pego com esse
roupãozinho quando eu estava me atirando em cima do Apollon como um cão
raivoso? O ressuscitador, o antigo herói se atira, como um vira"-latas
desgrenhado e tinhoso, em cima de seu lacaio, que se ri dele! Também
pelas lágrimas que há pouco, como uma mulher envergonhada, não consegui
conter, jamais vou lhe perdoar! Por tudo aquilo que agora estou
admitindo a você, também jamais vou \emph{lhe} perdoar! Sim, você, você
tem que responder por tudo isso sozinha, por ter aparecido assim, por eu
ser um pulha, por eu ser o mais asqueroso, ridículo, mesquinho,
estúpido, odioso de todos os vermes da terra, que não são em nada
melhores do que eu mas que, sabe o diabo por que, jamais ficam
embaraçados; enquanto eu, por toda minha vida, vou receber piparotes de
qualquer lêndea --- essa é a minha característica! E o que tenho a ver se
você não entende nada disso? E o que tenho, pois bem, o que tenho a ver
com você e com o fato de você estar ou não se arruinando naquele lugar?
Agora você entende que, depois de dizer isso, vou odiá"-la por ter estado
aí e escutado? Pois uma pessoa só se expressa assim uma vez na vida, e
quando está histérica!.. O que você ainda quer? Depois disso tudo, por
que você fica plantada na minha frente, me atormentando, e não vai
embora?

Mas então se produziu, de repente, uma estranha circunstância.

Eu estava tão habituado a pensar e imaginar tudo como nos livros, e ver
tudo no mundo como tinha fabricado nos sonhos, que não entendi
imediatamente aquela estranha circunstância. Acontecera o seguinte:
Liza, ofendida e esmagada por mim, entendeu muito mais do que eu
imaginava. De tudo aquilo, compreendera o que a mulher compreende antes
de tudo, se seu amor é franco: que eu era infeliz.

A sensação de susto e ofensa em seu rosto deram lugar, inicialmente, a
uma perplexidade amargurada. Quando comecei a me chamar de canalha e
pulha, e minhas lágrimas correram (proferi toda essa tirada entre
lágrimas), todo seu rosto foi percorrido por uma convulsão. Quis
levantar"-se, deter"-me; quando terminei, não prestou atenção em meus
gritos de ``por que voê está aqui, por que não vai embora!'', mas em que
devia ter sido muito difícil para mim dizer aquilo tudo. Estava muito
amedrontada, coitada: considerava"-se infinitamente inferior a mim; como
poderia se enfurecer, se ofender? Levantou"-se de repente da cadeira, em
impulso irresistível e, precipitando"-se toda para mim, mas ainda tímida
e seu ousar sair do lugar, estendeu"-me os braços\ldots{} Daí meu coração
revirou. Então ela se atirou de repente na minha direção, enlaçou meu
pescoço e se pôs a chorar. Também não me contive e me desfiz em pranto,
como jamais havia ocorrido\ldots{}

--- Não me deixam\ldots{} Eu não posso ser\ldots{} bom! --- mal consegui proferir,
depois fui até o sofá, onde caí de bruços, passando um quarto de hora em
soluços, em autêntica histeria. Ela se apertou contra mim, abraçou"-me e
ficou como que paralisada nesse abraço.

Mas, mesmo assim, a coisa era que a histeria tinha que passar. Então
(estou escrevendo uma verdade asquerosa), deitado de bruços no sofá, com
a cara enfiada com força na minha imunda almofada de couro, comecei a
sentir aos poucos, de longe, mas irresistivelmente, que agora seria
embaraçoso levantar a cabeça e olhar direto nos olhos de Liza. Do que
tinha vergonha? Não sei, mas tinha vergonha. Também passou por minha
cabeça perturbada que os papéis agora haviam se invetido completamente,
que agora ela era a heroína, enquanto eu agora era uma criatura tão
humilhada e esmagada quanto ela tinha sido naquela noite, quatro dias
atrás\ldots{} Tudo isso me ocorreu ainda naqueles instantes em que estava de
bruços no sofá!

Meu Deus! Será que então eu a invejava?

Não sei, até hoje não consegui decidir, e claro que então conseguia
entender menos ainda do que agora. Pois não consigo viver sem exercer
poder e tirania contra alguém\ldots{} Porém\ldots{} porém nada se explica com
raciocinios e, consequentemente, não há motivo para raciocinar.

Entretanto, venci a mim mesmo e ergui a cabeça: era necessário
levantá"-la em algum momento\ldots{} Pois bem, até agora estou seguro de que
exatamente por ter vergonha de olhar para ela, em meu coração de repente
se acendeu e se inflamou um outro sentimento\ldots{} um sentimento de
dominação e posse. Meus olhos reluziam de paixão, e apertei suas mãos
com força. Como eu a odiava, e como era atraído por ela naquele
instante! Um sentimento reforçava o outro. Parecia quase uma
vingança!\ldots{} Eu seu rosto refletiu"-se primeiro algo como uma
perplexidade, quase como medo, mas só por um momento. Abraçou"-me com
entusiasmo e ardor.

\section{X}

Um quarto de hora mais tarde, eu corria para frente e para trás, no
quarto, em impaciência furiosa, aproximando"-me do biombo a cada instante
e espiando Liza por uma pequena fresta. Estava sentada no chão, com a
cabeça na cama, e devia estar chorando. Só que não ia embora, e isso me
irritava. Dessa vez, ela já sabia de tudo. Eu a insultara
definitivamente, mas\ldots{} não há o que contar. Ela adivinhara que meu
ataque de paixão tinha sido exatamente uma vingança, uma nova
humilhação, e que a meu ódio antigo e quase indefinido fora acrescido
agora um ódio \emph{pessoal,} contra ela, \emph{invejoso\ldots{}} Aliás, não
garanto que ela tenha entendido tudo isso de forma exata; em
compensação, entendeu muito bem que sou um homem abominável e,
principalmente, que não estou em condições de amá"-la.

Sei que vão me dizer que isso é improvável, é improvável ser tão mau e
estúpido como eu; talvez ainda acrescentem que era improvável não amá"-la
ou, pelo menos, não dar valor a esse amor. Por que improvável? Primeiro,
eu já não podia amar, pois amar, para mim, significava tiranizar e
exercer supremacia moral. Por toda a vida não pude sequer imaginar outro
tipo de amor, chegando ao ponto de hoje achar, às vezes, que o amor
consiste em conceder ao objeto amado o direito de exercer tirania sobre
si. Nos meus sonhos do subsolo eu também não imaginava o amor como outra
coisa senão uma luta que sempre começava com o ódio e terminava com a
submissão moral, e depois não conseguia mais imaginar o que fazer com o
objeto submetido. E o que há de improvável quando eu já estava tão
corrompido do ponto de vida moral, tão desacostumado da ``vida viva'',
que há pouco pensara em ralhar com ela e envergonhá"-la por ter vindo à
minha casa para ouvir ``palavras piedosas''; mas não adivinhara que ela
não tinha vindo para ouvir palavras piedosas, mas para me amar, pois,
para a mulher, no amor consiste toda a ressurreição, toda a salvação de
não importa qual destruição e toda a regeneração, que não tem como se
manifestar de outra forma. Aliás, eu já não a odiava tanto ao correr
pelo quarto e espiar pela fresta do biombo. Era apenas insuportavelmente
duro que ela estivesse ali. Queria que ela desaparecesse. Desejava
``tranquilidade'', desejava ficar sozinho no subsolo. Por falta de
hábito, a ``vida viva''\footnote{O conceito de ``vida viva'' estava
  difundido na literatura e no jornalismo do século \textsc{xix}. Encontra"-se nos
  eslavófilos Khomiakov, Samárin e Kirêievski, mas também em Turguêniev
  e Herzen. Pode"-se julgar o significado dessa ideia em Dostoiévski
  pelas palavras do personagem Versílov no romance \emph{O adolescente}
  (1875). Ele diz: ``\ldots{} a viva vida, ou seja, não intelectual, nem
  fabricada {[}\ldots{}{]} deve ser algo terrivelmente simples, o mais
  ordinário, que se lança à vista a cada dia e a cada instante\ldots{}'' {[}\textsc{n.\,e.}{]}} me oprimia tanto que era difícil até respirar.

Transcorreram, porém, alguns minutos, e ela ainda não se levantara, como
se estivesse desmaiada. Tive a desonestidade de bater de leve no biombo,
para lembrá"-la\ldots{} Ela estremeceu de repente, saltou do lugar e se
precipitou em busca de seu lenço, seu chapéu, peliça, como se fosse
escapar de mim\ldots{} Dois minutos depois, saiu lentamente de trás do biombo
e me lançou um olhar pesado. Dei um riso raivoso, aliás, forçado,
\emph{por decoro,} e desviei de seu olhar.

--- Adeus --- ela disse, encaminhando"-se para a porta.

Corri até ela de repente, agarrei sua mão, abri"-a, coloquei\ldots{} e depois
voltei a fechá"-la. Logo em seguida, virei"-me e dei um pulo até o outro
canto, para pelo menos não ver\ldots{}

Naquele instante, tive vontade de mentir: escrever que tinha feito
aquilo sem querer, inconsciente, perdido, de bobeira. Mas não quero
mentir e, por isso, digo com franqueza que abri sua mão e coloquei lá\ldots{}
de raiva. Ocorreu"-me fazê"-lo quando estava correndo pelo quarto, para
frente e para trás, com ela sentada atrás do biombo. Mas o que posso
dizer com certeza é que cometi essa crueldade, ainda que deliberada, não
por meu coração, mas por minha cabeça ruim. Essa crueldade era tão
afetada, tão cerebral, deliberadamente fabricada, \emph{livresca,} que
não aguentei um instante; primeiro pulei para o canto, para não ver,
depois, envergonhado e desesperado, lancei"-me no encalço de Liza. Abri a
porta do saguão e me pus a apurar o ouvido.

--- Liza! Liza! --- gritei para a escadaria, porém hesitante, a meia"-voz\ldots{}

Não houve resposta; tive a impressão de ouvir seus passos nos degraus de
baixo.

--- Liza! --- gritei, mais alto.

Sem resposta. Naquele momento, porém, ouvi de baixo a dura porta de
cristal que dava para a rua abrir"-se pesadamente, com um ganido, e se
fechar com força. O barulho se ergueu pela escadaria.

Ela tinha partido. Regressei para o quarto, refletindo. Sentia um peso
horrível.

Parei perto da mesa, ao lado da cadeira em que ela tinha se sentado, e
lancei um olhar insensato para a frente. Passou um minuto e, de repente,
tremi todo: justo na minha frente, em cima da mesa, avistei\ldots{} em suma,
avistei uma nota azul amarrotada de cinco rublos, a mesma que enfiara em
sua mão há instantes. Era \emph{aquela} nota; não podia ser outra; não
havia outra na casa. Ela deve ter conseguido jogá"-la na mesa no momento
em que eu saltei para o outro canto.

E daí? Eu podia esperar que ela faria isso. Podia esperar? Não. Eu era
tão egoísta, desrespeitava tanto as pessoas que nem podia imaginar que
ela o fizesse. Isso eu não aguentei. Passado um minuto, fui me vestir,
feito doido, joguei em cima de mim o que pude pegar às pressas e me
precipitei, correndo, atrás dela. Ela não podia ter dado uns duzentos
passos quando me pus a correr pela rua.

Estava calmo, a neve caía de forma quase perpendicular, revestindo a
calçada e a rua deserta de uma almofada. Não havia passantes, não se
ouvia um som. Os lampiões tremeluziam tristes e inúteis. Corri duzentos
passos, até o cruzamento, e parei.

``Para onde ela foi? E por que estou correndo atrás dela? Por quễ?
Desabar na sua frente, soluçar de arrependimento, beijar seus pés,
implorar perdão! Eu também queria isso; todo meu peito se desfazia em
pedaços, e jamais, jamais recordarei aquele momento com indiferença. Mas
para quê? --- refletia. --- Será que não vou odiá"-la, talvez, amanhã mesmo,
exatamente por ter beijado seus pés hoje? Será que vou lhe proporcionar
felicidade? Será que não fiquei sabendo hoje, pela centésima vez, o
quanto valho? Será que não vou torturá"-la?'' Fiquei parado, na neve,
examinando as brumas momentaneamente, e pensando nisso.

``E não será melhor, não será melhor --- eu fantasiava, já em casa,
depois, abafando com a fantasia a dor viva em meu coração ---, não será
melhor se ela carregar o insulto consigo para sempre? O insulto é uma
purificação; a consciência mais cáustica e dolorida! Amanhã mesmo eu
teria poluído sua alma e fatigado seu coração. O insulto, porém, agora
jamais morrerá, e por mais asquerosa que seja a imundície que a espera,
o insulto há de elevá"-la e limpá"-la\ldots{} pelo ódio\ldots{} hum\ldots{} pode ser que
também pelo perdão\ldots{} A propósito, será que tudo isso vai facilitar para
ela?''

E, de fato, agora me faço uma pergunta ociosa: o que é melhor, uma
felicidade barata ou um sofrimento elevado? Então, o que é melhor?

Isso me vinha à mente quando estava sentado em casa, naquela noite, mal
sobrevivendo à dor na alma. Jamais suportei tamanho sofrimento e
arrependimento; poderia, porém, haver alguma dúvida, quando saí correndo
do apartamento, de que voltaria para casa no meio do caminho? Nunca mais
encontrei Liza, nem ouvi falar dela. Aconteceu"-me também de ficar muito
tempo satisfeito com a \emph{frase} sobre os proveitos do insulto e do
ódio, embora naquela oportunidade tenha quase adoecido de angústia.

Mesmo agora, alguns anos depois, tudo isso me vem à memória de um jeito
muito ruim. Agora, muita coisa ruim me vem à memória, mas\ldots{} não é o
caso de terminar aqui as ``Memórias''? Acho que cometi um erro ao
começar a escrevê"-las. Pelo menos, fiquei com vergonha por todo tempo
que escrevia essa \emph{novela;} portanto, isso não é mais literatura,
mas castigo corretivo. Afinal, contar, por exemplo, em longos relatos,
como malbaratei minha vida em depravação moral, em um canto, por
insuficiência de meios, desacostumando"-me da vida e com uma vaidade
irada no subsolo, meu Deus, não é interessante; um romance precisa de
heróis, e aqui estão \emph{deliberadamente} reunidos todos os traços do
anti"-herói e, principalmente, tudo isso vai causar uma impressão
desagradabilíssima, pois todos nós nos desacostumamos da vida,
claudicamos todos, uns mais, outros menos. Estamos desacostumados a
ponto de sentir, vez ou outra, uma repugnância pela verdadeira ``vida
viva'', por isso não podemos suportar quando nos lembram dela. Afinal,
chegamos ao ponto de considerar a verdadeira ``vida viva'' quase um
trabalho, quase um serviço, concordando no íntimo que nos livros é
melhor. E por que de vez em quando temos uns formigamentos,
extravagâncias, o que pedimos? Nós mesmos não sabemos. Seria pior para
nós se atendessem a nossos rogos extravagantes. Bem, experimentemos,
bem, dêem"-nos, por exemplo, mais independência, desatem nossas mãos,
ampliem o círculo de atividades, enfraqueçam a tutela, e nós\ldots{} posso
assegurar: imediatamente pediremos o retorno da tutela. Sei que os
senhores talvez vão se zangar comigo por isso, gritar, bater os pés,
afirmando: ``Diga"-o por si só e por suas misérias do subsolo, mas não
ouse dizer \emph{'todos nós'.''} Perdão, senhores, não estou me
justificando com esse \emph{todos.} No que tange a mim, apenas levei ao
extremo, em minha vida, o que os senhores não ousaram levar até a
metade, tomando covardia por sensatez e, dessa forma, consolando e
enganando a si mesmos. De modo que talvez eu seja ainda ``mais vivo'' do
que os senhores. Pois olhem com mais atenção! Sabemos afinal onde o que
é vivo vive agora, o que ele é, como se chama? Deixem"-nos sós, sem
livros, e imediatamente vamos nos confundir e nos perder; não saberemos
a quem nos unir, a quem seguir; o que amar e o que odiar, o que
respeitar e o que desprezar. Incomodamo"-nos até em ser gente, gente com
corpo e sangue real, \emph{próprio;} temos vergonha disso, consideramos
uma ignomínia e fazemos de tudo para ser uma espécie inexistente de
homens gerais. Somos natimortos, pois há tempos não nascemos de pais
vivos, e isso nos agrada cada vez mais. Estamos tomando gosto por isso.
Logo criaremos um jeito de nascer das ideias. Mas chega; não quero mais
escrever ``do Subsolo''\ldots{}

Aliás, as ``memórias'' desse paradoxalista não acabam aqui. Ele não
aguentou e seguiu adiante. Só que também achamos que aqui devem parar.
