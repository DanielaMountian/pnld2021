\SVN $Id: INTRO.tex 12208 2013-06-19 19:52:46Z leda $ 

\chapter[Introdução, por Christian Werner]{Introdução}
\hedramarkboth{Introdução}{Christian Werner}

\epigraph{Trepava de ser o mais honesto de todos, ou o mais danado, no
tremeluz, conforme as quantas. Soava no que falava, artes que falava, diferente
na autoridade, mas com uma autoridade muito veloz.}{J.~Guimarães Rosa,
\textit{Grande sertão: veredas}}


\section{fundamentos do poema}

Junto com Homero, Hesíodo é um poeta grego associado aos séculos
\textsc{viii} e \textsc{vii} a.~C., período em que, na Grécia Antiga, se sedimentou
uma série de fenômenos culturais e políticos (a cidade"-estado ou a \textit{pólis}, 
o oráculo de Delfos como santuário de
todos os gregos, os Jogos Olímpicos) que não só marcaram as sociedades
gregas nos séculos posteriores, mas também respondem pelas primeiras
associações que fazemos ao pensar nessa civilização. Entre esses fenômenos,
destaca"-se o politeísmo grego, ou seja, a base do poema em questão.
Assinale"-se, porém, desde agora, que o poema por si só não vale como uma
espécie de tratado teológico desse fenômeno – muito mais complexo, no que diz
respeito a práticas cultuais e tradições mitológicas, que a organização
sincrônica e diacrônica construída no poema.

Depois de alguns séculos em que, nos territórios onde se falavam diferentes
dialetos gregos, a escrita deixou de ser usada (aproximadamente, do final do
século \textsc{xii} a.~C. até o final do \textsc{ix} a.~C.), a introdução
paulatina e a adaptação de um alfabeto de origem fenícia motivou
a modificação de práticas sociais que a história e a arqueologia só
conseguem reconstruir com muitas e grandes lacunas. Assim, não sabemos nem
quando nem por que os poemas de Homero (\textit{Ilíada} e \textit{Odisseia}) e
os de Hesíodo\footnote{ Além de \textit{Teogonia} e \textit{Trabalhos e dias}, outros
poemas foram atribuídos a ele; desses, só \textit{Escudo de Héracles} chegou
completo até nós.} receberam sua primeira redação.

\section{performance e autoridade do poeta}

Fato é que, não só nos séculos \textsc{viii--vii} a.~C., mas também nos
seguintes, poemas mais ou menos musicados de toda sorte eram apresentados
oralmente em ocasiões específicas, várias delas associadas ao calendário
religioso ou certas práticas políticas da cidade. No que diz respeito à
poesia hesiódica, esse contexto de performance é em larga medida desconhecido
por nós, o que certamente representa empecilho para a interpretação do poema
que não deveria ser subestimado, já que a difusão dos poemas por meio da
leitura passou a ser mais praticada apenas vários séculos depois. Quando, como
e por que as primeiras plateias foram entretidas com a \textit{Teogonia} e
poemas semelhantes, disso é provável que jamais tenhamos notícias mais
precisas.

Certamente é significativo que o narrador da \textit{Teogonia} – ao contrário
do narrador dos poemas homéricos – se nomeie no início do poema (mas apenas uma
única vez), no momento mesmo em que é narrado seu encontro singular com a
entidade religiosa tradicional que confere autoridade a seu canto e garante a
precisão de seu conteúdo: as Musas. Os primeiros 115 versos do poema compõem um
proêmio, no qual se celebram essas divindades (1--103) e se demarca
explicitamente o conteúdo do canto a seguir (104--15). O trecho se assemelha a
uma forma poético"-religiosa tradicional em várias sociedades antigas: o hino
em honra de um deus, que, em algum momento na Grécia Antiga, ganhou uma versão
narrativa a partir da tradição épica – são os hinos homéricos longos ou médios.
O que há de muito particular nesse hino da \textit{Teogonia}, porém, é que
somente os gregos conheceram essas divindades coletivas responsáveis por uma
esfera cultural que podemos chamar de poesia, mas que envolvia também música e
dança. 

Ao celebrar as Musas antes de apresentar o canto que elas propiciam,\footnote{ A
cosmogonia e teogonia que começam no verso 116.} o poeta também fala da
relação que há entre ele próprio e essas divindades; pois o valor (de verdade),
ou seja, a autoridade do canto que apresenta depende dela. Como pode então
um mortal falar de eventos pretensamente reais que não presenciou, tal como o
surgimento do mundo conhecido e de todas as divindades, como dos mortais
que com elas dormiram, se não apresentar e fundamentar sua relação com certa
autoridade transcendente? Nesse sentido, não é mais possível, para nós,
saber com certeza se algum dia houve um poeta chamado Hesíodo, autor do poema
que conhecemos, ou se “Hesíodo” teria sido uma autoridade mítica inseparável de
uma certa tradição poética, reencarnada a cada apresentação do poema, um pouco
como o ator que reencarnaria, com uma máscara ritual, nas apresentações
teatrais atenienses no século \textsc{v} a.~C., as figuras tradicionais do
mito. Assim, a iniciação no canto, conduzida pelas Musas, pela qual teria
passado o poeta Hesíodo (9--34), também faria parte desse contexto mítico.

\section{zeus}

Outro elemento saliente no proêmio é Zeus. Como soberano dos deuses e dos
homens, deus responsável pela forma final do cosmo  
(físico e sociopolítico) e por sua manutenção, não é
raro ele desempenhar algum papel nos hinos aos deuses que conhecemos, sobretudo
nos hinos homéricos maiores. Sua presença nesse proêmio, porém, é ubíqua: não
só como pai das Musas e seu público primeiro e principal (não nessa ordem na
sequência do poema), mas também como o deus particularmente associado ao poder
político que, no mundo humano, é exercido pelos reis. Não surpreende, assim,
que as Musas estejam associadas não só aos poetas
(94--103), mas também aos reis (80--93), figura que, no contexto hesiódico, não
representa um monarca com amplos poderes, mas alguém que age sobretudo na
função de um juiz (Gagarin, 1992) na esfera
pública. O tipo de poder real exercido por Zeus no poema, absoluto e
hereditário, não é homólogo àquele dos líderes políticos da época. O rei
humano é antes de tudo um aristocrata com prestígio local que participa da
administração da justiça. Que reis e poetas, porém, são figuras dissociáveis,
isso fica claro no destaque dado a Apolo. De qualquer forma, o proêmio sugere
que os poetas são figuras bastante próximas dos reis.

\section{o abismo}

Para chegar a Zeus e o modo como este controla o cosmo, tema central do poema,
Hesíodo inicia do começo, ou seja, de Abismo (116), um espaço vazio cuja
delimitação primeira surge na sequência, Terra. Não se trata, porém, da Terra
tal qual a conhecemos, mas de uma espaço físico ainda descaracterizado, ou
melhor, marcado pela sua função futura: ser o espaço de atuação dos deuses
responsáveis pelo equilíbrio cósmico, que vai do Olimpo ao ínfero Tártaro.
Antes de Terra começar a gerar suas formas particulares (Montanhas e Mar) e das
divindades aparecerem, duas coisas fundamentais são necessárias: a presença de
Eros (120), desejo sem o qual não há geração, e as potências que permitem a
sucessão temporal, Escuridão, Noite, Éter e Dia (123--25).

Todos os deuses descendem de duas linhagens principais, a de Abismo e a de
Terra, mas entre elas não há nenhuma união. Os descendentes de Abismo são, em
sua maioria, potências cuja essência é negativa (Noite, Morte, Agonia etc.);
várias delas, além disso, expressam ações e emoções que permeiam os eventos
violentos narrados na sucessão de gerações da linhagem de Terra (Disputa,
Batalhas, Brigas etc.). A linhagem de Abismo, portanto, através da descendência
de Noite e Disputa, revela que a separação entre Terra e Abismo nunca é total
(as ações e emoções representadas como descendência de Abismo são executadas ou
sentidas pelos descendentes de Terra). Isso ilustra uma constante no
poema: o encadeamento das linhagens entre si e também delas com as histórias
que se sucedem mostra um poema no qual os catálogos dos deuses nascentes e as
narrativas em que os deuses estão envolvidos não devem ser separados. Trata"-se
de uma articulação de imagens e ideias que pressupõe uma temporalidade própria,
na qual se mesclam o tempo da narrativa genealógica, o tempo da sucessão de um
deus"-rei e o tempo da narração. É a partir disso que o leitor
deve entender, por exemplo, o fato de que um deus às vezes aparece como
personagem no poema antes do narrador mencionar seu nascimento propriamente
dito.

\section{genealogias divinas}

No poema, teogonia e cosmogonia são inseparáveis à medida que o espaço se
constitui e as genealogias divinas se sucedem. As divindades que passam pelo
poema – mais de 300 – são de diversos tipos no que diz respeito a cultos e
mitos: (1) os deuses do panteão (sobretudo os Olímpicos, como Zeus, Apolo,
Atena e Ártemis), cultuados pela Hélade, mas de uma forma mais específica que
aquela com que aparecem no poema (por exemplo, vinculados a um certo lugar ou
templo); (2) deuses presentes nas histórias míticas, com papel maior ou menor
nelas, mas que provavelmente nunca foram exatamente objetos de culto (Atlas e,
provavelmente, os Titãs); (3) partes do cosmo divinizadas (Terra, Noite,
Montanhas; algumas eram cultuadas); (4) personificações (elementos que, para
nós, são abstratos, mas não o eram para os gregos); (5) aqueles sobre os quais
nada sabemos fora de Hesíodo, ou seja, podem ser parte de um recurso típico
dessa tradição, que permitiria a “criação” de divindades para compor catálogos
ou expressar características de uma linhagem.

Essa tipologia, porém, não deve ser tomada como algo estático e invariável.
Eros, por exemplo, pode ser pensado como um deus de culto ou não. Como já
afirmado anteriormente, o próprio poema não é um retrato de uma estrutura
religiosa fixa. Pelo contrário, ele e sua tradição deveriam ser
antes pensados como uma tentativa de enquadrar, de dar uma certa forma a uma
vivência religiosa que é essencialmente plural no tempo e no espaço. O lance
astuto incorporado pela tradição é justamente procurar apresentar como um
sistema obviamente fixo algo que é necessariamente variável.

\section{afrodite}

Um dos modos do poeta expressar o que cada divindade tem de específico é a
derivação do seu nome e de seus epítetos. A construção mais desenvolvida diz
respeito a Afrodite, explorada quando se narra seu nascimento a partir do
esperma de Céu:

\settowidth{\versewidth}{cipriogênita, pois nasceu em Chipre cercado-de-mar;}
\begin{verse} Primeiro da numinosa Citera achegou-se,\\ e então de lá atingiu o
oceânico Chipre.\\ E saiu a respeitada, bela deusa, e grama em volta \\ crescia
sob os pés esbeltos; a ela Afrodite\\ espumogênita e Citereia bela-coroa\\
chamam deuses e varões, porque\\> na espuma [\textit{aphros}]\\ foi criada;
Citereia, pois alcançou Citera;\\ cipriogênita, pois nasceu em\\> Chipre
cercado-de-mar;\\ e ama-sorriso [\textit{philommedea}],\\> pois da genitália
[\textit{medea}] surgiu.\\ \end{verse} \attrib{Teogonia, 192--200}


Ora, à medida que o narrador, devido ao encontro que teve com as Musas, garante
estar falando a verdade, ao mostrar, através do próprio nome (aceito em toda a
Hélade) do deus, que as histórias que ele conta como que estão inscritas na
identidade verbal mesma do deus, ele confronta outras histórias – de outras
tradições – que não revelariam o mesmo conhecimento profundo e inequívoco da
realidade por ele dominado. A filiação da Afrodite de Homero – ela é filha de
Zeus e de Dione – como que sucumbe às “provas” dadas na \textit{Teogonia}, cuja
lógica só tem espaço para uma Afrodite, a filha de Céu.

O surgimento de Afrodite é um dos nascimentos que marcam o fim da supremacia de
Céu sobre o cosmo incipiente, ou seja, um momento de crise que antecede o
equilíbrio cosmológico verificado ainda hoje pelos ouvintes do poema no seu
cotidiano. Depois de Céu, também Crono, deus patriarca detentor do poder
soberano, é derrotado; somente Zeus, como rei dos deuses e homens, sempre tem
sucesso nos conflitos que enfrenta. O que os chamados “mitos de sucessão”,
fundamentais para o entendimento do poema, têm em comum é que a castração de
Céu, o nascimento de Zeus possibilitado pelo truque da pedra aplicado por Reia
e o combate de Zeus contra os Titãs e, posteriormente, Tifeu, guardam diversos
graus de semelhança com mitos equivalentes transmitidos por outras culturas
antigas do Oriente (babilônios, hurro"-hititas, hebreus etc.). O intercâmbio
verificado entre essas culturas problematiza, assim, a origem necessariamente
nebulosa, mas certamente não helenocêntrica, do poema (ou pelo menos de partes
dele). A maioria dos intérpretes concorda, hoje, que, de Homero e Hesíodo a
Platão, não deve ter havido nada parecido com um “milagre grego”, ainda que não
possamos sempre rastrear com precisão como teriam ocorrido os diversos casos de
intercâmbio entre as culturas orientais e a grega.

\section{terra e monstros}

Os mitos de sucessão são permeados por um par de opostos complementares
fundamental na mitologia, vale dizer, na cultura grega: “astúcia” e “força”. É
ele, por exemplo, que subjaz à oposição entre os heróis Odisseu e Aquiles; o
primeiro, o astuto por excelência, o segundo, o herói grego mais temido pelos
troianos devido à sua força. Também é essa oposição que mostra, em diversas
fábulas, animais mais fracos fisicamente derrotando os mais fortes ou velozes.
No caso da \textit{Teogonia}, desde o início a astúcia tem a particularidade de
ser uma característica essencialmente feminina. É de Terra o plano ardiloso que
permite a derrota de Céu; Farsa é filha de Noite; e Astúcia – além de Persuasão
– é uma das dezenas de filhas de Oceano.

Por outro lado, é a Terra que está ligada à geração dos seres que podemos
chamar de monstros (270--335): Équidna, Hidra de Lerna, Leão de Nemeia, Medusa,
Pégaso, Cérbero, Quimera etc. O que caracteriza tais criaturas como uma
coletividade é que elas não se assemelham nem a deuses, nem aos homens, nem aos
animais, mas são sempre seres estranhamente mistos, dotados – assim como sua
ancestral primeira – de um inominável, enorme poder, algo que faz deles seres
incapazes de serem conquistados pelos mortais, ou seja, “impossíveis”
(\textit{amêkhanos}). Os únicos que os venceram foram certos heróis, homens
muito superiores em certas qualidades do que os homens de hoje e que, além
disso, foram auxiliados por deuses.

Pela lógica da narrativa, os monstros parecem ser uma espécie de tentativa
malsucedida de continuar o desenvolvimento do cosmo (Clay, 2003), já que, em sua
maioria, não têm função alguma salvo contribuírem para a fama do herói que os
derrotou. Através deles, porém, mostra"-se que assim como, no plano humano,
mortais comuns se opõem a heróis, no divino, deuses se opõem a monstros. Além
disso, como notou Pietro Pucci (em Montanari, Rengakos \& Tsagalis), alguns
deuses da geração de Zeus utilizam, eles próprios, algum monstro para obter
determinado fim pessoal, o que sinaliza que o equilíbrio cósmico continua
instável.

\section{rios e oceaninas}

Como que a contrabalançar o peso negativo desses monstros, na sequência nascem
duas coletividades benfazejas, os Rios e as Oceaninas (337--70); e, entre essas
últimas, destacam"-se duas figuras femininas: Estige e Hécate (383--452). Ambas
aparecem na narrativa, de forma anacrônica, para serem cooptadas por Zeus, cujo
nascimento ainda não ocorreu. Isso se deve, como já foi mencionado acima, à
lógica própria do poema. As duas divindades femininas não só se opõem à
negatividade essencialmente feminina dos monstros, mas também preparam a
narrativa por vir. Estige mesma e seus filhos antecipam a vitória cósmica
de Zeus e o novo equilíbrio que ele vai instaurar e manter. Esse equilíbrio,
porém, não é resultado de uma tábula rasa, mas dá continuidade ao que já fora
equilibrado durante a supremacia de Crono. 

Hécate, por sua vez, é a deusa que permite a primeira irrupção mais substancial
dos humanos no poema. Como o objetivo do poema é revelar e celebrar o cosmo e os
deuses, é esperada a posição absolutamente marginal que o gênero humano ocupa
em relação a eles. Os homens e seu modo de vida são os protagonistas de outro
poema atribuído a Hesíodo, \textit{Trabalhos e dias}. Isso não significa que,
do ponto de vista dos próprios deuses, ou seja, da \textit{Teogonia}, as
características da fronteira que separa deuses e homens não sejam relevantes.
Essas aparecem com clareza em dois episódios que emolduram o nascimento de
Zeus, a celebração de Hécate e a história de Prometeu.

\section{heróis}

Se aos heróis – esses humanos (mortais) que estão no meio do caminho entre
deuses e homens – é dada uma razão de ser durante o catálogo de monstros, a
relação entre Zeus e Hécate, num momento do poema em que se enfatiza o
equilíbrio cósmico resultante das responsabilidades diversas atribuídas a cada
deus, revela que esse equilíbrio é indissociável da presença, na terra, dos
homens. A fim de se figurar o modo de
funcionamento do universo dos deuses através da sequência de eventos que a ele
levou, utiliza"-se também um retrato simplificado e razoavelmente genérico das
práticas cultuais humanas. Deuses, cosmo e homens não existem um sem o outro. O
trecho dedicado a Hécate, porém, revela também que a vida humana, mais que
marcada por um certo equilíbrio, é permeada pelo imponderável: por mais que os
homens propiciem os deuses, nada garante que serão auxiliados por eles. 

Não possuímos nenhum testemunho histórico independente da \textit{Teogonia} que
aponte para a importância cultual, mesmo que apenas local, de Hécate, sugerida
pelo destaque que lhe é dado no poema. Isso é um forte indício de que
comentadores como Clay (2003) estão corretos ao defender que a figura dessa
deusa é usada para se falar de Zeus e da relação entre os homens e os deuses
inaugurada por ele. Menos certa é a relação entre o nome de Hécate, a maneira
como o poeta se refere ao seu modo de atuação (“se ela quiser” etc.) e o acaso.

\section{prometeu e zeus}

O nascimento de Zeus narrado logo depois (453--91) é o evento que permite a
ascensão do terceiro soberano dos deuses. A astúcia de Terra é a responsável
pela castração de Céu, o nascimento dos seus filhos, os Titãs, e a tomada de
poder por parte do filho mais novo, Crono.\footnote{ Repare na importância que geralmente
têm os filhos que nascem por último nas linhagens do poema.} De forma homóloga,
é a astúcia da esposa de Crono, Reia, auxiliada pelos conselhos de Céu e Terra,
que permite que seus filhos vejam a luz do Sol e que Zeus destrone o pai. Dessa
vez, porém, há uma verdadeira competição entre astutos: como todo bom rei,
Crono é previdente, e, aprendendo com o erro de seu pai, decide engolir todos
os filhos paridos por sua esposa. Ela, porém, o ludibria no nascimento de Zeus,
de sorte que esse, através de uma série de manobras contadas rapidamente no
poema, pode ocupar a regência do cosmo. Ainda que, pelo menos em parte, nesse
momento da narrativa Zeus não seja representado como um agente deliberando
sozinho, seu poder é imediatamente ligado às esferas da
astúcia e da força. Por enquanto, sua astúcia ainda é aquela da mãe e da avó; sua
força, no entanto, está ligada ao seu primeiro ato como soberano, do ponto de
vista da lógica da narrativa: a libertação dos Ciclopes (501--6), aqueles que
lhe fornecerão os raios e o trovão, atributos que simbolizam seu poder e
apontam para sua ligação com o céu.

O primeiro conflito resolvido por Zeus envolve a astúcia (507--616). Trata"-se
do momento em que deuses e homens se distinguiram, se separaram em definitivo
por ocasião de um banquete festivo para o qual Prometeu repartiu a carne de um
boi. Marcam esse evento a origem do sacrifício, a conquista do fogo e a criação
da mulher. O texto não procura descrever detalhadamente essa raça humana que
não dominava o fogo, ainda compartilhava da companhia dos deuses e não conhecia
a reprodução sexual. 
O narrador deixa claro que Zeus aceita a repartição da carne do boi feita por
Prometeu para o banquete um vez que ele tinha em mente males destinados aos homens
mortais (551--52). Os homens, talvez guerreiros gigantes nascidos da terra
(interpretação possível a partir dos versos 50 e 185-87), em conluio com
Prometeu, representam, para Zeus, uma ameaça que precisa ser dominada antes que
seja tarde demais, pois, nesse momento de sua regência, Zeus ainda precisa
consolidar seu poder. 

Outro momento fundamental da história de Prometeu é a criação da primeira
mulher. Ao contrário do que ocorre em \textit{Trabalhos e dias}, aqui o
narrador não informa seu nome (lá, Pandora). Como em todos os eventos que
marcam o episódio de Prometeu, bem e mal estão indissociavelmente ligados. 
Segundo Vernant (1992), os ossos que não podem ser digeridos na divisão do boi
do primeiro sacrifício são encobertos pela gordura que solta delicioso aroma,
ao passo que a carne é disfarçada sob o repelente estômago. 
\footnote{ Assinale"-se que o
disfarce – e, consequentemente, a habilidade de reconhecer o que está
disfarçado – também é do domínio da astúcia). Os ossos (e o aroma da gordura
queimada.} 
Os ossos e o aroma da gordura queimada são um sinal da imortalidade divina, ao passo que a carne deliciosa
(o alimento perecível) comida pelos homens aponta para sua mortalidade; a
adoção da carne na sua dieta, escondida no estômago do boi, deixa claro que os
homens são escravos de seu próprio estômago e precisam satisfazê"-lo se não
quiserem perecer. 

No caso da Mulher, ela é dada aos homens em troca do fogo: se o fogo permite
que os homens sejam civilizados e não comam carne crua, a mulher terá que ser
por eles alimentada, caso queiram sobreviver por intermédio de um herdeiro. De
fato, fogo e mulher precisam ser constantemente alimentados para que o homem
não pereça. O sacrifício, o fogo e a bela mulher, portanto, indicam que há
elementos que apontam para uma certa presença do divino no seio da vida humana,
mas eles são tão tênues como a fumaça que sobe do sacrifício para o céu e tão
artificiais quanto os enfeites da coroa da primeira mulher, contra a qual o
homem não tem defesa alguma.

\section{titanomaquia}

Após essa separação entre deuses e homens levada a cabo por Zeus graças à
astúcia, a separação seguinte, entre os deuses da geração de seu pai, os Titãs,
e os da sua própria, os Olímpicos, é conseguida graças a uma supremacia
alcançada na esfera da força. O episódio conhecido como Titanomaquia (617--720)
mostra que o cosmo ficou mais complexo que quando sobre ele regia Céu, pois se,
para vencer seu pai, num primeiro momento Zeus contou com pelo menos dois ardis
arquitetados pela mãe e pela avó – entregar a Crono uma pedra no lugar do bebê
Zeus e, posteriormente, fazê"-lo vomitar todos os irmãos de Zeus que com ele
lutarão contra os deuses mais velhos –, num segundo momento a astúcia não será
suficiente.

É de novo Terra quem aconselha ao neto libertar aqueles que haviam sido presos
por Céu e assim mantidos por Crono abaixo da terra, os Cem"-Braços. Trata"-se
de uma força descomunal que os dois soberanos anteriores acharam por bem
simplesmente manter paralisada. Zeus, porém, consegue convencê"-los a serem
seus aliados e eles se mostram decisivos no combate contra os Titãs, gratos por
serem trazidos de volta à luz.

Luz e trevas marcam, como polaridades, toda a Titanomaquia, pois os Titãs, uma vez
vencidos, ocupam o espaço subterrâneo onde antes estavam os Cem"-Braços -- que
agora têm uma honra, uma função no cosmo, a de serem os eternos guardas dos
deuses outrora poderosos. Essa polaridade também prepara o episódio seguinte,
pois o esforço de Zeus para vencer os Titãs como que coloca o cosmo de volta ao
seu estado inicial. Terra, céu, mar e Tártaro: todos os espaços são atingidos
pelo fogo dos raios de Zeus, o que representa uma recriação do mundo por meio
da força. Não é por acaso que Abismo volta à cena (700 e 814) e que as imagens
e sons desse conflito cataclísmico são amplificados para o ouvinte por meio de
uma imagem que remete à união primordial entre Terra e Céu (700--5).

Uma vez finalizada a guerra, o narrador nos narra, pela primeira vez, como é a
geografia das terras ínferas (721--819). Não que antes nada lá houvesse. Mas,
com o aprisionamento dos Titãs, a essa parte do cosmo é conferida sua
estabilidade e Zeus pode finalmente aparecer como o organizador último de todos
os espaços. É por essa razão que de deuses como Sono e Morte e Noite e Dia,
cujas funções cósmicas os ligam ao Tártaro, finalmente se fala mais longamente,
mais uma vez mostrando de que forma polos positivos e negativos da realidade
estão interligados. É precisamente por isso que também nesse momento do poema
descreve"-se a função de Estige, ligada \mbox{a uma} jura divina que, quando quebrada
por um deus, o leva a uma morte virtual por dez anos. A ligação entre Estige e
Zeus mostra que também o juramento – uma instituição social fundamental também
entre os homens – é instituído pelo rei dos deuses e homens para bem
administrar o mundo divino, onde conflitos não são excepcionais.

\section{zeus e tifeu}

Curiosamente, porém, Zeus ainda terá que enfrentar mais um conflito armado, a
luta contra Tifeu (820--80). Por um lado, como nos dois poemas épicos que
conhecemos, a \textit{Ilíada} e a \textit{Odisseia}, o maior herói se revela
quando derrota um inimigo poderoso com suas próprias mãos. Por outro lado, esse
inimigo é, estranhamente, filho do próprio Tártaro com Terra. Que a fertilidade
exacerbada desta tenha gerado um ser para destronar o novo senhor do cosmo,
isso não surpreende, pois a eminência parda feminina foi peça fundamental na
deposição de Céu e Crono; que aquele seja o pai, isso sim é curioso, pois, até
esse momento da narrativa, dele apenas se falou como um espaço. É como se, pela
lógica da narrativa hesiódica, só agora ele tivesse adquirido o estatuto pleno
de divindade e precisasse se envolver em um conflito que garantisse que sua
forma não se alteraria.

Tifeu, por sua vez, adquire, graças à lógica da narrativa, o lugar de filho de
Zeus, pois o rei anterior fora deposto por seu filho (sempre ligado à Terra). O
conflito contra os Titãs, porém, já mostrou que a manipulação da astúcia e da
força, no grau superlativo em que o faz Zeus, não deixa espaço para a
possibilidade de derrota, mesmo que o adversário também seja \mbox{muito forte} –
Tifeu tem cabeças com olhos de onde sai fogo – e muito astuto – suas cem
cabeças produzem todo tipo de som, sendo que a metamorfose é um elemento mítico
típico do universo da astúcia. Além disso, esse combate singular entre a
criatura monstruosa e Zeus também permite que Terra, derradeiramente, seja
derrotada e esterilizada. O fogo de Zeus como que a derrete: de criadora de
metal e artífice metalúrgica, Terra como que se transforma, graças ao fogo
aniquilador de Zeus, no metal que é manipulado por artesãos machos (861--67).

\section{zeus e as esposas}

Uma vez derrotada a astuta Terra, que imediatamente torna"-se aliada de Zeus
(891), a primeira providência do soberano é casar e devorar sua esposa,
Astúcia, de sorte que essa gerasse um deus macho mais forte que ele
(886--900). Leonard Muellner (1996) mostrou como esse episódio arremata todos os
conflitos dinásticos narrados até então. Zeus não devora seu primeiro filho ou
obriga que sua esposa o guarde no ventre, mas assimila o elemento feminino em
si mesmo, Astúcia-Atena. Com isso, Zeus se torna um andrógino perfeito, e não
um disforme emasculado. A astúcia revela"-se mais uma vez essencialmente
feminina, mas para sempre assimilada pelo próprio rei. A filha produzida pelo
rei não só não é um macho – e foram sempre jovens machos que derrotaram seus
pais –, mas é uma virgem, ou seja, uma deusa que não irá produzir ameaça ao
\textit{status quo}. Por fim, ao ingerir a esposa grávida do primeiro filho, ele
bloqueou a previsão de que, depois de Atena, Astúcia geraria um filho mais
forte que o pai. Pela primeira vez, o rei dos deuses consegue “desparir” de
forma perfeita e acabada.

E somente agora nasce, de Zeus e várias de suas esposas, a linhagem dos deuses
responsáveis pelo que há de bom no cosmo (901--17): Norma, Decência, Justiça,
Paz, Radiância, Alegria, Festa e Musas, notável prole antípoda aos filhos de
Noite e Disputa. A última esposa de Zeus, Hera, é aquela que, de acordo com a
lógica do poema, representa a maior ameaça a Zeus. Mas tanto o filho mais
perigoso que os dois têm juntos quanto aquele que, como que emulando Zeus no
caso de Atena, Hera tem sozinha, Hefesto, não representam adversários fortes o
suficiente contra a filha que mais se assemelha ao pai e está completamente
alinhada com ele, Atena, senhora da guerra mas também da astúcia (921--29).

\section{catálogo de casamentos e filhos}

É nesse sentido que se deve entender o longo catálogo que finaliza o poema e
que tem três partes: os casamentos de Zeus e os filhos deles resultantes
(901--29); um catálogo mais abrangente de casamentos divinos (930--61), que
revelam, de forma sumária, um panteão muito bem organizado e potencialmente
harmônico (como que servindo de epítome, o casamento entre Ares e Afrodite
produz, por um lado, os machos Afugentador e Susto, mas, por outro, Harmonia);
e finalmente um catálogo de deusas que se uniram a mortais (962--1020). Ora,
com as deusas fêmeas que se unem a machos mortais, o princípio de ruptura por
excelência ao longo do poema agora desloca"-se ao mundo dos homens, mais
precisamente, ao mundo dos heróis. É nesse mundo que filhos serão mais fortes
que os pais, podendo, no limite, o que atesta Telégono, filho de Circe e
Odisseu, matá"-los.

Para concluir, menciono que há uma discussão sem fim sobre o verso em que a versão
“original” da \textit{Teogonia} teria terminado. Autores como Jenny Clay e
Adrian Kelly mostraram que os catálogos analisados compõem um final muito
adequado ao poema. Assim, provavelmente somente os quatro ou possivelmente dois
últimos versos foram acrescentados ao poema em um certo momento de sua
transmissão, para introduzir um outro poema atribuído a Hesíodo, o
\textit{Catálogo das mulheres} (obra que chegou a nós através de fragmentos),
que procurava dar uma visão geral da idade dos heróis a partir das mulheres que
com deuses dormiram por toda a Grécia. Esse catálogo era, possivelmente,
concluído pelo catálogo de pretendentes de Helena, cujo casamento redundou no
grande cataclisma que foi a guerra de Troia, que metonimicamente podia ser
pensada, na Antiguidade, como o fim da época dos heróis.


 \section{da tradução}
  
  
  O texto aqui traduzido é, basicamente, aquele proposto por Martin L. West.
  Algumas soluções de outros filólogos, porém, foram incorporadas, sobretudo as
  de Glenn Most, contrário a um excesso de correções ao texto transmitido.
  
  O principal problema para o tradutor da \textit{Teogonia} diz respeito aos
  nomes das divindades. Não se buscou nenhum tipo de padronização muito rígida,
  ou seja, ficou"-se entre os extremos de traduzir (quase) todos os nomes e
  (quase) nenhum nome. De forma geral, os principais critérios foram o bom
  senso, o conhecimento do leitor e a sonoridade.
  
  Algo do que se perde na tradução –- por exemplo, figuras etimológicas –- é
  recuperado nas notas.
  
  Por questões didáticas, utilizou"-se dois tipos de sinais para indicar 
  a divisão de unidades narrativas. O recuo de parágrafo é equivalente 
  a um novo parágrafo em uma narrativa
  em prosa; o sinal \P{} indica uma pausa maior no contexto de um
  uma \textit{performance} oral. Não é possível saber, entretanto, se tais
  marcações perpassaram tais performances orais do poema.

  A numeração das notas de rodapé em forma de lemas segue o número 
  que indica um verso ou um conjunto de versos do poema.

\begin{bibliohedra} \tit{ALLAN}, W. (2006) “Divine justice and cosmic order in
early Greek Epic.” Journal of Hellenic Studies 126: 1--35
  
\tit{ARNOULD}, D. (2009) “Les noms des dieux dans la Théogonie d’Hésiode:
étymologies et jeux de mots”. Revue des études grecques 122: 1--14
  
\tit{BLAISE}, F.; \textsc{judet de la combe}, P.; \textsc{\textsc{rousseau}},
P. (org.) \textit{Le métier du mythe: lectures d’ Hésiode}. Lille: Presses
Universitaires du Septentrion
  
\tit{BOYS-STONES}, G. R.; \textsc{haubold}, J. H. (2010) \textit{Plato and
Hesiod}. Oxford: Oxford University Press
  
\tit{BRANDÃO}, J. L. (2000) “As Musas ensinam a mentir''. (Hesíodo, Teogonia,
27--28). Ágora 2: 7--20
  
\tit{CLAY}, J. S. (2003) \textit{Hesiod’s cosmos}. Cambridge: Cambridge
University Press
  
\tit{DETIENNE}, M.; \textsc{vernant}, J.-P. (2008) \textit{Métis: As astúcias
da inteligência.} Trad. Filomena Hirata. São Paulo: Odysseus
  
  
\tit{GAGARIN}, Michael (1992) “The poetry of justice: Hesiod and the origins of
Greek law” Ramus 21: 61--78
 
\tit{HESIOD} (2006) \textit{Theogony, Works and days, testimonia}. Edited and
translated by Glenn W. Most. Cambridge, Mass.: Harvard University Press

\tit{HESÍODO} (2013) \textit{Trabalhos e dias.} Tradução, introdução e notas:
C. Werner. São Paulo: Hedra
  
\tit{JANDA}, M. (1997) \textit{Über ‘Stock und Stein’: die indogermanischen
Variationen eines universalen Phraseologismus.} Röll: Dettelbach
  
\tit{JANKO}, R. (1982) \textit{Homer, Hesiod and the Hymns.} Cambridge:
Cambridge University Press
  
\tit{KELLY}, A. (2007) “How to end an orally-derived epic poem?” Transactions
of the American Philological Association 137: 371--402
  
\tit{KONING}, H. (2010) \textit{Hesiod: The other poet. Ancient reception of a
cultural icon.} Leiden: Brill
  
\tit{LAMBERTON}, R. (1988) \textit{Hesiod.} New Haven: Yale University Press
  
\tit{LECLERC}, M.-C. (1993) \textit{La parole chez Hésiode: à la recherche de
l’harmonie perdue.} Paris: Belles Lettres
  
\tit{LEDBETTER}, G. M. (2003) \textit{Poetics before Plato: interpretation and
authority in early Greek theories of poetry.} Princeton: Princeton University
Press
  
\tit{MACEDO}, J. M. (2010) \textit{A palavra ofertada: um estudo retórico dos
hinos gregos e indianos.} Campinas: Edunicamp
  
\tit{MARTIN}, R. P. (1984) “Hesiod, Odysseus, and the instruction of princes”.
TAPA 114: 29--48
  
\tit{MONTANARI}, F.; \textit{rengakos}, A.; \textit{tsagalis}, C. (2009)
\textit{Brill’s companion to Hesiod.} Leiden: Brill
  
\tit{MUELLNER}, L. C. (1996) \textit{The anger of Achilles: mênis in Greek
epic.} Ithaca: Cornell University Press
  
\tit{NAGY}, G. (1990) “Hesiod and the poetics of Pan-Hellenism”. In:
\textit{Greek mythology and poetics.} Ithaca: Cornell University Press 
  
\tit{PUCCI}, P. (1977) \textit{Hesiod and the language of poetry.} Baltimore:
Johns Hopkins University Press
  
\titidem{}. (2007) \textit{Inno alle Muse (Esiodo, Teogonia, 1--115)}: texto,
introduzione, traduzione e commento. Pisa: Fabrizio Serra
  
\tit{RIJKSBARON}, A. (2009) “Discourse cohesion in the proem of Hesiod’s
Theogony”. In: \textsc{bakker}, S.; \textsc{wakker}, G. (org.) Discourse
cohesion in Ancient Greek. Leiden: Brill
  
\tit{ROCHA}, Júlio C.; \textsc{rosa}, André H.; \textsc{ribeiro} Jr.,
\textsc{Wilson} A. (2011) \textit{Hinos homéricos: tradução, notas e estudo.}
São Paulo: Edunesp
  
\tit{ROWE}, C. J. (1983) “ ‘Archaic thought’ in Hesiod”. Journal of Hellenic
Studies 103: 124--35
  
\tit{SERRA}, O. (2006) \textit{Hino Homérico a Hermes \textsc{iv}.} São Paulo:
Odysseus
  
\tit{SNELL}, B.  (2001) “O mundo dos deuses em Hesíodo”. In: \textit{A cultura
grega e as origens do pensamento.} São Paulo: Perspectiva
  
\tit{TORRANO}, J. (1992) \textit{Hesíodo: Teogonia. A origem dos deuses.}
Estudo e tradução. 2a edição. São Paulo: Iluminuras
  
\tit{VERDENIUS}, W. J. (1972) “Notes on the proem of Hesiod’s Theogony”.
Mnemosyne 25: 225--60
  
\tit{VERNANT}, J.-P. (1992) \textit{Mito e sociedade na Grécia antiga.} Rio de
Janeiro: José Olympio
  
\tit{VERSNEL}, H. S. (2011) \textit{Coping with the gods: wayward readings in
Greek theology.} Leiden: Brill
  
\tit{WEST}, M. L. (1966) \textit{Hesiod, Theogony: Edited with prolegomena and
commentary.} Oxford: Oxford University Press
  
\tit{WHEELER}, G. (2002) “Sing, Muse\ldots{}: the introit from Homer to Apollonius”.
Classical Quarterly 52: 33--49
  
\tit{WOODWARD}, R. D. (2007) “Hesiod and Greek myth” In: \textsc{woodward}, R.
D. (org.) \textit{The Cambridge companion to Greek mythology.} Cambridge:
Cambridge University Press
  
\end{bibliohedra}

