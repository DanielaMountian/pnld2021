\documentclass[11pt]{hedrabook}
\usepackage[brazilian]{babel}
\usepackage{ucs}
\usepackage[utf8x]{inputenc}
\usepackage[T1]{fontenc}
\usepackage{hedracrop}
\usepackage{hedrabolsolayout}
\usepackage[protrusion=true,expansion]{microtype}
\usepackage{comment,lipsum,footmisc}
\usepackage[minionint,mathlf]{MinionPro}

\makeatletter
  \renewenvironment{frontispiciopage}[2]{
    \thispagestyle{empty}
    \def\@ano{#2}
    \vspace*{\stretch{1}}
    \begin{fancytable}{#1}%
      \putline{\hspace*{36mm}}{}
      \ThirdpageAuthorCathegory               %comentar quando não houver AUTOR
      \putline{T\'itulo}{\textsc{\@title}}}{
      \putline{S\~ao Paulo}{\@ano\vspace{5mm}}
      \bigline{}{}{
        \setlength{\unitlength}{1mm}%
        \hspace{1mm}\begin{picture}(25,25)(0,2.2)%
          %\color{white}
          \put(0, 0){\line(1, 0){2}}
          \put(0, 0){\line(0, 1){2}}
          \put(23, 0){\line(1, 0){2}}
          \put(25, 0){\line(0, 1){2}}
          \put(0, 25){\line(1, 0){2}}
          \put(0, 23){\line(0, 1){2}}
          \put(23, 25){\line(1, 0){2}}
          \put(25, 23){\line(0, 1){2}}
          \put(5.8,13){\makebox(0,0)[l]{\fontencoding{OT1}\fontfamily{ptm}\selectfont\logosize{}hedra}}%
        \end{picture}}
      \bigline{}{%
        \setlength{\unitlength}{\baselineskip}%
        \begin{picture}(0,0)(0,0)%
          \put(0,1){\makebox(0,0)[l]{\interno@traco}}%
          \put(0,2.5){\makebox(0,0)[l]{\interno@traco}}%
          \put(0,3.5){\makebox(0,0)[l]{\interno@traco}}%
          \put(0,5){\makebox(0,0)[l]{\interno@traco}}%
          \put(0,8.7){\makebox(0,0)[l]{\interno@traco}}%
        \end{picture}}{%
        \setlength{\unitlength}{\baselineskip}%
        \hspace{\stretch{1}}\begin{picture}(0,0)(-4,0)%
          \put(-4,1){\makebox(0,0)[l]{\interno@traco}}%
          \put(-4,3){\makebox(0,0)[l]{\interno@traco}}%
          \put(-4,4){\makebox(0,0)[l]{\interno@traco}}%
          \put(-4,10){\makebox(0,0)[l]{\interno@traco}}%
          \put(-4,11){\makebox(0,0)[l]{\interno@traco}}%
          \put(-4,12){\makebox(0,0)[l]{\interno@traco}}%
          \put(-4,16){\makebox(0,0)[l]{\interno@traco}}%
          \put(-4,17){\makebox(0,0)[l]{\interno@traco}}%
          \put(-4,21){\makebox(0,0)[l]{\interno@traco}}%
          \put(-4,23){\makebox(0,0)[l]{\interno@traco}}%
          \put(-4,24){\makebox(0,0)[l]{\interno@traco}}%
        \end{picture}}
    \end{fancytable}}
\makeatother

\AtBeginDocument{%
  \selectlanguage{brazilian}
  \fontsize{11pt}{13.2pt}\selectfont
  \parskip=0pt
  \setlength{\unitlength}{1mm}%
  \setcounter{secnumdepth}{-2}%
  \pagestyle{plain}}


\begin{document}

\author{Johann Wolfgang von Goethe}
\title{Os sofrimentos do jovem Werther}
\begin{frontispiciopage}{4cm}{2013}
  %comentar/modificar linhas abaixo conforme necessário
%  \putline{Tradução}{\textsc{}}
%  \putline{Introdução}{\textsc{}}
%  \putline{Organização e tradução}{\textsc{}}
\end{frontispiciopage}


\section{Sobre o autor} 

Johann Wolfgang von Goethe (Frankfurt am Main, 1749 --- Weimar, 1832)
licenciou-se em direito por insistência de seu pai, jurista renomado, e
nunca chegou a exercer plenamente a profissão. Aos 25 anos, com a
publicação de \textit{Os sofrimentos do jovem Werther}, Goethe
tornou-se imediatamente célebre. Um ano depois, a convite do duque
Carlos Augusto, mudou-se para Weimar, onde fixou residência até o fim
da vida, com raras ausências, como a viagem à Itália de 1786 a 1788. Os
inúmeros poemas, dramas, romances, traduções, ensaios sobre arte e
natureza que produziu definiram o cenário cultural da época, que passou
a ser conhecida como \textit{Goethes Zeit}, “tempo de Goethe”, em que
ocorreram os movimentos clássico e romântico, além de Tempestade e
Ímpeto (\textit{Sturm und Drang}). Goethe foi grande poeta dramático
com seu \textit{Fausto} (\textit{Faust}), concluído em 1808, e
considerado seu texto mais relevante, grande narrador com \textit{Os
anos de aprendizado de Wilhelm Meister}, grande poeta em sua extensa
produção lírica, memorialista em \textit{Poesia e} verdade. Em vida,
foi mais conhecido como autor de \textit{Werther}. Ao lado de Schiller,
Goethe permanece o grande poeta da língua alemã. 

\section{Sobre o enredo}

O livro conta a história da paixão de Werther por Carlota. Esta, desde o
momento em que ele a conheceu, era uma mulher comprometida, que se
casaria em breve com Alberto. Werther, entretanto, não se dobra às
convenções sociais, ignora tais obstáculos e se deixa levar pelo desejo
de possuir Carlota, que também gosta dele, e às vezes parece-lhe que
tal adoração vai além da amizade e identificação espiritual que os une.


O livro é dividido em duas partes e cada parte é composta por cartas (36
e 39, respectivamente), enviadas entre 4 de maio de 1770 e 17 de
dezembro de 1771. Há ainda uma parte chamada “Do editor ao leitor”, que
inclui por meio de cartas jamais enviadas e de depoimentos de Alberto e
Carlota. 

Werther é um personagem de matiz sonhador e furioso, narrador em
primeira pessoa de sua própria experiência, que o levaria a um abismo,
e que ele não foi capaz de evitar. Werther recusa toda análise fria de
sua experiência amorosa, porque ele não está interessado em ordenar sua
vida interior, mas quer se entregar aos devaneios íntimos ocasionados
por seu desejo insatisfeito e improvável de realizar-se. Órfã de mãe,
Carlota é uma moça que prometeu cuidar de seus pequenos irmãos e de seu
pai. Ela está comprometida com Alberto, e esse compromisso foi
abençoado pela mãe, que é, para Carlota, uma figura venerável, quase
santificada. Assim, Carlota encarna um ideal de pureza, devoção e amor
desinteressado. Ela corresponde, em certa medida, ao sentimento de
Werther. Já Alberto é a imagem do equilíbrio, e forma uma oposição com
Werther. Ele dedica a Carlota um amor convencional, respeitoso e
comprometido. Por isso, Werther não encontra muitos meios de criticá-lo
como parceiro amoroso, exceto pelo caráter médio, protocolar e adequado
de seu amor. Guilherme é o destinatário das cartas de Werther, por meio
das quais tomamos conhecimento de seu drama, através do editor, que não
sabemos se é o próprio Guilherme. É Guilherme quem aconselha Werther a
trabalhar como secretário na embaixada. Também o aconselha a abandonar
suas pretensões amorosas e conformar-se com a condição comprometida de
Carlota. O relato não inclui as cartas de Guilherme, mas sabemos de
algumas de suas opiniões porque Werther as repete.

Vale a pena indicar aos alunos algumas cenas particularmente
importantes. O primeiro encontro com Carlota (carta 10, pp. 39-41), que
será tão decisivo para o desenvolvimento da narrativa. A discussão com
Alberto, a respeito do suicídio (carta 29, pp. 68-75). A cena do
alagamento (carta 74), que exemplifica o desejo de Werther de
desfazer-se na Natureza. O encontro com o desvairado (carta 70), cujo
destino antecipa o de Werther. O período como funcionário da embaixada,
com a cena em que pedem para Werther retirar-se de uma recepção social 
(carta 40), porque ela mostra a dificuldade do personagem para
integrar-se e cumprir as convenções sociais. A leitura das traduções de
Ossian, feitas por Werther (“Do editor ao leitor”, 142-151), porque
mostram todo o romantismo do personagem, quando traz à narrativa esses
cantos esses cantos de antigos poetas gaélicos, isto é, das ilhas da
Grã-Bretanha. 

A narrativa de Goethe trata da urgência, da exigência de deixar correr
livremente as aspirações afetivas dos homens, que deveriam libertar-se dos
condicionamentos sociais, educativos e morais em favor de uma experiência
ilimitada da própria interioridade. Daí a frequência com que o personagem
declara-se incapaz de expressar aquilo que sente, indicando que sua vida
interior excede os limites racionais da expressão letrada. 


\section{Sugestões de atividades} 

\paragraph{1.}

A narrativa de Goethe é epistolar, isto é, uma narrativa feita por
meio de cartas. Há outros livros que se utilizaram da mesma estratégia,
como \textit{A nova Heloísa}, de Rousseau, e \textit{As relações
perigosas}, de Chordelos de Laclos. Após a leitura do livro, propor aos
alunos a redação da carta de Guilherme à qual Werther responde com sua
carta de número 67. O texto desta carta contém indicações do que teria
dito Guilherme em sua última carta. É importante limitar a extensão da
atividade, por exemplo, a 30 linhas, pouco mais ou menos. Também é
possível criar variações desta atividade, com cartas de Carlota a
Werther, de Werther a Alberto, ou mesmo de personagens inventados de
forma coerente, por exemplo, não se fala do pai ou do avô de Werther,
mas eles existiram.

\paragraph{2.}

\textit{Os sofrimentos do jovem Werther} deve muito de sua
importância ao fato de ser uma das primeiras obras de arte do
Romantismo. É importante instruir os alunos sobre as características do
Romantismo, de preferência em oposição às características do
Classicismo. Uma boa síntese de tais características é apresentada pelo
professor Benedito Nunes:

\begin{quote}
  “A essa concepção do mundo, preponderantemente \textit{metafísica} e
  \textit{idealista}, percorrida por um \textit{afã de totalidade e de
  unidade}, próprio da sensibilidade conflitiva que a impulsionou, e
  polarizada por \textit{sentimentos extremos e atitudes antagônicas},
  comportando uma \textit{vivência da Natureza física}, um \textit{senso
  do tempo} e um \textit{poder mitogênico}; a essa concepção do mundo,
  que separou do universo cultural a literatura e a arte,
  transformando-as na estância privilegiada de uma só atividade poética,
  supraordenada das correlações significativas da cultura,
  concomitantemente ligada à \textit{afirmação do indivíduo e ao
  conhecimento da Natureza}; a essa concepção do mundo corresponde o
  Romantismo estritamente considerado […].” (Nunes, Benedito, “A visão
  romântica”, em Ginsburg, J. (org), \textit{O romantismo}. São Paulo:
  Perspectiva, 1978, p. 53) 
\end{quote}

As expressões destacadas em itálico representam características
dominantes do Romantismo. Sugere-se aqui dividir a turma de alunos em
grupos de quatro, escrever as características destacadas em fichas e
distribuí-las aos alunos. Em seguida, cada grupo deve relacionar a
característica recebida a cenas, personagens ou aspectos da narrativa.
Espera-se como resultado que os alunos percebam como essas tendências
dominantes de época determina o desenho do enredo e do caráter dos
personagens. 

\paragraph{3.}

Para uma atividade de interpretação de texto, pedir que os alunos
leiam a seguinte passagem do romance:

\begin{quote}
  “Minha avó costumava contar-me uma história da montanha de imã: os
  navios que se chegavam muito àquela montanha, de repente se
  desguarneciam das ferragens; os pregos voavam ao monte, e os
  desgraçados marinheiros morriam afogados entre as pranchas
  desconjuntadas.” (p. 64)
\end{quote}

Este tipo de passagem é muito importante por antecipar, de maneira
figurada, desenvolvimentos do enredo. Os alunos devem, em uma atividade
oral, refletir sobre como se relacionam esta pequena história contada
por Werther e a trajetória do personagem. 

\paragraph{4.}

No trecho a seguir, pertencente à carta 15, Werther faz uma reflexão
a partir da experiência de leitura de contos de fada para os irmãos de
Carlota:

\begin{quote}
  “[...] estes contos servem-me de lição; fico surpreendido de ver a
  reação que causam às crianças. Quando me esqueço de alguma
  particularidade e que na segunda vez altero, eles me dizem: Não era
  assim que me contou a outra vez; de tal forma que me tenho habituado a
  recitar as minhas histórias em termos invariáveis e até com a mesma
  cadência.” (p. 76)
\end{quote}

Na visão romântica a Natureza representa um valor muito importante
porque o homem romântico interpreta a Natureza como manifestação
espontânea do Espírito do mundo. A essa ideia se relaciona uma noção de
arte segundo a qual ela deve ser também espontânea, como a Natureza.
Após pedir aos alunos a releitura silenciosa da carta 15, discutir como
as características do romantismo se relacionam à reflexão de Werther.



\section{Indicações bibliográficas}

\begin{description}\labelsep0ex\parsep0ex
\newcommand{\tit}[1]{\item[\textnormal{\textsc{\MakeTextLowercase{#1}}}]}
\newcommand{\titidem}{\item[\line(1,0){25}]}

\tit{Bosi}, Alfredo. \textit{História concisa da literatura brasileira}. 33ª
ed. São Paulo: Cultrix, 1994.

\tit{Candido}, Antonio. \textit{O Romantismo no Brasil}. 2ª ed. São Paulo:
Humanitas, 2004

\tit{Citelli}, Adilson. \textit{Romantismo}. São Paulo: Ática, 2007.

\tit{Cunha}, Cilaine Alves da. “Visões do Romantismo”. Revista Cult, nº 61. Na
internet:

\tit{}http://revistacult.uol.com.br/home/2010/03/visoes-do-romantismo-por-antonio-candido/.

\tit{Ginsburg}, J. (org), \textit{O Romantismo}. São Paulo: Perspectiva, 1978.

\tit{Safranski}, Rüdiger. \textit{Romantismo: uma questão alemã}, trad. Rita
Rios. São Paulo: Estação Liberdade, 2010.

\tit{Thompson}, E. P. \textit{Os românticos --- A Inglaterra na era
revolucionária}, trad. Sérgio Moraes de Rêgo Reis. Rio de Janeiro:
Civilização Brasileira, 2002.

\end{description}

\paragraph{Na internet}

\begin{description}\labelsep0ex\parsep0ex
\newcommand{\tit}[1]{\item[\textnormal{\textsc{\MakeTextLowercase{#1}}}]}
\newcommand{\titidem}{\item[\line(1,0){25}]}

\tit{}\noindent No site You Tube é possível encontrar muitas aulas sobre o Romantismo. 

\end{description}

\end{document}
