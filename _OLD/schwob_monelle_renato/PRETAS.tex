\SVN $Id: PRETAS.tex 8599 2011-03-17 18:09:18Z bruno $
\begin{resumopage}

\item[Marcel Schwob] (Chaville, 1867--Paris, 1905) foi ficcionista, ensaísta e tradutor francês. 
Com formação intelectual erudita, ocupou lugar de destaque nos meios literários parisienses
nos anos 1890, tendo convivido intimamente com escritores como Paul Claudel,
Guy de Maupassant, Jules Renard e Alfred Jarry, entre outros. Traduziu autores
latinos como Luciano de Samósata, Catulo e Petrônio, mas tinha especial
predileção por escritores de língua inglesa, como Defoe, Stevenson, Meredith e
Whitman. Entre suas obras mais importantes estão \textit{Cœur Double}
(Coração duplo, 1891), \textit{Le Roi au masque d’or} (O rei
da máscara de ouro, 1892), \textit{Le Livre de Monelle} (O livro de
Monnelle, 1894), \textit{La Croisade des enfants} (A cruzada das
crianças, 1896) e \textit{Vies imaginaires} (Vidas imaginárias, 1896).

\item[O livro de Monelle] (1894) é uma obra única, imune a classificações. Originalmente, os dezessete 
contos que formam a obra foram publicados entre 1892 e 1894 no \textit{L'Écho de Paris}, sendo reunidos num volume 
por Léon Chailley em 1894. \mbox{O livro} se organiza sob a forma de um tríptico, no qual Schwob cria uma mistura sutil de
gêneros: conto, poema em prosa e texto profético sob forma de versículos. Cada
uma de suas partes se organiza, por sua vez, em fragmentos independentes que se
diferenciam não apenas por sua forma, mas também por seus temas, unidos, no
entanto, por um fio condutor: Monelle, misterioso personagem feminino. \textit{O livro
de Monelle} é um livro de luto que fascinou, à época de sua
primeira publicação, nomes como Mallarmé e Anatole France. Uma das obras mais
conhecidas de Marcel Schwob, foi várias vezes adaptado para
teatro e rádio. Inédito em português.
        
\item[Claudia Borges Faveri] é professora de Literatura Francesa e de Teoria e
História da Tradução da Universidade Federal de Santa Catarina (\versal{UFSC}). Com
doutorado em Letras pela Universidade de Nice Sophia Antipolis --- França e
pós-doutorado em Literatura pela Universidade Federal de Minas Gerais,
dedica-se a pesquisas na área de Teoria e História da Tradução e Literatura
Francesa Traduzida.

\end{resumopage}

