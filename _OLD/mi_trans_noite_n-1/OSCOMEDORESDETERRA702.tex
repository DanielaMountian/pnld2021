\chapterspecial{{Os} {comedores} {de} {terra}}{}{}
 

 

\letra{E}{ssa é a história} dos nossos antepassados que aos poucos se
multiplicaram. Ela começa na época em que não havia Yanomami como os de
hoje. Os Comedores de Terra sofriam, porque eles comiam terra. Os
primeiros que surgiram sofreram. Nós também quase que teríamos sofrido,
como as minhocas, por cavar a terra e tomar vinho de barro, se não
fossem os acontecimentos que seguem. 

Só havia os Comedores de Terra. Eles não conheciam os alimentos que hoje
nos alimentam, apesar de serem muitos, como ingá, como \emph{naɨ},
como conori. Havia cabari, bacaba, mas eles não sabiam tomar vinho de
bacaba, tomavam vinho de barro e de flores; depois de cortá"-las lá em
cima, eles tomavam vinho delas. Devoravam as embaúbas novas, e as
chamavam de comida. Se nossos antepassados tivessem surgido nessa época,
nós estaríamos sofrendo hoje. 

Quem descobriu os alimentos comestíveis? Morava com eles Horonamɨ,
aquele cujo nome aparece no início, na origem. Ele mostrou a todos os
alimentos que até hoje nós comemos. Depois de perguntar, experimentar e
carregar os alimentos por todos os cantos, ele ensinou os Comedores de
Terra a comê"-los. Foi ele, Horonamɨ, não outro. Assim foi.

Tudo isso não aconteceu embaixo deste céu, mas do céu que caiu e amassou
os primeiros habitantes. Abriram o céu e assim nossos antepassados
surgiram. O céu caiu, mas antes ele estava lá em cima, antes da
existência dos nossos antepassados, antes de algum \emph{napë}, entre
nós, perguntar assim:

--- Tudo bem? 

 Eles morriam de fome, pois comiam terra, flores, frutas,
excrementos de minhoca, folhas novas de cabari. É essa a história dos
ancestrais. Os ancestrais no início não comiam os alimentos que comemos
hoje. Eles comiam a pasta que se forma nas árvores junto às casas de
cupim. Dizem que a comiam com gula. Apesar de só comerem isso, não
ficavam doentes, pois não existia malária, e não precisavam curar
ninguém, pois não havia doença, não havia dor, nem tosse, portanto não
havia necessidade de remédio --- não havia doença, pois não
havia \emph{napë}. Viviam bem, sem doenças, até terem muitos cabelos
brancos. As mulheres ficavam velhas até terem a cabeça branca, pois não
havia doença. 

Era assim, no início: não sofriam com conjuntivite, nem com feridas, nem
tinham marcas de furúnculo. Tinham a pele bonita e somente sofriam de
fome, por causa da terra que comiam. Apoiavam"-se em paus para andar, por
causa da fome. Assim era. Nessa época, não sabiam comer carne, mas eles
estavam bem e, quando um velho morria, ninguém chorava. Não choravam por
causa de um velho morrendo de doença, pois ninguém morria de doença. Nem
havia cobra para picar, dar dor e matar. Eles viviam bem. Os espíritos
não pegavam a alma de ninguém para matar. Era assim. Eles não ficavam
fracos com diarreia, isso não acontecia, apesar de eles não tomarem
remédios. 

Era assim quando não existia \emph{napë,} antes de os \emph{napë} se
misturarem; nessa época, os \emph{napë} existiam? Sabemos que não! Não
existiam. 

Os rios, apesar de serem grandes, dizem que eram vazios. Dizem que não
se escutava o som de motor subindo o rio fazendo ``Tu, tu, tu, tu, tu,
tu!''

``Ũ, ũ, ũ, ũ, ũ!'' Não se escutava o som do avião, por isso os velhos
não morriam de doença. Morriam de cegueira. Era assim que morriam, por
causa da cegueira. Tornavam"-se cegos e a respiração parava, não por
causa de doença, mas de fome. Isso só aconteceria depois. Aconteceu
assim. 

Ninguém dizia:

--- Alguém lá pegou doença e morreu; eles estão chorando lá! 

Mesmo quando tinham cabelos brancos, eles andavam saudáveis. Morriam de
velhice. Ficavam cegos, os olhos secavam, o sangue acabava, por isso
morriam. Mandavam deixá"-los mortos fora do xapono, para que voltassem
como mortos"-vivos nessa hora. Retornavam sempre na forma de
mortos"-vivos, quando não havia \emph{napë} entre eles. Os ancestrais
ficavam alegres por comer frutas, não era como agora. Quando comeram
carne, eles endoideceram e passaram mal. Não havia fogo e comiam cru.
Endoideceram por comer cru.

Depois de eles aprenderem a comer os verdadeiros alimentos, eles se
tornaram como nós. Tornaram"-se assim, comendo carne cozida. Quando
aconteceu, as crianças se multiplicaram, saudáveis, em um e outro
xapono\emph{..}. Fizeram um grande xapono, outros se agruparam, e não
pararam de se multiplicar, todos saudáveis.

 
