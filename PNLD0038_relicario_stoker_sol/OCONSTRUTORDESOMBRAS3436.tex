%!TEX root=LIVRO.tex
\chapterspecial{O Construtor de Sombras}{}{}
 

O solitário Construtor de Sombras sempre fica observando tudo em sua morada
solitária.

As paredes são de nuvem, e, em volta e através delas, sempre mudando de figura enquanto vão, passam as sombras escuras de todas as coisas que já foram.

Esse círculo infinito, sombrio, rodopiante e movente é chamado de
\emph{A Procissão do Passado Morto}. Nela, tudo é tal como já foi no
grande mundo. Não há mudanças em parte alguma; pois cada momento, à
medida que passa, manda sua sombra para as fileiras dessa turva
Procissão. Aqui há pessoas que se movem e acontecimentos -- preocupações --
pensamentos -- tolices -- crimes -- alegrias -- tristezas -- lugares --
cenas -- esperanças e medos, e tudo isso perfaz a soma da vida com todas as
suas luzes e sombras. Cada imagem na natureza em que a sombra mora -- e
isso são todas as coisas -- aqui tem seu fantasma obscuro.

\imagemmedia{}{./img/07.png}


Aqui, todas as imagens mais belas e mais tristes de se ver -- a
escuridão que permeia um milharal ensolarado quando com a brisa aparece o
balanço escuro das espigas cheias se dobrando e se endireitando; a onda na
superfície vítrea de um mar de verão; a vastidão escura que jaz além e
fora da ampla trilha da luz da lua na água; a renda de brilho e de
sombra que cintila sobre a estrada à medida que se passa por ela no
outono quando a luz da lua cai através dos galhos nus das árvores
pendidas à margem; a sombra fresca e tranquila sob as grossas árvores no
verão quando o sol está flamejando acima do preparador de feno
trabalhando; as nuvens negras que esvoaçam atravessando a lua, escondendo
sua luz, que depois reaparece vazia e fria; o fusco do violeta e do
preto que se alça no horizonte quando no verão a chuva se aproxima; os
recessos escuros e as cavernas sombrias de onde as cachoeiras chiando
se arrojam ao lago abaixo --, todas essas imagens sombrias, e mais mil
outras que chegam dia e noite, circulam na Procissão entre as coisas que já
foram.

Aqui, também, cada ato que qualquer humano realize, cada pensamento --
bom ou mal -- cada desejo, cada esperança -- tudo o que é secreto --
está retratado, e se torna um registro duradouro que não pode ser
destruído; pois, a qualquer momento, o Construtor de Sombras pode
incitar, com sua mão espectral, qualquer um -- dormindo ou acordado -- a
contemplar o que é retratado do Passado Morto, na distância obscura e
misteriosa que abarca sua morada solitária.

Nessa Procissão do Passado Morto sempre em movimento há somente um lugar
no qual os fantasmas que circulam não estão presentes, e no qual as
paredes de nuvem estão dissipadas. Há aqui uma grande escuridão, densa e
profunda, e cheia de trevas, e além da qual jaz lá fora o grande mundo
real.

Essa escuridão é chamada de O Portal do Horror.

A distância, a Procissão toma o seu curso a partir do portal e, seguindo em
seu caminho, faz um círculo e retorna à escuridão; os fantasmas sombrios
derretem"-se novamente em trevas misteriosas.

Algumas vezes, o Construtor de Sombras atravessa as paredes vaporosas de
sua morada e mistura"-se nas fileiras da Procissão. E~algumas vezes uma
forma invocada pelo gesto de sua mão espectral, com um passo silencioso,
achega"-se saindo da névoa e para a seu lado. Algumas vezes, o
Construtor de Sombras invoca num corpo adormecido uma alma que sonha;
então, por certo tempo, o vivo e o morto ficam face a face, e os homens
chamam isso de sonho do Passado. Quando isso acontece, amigo encontra
amigo ou inimigo encontra inimigo; e à alma do sonhador vem uma
lembrança feliz e há muito desaparecida, ou a agonia inquieta do
remorso. Mas nenhum espectro atravessa a parede enevoada, com a única
exceção do Construtor de Sombras; e nenhum ser humano -- mesmo em sonho
-- pode entrar na obscuridade por onde se move a Procissão.

Assim vive o solitário Construtor de Sombras em meio a suas trevas, e sua
habitação é povoada por um passado espectral.

Seu único povo é o do passado; pois, apesar de criar sombras, elas
não vivem com ele. Seus filhos partem imediatamente para suas casas no
grande mundo, e ele não sabe mais delas até que, na completude do tempo,
se juntem à Procissão do Passado Morto e cheguem, por sua vez,
as paredes enevoadas de seu lar.

\imagemmedia{}{./img/08.png}


Para o Construtor de Sombras não há noite ou dia, nem estações do ano;
mas, para todo o sempre, a silenciosa Procissão do Passado Morto gira
em volta de sua morada solitária.

Algumas vezes ele se senta e medita com os olhos fixos e fitos, nada
fitando; e então, lá no mar, há uma calmaria desnublada ou a treva negra
da noite. Em direção ao distante Norte ou Sul, durante longos meses, ele
nunca fica observando, e então a quietude da noite ártica reina sozinha. Quando os
olhos em devaneio se tornam novamente conscientes, o silêncio duro se
suaviza em sons de vida e de luz.

\imagemmedia{}{./img/09.png}

Algumas vezes, com um franzido em seu rosto e um olhar duro, que raia e
lampeja relâmpagos negros, o Construtor de Sombras impele"-se resoluto à
sua tarefa, e por todo o mundo as sombras marcham densas e rápidas.
Sobre o mar se arroja o negror da tempestade; luzes baças bruxuleiam
dentro de cabanas remotas em pântanos solitários, e até mesmo nos
palácios dos reis sombras negras passam e voam e deslizam por todas as
coisas -- sim, através dos próprios corações dos reis --, pois o
Construtor de Sombras torna"-se, então, horrível de se olhar.

De vez em quando, entre longos intervalos, o Construtor de Sombras, à
medida que completa sua tarefa, demora"-se em seu trabalho como se o
amasse. Seu coração anseia pelos filhos de sua vontade, e ele gostaria
de guardar ao menos uma sombra para servir"-lhe de companhia em sua
solidão. Mas, nesses momentos, a voz do Grande Presente está sempre soando em seus ouvidos, impondo"-lhe pressa. A~voz gigante ribomba:

``Adiante, adiante.''

Enquanto as palavras soam nos ouvidos do Construtor de Sombras, a sombra
total desvanece debaixo de suas mãos e, passando sem ser vista pelo
Portal do Horror, mistura"-se no grande mundo lá fora, no qual deverá
desempenhar seu papel. Quando, na completude do tempo, essa sombra
adentra as fileiras da Procissão do Passado Morto, o Construtor de
Sombras a reconhece e dela se lembra; mas em seu coração morto não há
brilho de recordação amável, pois ele só pode amar o Presente, que
sempre escapa de seu alcance.

E, ah!, é uma vida solitária a que o Construtor de Sombras leva; e nas
trevas estranhas, tristes, solenes, misteriosas e silenciosas que o
envolvem, ele trabalha dura e constantemente em sua tarefa solitária.

Mas, algumas vezes, também o Construtor de Sombras tem suas alegrias.
Sombras bebês surgem, e imagens ensolaradas, iluminadas com doçura e
amor, escorregam de seu toque, e se vão.

Diante do Construtor de Sombras, imerso em sua tarefa há um espaço onde não
há nem luz nem escuridão, nem alegria nem melancolia. Tudo o que toca
esse espaço desaparece como montes de areia que se desfazem quando a maré
chega, ou como palavras escritas na água. Nesse lugar, todas as coisas perdem seu
ser e se tornam parte do grande \emph{Não-É}; e essa linha terrível de
mistério é chamada de O Limiar. Tudo o que adentra nele desaparece; e
tudo o que dele emerge está completo ao chegar e passar para o grande
mundo como algo a cumprir seu curso. Diante do Limiar, o próprio
Construtor de Sombras é como nada; e nessa força absorvente do Limiar
está aquilo que ele não consegue controlar ou dominar.

Em sua tarefa, o Construtor de Sombras faz invocações; e do nada
impalpável do Limiar provém o objeto de seu desejo. Algumas vezes, a
sombra irrompe cheia e fresca, e subitamente se perde nas trevas do
Portal do Horror; outras vezes, cresce suave e imperceptivelmente,
tornando"-se mais repleta à medida que se aproxima, e então se dissolve nas trevas.

\smallskip
O solitário Construtor de Sombras está trabalhando em sua morada
solitária; à sua volta, além das paredes vaporosas, impelindo"-se para
adiante como sempre, está a Procissão do Passado Morto movendo"-se em
círculo. A~tempestade e a calmaria foram invocadas do Limiar, e se
foram; e agora, nesse momento calmo e melancólico, o Construtor de
Sombras interrompe sua tarefa, e fica a desejar, desejar, até que seu
anseio saudoso e solitário receba uma resposta do nada do Limiar.

Dele cresce a sombra de um pé de Bebê, pisando com um andar cambaleante
em direção ao mundo; depois, vem o pequeno corpo roliço e a cabeça
grande, e o Bebê sombra se move adiante, oscilando e equilibrando"-se com
passos incertos. Rápidas por detrás dele vêm as mãos da Mãe, estendidas
num gesto amoroso de ajuda para que ele não caia. Ele dá um passo -- dois --,
e está caindo; mas os braços da Mãe são rápidos e as mãos
delicadas o mantêm firmemente em pé. A~Criança vira"-se e cambaleia
novamente para os braços de sua Mãe.

Novamente luta para andar; e novamente as mãos vigilantes da Mãe
estão prontas. Dessa vez, ele não precisa de ajuda; mas, quando a
corrida acaba, a Criança sombra se volta mais uma vez, docemente, para o
colo de sua Mãe.

Mais uma vez luta para andar, e anda corajosa e firmemente; mas as
mãos da Mãe se detêm junto ao corpo dela, agitadas, enquanto uma lágrima
desce pela sua face, embora essa face esteja agraciada por um sorriso.

O Bebê sombra vira"-se e desvia um pouco do caminho. Então, sobre o
Nada enevoado no qual caem as sombras, voa rapidamente a sombra
tremeluzente de uma pequena mão acenando; e adiante, com passos firmes,
a sombra dos pezinhos se move, saindo para as trevas enevoadas do Portal
do Horror, e vai"-se embora.

Mas a sombra da Mãe não se move. As mãos estão pressionadas contra o
coração, o rosto amável está voltado para cima em reza, e grandes
lágrimas se rolam pelas suas faces. Então, sua cabeça se arqueia para
baixo à medida que os pezinhos passam para além de seu alcance;
e a Mãe se curva cada vez mais para baixo, chorando, até se deitar de bruços.

Enquanto lança seu olhar, o Construtor de Sombras vê as
sombras desaparecendo, desaparecendo, e somente o terrível nada do
Limiar está ali.

Então, nesse mesmo instante, na Procissão do Passado Morto, rondam em
torno das paredes enevoadas as sombras do que já foi -- a Mãe e a
Criança.

\smallskip
Agora, do Limiar sai um Jovem com passo corajoso e animado; e à medida
que sua sombra cai no véu de névoa, a vestimenta e o porte
proclamam"-no um jovem marinheiro. Perto dessa sombra está outra -- a da
Mãe. Mais velha e mais magra, como que por causa da vigília, mas ainda a
mesma. As velhas mãos afetuosas arrumam com graça o lenço que enlaça
frouxamente o pescoço descoberto; e as mãos do Garoto se estendem, tomam o
rosto da Mãe, e trazem"-no para frente para lhe dar um beijo. Os braços
da Mãe flutuam ao redor de seu Filho, e ambos se unem num abraço
apertado.

A Mãe beija seu Garoto diversas vezes; e eles permanecem juntos, como se
fosse impossível separá"-los.

De repente, o Garoto vira"-se como se tivesse ouvido um chamado. A~Mãe
agarra mais apertado. Ele parece protestar carinhosamente; mas os braços
afetuosos seguram com mais firmeza, até que, com delicada força, ele se
desprende. A~Mãe dá um passo adiante, e estende as mãos finas, tremendo
numa agonia sofrida. O~Garoto para, prostra"-se sobre um dos
joelhos; então, arremessando suas lágrimas para longe, ajeita seu chapéu
e se apressa, enquanto a Mãe cai novamente de joelhos, e~chora.

E assim, mais uma vez, lentamente, as sombras da Mãe e da Criança
crescem na completude do tempo, atravessam o Portal do Horror e
circulam entre os fantasmas na Procissão do Passado Morto -- a Mãe
seguindo sem descanso os passos acelerados de seu Filho.

\smallskip
Na longa pausa que se segue, enquanto o Construtor de Sombras observa,
tudo parece mudado. Do Limiar chega uma névoa, tal qual a que se suspende
algumas vezes sobre a superfície de um mar tropical.

Aos poucos a névoa se afasta, e a proa de um portentoso navio, negra e
grande, desliza para frente. As sombras das grandes velas repousam
fracas nas profundezas gélidas do mar enquanto os panos oscilam
indolentes no ar sem brisa. Sobre a amurada há silhuetas apáticas
esperando que um vento venha. A~névoa no mar se dissipa lentamente; e
pelas sombras escuras de homens ao abrigo do clarão do sol e, arejando"-se
com seus largos chapéus de marinheiro, fica claro que o calor é terrível.

Agora, ao longe, atrás do navio, eleva"-se sobre o horizonte uma nuvem
negra, não maior do que a mão de um homem, mas avançando rapidamente em
grande velocidade. Também ao longe, diante do navio, surge a cumeeira de
um recife de coral, que mal pode ser vislumbrado acima da água vítrea e que 
vai escurecendo nas profundezas lá embaixo.

As pessoas a bordo não veem nem uma coisa nem outra, pois se abrigam sob
toldos e ficam a ansiar por brisas frescas.

Cada vez mais rápido a nuvem negra chega, deslizando cada vez mais veloz,
ficando cada vez mais escura e mais vasta conforme se aproxima.

Então, as pessoas a bordo parecem reconhecer o perigo. Sombras
apressadas voam pelos deques; sombras de homens sobem sombras de
escadas. O~agitar das grandes velas vai cessando à medida que, uma a
uma, elas são recolhidas por mãos determinadas.

Porém, mais rápido do que as mãos dos homens consigam trabalhar, a tempestade
vem impetuosamente.

Avança com ímpeto, e coisas terríveis a acompanham logo atrás; escuro breu --
ondas gigantes quebrando e voando para o alto -- a espuma do mar
varrendo os céus -- as grandes nuvens rodopiando em fúria. E, no centro
dessas sombras que voam, rodopiam e enlouquecem, balança a sombra do
navio.

Com a negra escuridão dos céus abarcando tudo, o ímpeto da tempestade
sombria irrompe através do Portal do Horror.

Enquanto espera, e olha e vê o ciclone rodopiando entre as sombras na
Procissão do Passado Morto, o Construtor de Sombras, mesmo em seu
coração morto, sente um pesar de dor pelo corajoso Garoto Marinheiro
arremessado às profundezas, e pela ansiosa Mãe sentada sozinha em casa.

\smallskip
Novamente, vinda do Limiar, uma sombra avança, tornando"-se mais escura à
medida que se aproxima, mas muito, muito fraca a princípio; pois aqui o
sol é forte, e quase não há espaço para sombras na pedra nua que parece
se erguer do brilho e do cintilar das profundezas do mar ao redor.

Na pedra solitária está em pé um Garoto Marinheiro; está magro e
delgado, e suas roupas são somente alguns poucos trapos. Protegendo seus
olhos com a mão, olha para o mar, onde, muito longe, o céu aberto
mergulha para encontrar o mar ardente. Mas nenhuma mancha no horizonte
-- nenhum brilho distante de uma vela branca -- lhe dá um raio de
esperança.

Por muito, muito tempo espreita, até que, exausto, senta"-se na pedra
e curva sua cabeça por um momento como em desespero. À~medida que o
mar baixa, ele colhe da pedra o marisco que chegara durante a maré.

Assim o dia se esvai e a noite vem; e, no céu tropical, as estrelas
penduram"-se como lampiões.

No silêncio frio da noite, o abandonado Garoto Marinheiro descansa --
dorme e sonha. Sonha com o lar -- com braços amorosos e abertos
para encontrá"-lo -- com a farta comida de banquetes -- com campos verdes e
galhos a balançar, e com a felicidade do amor protetor~de sua Mãe.
Pois, em seu sonho, o Construtor de Sombras invoca sua alma onírica e
lhe mostra todas essas bênçãos passando incessantemente na Procissão do
Passado Morto, consolando"-o, assim, para que não desespere e morra.

\imagemmedia{}{./img/10.png}

Assim se passam muitos dias cansativos; e o Marinheiro permanece na
pedra solitária.

Ao longe pode enxergar somente uma colina que parece se erguer acima da
água. Certa manhã, o céu encoberto e o ar abafado prometendo uma
tempestade, a montanha distante parece mais próxima. Ele pensa em
tentar alcançá"-la nadando.

Enquanto está decidindo, a tempestade corre sobre o horizonte e o
arrasta de sua pedra solitária. Ele nada com coração valente; mas, bem
no momento em que sua força se esvai, é jogado pela fúria da tempestade
numa praia de areias macias. A~tempestade passa, seguindo seu caminho, e
as ondas o deixam no alto e no seco. Ele entra pelo interior daquela
terra, onde, dentro de uma caverna nas rochas, encontra abrigo e
mergulha no sono.

O Construtor de Sombras, enquanto vê tudo isso acontecer nas sombras
projetadas nas nuvens, na terra e no mar, alegra"-se em seu coração
morto, porque a Mãe solitária talvez não espere em vão.

\smallskip
Assim, o tempo segue em frente, e muitos, muitos dias tediosos vão
passando. O~Garoto se torna um jovem Homem, vivendo na ilha solitária;
sua barba cresceu, e está vestido com uma roupa feita de folhas. Por todo o
dia, exceto quando está trabalhando para conseguir alimento para comer,
observa do topo da montanha algum navio que possa vir. Enquanto fica lá
vigiando o mar, o sol delineia sua sombra pela encosta abaixo, de forma
que, ao entardecer, à medida que o sol se põe nas águas, a sombra do
Marinheiro solitário se alonga e se alonga, até que, por fim, traça uma
linha escura por toda a encosta até a beira da água.

O coração do Homem solitário se torna mais e mais melancólico enquanto
espera e observa, com o tempo passando tediosamente e dias e noites
incontáveis indo e vindo.

\smallskip
Chega uma hora em que ele começa a ficar cada vez mais fraco. Por fim,
fica doente, à beira da morte, e permanece por muito tempo moribundo.

Até que essas sombras desvanecem.

\smallskip
Do Limiar cresce a sombra de uma velha mulher, magra e desgastada,
sentada dentro de uma cabana solitária em um penhasco protuberante. Na
janela, uma lamparina queima à noite para dar boas"-vindas àquele que está
perdido, caso ele algum dia retorne, e para guiá"-lo ao lar de sua Mãe.
Junto à lamparina, a Mãe fica de vigia, até que, fatigada, mergulha no
sono.

Enquanto dorme, o Construtor de Sombras invoca sua alma adormecida
com o acenar de sua mão espectral.

A mão fica a seu lado na morada solitária, enquanto à volta deles, através
da parede de névoa, segue adiante a Procissão do Passado Morto.

Enquanto ela olha, o Construtor de Sombras levanta sua mão espectral
para apontar para a visão de seu Filho.

Mas os olhos da Mãe são mais rápidos até mesmo que a mão espectral que
evoca todas as sombras da tempestade impetuosa, e, antes que a mão
esteja levantada, ela vê seu Filho entre as Sombras do Passado. O~coração da Mãe se enche de uma alegria inefável quando o vê vivo e
saudável, apesar de prisioneiro em mares tropicais.

Mas, ah!, ela não sabe que na turva Procissão passam somente as coisas
que já foram, e que, apesar de o Marinheiro solitário ter vivido no
passado, no presente -- naquele exato instante -- ele pode estar morrendo
ou estar morto.

A Mãe estende seus braços a seu Filho; mas, no mesmo instante, sua alma
dormente perde de vista a turva Procissão e desaparece da morada
solitária do Construtor de Sombras. Pois, quando fica sabendo que
seu Garoto está vivo, surge uma grande dor por saber que ele está
sozinho, que espera e procura por ajuda; o coração impaciente da Mãe é
vencido pelo pesar, e ela acorda com um grito amargo.

Então, quando se levanta e olha a manhã para além da lamparina que se
apaga, a Mãe sente que teve no sono a visão de seu Filho, e que ele está vivo
e espera por ajuda; e seu coração se abrasa com grande resolução.

\smallskip
Rapidamente, então, surgindo do Limiar, flutuam muitas sombras\ldots{}

\smallskip
Uma Mãe solitária apressando"-se com pés ligeiros para uma cidade
distante.

\smallskip
Homens sérios rejeitando, mas não indelicadamente, uma mulher ajoelhada
suplicando com as mãos levantadas.

\smallskip
Homens severos enxotando de suas portas uma Mãe a rezar.

\smallskip
Um bando de garotos selvagem e garotas más e imprudentes perseguindo
uma mulher apressada pelas ruas.

\smallskip
A sombra da dor num coração de Mãe.

\smallskip
A chegada de uma nuvem negra de desespero, mas que permanece bem longe \mbox{--,}
pois não consegue penetrar na luz solar e radiante da decisão da Mãe.

\smallskip
Dias cansativos que têm sua própria miríade de sombras.

\smallskip
Noites solitárias -- desejo negro -- frio -- fome e dor; e através de
todas essas sombras tenebrosas, a sombra rápida dos pés ligeiros da Mãe.

\smallskip
Uma longa, longa fila com imagens se aproxima cada vez mais,
próxima da Procissão, até que o coração morto do Construtor de Sombras se
torna gélido e seus olhos fulminantes observam selvagemente todos
aqueles que provocam dor e provações ao coração fiel da Mãe.

E assim todas essas sombras flutuam para dentro de uma névoa negra, e
perdem"-se nas trevas do Portal do Horror.

\smallskip
Outra sombra sai da névoa\ldots{}

Um Velho está sentado em sua poltrona. A~luz crepitante da lareira
projeta sua imagem, dançando de forma estranha, na parede do quarto. Ele
é velho, pois os grandes ombros estão curvados, e o rosto nobre e
majestoso está cheio das linhas dos anos. Há outra sombra no
quarto; é a da Mãe -- ela está ao lado da mesa e conta a sua
história. Suas mãos finas apontam para onde, na distância, ela sabe que seu
Filho é um prisioneiro em mares solitários.

O Velho levanta"-se; o entusiasmo do coração da Mãe tocou"-o, e à sua
memória volta rápido o velho amor, a energia e o valor de sua
juventude. A~grande mão se levanta, fecha"-se e bate na mesa com um golpe
poderoso, como que declarando uma promessa irrevogável. A~Mãe cai de
joelhos -- segura a grande mão e a beija; depois, fica em pé, ereta.

Outros homens entram -- eles recebem ordens -- saem apressados.

\smallskip
Então chegam muitas sombras, cujos movimento, rapidez e firme propósito
significam vida e esperança.

\smallskip
Ao pôr do sol, quando os mastros lançam longas sombras nas águas do cais,
um navio grande zarpa em sua jornada a mares tropicais. As sombras dos
homens rapidamente esvoaçam acima e abaixo do cordame e por todo o
convés.

Quando as sombras rodam em volta do cabrestante, a âncora se levanta e
em direção ao pôr do sol desliza o grande barco a vela.

Na proa, como uma figura de Esperança, está a Mãe, fitando com olhos
ávidos o horizonte longínquo.

Então, essa sombra desvanece.

\smallskip
Um grande navio se move, com alvas velas enfunadas pela brisa. Na proa
está a Mãe, fitando sempre a distância diante de si.

Tempestades vêm, e o navio corre na frente do pé de vento; mas não se
desvia, pois a Mãe, com a mão estendida, aponta o caminho, e o
timoneiro, balançando junto a seu timão, obedece a mão.

Então, essa sombra também desvanece.

\smallskip
As sombras dos dias e das noites chegam numa rápida sucessão, e a Mãe
procura continuamente o seu Filho.

\smallskip
Então, os registros de uma jornada próspera desvanecem numa sombra
fraca, turva, enevoada, através da qual uma silhueta sozinha se destaca
claramente -- a Mãe vigilante na proa do navio.

\smallskip
Agora, do Limiar crescem as sombras da ilha montanhosa e do navio se
aproximando. Na proa, a Mãe se ajoelha, alerta e apontando. Um bote é
baixado. Homens saltam a bordo com pés ávidos; mas, antes deles todos,
está a Mãe. O~bote se aproxima da ilha; a água se torna rasa e, na praia
branca e quente, os homens saltam para a terra.

Mas a Mãe ainda está sentada na proa do bote. Em suas longas e ansiosas
horas de agonia, viu, em seus sonhos, seu Filho bem ao longe, de pé e
observando; ela o viu balançar seus braços com grande alegria à medida
que o navio assoma sobre a linha do horizonte; ela o viu em pé na praia,
esperando; ela o viu correndo através da rebentação, de modo que a
primeira coisa que o solitário Garoto Marinheiro tocaria seriam as mãos
amorosas de sua Mãe.

Mas, ai de seus sonhos!\ldots{} Não há nenhuma silhueta com braços acenando
alegremente no pico da montanha -- nenhuma silhueta sôfrega está à
beira da água ou se atira contra a rebentação para encontrá"-la. O~coração dela enregela e arrepia de medo.

Ela chegara mesmo tarde demais?

Os homens deixam o bote, consolando"-a enquanto se afastam com apertos de mão
e toques amáveis no ombro. Ela os apressa, gesticulando, e permanece de
joelhos.

\smallskip
O tempo passa. Os homens escalam a montanha; procuram, mas não encontram
o Garoto Marinheiro, e, com pés lentos e hesitantes, retornam ao bote.

\smallskip
A Mãe os ouve vindo de longe e se levanta para encontrá"-los. Eles baixam
suas cabeças. Os braços da Mãe se erguem, atirados para cima em angústia
e desespero, e ela cai desmaiada no bote.

Num instante, o Construtor de Sombras invoca o espírito dela para que
saia de sua forma humana sem sentidos, e aponta para uma forma que
passa, imóvel, na Procissão do Passado Morto.

Então, mais rápido que a luz, a alma da Mãe voa de volta, tomada
por uma alegria recém"-descoberta.

Ela se levanta do bote -- salta para a terra. Os homens a seguem,
perplexos.

Ela corre pela costa com pés ligeiros; os marinheiros vêm logo atrás.

Ela para defronte a entrada de uma caverna, parcialmente escondida por
arbustos rasteiros. Aqui, sem se virar, gesticula para os homens
esperarem. Eles param e ela entra.

\smallskip
Por alguns momentos, uma escuridão macabra verte do Limiar; e então uma
visão triste, triste, surge e passa\ldots{}

Uma caverna à meia luz, escura -- um homem esgotado deitado de bruços, e em agonia
uma Mãe curvada sobre a forma humana fria. Ela pousa a mão no peito gelado. Mas, ah!, não consegue sentir o batimento do coração
que ela ama.

Num gesto violento, com o coração arrasado, ela se atira sobre o corpo
de seu Filho e o segura forte, forte -- como se o abraço apertado de uma
Mãe fosse mais forte do que o abraço da Morte.

\smallskip
O coração morto do Construtor de Sombras fica vivo de dor à medida que
se afasta da triste imagem; e, com olhos ansiosos, olha para onde, atrás
do Portal do Horror, a Mãe e o Filho devem ir para se juntar às fileiras
sempre crescentes da Procissão do Passado Morto.

Lentamente, lentamente vem passando a sombra da fria forma humana do
Marinheiro.

\smallskip
Porém, mais rápidos que a luz, vêm os pés ligeiros da Mãe. Os braços
tão fortes de amor estão estendidos -- as mãos finas seguram a sombra
errante de seu Filho e o arremessam de volta para além do Portal do
Horror -- para a vida -- e a liberdade -- e o amor.

\smallskip
O solitário Construtor de Sombras sabe agora que os braços da Mãe são
mais fortes que o abraço da Morte.


\imagemmedia{}{./img/11.png}