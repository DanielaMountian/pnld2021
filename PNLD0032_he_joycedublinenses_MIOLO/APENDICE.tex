
%Apêndice

%\makeatletter
%\renewcommand\@endpart{\vfil\clearpage}
%\makeatother

\part{Três cartas sobre Dublinenses}

%\begin{figure}[c]
%\begin{center}
%\includegraphics[width=.6\textwidth]{joyce1.jpg}
%\end{center}
%\end{figure}

\chapter*{\ }

\pagestyle{plain}

\addcontentsline{toc}{chapter}{A Stanislaus Joyce}

{\raggedright\large
\textsc{A Stanislaus Joyce (cartão-postal)}\\\smallskip\normalsize
\textit{22 de janeiro de 1911}\hfill Barriera Vecchia 32, \textsc{iii}
\par}

\bigskip

Talvez te interesse saber que o lançamento de \textit{Dublinenses}, anunciado
pela terceira vez ontem, 20 de janeiro, foi novamente postergado \textit{sine
die} e sem uma palavra de explicação. Conheço a fama e a tradição de meu país
bem demais para me surpreender com o recebimento de três mal-traçadas linhas em
resposta a cinco anos de constantes serviços à minha arte e constante espera e
indiferença e deslealdade em troca dos cento e cinquenta mil francos de dinheiro
continental que arranjei para o bolso de irlandeses e irlandesas famintos desde
que me expulsaram de seu hospitaleiro lamaçal há seis anos.

\medskip

{\raggedleft\scshape
Jim
\par}

%\&\&\&\&

\clearpage

\openany

\addcontentsline{toc}{chapter}{Ao Editor}

\chapter*{\ }

{\raggedright\large
\textsc{Ao Editor}\footnote{ Joyce enviou cópias desta carta a muitos jornais; a carta foi
publicada na íntegra no \textit{Sinn Fein} de Dublin, no dia 2 de setembro de
1911 e, com a passagem polêmica excluída, no \textit{Northern Whig} de Belfast,
no dia 26 de agosto de 1911.}\\\smallskip\normalsize
\textit{17 de agosto de 1911}\hfill Via della Barriera Vecchia 32, \textsc{iii},\\\raggedleft Trieste (Áustria)
\par}

\bigskip

Senhor, posso lhe pedir que publique esta carta que lança alguma luz sobre a
atual condição de um autor na Inglaterra e na Irlanda?

Quase seis anos atrás, o senhor Grant Richards, editor de Londres, assinou um
contrato comigo para a publicação de um livro de contos escritos por mim,
intitulado \textit{Dublinenses}. Cerca de dez meses depois ele me escreveu
pedindo que excluísse um dos contos e algumas passagens em outros que, como ele
disse, seu tipógrafo se recusava a compor. Declinei de ambos os pedidos e
iniciou-se uma correspondência entre mim e o senhor Grant Richards que duraria
mais de três meses. Procurei um jurista internacional em Roma (onde eu morava na
época) e ele me aconselhou a acatar as exclusões. Declinei novamente e o
manuscrito foi devolvido, uma vez que o editor se recusava a publicá-lo não
obstante sua palavra empenhada em letras impressas, e eu com o contrato em mãos.

Seis meses depois, um certo senhor Hone [Joseph Maunsel Hone] me escreveu de
Marselha pedindo que eu enviasse o manuscrito aos senhores Maunsel, editores de
Dublin. Foi o que fiz: e depois de cerca de um ano, em julho de 1909, os
senhores Maunsel assinaram um contrato comigo para a publicação do livro até no
máximo 1º de setembro de 1910. Em dezembro de 1909, um funcionário dos senhores
Maunsel me implorou que alterasse uma passagem em um dos contos,
``Dia de hera na sede do comitê'',
onde se fazia alusão a Edward~\textsc{vii}. Concordei em fazê-lo, muito a contragosto, e
alterei uma ou duas frases. Os senhores Maunsel continuaram postergando a data
de publicação, e por fim me escreveram, pedindo que excluísse a passagem ou que
a mudasse radicalmente. Declinei fazer as duas coisas, observando que o senhor
Grant Richards de Londres não fizera nenhuma objeção à passagem quando Edward~\textsc{vii} 
era vivo e que eu não via por que um editor irlandês haveria de fazer
qualquer objeção à mesma passagem agora que Edward~\textsc{vii} havia entrado para a
história. Sugeri a intermediação de um juiz ou que suprimissem a passagem com
uma nota prévia de minha autoria, explicando, mas os senhores Maunsel não
concordaram com nenhuma das duas coisas. Como o senhor Hone (que me escrevera a
princípio) se eximiu de qualquer responsabilidade no assunto e de qualquer
ligação com a editora, pedi a opinião de advogados de Dublin que me aconselharam
a excluir a passagem, informando-me que, como eu não tinha domicílio no Reino
Unido, não podia processar os senhores Maunsel por quebra de contrato a não ser
que pagasse cem libras à corte e que, mesmo que eu pagasse as cem libras à corte
e os processasse, não tinha chance de conseguir um veredicto favorável com um
júri de Dublin se a passagem em questão fosse considerada de algum modo ofensiva
ao falecido rei. Escrevi então ao atual rei, George~\textsc{v}, anexando uma prova
impressa do conto com a passagem destacada e suplicando que ele me informasse se
em sua opinião a passagem (certas alusões feitas por um personagem do conto no
idioma de sua classe social) deveria ser impedida de publicação por ser ofensiva
à memória de seu pai. O secretário particular de Sua Majestade enviou-me esta
resposta:

\begin{quote}
Buckingham Palace

O secretário particular recebeu ordens de acusar o recebimento da carta do senhor
James Joyce no dia 1º passado e informá-lo de que não faz parte do
protocolo de Sua Majestade expressar sua opinião em tais casos. Os anexos seguem
aqui devolvidos.
\end{quote}

Eis a passagem em disputa:\footnote{ Joyce recortou esse trecho da prova impressa e colou na segunda página da carta. 
[Nota de R.~Ellmann.]}

\begin{quote}\normalsize
--- Escute aqui, John --- disse Mr.~O’Connor.  --- Por que haveremos de dar
boas-vindas ao rei da Inglaterra? O próprio Parnell não\ldots{}

--- Parnell --- disse Henchy --- está morto.  --- Agora, eu acho o seguinte: o
sujeito é impedido pela maldita mãe de ser coroado e só consegue sentar no
trono quando já está grisalho.  É um homem sensível, e deseja o nosso bem.  A
meu ver é um ótimo sujeito e não é dado a frescuras.  Ele deve pensar:
\textit{A velha nunca se deu ao trabalho de visitar esses malditos irlandeses.
Meu Deus, eu vou até lá ver como eles são}.  E nós vamos insultar o
sujeito no momento em que vem nos fazer uma visita de cortesia?  Hein?  Não é,
Crofton?

Mr.~Crofton assentiu com a cabeça.

--- Mas, afinal de contas --- disse Mr.~Lyons em tom de discordância ---, você
sabe, a vida do rei Eduardo não é lá\ldots{}

--- O que passou, passou --- disse Mr.~Henchy.  --- Eu o admiro.  É um sujeito
comum, como você e eu.  Gosta de um trago e é um tanto mulherengo, talvez, mas
tem espírito esportivo.  Diabo, será que nós irlandeses não conseguimos jogar limpo?
\end{quote}

Escrevi este livro há sete anos e, como não consigo imaginar, de nenhum lado,
nenhuma possibilidade de ver meus direitos serem defendidos, eu aqui doravante
concedo aos senhores Maunsel publicamente permissão de publicar esse conto com
quaisquer mudanças ou supressões que bem entenderem e espero que o que eles
venham a publicar se pareça com o que eu escrevi depois de muito pensar e depois
de muito tempo. A atitude deles como editores irlandeses poderá ser julgada pelo
público irlandês. Eu, como escritor, protesto contra os sistemas (legal, social
e cerimonial) que me levaram a este passo. Grato pela sua cortesia, sou, senhor,
seu servidor obediente, 

{\raggedleft\scshape
James Joyce
\par}

%\&\&\&\&

\clearpage

\openany

\addcontentsline{toc}{chapter}{Uma história curiosa}

\chapter*{\ }

{\raggedright
\textit{30 de novembro de 1913}\\\medskip
\par}

{\centering\large
\textsc{Uma história curiosa}\footnote{ Publicado pela primeira vez na revista \textit{Egoist}
(Londres, 1.2, 15 de janeiro de 1914, pp.~26-7) por Ezra Pound.}
\par}

\medskip

A carta a seguir, que contava a história de um livro de contos, foi enviada por
mim à imprensa do Reino Unido há dois anos. Foi publicada por dois jornais até
onde sei: \textit{Sinn Fein} (Dublin) e \textit{Northern Whig} (Belfast).

\bigskip

{\raggedright
\textit{17 de agosto de 1911}\\\smallskip
\par}

Esperei nove meses após a publicação desta carta. Então fui à Irlanda e entrei em
negociação com os senhores Maunsel. Eles me pediram para excluir do conjunto o
conto ``Um encontro'', passagens de ``Dois galãs'', ``A pensão'', ``Um caso
triste'' e que eu mudasse em todas as ocorrências os nomes
dos restaurantes, confeitarias, estações de trem, bares, lavanderias,
escritórios e outros comércios. Após ter argumentado dia após dia contra esse
ponto de vista durante seis semanas e apresentado o assunto a dois
advogados (os quais, apesar de me informarem que a editora incorreu em quebra de
contrato, recusaram meu caso e nem mesmo permitiram que seus nomes fossem
associados ao meu caso sob qualquer aspecto), cedi desesperado diante de todas
as alterações desde que o livro fosse lançado sem mais demora e que o texto
original pudesse ser restaurado em edições futuras, se tal viesse a ocorrer.
Então os senhores Maunsel pediram que eu lhes depositasse no banco mil libras
como caução, ou que lhes arranjasse duas promissórias de quinhentas libras cada.
Recusei-me a fazer as duas coisas; então eles me escreveram, informando que não
publicariam mais o livro, com ou sem alterações, e que se eu não lhes fizesse
uma proposta para cobrir seus custos de impressão, eles me processariam para
recuperar o investimento. Ofereci-me a pagar sessenta por cento do custo de
impressão da primeira tiragem de mil exemplares, se a edição fosse entregue onde
eu determinasse. Esta oferta foi aceita, e consegui com meu irmão em Dublin que
publicasse e vendesse o livro para mim. No dia em que a proposta e o acordo
seriam assinados, os editores me informaram que a questão chegara a um impasse,
pois a gráfica se recusava a entregar o serviço. Fui então conversar na gráfica.
O responsável me disse que a gráfica havia resolvido receber o dinheiro
antecipadamente. Perguntei se ele poderia entregar toda a tiragem para uma
empresa de Londres, ou no continente, se fosse integralmente indenizado. Ele
disse que os exemplares jamais sairiam de sua oficina, e que as matrizes já
haviam sido quebradas e que a edição inteira, os mil exemplares, seriam
incinerados no dia seguinte. Saí da Irlanda no dia seguinte, trazendo comigo um
único exemplar do livro que consegui das mãos do editor.

\medskip

{\raggedleft
\textsc{James Joyce}\\
Via Donato Bramante 4, \textsc{ii}, Trieste, Áustria
\par}


\vspace*{30mm}

{\raggedright\itshape
Tradução de Alexandre Barbosa de Souza
\par}
