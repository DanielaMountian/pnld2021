\textbf{Hesíodo} foi um poeta grego arcaico e, assim como ocorre com Homero,
não é possível provar que ele tenha realmente existido. Segundo certa tradição,
porém, teria vivido por volta dos anos 750 e 650 a.C.  Supõe-se, a partir de
passagens do poema \textit{Trabalhos e dias}, que o pai de Hesíodo tenha
nascido no litoral da Ásia e viajado até a Beócia, para instalar-se num
vilarejo chamado Ascra, onde teria nascido o poeta; supõe-se também que ele
tenha tido um irmão, Perses, que teria tentado se apropriar, por meios ilegais,
de uma parte maior da herança paterna do que a que lhe cabia, exigindo ainda
ajuda de Hesíodo. Acredita-se que a única viagem que Hesíodo teria realizado
tenha sido a Cálcis, com o objetivo de participar dos jogos funerários em honra
de Anfidamas, dos quais teria sido o ganhador e recebido um tripé pelo
desempenho na competição de cantos. Apenas três das obras atribuídas a Hesíodo
resistiram ao tempo e chegaram às nossas mãos: são elas os \textit{Trabalhos e
dias}, a \textit{Teogonia} e \textit{O escudo de Héracles}.

\textbf{Trabalhos e dias} (em grego \textit{Erga kai Hamerai}) é um poema épico
de 828 versos em que são contados alguns dos mitos gregos mais conhecidos até
hoje, como o de Prometeu e o de Pandora. Diferente da \textit{Teogonia}, que
apresenta a origem dos deuses, este poema é voltado para a condição dos
mortais, explicitando suas necessidades e limitações, com foco no trabalho
agrícola baseado nas estações do ano. Com a ajuda das Musas, o poeta narra em
primeira pessoa e se dirige a seu irmão Perses, na tentativa de ensinar a ele
verdades divinas a respeito das práticas humanas.

\textbf{Teogonia} (em grego \textit{theogonia}: \textit{theos} $=$ deus +
\textit{genea} $=$ origem) é um poema de 1022 versos hexâmetros datílicos que
descreve a origem e a genealogia dos deuses. Muito do que sabemos sobre os
antigos mitos gregos é graças a esse poema que, pela narração em primeira
pessoa do próprio poeta, sistematiza e organiza as histórias da criação do
mundo e do nascimento dos deuses, com ênfase especial a Zeus e às suas façanhas
até chegar ao poder. A invocação das Musas, filhas da Memória, pelo aedo
Hesíodo é o que lhe dá o conhecimento das coisas passadas e presentes e a
possibilidade de cantar em celebração da imortalidade dos deuses; e é a partir
daí que são narradas as peripécias que constituem o surgimento do universo e de
seus deuses primordiais.  

\textbf{Christian Werner} é professor livre-docente de língua e literatura
grega na Faculdade de Letras da Universidade de São Paulo (\versal{USP}).
Publicou, entre outros, \textit{Duas tragédias gregas: Hécuba e Troianas}
(Martins Fontes, 2004).







