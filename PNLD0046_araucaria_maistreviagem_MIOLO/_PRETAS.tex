\textbf{Xavier de Maistre} (Chambéry, 1763--São Petersburgo, 1852)
nasceu em família nobre, na Savóia hoje francesa, e teve uma educação
refinada. Interesses variados marcaram profundamente sua vida e
sua obra. Na juventude chegou a interessar-se pelas novas invenções e
foi um dos primeiros do seu país a fazer um voo de balão. Mas logo
alista-se no exército italiano e a disciplina militar acaba por
conduzir sua vida a partir de então. Passou algum tempo na Itália, onde
sua família também fugia da Revolução Francesa, e depois se estabeleceu
em São Petersburgo. Anos mais tarde, a Rússia invadida por Napoleão,
Xavier de Maistre juntou-se novamente às fileiras que bateriam o
imperador francês, participando dos embates até Waterloo. Casou-se com
uma fidalga russa com quem teve quatro filhos que perdeu cedo. Embora a
pintura tenha sido sua grande paixão, foi a literatura, e especialmente
a \textit{Viagem em volta do meu quarto} e sua continuação, a
\textit{Expedição noturna em volta do meu quarto}, que fez com que
alcançasse fama e reconhecimento. Sua \textit{Viagem\ldots} é lembrada por
Machado de Assis na dedicatória do seu \textit{Memórias póstumas de Brás Cubas}.   

\textbf{Sandra M. Stroparo} é graduada em Letras/Francês pela Universidade
Federal do Paraná (\textsc{ufpr}), onde também fez o mestrado em Letras.
Trabalha como professora de Teoria Literária e Literatura Brasileira na
\textsc{ufpr} desde 1998. Dentre alguns trabalhos de tradução, publicou
\textit{Axël}, de Villiers de L’Isle Adam, pela Editora \textsc{ufpr}. É
atualmente doutoranda em Teoria Literária pela Universidade Federal de
Santa Catarina (\textsc{ufsc}), onde desenvolve um trabalho de estudo e tradução
da correspondência de Stéphane Mallarmé. 


