\documentclass[11pt]{hedrabook}
\usepackage[brazilian]{babel}
\usepackage{ucs}
\usepackage[utf8x]{inputenc}
\usepackage[center,cam,a4]{hedracrop}
\usepackage{hedrabolsolayout,hedraextra}
\usepackage[chapterdot]{hedratoc}
\usepackage[protrusion=true,expansion]{microtype}
\usepackage{comment,lipsum,footmisc,pdfpages}

\geometry{bottom=1cm,%textheight=148mm,%
		top=10mm,bottom=10mm,
		paperwidth=143mm,%
		paperheight=222mm}

\begin{document}
\SVN $Id: TEMPLATE.tex 6339 2010-04-28 17:52:17Z bruno $


\selectlanguage{brazilian}
\title{Sermões, vol.~i} 
\author{Antonio Vieira} 
\begin{blackpages}
	\maketitle
	\begin{techpage}{5cm}
		\vspace{-1cm}
		\putline{Copyright}{Hedra \the\year}
		\putline{Tradução$^\copyright$}{   } 
%		\putline{Título original}{}
%		\putline{Edição consultada}{\emph{ }, }
%		\putline{Primeira edição}
%		\putline{Indicação}{}
%		\putline{Agradecimento}
		\putline{Corpo editorial}{
			Adriano Scatolin,\\
			Alexandre B.~de Souza,\\
			Bruno Costa,
			Caio Gagliardi,\\
			Fábio Mantegari,
			Iuri Pereira,\\
			Jorge Sallum,
			Oliver Tolle,\\
 		   Ricardo Musse,
			Ricardo Valle
		}
		\ \\

		\putline{Dados}{\fichacatalografica{H331}% Código de autor
				% Autor (data nascimento--morte)
				{Heine, Heinrich (1797–1856)}% 
				% Dados bibliográficos
				{O Rabi de Bacherach. / Heinrich Heine. 
				Tradução e organização de Marcus
				Vinicius  Mazzari. – São Paulo: Hedra, 2009. (Estudos Libertários).}%
				% Isbn
				{978-85-7715-131-8}%
				% Catalogação. Ex. 1.~Blabla.(Não esquecer ~)
				{1.~Literatura Alemã. 
				2.~Romance Histórico. 3.~Religião. 4.~Judaismo. 5.~Ódio Racial. 
				6.~Antissemitismo. 
				\textsc{i}.~Título. 
				\textsc{ii}.~Série. \textsc{iii}.~Mazzari, Marcus Vinicius, Tradutor.}%
				%CDU
				{830}
				%CDD 
				{833.7} %
			       }
		\direitos
		\dadoseditora
		\depositolegal
	\end{techpage}
	\begin{frontispiciopage}{4cm}{}	
	\putline{\hspace{12ex}Organização}{\textsc{Alcir Pécora}}
	\end{frontispiciopage}

\end{blackpages}
\setcounter{tocdepth}{0}     % amplitude da presença das partes no índice
\setcounter{secnumdepth}{-2} % amplitude da numeração das partes


% Favor não alterar o segundo parâmetro (\baselineskip). 
% Para acertar entrelinhas, usar comando \linespread
%\fontsize{10.5pt}{\baselineskip}\selectfont
\baselineskip=13.2pt   % Parâmetro válido apenas para corpo 11, para outros, acrescer 20% do valor do corpo%%

\paginabranca
\includepdf[pages=3-284]{TOMO2A.pdf}
\includepdf[pages=1-299]{TOMO2B.pdf}

\SVN $Id: FINAIS.tex 13024 2014-02-17 16:55:20Z bruno $

\SVN $Id: PUBLICIDADE.tex 7874 2010-12-07 16:31:20Z oliveira $ 
\pagebreak
\pagestyle{empty}
\textsc{títulos publicados}
\begin{enumerate}
\setlength\itemsep{0.1mm}
%Se o corpo for 11 ou maior, colocar \scripsize
{
\def\normalsize{\fontsize{7}{7}\selectfont}\normalsize
\item Patativa do Assaré
\item Cuíca de Santo Amaro
\item Manoel Caboclo
\item Rodolfo Coelho Cavalcante
\item Zé Vicente
\item João Martin de Athayde
\item Minelvino Francisco Silva
\item Expedito Sebastião da Silva
\item Severino José
\item Oliveira de Panelas
\item Zé Saldanha
\item Neco Martins
\item Raimundo Santa Helena
\item Téo Azevedo
\item Paulo Nunes Batista
\item Zé Melancia
\item Klévisson Viana
\item Rouxinol do Rinaré
\item J. Borges
\item Franklin Maxado
\item José Soares
\item Francisco das Chagas Batista
\vfill
}%
\end{enumerate}


\pagebreak

%\begin{blackpages}
\thispagestyle{empty}
\begin{techpage}{42mm}
		\putline{Edição}{Bruno Costa}
		\putline{Coedição}{Iuri Pereira e Jorge Sallum}
		\putline{Capa e projeto gráfico}{Júlio Dui e Renan Costa Lima}
%		\putline{Imagem de capa}{}
		\putline{Programação em LaTeX}{Bruno Oliveira}
		\putline{Preparação}{Denise Pessoa}
		\putline{Revisão}{Bruno Costa e Luan Maitan}
		\putline{Assistência editorial}{Luan Maitan}
          
		\putline{Colofão}{Adverte-se aos curiosos que se
			imprimiu esta obra em nossas oficinas em \today, em papel 
			\mbox{off-set} 90~g/m²,
			composta em tipologia Minion Pro, 
			em \textsc{gnu}/Linux (Gentoo, Sabayon e Ubuntu), 
			com os softwares livres 
			\LaTeX, De\TeX, \textsc{vim}, Evince, Pdftk, 
			Aspell, \textsc{svn} e \textsc{trac}.}

\end{techpage}
%\end{blackpages}


\ifdefined\printcheck\printcheck\fi

\end{document}
