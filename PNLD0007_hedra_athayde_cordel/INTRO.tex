\chapter[Introdução, por Mário Souto Maior]{Introdução}

\section{Nascimento}

Filho de Belchior Martins de Lima e de dona Antônia
Lima de Athayde, nasceu no dia 23 de junho de 1877, na
então vila Cachoeira de Cebola, hoje denominada Itaituba,
no município de Ingá, na Paraíba, o menino que na pia
batismal recebeu o nome de João Martins de Athayde, e
que no dizer de Waldemar Valente, ``é o príncipe dos poetas
populares do Norte do Brasil''.\footnote{ Waldemar
Valente, ``João Martins de Athayde: um
depoimento'', \textit{Revista Pernambucana de
Folclore}, Recife, maio/agosto 1976.}

A data de seu nascimento envolve muitas controvérsias.

Átila Augusto F. de Almeida e José Alves Sobrinho, no
\textit{Dicionário bio-bibliográfico de repentistas e poetas de
bancada},\footnote{ Átila Augusto F. de Almeida \& José Alves Sobrinho,
\textit{Dicionário bio-bibliográfico de repentistas e poetas de
bancada}. João Pessoa: Editora Universitária, 1978, p. 71.}
mencionam o ano de 1880 como o do seu
nascimento.

O poeta popular baiano Minelvino Francisco Silva, por
sua vez, registra em \textit{Vida, profissão e morte de João Martins
de Athayde}:\footnote{ Silva, Minelvino Francisco. \textit{Vida, profissão e
morte de João Martins de Athayde}. Salvador: Fundação
Cultural do Estado da Bahia, s.d., p. 1.}


\begin{verse}

A 24 de junho 1880\\*
João Athayde nasceu\\*
No estado da Paraíba\\*
Onde era o berço seu,\\*
Cachoeira de Cebola\\*
Conforme Deus concedeu

\end{verse}

\noindent registrando o dia 24 de junho de 1880 como a data natalícia
do vate paraibano.

Cavalcanti Proença, um dos experts em literatura de
cordel, no seu \textit{Literatura popular em versos},\footnote{ Proença,
Cavalcanti. \textit{Literatura popular em verso}. Rio de Janeiro: Fundação
Casa de Rui Barbosa, 1964, p. 569.} que organizou
para a Fundação Casa Rui Barbosa, também registra o
nascimento do poeta no ano de 1880.

Acontece que, por ocasião da missa celebrada em
sufrágio de sua alma, a família de João Martins de Athayde
fez distribuir entre os presentes o clássico \textit{santinho}, no qual
se encontra registrada a data de 23 de junho de 1877 como
a de seu nascimento

Acredito que ninguém pode precisar com mais acerto a
data em que nasceu um de seus entes queridos do que sua
própria família. É esta a razão pela qual adotei como data
de nascimento de João Martins de Athayde o dia 23 de
junho de 1877.

Mas a controvérsia não terminou ainda.

Numa entrevista concedida ao jornalista Paulo Pedrosa,
do \textit{Diário de Pernambuco},\footnote{ Pedrosa, Paulo.
``Cangaceiros e Valentões''. \textit{Diário de
Pernambuco}, Recife, 16 jan. 1944.} o poeta afirmou haver nascido
no dia 23 de junho de 1880.

Fui informado de que o folclorista Liedo Maranhão tem
uma fita gravada na qual o poeta afirma haver nascido em
1878.

A verdade é que naquele tempo ainda não se promulgara
a lei que estatuía a obrigatoriedade do registro civil em
cartório próprio.

Talvez nem mesmo o próprio poeta soubesse, com
exatidão, a data em que nasceu.

Mas, em 1877, 1878 ou 1880, no dia 23 ou 24 de junho,
João Martins de Athayde nasceu e foi um dos maiores e
mais queridos poetas populares do Nordeste.

\section{Vida}

A cidade de Ingá, nome indígena que significa ``cheio
d'água'', situada em plena zona da
caatinga, a 85
quilômetros em linha reta de João Pessoa, Paraíba, de clima
quente e seco, e temperatura entre 22 e 34° C, foi habitada
inicialmente pelo português Manuel da Costa Travassos, que
obtivera permissão, em tempos remotos, para explorar
aquelas terras. Ali fixou residência, erigiu uma capela sob a
invocação de Nossa Senhora da Conceição, dedicando-se à
criação de gado e à agricultura, registra Coriolano de
Medeiros em seu \textit{Dicionário corográfico do estado da Paraíba}.

Em torno da igrejinha a povoação foi crescendo até que
passou à categoria de vila, com o nome de Vila do Imperador.

Outros historiadores acham que nos meados do século
XVII já ali residiam, em suas fazendas de criação, Francisco
de Arruda Câmara, Gaspar Correia e Cosma Tavares
Leitão, viúva do sertanista Teodósio de Oliveira Ledo.

Em 1864, a Vila do Imperador, pela Lei Provincial nº
3, teve seu nome substituído pelo de Ingá, que conserva
até hoje.

Em virtude de seu desenvolvimento, a Vila de Ingá foi
elevada à categoria de município, formado pelos distritos
de Ingá (sede do município), Riachão do Bacamarte, Serra
Redonda e Cachoeira de Cebola. Este foi o mundo onde o
poeta João Martins de Athayde nasceu, viveu sua infância
e parte de sua adolescência.

O velho Belchior Martins de Luna era um pequeno
agricultor e tirava da terra o sustento de sua família,
composta pela mulher e três filhos.

Com a morte da esposa, o velho Belchior casou pela
segunda vez e seus filhos não tiveram sorte com a madrasta,
que judiava dos enteados a ponto de deixá-los quase nus e
em geral mal alimentados. Sem que o pai soubesse, cada
menino recebia da madrasta apenas uma xícara de farinha
de mandioca, sem um pedacinho de carne sequer.

Um dia, quando o menino João Martins de Athayde
voltava do roçado, desmaiou de fome, no caminho,onde
foi encontrado por conhecidos de seu pai que ajudaram a
trazê-lo para casa.

Mesmo sem frequentar a escola, o poeta andava com
uma carta de ABC no chapéu, pois seu maior sonho era
aprender a ler e a escrever. E tanto era assim, que ele saía
perguntando as letras às pessoas e, como não tinha caderno
nem lápis, escrevia no chão, com o dedo.

Não se sabe se ele fugiu de casa ou se teve o
consentimento do pai para vir tentar a vida no Recife, onde
tinha um parente ou conhecido que era pequeno
comerciante. No Recife, trabalhou no comércio e, dizem,
até em fábrica não sei de quê.

Sabendo ler e escrever, já ``taludinho'', João Martins de
Athayde começou a escrever seus primeiros versos e
imprimir seus primeiros folhetos, que vendia nas feiras e
nos mercados do Recife.

Com o dinheiro da venda dos folhetos e com o que
ganhava nos empregos, conseguiu comprar uma pequena
impressora manual, uma guilhotina para cortar o papel
dos folhetos, alugar uma casa, contratar vários empregados,
de vez que a procura de seus folhetos era tão grande e ele
procurava atender à freguesia e aos seus 
agentes-vendedores em diversas cidades do Nordeste.

Estava, assim, feito o poeta popular João Martins de
Athayde, que, com o apurado dos folhetos que escrevia,
comprou a casa onde morava e uma máquina melhor.

Sua fama de poeta popular corria solta pelas feiras e
pelos mercados do Nordeste.

Depois de cada dia de trabalho que, às vezes, se prolongava
até a madrugada, ele tinha uma grande paixão que era o
cinema. Vez por outra gostava de assistir aos filmes exibidos
no Glória e no Ideal, cinemas existentes naquela época, no
Pátio do Mercado São José e no Pátio do Terço, do Recife.

Interessante é o fato de o poeta, pelo que me consta,
nunca haver escrito um folheto baseado em algum filme.
Pode ser até que tenha usado na capa de alguns folhetos
fotografias de artistas famosos da época.

Morava na Paraíba (nome também dado, antigamente,
à cidade de João Pessoa) uma moça muito bonita chamada
Sofia Cavalcanti, filha da viúva dona Luísa. Sua irmã,
casada, morava no Recife e tanto fez que trouxe sua mãe e
Sofia para tentarem a vida lá, que era uma cidade maior,
com mais possibilidades de se encontrar trabalho. Uma
conhecida delas, chamada dona Ritinha, falou com dona
Luísa: ``Luísa, você consente que eu leve Sofia para
trabalhar numa casa que eu conheço há muito tempo?''

Dona Luísa, depois de muito relutar, deu seu
consentimento. E a mocinha Sofia foi apresentada, então,
ao poeta João Martins de Athayde, que não somente lhe
deu a vaga na gráfica, como também ficou apaixonado por
sua beleza.

E a partir de 1923, Sofia passou a trabalhar na gráfica
do poeta, que, vendendo folhetos, conseguiu comprar duas
ou três casas e adquirir várias máquinas. O poeta ficou de
olho na mocinha Sofia, até que um dia com ela se casou,
passando a morar num segundo andar. Como o poeta era
muito ciumento, ela passou vinte anos assim, enclausurada,
sem quase sair de casa, sem conversar com ninguém. Se
uma outra mulher chegasse na casa do poeta pra falar com
dona Sofia, ele perguntava logo se havia trazido algum
bilhete para ela, tamanho era o ciúme dele.

Dos filhos do primeiro casamento e de como era a vida
do poeta que passava noites em claro, ora escrevendo seus
folhetos, ora imprimindo-os, e outras informações mais
detalhadas sobre sua vida poderão ser encontradas na
entrevista de dona Sofia Cavalcanti de Athayde, sua viúva, 
realizada durante o ciclo de estudos promovido pela
Fundação Joaquim Nabuco, por ocasião das comemorações
do centenário do nascimento do poeta, em 1980.

Vejamos, agora, como um outro poeta popular,
Minelvino Francisco da Silva, conta a vida do grande poeta
nordestino no seu folheto \textit{Vida, profissão e morte de João
Martins de Athayde}, com o qual ganhou o Prêmio de
Literatura de Cordel, instituído pela Fundação Cultural
do Estado da Bahia:

\pagebreak

\begin{verse}

\textsc{Vida, profissão e morte\\ de João Martins de Athayde}\\*

\bigskip

Como humilde trovador\\*
Do estado da Bahia,\\*
Vou falar de um personagem\\*
Que muita gente aprecia,\\*
João Martins de Athayde --\\*
Um Gênio da Poesia

A 24 de junho 1880\\*
João Athayde nasceu\\*
No estado da Paraíba\\*
Onde era o berço seu,\\*
Cachoeira de Cebola\\*
Conforme Deus concedeu

Crescendo João Athayde\\*
Lá pelo alto sertão\\*
Se dispôs a trabalhar\\*
A fim de ganhar o pão,\\*
Enfrentou diversas artes\\*
Porém não fez profissão

Depois de muito lutar\\*
Trabalhando noite e dia,\\*
Em fábrica como operário,\\*
Sem a menor garantia\\*
Achou que devia explorar\\*
A Arte da Poesia

Fez da caneta uma enxada\\*
E a roça da inspiração,\\*
Da poesia popular\\*
Fez a sua plantação,\\*
No campo fértil das Letras\\*
De sua imaginação

Mudou-se para Pernambuco\\*
Em Recife se firmou\\*
Escrevendo seus livrinhos\\*
Por todo canto espalhou\\*
Aí financeiramente\\*
Bastante se melhorou

Leandro Gomes de Barros\\*
Um trovador de cartaz\\*
Viajou pra eternidade\\*
Para voltar nunca mais,\\*
João comprou da viúva\\*
Seus direitos autorais

Seiscentos mil réis por tudo\\*
Essa viúva cobrou,\\*
Athayde achando caro\\*
Mas mesmo assim concordou\\*
Recebendo o documento\\*
Esta quantia entregou

E daí continuou\\*
Publicar pra mais de mil\\*
Romances de todo assunto\\*
Pra militar e civil\\*
E tornou-se conhecido\\*
Em todo o nosso Brasil

O nome apresentava\\*
No livro como editor,\\*
A maior parte do povo\\*
Não conhecendo o autor\\*
Dizia -- João Athayde\\*
É o maior trovador!

Tempos e tempos depois\\*
João Martins se desgostou,\\*
No assunto de família\\*
Um soneto publicou\\*
Que o mesmo tinha o título:\\*
\textit{Estrela que se apagou}

O assunto do soneto\\*
Se mostrava apaixonado\\*
Dizia que neste mundo\\*
Tornou-se um pobre coitado,\\*
Por não casar com a primeira\\*
Mulher que ele foi amado

Seja por sim ou por não\\*
Ninguém sabe o que se deu,\\*
O certo é que Athayde\\*
Com a arte esmoreceu,\\*
Seus direitos autorais\\*
A Zé Bernardo vendeu

Por vinte contos de réis\\*
Vendeu naquele momento\\*
Seus direitos autorais\\*
Com todo contentamento,\\*
Recebeu o apurado\\*
E assinou o documento

No Ano 55\\*
Na capital da Bahia\\*
No Congresso dos Poetas\\*
Tivemos a primazia\\*
De ver Martins de Athayde\\*
Dar-nos bastante alegria

Nosso Congresso queria\\*
Trazer ele de avião\\*
Mostrando sua humildade\\*
Respondeu logo que não,\\*
Que de ônibus para ele\\*
Dava mais satisfação

No dia 1º de julho\\*
João Athayde chegou\\*
Lá na rua Carlos Gomes\\*
Que o Congresso realizou\\*
Houve uma salva de palmas\\*
No salão quando ele entrou

Todo povo reunido\\*
Quando entrou o trovador\\*
João Martins de Athayde\\*
Homem de grande valor\\*
Renderam logo homenagem\\*
Chamando-o de professor

Um dizia: Professor\\*
Com seus livros aprendi\\*
Professor -- dizia outro,\\*
É prazer vos ver aqui\\*
{}-- Meu professor, vos saúdo\\*
Outro falava dali

Aí eu cheguei na hora\\*
Como humilde trovador\\*
Abracei ele, dizendo:\\*
Parabéns meu professor,\\*
Por todas as suas obras\\*
De grandioso valor

Graças a Deus com seus versos\\*
Eu também aprendi a ler\\*
Porque um livro escolar\\*
Eu não podia obter,\\*
Meu pai não podia comprar\\*
Para eu ir aprender

Eu comprei um ABC\\*
Quando aprendi soletrar\\*
Os seus livrinhos versados\\*
Comecei logo estudar,\\*
Eu fui lendo e fui gostando\\*
Comecei me desasnar

Fui assim continuando\\*
Com grande satisfação\\*
Lendo o livro \textit{Zé Pretinho},\\*
\textit{Uma festa no sertão},\\*
\textit{A imperatriz Porcina},\\*
\textit{Juvenal com o dragão}

Aí eu adquiri\\*
Um pouquinho de saber\\*
Que meus livros de histórias\\*
Eu também dei pra escrever\\*
Tornei-me profissional\\*
Não tenho tempo a perder

Usando da minha máquina\\*
Bati uma fotografia\\*
De João Martins de Athayde\\*
Um Gênio da Poesia\\*
Pra ficar como lembrança\\*
No Estado da Bahia

Terminou nosso Congresso\\*
João Athayde voltou\\*
Com seu filho a Pernambuco\\*
Com poucas horas chegou,\\*
Daí pra cá eu não sei\\*
O que foi que se passou

Tempos depois em Brasília\\*
Encontrei um sobrinho seu,\\*
João Athayde Sobrinho\\*
Me contou o que se deu,\\*
No ano 59\\*
João Athayde morreu

No estado de Pernambuco\\*
Em Limoeiro a cidade\\*
João Martins de Athayde\\*
Foi pra Terra da Verdade\\*
A todos os seus colegas\\*
Deixando muita saudade

Eu imploro a Jesus Cristo\\*
Filho da Virgem Pura\\*
Que perdoe todas as culpas\\*
Dessa pobre criatura\\*
Que passou a sua vida\\*
Só espalhando a cultura

Ó Deus de misericórdia\\*
Ouça a minha oração\\*
Vós sois o Deus de Isac\\*
De Jacó e de Abraão\\*
Aceite João Athayde\\*
Em vossa Santa Mansão

Essa mesma petição\\*
Eu faço à Virgem Maria\\*
Que rogue ao seu bento filho\\*
Por sua sabedoria\\*
Que perdoe todos os pecados\\*
De quem semeia a poesia

\end{verse}

O poeta João Martins de Athayde faleceu às seis horas
do dia 7 de agosto de 1959, na cidade de Limoeiro,
Pernambuco, vítima de uma embolia cerebral, deixando
a viúva Sofia Cavalcanti de Athayde e oito filhos: Josefa
Augusta de Athayde Dornelas, João Martins de Athayde
Filho, Manoel Cristiano de Athayde, Maria José de
Athayde, Ceci de AthaydeMontenegro, João Oliveira de
Athayde, Fernando Oliveira de Athayde, Carlos Oliveira
de Athayde e Marcus VinÍcius de Athayde.

\section{Obra}

Ninguém sabe, com absoluta certeza, quantos folhetos
foram escritos e publicados por João Martins de Athayde.
Sua gráfica, trabalhando a todo vapor, quase que
semanalmente lançava um título novo ou mais uma
edição/impressão de um folheto que, na época, estivesse
fazendo sucesso. As encomendas recebidas de seus agentes-
vendedores espalhados por todo o Nordeste chegavam
quase todos os dias, e ele procurava entregar, imprimindo
durante as madrugadas.

Consegui inventariar os seguintes folhetos do poeta:
\textit{Amor de perdição}, \textit{A vitória da revolução brasileira}, \textit{O triste
fim de um orgulhoso}, \textit{Os últimos dias da humanidade ou O
fim do mundo}, \textit{Uma viagem ao céu}, \textit{A vida de Nascimento
Grande}, \textit{Amor de pirata}, \textit{O amor de um estudante ou O poder
da inteligência}, \textit{A filha do boiadeiro}, \textit{O fim do mundo}, \textit{A
morte de Lampião}, \textit{A órfã abandonada}, \textit{A nobreza de um
ladrão}, \textit{História da imperatriz Porcina}, \textit{História de José do
Egito}, \textit{Peleja de João Athayde com Raimundo Pelado}, \textit{O
Jeca na praça}, \textit{Juvenal e o dragão}, \textit{A lamentável morte de
padre Cícero Romão Batista -- O patriarca do Juazeiro},
\textit{Lampião em Vila Bela}, \textit{As proezas de Lampião}, \textit{Sacco e
Vanzetti aos olhos do mundo}, \textit{Uma noite de amor}, \textit{A paixão
de Madalena}, \textit{A sorte de uma meretriz}, \textit{Proezas de João
Grilo}, \textit{O lobo do oceano}, \textit{História de Natanael e Cecília}, \textit{A
rainha que saiu do mar}, \textit{O retirante}, \textit{Romance do escravo
grego}, \textit{O Recife novo}, \textit{O prisioneiro do castelo da rocha negra},
\textit{Proezas de Lampião na cidade de Cajazeiras}, \textit{A garça
encantada}, \textit{Romance do príncipe que veio ao mundo sem ter
nascido}, \textit{Rachel e a fera encantada}, \textit{O segredo da princesa},
\textit{Discussão de um crioulo com um padre}, \textit{História de um
pescador}, \textit{O lobo do oceano}, \textit{A menina perdida}, \textit{O monstro
do Rio Negro}, \textit{O romance de um sentenciado}, \textit{Um amor
impossível}, \textit{A dama das camélias}, \textit{História de Paulo e Maria},
\textit{O dia de juízo}, \textit{Discussão de José Duda com João Athayde},
\textit{A grande surra que levou Cordeiro Manso de João Athayde
por desafiá-lo}, \textit{A chegada de Lampião e Maria Bonita a
Maceió e Corisco vingando o chefe}, \textit{Meia-noite no cabaré}, \textit{O
toureiro de Umbuzeiro ou O curandeiro misterioso}, \textit{O
namoro de um cego com uma melindrosa da atualidade}, \textit{A
princesa sem coração}, \textit{A grande batalha no reino da bicharia},
\textit{O homem do pulso de ferro}, \textit{A vida e os novos sermões do
padre Cícero}, \textit{Peleja de João Athayde com José Ferreira
Lima}, \textit{A pérola sagrada}, \textit{O primeiro debate de Patrício com
Inácio da Catingueira}, \textit{Mabel ou Lágrimas de mãe}, \textit{O poder 
oculto da mulher bonita}, \textit{João Batista Lusitano
desmascarado na sua mentirosa profecia do ano de 17},
\textit{História de Roberto do Diabo}, \textit{História de um homem que
teve uma questão com Santo Antônio}, \textit{O prêmio do sacrifício
ou Os sofrimentos de Lindoia}, \textit{Peleja de Ventania com Pedra
Azul}, \textit{Quatro poetas glosados: Ugulino, Romano, Nogueira
e o velho Mufumbão}, \textit{Peleja de Serrador e Carneiro}, \textit{O
homem que nasceu para não ter nada}, \textit{Martelo de José Duda
e Joaquim Francisco em Itabaiana}, \textit{Germano e Mufumbão},
\textit{Peleja de Laurindo Gato com Marcelino Cobra Verde},
\textit{História da princesa Elisa}, \textit{Peleja de Bernardo Nogueira
com Preto Limão}, \textit{História da moça que foi enterrada viva
ou a infeliz Sofia}, \textit{Peleja de Manoel Raymundo com Manoel
Campina}, \textit{História de Dimas -- O bom ladrão}, \textit{A guerra
dos animais}, \textit{Peleja de Antônio Machado com Manoel
Gavião}, \textit{Peleja de Patrício com Inácio da Catingueira}, \textit{O
prêmio da inocência}, \textit{A infelicidade de dois amantes}, \textit{História
do valente Vilela}, \textit{O efeito da passagem do eclipse total do
Sol e o alarme dos que não tinham visto o fenômeno},
\textit{Discussão de João Athayde com Leandro Gomes}, \textit{Alzira --
A morta viva}, \textit{O casamento do bode com a raposa}, \textit{História
de um rico avarento}, \textit{Fugida da princesa Beatriz com o conde
Pierre}, \textit{História da escrava Guiomar}, \textit{História de Joãozinho
e Mariquinha}, \textit{Décimas amorosas}, \textit{O bataclan moderno}, \textit{A
filha do bandoleiro}, \textit{O casamento do calango}, \textit{História de
Balduíno e o estudante que se vendeu ao diabo}, \textit{História do
menino da floresta}, \textit{As felicidades que oferece o casamento},
\textit{O balcão do destino ou a menina da ilha}, \textit{A filha das selvas},
\textit{O azar na casa do funileiro}, \textit{A desventura de um analfabeto
ou o homem que nunca aprendeu a ler}, \textit{Em homenagem às
mulheres}, \textit{O fantasma do castelo}, \textit{A condessinha roubada},
\textit{Discussão de João Athayde com João de Lima}, \textit{A fada e o
guerreiro}, \textit{Discussão de João Athayde com Mota Júnior},
\textit{Discussão de um operário com um doutor}, \textit{Doutor Caganeira},
\textit{A entrada de padre Cícero no céu vista por uma donzela de
13 anos}, \textit{Uma festa no sertão} e \textit{História do negrão André
Cascadura}.

Como se vê, João Martins de Athayde não era um poeta
cuja temática fosse o sobrenatural, apesar de alguns de seus
folhetos enfocarem o céu, padre Cícero, o Diabo ou o
inferno. Não era, também, o poeta do circunstancial, de
fazer um jornalismo paralelo, como José Costa Leite, o
poeta-repórter. Era, sim, um poeta voltado para o amor,
para a aventura, para o grotesco, para o mundo da
imaginação.

Mas seus folhetos, no que se refere à autoria, geraram
dúvidas com a venda que o poeta fez aos herdeiros de José
Bernardo da Silva dos direitos autorais dos seus folhetos,
que passaram a trazer impressos na capa, como também
na primeira página, os dizeres: ``JOÃO MARTINS DE
ATHAYDE. Proprietárias: Filhas de José Bernardo da
Silva''.

Entendem alguns estudiosos que José Bernardo da Silva
tenha assumido, com a compra feita, a autoria dos folhetos.

Sobre a autoria dos folhetos de José Bernardo da Silva
ou de João Martins de Athayde, acreditamos que ela só
poderia ser elucidada se fosse feito, a cargo de linguistas e
outros especialistas no assunto, um sério estudo do
vocabulário, dos temas de predileção, das rimas e outros
recursos técnicos que fogem ao meu conhecimento.

Na realidade, a obra de João Martins de Athayde, como
poeta popular, é uma das mais significativas e ricas do
Nordeste.

\begin{flushright}
\textit{Mário Souto Maior}\\
\end{flushright}


\section{Bibliografia}

\begin{description}\labelsep0ex\parsep0ex
%\newcommand{\tit}[1]{\item[\textnormal{\textsc{\MakeTextLowercase{#1}}}]}
%\newcommand{\titidem}{\item[\line(1,0){25}]}

\item Almeida, Átila Augusto F. de \& Sobrinho, José Alves.
\textit{Dicionário bio-bibliográfico de repentistas e poetas de
bancada}. João Pessoa: Editora Universitária, 1978.

\item Athayde, João Martins de. \textit{O trovador do Nordeste} (primeira
série). Com notas e comentários de Waldemar Valente.
Recife: 1937.

\item Medeiros, Coriolano de. \textit{Dicionário corográfico da Paraíba}.
João Pessoa: 1947.

\item Pedrosa, Paulo. ``Cangaceiros e valentões''. \textit{Diário de
Pernambuco}, Recife, 16 jan. 1944.

\item Proença, Cavalcanti. \textit{Literatura popular em verso}. Rio de
Janeiro: Fundação Casa de Rui Barbosa, 1964.

\item Silva, Minelvino Francisco. \textit{Vida, profissão e morte de João
Martins de Athayde}. Salvador: Fundação Cultural do
Estado da Bahia, s.d.

\item Valente, Waldemar. ``João Martins de Athayde: um
depoimento''. \textit{Revista pernambucana de folclore}, Recife,
maio-agosto, 1976.

\end{description}