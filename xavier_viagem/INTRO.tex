\chapter[Introdução, por Sandra M. Stroparo]{introdução}
\hedramarkboth{introdução}{sandra m. stroparo}
\vspace*{2ex}

\textsc{Viagem em volta do meu quarto} e \textit{Expedição noturna em
volta do meu quarto}, embora se completem, são obras escritas com um
espaço de tempo considerável entre si. A primeira foi publicada em
1794\footnote{ Esta é a data do anúncio feito em Turim, mas a data real
é 1795, quando Joseph de Maistre providencia a edição em Lausanne, na
Suíça.}  e sua continuação apareceu somente em 1825. O intervalo de 31
anos entre os dois relatos deixou marcas interessantes que podem ser
percebidas, se não nos temas desenvolvidos e em uma melancolia quase
permanente na \textit{Expedição noturna\ldots} -- apesar de algumas
situações descritas beirarem muitas vezes o absurdo cômico --,
especialmente nas diferenças de linguagem entre os dois textos. Grande
admirador de Laurence Sterne, que é a principal referência para esta
obra, o autor deixa a retórica do século \textsc{xviii} mais evidente na
\textit{Viagem\ldots}, nos circunlóquios e repetições do narrador. No
segundo livro a linguagem está mais leve, mais fluente, as frases se
organizam de maneira mais clara.

 Os dois textos, a \textit{Viagem\ldots} e a \textit{Expedição\ldots} -- ou as
duas partes de um só, como geralmente são apresentados hoje -- são
discursos ficcionais que narram exatamente o que os títulos indicam:
uma viagem feita dentro de um quarto. Não se trata, claro, de um
romance de aventuras. 

Os romances de viagem estavam em voga na Europa. Na verdade, desde as
narrativas de Marco Pólo, e depois com a descoberta da América, o
europeu estava fascinado pela possibilidade da evasão. Já no século
\textsc{xix}, durante o romantismo, a viagem será inclusive tema romanesco: a
viagem como forma de construção de conhecimento, a viagem como processo
de amadurecimento do herói, a viagem fantástica e de ficção futurista,
a viagem como evasão.

Mas esta, proposta pelo narrador de Xavier de Maistre, é diferente. São
ao menos três capítulos gastos, no primeiro volume, para defender esse
método intramuros e tentar explicá-lo como interessante e produtivo. O
leitor pode não se convencer logo no início, mas a companhia do
narrador não será das piores e para o nosso tempo, mais consciente das
várias possibilidades de \textit{trips} inventadas e provocadas por
nós, considerar uma viagem puramente intelectual e especulativa não
gera dificuldade. A metáfora que se cria, claro, é literária, ou, em
alguns momentos mais inspirados, de discussão filosófica --
especialmente se usamos o sentido largo que o século \textsc{xviii} usou para a        %T:"Usamos" e, logo em seguida, "usou" na mesma frase. 
filosofia. Assim, nada mais peculiar ao século \textsc{xviii} que uma obra que combina
pretensões intelectuais e discussões sobre a natureza do homem e da
sociedade sua contemporânea com o mais puro tom mundano do discurso de                %T: Este "sua" me parece sobrando ou carecendo de pontuação. 
corte. Uma mostra divertida desse convívio é, por exemplo, a descrição
de um sonho que afligiu o narrador, feita no capítulo \textsc{xlii}.

 O segundo volume, a \textit{Expedição\ldots}, no entanto, será mais
introspectivo. Podemos nos perguntar: mais que uma obra escrita sobre o
próprio quarto? E na expedição noturna há inclusive uma janela aberta,
o olhar passeia pelas estrelas assim como pela sacada próxima onde está
uma bela vizinha\ldots\ Mas há mais ``exterior'' no cenário ao mesmo tempo em
que desejos, alegrias e frustrações são tratados a partir sempre de uma
crescente subjetividade. A linguagem é mais direta, a frase é menos
sinuosa: o século \textsc{xix} prefere sentenças menos tortuosas\ldots\ e a
individualidade passa a ser explorada. Dessas particularidades emerge
finalmente um texto mais pessoal que o do primeiro volume e, condizendo
com algumas das reflexões desenvolvidas, mais melancólico. 

A vida do autor nos dá pistas para vários temas desenvolvidos em suas
obras, embora alguns dados da sua biografia não sejam muito precisos.
Começamos com a data de seu nascimento, por exemplo, que varia entre
1760 e 1763, sendo essa última, aparentemente, a mais correta. 

Nasce em Chambéry, naquele momento capital dos estados da Savóia, em uma
família nobre. A Savóia tem uma história autônoma que remonta à Idade
Média e no século \textsc{xv} torna-se um estado independente. A Renascença
francesa de Francisco \textsc{i} não tarda, no entanto, a dividir os poderes na
região, e mais tarde, no período entre 1792 e 1815, e então desde 1860,
a maior parte da região da antiga Savóia torna-se francesa. 

O berço nobre garantiu-lhe uma educação sofisticada, sendo a própria
família, aparentemente, o primeiro grande estímulo. Seus biógrafos, no
entanto, são unânimes em descrevê-lo como um diletante dono de uma
vida itinerante e algo indolente, especialmente em contraste com o
restante da família. O pai, magistrado, descende de uma linhagem de
toga. A mãe, católica fervorosa, inspirará os filhos, fazendo de dois
deles -- foram dez, ao todo -- um abade e uma religiosa. O filho mais
velho, Joseph, será um grande teórico contrarrevolucionário, filósofo e
homem ativo,  um anti-iluminista cuja obra tem sido relida
contemporaneamente. 

 Joseph parece também tê-lo ajudado a escrever o seu \textit{Prospectus
de l’expérience aérostatique de Chambéry}, publicado em 1784, que
descreve um voo de balão realizado por ele, Xavier, e seu amigo Louis
Brun,\footnote{ Onze anos antes, os irmãos Mongolfier haviam realizado o
primeiro voo tripulado.} e que lhe deu certa fama na cidade natal. O
gosto por objetos voadores aparece no capítulo \textsc{ix}, da \textit{Expedição
noturna\ldots}, na frustrante -- mas adequadamente hilária -- pomba mecânica
construída pelo narrador. 

Engajou-se no serviço militar embora tenha começado a carreira
lentamente. Ainda em 1784 é voluntário na região do Piemonte: nesse ano
deixa a casa da família, firmando residência em Turim, onde permanecerá
vários anos. É desse período a \textit{Viagem em volta do meu quarto}
(1794), editada na Suíça. O livro obteve uma repercussão imediata,
assim como o desejo declarado dos leitores de que o autor desse
seguimento àquela história. Só mais ao final da vida, no entanto, é que
Xavier terá uma ideia mais clara do alcance e da popularidade da obra. 

Pouco antes, ainda em 1789, a Revolução Francesa marcará profundamente a
vida do autor e de sua família, obrigada a abrigar-se em território
italiano. Mesmo no texto leve da \textit{Viagem\ldots} sua reação à
revolução se faz presente -- já na \textit{Expedição\ldots}, anos depois,
seu posicionamento fica claro desde o início. Com a invasão francesa,
ligou-se ao exército russo, na Itália naquele momento. Mais tarde, em
torno de 1799, depois de uma campanha de alguma vitórias e grandes
derrotas, retira-se para a Rússia, ligado que estava ao general
Suvarov, seu comandante. Estabelece-se então em São Petersburgo, onde
parece muito mais disposto a trabalhar com suas telas e pincéis que com
a pena.

Embora fosse esse o seu desejo, não consegue sobreviver às custas da
pintura e acaba se resignando à vida militar e burocrática, acomodada
mas confortável: foi diretor de museu e biblioteca, enquanto
participava intensamente da vida mundana da corte russa. Seu irmão
Joseph é então enviado a São Petersburgo como embaixador da Sardenha na
corte do tsar e essa proximidade com o irmão mais velho dá novo fôlego
a Xavier. 

Galga vários postos no exército russo, alcançando posições de destaque.
Casa-se, em São Petersburgo, com uma das damas de honra da imperatriz,
Sophie Sagriatzki, o que lhe garante uma renda segura. As guerras
napoleônicas engajam a armada russa e Xavier de Maistre participa delas
até Waterloo. Em carta a seu irmão, o autor afirma que ``os cadáveres
obstruíam o caminho que, de Moscou até a fronteira, tinha ar de um
campo de batalha contínuo''.  Terminada a guerra ele abandona o
exército, seu último posto tendo sido o de general. 

Passará os anos seguintes longe da capital. Em 1825, após ter perdido
dois de seus quatro filhos, abandona a Rússia em direção à Itália, onde
permanecerá por treze anos. É desse período, de volta a Turim, a
continuação da \textit{Viagem}, a que ele dará o nome de
\textit{Expedição noturna em volta do meu quarto}. Fará várias viagens
à Savóia, agora já dividida entre a França e a Sardenha. Na Itália,
perderá seus outros dois filhos, o que determinará a tristeza de seus
últimos anos. Em 1838, fazendo a vontade de sua mulher, começa a volta
para São Petersburgo, mas não sem antes viajar ainda uma vez para a
França e, pela primeira vez em sua vida, para Paris. 

Na capital francesa descobre, aparentemente bastante surpreso, a fama e
o sucesso que sua obra, especialmente a \textit{Viagem\ldots} e a
\textit{Expedição\ldots}, lhe tinham proporcionado. Descobre também que
algumas ``continuações'' apócrifas concorriam com as suas. Fornece uma
entrevista a Sainte-Beuve\footnote{ Charles-Augustin Sainte-Beuve
(1804--1869), escritor e crítico francês, foi o mais importante crítico
de sua época, desenvolvendo a teoria, rejeitada a partir do início do
século \textsc{xx}, de que a obra de um autor deve ser estudada como um reflexo
da sua vida. O texto que escreve sobre Xavier de Maistre a partir desse
encontro segue essa disposição teórica, muito ao gosto da época,
expondo -- ou tentando expor -- tanto o próprio autor como sua obra aos
seus leitores.} que escreverá um ensaio sobre ele, publicado
inicialmente na \textit{Revue des deux mondes}, depois como prefácio à
edição da obra completa do autor e finalmente como um capítulo da obra
em cinco volumes \textit{Portraits contemporains}. 

Viverá em São Petersburgo até 1852, onde morre com a idade de 89 anos,
tendo perdido também sua mulher um ano antes. As outras obras publicadas 
em vida do autor são \textit{Le Lépreux de la cité d’Aoste} (1811), \textit{La Jeune sibérienne} 
(1825) e \textit{Les Prisonniers du Caucase} (1825). Hoje temos acesso também a alguns 
poucos poemas e parte de sua correspondência.  
A vida longa e de realizações contidas parece ter deixado traços em seus
livros e seus biógrafos e críticos costumam enxergar em suas obras mais
famosas, a \textit{Viagem\ldots} e a \textit{Expedição\ldots}, a construção
de uma voz narrativa cindida entre alguma pretensão intelectual e a
atração leviana da sociedade, assim como entre a evasão e o
confinamento, características claramente determinantes para os temas
dos dois livros. 

Detido em Turim, por ter participado de um duelo, Xavier de Maistre
começa sua \textit{Viagem\ldots} Nesse primeiro volume a clausura é
tratada ironicamente: criando seu narrador como um oficial na mesma
situação -- aprisionado em seu quarto --, Xavier de Maistre deixa
correr a pena para situações prosaicas assim como para questões que ele
chama de filosóficas, fazendo com que umas decorram das outras. Assim,
para o narrador, os dias limitados pela prisão domiciliar são ocupados
pelo ócio confortável e diletante, em que a ``viagem'' literária faz o
papel de distração, mero passatempo intelectual. Esse pode ser o tom
geral da narrativa, mas os assuntos em que ela se embrenha acabam por
tocar em questões interessantes que, embora não sejam desenvolvidas
seriamente, são suficientes para nos dar uma amostra dos seus
interesses e, sem dúvida, dos interesses que tocavam seus
contemporâneos -- e os nossos, hoje. 

E são assuntos variados. Fechado no quarto o narrador discute a
possibilidade da própria obra, fazendo da metalinguagem a abertura do
livro e a linha mestra que sustenta a voz narrativa na passagem de um
capítulo a outro, em que os assuntos não necessariamente se seguem. E a
retórica do texto busca em uma empática reflexão a aprovação e
concordância do leitor\ldots\ até que este esteja definitivamente
``fisgado''. Já de imediato a ideia da viagem se impõe, sendo que o
desafio é a descrição de quarenta e dois dias de ``descobertas'' feitas a
partir de deambulações em volta do quarto. 

Toda a força argumentativa do século \textsc{xviii} se coloca aqui a serviço da
literatura. Laurence Sterne é o principal parâmetro literário, assim
como todos os escritos moralistas do período: fazer de qualquer tema,
motivo, dúvida ou certeza uma razão para longas elucubrações é o método
preferido da época, que vai do salão ao texto filosófico e às melhores
páginas literárias. Para quem se lembrou de algo parecido, qualquer
semelhança com as \textit{Memórias póstumas de Brás Cubas} não é mera
coincidência.

Os capítulos são curtos e os assuntos, vários. Rosine, a cachorrinha
\textit{poodle} do narrador, serve para uma explanação sobre o amor dos homens
pelos animais -- e vice-versa. O criado Joannetti é motivo para boas
reflexões sobre as relações humanas, que não escondem muitas vezes a
prepotência e a falta de modos do ``patrão''. Quadros na parede permitem
uma viagem pela memória e pelas histórias com as mulheres\ldots\ motivo,
aliás, constante, assim como é constante a menção à inconstância e volubilidade femininas. 

Mas os conflitos internos humanos dão motivo para discussões divertidas
que anunciam grandes questões da psicologia: a teoria da ``alma'' e da
``besta'', uma versão com exemplos cômicos da suposta divisão entre corpo
e alma, a hipótese da bipartição -- sim,  só ``bi'', ainda estamos nos
setecentos -- inerente a todos nós domina vários capítulos da
\textit{Viagem\ldots}. Há também outros conflitos, mais seriamente
tratados, como o papel que os amigos -- sua cachorrinha e, quase, seu
criado -- cumprem em nossa vida  e a falta que eles podem fazer, o vazio
que eles podem deixar. No capítulo \textsc{xxi} da viagem, a morte de um
verdadeiro amigo ganha a sua homenagem, como uma pausa respeitosa no
meio do discurso\ldots\ A melancolia, a tristeza talvez, desse capítulo só
se compara àquelas presentes na \textit{Expedição\ldots}. 

O século das luzes abre então espaço para o sentimentalismo. Embora o
amor por um discurso de argumentação, raciocínio, especulação sobre
toda e qualquer ideia, fazendo desse processo um valor principal e a
garantia de sustentação da obra não tenha se perdido entre a
\textit{Viagem\ldots} e a \textit{Expedição\ldots},  o segundo volume se
revela muito mais sentimental.

Escrita vários anos depois, a \textit{Expedição\ldots} apresenta o ``mesmo''
narrador, mais velho, fazendo comentários à primeira viagem, já tão
conhecida, e tentando repetir o modelo em uma viagem noturna. Uma noite
só, outro tipo de desafio narrativo. Mas essa noite é quase tão longa
quanto os outros quarenta e dois dias. E em alguns momentos mais séria,
com toques melancólicos que beiram a solidez da tristeza no final do
volume. A linguagem, como já foi dito, apresenta grandes mudanças. Mas
os textos não são brutalmente diferentes: é como um antes e depois,
como olhar para fotos de um amigo que conhecemos na juventude.
Reconhecemos os traços, descobrimos a passagem do tempo. A passagem do
tempo, o envelhecimento, a perda irremediável daqueles a quem amamos, a
impossibilidade da felicidade e o ridículo da infelicidade fecham o
volume dando ao narrador um tom mais sábio que aquele alcançado no
primeiro volume. O capítulo \textsc{xii} exemplifica a inevitabilidade das
``tristezas e amarguras'' na vida do narrador. Mais sábio, menos
esperançoso. 

O início do século \textsc{xix}, no entanto, não tira dessa voz o interesse pelo
discurso fluente e pelos assuntos variados. Reflexões de caráter social
-- e celeste (no sentido astronômico e não religioso do termo) -- podem
levar, por exemplo, à elaboração de utopias de governo\ldots\ como a que
decidiria que todos os cidadãos deveriam, obrigatoriamente, olhar para o céu: 

\begin{hedraquote}
-- Oh! Se fosse soberano de um país [\ldots], faria a cada noite
soar o toque do sino e obrigaria meus súditos de todas as idades, de
todos os sexos e todas as condições a se colocarem à janela para olhar
as estrelas.
\end{hedraquote}

O grau de devaneio presente na afirmação (e no restante da discussão
sobre o assunto) não é muito distinto de sistemas inspiradores do
socialismo utópico, como o de Fourier, o pensador francês que inventou
os falanstérios como projeto de sociedade ideal. Tal fato denota acima
de tudo uma atualidade de alcance muito interessante. E neste ponto é
também possível enxergar uma proximidade com o pensamento romântico,
tão dado à evasão quanto às idealizações, fossem elas sociais ou
amorosas.

 A vizinha que cantava no balcão abaixo do apartamento do narrador é
outro exemplo, representando o ligeiro motivo lírico-amoroso da
\textit{Expedição\ldots} Ouvindo um rápido trecho do seu canto o narrador
se motiva a espiá-la e coloca-se então em circunstâncias periclitantes.
Poderíamos perceber aqui uma ironia suave aos desvarios amorosos dos
primeiros romances românticos provavelmente lidos pelo autor? Ou
simplesmente percebemos na obra a atualização e a incorporação de um
gosto da época? Se comparamos ao volume de 31 anos antes, há diferenças
interessantes que podem ser registradas. Enquanto no primeiro volume a
demorada \textit{toilette} de uma dama é motivo para uma quase
``humilhação'' do narrador, que não suporta não ser o centro das atenções
da sua dama, no segundo, o narrador se vê longamente admirando um
chinelinho\ldots 

O quarto é permissivo\ldots\  Há estudos que aproximam e opõem a obra de
Xavier de Maistre à \textit{Filosofia na alcova}, de Sade, partindo da
coincidência do confinamento como tema narrativo para os diferentes
resultados alcançados pelos autores. O \textit{boudoir} é comum ao
mundo aristocrático a que ambos pertencem, o resultado materialista de
Sade é diferente do de Xavier de Maistre, mas o princípio do prazer
rege ambas as obras. Considere-se, como complicador, o capítulo \textsc{xxxii}
da \textit{Viagem\ldots} e a ironia posta sobre as violências e
radicalidades revolucionárias. Finalmente, podemos ainda perceber nas
obras de Xavier de Maistre um grande embate entre o conforto
intelectual do confinamento e o desejo pelo mundo de fora, da
atividade, da vida. A oposição clara entre o que se ``pensa'' sobre o
mundo e como o mundo de fato se apresenta. Se há uma ironia
reconhecível na descrição do trato amoroso, a própria especulação
filosófico-humanista é também posta à prova: como racionalizar o
sentimento? O amor, a perda ou a mera frustração? O formato escolhido
para o texto, percebemos então, acaba se mostrando mais adequado ao
índice de ironia da primeira parte do que a certas manifestações mais
sensíveis que podemos encontrar na segunda. Mas mesmo esse detalhe é
revelador da dificuldade imposta pela resposta dada à pergunta acima. O
século \textsc{xix} ainda levará algumas décadas para chegar ao seu melhor
romance.

Mas Flaubert leu Xavier de Maistre e, sem levar em conta a presença da
ironia na obra, põe Bouvard e Pécuchet para desprezá-lo. O problema
aparentemente seria a presença excessiva da personalidade do autor\ldots\
Claro que para o escritor de \textit{Madame Bovary} o desaparecimento
do autor -- e mesmo do narrador -- era uma questão crucial. Seguindo
um outro ponto de vista, um crítico contemporâneo, Daniel Sangsue, já considerou a obra
uma narrativa ``excêntrica'' que por suas qualidades particulares
procuraria criticar e ironizar o gênero romanesco clássico, aquele onde
tudo é grande: o sentimento, a nobreza dos personagens, a aventura.
Seria uma composição caracterizada por ``digressões, uma hipertrofia do
discurso narrativo e uma atrofia da história contada, um questionamento
sobre as personagens'' etc. A ``estética do detalhe'', escolhida para as
longas descrições do quarto e do mundo a partir do quarto, seria na
verdade uma representação irônica, porque ela ``modifica as escalas de
avaliação da nobreza das coisas'' e acabaria por alcançar a simpatia do
leitor deslocando sua atenção para banalidades cotidianas como a
cachorrinha Rosine ou o simples ato de acordar. Além do romance, a
narrativa de viagens também tem aqui seus paradigmas questionados: o
isolamento do quarto possibilita a minúcia atenta que se perde nas
grandes distâncias e panoramas. A obra colocaria em discussão,
portanto, metalinguisticamente, o próprio gênero a que pertenceria ou,
no mínimo, de que fazia parte de forma tangencial.

Romance ou antirromance, narrativa excêntrica, narrativa de viagem ou
narrativa heroica e cômica sobre viagem: essas são algumas das
classificações levantadas pela crítica contemporânea. Mas talvez
possamos realmente classificar Xavier de Maistre e sua obra na
categoria de \textit{libre-pensée}. Sem haver desenvolvido nenhuma
doutrina específica, sem ter defendido nenhuma causa maior, ele acabou
por criar uma obra que passeia por várias dessas possibilidades e que
não abre mão de tratar delas, mesmo se às vezes levianamente,
transformando os assuntos mais sérios em uma quase risada de salão.  O
discurso da individualidade realizada através da digressão pode talvez
ser o principal definidor das duas obras, assim como a construção lenta
da opinião pessoal, a permissão à perspectiva discordante, o
desenvolvimento tolo e/ou perspicaz dos mesmos assuntos. 

Ao leitor o aprofundamento e as escolhas. Mas sem esquecer que as
viagens podem ser divertidas. Antes de considerá-las filosóficas,
poderíamos chamar certas discussões propostas pelo livro de
profundamente humanistas. Aí talvez um dos grandes interesses da obra e
o cheiro de modernidade que ela oferece: não é difícil para
nós, passados já alguns anos dos 2000, nos sentirmos tocados por ela e,
em alguns momentos, de forma ora cômica, ora encabulada, nos
identificarmos com esse narrador viajante.

\section{Nota sobre a tradução}

A obra de Xavier de Maistre pode impor, hoje, algumas dificuldades 
para a tradução e a leitura. Ela acompanha o processo de descobertas da 
prosa do século \textsc{xviii} e no seu caso a influência da obra de Sterne oferece 
ainda um outro detalhe: frases longas, descritivas e infinitamente encadeadas 
graças a muitas conjunções e uma pontuação complexa. Algo disso, claro, precisava 
permanecer no texto para os leitores de língua portuguesa.
	
Temos no Brasil uma tradução já clássica, a de Marques Rebelo, onde o tom escolhido 
oferece um gosto ``de época'' ao manter muito do ritmo original com um vocabulário e 
formas de tratamento correspondentes a essa escolha. Na presente tradução, ao contrário, 
alguns procedimentos experimentam uma maior contemporaneidade -- e vernaculidade -- tentando 
ao mesmo tempo manter a identidade do texto original. A mudança do título (que permanecia o 
mesmo de uma tradução portuguesa do século \textsc{xix}) e opções como o uso do pronome ``você'' 
substituindo a clássica escolha pelo ``vós'' são um exemplo dessa postura, que se aplica a 
toda extensão do texto (notem-se os traços de diálogos do narrador com supostas interlocutoras 
epistolares e com seus possíveis leitores). 

\begin{bibliohedra}
\vspace{2ex}

\tit{bosi}, Alfredo (org.). \textit{Machado de Assis}. São Paulo: Ática, 1982.

\tit{compagnon}, Antoine. \textit{Les Antimodernes: de Joseph de Maistre à
Roland Barthes}. Paris: Gallimard, 2005.

\tit{maistre}, Xavier. \textit{Oeuvres complètes du Comte Xavier de Maistre:
édition précédée d’une notice de l’auteur par M. Sainte Beuve (1839).}
Paris: Garnier Frères, 1889. Disponível em: <http://gallica.bnf.fr> 

\tit{--------}. \textit{Oeuvre complète}. Paris: Éditions du Sandre,
2005.

\tit{sangsue}, Daniel. \textit{Le Récit excentrique. Gautier, de Maistre,
Nerval, Nodier}. Paris: José Corti, 1989. 

\tit{starobinski}, Jean. \textit{As máscaras da civilização: ensaios}. São
Paulo: Companhia das Letras, 2001. Trad. Maria Lúcia Machado. 

\tit{vovelle}, Michel (org.). \textit{O homem do iluminismo}. Lisboa: 
Ed.~Presença, 1992. Trad.:Maria Georgina Segurado.
\end{bibliohedra}



