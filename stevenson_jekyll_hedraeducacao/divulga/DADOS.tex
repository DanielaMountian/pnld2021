\titulo{O estranho caso do Dr.~Jekyll e Mr.~Hyde} % Em minúsculas!!!
\autor{Robert Louis Stevenson}  % Apenas sobrenome, se for o caso. Verificar capa.% 
\organizador{Organização e tradução}{Braulio Tavares}   %Conferir se é apenas {Organização}; {Organização e tradução} ou apenas {Tradução}%
\isbn{978-85-7715-262-9}
\preco{ 28}   % Ex.: 14. Não usar ",00"%
\pag{176}   % Número de páginas
\release{\textbf{O estranho caso do Dr. Jekyll e Mr. Hyde} narra a história 
de um respeitado médico, cujas relações com um homem sórdido, 
de aparência grotesca, faz com que seus amigos desconfiem de que 
ele é vítima de chantagem. Empenhados em ajudar o Dr. Jekyll
a libertar-se desse suposto explorador, começam a investigar os 
vínculos entre os dois homens. Essa seria a dimensão mais 
superficial deste clássico do horror psicológico.
Mas há algo que torna a história mais assustadora. É que Stevenson
busca uma dimensão alegórica, e é tal dimensão que o faz localizar 
a maldade no seio mesmo da bondade, e o monstruoso surgir da mais 
plácida aparência humana. O livro teve ainda o mérito de abordar 
temas que se tornariam basilares para a psicanálise, que apenas
nascia. 
 
\textbf{Esta edição traz como complementos ao texto um conjunto de ensaios 
inéditos em português.} Trata-se de depoimentos escritos pelo próprio escritor 
-- dentre os quais o fundamental ``Um capítulo sobre o sonho'' --, 
por sua esposa e por seu afilhado, sobre a gênese da narrativa, surgida em um sonho 
e escrita em poucos dias de atividade ininterrupta. Acrescentamos ainda dois 
ensaios escritos por pesquisadores contemporâneos de Stevenson sobre as cisões 
de personalidade, ``A personalidade multiplex'' e ``As desintegrações do ego''. 

\noindent\textbf{Robert Louis Stevenson} (Edimburgo, 1850---Samoa, 1894) 
descende de uma família de engenheiros que construiu alguns dos 
faróis da costa escocesa. Em 1857, seus pais transferem-se para 
Edimburgo. Aos dezessete anos, Stevenson ingressa no curso de 
engenharia, mas, abandonando-o, passou a fazer direito e se formou em 1875. 
Seus dois primeiros livros, \textit{An Inland Voyage} (1878) e \textit{Travels with a Donkey in the 
Cevennes} (1879), são descrições de suas viagens. Em 1878, viaja 
à Califórnia ao encontro de Fanny Van de Grift Osbourne, que 
conhecera dois anos antes, e com quem se casaria. Publicou \textit{New 
Arabian Nights} em 1882, reunindo seis narrativas escritas 
entre 1877 e 1880. Por estas narrativas, é considerado um dos 
primeiros cultores ingleses do conto. Seu volume de 1887, 
\textit{The Merry Men and Other Tales and Fables}, traz o conto ``Markheim'', 
que antecipa o tema do duplo que estaria na base de \textit{O estranho caso 
do Dr. Jekyll e Mr. Hyde}. \textit{A Ilha do tesouro} (\textit{The Treasure Island}, 
1883), seu maior sucesso, é a primeira de uma série de narrativas 
de aventura, que inclui \textit{The Black Arrow}, do mesmo ano, e \textit{Kidnapped} 
(1886). Em 1888, vai morar em Samoa, onde permanece os últimos seis anos de sua vida.  

\noindent\textbf{Braulio Tavares} (1950) é escritor e compositor. Dentre os livros que publicou estão \textit{A Pedra do 
Meio-Dia ou Artur e Isadora} (1998), \textit{O homem artificial} (1999) e \textit{Os martelos de Trupizupe}
(2004). Organizou as antologias \textit{Páginas de sombra: contos fantásticos brasileiros} (2003), 
\textit{Contos fantásticos no labirinto de Borges} (2005) e \textit{Freud e 
O Estranho --- contos fantásticos do inconsciente} (2007). Pesquisador da história da literatura fantástica 
do Brasil, organizou a compilação bibliográfica \textit{Fantasy, Fantastic and 
Science Fiction Literature Catalog} (1992). Traduziu \textit{A máquina do tempo}, 
de H.G.~Wells (2010), entre muitos outros. É membro desde 1990 da Science 
Fiction Research Association, sediada nos EUA. Seu livro \textit{A invenção do mundo pelo Deus-Curumim} 
recebeu o prêmio Jabuti para literatura infantil em 2009.

}


