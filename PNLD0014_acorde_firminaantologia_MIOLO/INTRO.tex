\chapterspecial{Vida e obra}{}{Rodrigo Jorge Ribeiro Neves}

\section{Sobre a autora}

Maria Firmina dos Reis~nasceu no dia 11 de março de 1822, ano
em que o Brasil se tornava independente de Portugal, em São Luís do
Maranhão. Filha de Leonor Felipa dos Reis, escrava alforriada e, segundo
seu registro de óbito, João Pedro Esteves, homem abastado e sócio do
antigo dono de sua mãe. Considerada a primeira escritora negra do
Brasil, foi também professora primária, compositora, musicista e
criadora da primeira escola de meninas e meninos do país, fundada em
Maçaricó, povoado próximo ao município maranhense de Guimarães; as
aulas, gratuitas, eram ministradas dentro de um barracão na propriedade
de um senhor de engenho. No entanto, a escola mista não chegou a durar
três anos em decorrência da insatisfação gerada na cidade.

A escritora cresceu e viveu em meio a uma
sociedade elitista, escravocrata e patriarcal. O estado do Maranhão era
mais um a expressar seu elitismo por meio do acesso limitado ao ensino.
Na época, só havia os cursos de Medicina e Direito, portanto os que
publicavam livros faziam parte de um grupo extremamente restrito. A
maioria dos escritores eram homens brancos, economicamente
privilegiados, com acesso ao estudo das letras e recursos para a
publicação de seus trabalhos. Maria Firmina dos Reis viveu a
Independência do Brasil em 1822, a promulgação da 1ª Constituição em
1824, a~Lei Eusébio de Queiroz de 1850, a abolição da escravidão em
1888, a~Proclamação da República em 1889, assim como todas as mudanças
que surgiram no país e no mundo com a virada do século.

Conheceu a literatura ao mudar"-se para a vila de São José
de Guimarães, em 1830. Sua relação com parentes ligados ao meio
cultural, como o gramático Sotero dos Reis, somada ao autodidatismo,
construíram seu amor pelas letras. Pela via da ficção, Firmina foi a
primeira a colocar o negro como sujeito humanizado, munido de voz capaz
de relatar suas tragédias como instrumento de denúncia à escravidão.
Pequena, de rosto redondo, olhos escuros e cabelos crespos, escreveu sua
obra mais conhecida com o pseudônimo de ``Uma Maranhense''. A partir da
publicação de \emph{Úrsula} em 1859, apontado como o primeiro romance
abolicionista do Brasil, Maria Firmina dos Reis passou a contribuir para
a imprensa local com textos e poemas, além de escrever um conto, uma
novela, um livro de poesias e diversas composições musicais. Entre suas
principais obras estão \emph{Úrsula} (1859), \emph{Cantos à beira"-mar}
(1871) e ``Hino da libertação dos escravos'' (1888).

Conta"-se que, quando foi tomar posse como a primeira mulher a ser
aprovada em um concurso público no Maranhão para o cargo de professora
de primário, com pouco mais de 20 anos, Firmina recusou o transporte em
uma liteira carregada por escravizados, preferindo ir a pé: ``Negro não
é animal para se andar montado nele''. Maria Firmina dos Reis, única
mulher dentre os bustos de importantes escritores maranhenses
homenageados na Praça do Pantheon, São Luís, morreu no dia 11 de
novembro de 1917 aos 95 anos, cega e pobre, na casa de Mariazinha,
ex"-escravizada e mãe de um de seus filhos de criação.


Segundo Régia Agostinho da Silva, professora da Universidade Federal do Maranhão e autora do artigo ``A mente, essa ninguém pode escravizar: Maria Firmina dos Reis e a escrita feita por mulheres no Maranhão'', na literatura da escritora ``os escravos são nobres e generosos. Estão em pé de igualdade com os brancos e, quando a autora dá voz a eles, deixa que eles mesmos contem suas tragédias. O que já é um salto imenso em relação a outros textos abolicionistas.''\footnote{Dispnível em: https://bit.ly/3rCboDV}. Para Régia da Silva, esse é um dos prováveis motivos pelos quais a obra da autora passou tantas décadas esquecido e desconhecido do grande público:

\begin{quote}
O assunto de que tratava era insalubre demais, uma fala antiescravista em uma das províncias mais escravistas do Brasil. Não a levaram a sério localmente, não queriam ouvi-la falando. E ela não teve como levar seu texto para outros lugares.\footnote{Ibidem.}
\end{quote}

Foi somente em 1962, em um sebo do Rio de Janeiro, que a obra de Maria
Firmina dos Reis foi recuperada pelo historiador paraibano Horácio de
Almeida. Nos registros oficiais da Câmara dos Vereadores de Guimarães,
sua gravura é, na realidade, a de uma mulher branca, tal como seu busto
no Museu Histórico do Maranhão, retrato de uma mulher de nariz fino e
cabelos lisos. Através dos novos olhares lançados sobre os estudos da
literatura afro"-brasileira e da literatura escrita por mulheres, vida e
obra de Maria Firmina dos Reis passam a ser resgatadas, integrando"-a,
aos poucos, ao cânone literário brasileiro.

Outra importante fonte de informações sobre a vida autora é seu famoso ``Álbum'', compilação de anotações pessoais da autora que foi disponibilizado ao seu biógrafo, José Nascimento Morais Filho, por um dos filhos de criação da autora, Leude Guimarães. O comentário da professora e pesquisadora Maria Helena Pereira Toledo Machado acerca do ``Álbum'' é elucidativo de alguns importantes aspectos biográficos de Maria Firmina:

\begin{quote}
O ``Álbum'' agrupa diferentes tipos de registro, que seguem os parâmetros usuais da redação de diário característicos do século \textsc{xix}: anotações de datas familiares e comemorativas, de visitas e participações em eventos sociais, como casamentos, de partidas e chegadas de viagem, e reflexões sobre a vida da autora.
Nesse conjunto, chamam atenção as entradas referentes às sucessivas perdas dos filhos adotivos ou afilhados (Firmina se encarregou da criação de onze crianças, algumas delas filhas de escravas/os), cujas notas revelam o fundo sofrimento que cada uma das mortes acarretou.
Manifestando incompreensão trágica e esforço de conformação com a vontade divina, esses apontamentos invariavelmente terminam com a frase ``que a terra lhe seja leve'', imprimindo um tom ainda mais soturno à redação. Por fim, as reflexões sobbre sua vida exprimem um profundo senso de solidão, fragilidade e melancolia, expresso em paisagens noturnas, visões do infinito e do vazio, e aguda consciência de sua individualidade. Esse é considerado o primeiro diário redigido por uma mulher a ser publicado no Brasil.\footnote{\textsc{machado}, Maria Helena Pereira Toledo.``Maria Firmina dos Reis: invisibilidade e presença de uma romancista negra no Brasil do século \textsc{xix} ao \textsc{xx}''. In: \textsc{reis}, Maria Firmina dos. \textit{Úrsula}. São Paulo: Penguin Classics Companhia das Letras, 2018, p.\,11--12.}
\end{quote}

\section{Sobre a obra}

Esta edição é uma miscelânea de gêneros literários praticados por Maria
Firmina dos
Reis, entre a prosa e a poesia. Ela reúne os textos, em prosa, ``A
escrava'' e ``Gupeva''; e, em verso, extraídos do livro \emph{Cantos à
beira"-mar} (1871) e da coletânea \emph{Parnaso maranhense} (1861), além
do ``Hino à liberdade dos escravos'' (1888). Apresentamos os textos na
mesma ordem que acabamos de mencionar, sem nenhuma hierarquização
cronológica ou de importância, apenas dividimos por gênero para conferir
uma coesão estrutural na organização da antologia.

O conto ``A escrava'' foi publicado, pela primeira vez, na edição nº 3
da \emph{Revista Maranhense}, em novembro de 1887. No século \textsc{xx}, vem
sendo editado nas mais diversas antologias sobre a autora, denotando seu
lugar de destaque no conjunto até então conhecido de sua obra e na
história da literatura brasileira. A narrativa tem caráter abolicionista
e marca uma fase mais amadurecida da autora. Ao colocar como
protagonista uma mulher negra escravizada, que fugiu de seu algoz e
relata sua própria história para a narradora, descrita apenas como ``uma
senhora'', Firmina dos Reis constrói um relato não apenas de escravidão,
mas, sobretudo, de resistência e liberdade.

Quanto a ``Gupeva'', trata"-se de uma novela publicada, pela primeira vez
e de forma incompleta, em capítulos no semanário \emph{O Jardim
Maranhense}, entre outubro de 1861 e janeiro de 1862. O texto foi
publicado em versão completa e revista em mais dois periódicos pela
escritora, no jornal \emph{Porto Livre}, em 1863, e em \emph{Eco da
Juventude}, em 1865. Ambientada na Bahia, a novela narra a história do
indígena Gupeva e da filha de sua esposa, Épica. Com a morte da mulher
logo após o parto, Gupeva batiza a criança com o mesmo da mãe e passa a
cuidar dela como pai. A paixão de Épica, a filha, por um marinheiro
francês traz à tona um passado repleto de conflitos.

O livro de poesia \emph{Cantos à beira"-mar} foi publicado, pela primeira
vez, em 1871, pela Typografia do Paiz. Dos 56 poemas, foram selecionados
29 para esta antologia. Nos versos líricos, estão presentes muitos dos
temas relacionados ao romantismo brasileiro, como a exaltação da terra,
o nacionalismo, a idealização e a impossibilidade do sentimento amoroso.
Além disso, há poemas com crítica à sociedade patriarcal e ao papel
destinado às mulheres na sociedade. Há ainda um poema indianista, ``Por
ocasião da tomada de Villeta e ocupação de Assunção'', dialogando com a
novela ``Gupeva''.

Os poemas ``Por ver"-te'' e ``Minha vida'' transcritos de uma coletânea
com textos outros poetas da geração de Firmina dos Reis, como Gonçalves
Dias e Sotero dos Reis. O livro \emph{Parnaso maranhense} foi publicado
em 1861 pela Tipografia do Progresso, em São Luís, com organização de
Gentil Homem de Almeida Braga, Antônio Marques Rodrigues, Raimundo de
Brito Gomes de Sousa, Luís Antônio Vieira da Silva, Joaquim Serra e
Joaquim da Costa Barradas. Por fim, temos um dos textos poéticos mais
famosos da autora, o ``Hino à liberdade dos escravos'', gênero bastante
presente em sua obra, composto por ocasião da Abolição da Escravatura,
em 1888.

Para o estabelecimento do texto, cotejamos os textos publicados em
\emph{Maria Firmina dos Reis: fragmentos de uma vida}, de 1975, edição
organizada por Nascimento Morais Filho, e em \emph{Úrsula}, de 2009, da
editora \textsc{puc}-Minas, e \emph{Úrsula e outras obras}, de 2019, das Edições
Câmara. A escrita foi atualizada conforme o Novo Acordo Ortográfico, mas
foram mantidas colocações pronominais da época e grafias como ``soidão''
(para ``solidão''), para não romper com o ritmo e a métrica do texto.

\section{Sobre o gênero}

É importante ressaltar que as características do Romantismo permeiam a
obra de Maria Firmina. \emph{Úrsula} trata de um amor conturbado entre
dois jovens brancos, no entanto da protagonismo a certos personagens
escravizados, expondo suas reflexões acerca das injustiças presentes em
uma sociedade escravocrata e patriarcal. Questões sociais e
abolicionistas fazem parte da terceira fase do Romantismo, também
conhecida como Condoreirismo. ``Gupeva'', por outro lado, integra
particularidades do indianismo, uma das tendências mais marcantes da
fase romântica. Por fim, grande parte de seus poemas publicados em
\emph{Cantos à beira-mar} exprimem uma inquietação diante do
autoritarismo vigente, fruto do patriarcado escravocrata, e o eu lírico
feminino manifesta a agonia e a melancolia tão presentes na produção do
período romântico.

O primeiro texto em prosa da autora aqui reunido, ``A escrava'', pode ser classificado como um conto, que, na definição do crítico Massaud Moisés:

\begin{quote}
O conto é, do ângulo dramático, unívoco, univalente. [\ldots]
Etimológicamente preso à linguagem teatral,
``drama'' significava ``ação''. E com o tempo passou a designar
toda peça destinada à representação. Na época romântica, dado o
princípio da fusão de gêneros, entendia-se por drama o misto de
tragédia e comédia. Transferido para a prosa de ficção, o termo
``drama'' entrou a significar ``conflito'', ``atrito''. Nesse caso,
``ação'' ``cortflito'' se tonaram equivalentes, uma vez que toda
ação pressupõe conflito, e este, promove a ação, ou por meio dela
se manifesta; em suma, ambos se implicam mutuamente.

O conto é, pois, uma narrativa unívoca, univalente: constitui
uma \textit{unidade dramática}, uma \textit{célula dramática}, visto gravitar ao
redor de um só conflito, um só drama, uma só ação. Caracteriza-se,
assim, por conter \textit{unidade ação}, tomada esta como a sequência de atos praticados pelos protagonistas, ou de acontecimentos de
que participam. A ação pode ser externa, quando as personagens se
deslocam no espaço e no tempo, e interna, quando o conflito se
localiza em sua mente.\footnote{\textsc{moisés}, Massaud. \textit{A criação literária}. São Paulo: Cultrix, 2006, p.\,40.}
\end{quote}

Já a segunda narrativa, ``Gupeva'', aproxima"-se do gênero literário novela.
Ainda na linha argumentativa de Massaud Moisés, em comparação com o conto a novela é
essencialmente multívoca, polivalente: ``constitui"-se de uma série de unidades ou células dramáticas. De onde se segue que a primeira característica estrutural da novela é sua pluralidade dramática: ao invés do conto, que gira em torno de um conflito, a novela focaliza vários. E cada um deles apresenta começo, meio e fim''.\footnote{\textsc{moisés}, Massaud. \textit{A criação literária}. São Paulo: Cultrix, 2006, p.\,113.}

Ainda nas palavras do crítico literário:

\begin{quote}
O novelista não esgota por completo o conteúdo de uma unidade para depois efetuar o mesmo com as seguintes: no fim de cada episódio, procura deixar sementes de mistério ou conflito para manter aceso o interesse do leitor. É raro que esvazie o recheio dramático duma célula antes de prosseguir, pois frustraria a curiosidade do leitor.

{[}\ldots{]}

Em suma multiplicidade dramática, numa corrente horizontal. Por isso, o número de páginas pode crescer à vontade: a pluralidade pressupõe uma estrutura aberta, de modo que novos episódios possam adicionar"-se numa cadeia sucessiva, assim como o fim provisório da narrativa implica a multivocidade.\footnote{Ibid., p.\,114.}
\end{quote}

Após esclarecer a ação na novela, Moisés coloca em perspectiva o tempo novelesco. Afirma que a estrutura linear e plural da novela lhe impõe uma limitação temporal, que faz com que esse gênero não aborde a história da personagem desde seu nascimento, mas reduz"-lhe o passado a poucas linhas, essenciais para compreender"-lhe as ações e seu modo de ser, supreendendo a personagem no momento em que está madura par agir.

\begin{quote}
O tempo da novela é o histórico, assinalado pelo relógio ou pelo calendário, ou pelas convenções sociais. O presente é categoria dominante, em que pese às referências sumárias ao pretérito. Tudo se passa como se os dias, as semanas, os meses e os anos, de efêmera importância, significassem muito. A ação desenrola"-se por inteiro no presente, aqui e agora: condensado o pretérito em breves anotações.\footnote{Ibid., p.\,115.}
\end{quote}

Por fim, ao abordar o espaço da novela, Moisés ressalta o dinamismo acelerado desse gênero literário, causado pela sucessão de episódios, que implica a ausência de uma unidade espacial. É no deslocamento físico, continua o crítico, que as personagens procuram dar cabo da angústia ou atender ao apelo da aventura, criando, mesmo em uma única cidade, uma pluralidade de espaços que distingue a novela. Segundo o crítico tal dinanismo espacial aproxima a novela do conto:

\begin{quote}
À semelhança do conto, a estrutura da novela caracteriza"-se por ser plástica, concreta, horizontal. Assumindo as mais das vezes a perspectiva da terceira pessoa, o autor se coloca fora dos acontecimentos, ou concede a uma personagem a direção da narrativa. A vida imaginária sobrepõe"-se à vida observada: o novelista concentra"-se em multiplicar os expedientes narrativos, formulando sucessivas células dramáticas, sem atentar para os imperativos da verossimilhança. O enredo, além de visível, não esconde nada, não dissimula profundidades dramáticas ou psicológicas: com o predomínio da ação, tudo o mais se torna menos significativo.\footnote{Ibid., p.\,117--118.}
\end{quote}

A segunda parte desse volume é composto pela poesia de Maria Firmina dos Reis.
Apesar de seguir alguns \textit{tópos} da poesia romântica de então, como a subjetividade exacerbada, a descrição lírica da natureza e o olhar melancólico, 
sua poesia difere"-se pela forte marca da subjetividade feminina, inquieta diante da realidade do século \textsc{xix} eivada pela presença do patriarcado escravocrata.

Para uma primeira definição de poesia enquanto gênero literário, poder"-se"-ia recorrer à definição do professor Domingos Paschoal Cegalla, para quem ``poesia é a linguagem subjetiva, carregada de emoção e sentimento, com ritmo melódico constante, bela e indefinível como o mundo interior do poeta visa a um efeito estético''.\footnote{\textsc{cegalla}, Domingos Paschoal. \textit{Novíssima Gramática da Língua Portuguesa}. São Paulo: Companhia Editora Nacional, 2008, p.\,640}

Aprofundando um pouco essa definição, o crítico Antonio Candido expanse a definição de poesia ao diferenciá"-la do verso.
Para o crítico, a poesia enquanto ato criador do artista independe da forma métrica do verso, que passa a ser apenas um dos registros possíveis do poético:

\begin{quote}
A poesia não se confunde necessariamente com o verso, muito menos com o verso metrificado. Pode haver poesia em prosa e poesia em verso livre. [\ldots]
Pode ser feita em verso muita coisa que não é poesia.\footnote{\textsc{candido}, Antonio. \textit{O estudo analítico do poema}. São Paulo: Terceira leitura, 1993, p.\,13--14.}
\end{quote}

Delineada, de forma breve e geral, a forma poética, pode"-se pensar agora em seus três gêneros básicos: lírico, épico e dramático.
Para o crítico Anatol Rosenfeld, a lírica é o gênero mais subjetivo, no qual uma voz central exprime um estado de alma traduzido em orações poéticas.
Seria a expressão de emoções e experiências vividas, ``a plasmação imediata das vivências intensas de um Eu no encontro com o mundo, sem que se interponham eventos distendidos no tempo (como na Épica e na Dramática)''.\footnote{\textsc{rosenfeld}, Anatol. \textit{O teatro épico}. São Paulo: Perspectiva, 2006, p.\,22.}

Devido a essa característica central da lírica, a expressão de um estado emocional, Rosenfeld considera que o eu"-lírico, nesse gênero, não se delineia enquanto um personagem. Embora possa evocar personagens e narrar acontecimentos, a lírica entendida enquanto gênero puro afasta"-se sobremaneira da apreensão objetiva do mundo, que não existe independente da subjetividade intensa que o apreende e exprime. Assim, na lírica prevalece a fusão entre o sujeito e o objeto, que serve mais a realçar os estados profundos de alma do poeta.
Sobre os aspectos formais do gênero, Rosenfeld nota:

\begin{quote}
À intensidade expressiva, à concentração e ao caráter ``'imediato'' do poema lírico, associa"-se, como traço estilístico importante, o uso do ritmo e da musicalidade das palavras e dos versos. De tal modo se realça o valor da aura conotativa do verbo que este muitas vezes chega a ter uma função mais sonora que lógico"-denotativa. A isso se liga a preponderância da voz do presente que indica a ausência de distância, geralmente assocaida ao pretérido. Este caráter do imediato, que se manifesta na voz do presente, não é, porém, o de uma atualidade que se processa e distende através do tempo (como na Dramática) mas de um momento ``eterno''.\footnote{Ibidem, p.\,23.}
\end{quote}



\subsection{Sobre nossa equipe}

Rodrigo Ribeiro Neves é crítico literário e pesquisador, com doutorado em Estudos de Literatura e mestrado em Letras pela Universidade Federal Fluminense (\textsc{uff}). Foi pesquisador visitante na Princeton University, nos \textsc{eua}, e bolsista da Fundação Casa de Rui Barbosa. Atuou como docente de literatura brasileira na Universidade Federal Fluminense (\textsc{uff}) e na Universidade Federal do Rio de Janeiro (\textsc{ufrj}). Desenvolveu pesquisa de pós"-doutorado no Instituto de Estudos Brasileiros da Universidade de São Paulo (\textsc{ieb"-usp}) e na Universidad de Alcalá, na Espanha.