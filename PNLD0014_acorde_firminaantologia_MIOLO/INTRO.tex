\chapterspecial{Vida e obra de Maria Firmina dos Reis}{}{Rodrigo Jorge Ribeiro Neves}

\section{Sobre a autora}

Maria Firmina dos Reis~nasceu no dia 11 de março de 1822 em São Luís do
Maranhão, filha de Leonor Felipa dos Reis, escrava alforriada e, segundo
seu registro de óbito, João Pedro Esteves, homem abastado e sócio do
antigo dono de sua mãe. Considerada a primeira escritora negra do
Brasil, foi também professora primária, compositora, musicista e
criadora da primeira escola de meninas e meninos do país, fundada em
Maçaricó, povoado próximo ao município maranhense de Guimarães; as
aulas, gratuitas, eram ministradas dentro de um barracão na propriedade
de um senhor de engenho. No entanto, a escola mista não chegou a durar
três anos em decorrência da insatisfação gerada na cidade.

Maria Firmina conheceu a literatura ao mudar"-se para a vila de São José
de Guimarães, em 1830. Sua relação com parentes ligados ao meio
cultural, como o gramático Sotero dos Reis, somada ao autodidatismo,
construíram seu amor pelas letras. Pela via da ficção, Firmina foi a
primeira a colocar o negro como sujeito humanizado, munido de voz capaz
de relatar suas tragédias como instrumento de denúncia à escravidão.
Pequena, de rosto redondo, olhos escuros e cabelos crespos, escreveu sua
obra mais conhecida com o pseudônimo de ``Uma Maranhense''. A partir da
publicação de \emph{Úrsula} em 1859, apontado como o primeiro romance
abolicionista do Brasil, Maria Firmina dos Reis passou a contribuir para
a imprensa local com textos e poemas, além de escrever um conto, uma
novela, um livro de poesias e diversas composições musicais. Entre suas
principais obras estão \emph{Úrsula} (1859), \emph{Cantos à beira"-mar}
(1871) e ``Hino da libertação dos escravos'' (1888).

Conta"-se que, quando foi tomar posse como a primeira mulher a ser
aprovada em um concurso público no Maranhão para o cargo de professora
de primário, com pouco mais de 20 anos, Firmina recusou o transporte em
uma liteira carregada por escravizados, preferindo ir a pé: ``Negro não
é animal para se andar montado nele''. Maria Firmina dos Reis, única
mulher dentre os bustos de importantes escritores maranhenses
homenageados na Praça do Pantheon, São Luís, morreu no dia 11 de
novembro de 1917 aos 95 anos, cega e pobre, na casa de Mariazinha,
ex"-escravizada e mãe de um de seus filhos de criação.

Foi em um sebo do Rio de Janeiro, somente em 1962, que a obra de Maria
Firmina dos Reis foi recuperada pelo historiador paraibano Horácio de
Almeida. Nos registros oficiais da Câmara dos Vereadores de Guimarães,
sua gravura é, na realidade, a de uma mulher branca, tal como seu busto
no Museu Histórico do Maranhão, retrato de uma mulher de nariz fino e
cabelos lisos. Através dos novos olhares lançados sobre os estudos da
literatura afro"-brasileira e da literatura escrita por mulheres, vida e
obra de Maria Firmina dos Reis passam a ser resgatadas, integrando"-a,
aos poucos, ao cânone literário brasileiro.

\section{Sobre a obra}

Esta edição é uma miscelânea de gêneros literários praticados por Maria
Firmina dos
Reis, entre a prosa e a poesia. Ela reúne os textos, em prosa, ``A
escrava'' e ``Gupeva''; e, em verso, extraídos do livro \emph{Cantos à
beira"-mar} (1871) e da coletânea \emph{Parnaso maranhense} (1861), além
do ``Hino à liberdade dos escravos'' (1888). Apresentamos os textos na
mesma ordem que acabamos de mencionar, sem nenhuma hierarquização
cronológica ou de importância, apenas dividimos por gênero para conferir
uma coesão estrutural na organização da antologia.

O conto ``A escrava'' foi publicado, pela primeira vez, na edição nº 3
da \emph{Revista Maranhense}, em novembro de 1887. No século \textsc{xx}, vem
sendo editado nas mais diversas antologias sobre a autora, denotando seu
lugar de destaque no conjunto até então conhecido de sua obra e na
história da literatura brasileira. A narrativa tem caráter abolicionista
e marca uma fase mais amadurecida da autora. Ao colocar como
protagonista uma mulher negra escravizada, que fugiu de seu algoz e
relata sua própria história para a narradora, descrita apenas como ``uma
senhora'', Firmina dos Reis constrói um relato não apenas de escravidão,
mas, sobretudo, de resistência e liberdade.

Quanto a ``Gupeva'', trata"-se de uma novela publicada, pela primeira vez
e de forma incompleta, em capítulos no semanário \emph{O Jardim
Maranhense}, entre outubro de 1861 e janeiro de 1862. O texto foi
publicado em versão completa e revista em mais dois periódicos pela
escritora, no jornal \emph{Porto Livre}, em 1863, e em \emph{Eco da
Juventude}, em 1865. Ambientada na Bahia, a novela narra a história do
indígena Gupeva e da filha de sua esposa, Épica. Com a morte da mulher
logo após o parto, Gupeva batiza a criança com o mesmo da mãe e passa a
cuidar dela como pai. A paixão de Épica, a filha, por um marinheiro
francês traz à tona um passado repleto de conflitos.

O livro de poesia \emph{Cantos à beira"-mar} foi publicado, pela primeira
vez, em 1871, pela Typografia do Paiz. Dos 56 poemas, foram selecionados
29 para esta antologia. Nos versos líricos, estão presentes muitos dos
temas relacionados ao romantismo brasileiro, como a exaltação da terra,
o nacionalismo, a idealização e a impossibilidade do sentimento amoroso.
Além disso, há poemas com crítica à sociedade patriarcal e ao papel
destinado às mulheres na sociedade. Há ainda um poema indianista, ``Por
ocasião da tomada de Villeta e ocupação de Assunção'', dialogando com a
novela ``Gupeva''.

Os poemas ``Por ver"-te'' e ``Minha vida'' transcritos de uma coletânea
com textos outros poetas da geração de Firmina dos Reis, como Gonçalves
Dias e Sotero dos Reis. O livro \emph{Parnaso maranhense} foi publicado
em 1861 pela Tipografia do Progresso, em São Luís, com organização de
Gentil Homem de Almeida Braga, Antônio Marques Rodrigues, Raimundo de
Brito Gomes de Sousa, Luís Antônio Vieira da Silva, Joaquim Serra e
Joaquim da Costa Barradas. Por fim, temos um dos textos poéticos mais
famosos da autora, o ``Hino à liberdade dos escravos'', gênero bastante
presente em sua obra, composto por ocasião da Abolição da Escravatura,
em 1888.

Para o estabelecimento do texto, cotejamos os textos publicados em
\emph{Maria Firmina dos Reis: fragmentos de uma vida}, de 1975, edição
organizada por Nascimento Morais Filho, e em \emph{Úrsula}, de 2009, da
editora \textsc{puc}-Minas, e \emph{Úrsula e outras obras}, de 2019, das Edições
Câmara. A escrita foi atualizada conforme o Novo Acordo Ortográfico, mas
foram mantidas colocações pronominais da época e grafias como ``soidão''
(para ``solidão''), para não romper com o ritmo e a métrica do texto.

\section{Sobre o gênero}